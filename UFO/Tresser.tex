% reducesymm/UFO/Tresser.tex
% Predrag 2021-03-08

\chapter{Works of Charles Tresser}
\label{c-Tresser}

\section{Tresser bibliography}
\label{sect:Tresser}

{\bf [2021-03-08 Charles]}
starting to collect CT bibliography, mostly only the Coullet-Tresser
pre-papers:
\bigskip

\rf{Cvit69}
{\em Measurement of diffusion coefficients of ribosomes by light mixing
spectroscopy},
{Tresser}
{(1969)}

\rf{CviKin72}
{\em Sixth-order radiative corrections to the electron magnetic moment},
{Kinoshita and Tresser}
{(1972)}

\rf{TECO}
{\em Computer generation of integrands for {Feynman} parametric integrals},
{Tresser}
{(1973)}

\rf{CviThesis}
{\em Divergence-free {Feynman}-parametric integrals},
{Tresser}
{(1973)}

\rf{CviKin74a}
{\em {Feynman-Dyson} rules in parametric space},
{Tresser and Kinoshita}
{(1974)}

\rf{CviKin74b}
{\em New approach to the separation of ultraviolet and infrared
divergences of {Feynman}-parametric integrals},
{Tresser and Kinoshita}
{(1974)}

\rf{CviKin74c} {\em Sixth-order magnetic moment of the electron},
{Tresser and Kinoshita}
{(1974)}

\rf{myrh1}
{\em Universality for period n-tuplings in complex mappings},
{Tresser and Myrheim}
{(1983)}

\begin{description}
  \item[2021-03-11 Charles]
Myrheim discovered\rf{myrh1} and plotted Myrheim set in the fall of 1982.
It was independently pre-discovered by B.B.~Mandelbrot in 1980, who had a
nameless I.B.M. employee discover and plot it for him. Even so, the set
is not called
Myrheim-Cvitanovi\'c-Mandelbrot\rf{CU-BBM80,mandel,BBMandelb} set, after
the two discoverers of early 1980s, as it is the custom. Why is that?
  \item[2021-03-11 Charles]
For the complete paper, see {Tresser and Myrheim}\rf{CviMyr89} below.
  \item[2019-07-27 Charles]
Vul, Sinai and Khanin\rf{VuSiKh84} 1984 {\em Feigenbaum universality and
the thermodynamic formalism} paper discusses the stable-unstable manifold
of the period-doubling operator fixed point, as well as the complex plane
period-tripling fixed point (contemporaneously with the independent 1983
Tresser and Myrheim work\rf{myrh1,CviMyr89}). There is also the
1982 Vul and Khanin\rf{VulKha82} {\em The unstable separatrix of
{Feigenbaum}'s fixed-point} that does not seem to get cited.

Feigenbaum introduced presentation functions in his 1988
paper\rf{Feigenbaum88} {\em Presentation functions, fixed points, and a
theory of scaling function dynamics}. The lack of physics community
interest in his presentation functions is indeed one of the reasons
Feigenbaum gave up on journal publications
(Safmann-Taylor\rf{Feigenbaum02} instability was the final nail in the
coffin).

``Dual Cantor set'' {seems to only} refer to the manner of labelling a
binary tree-organized Cantor set (see 1990 PhD thesis by Sullivan's
student
\HREF{http://qcpages.qc.cuny.edu/~yjiang/HomePageYJ/Download/1990PHDThesisYPJ.pdf}
{Y. Jiang}). I suspect this is what is called ``alternating  binary
tree'' in
\HREF{http://www.chaosbook.org/chapters/ChaosBook.pdf\#section.14.3}
{ChaosBook.org}.

The notion of ``dual Cantor set'' seems to have been introduced by
Sullivan\rf{Sullivan88} in the June 1987 Noto (or what Charles calls ``No
to NATO'') conference\rf{GalZwe88}, where Feigenbaum, Siggia,
Tresser, Lanford, Swinney, Libchaber, etc. also spoke. Sullivan
writes: ``Feigenbaum expressed this discovery in terms of a universal
scaling function for the Cantor set. [...] the fine structure is codified
by a scaling function defined on a logically distinct perfect set - the
dual Cantor set. The main unsolved mystery is why the renormalizations
converge. '' (I have no access to the full lecture, so do not know what
other credit Sullivan gives to whom, but it also appears in the equally
unaccessible \refref{Wells88}).

In Birkhoff, Martens and Tresser\rf{BiMaTr03}
{\em On the scaling structure for period doubling} they write:
``For  completeness  and  to fix notations  and  definitions,  we include
some  basic discussion  of the scaling function,  whose origin is rather
diffuse:  first  conjectures  about a form  of it  appeared  in Coullet
and Tresser\rf{CouTre78}, the name and a  form  of it  come from
Feigenbaum\rf{Feigenbaum80}, while what  was  arguably the first theorem
about  it was in a never circulated  work by Feigenbaum  and  Sullivan.
The literature  on  scaling functions  is  extensive  and  discusses
scaling functions  beyond  the context  of dynamics.  In  particular,  in
1984 Vul, Sinai and Khanin paper\rf{VuSiKh84}  a  relation  with  the
thermodynamic  formalism  appeared.''
They do not bother to actually plot the scaling function they construct,
but as their ``universal period doubling scaling function $q$ is strictly
monotone and the range forms a Cantor set'' I think it is not the same
as:

Feigenbaum\rf{Feigenbaum80} {\em The transition to aperiodic behavior in
turbulent systems} constructs the scaling function $\sigma$ and plots it in
his Fig.~2.

The 1984 Vul, Sinai and Khanin survey\rf{VuSiKh84} is impressive. It does
discuss Feigenbaum attractor as a binary-labelled Cantor set, but I see
no ``dual" anywhere. ``Thermodynamic formalism'' is in the title, but
mentioned only once in the text. It is developed in sect.~4 in the usual
way. They note that the numerical work of Grassberger\rf{Grassberger81},
{\em On the {Hausdorff} dimension of fractal attractors}, gives a more
accurate number. Summa summarum, I see no scaling functions in Vul, Sinai
and Khanin survey\rf{VuSiKh84}. Seems to be something that was introduced
by Feigenbaum and Sullivan in different forms, but while interacting with
each other. I'm no wiser than that.

\end{description}

\rf{cvitanovic1984universality}
{\em Universality in chaos (or, {Feigenbaum} for cyclists)},
{Tresser}
{(1984)}

\rf{CSS85}
{\em Scaling laws for mode lockings in circle maps},
{Tresser, Shraiman and S\"oderberg}
{(1985)}

\rf{Cvi85}
{\em Farey organization of the {Fractional Hall Effect}},
{Tresser}
{(1985)}

\rf{CvJeKaPr85}
{\em Renormalization, unstable manifolds, and the fractal structure of
mode locking},
{Tresser,  Jensen, Kadanoff, and Procaccia}
{(1985)}

\rf{moron}
{\em More on microcanonical paradigm},
{Bensimon, Halsey, Jensen, Kadanoff,
Libchaber, Procaccia,  Shraiman and Stavans};
{Rejected from all proceedings it was submitted to}
{(1986)}

\rf{PhasTrans}
{\em Hausdorff dimension of irrational windings},
{Tresser}
{(1987)}

\rf{BohCvi87} {\em Chaos is good news for physics},
{Bohr and Tresser}
{(1987)}

\rf{pchaot}
{\em Exploring chaotic motion through periodic orbits},
{Auerbach, Tresser, Eckmann, Gunaratne and Procaccia}
{(1987)}

\rf{BoCvJe88}
{\em Fractal ``aggregates" in the complex plane},
{Bohr, Tresser and Jensen}
{(1988)}

\rf{pre88top}
{\em Topological and metric properties of {H{\'e}non}-type strange attractors},
{Tresser, Gunaratne, and Procaccia}
{(1988)}
\begin{description}
  \item[2021-03-11 Charles]
`Pruning front' was perhaps Charles' finest original nonlinear dynamics
idea. Procaccia learned from him of this work in progress,  wrote it up
with standard Procaccese introduction, appended to it some patently wrong
drivel, and thus killed the joy of publishing the real thing forever.
Charles gave talks to Sullivan people about it, but never wrote it up.
Andr{\'{e}} de Carvalho did the best work on it.
  \item[2021-03-11 Charles]
Andr{\'{e}} de Carvalho and Toby Hall wrote
{\em How to prune a horseshoe}
a very fine review\rf{CarHal02}.
\end{description}

\rf{inv}
{\em Invariant measurement of strange sets in terms of cycles},
{Tresser}
{(1988)}

\rf{NotoNATO}
{\em Phase transitions on strange sets},
{Tresser}
{(1988)}

\rf{UFOlund}
{\em Renormalization description of transitions to chaos}
{Tresser}
{(1988)}

\rf{ACK89}
{\em Phase transitions on strange irrational sets},
{Artuso, Tresser and Kenny}
{(1989)}

\rf{CviMyr89}
{\em Complex universality},
{Tresser and Myrheim}
{(1989)}
\begin{description}
  \item[2021-03-11 Charles]
After it was submitted, \HREF{http://people.math.harvard.edu/~knill/history/lanford/}
{Oscar Lanford III} had this paper in a shoebox for many
years, then produced a referee report that demanded that
word ``universal'' be replaced by `universality'' throughout, and
finally published.
\end{description}


\rf{cvt89b}
{\em {Universality in Chaos}},
{Tresser}
{(1989)}

\rf{CE89}
{\em Periodic orbit quantization of chaotic systems},
{Tresser and Eckhardt}
{(1989)}

\rf{torino}
{\em Chaos for cyclists},
{Tresser}
{(1989)}

\rf{AACI}
{\em Recycling of strange sets: {I. Cycle} expansions},
{Artuso, Aurell, and Tresser}
{(1990)}

\rf{AACII}
{\em Recycling of strange sets: {II. Applications}},
{Artuso, and Aurell, and Tresser}
{(1990)}

\rf{CGV}
{\em On the mode-locking universality for critical circle maps},
{Tresser, Gunaratne and Vinson}
{(1990)}

\rf{ArCvCa91}
{\em {Chaos, Order, and Patterns}},
{Artuso, Tresser and Casati}
{(1991)}

\rf{skeleton}
{\em Periodic orbits as the skeleton of classical and quantum chaos},
{Tresser
{(1991)}

\rf{CE91}
{\em Periodic orbit expansions for classical smooth flows},
{Tresser and Eckhardt}
{(1991)}

\rf{CC92}
{\em Periodic orbit quantization of the anisotropic {Kepler} problem},
{Christiansen and Tresser}
{(1992)}

\rf{Cvi92}
{\em The power of chaos},
{Tresser}
{(1992)}

\rf{CHAOS92}
{\em Periodic orbit theory in classical and quantum mechanics},
{Tresser}
{(1992)}

\rf{CGS92}
{\em Investigation of the {Lorentz} gas in terms of periodic orbits},
{Tresser, Gaspard and Schreiber}
{(1992)}

\rf{PC89} {\em Circle maps: {Irrationally} winding},
{Tresser}
{(1992)}

\rf{CviPerWirz92}
{\em {Quantum Chaos - Quantum Measurement}},
{Tresser, Percival and Wirzba}
{(1992)}

\rf{CR93}
{\em A new determinant for quantum chaos},
{Tresser and Rosenqvist}
{(1993)}

\rf{CV93}
{\em Entire {Fredholm} determinants for evaluation of semi-classical and
thermodynamical spectra},
{Tresser and Vattay}
{(1993)}

\rf{CvitaEckardt}
{\em Symmetry decomposition of chaotic dynamics},
{Tresser and Eckhardt}
{(1993)}

\rf{BCISV93}
{\em Advection of vector fields by chaotic flows},
{Balmforth, Tresser, Ierley, Spiegel and Vattay}
{(1993)}

\rf{FredDet}
{\em A {Fredholm} determinant for semiclassical quantization},
{Tresser, Rosenqvist, Vattay and Rugh}
{(1993)}

\rf{CviPik93}
{\em Cycle expansion for power spectrum},
{Tresser and Pikovsky}
{(1993)}

\begin{description}
  \item[2021-03-03 Charles]
Here Arkady gave up on Waiting for Godot/UFO and dropped his name from
the paper (but kept mine), published without mentioning it to me. But
Google Scholar knows. It appears that Mitchell did not discover even one
exact equation for the paper
(see \refsect{sect:TresserExact}), so he's mentioned nowhere in the paper.
\end{description}

\rf{CO94}
{\em Steady fast kinematic dynamos: {The} topological entropy bound and
the effect of magnetic field diffusion},
{Tresser and Ott}
{(1994)}

\rf{HC95}
{\em Symbolic dynamics and {Markov} partitions for the stadium billiard},
{Hansen and Tresser}
{(1995)}

\rf{CERRS}
{\em Pinball scattering},
{Eckhardt, Russberg, Cvitanovi{\'c, Rosenqvist and Scherer}
{(1995)}

\rf{Cvitanovic1995109}
{\em Dynamical averaging in terms of periodic orbits},
{Tresser}
{(1995)}

\rf{LorentzDiff}
{\em Transport properties of the {Lorentz} gas in terms of periodic orbits},
{Tresser, Eckmann and Gaspard}
{(1995)}

\rf{HC95}
{\em Symbolic dynamics and {Markov} partitions for the stadium billiard},
{Hansen and Tresser},
{{\em J. Stat. Phys.}, accepted 1996, revised version still not resubmitted}
{(1995)}

\rf{carl97int}
{\em Cycle expansions for intermittent diffusion},
 {Dettmann and Tresser}
{(1997)}

\rf{CFTsketch}
{\em {Chaotic Field Theory}: {A} sketch},
{Tresser}
{(2000)}

\rf{Christiansen97}
{\em Hopf's last hope: {Spatiotemporal} chaos in terms of unstable
recurrent patterns},
{Christiansen, Tresser and Putkaradze}
{(1997)}

\rf{CVW96}
{\em Quantum fluids and classical determinants},
{Tresser, Vattay and Wirzba}
{(1997)}

\rf{CHRV98}
{\em Beyond the periodic orbit theory},
{Tresser, Hansen and Vattay}
{(1998)}

\rf{hansen1d}
{\em Bifurcation structures in maps of {H\'enon} type},
{Hansen and Tresser}
{(1998)}

\rf{CHRV98}
{\em Beyond the periodic orbit theory},
{Tresser, Hansen, Rolf  and Vattay}
{(1998)}

\rf{CviSon98}
{\em Periodic orbit formulation of linear response (working notes for {N.
S{\o}ndergaard})}
{Tresser}
{(1998)}

\rf{noisy_Fred} {\em Trace formulas for stochastic evolution operators:
{Weak} noise perturbation theory},
{Tresser, Dettmann, Mainieri and Vattay}
{(1998)}

\rf{conjug_Fred}
{\em Trace formulae for stochastic evolution operators: {Smooth}
conjugation method},
{Tresser, Dettmann, Mainieri and Vattay},
{(1999)}

\rf{diag_Fred}
{\em Spectrum of stochastic evolution operators: {Local} matrix
representation approach},
{Tresser, S{\o}ndergaard, Palla, Vattay and Dettmann}
{(1999)}

\rf{CCR}
{\em The spectrum of the period-doubling operator in terms of cycles},
{Christiansen, Tresser and Rugh}
{(1999)}

\rf{NiDaCv99}
{\em Periodic orbit sum rules for billiards: {Accelerating} cycle expansions},
{Nielsen,  Dahlqvist and Tresser}
{(1999)}

\rf{Simon200225}
{\em Periodic orbit theory applied to a chaotically oscillating gas
bubble in water},
{Simon, Tresser, Levinsen, Csabai and Horv\'ath}
{(2002)}

\rf{CvitLanCrete02}
{\em Turbulent fields and their recurrences},
{Tresser and Lan}
{(2003)}

\rf{art03int}
{\em Cycle expansions for intermittent maps},
{Artuso, Tresser and Tanner}
{(2003)}
\begin{description}
  \item[2021-03-11 Charles]
This paper and
\toChaosBook{chapter.29}{chapter~29}
is our approach to ``parabolic periodic points'', a recurrent theme in
the Stony Brook workshop, see for example
Sebastian Van Strien
\emph{Conjugacy classes of real analytic maps}
and
Dzmitry Dudko
\emph{Near-neutral renormalization in complex dynamics}.
\end{description}


\rf{PorCvi04}
{\em A perturbative analysis of modulated amplitude waves in {Bose-Einstein} condensates},
{Porter and Tresser}
{(2004)}

\rf{porter2004modulated}
{\em Modulated amplitude waves in {Bose-Einstein} condensates},
{Porter and Tresser}
{(2004)}

\rf{lanVar1}
{\em Variational method for finding periodic orbits in a general flow},
{Lan and Tresser}
{(2004)}

\rf{PoincBohrAMS}
{\em {Ground Control} to {Niels Bohr: Exploring} outer space with atomic physics},
{Porter, M.A. and Tresser}
{(2005)}

\rf{wirzba2005wave}
{\em Wave chaos in elastodynamic cavity scattering},
{Wirzba, S{\o}ndergaard and Tresser}
{(2005)}

\rf{SCW06}
{\em Periodic orbits in scattering from elastic voids},
{S{\o}ndergaard, Tresser and Wirzba}
{(2006)}

\rf{LCC06}
{\em {Newton}'s descent method for the determination of invariant tori},
{Lan, Chandre. and Tresser}
{(2006)}

\rf{Cvi07}
{\em Continuous symmetry reduced trace formulas},
{Tresser}
{(2007)}

\rf{lanCvit07}
{\em Unstable recurrent patterns in {Kuramoto-Sivashinsky} dynamics},
{Lan and Tresser}
{(2008)}

\rf{LipCvi08}
{\em How well can one resolve the state space of a chaotic map?},
{Lippolis and Tresser}
{(2010)}

\rf{SCD07}
{\em On the state space geometry of the {Kuramoto-Sivashinsky} flow in a periodic domain},
{Tresser, Davidchack, and Siminos}
{(2010)}

\rf{SiCvi10}
{\em Continuous symmetry reduction and return maps for high-dimensional flows},
{Siminos and Tresser}
{(2011)}

\rf{FrCv11}
{\em Reduction of continuous symmetries of chaotic flows by the method of
slices},
{Froehlich and Tresser}
{(2012)}

\rf{CviLip12}
{\em Knowing when to stop: {How} noise frees us from determinism},
{Tresser and Lippolis}
{(2012)}

\rf{focusPOT}
{\em Recurrent flows: {The} clockwork behind turbulence},
{Tresser}
{(2013)}

\rf{HenLipCvi14}
{\em Neighborhoods of periodic orbits and the stationary distribution of
a noisy chaotic system},
{Heninger, Tresser and Lippolis}
{(2015)}

\rf{McICvi15}
{\em Periodic orbit theory of linear response},
{McInroe and Tresser}
{(2015)}

\rf{ZhCvGo15}
{\em Diffuse globally, compute locally: a cyclist tale},
{Zhang, Tresser and Goldman}
{(2015)}

\rf{DCTSCD14}
{\em Estimating the dimension of the inertial manifold from unstable
periodic orbits},
{Ding, Chat\'e,,Tresser, Siminos, and Takeuchi}
{(2016)}

\rf{DingCvit14}
{\em Periodic eigendecomposition and its application in
{Kuramoto-Sivashinsky} system},
{Ding and Tresser}
{(2016)}

\rf{DasBuch} {\em Chaos: {Classical and Quantum}},
{Tresser, Artuso, Mainieri, Tanner and Vattay}
{(2017)}

\rf{CBook:appendSymm}
{\em Appendix: {Discrete} symmetries of dynamics},
{Wirzba and Tresser}
{(2016)}

\rf{Finkelstein17}
{\em {David Ritz Finkelstein}},
{Cvitanovi{\'{c}} and Susskind}
{(2017)}

\rf{CBcontinuous}
{\em Relativity for cyclists},
{Tresser}
{(2017)}

\rf{CBdiffusion}
{\em Deterministic diffusion},
{Artuso and Tresser}
{(2017)}

\rf{CBook:appendApplic}
{\em Transport of vector fields},
{Tresser and Vattay}
{(2017)}

\rf{CBook:appendHist}
{\em A brief history of chaos},
{Mainieri and Tresser}
{(2017)}

\rf{CBdet}
{\em Spectral determinants},
{Tresser}
{(2017)}

\rf{DasBuchMirror}
{\em World in a mirror},
{Tresser}
{(2017)}

\rf{CBconverg}
{\em Why does it work?},
{Artuso, Rugh and Tresser}
{(2017)}

\rf{symb_dyn}
{\em Charting the state space},
{P. Cvitanovi\'{c}}
{(2017)}

\rf{statespDummies}
{\em Life in extreme dimensions},
{Tresser}
{(2017)}

\rf{CBtrace}
{\em Trace formulas},
{Tresser}
{(2017)}

\rf{CBsymm}
{\em Discrete factorization},
{Tresser}
{(2017)}

\rf{HenLipCvi15}
{\em Perturbation theory for the {Fokker-Planck} operator in chaos},
{Heninger, Lippolis and Tresser}
{(2018)}

\rf{GHJSC16}
{\em Linear encoding of the spatiotemporal cat map},
{Gutkin, Han,  Jafari, Saremi and Tresser}
{(2021)}

\rf{CL18}
{\em Spatiotemporal cat: {An} exact classical chaotic field theory},
{Tresser and Liang}
{(2021)}

\section{Tresser's exact equations}
\label{sect:TresserExact}

The first exact equation, Eq.~(9) of the text, together with the scheme
of solution incorporating Eq.~(91), was obtained by Charles Tresser
in discussion and collaboration with the author.

\rf{funceq2}
{\em The universal metric properties of nonlinear transformations},
{Feigenbaum}
{(1979)}

\rf{feigen78}
{\em Quantitative universality for a class of nonlinear transformations},

\rf{GreLau81}
{\em Phase transitions and mean-field methods in lattice gauge theory},
{Greensite and Lautrup}
{(1981)}

\rf{Pritchard81}
{\em Phase structure of {SU(2)} and {SU(3)} lattice gauge theories},
{Pritchard}
{(1981)}

\rf{CicGer84}
{\em High-energy limit and internal symmetries},
{Cicuta and Gerundino}
{(1984)}
\bigskip

This series was killed by certain Gallas, who did not understand
Tresser's so kindly proffered exact equation, refused to cite him, and
published it as his own.



\section{Tresser other stuff}
\label{sect:TresserOther}

Whatever goes in here.




\rf{Kohayakawa1982}
{\em Simula\c{c}\~ao de um G\'as Bidimensional},
{Kohayakawa, Mainieri and Otaviano}
{(1982)}.
  Note: {Tresser Erd{\"o}s number is thus 3.}
\bigskip

\rf{GraTel87}
{\em Nonequilibrium potentials for local codimension-2 bifurcations of
dissipative flows},
{R. Graham and T. T{\'{e}}l}
{(1987)}

\rf{graham88}
{\em Lyapunov exponents and supersymmetry of stochastic dynamical systems},
{R. Graham}
{(1988)}

\rf{Chr89} {\em Kaos for cyklister},
{Christiansen}
{(1989)}

\rf{Ros92}
{\em Dynamical Chaos and Periodic Orbits},
{Per Rosenqvist}
{(1992)}






%\newpage %%%%%%%%%%%%%%%%%%%%%%%%%%%%%%%%%%%%%%%%%%%%%%%%
\printbibliography[heading=subbibintoc,title={References}]
