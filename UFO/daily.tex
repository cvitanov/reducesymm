% reducesymm/UFO/daily.tex
% Predrag 2019-07-23


\chapter{Daily log}
\label{c-dailyBlog}

Comments to the New York Times
\HREF{https://www.nytimes.com/2019/07/18/science/mitchell-feigenbaum-dead.html}
{obituary} are all very kind with almost no stupid trolling. Mitchell
would have been happy to read them.

\begin{description}
\item[2019-07-19 John K., Nashville]
I was in a class that Mitchell Feigenbaum taught at Cornell.  He was
extremely intense and amazingly hard-working.  Once, on a Saturday
morning, he came into class and started a lecture.  Seeing that the
students didn't seem to be following him, he decided to give an impromptu
lecture on thermodynamics.  He gave a very lucid talk, rapidly filling up
several chalk boards with chalk dust flying and a sweat developing on his
brow.

Once, while looking out the physics library window, I saw Mitchell come
driving in the parking lot, not far from the location of the photograph
in his obituary of him dropping papers while walking with students.  He
was driving what appeared to be a brand new yellow convertible Mercedes
with the top down and his mane of hair blowing in the wind.  It was
rumored that he had bought it with prize money.

\item[2019-07-19 whaddoino, Kafka Land]
Comment to New York Times obit:

Feigenbaum was a grad student at MIT and a post-doc at Cornell. You don't
get to go there unless you are pretty good to begin with. Cornell physics
in particular was humming with activity in the 1970's. Ken Wilson, Robert
Richardson, David Lee and David Oscheroff did work that would win them
all Nobels, Michael Fisher would win the Wolf prize, and many others,
David Mermin, Robert Pohl, Toichiro Kinoshita, etc. would become members
of the National Academy.

Again, not every person who goes to MIT or Cornell for grad school is a
genius of the Feigenbaum caliber, but chances are if you are
intrinsically smart, you will be noticed. And Peter Carruthers at Los
Alamos was no mere ``administrator." A class 1 physicist, well able to
see real talent.

\end{description}

There is a Rashomon of rememberences ahead

\begin{description}
\item[2019-07-19 Kisu] I met Mitchell at your place in Copenhagen. I
always remember him for only eating processed vegetables (\ie, meat) and
getting his vitamins from cigarettes.

\item[2019-07-26 Jürgen Kurths] % <kurths@pik-potsdam.de>
 I bought him at a special ``Stand" near the Brandenburg Gate a
bratwurst to eat from the hand. He was so happy about.

\item[2019-07-28 David]
Your story about Mitchell enjoying eating a bratwurst in Berlin
resonates well with my own experience on an occasion (many years ago)
when he and I were together in Berlin and each had not just a bratwurst
but  a Maß of Pilsner Urquell vom Fass—that really made his day.

\end{description}



%\newpage %%%%%%%%%%%%%%%%%%%%%%%%%%%%%%%%%%%%%%%%%%%%%%%%
\printbibliography[heading=subbibintoc,title={References}]
