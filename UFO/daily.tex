% reducesymm/UFO/daily.tex
% Predrag 2019-07-23


\chapter{Daily log}
\label{c-dailyBlog}

Comments to the New York Times
\HREF{https://www.nytimes.com/2019/07/18/science/mitchell-feigenbaum-dead.html}
{obituary} are all very kind with almost no stupid trolling. Mitchell
would have been happy to read them.

\begin{description}
\item[2019-07-19 John K., Nashville]
I was in a class that Mitchell Feigenbaum taught at Cornell.  He was
extremely intense and amazingly hard-working.  Once, on a Saturday
morning, he came into class and started a lecture.  Seeing that the
students didn't seem to be following him, he decided to give an impromptu
lecture on thermodynamics.  He gave a very lucid talk, rapidly filling up
several chalk boards with chalk dust flying and a sweat developing on his
brow.

Once, while looking out the physics library window, I saw Mitchell come
driving in the parking lot, not far from the location of the photograph
in his obituary of him dropping papers while walking with students.  He
was driving what appeared to be a brand new yellow convertible Mercedes
with the top down and his mane of hair blowing in the wind.  It was
rumored that he had bought it with prize money.

\item[2019-07-19 whaddoino, Kafka Land]
Comment to New York Times obit:

Feigenbaum was a grad student at MIT and a post-doc at Cornell. You don't
get to go there unless you are pretty good to begin with. Cornell physics
in particular was humming with activity in the 1970's. Ken Wilson, Robert
Richardson, David Lee and David Oscheroff did work that would win them
all Nobels, Michael Fisher would win the Wolf prize, and many others,
David Mermin, Robert Pohl, Toichiro Kinoshita, etc. would become members
of the National Academy.

Again, not every person who goes to MIT or Cornell for grad school is a
genius of the Feigenbaum caliber, but chances are if you are
intrinsically smart, you will be noticed. And Peter Carruthers at Los
Alamos was no mere ``administrator." A class 1 physicist, well able to
see real talent.

\end{description}

There is a Rashomon of remembrances ahead

\begin{description}

\item[2019-08-07 Predrag, Kathy]
% Kathy Hammond <kathy.d.hammond@gmail.com>
Micheline Kahan-Lide is super important - lover from when Mitchell was at
MIT, was Mitchell's fianc\'e and great love. She is the source of all of
Mitchell's fresh vegetables (the storied Belgian Fries that required
liters of finest oils) as well the processed vegetables: The Belgian Pork
Chops.

Micheline is married to Lionel Lide, a Frenchman - with some kind of
coffee importing business. Their address is
\begin{quote}
Lot 47
Terres Basses,
St. Martin
French West Indies
\end{quote}

\item[2019-07-19 Kisu] I met Mitchell at your place in Copenhagen. I
always remember him for only eating processed vegetables (\ie, meat) and
getting his vitamins from cigarettes.

\item[2019-07-26 Jürgen Kurths] % <kurths@pik-potsdam.de>
 I bought him at a special ``Stand" near the Brandenburg Gate a
bratwurst to eat from the hand. He was so happy about.

\item[2019-07-28 David]
Your story about Mitchell enjoying eating a bratwurst in Berlin
resonates well with my own experience on an occasion (many years ago)
when he and I were together in Berlin and each had not just a bratwurst
but  a Maß of Pilsner Urquell vom Fass—that really made his day.

\item[2019-10-05 Gunilla Feigenbaum] % <gunilla450@rcn.com>
I seem to recall he knew Frank Wilzcek.

\item[2019-12-23 Josh Tenenbaum] % <josh.tenenbaum@gmail.com>
I met Feigenbaum once, when giving a talk at Rockefeller.  Honestly it
wasn't a great talk, and I felt bad because I was really hoping he'd find
it interesting. I was in college during the peak of the ``chaos'' boom of
the late 80s early 90s so I was definitely familiar with his seminal
work.  I would be very interested to learn more about his work on vision.

\item[2019-12-29 Dennis Sullivan]
%to Yakov, Konstantin, Mikhail, John, David, me, Pierre, Misha
{\bf [2019-07-26]} entry above, continued,
for the special conference on renormalization and Feigenbaum's numerical discovery

\bigskip

\textbf{Recalling learning about dynamics and\\
        using it to study period doubling renormalization}.
\\                          by  Dennis Sullivan

\medskip

I remember well, many mathematical  and  stimulating moments  from the
early 70's  continuing  entirely through the 80's while learning about
the subject of dynamics and period doubling bifurcations  in dynamics.

There was Bruce Knight, a physicist turned biologist in Hartline's  lab
at Rockefeller University , asking new questions  in the  early 70's
such as

"In a  typical one parameter family of  circle diffeomorphisms  does the
set of parameters with irrational rotation number have positive
probability".

This should be so if their lab's  model of animal vision  concerning
nerve cell firing in the retina  would be valid. Bruce explained that
snow blindness was caused by phase locking  of retinal nerve firing
related to the  model's circle diffeomorphism having a rational rotation
number.

There was Michel Herman in Paris  discussing in 1975 the argument of
Arnaud Denjoy (1884-1974)  obtaining   bounded non linearity  for long
iterations of  one dimensional mappings on disjoint intervals. This,
showing  Poincare's  semi conjugacy  H  of any circle  diffeomorphism F
without periodic points to an irrational rotation of the circle had to
actually be a  topological conjugacy in the $C^2$ case. The argument used
the "nonlinearity one form" . $N(F) =  d logF' =  F"/F'dx$ and its chain
rule  for iterated composition  $N(GF) =  N(F) + F*N(G)$.   The
nonlinearity one form N measures the  F distortion of the affine
geometry.


Later there was  Michel's  phd student Jean Christophe Yoccoz discussing
Herman's  tour de force argument that H is actually a diffeomorphism  for
a full measure set of rotation numbers   when  F  is at least $C^3$. This
work answered affirmatively the  conjecture of Vladimir Arnold. Jean
Christophe  pointed out  that some of  Michel's   courageous
calculations of the third derivatives of high iterates of F  could be
explained  using the chain rule for the  Schwarzian derivative quadratic
differential,
$S(F)= DN -1/2 NN = (F'''/F' - 3/2 (F"/F')^2) dx^2$.
The chain rule  is  structurally the same as above $S(GF) = S(F) + F*S(G)$.  Besides having a sign (the value at a point is multiplied by h' squared for a change of charts mapping h), S(F) measures the F distortion of cross ratio  or rather the F distortion of the  projective geometry.

 One was struck by Herman's theorem for the golden rotation number, that first returns  close to the  starting point were on opposite sides for successive fibonacci number  iterates and  that the ratios of these smaller and smaller distances  converged to the golden ratio.  A couple of years  were spent trying to get  a  conceptual proof which came  eventually:
use Denjoy to control the nonlinearity of  renormalized mappings (these being: restrict a high iterate to an interval between successive closest returns which also glues the endpoints together giving a circle diffeomorphism with a Gauss map shifted rotation number) ,
use the Schwarzian chain rule to see  their  projective distortions  became  arbitrarily small - so up to projective changes they were  closer and closer  to rotations , use the ergodicity of the gauss map shifting continued fractions  and  then to finish use  Kolmogorov- Arnold- Moser  perturbation theorem that a smooth perturbation of a rotation for a positive measure set of rotation numbers  (with estimates) is smoothly conjugate to a rotation.

 There was  Pierre Deligne's  Bourbaki presentation of Michel Herman's  first result where every sentence advanced decisively into the dense proof, the first sentence being: "for a homeomorphism of the circle without periodic points  each point's orbit has the same cyclic order  and this is the same as that of a unique irrational rotation of the circle."

These results culminated in large measure the study of invertible dynamics in one  real dimension.
Invertible dynamics in more dimensions including dimension two  is still quite open.

 Fortunately , an  article by biologist Robert May  generated mathematical interest in non invertible  dynamics in dimension one, specifically the unimodal maps which when iterated had orbits which  modeled the patterns of population dynamics.

    A  1975 lecture by Stephan Smale in Aspen discussed these  dynamical patterns , especially the  sequence of period doubling bifurcations. Smale asked a new kind of question about the limit of these bifurcations. Mitchell Feigenbaum,  a physicist, was in the audience and took up Smale's question and recast it.

    There were the  physicists   Oscar Lanford, Henri Epstein and David Ruelle  studying a functional equation  for unimodal mappings  even during scientific meetings at the IHES which in words said :
" f composed with f / restricted to a sub interval was linearly conjugate to f." This, being related to Feigenbaum's discovery.

  In 1976 at the IAS  there was Thurston's work on surface diffeomorphisms   foreshadowing  his 1980 geometrization conjecture in 3D, and  there was the Milnor -Thurston kneading sequence for unimodal maps reflecting the Sharkovsky ordering whose beginning  was the  period doubling bifurcation
cascade.

  In 1977 there  was an interview of Lipman Bers where he  revealed his preoccupation with the Ahlfors measure problem for the  Poincare' limit set of a finitely generated (fg)  Kleinian group ( Kleinian group by definition: any discrete subgroup of PSL(2,C)). This suggested  studying  a related  stronger problem : to show ergodicity on the limit set for any fg  Kleinian group.

  There  were revelations  from Thurston and Gromov  about the d-sphere at infinity of hyperbolic space of dimension d+1 with its quasi conformal geometry. This gave insight into the dynamics of the group action  at infinity.

  This activity followed  in the wake of  a rejuvenation of the theory of Kleinian groups in the 60's by  Lars Ahlfors and Lipman Bers .
 They took vigorous advantage of the analysis behind the 2D  Gauss theorem about conformally flat coordinates  which was extended to  bounded shape measurable metrics by Morrey using quasi conformal  homeomorphisms.
Their use of this theory  suggested a powerful dynamical tool.

 There was  interest in the  study of random walks for the example of hyperbolic space  including Sinai's understanding that random paths deviated logarithmically  from hyperbolic geodesics.  Such ideas  could be used  to  study  ergodic properties of Kleinian groups  combined with Lucy Garnett's brownian motion on the leaves of a foliation.

 There was  Wilhelm Magnus'  "necklace of  four tangent circles" example of Poincare's perturbations of a Fuchsian group (by definition ,a discrete subgroup of PSL(2,R)) generated by inversions in  these four to a Kleinian group  producing a fractal Poincare'  limit curve  which in his book was reported to have been called by Poincare's generation "eine so-genannt  kurve."

 There was   Rufus Bowen's quite  analogous  fractal circle Julia set for a perturbation of z---> zz.  ( One could check that the  infinitesimal  version of the conjugacy back  to a round circle was  exactly Riemann's  "powers of two" nowhere differentiable lacunary   Fourier series.)

This analogy  plus a  remark of Michel Hermann that the complement of the poincare limit set is also obviously defined by normal families  stimulated a preoccupation with  an  analogy between Kleinian groups and holomorphic dynamics in one variable.  This  analogy  was  also made  before WWI by Fatou, Julia  et al  but  this analogy was lost when the subject died after WWI .  Guckenheimer   studied holomorphic dynamics in his Berkeley thesis  circa 1970 and Hubbard studied ' Newton's method for root finding of complex polynomials to help Cornell undergraduates  see math complexity in an easily stated problem . The beautiful computer pictures did get  Adrien Douady interested in these matters.

Rather soon nice questions started flowing forth from Douady and Hubbard  some going back to Fatou and Julia 65 years before  and some of these could be treated with the Ahlfors Bers technique from Kleinian groups. The Ahlfors finiteness theorem was analogous to the nonwandering question of Fatou which surrendered to the analogue of Ahlfors' proof .

 There were two ideas  from Thurston and Gromov.
 Gromov said Ahlfors theorem should be obvious for fg groups because they could only have finitely many deformations.
 Thurston  explained that  an Ahlfors Bers  deformation of a domain whose frontier was in the limit set (or the  julia set ) could be regarded as a deformation rel boundary  because each point is dynamically marked in the limit set or the Julia set.



 In the  two examples of fractal circles  above (Magnus and Bowen) the Hausdorff dimension's excess above one was a measure of the size of the perturbation.  The same was true for  Poincare's perturbations of Fuchsian groups to Kleinian groups.
Also in hyperbolic systems like the horseshoe the Hausdorff dimension was a variable parameter related to the size of the deformation.

  One also  heard about  the  Feigenbaum number  measuring the rate of period doubling bifurcations and the numerically observed rate of convergence to a fixed point mapping of the period doubling renormalization, the fixed point being the above mentioned solution to the functional equation:
" f :[0,1] ---> [0,1]  and f composed with f / restricted to an invariant  sub interval  is linearly conjugate to f."



One day in 1982  there was a decisive exchange :  Charles Tresser said  "The Hausdorff dimension of the period doubling critical orbit Cantor set changes numerically if one changes the exponent of the critical point "  My  reaction was " Do you mean to say  the Hausdorff dimension of the Cantor set doesn't change if you don't change the  exponent of the critical point while deforming among period doubling maps"  Charles  responded "Numerically, no, it doesn't change if you don't change the exponent of the critical point."

This was amazing  because of the usual variability of Hausdorff dimension under  the perturbations mentioned above.  This suggested a  new rigidity result reminiscent of Hermann's golden number geometry of closest first returns for golden rotation number.

 Here was a  numerically true , very beautiful ,but  mathematically unproved theorem.  One could  work on  this indefinitely full time without worry. There would be no counterexamples, it was true, and any proof could well teach new things about mathematics and dynamics.

 Seeking out   Mitchell Feigenbaum and Predrag Cvitanovic , the formulator of  the functional equation,  one learned about the  Feigenbaum constant  and  Mitchell's route to  its discovery. It was fascinating.

For example, Feigenbaum at Cornell had adopted the language of renormalization from quantum field theory in physics especially
 Ken Wilson's renormalization iteration to a fixed point of the physical  action.
Leo Kadanov likened the period doubling maps at the limit of period doubling bifurcations to the onset of turbulence in hydrodynamics because  they were on the  boundary of chaotic dynamics.
This picture  might fit  with  Flores Takens' and David Ruelle's chaotic attractor picture of turbulence from 1971.

Albert  Libchaber reported 1980  he had experimentally measured (at
Rockefeller University ) Feigenbaum's constant to three decimal places
....in an infinite degree of freedom  fluid experiment   where an
increasing temperature difference in a bath of liquid mercury  induced
doubling bifurcations of convection rolls  which were created
geometrically  faster at a rate 4.68...  leading to a  complex structure
in the fluid.
\\(2019-12-29 John Guckenheimer: Libchaber was still in France at the Ecole
Normale Superieure when he did his experiments measuring a period
doubling sequence in liquid helium. He moved first to Chicago and then to
Princeton before resettling at Rockefeller.)

Predrag said for him , understanding turbulence, was the main problem in physics.

 This naturally intensified one's interest in the nature of the mathematical structure of the period doubling cascade.




 Synthesis and more examples:

Hermann's rigidity for circle diffeomorphisms   seemed to be  an infinite dimensional  parameter  version  of  Mostow's finite dimensional  parameter  rigidity of closed hyperbolic manifolds of dimension  at least three ( hyperbolic meaning constant sectional curvature -1)  where topology or combinatorics ( in this case the universal covering group as an abstract group)  implied a fine scale (at infinity ) geometric  rigidity and then by ergodicity a global rigidity ie isometry.

Moreover in 2D,  Mostow's  rigidity theorem  was false, because of Riemann's moduli space with  6g-6 real parameters for hyperbolic surfaces.  Yet  the argument worked up to the very end where lebesgue measure classes were not respected in the one dimensional infinity of the universal cover.  Ergodic theory  and lebesgue spaces as in Rochlin's  basic paper   thus came to prominence in these geometric/topological questions, especially  for  the dynamical structures at infinity.

  A paper by   Kostia Khanin and Yakob Sinai  showed the period doubling cantor set of the numerically observed fixed point mapping  of period doubling renormalization, described by the functional equation,  had a precise fine structure defined by a scaling function on a dual cantor set. And they showed  that a  topological conjugacy between two maps of period doubling type , assuming renormalization convergence , had to be smooth at the critical point.

Here was  a  putative  very precise  infinite dimensional parameter space  example of  a mostow- hermann type rigidity of geometric structure given the topological form.


In a talk in the Einstein Chair Seminar by  Sebastian Van Strien on work with  Welington deMelo  one  learned the negative schwarzian condition on a one dimensional map  implied for a four tuple of points on the line the poincare length of the inner interval relative to the enclosing interval  had to decrease under the map.  This was a memorable moment because the difficulty with one dimensional unimodal maps was  that the  total mass of non linearity available  for compositions along disjoint intervals  passing near the critical point was unbounded. But since the  schwarzian is negative near the critical point , one  now had  a chance to control the geometry.

 A  paper by John Guckenheimer showed that  under  period doubling renormalization of a special class of unimodal maps ,  one condition being negative schwarzian everywhere, the structure of the local cantor set stayed  bounded as one descended down the hierarchy of scales.

 Synthesizing several ideas above :

 The Denjoy argument away from the critical point ,   the negative Schwarzian near the quadratic critical point   and  the idea of  maximal  space  expressed in terms of  poincare length  of  the  hierarchical intervals of the cantor set  allowed  an unconditional  proof that the geometric structure of the cantor set was bounded as one descended down the scales.

 During a  Rademacher Lecture  at U Penn in 1986 on period doubling Eugenio Calabi asked  what the bounds just mentioned depended on. This was a  great question. It turned out  one could see  if the smoothness of the map exceeded one plus the hausdorff dimension of the cantor set, the bounds were eventually universal.
These became known as the "beau" real bounds.
(bounded and eventually universally so).

  Benefitting  from the Guckenheimer paper and the Khanin-Sinai paper, a picture emerged that the renormalization  maps  were compositions of  very many strongly almost linear maps on tiny intervals ( and this part could be  ignored) plus a few well spaced  nonlinear maps on intervals on the slope of the mountain near the critical point and this entire structure or picture stayed bounded as one descended down the scales  by  iterated renormalization of the mapping.
(  definition of a renormalization :  iterating a power of two times , restricting to  appropriate intervals  and linearly rescaling up to standard size).

One could  now  advance a little and see  using  these  "beau" real bounds that  firstly, limits of renormalization became real analytic for critical exponent two and secondly, that the inverse branches had complex extensions to the entire upper half plane. These  mappings comprised  the Epstein class because Henri Epstein had studied these numerically and mathematically . This is revisited below.


 Where does the  actual convergence come from?

In the meantime  Douady and  Hubbard  had been studying  the complex quadratic family and the little Mandelbrot sets in the  big Mandelbrot bifurcation diagram ( now readily available using computers) were explained by them to be the effect of renormalization. This  renormalization   being  associated to some  iterate of the critical point  recurring near to itself while a tiny neighborhood wrapped  twice over  a larger neighborhood.
 A crucial, but beautifully coordinate free , invariant (the modulus of a domain diffeomorphic to an annulus) was the amount of extra wrapping space one had around the critical point of the renormalized mapping.

The  beau real  bounds , the Epstein class limits, the classification by
Douady Hubbard of the just described quadratic-like mappings in the
period doubling context  \&  a general discussion of scaling functions on
dual cantor sets were respectively described in  the 1986 ICM  report  as
two  of four appearances of  quasiconformal mappings  in geometry,
topology and dynamics.

 (Explanation  of "quasi conformal " in dimension one: first note any diffeomorphism in real dimension one is infinitesimally  scaling , possibly composed with flipping   and this is a similarity transformation  ie a diffeomorphism in real dimension one is conformal.  Next  the beau bounds showed  topological conjugacies between smooth  period doubling mappings were more than just homeomorphisms between the critical orbit  cantor sets, they were the one dimensional versions of quasi conformal mappings , called quasi symmetric mappings ( namely the just mentioned  infinitesimal similarity of diffeomorphisms  is only boundedly distorted at a fine enough scale. These mappings have lot of properties  but  they do not  preserve the lebesgue measure class  as opposed to  their higher dimensional  quasi conformal analogues. )

The quest to explain  period doubling renormalization convergence became a combination of searching for  a  natural domain of  definition for renormalization  together with a concomitant deformation principle that would  force  convergence toward  a fixed point.

The convergence into the Epstein class  plus  an analysis of the geometry of the Epstein limit suggested one  might find the natural renormalization domain using the Douady -Hubbard classification of  "quadratic like  holomorphic mappings".

Their classification was  by a  real analytic expanding dynamics of degree two  on the circle.  This described the germ of complex dynamics just outside the complex julia set.
Out of this data  it was  natural to construct  the  inverse limit of the two to one  circle dynamics, ie the natural extension, and from its germ of complex extension minus  the inverse limit itself , modulo the  action of the extension to obtain a solenoidal Riemann surface lamination ( which means locally a Riemann surface cross a totally disconnected space). This Riemann surface lamination with its complex structure classified the original quadratic like mapping in its (hybrid) quasiconformal deformation class.

 One  can build a separable Banach manifold  Teichmuller space for  compact solenoidal Riemann surfaces with leaves whose universal covering space is the disk.

Using the Teichmuller description of tangent space and cotangent space in terms of Beltrami coefficients and holomorphic quadratic differentials one   can   build a convergence criterion to a fixed point  given   "beau complex bounds" on the  available modulus of an annulus  around the critical point.

 In that time period  Yacob Sinai  suggested  a trip to the  Soviet  Union to talk about these developments. It seemed natural to try to  finish the proof first and this proof was taking its time.

 Then a  funny thing  happened. After months  without progress  getting  the annular  Koebe space around the renormalized to be quadratic like mapping, one day the idea came to use Thurston's method.  Namely, to get the picture  first , then to prove it.  For an  Epstein class mapping one  can take  the preimage of the upper half plane under one branch of the inverse getting a quarter plane. One then repeats this on the quarter plane, getting another right angle in the domain with a certain picture. One repeats again and again to get  more and more  corners . Pretending to be Thurston  one could  drew a picture of what the evolving domain would look like without worrying about how to really prove it. It was a nice picture of a scalloped figure with a roughly geometrically self similar  deployment of corners. Ok, but is this what happens?
Then remembering  Henri Epstein's work  on these mappings  I  bolted up the stairs at IHES to his office to find him. He was not there, but the door was open as all doors were at IHES. There  was his preprint stack on this work  and  opening  one copy to  his  figures the computer drawing of the renormalized mapping  was very much like the figure just drawn on the blackboard downstairs.

Hooray! One had the  picture to be proved.

It  still took time  and  the last  moment  happened at the Orsay pool where Welington deMelo was relentlessly  asking   about the proof.  While describing  the scalloped  domains in the upper half plane  there was a pause and Welington said "Isn't that part of the  map injective".  It was injective and suddenly the proof was finished by the Koebe distortion theorem.

The complex bounds were achieved and they were also " beau."  The three steps required were complete: the beau real bounds , the beau complex bounds  both needed to show every sufficiently smooth  real quadratic mapping was limiting to  a holomorphic quadratic like mapping with Koebe space bounded from below, The natural Teichmuller space with its differential calculus in terms of holomorphic quadratic differentials and deformation theory gave a principle that forced convergence to a fixed point.

This gave convergence ,  not  yet the  exponential rate of convergence  predicted by Feigenbaum,   but this convergence  was enough to obtain   an  analogue of Mostow-Herman rigidity that a topological or combinatorial condition could impose geometric rigidity.
 The critical orbit Cantor sets  of $C^2$ smooth period doubling maps are  geometrically rigid . They are  all  have  the scaling function of the solution to the functional equation,  as described in the paper of Khanin-Sinai , since they converge under renormalization to this fixed point.




 In the dynamics seminar at Orsay,  Maguerite Flexor wrote up a French exposition  of the proof  of the complex bounds.  The  solenoidal Riemann surface  description  went to  the Milnor 60 fest  volume.  Welington  deMelo and Sebastian Van Strien wrote the  book   "One dimensional dynamics" that  treated  the  entire proof.

Now it was 1990, eight years after Tresser's remark. The Soviet Union had changed  dramatically.

  Misha Lyubich  had come from the Ukraine.
He was interested  and involved in all of the ideas that had emerged so far in complex dynamics.
   Misha  made a  dramatic  improvement  to the situation. By adopting an appropriate  complex coordinate viewpoint for complex mappings and using  the proved convergence , the beau  real bounds and the  beau complex bounds using holomorphic contraction he showed the  exponential convergence justifying the Feigenbaum  numerical discovery."

Holomorphic dynamics in one variable has  flourished since that time.

At a talk at IHES   around this time by physicists using fourier modes to study the 3D Navier Stokes Equation one learned about the lack of a mathematical answer to the question of long time existence of solutions in certain classes. That  this beautiful and very useful equation held such mysteries was astonishing and this question was well worthy of study.   For example, why is this 3D problem so hard? The same Navier Stokes  problem in 2D could be treated by exactly the same analysis tools that treated the Beltrami equation so important in the Ahlfors Bers treatment of Teichmuller theory.  In addition half of the non trivial theory of quasiconformal mappings extended to higher dimensions as in Mostow rigidity for  3D. One could start over studying in this domain.

  The  trip to Russia  happened finally  at Rochlin's centenary Congress last August in St. Petersburg and the talk was on geometry in 3D.

\begin{center}
The end
\end{center}

\item[2019-12-29 Predrag]
For me the story all along and always has been about the language /
mathematics we need to develop to understand instabilities (patterns,
shapes) of infinite-dimensional strongly nonlinear field theories;
Quantum ChromoDynamics, fluid turbulence, neuronal systems.

Mitchell's role was crucial - here and there, a simple, provocative
rhetorical question which would puncture the snugness of my quantum-field
theorist indoctrination, then reshape decades of my work.

A few jogs to your memory

* To put it in a very polite way, you were the only pure mathematician I
learned something original from about renormalization - as
\HREF{http://ChaosBook.org/~predrag/papers/JPA23-90.pdf} {JPA23-90.pdf}
says: "The basic idea of relating the Feigenbaum constant $\delta$ to the
scaling (or the presentation) function is due to Sullivan". I wrote that
up in Sect 4 Period doubling repeller of
\HREF{http://ChaosBook.org/~predrag/papers/AAC-II.pdf} {AAC-II.pdf}, and
you published it later with Jiang. What you taught me is that on should
keep the signs of the period-doubling operator cycles (ie, the sign of
the slope of the negative-slope branch of the presentation function in
dynamical zeta-functions weights), and not take the absolute value, as
Ruelle did - in statistical mechanics all weights / probabilities are
positive.

I have a clear memory of the moment I understood you - you gesticulating
that when you fold, you subtract. We might have been standing in a room
in IHES, or in your apartment. That's all it took. Deep. Keep the minus
sign, kid!

* Do not forget pruning fronts (there are 2 videos of my talks on that in
your video collection). How to count the admissible patterns is a crucial
part of the turbulence puzzle. Andre de Carvalho made me so happy by
showing that my hand drawn illustrations of pruning front -  Figure15.11
here: \HREF{http://ChaosBook.org/chapters/smale.pdf} {smale.pdf} were
(and still are - have not fixed the drawing yet) impossible.

\item[2019-12-29 John Guckenheimer]  % <jmg16@cornell.edu>
a story about Feigenbaum and the history of period doubling:

At a 1982 conference in Sitges, Mitchell and I discussed differences
between the way mathematicians and physicists approached their research.
He made two comments that stuck with me:

1) Given the choice between a rigorous proof for the existence of the
period doubling ratio and the fixed point function vs an algorithm that
would compute these quantities to arbitrary precision, he preferred the
algorithm.

2) Theoretical condensed matter physicists all converged on the problem
of the day, published ideas still being developed and expected 90\% to be
wrong. Mitchell was comfortable that things which were correct would survive
and the rest would be discarded. Mathematicians, on the other hand, went
to great lengths to (unsuccessfully) avoid mistakes. I think this
comparison is an intriguing topic in the sociology of science.

\newpage %TEMP
\item[2018-05-03 Bj{\"o}rn]
\HREF{https://arxiv.org/search/?searchtype=author&query=Winckler\%2C+B}
{Winckler}
{\em The {Lorenz} renormalization conjecture}\rf{Winckler19}, \\
\arXiv{1805.01226}
(should also have a look at Martens and Winckler\rf{MarWin16} {\em
Physical measures for infinitely renormalizable {Lorenz} maps}, with a
``more complete list of references'')
writes: ``
Renormalization in low-dimensional dynamical systems [...] which adhere
to this paradigm, such as unimodal maps
\cite{Sullivan92,McMullen96,AviLyu11}, critical circle maps
\cite{deFaria92,Yampolsky03} [...]
''
\\(Predrag: I get it, the deal is not to cite the original papers,
but the more recent papers where it is done right)

``
Renormalization was introduced to dynamics by Coullet and
Tresser\rf{CouTre78} and independently by Feigenbaum\rf{feigen78}.
Working numerically, they discovered that unimodal maps on the boundary
of chaos exhibit universal phenomena and showed how this could be
explained by an associated dynamical system acting on these maps, called
the period-doubling operator, having a hyperbolic fixed point.
''
\\(Predrag: So, Mitchell does this in 1975 - see
\HREF{http://chaosbook.org/~predrag/papers/universalFunct.html} {here}
and \HREF{http://chaosbook.org/chapters/appendHistRemA1_2.pdf} {here}, our
friends Charles and Pierre do it a couple of years later, and somehow
``Renormalization was introduced to dynamics by'' them?)

``The former authors also predicted that these phenomena could be
measured in physical experiments and this has since been confirmed in
many different settings Maurer and Libchaber\rf{MauLib79}, Linsay\rf{Linsay81}.
''
\\(Predrag: again, not the historical sequence, or why would Feigenbaum
and Libchaber, and not somebody else, share a Wolf prize, were it not
that Feigenbaum's theory was experimentally confirmed by Libchaber?
BTW, Linsay\rf{Linsay81} is not the obvious reference - he was not
in my selection of reprints of experimental papers\rf{cvt89b})

From then on I enjoy reading the paper, Sect.~1.1. {\em Informal overview
of renormalization}, Figure~2. \emph{Examples of three different classes
of the first-return maps}, etc.

\item[2020-03-15 Predrag] Shi\rf{Shi19} {\em Multi-modal-extended
solutions of the {Feigenbaum} equation} has a useful, perhaps up-to-date
overview of the literature in the introduction.

\item[2020-03-15 Predrag] Read



\end{description}



%\newpage %%%%%%%%%%%%%%%%%%%%%%%%%%%%%%%%%%%%%%%%%%%%%%%%
\printbibliography[heading=subbibintoc,title={References}]
