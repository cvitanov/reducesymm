% reducesymm/UFO/history.tex
% Predrag 2019-07-28


\chapter{Historical notes}
\label{c-history}

\begin{quote}
Somehow the wondrous promise of the earth is \\
that there are things beautiful in  it, \\
things wondrous and alluring, \\
and by virtue of your trade you want to understand them. \\
Mitchell J.  Feigenbaum [G1eick, p.~187]
\end{quote}

\section{Bibliography}
\label{sect:Others}

\begin{description}
\item[2019-07-27 Predrag]
starting to collect ``about" MJF bibliography

\rf{AnBeMe19}
{An, Bergada and Mellibovsky}
{\em The lid-driven right-angled isosceles triangular cavity flow},
{2019}
is an example of a contemporary fluid dynamics in which the transition to
chaos follows the period-doubling scenario. However, authors say that the
``details are beyond the scope of this study'' so it is not clear how
many doublings they have observed.

\rf{AubDal02}
Aubin and Dalmedico
{\em Writing the history of dynamical systems and chaos: {Longue}
dur{\'{e}}e and revolution, disciplines and cultures},
{2002},

\rf{AviLyu06}
{Avila, A. and Lyubich, M.},
  {\em Examples of {Feigenbaum Julia} sets with small {Hausdorff} dimension},
{2006},

\rf{AviLyu07}
{Avila, A. and Lyubich, M.},
  {\em Hausdorff dimension and conformal measures of {Feigenbaum Julia} sets},
{2007},

\rf{AviLyu15}
{Avila, A. and Lyubich, M.},
{\em Lebesgue measure of Feigenbaum Julia sets},
{2015},

\rf{Barbaro07}
{G. Barbaro},
{\em Formal solutions of the {Cvitanovi{\'c}-Feigenbaum} equation},
{2007},

\rf{Briggs89}
{Briggs, K.},
{\em How to calculate the {Feigenbaum} constants on your {PC}},
{1989},

\rf{Briggs91}
{Briggs, K.},
{\em A precise calculation of the {Feigenbaum} constants},
{1991},

\rf{BDS98}
{Briggs, K. M. and Dixon, T. W. and Szekeres, G.},
{\em Analytic solutions of the {Cvitanovi{\'c}-Feigenbaum} and {Feigenbaum-Kadanoff-Shenker} equations},
{1998},

\rf{Broadhurst99}
{D. Broadhurst},
{\em Feigenbaum constants to 1018 decimal places},
{1999},

\rf{Buff99}
{X. Buff},
{\em Geometry of the {Feigenbaum} map},
{1999},

\rf{CamEps81}
{Campanino, M. and Epstein, H.},
{\em On the existence of {Feigenbaum's} fixed point},
{1981},

\rf{CaEpRu82}
{M. Campanino and H. Epstein and D. Ruelle},
{\em On {Feigenbaum}'s functional equation {$g \circ g(\lambda x) + \lambda g(x) = 0$}},
{1982},

\rf{coppersmith:52}
{S. N. Coppersmith},
{\em A simpler derivation of {Feigenbaum}'s renormalization group equation for the period-doubling bifurcation sequence},
{1999},

\rf{CouPom16}
{Coullet and Pomeau},
{\em History of chaos from a {French} perspective},
{2016},

\rf{cvitanovic1984universality}
{Cvitanovi{\'c}, P.},
{\em Universality in chaos (or, {Feigenbaum} for cyclists)},
{1984},

\rf{DudYam15}
{Dudko, A. and Yampolsky, M.},
  {\em Poly-time computability of the {Feigenbaum Julia} set},
{2015},

\rf{Groen86}
{J. Groeneveld},
{\em On constructing complete solution classes of the {Cvitanovi{\'c}-Feigenbaum} equation},
{1986},

\rf{EpsLas81}
{Epstein, H. and Lascoux, J.},
{\em Analyticity properties of the {Feigenbaum} function},
{1981},

\rf{FrKhaMa03}
{Frisch, U. and Khanin, K. and Matsumoto, T.},
{\em Multifractality of the {Feigenbaum} attractor and fractional derivatives},
{2005},

\rf{Gaidashev11}
{Gaidashev, D.},
{\em On analytic perturbations of a family of {Feigenbaum}-like equations},
 {2011},

\rf{GaJoMa16}
{Gaidashev, D. and Johnson, T. and Martens, M.},
{\em Rigidity for infinitely renormalizable area-preserving maps},
{2016},

\rf{GrMaViFe81}
{J. M. Greene and R. S. MacKay and F. Vivaldi and M. J. Feigenbaum},
{\em Universal behaviour in families of area-preserving maps},
{1981},

\rf{KuzOsb02}
{P. Kuznetsov and A. H. Osbaldestin},
{\em Generalized dimensions of {Feigenbaum}'s attractor from renormalization-group functional equations},
{2002},

\rf{KuKuSa05}
{Kuznetsov, S. P. and Kuznetsov, A. P. and Sataev, I .R.},
{\em Review and examples of non-{Feigenbaum} critical situations associated with period-doubling},
{2005},

\rf{Lanford82S}
{{Lanford}, O. E.},
{\em A computer-assisted proof of the {Feigenbaum} conjectures},
{1982},

\rf{Lilja17}
{Lilja, D.},
{\em On the invariant cantor sets of period doubling type of infinitely
renormalizable area-preserving maps},
{2017},

\rf{Lyubich99}
{Lyubich, M.},
{\em {Feigenbaum-Coullet-Tresser} universality and {Milnor}'s hairiness conjecture},
{1999},

\rf{ManZoi18}
{Mancusi, D. and Zoia, A.},
{\em Chaos in eigenvalue search methods},
{2018},

\rf{Mathar10}
{Mathar, R. J.},
{\em {Chebyshev} series representation of {Feigenbaum's} period-doubling function},
{2010}, has many period-doubling references that are not yet in this list.

\rf{Molteni16}
{Molteni, A.},
{\em An efficient method for the computation of the {Feigenbaum} constants to high precision},
{2016},

\rf{NKHM17}
Noble, Karimeddiny, Hastings and Machta,
{\em Critical fluctuations of noisy period-doubling maps}
{2017},

\rf{OlLe11}
{Oliveira, D. F. M. and Leonel, E. D.},
{\em The {Feigenbaum}'s {$\delta$} for a high dissipative bouncing ball model},
{2011},

\rf{OprDan19}
{Oprea, I. and Dangelmayr, G.},
{\em A period doubling route to spatiotemporal chaos in a system of
{Ginzburg-Landau} equations for nematic electroconvection}
{2019},

\rf{Pinotsis10}
{Pinotsis, A. D.},
{\em Infinite {Feigenbaum} sequences and spirals in the vicinity of the
{Lagrangian} periodic solutions}
{2010},

\rf{Pollicott91}
{Pollicott, M.},
{\em A note on the {Artuso-Aurell-Cvitanovi{\'c}} approach to the {Feigenbaum} tangent operator},
{1991},

\rf{SKSF15}
{Savin, D. V. and Kuznetsov, A. P. and Savin, A. V. and Feudel, U.},
{\em Different types of critical behavior in conservatively coupled {H{\'{e}}non} maps},
{2015}

\rf{Sezgin06}
{F. Sezgin and T. M. Sezgin},
{\em On the statistical analysis of {Feigenbaum} constants},
{2006},

\rf{Skiadas16},
{Skiadas},
{{The Foundations of Chaos Revisited: From Poincar{\'{e}} to Recent Advancements}},
{2016}
\\
has a number of essays important for writing up the history of ``chaos'':
Ferdinand Verhulst, Jean-Marc Ginoux (I did not know that Poincar\'e
established existence of sustained oscillations necessary for radio
communication, and that the first International Conference of Nonlinear
Oscillations took place in 1933) and
Coullet and Pomeau\rf{CouPom16}.

\rf{TRGFOL15}
{Teixeira, R. M. N. and Rando, D. S. and Geraldo, F. C. and Filho, R. N. C. and de Oliveira, J. A. and Leonel, E. D.},
{\em Convergence towards asymptotic state in {1-D} mappings: {A} scaling investigation},
{2015},

\rf{Tsygvintsev_2002}
{A. V. Tsygvintsev and B. D. Mestel and A. H. Osbaldestin},
{\em Continued fractions and solutions of the {Feigenbaum-Cvitanovi{\'{c}}} equation},
{2002},

Victor Varin {\em Spectral properties of the period-doubling operator}
\arXiv{1202.4672} is unhappy with received wisdom (``This article was
declined by "Nonlinearity" on the ground of `Conflict of Interests',
since it criticizes implicitly some members of the editorial board'').

\rf{VulKha82}
{E. B. Vul and K. M. Khanin},
{\em The unstable separatrix of {Feigenbaum}'s fixed-point},
{1982},

\rf{VuSiKh84}
{Vul, E. B. and Sinai, Ya. G. and K. M. Khanin},
{\em Feigenbaum universality and the thermodynamic formalism},
{1984},

\rf{mathwFeigF}
{Weisstein, E. W.},
{\em Feigenbaum function},
{2012},

\rf{WeissFeig}
{Weisstein, E. W.},
{\em Feigenbaum constant},
{2012},

\rf{WeOv94}
{Wells, A. L. J. and Overill, R. E.},
{\em The extension of the {Feigenbaum-Cvitanovi{\'c}} function to the complex plane},
{1994},

\end{description}


There is a Rashomon of histories ahead

\begin{description}

\item[1989-05-17 Rhonda Roland Shearer]
{\em Bronzes - Chaos, New Science, New Art}\rf{Shearer89}

FOREWORD
by Mitchell J. Feigenbaum

The deep goal of physics has been to determine just what a description of
an object should be. It has invariably entailed, at least, enumerating
the locations of each "point" of the object in a continuum. This scheme
of depiction works altogether well when the object is just a few such
points, or alternatively when the object is many such points-adjoined to
comprise a regular geometric figure that is composed of a small number of
edges , or geometrical solids. Indeed, in order to test the rectitude of
the laws of physics, one eschews all complications in order to find the
simplest, most symmetrical configuration, so that all predictions can be
rendered mathematically precise.

Our ambient human environment has chosen otherwise, with high symmetry
and sparse simplicity the shibboleth of that which is contrived. While in
principle the pointwise description of a tree with many thousands of
leaves-each somewhat different from the other and none a simple geometric
form - is possible, this manner of description is reserved only to a god.
It is fatuous to say that since the laws are verifiably true, this
description of a tree is only a technicality: the mere writing down of
this wealth of precise in formation transcends all human resources so
that the thought of any further analysis is academic.

Yet the leaves have a humanly knowable content - they are neither few in
number nor blitheringly many of a random variety, and we recognize them
easily.

To date, we have succeeded in rendering one alternative to
the pointwise enumeration of complex forms. These ``fractal" objects
enjoy not a simple pointwise description, but rather a simple building
rule. A new feature is built out of an old feature-always by the same
rule, but in proportion to the size of the original. Suffice to say that
variants on this scheme have developed that indeed reconstruct some of
the complexities observed in physically created objects. Nevertheless,
they don't make recognizably true trees or ferns-they're always a little
too regular in their complexity. Somehow, ideas must be adduced - perhaps
from the observation of nature-if we are to embrace under the sway of our
science the world we know so handily.

Ms. Shearer's offering prods me along just such lines. It is confronting
to see, in juxtaposition, the simplicity of geometry and the non-random
complexity of the forms of earthly flora. It is intellectually
embarrassing to see what is so craftily allowed to the synthetic arts but
forbidden to the analytic sciences. It gnawingly cries out that there is
something we almost know that we have to learn . And it is beautiful and
joyous.


\item[1995-01-13 (?) Lyubich]
email to Predrag explaining why Predrag's name is no longer attached
to the fixed point equation.

\item[2009-05-04 Mitchell to Predrag] email: % feigenb@mail.rockefeller.edu

Predrag,

Where are you?  Back in the US?  I'd like to call and talk, after quite a
while now.

Here's the all but finished
\HREF{} {tube paper}.  I bet you can make your way through
the preface.

Mitchell


\item[2019-07-27 David Ruelle] %<ruelle AT ihes.fr>
Mitchell invited me to Los Alamos after the Ruelle-Takens paper, and I
met Cornelia, his first wife (I don't remember the exact date).  We kept
very good personal relations, but I don't remember any anecdote worth
relating.  After his famous work on the period doubling cascade, I
collaborated to a small 1982 note with M. Campanino and H.
Epstein\rf{CaEpRu82} {\em On {Feigenbaum}'s functional equation {$g \circ
g(\lambda x) + \lambda g(x) = 0$}}, but I watched the work on
Feigenbaum's equation mostly from some distance.  It may be of interest
to retrace the mathematical work by Lanford, Sullivan, etc. on this
subject.


\item[2019-07-29 Predrag]
On 2015 Coullet and Pomeau\rf{CouPom16}
{\em History of chaos from a {French} perspective}:

The good thing about Wolfram having his essays online, is that he can
tune them a bit if you need to. The French school certainly deserves the
credit Pomeau and Coullet ask for, both for what they had accomplished up
to the time described in their essay, and beyond 1978 for the central
role IHES-connected mathematicians and mathematical physicists played
thereafter.

So does the Soviet school.

Personally, by far the most thoughtful writing about the process of
discovery, exemplified by Mitchell's own period-doubling story, is his
``Twentieth Century Physics'' essay\rf{BrPiPa95IIIFeig}.

Fortunately for Wolfram, it's not his job to write The History of Chaos
:) . Because, as I am learning now, beyond physics / math divide, there
is also ongoing trench warfare between pure mathematicians and
mathematical physicists. These lingering resentments, following in
Mitchell's footsteps, will too be soon laid the rest.

The hilarious, very French thing is how they essay ends. You can just see
them saying, of fuck, who cares, so they abandon it on a %comma

,

\medskip

PS The essay is published by a predatory publisher, so the published
version isn't proofread either, and ends on a coma

\item[2019-07-29 Gemunu]
According to \HREF{https://history.aip.org/phn/11610002.html} {Swinney}
(see Oral Histories interview), they had no idea about the Ruelle-Takens'
work when Jerry and he did the experiment. They were trying to prove
Landau's path, could never get to a third independent frequency. Joel
Lebowitz got them in contact with Ruelle who was visiting NY, but they
could not follow Ruelle's argument. Later, Joel got them in contact with
Lorenz, and that is when they realized what was happening.

\item[2019-07-26 Dennis Sullivan] % <sullivan0212 AT gmail.com>
% To: David Ruelle <ruelle AT ihes.fr>
I recall during a scientific meeting at IHES you and Oscar or J{\'u}rg
were playing around and not paying attention

I asked what you were doing and you guys showed me the Cvitanovi{\'c} functional
equation without the story

later you explained Mitchell's Feigenbaum's period doubling
numerical discovery

there was also Coullet-Tresser critical opalescence related discovery
mentioned below

at the time  many of us were focusing on Thurston-Gromov hyperbolic
geometry and  those with a dynamical bent on limit sets of Kleinian
groups

there was the idea with Rufus Bowen to study dynamical bifurcation on the
more explicit case of limit sets of Kleinian groups,
because an analogy could be made between that subject and the Fatou-Julia
theory of holomorphic iteration

Tresser came to visit me in NY to try out computations of the Poincar\'e
limit sets

This idea failed for various reasons related to the sad departure of
Rufus Bowen and the computer difficulties of large groups.

I really got involved personally in period doubling in 1981 when Tresser
told me the Hausdorff dimension  of the period doubling cantor set changed
when you changed the critical exponent.

I was shocked because of my experience with the horseshoe Cantor set
whose geometrical structure depended on the choice of the mapping.

I said you mean it doesn't change if you don't change the exponent.

He said of course and recounted his work with Coullet on critical
opalescence.

For me this was fantastic :
an infinite dimensional version Mostow rigidity and an extension of
Michel Hermann's solution of the Arnold conjecture.

I dropped all my projects and worked on this full time till 1990 to get a
math formulation and proof of geometric  rigidity \emph{of the period
doubling cantor set} (this was expressed in terms of the scaling function
on the dual Cantor set introduced by Khanin and Sinai\rf{VuSiKh84}), the
proof derived the beau bounds real and complex and also used a
Teichmuller space of laminations.

The proof only works until today for even integer exponents because of
the Teichmuller step (in this complex proof the concept of quadratic like
mappings due to Douady and Hubbard was crucial), but gave the convergence
of renormalization.

In the early 1990s an elegant device was introduced by Lyubich to use the
above information to get the exponential convergence thus proving the
Feigenbaum universality of his number $4.67\cdots$ .

I wish the physicists in the middle of things would describe the story on
the physics side so we could better understand what the meaning of all of
mathematics might be and perhaps find more mathematics including a real
variable discussion that would include all real exponents greater than
one.

\item[2019-07-27 Predrag]
In the above, Sullivan has forgotten to mention Collet-Eckmann
book\rf{eckman80map}, and Collet-Eckmann-Lanford\rf{collet80} {\em
Universal properties of maps on an interval} proof of Feigenbaum
equation for $|x|^{1+\epsilon}$.

\item[2019-07-27 Gemunu]
Sullivan writes: ``... this was expressed in terms of the scaling
function on the dual Cantor set introduced by Khanin and Sinai.'' But I
thought Mitchell invented the trajectory scaling function (which he was
very proud of) first, before Sinai and Khanin?

\item[2019-07-27 Predrag]
Vul, Sinai and Khanin\rf{VuSiKh84} 1984 {\em Feigenbaum universality and
the thermodynamic formalism} paper discusses the stable-unstable manifold
of the period-doubling operator fixed point, as well as the complex plane
period-tripling fixed point (contemporaneously with the independent 1983
Cvitanovi{\'c} and Myrheim work\rf{myrh1,CviMyr89}). There is also the
1982 Vul and Khanin\rf{VulKha82} {\em The unstable separatrix of
{Feigenbaum}'s fixed-point} that does not seem to get cited.

Feigenbaum introduced presentation functions in his 1988
paper\rf{Feigenbaum88} {\em Presentation functions, fixed points, and a
theory of scaling function dynamics}. The lack of physics community
interest in his presentation functions is indeed one of the reasons
Feigenbaum gave up on journal publications
(Safmann-Taylor\rf{Feigenbaum02} instability was the final nail in the
coffin).

``Dual Cantor set'' {seems to only} refer to the manner of labelling a
binary tree-organized Cantor set (see 1990 PhD thesis by Sullivan's
student
\HREF{http://qcpages.qc.cuny.edu/~yjiang/HomePageYJ/Download/1990PHDThesisYPJ.pdf}
{Y. Jiang}). I suspect this is what is called ``alternating  binary
tree'' in
\HREF{http://www.chaosbook.org/chapters/ChaosBook.pdf\#section.14.3}
{ChaosBook.org}.

The notion of ``dual Cantor set'' seems to have been introduced by
Sullivan\rf{Sullivan88} in the June 1987 Noto (or what Predrag calls ``No
to NATO'') conference\rf{GalZwe88}, where Feigenbaum, Siggia,
Cvitanovi{\'c}, Lanford, Swinney, Libchaber, etc. also spoke. Sullivan
writes: ``Feigenbaum expressed this discovery in terms of a universal
scaling function for the Cantor set. [...] the fine structure is codified
by a scaling function defined on a logically distinct perfect set - the
dual Cantor set. The main unsolved mystery is why the renormalizations
converge. '' (I have no access to the full lecture, so do not know what
other credit Sullivan gives to whom, but it also appears in the equally
unaccessible \refref{Wells88}).

In Birkhoff, Martens and Tresser\rf{BiMaTr03}
{\em On the scaling structure for period doubling} they write:
``For  completeness  and  to fix notations  and  definitions,  we include
some  basic discussion  of the scaling function,  whose origin is rather
diffuse:  first  conjectures  about a form  of it  appeared  in Coullet
and Tresser\rf{CouTre78}, the name and a  form  of it  come from
Feigenbaum\rf{Feigenbaum80}, while what  was  arguably the first theorem
about  it was in a never circulated  work by Feigenbaum  and  Sullivan.
The literature  on  scaling functions  is  extensive  and  discusses
scaling functions  beyond  the context  of dynamics.  In  particular,  in
1984 Vul, Sinai and Khanin paper\rf{VuSiKh84}  a  relation  with  the
thermodynamic  formalism  appeared.''
They do not bother to actually plot the scaling function they construct,
but as their ``universal period doubling scaling function $q$ is strictly
monotone and the range forms a Cantor set'' I think it is not the same
as:

Feigenbaum\rf{Feigenbaum80} {\em The transition to aperiodic behavior in
turbulent systems} constructs the scaling function $\sigma$ and plots it in
his Fig.~2.

The 1984 Vul, Sinai and Khanin survey\rf{VuSiKh84} is impressive. It does
discuss Feigenbaum attractor as a binary-labelled Cantor set, but I see
no ``dual" anywhere. ``Thermodynamic formalism'' is in the title, but
mentioned only once in the text. It is developed in sect.~4 in the usual
way. They note that the numerical work of Grassberger\rf{Grassberger81},
{\em On the {Hausdorff} dimension of fractal attractors}, gives a more
accurate number. Summa summarum, I see no scaling functions in Vul, Sinai
and Khanin survey\rf{VuSiKh84}. Seems to be something that was introduced
by Feigenbaum and Sullivan in different forms, but while interacting with
each other. I'm no wiser than that.

\item[2019-07-29 Gemunu]
Some of the trajectory scaling function papers:
\begin{itemize}
  \item
Feigenbaum\rf{Feigenbaum80}
{\em The transition to aperiodic behavior in turbulent systems} (1980).
This precedes the 1984 construction
of Vul, Sinai and Khanin\rf{VulKha82}.
  \item
Gunaratne and Feigenbaum\rf{GunFei85} {\em Trajectory scaling function
for bifurcations in area-preserving maps on the plane} (1985)
  \item
Aurell\rf{Aurell87}
{\em Feigenbaum attractor as a spin system} (1987),
a follow up on Feigenbaum\rf{Feigenbaum80}, removes a redundancy in
the scaling function $\sigma$, and shows vi a thermodynamical formulation
that Feigenbaum\rf{Feigenbaum80} and Vul, Sinai and Khanin\rf{VulKha82}
scaling functions are the same.
\end{itemize}

\item[2019-10-05 Predrag]
Tsygvintsev\rf{Tsygvintsev06}
{\em Int{\'e}grabilit{\'e}, renormalisation et fractions continues}:
``[...] analysis of meromorphic integrability of the classical three body
problem. Then the period doubling renormalization in the asymmetric case
is treated. Some results in the continued fractions theory are presented
including the solving of the Ramanujan's conjecture about convergence of
limit periodic fractions.''

\item[2021-02-11 Predrag] conversation with Dan and Gemunu\\
\emph{predrag/reports/meetings/Feigenbaum/}
\emph{210211-211transcript.txt}\\

He does a horrendous PhD\rf{Feigenbaum70} at MIT with a disillusioned physicist
called Francis Low, who used to be great. Low and Gell-Mann wrote a
fundamental paper the renormalization theory. But by this time he
was jaded and felt that particle physics was not going anyplace.

So he does a horrendous particle physics thesis which I discovered was
actually published\rf{FeiLow71}, I didn't think it was published. Just a
complicated calculation on current conservation, something
with perturbation theory that required technical skill, but nothing
that you fall in love with. And he's smart, even though he does not
look like a success in physics, but he's obviously much smarter
than the average. So Francis Low sends him off to Hans Bethe, to
Cornell. Hans Bethe is running this amazing department at Cornell,
and Mitchell excels in doing New York Times crossword puzzles. Most
days of the week, he does it in like in three to five minutes, and
then, on the difficult one, the Saturday one, he might take 15
minutes. He just does them. So everybody understands he is smart.
He also cares about Schubert and he cares about German literature
and stuff like that so we become friends. I am a graduate student,
he is a postdoc, because you know we care about things Cultural
which are not physics.

And, I'm a success story,  I'm very good graduate student I'm
better than Porcogrande says a recommendation which makes
Porcogrande hate me forever, for the rest of our lives.
Whereas Feigenbaum, you know, he gives good journal clubs,
but I don't think he publishes anything and Bethe doesn't know what
to do so, he talks to Paul F. Zweifel in Black Hole, Virginia.
And so Mitchell is sent off to Black Hole because they need somebody to do
some calculations in transport by neutrons in a nuclear reactors
or whatever.

Their fee so he's gone, you know his total failure in physics, has
been sent of.

He had a good start, you know city college MIT
Cornell but now he has been sent off to the Black Hole and

But one of the Cornell professors called Peter Carruthers gets
hired by what I call death lab it's called Los Alamos National
laboratory you get hired by death lab to build up theory facility
outside the fence where they could bring some intelligent people
and eventually caught them and move them, on the other side so
suddenly there was gammon

But suddenly Peter has a medieval prerogative to just hire people
instead of running them through faculty committees. There's no
way you could ever hire Feigenbaum for any job that was ran through
faculty committees, because be intelligent and bank whatever.

That doesn't cut it is a professional

You know, only known for New York Times crossword very long
coherence secrets sentences, as many clauses directly from life in
century, you know nobody's even able to speak like this, but you
know totally unemployable so Carruthers now that he is a Lord he
just can't see So he's rescued from Black Hole.

You know what's the moral story, I mean there would be no
Feigenbaum if this is end of it, you know working government who
does he work what you know this mathematician in Black Hole
Virginia busy

He also is currently that is more for here to Congressional medal
of honor or maybe presidential medal of honor because he must have
done something good making bombs

Anyways so he's rescued, then you know, then he gets a pocket
calculator because these places infinite amount of money, so they
can afford a Hewlett-Packard

Because hundreds and hundreds of dollars, whereas graduate pause
pause pause the titania which has to have a girlfriend who works on
the assembly line of Hewlett Packard who can get the half price of
one pocket calculator modest one, compared to forget about where
multiply and divide a interchange on the keyboard

that's you know that's how we get into doing period doubling you
know he's doing it, he can program hundred and 10 steps, but I can
do 63 or something

So anyway

And then you know if he takes a random problem that nobody asked
him to do because we're all supposed to solve the core confinement
problem, and he does some totally useless i'm not supposed to use
the word, but you know some things that nobody should be hired for
etc, so the moral stories, there is no moral story it's totally
random process yeah and it turns out that even very gifted
intelligent person can survive in spite of it yeah it's not like fi
goodbyes a social, you know he definitely knows how to batter all
people,

but

You know, he doesn't play by the rules at all, he shouldn't be
allowed to be you know, on the Faculty of institution to venture
record no horse that's only after you get both prize or something
it really takes a lot Whatever the for buzzwords yeah but it turns
out even Bush is new color he's been built out of new color have to
blame your been infected by Feigenbaum so you have some kind of
values which are actually What we're supposed to be doing in
academia Illegal things because you have ethics and that's Of
course not i'm sure if you get our hands on me Yes, somebody can
actual look at

\emph{210211-211transcript.txt}\\
and
\HREF{https://www.youtube.com/watch?v=J7-_uqNKavc} {CC}\\

uh i mean i gave a talk at rutgers once right like i mentioned some of
them i i said that talk to you i i probably have it still okay uh so i
mean think about even this uh this uh photocopying money right yeah yeah
you know i mean who would have thought about i mean which physicist would
have thought about this that this is you know and

you know totally random
reason why he made maps but he did make map

right yeah and then the
precision of the earth and with the odysseus was one of those stars

so i remember very very well it must have been the fall of 1996. i'm
virtually sure of that date i went to one of leo's meetings in chicago
and mitchell gave this i had me on the edge of my chair when he was
talking about the the two pan am fighter the twa flight that crashed off
the coast of long island yeah so he was scarce that this was hushed up
this is that's right by american military and it's another example and it
was a stunning lecture and i i had a the wife of a friend of mine was on
the flight so i felt somewhat involved in it yeah and um yeah so uh you
know it's another example of what you're talking about yeah and the other
thing is you know when he was here i believe last time in atlanta oops we
lost him when he was on a fight

anyway he talked about the fish eye you know the vision of a fish right
it's perfect and it was an amazing lecture and you know everybody who was
there remembers this is an amazing lecture he had to be an idiot not to
appreciate how amazing it was it was all you know done just
extemporaneously again in long sentences with many clauses all you know
all correct if you wrote them down and he was on a vision of fisheye you
know which was one of his preoccupations he one of his preoccupations
since los alamos days was how to make a perfect lens and how badly lenses
are made etc and so on so you know many things that he picked up we
consider engineering for him it was not a big problem because his
undergraduate degree was in electrical engineering and he didn't have
hangs up hang ups about that apparently

what sort of family background did he come from

father was a chemical engineer you know had a solid middle class job for uh some major company yeah which one the mother probably was a school teacher and she worked but you know she was a force of nature she was totally dominating and you know suburban or something yeah yeah and uh you know he had a brother and sister he never talked to his brother brother doesn't didn't talk to her family at all sister they had to deal with to get two of them because mother was going bananas for at least like 20 years until she died uh so it was a typical you know jewish family from whatever this neighborhood in new jersey no no it's in new york but uh new york city but on the on the you know shore of the beginning of the long island whatever it is not rockaway or something that that's fine that's fine man yeah but you know it's this kind of neighbor i think also for mitchell i might be wrong but yeah remember oh it was the same kind of you know take a train to school and then go to city college and one of the many city college amazing you know students from this kind of background so it was uh you know he had a good relationship to his father he liked to talk to him

so it was nothing strange and you know they let them do things like they let him put

you know wagner lps onto the lp player when he was or you know or 78 whatever these things were called you know when he was a three-year-old or four-year-old or something yeah and he hated uh

post-puberty kind of school because he thought all these boys were ideas they only care about baseball or soccer whatever the [---] they care about so uh he had no use for them and you know he got happier by the time he got to mit so yeah he went to samuel tilden high school in brooklyn i see i don't know that one i don't know he's born in philadelphia

you must know that not really but yeah oh yeah but you know he was a typical jewish kid of the period yeah i live and uh you know his main influence at mit was the neuroscientist um the one who discussed him with hillary um i can't i know who you mean yeah you know there's a famous uh debate with timothy

on one of the big auditoriums at mit so that was actually the main you know intellectual influence not francis law i think

when he was a student then you know he got himself a belgian girlfriend and learn how to make belgium pork chops and stuff like that and they're forever friends

maybe she looks fancy at some point you could talk about some of this stuff you know i mean it's um

idea of talking about the uh you know the various subjects he worked on would probably interest a lot of people who don't know yeah what i found interesting is that you know everybody was convinced that he was a one idea man and did nothing uh other than period dublin which is just not fair because he did 100 random things and for some you know yeah for this one resonated because there was lots of uh physicists who have been trained in renewalization theory and they had nothing to do right i mean they were they ran out of steam so that hit that resonance kind of accidentally i mean but just happened to i don't know if you can say that at the meeting but i don't mind i mean nobody nobody will fire me at this point yes i can't do anything to me but you know yeah exactly so i would say it then i mean i would point out yeah that that's uh it sounds to me like a perfectly valid observation yeah you know it's and it's interesting you know he was really not a mathematician he didn't care about proving theorems but that work you know live lives in mathematics rather than in physics today not so uncommon is it that it goes like that well you know it's easy to teach it in non-linear courses but you know people don't know what you know it's just about part of life now because if you have some bifurcations onset okay it's one of the things that can happen and everybody knows it was no what i mean it seems not so uncommon that after it becomes basic knowledge then it becomes um a sort of thing among some mathematicians

what do you have in mind well renormalization like um you know from reading conan's piece that you sent me yeah you know that it became you know this you know this problem of deep mathematics when as a physics problem it had been sort of you know understood yeah but that's the one example i know but you know oh another example i i mean i it was uh several years ago i read some you know piece by mathematician on the kpz equation and he said there was this big renaissance you know in the mathematics yeah in love nourish something it wasn't that but that's that's another thing you could say the loner thing i mean that was what that's what conan mentioned the um the the sle equation what is it schramm loner

which doesn't look very physical when you actually look at it yeah

i know a little bit about that i mean the loner stuff was a very physical sort of system and thereby by the time it became sram loner for somebody like me it doesn't seem very physical anymore but it gives rise to interesting problems yeah yeah

because now that you mentioned i can't think of physics influencing mathematics really you know i thought i thought until beginning of 20th century these are the same subject right but

natural philosophy

well i mean a lot of models in statistical physics for example they become math problems i mean they become problems that mathematical physicists work on and you know yeah but you you like me mathematics you go to oh i don't know but there's this group of mathematical physicists who don't strike me as doing any physics yeah yeah no i understand you know conformal theory this and that i mean there's lots of math being used uh yeah you know so written atiya made sure that written got a feels metal

and i i just don't know whether this really affects you know what too oh i thought i don't know

so you're suggesting that these mathematical physicists i'm speaking of are sort of um irrelevant no i mean it's technically very hard and you know it's we can appreciate it you know some of the proofs they can do and stuff like that no i'm just thinking that um

you know i view ourselves as users of mathematics you know in the sense of being a language that we

code everything in right but uh that we actually have created mathematics that that is developed to me renormalization theory is kind of i think created ideas that have become mathematical objects of interest to mathematicians i think in minimalization that's quite true right and that's what i that's my impression yeah um

but you know if you've

hey

anyway you know mitchell was very good in let's say complex analysis and you know lots of mathematics he taught mathematical methods of course in rockefeller that probably killed everybody and you know i didn't i didn't know you had to teach them he might have even volunteered to do it

sorry my internet crashed okay we don't see you yet there you are yeah what happened internet crest oh really yeah okay so we're talking about you know mitchell and mathematics so um at rockefeller he taught kind of math methods course which was mostly optics so what did he teach there was some course he taught at some point there's some some yeah he did he did stochastic processes and something i i remember well that he did earlier because of the financial markets you know because of his work so he you know he gave nice lectures on ito calculus right yes in copenhagen that one of our students or postdocs wrote down so we have them on the internet from company i'm going to send you uh i just looked at this uh talk that i had but i don't have uh i haven't written anything i just put some pictures and talked about it so anyway if you want to i can probably write them so you know there's a good finkelstein story

which is uh an edge beagle i hope i get i'm getting it right so

it totally look you have a bunch of transparency i'll flip them on in a random order and you talk

and because i gave that talk so you know i had was just throwing a random transparency from a stack of transparency and finkelstein just went on smoothly without skipping your step

so before my graduate students sometimes when i'm going through some particularly awful literature i say you show me the title i'll tell you what's in the paper

anyway the only thing that then and i could think about is that renovation theory has gone into deep mathematics you know started in physics but went that way are the other things that physics has created that you know created significant mathematics because i view myself as a user of mathematics

okay so what uh maybe random matrix theory i mean that seems to be a branch of mathematics numerics yeah well uh i mean of course gaussian gaussian surfaces and riemannian surfaces were important for einstein right no no we're not talking in those days they were not separate you know they they got really divorced someplace in gerting and in 1910s and 20s until that time you know everybody had education in both

i'm thinking more of a late 20s oh i mean what what about chaos for example you know take a guy like lorenz lorenz was just um you know basically trying to show that you couldn't predict the weather right yeah and um and then the you know know about him but still it had a great effect i guess i have to define you know what mathematics means when they take the subject that we know intimately for example we have created it and made it so deep that we don't understand anything yeah that's called mathematics that's right yeah i mean chaos has it happened you know it's ergotic theory it's a serious theory so maybe you're right but ergonic theory precedes physics scales right it's a care suggestion injection from the side right maybe right

so where did organic theory start well usually we think of burkhoff in 1930s or so well for neumann berkoff yeah teacher of lorenz i think that's right or maybe it's the son or not it was no it was berkoff masters thesis supervisor yeah but the motivation came from statistical mechanic mechanics right and and maxwell and boltzmann right yeah right

i mean it took a while to get there but yeah things move more slowly

but you know this thing's like super super whatever string no way you know the name changes what to call so that seems like a very ugly application of beautiful mathematics you know i i don't see it as something that drives mathematics it's just you know no but

the ideas of bifurcation theory and symmetry breaking really came from physics right like from euler euler but then later taylor's work

when i read some russian texts they say it all comes from leningrad

or this is this particular lab of radio electronics somewhere in moscow really yeah well people you know like andrew knox of the andrew bifurcation all of the for all the names we know there's all it's always a russian name there's always a russian name for that thing yeah remember all of these things started with uh taylor's experiments right uh on this convection yeah but you know that's applied math do mathematicians care about applied methods

no i mean it is i see your points yeah

anyway it doesn't matter but uh you know mitchell's work had influence surprisingly in mathematics rather than in physics right yeah as a lasting intellectual effort it's here

as a continuing intellectual operation yeah yeah i mean in physics it's kind of part of corpus by now you know that's part of the things you need to know but there hasn't been significant

developments i mean it's not a bad thing that's okay so it's okay to be part of a corpus

it's pretty good to have your own number the funny thing about his own number is you know it was so hard to for me to comprehend you could have own number that after i wrote the universal equation blah blah then i spent

weeks a month yeah institute has an amazing library because it is instituted advanced study yeah you know going through all kinds of things just to see the composition of two functions giving the function back because you know how could not this be written it's all

seriously that's what i would have done too i was sitting in this library beautiful library you know uh and just go and talk all kinds of things books you know because at this stage you don't have the word for it right i mean what are thing so some kind of functional composition or iteration of functions or something okay so i have a story for you on that yeah remember george george schmidt right the guy who was that somewhere in new jersey yeah yeah i remember that this guy did the hamiltonian uh renovation group right yeah and then it turns out there's a transition from hamiltonian to no no that's what they did the transition from hamiltonian to distributive system okay and that's how i can run your universe so okay it turns out that the number is two two two yeah uh so

you can't call two schmidt number right

the little boy schmidt number

yeah that's right that's part of being lucky

who invented two two well maybe bull you know it's the basis of boolean base two so it could be called bull schmidt number

7th century lineage of number two

she actually has i think he has a paper on number pi over four already oh yeah my dad yeah no because you know it's a it's an air if i know it's important not not in there it's stokes yeah the stocks stocks motivating pi over four three and five get more attention i think oh yeah yeah because but imagine the world without number two

well two gets to be ordinary

now the big question is one a prime or not that's the

question and and who gets credit for that well it's obviously god there is the word and there is one nameless arab gets credit for zero

you know the wonderful thing and i totally recommend if you want me to send you the list to uh there's this very good russian mathematician at rutgers who you all know yeah relatively young one who is very good popularizer or mathematics at a good mathematician anyway in in his course on complex analysis he gives the history of i and it's beautiful uh it's totally beautiful so the interesting thing that uh turns out is you know it comes in renaissance italy and it's a blood sport i mean people kill each other for you know for the secret uh roots how do you get it solve a quartic equation and stuff like that and you know there's a whole history of duels and people going basically like you know wrestling federation you go to florence and you have a two of you are solving uh quarter equation you know side by side and who wins wins and that's it so they had to use square root of minus one eventually they figure out if they use it it's easier to do this stuff so they invented i now that was very much my line so the cart you know called it imaginary as an insult and he renamed other numbers of real numbers just to make distinction clearly between you know real numbers and what's imagining uh but what's the funniest thing about this renaissance italians they couldn't deal with a num negative numbers minus one so they invented imaginary but they couldn't do negative because the way to solve these equations is they did it geometrically they're creating you know shapes and volumes in higher dimensions which they're putting together to solve the equations so you know they really thought of this solving these equations geometrically and it's very hard to think of negative volumes so when they stated solution or cubic or quartic it was given as a list of cases in the cases were designed in such a way that you would put on the right hand side of equation everybody who would be otherwise minus in the left-hand side equation so all everybody was positive and then they worked and sold you know each case separately just because you know either they did not know minus or they did not feel comfortable with manipulating you know the way they thought about these things so that's you know amazing thing and this minus it only took vessels danish mathematician vessel not to be a nationalist and and this other german called gauss you know they made it a real thing i mean since then we never looked back but but in the beginning it was so difficult so i always wondered you know why is this totally obvious thing called fundamental theorem or algebra you know it says that the number of roots of polynomials is order upon a molar yeah so what's fundamental about that you know why is that it but it turns out yeah you know this is truly revolutionary to realize that uh roots were there they're just complex yeah they're complex it was fundamental but you know looking back we are so used to complex numbers we don't think about it

so let's talk about quantum mechanics and firearms so he had strong opinions and everybody was a schmuck but uh people who thought and i i'm sorry but you might remember i forget it's particles or maybe those waves and everybody else is a schmuck and you know i didn't i listened like left here and forget that is right here so i don't know whether he that the only way to understand it is this particles and not the waves or vice versa gamma do you remember uh i mean i he came to houston once and i had i really invited him to dinner and there was a particle physics guy who worked that son and he was talking to him for hours telling him about these particles and waves i don't remember i wasn't listening so

yes that's one time he went to extended discussion about no i mean he was very opinionated on the subject you know for me it's it's one or the other you know i mean i can always take a full respect and what the hell is the problem but for him it's a very tangible thing i mean it was really yeah i remember and you know he had a strong opinion which i don't remember what it was

and then those things he obviously wasted these you know words on on uh i shouldn't have talked to somebody more serious but go ahead no no so i mean he thought that nuclear physics you couldn't do anything with it right you solved the single particle case and then then the others you couldn't do it

i mean if you take a real new piece you cannot solve it right and yeah so what was the point of doing this was was one of his complaints and yeah i had to ask my friend what he what he told him about

i was the host so i was taking care of all things and i heard them talking because you know usually i don't think i'm literally in quantum physics context but and the other thing there was a recurring theme and you know this always failed totally we were somehow supposed to use our brains on you know biology or physics and biology but we were always disgusted with every single talk on the subject right yes so one thing one thing he told me once was that uh that he didn't think we are ready for that

to understand biology i mean even with genetics and whatever he said yeah yeah so how did he feel about having a center for physics and biology it was a political thing just to get some money you know to talk rockefellers into dumping some money into that thing

i mean i think center was reasonably good yeah no i mean he did good people and and he had this great idea that to make sure these guys don't spend time writing grant proposals right yeah it's always a good idea

so i guess the other thing that itama said a story about mitchell when he went to rockefeller went to visit him and then he asked where's the library and he had no idea where the library was okay so itamar wanted to get the get some dla calculation uh done and mitchell said oh just wait let me do it and he thought about it for half an hour and uh gave it to me the answer the library he was totally outraged that uh you know it was lyft chabert somebody who destroyed the library there was a very beautiful library very old library physics library and i think somebody decided just to shut it down so you know the every institution has one of those one of those people right yeah so i remember him not being happy about that even though i didn't use this book

but did mitchell read any papers anybody else's papers no i think he must he he actually had some papers sometimes uh i think he probably skimmed through them right

but he wanted to talk to i mean he talked to people and he understood exactly what uh what they had done i mean sudan it was embarrassing uh i mean after i i was in houston i had i had worked on a couple of other problems okay and i go to visit him and talk to him and within half an hour he knows exactly what i have done and whatever what else i should have done

maybe he knew it faster and was being polite would be

okay so give me the link to this recording i'll have to figure out oh yeah gamma you're going to say something yeah no i have sent you the the my uh the talk i gave attractors okay for me till 75th 70th birthday so uh yeah i haven't written anything so let me try this can i call it do clouds sold pdes or do you have some variant like that clouds uh no i didn't talk about that i no i i have to give a title to this yeah you know do clouds off pde i think that's nice yeah actually that's that's nice you know that was his in-flight i mean he influenced me a thousand ways but you know it was this thing that where i thought i knew something and he would ask me a question you know socratically so this was uh you know this is the most significant question to me the clouds seoul pbs you know do you believe there is a super computer in the sky that crunches never stokes i mean seriously do you believe that they sold these equations right do you believe that's how they solve them it's really a cellular automata machine yeah what it's really a cellular automaton machine it could be in some sense but it has to be global it cannot be doing it locally yeah uh and you know i think it has to do it in space time but but that you know that idea was stuck into my head and also this pruning you know he was trying to understand the organization of a non-periodic orbits right so as far as i know he's the first person who told me that anon has periodic orbits you know oh i see i just i didn't think about it but anyway they're organized you know yeah some place you're the one who got the pruning front right yeah but you know it comes from there trying to understand how they organize you know the the question comes from it to me or you know provocation

no really because you know no actually back in uh so he talked to me about this in 83 i think right and at the time the only dynamical invariant we were talking about was uh well yapunov exponents and fractal dimensions yeah yeah and so i had asked him for a problem to work on and so i mean like what a couple of days he told me about this okay his point was that in statistical mechanics you have the boltzmann density and then use that to calculate statistical quantities right yeah i mean we don't run uh uh thousands of uh different systems whereas at the time in like to get fractal dimensions or the open of exponents you just took an orbit and ran it for a zillion times and then did the calculation yeah yeah so he said this is not the way to do it you need to be able to get these local densities and get somehow be able to calculate yeah yeah we ended up doing that but that's i mean okay in his mind it was very clear to him but what to do he just asked me to do it

so i think it's a good title

the title uh the cloud soft pde oh actually i like that is is there a super computer in the sky

he computed all the time by the idea that computers understand anything was not

mutual

i mean he was really you know an intelligent engineer because you know he was programming these totally mindless things like how do you make the best possible map of latin america and you know he was doing it in whatever not machine code but not much higher and you know it required like days of putting programs together and keeping you know 50 pointers in yourself etc uh you know and he did he did his things very massively these programs and calculations he also sped up this uh stochastic calculations by factors of 100 stuff like that by thinking but but also implemented you know but implementing you know it's really yeah true no actually he gave a talk about this in chicago right

about what uh about using this uh solution methods for that for finance i guess yeah and the finance bit people didn't understand why i mean yet use these fancy integrations to solve the equations they thought you could get a i'll get a bigger computer and do it yeah yeah

schmucks

absolutely disgusting

you have to tell them they'll make money faster than the next guy that'll get yeah we wrote more on what is it called [---] algorithm or whatever

universal scaling whatever it was okay i'll finally get published now i'll try to figure out how to get this this is probably going to be a very big file so i'll try to figure out what to do oh what you do is you just send me a link it's in the cloud so i can send the link to the file okay if i can it might

it might be in some mit protected domain so i'll do what i can no no no it's okay it's a zoom it's owned by national security

putin still has access he had it until three weeks ago okay they might have changed their password i'll uh be in touch that it may be that well you know i believe zoom even creates a um um what's it called uh transcript transcript yes they can transcribe me i'll be very impressed but uh i'm totally impressed that google transcribes me almost correctly on on youtube videos it's amazing how good i've gotten seriously i mean i'm totally impressed okay see you guys next week then okay yeah maybe david would be curious actually we should be thinking about the dates for the conference next week well then we just put some cold water onto that lady go ahead then yeah you weren't there coming in when we talked about no no no no no no no i'm just i'm just i'm just concerned that um you know maybe january is going to be too early we should wait a couple of months yeah see how things develop or too many of these people will still be

alive actually the one thing we need to figure out is how to ask dead people to give zoom talks

uh re-record them now

we could be very like gentle and say you know just in case do you mind recording it right now because we would like to you know test our technology

okay see you guys next week





\end{description}



%\newpage %%%%%%%%%%%%%%%%%%%%%%%%%%%%%%%%%%%%%%%%%%%%%%%%
\printbibliography[heading=subbibintoc,title={References}]
