% reducesymm/UFO/history.tex
% Predrag 2019-07-28


\chapter{Historical notes}
\label{c-history}

\section{Other authors}
\label{sect:Others}

\begin{description}
\item[2019-07-27 Predrag]
starting to collect ``about" MJF bibliography

\rf{AubDal02}
Aubin and Dalmedico
{\em Writing the history of dynamical systems and chaos: {Longue}
dur{\'{e}}e and revolution, disciplines and cultures},
{2002},

\rf{Barbaro07}
{G. Barbaro},
{\em Formal solutions of the {Cvitanovi{\'c}-Feigenbaum} equation},
{2007},

\rf{Briggs89}
{Briggs, K.},
{\em How to calculate the {Feigenbaum} constants on your {PC}},
{1989},

\rf{Briggs91}
{Briggs, K.},
{\em A precise calculation of the {Feigenbaum} constants},
{1991},

\rf{BDS98}
{Briggs, K. M. and Dixon, T. W. and Szekeres, G.},
{\em Analytic solutions of the {Cvitanovi{\'c}-Feigenbaum} and {Feigenbaum-Kadanoff-Shenker} equations},
{1998},

\rf{Broadhurst99}
{D. Broadhurst},
{\em Feigenbaum constants to 1018 decimal places},
{1999},

\rf{Buff99}
{X. Buff},
{\em Geometry of the {Feigenbaum} map},
{1999},

\rf{coppersmith:52}
{S. N. Coppersmith},
{\em A simpler derivation of {Feigenbaum}'s renormalization group equation for the period-doubling bifurcation sequence},
{1999},

\rf{cvitanovic1984universality}
{Cvitanovi{\'c}, P.},
{\em Universality in chaos (or, {Feigenbaum} for cyclists)},
{1984},

\rf{Groen86}
{J. Groeneveld},
{\em On constructing complete solution classes of the {Cvitanovi{\'c}-Feigenbaum} equation},
{1986},

\rf{CamEps81}
{Campanino, M. and Epstein, H.},
{\em On the existence of {Feigenbaum's} fixed point},
{1981},

\rf{CaEpRu82}
{M. Campanino and H. Epstein and D. Ruelle},
{\em On {Feigenbaum}'s functional equation {$g \circ g(\lambda x) + \lambda g(x) = 0$}},
{1982},

\rf{EpsLas81}
{Epstein, H. and Lascoux, J.},
{\em Analyticity properties of the {Feigenbaum} function},
{1981},

\rf{FrKhaMa03}
{Frisch, U. and Khanin, K. and Matsumoto, T.},
{\em Multifractality of the {Feigenbaum} attractor and fractional derivatives},
{2005},

\rf{Gaidashev11}
{Gaidashev, D.},
{\em On analytic perturbations of a family of {Feigenbaum}-like equations},
 {2011},

\rf{GrMaViFe81}
{J. M. Greene and R. S. MacKay and F. Vivaldi and M. J. Feigenbaum},
{\em Universal behaviour in families of area-preserving maps},
{1981},

\rf{KuzOsb02}
{P. Kuznetsov and A. H. Osbaldestin},
{\em Generalized dimensions of {Feigenbaum}'s attractor from renormalization-group functional equations},
{2002},

@InProceedings{KuKuSa05}
{Kuznetsov, S. P. and Kuznetsov, A. P. and Sataev, I .R.},
{\em Review and examples of non-{Feigenbaum} critical situations associated with period-doubling},
{2005},

\rf{Lanford82S}
{{Lanford}, O. E.},
{\em A computer-assisted proof of the {Feigenbaum} conjectures},
{1982},

\rf{Lyubich99}
{Lyubich, M.},
{\em {Feigenbaum-Coullet-Tresser} universality and {Milnor}'s hairiness conjecture},
{1999},

\rf{Mathar10}
{Mathar, R. J.},
{\em {Chebyshev} series representation of {Feigenbaum's} period-doubling function},
{2010},

\rf{Molteni16}
{Molteni, A.},
{\em An efficient method for the computation of the {Feigenbaum} constants to high precision},
{2016},

\rf{OlLe11}
{Oliveira, D. F. M. and Leonel, E. D.},
{\em The {Feigenbaum}'s {$\delta$} for a high dissipative bouncing ball model},
{2011},

\rf{Pollicott91}
{Pollicott, M.},
{\em A note on the {Artuso-Aurell-Cvitanovi{\'c}} approach to the {Feigenbaum} tangent operator},
{1991},

\rf{Sezgin06}
{F. Sezgin and T. M. Sezgin},
{\em On the statistical analysis of {Feigenbaum} constants},
{2006},

\rf{Tsygvintsev_2002}
{A. V. Tsygvintsev and B. D. Mestel and A. H. Osbaldestin},
{\em Continued fractions and solutions of the {Feigenbaum-Cvitanovi{\'{c}}} equation},
{2002},

\rf{VulKha82}
{E. B. Vul and K. M. Khanin},
{\em The unstable separatrix of {Feigenbaum}'s fixed-point},
{1982},

\rf{VuSiKh84}
{Vul, E. B. and Sinai, Ya. G. and K. M. Khanin},
{\em Feigenbaum universality and the thermodynamic formalism},
{1984},

\rf{mathwFeigF}
{Weisstein, E. W.},
{\em Feigenbaum function},
{2012},

\rf{WeissFeig}
{Weisstein, E. W.},
{\em Feigenbaum constant},
{2012},

\rf{WeOv94}
{Wells, A. L. J. and Overill, R. E.},
{\em The extension of the {Feigenbaum-Cvitanovi{\'c}} function to the complex plane},
{1994},

\rf{Avilaa}
{Avila, A. and Lyubich, M.},
  {\em Examples of {Feigenbaum Julia} sets with small {Hausdorff} dimension},
{2006},

\rf{AviLyu07}
{Avila, A. and Lyubich, M.},
  {\em Hausdorff dimension and conformal measures of {Feigenbaum Julia} sets},
{2007},

\rf{AviLyu15}
{Avila, A. and Lyubich, M.},
{\em Lebesgue measure of Feigenbaum Julia sets},
{2015},

\rf{DudYam15}
{Dudko, A. and Yampolsky, M.},
  {\em Poly-time computability of the {Feigenbaum Julia} set},
{2015},




\end{description}


There is a Rashomon of histories ahead

\begin{description}
\item[1995-01-13 (?) Lyubich]
email to Predrag explaining why Predrag's name is no longer attached
to the fixed point equation.

\item[2019-07-27 David Ruelle] %<ruelle@ihes.fr>
Mitchell invited me to Los Alamos after the Ruelle-Takens paper, and I
met Cornelia, his first wife (I don't remember the exact date).  We kept
very good personal relations, but I don't remember any anecdote worth
relating.  After his famous work on the period doubling cascade, I
collaborated to a small 1982 note with M. Campanino and H.
Epstein\rf{CaEpRu82} {\em On {Feigenbaum}'s functional equation {$g \circ
g(\lambda x) + \lambda g(x) = 0$}}, but I watched the work on
Feigenbaum's equation mostly from some distance.  It may be of interest
to retrace the mathematical work by Lanford, Sullivan, etc. on this
subject.

\item[2019-07-26 Dennis Sullivan] % <sullivan0212@gmail.com>
% To: David Ruelle <ruelle@ihes.fr>
I recall during a scientific meeting at IHES you and Oscar or J{\'u}rg
were playing around and not paying attention

I asked what you were doing and you guys showed me the Cvitanovi{\'c} functional
equation without the story

later you explained Mitchell's Feigenbaum's period doubling
numerical discovery

there was also Coullet-Tresser critical opalescence related discovery
mentioned below

at the time  many of us were focusing on Thurston-Gromov hyperbolic
geometry and  those with a dynamical bent on limit sets of Kleinian
groups

there was the idea with Rufus Bowen to study dynamical bifurcation on the
more explicit case of limit sets of Kleinian groups,
because an analogy could be made between that subject and the Fatou-Julia
theory of holomorphic iteration

Tresser came to visit me in NY to try out computations of the Poincar\'e
limit sets

This idea failed for various reasons related to the sad departure of
Rufus Bowen and the computer difficulties of large groups.

I really got involved personally in period doubling in 1981 when Tresser
told me the Hausdorff dimension  of the period doubling cantor set changed
when you changed the critical exponent.

I was shocked because of my experience with the horseshoe Cantor set
whose geometrical structure depended on the choice of the mapping.

I said you mean it doesn't change if you don't change the exponent.

He said of course and recounted his work with Coullet on critical
opalescence.

For me this was fantastic :
an infinite dimensional version Mostow rigidity and an extension of
Michel Hermann's solution of the Arnold conjecture.

I dropped all my projects and worked on this full time till 1990 to get a
math formulation and proof of geometric  rigidity \emph{of the period
doubling cantor set} (this was expressed in terms of the scaling function
on the dual Cantor set introduced by Khanin and Sinai\rf{VuSiKh84}), the
proof derived the beau bounds real and complex and also used a
Teichmuller space of laminations.

The proof only works until today for even integer exponents because of
the Teichmuller step (in this complex proof the concept of quadratic like
mappings due to Douady and Hubbard was crucial), but gave the convergence
of renormalization.

In the early 1990s an elegant device was introduced by Lyubich to use the
above information to get the exponential convergence thus proving the
Feigenbaum universality of his number $4.67\cdots$ .

I wish the physicists in the middle of things would describe the story on
the physics side so we could better understand what the meaning of all of
mathematics might be and perhaps find more mathematics including a real
variable discussion that would include all real exponents greater than
one.

\item[2019-07-27 Predrag]
In the above, Sullivan has forgotten to mention Collet-Eckmann
book\rf{eckman80map}, and Collet-Eckmann-Lanford\rf{collet80} {\em
Universal properties of maps on an interval} proof of Feigenbaum
equation for $|x|^{1+\epsilon}$.

\item[2019-07-27 Gemunu]
Sullivan writes: ``... this was expressed in terms of the scaling
function on the dual Cantor set introduced by Khanin and Sinai.'' But I
thought Mitchell invented the trajectory scaling function (which he was
very proud of) first, before Sinai and Khanin?

\item[2019-07-27 Predrag]
Vul, Sinai and Khanin\rf{VuSiKh84} 1984 {\em Feigenbaum universality and
the thermodynamic formalism} paper discusses the stable-unstable manifold
of the period-doubling operator fixed point, as well as the complex plane
period-tripling fixed point (contemporaneously with the independent 1983
Cvitanovi{\'c} and Myrheim work\rf{myrh1,CviMyr89}). There is also the
1982 Vul and Khanin\rf{VulKha82} {\em The unstable separatrix of
{Feigenbaum}'s fixed-point} that does not seem to get cited.

Feigenbaum introduced presentation functions in his 1988
paper\rf{Feigenbaum88} {\em Presentation functions, fixed points, and a
theory of scaling function dynamics}. The lack of physics community
interest in his presentation functions is indeed one of the reasons
Feigenbaum gave up on journal publications
(Safmann-Taylor\rf{Feigenbaum02} instability was the final nail in the
coffin).

``Dual Cantor set'' {seems to only} refer to the manner of labelling a
binary tree-organized Cantor set (see 1990 PhD thesis by Sullivan's
student
\HREF{http://qcpages.qc.cuny.edu/~yjiang/HomePageYJ/Download/1990PHDThesisYPJ.pdf}
{Y. Jiang}). I suspect this is what is called ``alternating  binary
tree'' in
\HREF{http://www.chaosbook.org/chapters/ChaosBook.pdf\#section.14.3}
{ChaosBook.org}.

The notion of ``dual Cantor set'' seems to have been introduced by
Sullivan\rf{Sullivan88} in the June 1987 Noto (or what Predrag calls ``No
to NATO'') conference\rf{GalZwe88}, where Feigenbaum, Siggia,
Cvitanovi{\'c}, Lanford, Swinney, Libchaber, etc. also spoke. Sullivan
writes: ``Feigenbaum expressed this discovery in terms of a universal
scaling function for the Cantor set. [...] the fine structure is codified
by a scaling function defined on a logically distinct perfect set - the
dual Cantor set. The main unsolved mystery is why the renormalizations
converge. '' (I have no access to the full lecture, so do not know what
other credit Sullivan gives to whom, but it also appears in the equally
unaccessible \refref{Wells88}).

In Birkhoff, Martens and Tresser\rf{BiMaTr03}
{\em On the scaling structure for period doubling} they write:
``For  completeness  and  to fix notations  and  definitions,  we include
some  basic discussion  of the scaling function,  whose origin is rather
diffuse:  first  conjectures  about a form  of it  appeared  in Coullet
and Tresser\rf{CouTre78}, the name and a  form  of it  come from
Feigenbaum\rf{Feigenbaum80}, while what  was  arguably the first theorem
about  it was in a never circulated  work by Feigenbaum  and  Sullivan.
The literature  on  scaling functions  is  extensive  and  discusses
scaling functions  beyond  the context  of dynamics.  In  particular,  in
1984 Vul, Sinai and Khanin paper\rf{VuSiKh84}  a  relation  with  the
thermodynamic  formalism  appeared.''
They do not bother to actually plot the scaling function they construct,
but as their ``universal period doubling scaling function $q$ is strictly
monotone and the range forms a Cantor set'' I think it is not the same
as:

Feigenbaum\rf{Feigenbaum80} {\em The transition to aperiodic behavior in
turbulent systems} constructs the scaling function $\sigma$ and plots it in
his Fig.~2.

The 1984 Vul, Sinai and Khanin survey\rf{VuSiKh84} is impressive. It does
discuss Feigenbaum attractor as a binary-labelled Cantor set, but I see
no ``dual" anywhere. ``Thermodynamic formalism'' is in the title, but
mentioned only once in the text. It is developed in sect.~4 in the usual
way. They note that the numerical work of Grassberger\rf{Grassberger81},
{\em On the {Hausdorff} dimension of fractal attractors}, gives a more
accurate number. Summa summarum, I see no scaling functions in Vul, Sinai
and Khanin survey\rf{VuSiKh84}. Seems to be something that was introduced
by Feigenbaum and Sullivan in different forms, but while interacting with
each other. I'm no wiser than that.

\end{description}



%\newpage %%%%%%%%%%%%%%%%%%%%%%%%%%%%%%%%%%%%%%%%%%%%%%%%
\printbibliography[heading=subbibintoc,title={References}]
