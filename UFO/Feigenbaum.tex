% reducesymm/UFO/Feigenbaum.tex
% Predrag 2019-07-23

\chapter{Mitchell J. Feigenbaum}
\label{c-Feigenbaum}

Mitchell Jay Feigenbaum (July 19, 1929 -- January 24, 2016)

For a bibliography, see
\HREF{http://www.physics.gatech.edu/user/david-finkelstein}
{GaTech} homepage,
his
\HREF{https://www.davidritzfinkelstein.com/publications.html}
{selected publications}.

\section{Feigenbaum notes}
\label{sect:Feigenbaum}

\begin{description}
\item[1971-01-13 Fred Cooper]

\rf{feigen78}
{\em Quantitative universality for a class of nonlinear transformations},
{M. J. Feigenbaum},

\rf{Feigenbaum87}
{\em Some characterizations of strange sets},
{Feigenbaum, M. J.},

\rf{Feigenbaum88}
{\em Presentation functions, fixed points, and a theory of scaling function dynamics},
{Feigenbaum, M. J.},

\rf{Feigenbaum88a}
{\em Presentation functions and scaling function theory for circle maps},
{M. J. Feigenbaum},

\rf{FeigKen83}
{M. J. Feigenbaum and R. D. Kenway},
{\em The onset of chaos},
{1984},

\rf{fpt}
{M. J. Feigenbaum, I. Procaccia and T. T{\'e}l},
  {\em Scaling properties of multifractals as an eigenvalue problem},
{1989},

\rf{funceq2}
{\em The universal metric properties of nonlinear transformations},
{M. J. Feigenbaum},
{1979},

\rf{GrMaViFe81}
{\em Universal behaviour in families of area-preserving maps},
{J. M. Greene, R. S. MacKay, F. Vivaldi and M. J. Feigenbaum},
{1981},

\rf{FeKaSh82}
{M. J. Feigenbaum, L. P. Kadanoff and S. J. Shenker},
  {\em Quasiperiodicity in dissipative systems: {A} renormalization group analysis},
{1982}

\rf{Feigenbaum_1982}
{M. J. Feigenbaum, L. P. Kadanoff, S. J. Shenker},
  {\em Quasiperiodicity in dissipative systems: {A} renormalization group analysis},
{1982},

\rf{Feigenbaum82}
{Feigenbaum, M. J.},
{\em Reflections of the {Polish} masters: {An} interview with {Stan Ulam} and {Mark Kac}},
{1982},
reprinted as
\rf{Feigenbaum85}
{Feigenbaum, M. J.},
  {\em An interview with {Stan Ulam} and {Mark Kac}},
  {1985}

\rf{FeiLow71}
{Feigenbaum, M. and Low, F. E.},
  {\em Current conservation and double-spectral representations for scattering of vector particles},
{1971},

\rf{Feigenbaum08}
{M. J. Feigenbaum},
{\em The theory of relativity - {Galileo's} child},
{2008},

\rf{Feigenbaum02}
{M. J. Feigenbaum},
{\em Pattern selection: {Determined} by symmetry and modifiable by long-range effects},
{2002},

\rf{DFHP00}
{Davidovitch, B. and Feigenbaum, M. J. and Hentschel, H. G. E. and Procaccia, I.},
  {\em Conformal dynamics of fractal growth patterns without randomness},
{2000},

\rf{VdSFC97}
{Verberg, R. and de Schepper, I. M. and Feigenbaum, M. J. and Cohen, E. G. D.},
  {\em Square root singularity in the viscosity of neutral colloidal suspensions at large frequencies},
{1997},

\rf{FePrDa01}
{Feigenbaum, M. J. and Procaccia, I. and Davidovich, B.},
  {\em Dynamics of finger formation in {Laplacian} growth without surface tension},

\rf{feigen79scal}
{M. J. Feigenbaum},
{\em The onset spectrum of turbulence},
{1979},

\rf{Feigenbaum83}
{Feigenbaum, M. J.},
{\em Universal behavior in nonlinear systems},
{1983},
also
\HREF{https://fas.org/sgp/othergov/doe/lanl/pubs/00818090.pdf}
{Universal behavior in nonlinear systems}

\rf{FeHa82}
{\em Irrational decimations and path-integrals for external noise},
{M. J. Feigenbaum and B. Hasslacher},
{1982},

\rf{BrPiPa95IIIFeig}
{Feigenbaum, M. J.},
{\em Computer-generated physics}
1995 in {Brown, Pippard  and Pais}
{\em {Twentieth Century Physics}}
is one of those books put together to extort money from libraries, and
not owned by anyone, contains a good essay by Feigenbaum. The copyright
might be expired(?) - we should consider uploading this text to arXiv
under Mitchell's name, it's pity it will just disappear in a huge volume
out of print... It's a rare genuine Feigenbaum gem:

\begin{quote}
This has me deeply skeptical of research awaiting to encounter ``emergent
behavior''. After all, if new organizations occur when an object becomes
``complex'' enough, then one has no need of a computer's service. Rather,
one merely needs to look up at the clouds; down at the pebbles and cracks
in the pavement and plants and other life; across onto the river, across
at the innumerable arbitrarily placed objects in one's environment, not
to mention the specific, however peculiarly random, placed hairs on one's
head. The world abounds with interacting complex entities and in quantity
and quality infinitely surpassing the puny simulations of all the world's
computers. It is hard to learn by seeing. So why should the scales
miraculously fall off when one stares at a computer screen?

The point is that without a conceptual schema to be able to notice what
is recurrent, however irregular, no subtle observations are generally
possible. Nevertheless, let me admit, fortuities and serendipities do
sometimes transpire.

[...]
As for myself in the early 1970's, I didn't care how the particles came
out ``today'', but skeptical of all our calculations, looked around for
ideas to calculate something correctly. Moreover, as all of modern
physics is concerned with interacting systems of an infinite number of
degrees of freedom, if one could face these questions with adequate
tools, one might begin to consider infinitely interconnected sets of
neurons, that is, the behavior of a nervous system.

[...]
Wilson told me in 1981 landing in Santa Barbara ``Working too much with a
computer makes you less smart than you used to be.''

[...]
But then too, that other phenomenon that defied all classical effort -
the turbulence in a fluid that, à la Kolmogorov, exhibits a smooth
spectral cascade of each scale size eddy intimately interacting with
those of approximately adjacent scales. Wilson, from the beginning,
imagined that his ideas could finally deliver the theory of fluid
turbulence. It has not. But a curious tangent did something else: It
delivered the first correct quantitative theory of the onset of
turbulence. Wilson’s guess had Carruthers set me to turbulence. Now a
different story will unfold.

\end{quote}

\item[2019-07-19 Predrag]
have to track down a bunch of references:\\
 M J Feigenbaum, Using nonlinear dynamics to make a new world atlas, in Towards the harnessing of chaos (Amsterdam, 1994), 1-9.
\\
M J Feigenbaum, Scaling function dynamics, in Chaos, order, and patterns, Lake Como, 1990 (New York, 1991), 1-23.
\\
M J Feigenbaum, Universal behavior in nonlinear systems, in Order in chaos, Los Alamos, N.M., 1982, Phys. D 7 (1-3) (1983), 16-39.
\\
M J Feigenbaum, Low-dimensional dynamics and the period doubling scenario, in Dynamical systems and chaos, Sitges/Barcelona, 1982 (Berlin, 1983), 131-148.
\\
Hammond Atlas of the World (Maplewood, NJ, 1992), 9.

\end{description}


Mitchell was very hard to understand. In
colloquia first 15 minutes one could follow, then it got fuzzy. They
tried to write a book on general relativity, but could not agree on
anything beyond the first chapter. Eventually Anderson wrote the book by
himself.

\section{Other authors}
\label{sect:Others}

\begin{description}
\item[1995-01-13 Lyubich, M.]

\rf{Barbaro07}
{G. Barbaro},
{\em Formal solutions of the {Cvitanovi{\'c}-Feigenbaum} equation},
{2007},


\rf{Briggs89}
{Briggs, K.},
{\em How to calculate the {Feigenbaum} constants on your {PC}},
{1989},

\rf{Briggs91}
{Briggs, K.},
{\em A precise calculation of the {Feigenbaum} constants},
{1991},

\rf{BDS98}
{Briggs, K. M. and Dixon, T. W. and Szekeres, G.},
{\em Analytic solutions of the {Cvitanovi{\'c}-Feigenbaum} and {Feigenbaum-Kadanoff-Shenker} equations},
{1998},

\rf{Broadhurst99}
{D. Broadhurst},
{\em Feigenbaum constants to 1018 decimal places},
{1999},

\rf{Buff99}
{X. Buff},
{\em Geometry of the {Feigenbaum} map},
{1999},

\rf{coppersmith:52}
{S. N. Coppersmith},
{\em A simpler derivation of {Feigenbaum}'s renormalization group equation for the period-doubling bifurcation sequence},
{1999},

\rf{cvitanovic1984universality}
{Cvitanovi{\'c}, P.},
{\em Universality in chaos (or, {Feigenbaum} for cyclists)},
{1984},

\rf{Groen86}
{J. Groeneveld},
{\em On constructing complete solution classes of the {Cvitanovi{\'c}-Feigenbaum} equation},
{1986},

\rf{CamEps81}
{Campanino, M. and Epstein, H.},
{\em On the existence of {Feigenbaum's} fixed point},
{1981},

\rf{CaEpRu82}
{M. Campanino and H. Epstein and D. Ruelle},
{\em On {Feigenbaum}'s functional equation {$g \circ g(\lambda x) + \lambda g(x) = 0$}},
{1982},

\rf{EpsLas81}
{Epstein, H. and Lascoux, J.},
{\em Analyticity properties of the {Feigenbaum} function},
{1981},

\rf{FrKhaMa03}
{Frisch, U. and Khanin, K. and Matsumoto, T.},
{\em Multifractality of the {Feigenbaum} attractor and fractional derivatives},
{2005},

\rf{GrMaViFe81}
{J. M. Greene and R. S. MacKay and F. Vivaldi and M. J. Feigenbaum},
{\em Universal behaviour in families of area-preserving maps},
{1981},

\rf{KuzOsb02}
{P. Kuznetsov and A. H. Osbaldestin},
{\em Generalized dimensions of {Feigenbaum}'s attractor from renormalization-group functional equations},
{2002},

@InProceedings{KuKuSa05}
{Kuznetsov, S. P. and Kuznetsov, A. P. and Sataev, I .R.},
{\em Review and examples of non-{Feigenbaum} critical situations associated with period-doubling},
{2005},

\rf{Lanford82S}
{{Lanford}, O. E.},
{\em A computer-assisted proof of the {Feigenbaum} conjectures},
{1982},

\rf{Lyubich99}
{Lyubich, M.},
{\em {Feigenbaum-Coullet-Tresser} universality and {Milnor}'s hairiness conjecture},
{1999},

\rf{Mathar10}
{Mathar, R. J.},
{\em {Chebyshev} series representation of {Feigenbaum's} period-doubling function},
{2010},

\rf{Molteni16}
{Molteni, A.},
{\em An efficient method for the computation of the {Feigenbaum} constants to high precision},
{2016},

\rf{OlLe11}
{Oliveira, D. F. M. and Leonel, E. D.},
{\em The {Feigenbaum}'s {$\delta$} for a high dissipative bouncing ball model},
{2011},

\rf{Pollicott91}
{Pollicott, M.},
{\em A note on the {Artuso-Aurell-Cvitanovi{\'c}} approach to the {Feigenbaum} tangent operator},
{1991},

\rf{Sezgin06}
{F. Sezgin and T. M. Sezgin},
{\em On the statistical analysis of {Feigenbaum} constants},
{2006},

\rf{Tsygvintsev_2002}
{A. V. Tsygvintsev and B. D. Mestel and A. H. Osbaldestin},
{\em Continued fractions and solutions of the {Feigenbaum-Cvitanovi{\'{c}}} equation},
{2002},

\rf{VulKha82}
{E. B. Vul and K. M. Khanin},
{\em The unstable separatrix of {Feigenbaum}'s fixed-point},
{1982},

\rf{VuSiKh84}
{Vul, E. B. and Sinai, Ya. G. and K. M. Khanin},
{\em Feigenbaum universality and the thermodynamic formalism},
{1984},

\rf{mathwFeigF}
{Weisstein, E. W.},
{\em Feigenbaum function},
{2012},

\rf{WeissFeig}
{Weisstein, E. W.},
{\em Feigenbaum constant},
{2012},

\rf{WeOv94}
{Wells, A. L. J. and Overill, R. E.},
{\em The extension of the {Feigenbaum-Cvitanovi{\'c}} function to the complex plane},
{1994},

\rf{Avilaa}
{Avila, A. and Lyubich, M.},
  {\em Examples of {Feigenbaum Julia} sets with small {Hausdorff} dimension},
{2006},

\rf{AviLyu07}
{Avila, A. and Lyubich, M.},
  {\em Hausdorff dimension and conformal measures of {Feigenbaum Julia} sets},
{2007},

\rf{AviLyu15}
{Avila, A. and Lyubich, M.},
{\em Lebesgue measure of Feigenbaum Julia sets},
{2015},

\rf{DudYam15}
{Dudko, A. and Yampolsky, M.},
  {\em Poly-time computability of the {Feigenbaum Julia} set},
{2015},




\end{description}

\section{Memoirs}
\label{sect:Memoirs}

\HREF{http://www.nasonline.org/member-directory/deceased-members/51874.html}
{Bram Pais} is an example of a NAS Memoir David and I are setting out to
write

\HREF{http://www-history.mcs.st-andrews.ac.uk/Search/historysearch.cgi?SUGGESTION=Mitchell+Feigenbaum\%20MacTutor&CONTEXT=1}
{MacTutor} search


A. Pais\rf{Pais00}
{\em Mitchell Jay Feigenbaum}, in {\em The genius of science}.
Can borrow it from
\HREF{https://archive.org/details/geniusofsciencea00pais} {here}.

Michael Berry
\HREF{https://michaelberryphysics.files.wordpress.com/2013/07/berry317.pdf}
{kvetches}: ``David Mermin told me that after reading Pais's colloquial
injunction `Go figure' in the chapter on Mitchell Feigenbaum, he did just
that - only to discover that the numbers in the table of period-doublings
are wrong. Reading the same chapter, I wondered why the author thought it
worth telling us that Feigenbaum was toilet-trained by six months.''


%\newpage %%%%%%%%%%%%%%%%%%%%%%%%%%%%%%%%%%%%%%%%%%%%%%%%
\printbibliography[heading=subbibintoc,title={References}]
