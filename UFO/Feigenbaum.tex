% reducesymm/UFO/Feigenbaum.tex
% Predrag 2019-07-23

\chapter{Mitchell J. Feigenbaum}
\label{c-Feigenbaum}

Mitchell Jay Feigenbaum (December 19, 1944 -- June 30, 2019)

In 2017 Mitchell was diagnosed with advanced laryngeal cancer, which was
treated with radiation. Although the cancer was apparently cured, the
after effects of radiation were very damaging, including the essential
destruction of his voice and serious swelling in his throat and
esophagus, which caused him severe difficulty breathing and stressed his
heart. On Friday, June 28 2019 he went into the hospital with breathing
difficulties and on Saturday morning, June 29th, he suffered a massive
heart attack from which he never regained consciousness. He passed away
on June 30.

For a bibliography, see
\HREF{http://XXX}
{XXX} homepage,
his
\HREF{https://XXX}
{XXX selected publications}.

\section{Feigenbaum bibliography}
\label{sect:Feigenbaum}

\begin{description}
%\item[1971-01-13 Fred Cooper]
\item[2019-07-27 Predrag]
starting to collect MJF bibliography

\rf{feigen78}
{\em Quantitative universality for a class of nonlinear transformations},
{M. J. Feigenbaum},

\rf{Feigenbaum87}
{\em Some characterizations of strange sets},
{Feigenbaum, M. J.},

\rf{Feigenbaum88}
{\em Presentation functions, fixed points, and a theory of scaling function dynamics},
{Feigenbaum, M. J.},

\rf{Feigenbaum88a}
{\em Presentation functions and scaling function theory for circle maps},
{M. J. Feigenbaum},

\rf{FeigKen83}
{M. J. Feigenbaum and R. D. Kenway},
{\em The onset of chaos},
{1984},

\rf{fpt}
{M. J. Feigenbaum, I. Procaccia and T. T{\'e}l},
  {\em Scaling properties of multifractals as an eigenvalue problem},
{1989},

\rf{funceq2}
{\em The universal metric properties of nonlinear transformations},
{M. J. Feigenbaum},
{1979},

\rf{GrMaViFe81}
{\em Universal behaviour in families of area-preserving maps},
{J. M. Greene, R. S. MacKay, F. Vivaldi and M. J. Feigenbaum},
{1981},

\rf{FeKaSh82}
{M. J. Feigenbaum, L. P. Kadanoff and S. J. Shenker},
  {\em Quasiperiodicity in dissipative systems: {A} renormalization group analysis},
{1982}

\rf{Feigenbaum_1982}
{M. J. Feigenbaum, L. P. Kadanoff, S. J. Shenker},
  {\em Quasiperiodicity in dissipative systems: {A} renormalization group analysis},
{1982},

\rf{Feigenbaum82}
{Feigenbaum, M. J.},
{\em Reflections of the {Polish} masters: {An} interview with {Stan Ulam} and {Mark Kac}},
{1982},
reprinted as
\rf{Feigenbaum85}
{Feigenbaum, M. J.},
  {\em An interview with {Stan Ulam} and {Mark Kac}},
  {1985}
He starts out with something we might use for the conference:

\begin{quote}
I am what is called a mathematical physicist. I take this to mean the
utilization of -and sometimes the attendant construction of- mathematics
in a context posed by physical reality.
[...]
I wanted to explore the (personal) ``philosophical'' views of just what
connections are in the back of theorists' minds that drive the work they
perform. It is hard, in understatement, to know a creator's internal
vantage point from the technical products in print.
[...]
It is a regrettable consequence of the medium of the
written word that the rich inflection of voice and gesticulation of
hand that so often color and amplify the words of these men are not
available to the reader
\end{quote}

Kac is a riot.

Ulam: ``My life, and Mark's too, occupies more than almost two per cent
of the recorded history of mankind. You see, fifty or sixty years is that
much. That it is a sizeable fraction of the whole history that we know
about is a strange and very terrifying thought.''

Kac: ``Let me put this in because I would like to
record it for posterity. I think there are two acts in mathematics.
There is the ability to prove and the ability to understand. Now the
actions of understanding and of proving are not identical. In fact, it is
quite often that you understand something without being able to
prove it. Now, of course, the height of happiness is that you
understand it and you can prove it. The next stage is that you don't
understand it, but you can prove it. That happens over and over
again, and mathematics journals are full of such stuff. Then there is
the opposite, that is, where you understand it, but you can't prove it.
Fortunately, it then may get into a physics journal. Finally comes the
ultimate of dismalness, which is in fact the usual situation, when you
neither understand it nor can you prove it.''

Ulam: ``Well, actually, computers are a marvelous tool, and there is
no reason to fear them. You might say that initially a mathematician
should be afraid of pencil and paper because it is sort of a vulgar tool
compared with pure thought.''

Kac: There are two principles of pedagogy which have to be adhered to.
One is, ``Tell the truth, nothing but the truth, but not the whole
truth.''" That I had from a former colleague who is now unfortunately
deceased. The other one is, ``Never try to teach anyone how not to commit
errors they are not likely to commit.''

Kac: ``I was financially somewhat less affluent than Stan-I was, as
Michael Cohen, one of our mutual friends, says, independently poor.''

Mitchell in his biosketch: ``he has an abiding interest in both the
nature of human experience and the nature of the human brain. One of his
distant hopes is that his new approach to chaotic phenomena may provide a
clue on how to model the complex processes of the brain.''

\medskip

David writes:
The ``dispersion relations'' paper\rf{FeiLow71} is
based on Mitchell’s 1970 PhD thesis,
Institute Archives - Noncirculating Collection 3 |
\HREF{http://library.mit.edu/F/MMI61IENTLBLXK9TLMRQABECSAR6MACJVV2YM1F4PXH1EIRGKH-01968?func=item-global&doc_library=MIT01&doc_number=000599836&year=&volume=&sub_library=ARC}
{Thesis Phy 1970 Ph.D}
(requires a login):
\\
Feigenbaum, Mitchell J.
{\em The relationship of Feynman parametrization to the double spectral
representation of scattering amplitudes for higher spin particles}.

Mitchell really hated his thesis, never published it, and dropped
particle physics like a stone.

\rf{FeiLow71}
{Feigenbaum, M. and Low, F. E.},
  {\em Current conservation and double-spectral representations for scattering of vector particles},
{1971},
\begin{quote}
Joseph Serene, who started Cornell physics grad school in the fall of 1969,
remembers Mitchell as a new postdoc who arrived in
1970. He came by asking us to have a look at the draft of his paper with
\HREF{http://web.mit.edu/physics/news/physicsatmit/physicsatmit_01_Low.pdf}
{Francis Low}.
The reason Joseph remembers that paper so clearly is that the paper had
only one reference. Asked about that, Mitchell said ``There is no need for
other references, they are all cited in that one.''

The paper was published\rf{FeiLow71}, wherein, to our profound chagrin,
we find that Phys. Review had forced Mitchell to double the number of
references.
\end{quote}

\rf{Feigenbaum08}
{M. J. Feigenbaum},
{\em The theory of relativity - {Galileo's} child},
{2008},

\rf{Feigenbaum02}
{M. J. Feigenbaum},
{\em Pattern selection: {Determined} by symmetry and modifiable by long-range effects},
{2002},

\rf{DFHP00}
{Davidovitch, B. and Feigenbaum, M. J. and Hentschel, H. G. E. and Procaccia, I.},
  {\em Conformal dynamics of fractal growth patterns without randomness},
{2000},

\rf{VdSFC97}
{Verberg, R. and de Schepper, I. M. and Feigenbaum, M. J. and Cohen, E. G. D.},
  {\em Square root singularity in the viscosity of neutral colloidal suspensions at large frequencies},
{1997},

\rf{FePrDa01}
{Feigenbaum, M. J. and Procaccia, I. and Davidovich, B.},
  {\em Dynamics of finger formation in {Laplacian} growth without surface tension},

\rf{feigen79scal}
{M. J. Feigenbaum},
{\em The onset spectrum of turbulence},
{1979},

\rf{Feigenbaum83}
{Feigenbaum, M. J.},
{\em Universal behavior in nonlinear systems},
{1983},
also
\HREF{https://fas.org/sgp/othergov/doe/lanl/pubs/00818090.pdf}
{Universal behavior in nonlinear systems}

\rf{FeHa82}
{\em Irrational decimations and path-integrals for external noise},
{M. J. Feigenbaum and B. Hasslacher},
{1982},

\rf{BrPiPa95IIIFeig}
{Feigenbaum, M. J.},
{\em Computer-generated physics}
1995 in {Brown, Pippard  and Pais}
{\em {Twentieth Century Physics}}
is one of those books put together to extort money from libraries, and
not owned by anyone, contains a good essay by Feigenbaum. The copyright
might be expired(?) - we should consider uploading this text to arXiv
under Mitchell's name, it's pity it will just disappear in a huge volume
out of print... It's a rare genuine Feigenbaum gem:

\begin{quote}
This has me deeply skeptical of research awaiting to encounter ``emergent
behavior''. After all, if new organizations occur when an object becomes
``complex'' enough, then one has no need of a computer's service. Rather,
one merely needs to look up at the clouds; down at the pebbles and cracks
in the pavement and plants and other life; across onto the river, across
at the innumerable arbitrarily placed objects in one's environment, not
to mention the specific, however peculiarly random, placed hairs on one's
head. The world abounds with interacting complex entities and in quantity
and quality infinitely surpassing the puny simulations of all the world's
computers. It is hard to learn by seeing. So why should the scales
miraculously fall off when one stares at a computer screen?

The point is that without a conceptual schema to be able to notice what
is recurrent, however irregular, no subtle observations are generally
possible. Nevertheless, let me admit, fortuities and serendipities do
sometimes transpire.

[...]
As for myself in the early 1970's, I didn't care how the particles came
out ``today'', but skeptical of all our calculations, looked around for
ideas to calculate something correctly. Moreover, as all of modern
physics is concerned with interacting systems of an infinite number of
degrees of freedom, if one could face these questions with adequate
tools, one might begin to consider infinitely interconnected sets of
neurons, that is, the behavior of a nervous system.

[...]
Wilson told me in 1981 landing in Santa Barbara ``Working too much with a
computer makes you less smart than you used to be.''

[...]
But then too, that other phenomenon that defied all classical effort -
the turbulence in a fluid that, à la Kolmogorov, exhibits a smooth
spectral cascade of each scale size eddy intimately interacting with
those of approximately adjacent scales. Wilson, from the beginning,
imagined that his ideas could finally deliver the theory of fluid
turbulence. It has not. But a curious tangent did something else: It
delivered the first correct quantitative theory of the onset of
turbulence. Wilson’s guess had Carruthers set me to turbulence. Now a
different story will unfold.

\end{quote}

\item[2019-07-19 Predrag]
have to track down a bunch of references:\\
 M J Feigenbaum, Using nonlinear dynamics to make a new world atlas, in Towards the harnessing of chaos (Amsterdam, 1994), 1-9.
\\
M J Feigenbaum, Scaling function dynamics, in Chaos, order, and patterns, Lake Como, 1990 (New York, 1991), 1-23.
\\
M J Feigenbaum, Universal behavior in nonlinear systems, in Order in chaos, Los Alamos, N.M., 1982, Phys. D 7 (1-3) (1983), 16-39.
\\
M J Feigenbaum, Low-dimensional dynamics and the period doubling scenario, in Dynamical systems and chaos, Sitges/Barcelona, 1982 (Berlin, 1983), 131-148.
\\
Hammond Atlas of the World (Maplewood, NJ, 1992), 9.

\end{description}


Mitchell was very hard to understand. In colloquia first 15 minutes one
could follow, then it got fuzzy. We thought of writing a book together,
but could not agree on anything, even the first chapter. Eventually
Mitchell wrote a totally different book by himself.

\section{Memoirs}
\label{sect:Memoirs}

\HREF{http://www.nasonline.org/member-directory/deceased-members/51874.html}
{Bram Pais} is an example of a NAS Memoir David and I are setting out to
write

\HREF{http://www-history.mcs.st-andrews.ac.uk/Search/historysearch.cgi?SUGGESTION=Mitchell+Feigenbaum\%20MacTutor&CONTEXT=1}
{MacTutor} search


A. Pais\rf{Pais00}
{\em Mitchell Jay Feigenbaum}, in {\em The genius of science}.
Can borrow it from
\HREF{https://archive.org/details/geniusofsciencea00pais} {here}.

Michael Berry
\HREF{https://michaelberryphysics.files.wordpress.com/2013/07/berry317.pdf}
{kvetches}: ``David Mermin told me that after reading Pais's colloquial
injunction `Go figure' in the chapter on Mitchell Feigenbaum, he did just
that - only to discover that the numbers in the table of period-doublings
are wrong. Reading the same chapter, I wondered why the author thought it
worth telling us that Feigenbaum was toilet-trained by six months.''

%\newpage %%%%%%%%%%%%%%%%%%%%%%%%%%%%%%%%%%%%%%%%%%%%%%%%
\printbibliography[heading=subbibintoc,title={References}]
