% summary.tex
% $Author$ $Date$

\section{Summary}
% \section{Conclusions}
\label{sect:rpo-sum}


We have presented a detailed investigation
of the topology of the
{\KS} \statesp\ for $L=22$ system size.
At first glance, turbulent dynamics visualized in the \statesp\ might appear
hopelessly complex, but under a detailed examination it is
much less so than feared: it is
pieced together from essentially {1-$d$ return maps}
connected by fast transient interludes.
{\KS} and \PCf\  \eqv, \reqv, \po s and 
\rpo s embody Hopf's vision:
repertoire of recurrent spatio-temporal
patterns explored by turbulent dynamics.
We used
the \eqva, \reqva\ and {\rpo}s as a probe to explore the
\statesp\  topology and chaotic dynamics.
\Eqva\ are important because they set up 
a coarse description of 
typical \statesp\ motions.

While in general
for $\tildeL$ sufficiently large
one expects many 
coexisting attractors in the \statesp%
%Hyman and Nicolaenko
\rf{HNZks86},
in numerical studies most random initial
conditions settle converge to the same chaotic attractor. 

At present the theory is in practice applicable only to systems
with a low intrinsic dimension.
If the system is very turbulent
(a description of its long time dynamics requires a space of high
intrinsic dimension) we are out of luck. 

In a long run, the hope is that the periodic orbit theory
can be applied to real-world
problems, such as moderately turbulent Navier-Stokes
flows, and use calculated results to match or predict
experimental data, or to check and modify the assumptions underlying specific
turbulence models.



