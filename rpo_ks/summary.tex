% summary.tex
% $Author$ $Date$

\section{Summary}
\label{sect:rpo-sum}

We have presented a detailed investigation
of the geometry of the
{\KS} \statesp\ for $L=22$ system size.
At first glance, turbulent dynamics visualized in the \statesp\ might appear
hopelessly complex, but under a detailed examination it is
much less so than feared: it is
pieced together from low dimensional % {1-$d$ return maps}
local unstable manifolds connected by fast transient interludes.
While in general
for $\tildeL$ sufficiently large
one expects many
coexisting attractors in the \statesp%
%Hyman and Nicolaenko
\rf{HNZks86},
in numerical studies most random initial
conditions converge to the same chaotic attractor.
{\KS} (and \pCf, see \refref{GHCW07})  \eqva, \reqva, \po s and
\rpo s embody Hopf's vision:
repertoire of recurrent spatio-temporal
patterns explored by turbulent dynamics.
We use these dynamically invariant solutions
as a scaffolding from which to explore the
\statesp\  topology and chaotic dynamics.
    \PC{must expand this, emphasize novelty, especially
        of the \statesp\ visualization. See \refref{GHCW07}
        for inspiration.
        }


The key new feature of the full, periodic domain
KS, with its continuous translational symmetry,
are the attendant continuous families of
\reqva\ (traveling waves) and \rpo s.
\Rpo s, in particular, will require rethinking dynamical systems
approach to constructing symbolic dynamics.




\PC{
The real motivation for all this is that if we understand \eqva\ as
$L \to \infty$ we might have an entry into $L = \infty$ periodic orbit
theory of KS.
   }
\PC{ I have not proofread appendices}
