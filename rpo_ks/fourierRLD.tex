% fourierRLD.tex
%
% Predrag               jun 20 2006
% $Author$ $Date$


The \KSe\ in terms of Fourier modes:
\beq
  \hat{u}_k = {\cal F}[u]_k = \frac{1}{L}\int_0^L u(x,t) e^{-ikx/\tildeL}dx\,,
  \qquad u(x,t) = {\cal F}^{-1}[\hat{u}] = \sum_{k\in{\mathbb Z}} \hat{u}_k e^{ikx/\tildeL}
\eeq
 is given by
\beq
  \dot{\hat{u}}_k = [(k/\tildeL)^2-(k/\tildeL)^4]\hat{u}_k -
  \frac{ik}{2\tildeL}{\cal F}[({\cal F}^{-1}[\hat{u}])^2]_k\,.
\eeq
Since $u$ is real, the Fourier modes are related by $\hat{u}_{-k} =
\hat{u}^\ast_k$.

The above system is truncated as follows: The Fourier transform
${\cal F}$ is replaced by its discrete equivalent
\beq
  a_k = {\cal F}_N[u]_k = \sum_{n = 0}^{N-1} u(x_n)
  e^{-ikx_n/\tildeL}\,,\qquad u(x_n) = {\cal F}_N^{-1}[a]_n
  = \frac{1}{N}\sum_{k = 0}^{N-1} a_k e^{ikx_n/\tildeL}\,,
\eeq
where $x_n = 2\pi\tildeL/N$ and $a_{N-k} = a^\ast_k$.  Since $a_0
= 0$ due to galilean invariance and setting $a_{N/2} = 0$ (assuming
$N$ is even), the number of independent variables in the truncated
system is $N-2$.  The truncated system looks as follows:
\beq
  \dot{a}_k = \pVeloc_k(a) = [(k/\tildeL)^2-(k/\tildeL)^4]a_k -
  \frac{ik}{2\tildeL}{\cal F}_N[({\cal F}_N^{-1}[a])^2]_k\,.
\ee{eq:KS}
where $k = 1,\ldots,N/2-1$.  Note that, since $a_k \in \mathbb{C}$,
\refeq{eq:KS} represents a system of
%, although in the Fourier transform we need
%to use $a_k$ over the full range of $k$ values from 0 to $N-1$.
The discrete Fourier transform ${\cal F}_N$ can be computed by FFT.
In Fortran and C, the routine {\tt REALFT} from Numerical Recipes
can be used.
% In Matlab, it is more convenient to use complex
% variables for $a_k$.  Note that Matlab function {\tt fft} is, in
% fact, the inverse Fourier Transform.

In order to find the Jacobian of the solution, or compute
Lyapunov exponents of the \KSe , one needs to solve the equation
for a displacement vector $b$ in the tangent space:
\beq
  \dot{b} = \frac{\partial \pVeloc(a)}{\partial a} b\,.
\eeq
Since ${\cal F}_N$ is a linear operator, it is easy to show that
\beq
  \dot{b_k} = [(k/\tildeL)^2-(k/\tildeL)^4]b_k -
  \frac{ik}{\tildeL}{\cal F}_N[{\cal F}_N^{-1}[a]\cdot
  {\cal F}_N^{-1}[b]]\,,
\ee{eq:KStan}
where the dot indicates componentwise product of two vectors, \ie,
$a\cdot b = \diag(a)\,b = \diag(b)\,a$.  This
equation needs to be solved simultaneously with \refeq{eq:KS}.

The Jacobian of the KS flow map, $J(a,t) = \partial f^t(a)/\partial
a$, where the partial derivatives need to be evaluated separately
with respect to the real and imaginary parts of the components of
complex-valued vector $a$.

%% To derive the equation for the matrix of variations, I use the fact
%% that ${\cal F}_N$ is a linear operator.  Since we need to
%% differentiate separately with respect to the real and imaginary
%% components of $a_k$, I use the notation $a_k = b_{2k-1} + ib_{2k}$.
%% \beq
%%   \frac{\partial \dot{a}_k}{\partial b_j} =
%%   [(k/\tildeL)^2-(k/\tildeL)^4]\delta_{kj} -
%%   \frac{ik}{\tildeL}{\cal F}_N[{\cal F}_N^{-1}[a]\cdot{\cal
%%   F}_N^{-1}[\delta_{kj}]]\,,\quad j = 1,\ldots,N-2
%% \eeq
%% where the dot indicates componentwise product, and the inverse
%% Fourier transform is applied separately to each column of
%% $\delta_{kj}$. Here, $\delta_{kj}$ is not a standard Kronecker
%% delta, but a complex valued $N\times N-2$ matrix:
%% \beq
%%   \delta_{kj} = \left(
%%   \begin{array}{ccccc}
%%   0 &  0 & 0 &  0 &\cdots\\
%%   1 &  i & 0 &  0 &\cdots\\
%%   0 &  0 & 1 &  i &\cdots\\
%% \multicolumn{5}{c}\dotfill \\
%%   0 &  0 & 0 &  0 &\cdots\\
%% \multicolumn{5}{c}\dotfill \\
%%   0 &  0 & 1 & -i &\cdots\\
%%   1 & -i & 0 &  0 &\cdots\\
%%   \end{array}  \right),
%% \eeq
%% with index $k$ running from 0 to $N-1$.  I admit, the notations are
%% a bit stretched here, but I find them convenient when coding this
%% equation using FFT.


\section{Symmetries imply possible existence of \rpo s}
\label{sec:SymRPO}
In a dynamical system $\dot{a} = v(a)$ with a strange invarian set, there exist
infinitely many periodic orbits
\[ f^\period{}(a) = a \]
characterized by period $\period{}$, which are dense within the invariant set.
Here $f^t$ is the flow map of the flow $v$, \ie, $a(t) = f^t(a)$ is the
solution of the flow $v$ with initial condition $a(0) = a$.

Let the dynamical system have symmetries represented by the operators
$\Sigma_{k,s}$, where $k \in {\cal K} \subset \mathbb{Z}^p$ are
parameters of discrete symmetries and $s \in {\cal S} \subset \mathbb{R}^q$
are parameters of continuous symmetries.  In other words,
\[ f^t(\Sigma_{k,s} a) = \Sigma_{k,s} f^t(a)\,. \]
In this case it is likely that, in addition to \po s, the dynamical system also
has \rpo s, characterized by the condition
\[ \Sigma_{k,s}f^\period{}(a) = a\,, \]
where, in addition to the period $\period{}$, the \rpo\ is also characterized by
the symmetry parameters $k$ and/or $s$.

In fact, if the symmetry is continuous, then it is much more likely
to find \rpo s, than it is to find exact \po s, since $s = 0$,
corresponding to the \po , is only one specific value in the
continuum of possible values of parameter $s$.

In the case of KS equation, which has continuous symmetry
$\Shift_{\shift/L}$ and discrete symmetry $\Refl$, it is possible to
find \rpo s that satisfy the following conditions
\[
  \Shift_{\shift/L}f^\period{}(a) = a\,
\quad\mbox{and}\qquad
  \Refl\Shift_{\shift/L}f^\period{}(a) = a\,
\]
The first condition is satisfied by \rpo s with shift $\shift$,
while \rpo s that satisfy the second condition are exactly periodic,
\ie\ $\shift = 0$, with period $2\period{}$.
