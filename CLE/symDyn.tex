% siminos/CLE/symDyn.tex
% $Author$ $Date$

\section{\label{s:symDyn} Symmetries of dynamical systems}

Consider a system of \ode s of the form
\beq
	\dot{\ssp} = \vf(\ssp)
	\label{eq:difeq}
\eeq
with $\vf$ a smooth vector field and $x\in\pS\subset\Rls{d}$.
Here we will be interested in the role continuous symmetries
play in dynamics.
    \PC{dropped:
    , so we assume $\Group$ is a compact Lie group.
    Any compact Lie group $\Group$ acting on $\Rls{d}$ is
    a subgroup of $\On{d}$, \cf\ for example \refref{golubII}.
    }
While the key concepts and methods we develop here are well
illustrated by the 1-parameter Lie \SOn{2} group, they are in
principle applicable to any compact Lie group, and the
generalization to translational and rotational symmetries of
PDEs such as \KS\ and \pCf\ is immediate.

A linear action $\LieEl$ is a symmetry of
\refeq{eq:difeq} if
\beq
	\vf(\LieEl x) =\LieEl \, \vf(x)
	\label{eq:equiv}
\eeq
for all $x\in\Rls{d}$. One  says that $\vf$ \emph{commutes}
with $\LieEl$ or that $\vf$ is $\LieEl$-\emph{equivariant}.
When $\vf$ commutes with the set of group elements
$\LieEl\in\Group$, the vector field $\vf$ is said to be $\Group$-equivariant.
Equivalently, the group action $\LieEl$ is said to be a {\em
symmetry} of dynamics if for every solution $x(t)=
\flow{t}{x}$, $\LieEl \, x(t)$ is also a solution. The finite
time flow $\flow{t}{\LieEl x}$ through $\LieEl x$ then
satisfies the equivariance condition
\beq\label{eq:equivFinite}
\flow{t}{\LieEl x}=\LieEl\flow{t}{x}
\,.
\eeq
In physics literature the term $invariant$ is most commonly
used; for example, in Hamiltonian systems a symmetry is
manifested as invariance of the Hamiltonian under the
symmetry group action.

An element of a compact Lie group
continuously connected to identity can be written as
\beq
\LieEl(\gSpace)=e^{\gSpace \cdot \Lg }
	\,,\qquad
\gSpace \cdot \Lg  = \sum \gSpace_a \Lg_a,\; a=1,2, \cdots, N
\,,
\ee{FiniteRot}
where
$\gSpace \cdot \Lg$
is a {\em Lie algebra} element,  and $\gSpace_a$ are the parameters
of the transformation. Repeated indices are summed throughout this
chapter, and the dot product refers to a sum over
Lie algebra generators. The Euclidian product of two vectors
$x,y$ will be indicated by $x$-transpose times $y$, \ie,
$x^T y = \sum_i^d x_i y_i$.
Finite transformations $ \exp(\gSpace \cdot {\Lg}) $ are
generated by sequences of infinitesimal steps of form
\beq
\LieEl(\delta\gSpace) \simeq 1 + \delta \gSpace \cdot \Lg
% \LieEl{}_i{}^j \simeq \delta_i^j +  \delta \gSpace_a \, (\Lg_a)_i^j
    \,,\quad
\delta\gSpace \in \reals^N
    \,,\quad
|\delta \gSpace| \ll 1
    \, ,
\ee{intsmLieTransf}
where $\Lg_a$, the {\em generators} of infinitesimal
transformations, are a set of $N$ linearly independent
$[d\!\times\!d]$ anti-hermitian matrices, $(\Lg_a)^\dagger =
- \Lg_a$, acting linearly on the $d$-dim\-ens\-ion\-al \statesp\
$\pS$.
The flow
at the \statesp\ point $\ssp$ induced by the action of the group
is given by the set of $N$ tangent fields
\beq
\groupTan_a(\ssp)_{i}= (\Lg_a){}_{ij} \ssp_j
\,.
\ee{GroupTangField}


% \subsection{$\SOn{2}$ equivariance}
% Predrag                           Sep 19 2009
% extracted from siminos/thesis/chapters/symInfm.tex
% Predrag                           Aug 22 2009
%       extracted from wilczak/blog/flow.tex

% A flow $\dot{\ssp}= \vel(\ssp)$ is $\Group$-equivariant
% \refeq{GvCommut} if
%
% \beq
% \vel(\ssp)=\LieEl^{-1} \, \vel(\LieEl \, \ssp)
% \,,\qquad \mbox{for all } \LieEl \in {\Group}
% \,.
% \ee{eq:FiniteRot}
For an infinitesimal transformation \refeq{intsmLieTransf}
the $\Group$-equivariance condition \refeq{eq:equiv}
becomes
\[
\vel(\ssp)
      \simeq
  (1-\gSpace \cdot \Lg) \, \vel(\ssp+\gSpace \cdot \Lg \, \ssp)
       = \vel(\ssp)- \gSpace \cdot \Lg \, \vel(\ssp)
             + \frac{d\vel}{d\ssp} \,\gSpace \cdot \Lg \, \ssp
\,.
\]
Denote
the group flow tangent field at \ssp\ by
$\groupTan_a(\ssp)_{i}= (\Lg_a){}_{ij} \ssp_j$. Thus the
infinitesimal, Lie algebra $\Group$-equivariance condition is
\beq
  \groupTan_a(\vel)  - \Mvar(\ssp) \, \groupTan_a(\ssp) =0
  \,,
\ee{inftmInv}
where $\Mvar = {\pde \vel}/{\pde \ssp}$ is the \stabmat.
The left-hand side,
\beq
{\cal L}_{\groupTan_a} \vel =
\left.\left(
  \Lg_a - \frac{\partial}{\partial y}(\Lg_a \ssp)
 \right) \vel(y)\right|_{y=\ssp}
 \,,
\ee{LieDeriv}
is known as
the {\em Lie derivative} of the dynamical flow
field $\vel$ along the direction of the infinitesimal
group-rotation induced flow $\groupTan_a(\ssp)= \Lg_a \ssp$.
The equivariance condition \refeq{inftmInv} states that the two
flows, one induced by the dynamical vector field $\vel$, and
the other by the group tangent field $\groupTan$, commute if
their Lie derivatives (or the `Lie brackets ' or `Poisson
brackets') vanish.


Any representation of a compact Lie group $\Group$ is fully
reducible, and invariant tensors constructed by contractions
of $\Lg_a$ are useful for identifying irreducible
representations. The simplest such invariant is
\beq
\Lg^T \cdot \Lg = \sum_\alpha C_2^{(\alpha)} \, \id^{(\alpha)}
\,,
\ee{QuadCasimir}
where $C_2^{(\alpha)}$ is the quadratic Casimir for
irreducible representation labeled $\alpha$, and
$\id^{(\alpha)}$ is the identity on the $\alpha$-irreducible
subspace, 0 elsewhere. The dot product of two tangent fields
is thus a sum weighted by Casimirs,
\beq
\groupTan(\ssp)^T  \cdot \groupTan(\ssp')
   = \sum_\alpha C_2^{(\alpha)} \ssp_i\, \delta_{ij}^{(\alpha)} \ssp'_j
\,.
\ee{dotProd}
