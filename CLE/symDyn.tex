% symmetry in dynamical systems

We consider a system of \ode s of the form
\beq
	\dot{\ssp} = \vf(\ssp)
	\label{eq:difeq}
\eeq
where $\vf: \Rls{n} \rightarrow \Rls{n}$ a $C^\infty$ mapping.

We call a group element $\gamma\in\On{n}$ a symmetry of
\refeq{eq:difeq} if for every solution $x(t)$, $\gamma x(t)$
is also a solution. Equivalently $\gamma$ is a symmetry of \refeq{eq:difeq}:
\beq
	\vf(\gamma x) =\gamma \vf(x)
	\label{eq:equiv}
\eeq
for all $x\in\Rls{n}$. We say that $\vf$ \emph{commutes} with
$\gamma$ or that $\vf$ is $\gamma$-\emph{equivariant}. When
$\vf$ commutes with all $\gamma\in\Group$ we say that $\vf$
is $\Group$-equivariant. The finite time flow
$\flow{t}{\gamma x_o}$ through $\gamma x_o$ satisfies the
equivariance condition $\flow{t}{\gamma x_o}=\gamma
\flow{t}{x_o}$ from definition of symmetry and uniqueness of
solutions. In physics literature the term $invariant$ is most
commonly used, mostly because in Hamiltonian systems symmetry
is manifested as invariance of the Hamiltonian under the
symmetry operation.



