% siminos/CLE/CLEsols.tex
% $Author$ $Date$

\subsection{\label{s:CLEsols} An example: Solutions of \cLe}

In the case of {\cLe}  the origin \EQV{0} is an \eqv\ of
\refeq{eq:CLe} for any value of the parameters. It is stable
for $0<\RerCLor<\rho_{1c}$ and unstable for
$\rho_{1c}<\RerCLor$, where\rf{FowlerCLE82}
\[
	\rho_{1c} = 1 + {(e+\ImrCLor)(e-\sigma \ImrCLor)}/{(\sigma+1)^2}
\,.
\]
At the bifurcation a pair of eigenvalues crosses the imaginary
axis with imaginary part
\beq
	\omega_c = {\sigma (e + \ImrCLor)}/{(\sigma+1)}
\,,
\ee{eq:omegaCLE}
and a \emph{relative equilibrium} \REQV{}{1} with constant
angular velocity $\omega_c$ is born. For $\omega_c =0$ the
\reqv\ degenerates to an \SOn{2}-orbit of \eqva. As the
existence of a \reqv\ in a system with \SOn{2} symmetry is
the generic situation, we follow \refref{BakasovAbraham93}
and set $\ImrCLor=0$ and $e \neq 0$.

As $\RerCLor$ is increased,  a secondary bifurcation from
\REQV{}{1} results in a \emph{\rpo} \refeq{RPOrelper1}, or,
more precisely, in the quasiperiodic 2-frequency
\emph{modulated traveling wave}\rf{Krupa90}. With further
increase in $\RerCLor$ the dynamics turns chaotic, with
infinity of unstable {\rpo s}. This is illustrated by the
portrait of \cLf\ \statesp\ in  \reffig{fig:CLE}, with the
\reqv\ \REQV{}{1} and three repetitions of \cycle{01} \rpo\
superimposed over a generic chaotic orbit. Repeats of
\cycle{01} trace out ergodically a torus, so in a system with
a $1$-dimensional continuous symmetry the organizational
blocks of a strange attractor are circles (\reqva) instead of
points (\eqva), and partially hyperbolic tori (\rpo s)
instead of closed loops (\po s). It is difficult to
understand the geometry of the flow by looking at such tori.
Dynamics is organized by the interplay of the stable and
unstable manifolds of \eqv\ \EQV{0} and \reqv\ \REQV{}{1} but
the symmetry induced drift along the direction of rotation
blurs the picture and the notion of recurrence becomes
relative. This confusing situation (as well as the
theoretical fact\rf{Cvi07} that dynamical zeta functions have
their support on \rpo s) motivates the search for effective
methods to project the dynamics onto a \reducedsp\ in what
follows.

To find the location of the \reqv\ it is convenient to work
on polar coordinates
\beq
(x_1,x_2,y_1,y_2,z) =
    (r_1 \cos\theta_1,r_1\sin\theta_1,
     r_2\cos\theta_2,r_2\sin\theta_2,z)
\,,
\label{eq:CartToPol}
\eeq
where $r_1 \geq 0 \,,r_2 \geq 0$.
In polar coordinates the \cLe\ equations  \refeq{eq:CLe} take form
\[ %\beq
\left(
\begin{array}{c}
\dot{r}_1\\
\dot{\theta}_1\\
\dot{r}_2\\
\dot{\theta}_2\\
\dot{z}
\end{array}
\right)
=
\left(
\begin{array}{c}
 -\sigma\left(r_1 - r_2\cos\theta\right) \\
 -\sigma\frac{r_2}{r_1}\sin \theta  \\
 -r_2 + r_1\left((\rho_1-z)\cos \theta - \rho_2 \sin\theta\right)\\
  e  + \frac{r_1}{r_2}\left((\rho_1-z)\sin\theta +\rho_2 \cos\theta\right)\\
 -b z + r_1 r_2\cos\theta
\end{array}
\right)
,
\] %\ee{eq:PolarCLe}
For
rotationally invariant flows the relative angle is
invariant (that is why one speaks of `relative' equilibria),
hence we introduce a variable $\theta = \theta_1-\theta_2$.
$\theta_1$ and
$\theta_2$ change in time, but at the \reqva\ the
difference between them is constant. This enables us to rewrite the \cLe\
as four coupled polar coordinate ODEs:
\beq
\left(
\begin{array}{c}
\dot{r}_1\\
\dot{r}_2\\
\dot{\theta}\\
\dot{z}
\end{array}
\right)
=
\left(
\begin{array}{c}
 -\sigma\left(r_1 - r_2\cos\theta\right) \\
 -r_2 + (\rho_1-z)r_1\cos \theta\\
  -e -\left(\sigma\frac{r_2}{r_1}
 +(\rho_1-z)\frac{r_1}{r_2}\right)\sin\theta\\
 -b z + r_1 r_2\cos\theta
\end{array}
\right)
\label{eq:PolarCLeTheta}
\eeq
where we have set $\rho_2=0$.







    %PC: Rebecca and I rederived these: they check.
\beq
\begin{split}
	\dot{r}_1 &=-\sigma (r_1 - r_2\cos\phi) \cont
	\dot{r}_2 &=-r_2 + r_1(\RerCLor -z)(\cos\phi-\ImrCLor\sin\phi) \cont
	\dot{z} &=  -b z+r_1 r_2\cos\phi \cont	
	\dot{\phi} &=-e-\frac{\sigma r_2 \sin\phi}{r_1}
             -\frac{r_1(\RerCLor-z) (\ImrCLor\cos\phi+\sin\phi) }{r_2}\,,
	\label{eq:CLePolar}
\end{split}
\eeq
where $\phi=\phi_1-\phi_2$ and the evolution equations for
$\phi_1,\phi_2$ are given by
\beq
\begin{split}
	\dot{\phi}_1 &=-\frac{\sigma r_2 \sin\phi}{r_1}\cont
	\dot{\phi_2} &= e +\frac{r_1\left(
         (\RerCLor -z)\sin\phi+\ImrCLor\cos\phi
                                \right)}{r_2}\,.
	\label{eq:CLeAngl}
\end{split}
\eeq

For simplicity we now turn to the ``laser case''
$e\neq0,\;\ImrCLor=0$. The condition for a \reqv\ is that all
time derivatives in \refeq{eq:CLePolar}, while
$\dot{\phi}_1=\dot{\phi}_2\neq 0$. If
$\dot{\phi}_1=\dot{\phi}_2=0$ we would have (a group orbit
of) equilibria. We get
\beq
\begin{split}
	z^{(1)} &= \frac{-e^2+(\RerCLor -1)(\sigma +1)^2}{(\sigma +1)^2}\cont
	r_1^{(1)} &= \sqrt{b z^{(1)}}\cont
	r_2^{(1)} &= \sqrt{b \left(e^2+(\sigma +1)^2\right)z^{(1)}}\cont
	\phi^{(1)} &= -\cos ^{-1}\left(\frac{\sigma +1}{\sqrt{e^2+(\sigma +1)^2}}\right)\,.
\end{split}
\eeq
Substituting in \refeq{eq:CLeAngl} we get
$\dot{\phi}_1=\dot{\phi}_2=e \sigma/(1 + \sigma)\neq 0$ for
$e\neq0$ and thus we have indeed a \reqv, not a group orbit
of \eqva.

Calculation  in polar coordinates $r_1,r_2,\phi,z$ of
stability eigenvalues for \REQV{}{1} for the set of
parameters we use here yields a weakly unstable spiral-out
\eqv\
\beq
	\eigRe[1]\pm i\eigIm[1]= 0.0938\pm 10.1945i,\,
    \eigExp[3]=-11.0009,\, \eigExp[4]= -13.8534\,.
	\label{eq:CLeREQBstab}
\eeq
In \ref{s:StabReq} we show how to calculate stability of
\reqva\ in equivariant variables, without change of
coordinates to polar or any other set of symmetry invariant
variables.
