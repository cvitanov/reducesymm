% siminos/CLE/symRedGeneral.tex
% $Author$ $Date$

%Discussion of symmetry reduction methods
% extracted from siminos/thesis/chapters/symODEs.tex

Given Lie group \Group\ acting smoothly on a $C^\infty$
manifold \pS, we can think of each group orbit as an
equivalence class. {\em Symmetry reduction} is the
identification of a unique point on a group orbit as the
representative of its equivalence class.

We shall refer to this space as the {\em \reducedsp}.
In the literature this is alternatively called the
\emph{desymmetrized \statesp},
 the \emph{orbit space}, or \emph{quotient space}
$\pS/\Group$ because symmetry has been `divided out.'
Symmetry group \Group\ of equivariant dynamics acts trivially
in reduced space, and the resulting dynamical system, called
by Gilmore and Lettelier\rf{GL-Gil07b} the \emph{image}, is
symmetry {\em invariant}, in the sense that its symmetry group is
the identity.
The mapping equivariant dynamics to invariant dynamics is
implemented
by methods such as
{\em \mframes},
{\em \csection s},
{\em \mslices},
{\em Hilbert bases},
\etc.
%
\ES{merge this:  Here we
first motivate the method by considering its
finite-rotations version, and then derive its differential
formulation, following \refref{rowley_reduction_2003}.
{\bf PC} irrelevant remark: I did not `follow'
\refref{rowley_reduction_2003}. I first derived the equations,
then identified them in earlier papers, as is what usually happens
}
    \PC{Cite literature that uses each of the above}


As the systems that we study here are not associated with a
variational principle, Noether's theorem does not
apply, and in general no conserved
quantities are associated with continuous symmetries of
such systems.
    \PC{include discussion, references from the blog here.}
    \PC{not sure about this: ``
Such a conserved quantity would restrict dynamical
trajectories to an invariant manifold locally transverse to
the direction of group action. In the contrary here the
system also evolves along the direction of group action.
    ''
Being on a constant energy surface does not mean we do
not evolve in time, for example.
{\bf ES:} You are right, this statement is not correct in
general. What I had in mind was cases such as
conservation of angular momentum in central force 		
problems fixes the plane of motion.
}


\PublicPrivate{}{
The stratification
of $\Manif$ induced by the group action is carried over to
the quotient space with each disconnected set in a stratum
mapped to the same manifold in quotient space. Yet, a
fundamental problem with symmetry reduction is that the orbit
space is in general not a manifold. Unless the action of the
group is free, group orbits do not have the same dimension
and different strata are mapped to manifolds of different
dimension. We will see this property of quotient space
manifest itself in different ways depending on the reduction
method but always introducing some singularity even though
there is nothing singular about $\Manif$ or the flow of the
dynamical system on it.
        \ES{Either define stratum, or simplify the discussion
		here.}
}

Perhaps the most appealing approach to symmetry reduction is
through the use of a Hilbert basis of invariant polynomials.
One computes a (non-unique) basis of linearly independent
polynomials, invariant under the action of the symmetry group
(\cf \refref{gatermannHab,ChossLaut00} for a discussion of
methods) and either rewrites the dynamics in this basis or
maps the solutions to the polynomial variables. The reader is
referred to the book of Gilmore and Lettelier\rf{GL-Gil07b}
for a very detailed discussion of symmetry reduction, through
the use of invariant polynomials.
    \ES{Also Chossat and Lauterbach?}
