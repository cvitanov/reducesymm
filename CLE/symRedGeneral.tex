% siminos/CLE/symRedGeneral.tex
% $Author$ $Date$

%Discussion of symmetry reduction methods
% extracted from siminos/thesis/chapters/symODEs.tex

Given Lie group \Group\ acting smoothly on a $C^\infty$
manifold \pS, we can think of each group orbit as an
equivalence class. {\em Symmetry reduction} is the
identification of a unique point on a group orbit as the
representative of its equivalence class. We call the set of
all such group orbit representatives the \emph{orbit
 space}, or  or \emph{quotient space} $\pS/\Group$ because 
symmetry has been ``divided out'' or
simply \emph{reduced space}. Symmetry group \Group\ of equivariant
dynamics acts trivially in reduced space, and the resulting dynamical
system, called \emph{image} of the original, is symmetry invariant,
in the sense that it's symmetry group is the identity.

\ES{merge this:  Here we
first motivate the method by considering its
finite-rotations version, and then derive its differential
formulation, following \refref{rowley_reduction_2003}.
%
We shall refer to $\pS/\Group$ as {\em `\reducedsp'}.
In the literature this is alternatively called
`desymmetrized {\statesp},'
% `reduced \statesp,'
`orbit space,'
`quotient space,'
or
`image space,'
obtained by mapping equivariant dynamics to invariant dynamics
by methods such as
`moving frames,'
`\csection s,'
`\slice s,'
`Hilbert bases,'
\etc.
}
    \PC{Cite literature that uses each of the above}


As the systems that we study here are not associated with a
variational principle, Noether's theorem does not
apply, and in general no conserved
quantities are associated with continuous symmetries of
such systems.
    \PC{include discussion, references from the blog here.}
    \PC{not sure about this: ``
Such a conserved quantity would restrict dynamical
trajectories to an invariant manifold locally transverse to
the direction of group action. In the contrary here the
system also evolves along the direction of group action.
    ''
Being on a constant energy surface does not mean we do
not evolve in time, for example.
{\bf ES:} You are right, this statement is not correct in
general. What I had in mind was cases such as
conservation of angular momentum in central force 		
problems fixes the plane of motion.
}


\PublicPrivate{}{
The stratification
of $\Manif$ induced by the group action is carried over to
the quotient space with each disconnected set in a stratum
mapped to the same manifold in quotient space. Yet, a
fundamental problem with symmetry reduction is that the orbit
space is in general not a manifold. Unless the action of the
group is free, group orbits do not have the same dimension
and different strata are mapped to manifolds of different
dimension. We will see this property of quotient space
manifest itself in different ways depending on the reduction
method but always introducing some singularity even though
there is nothing singular about $\Manif$ or the flow of the
dynamical system on it.
        \ES{Either define stratum, or simplify the discussion
		here.}
}
