% siminos/CLE/intro.tex
% $Author$ $Date$
%
In his seminal paper E. Lorenz\rf{lorenz} had reduced both
the continuous time and the spatial symmetries of the
3-dimensional Lorenz equations in order to construct a
1-dimensional return map that yields deep
insights\rf{tucker1-2} into the nature of chaos in this
paradigmatic flow. For strongly contracting, low-dimensional
flows Gilmore, Lefranc and
Letellier\rf{gilmore2003,GL-Gil07b} have systematized
construction of such discrete time return maps, through use
of topological templates, Poincar\'e sections (to reduce the
continuous time invariance) and invariant polynomial bases
(to reduce the spatial symmetries). They show that in
presence of spatial symmetries one has to  `quotient' the
symmetry and replace the dynamics by a physically equivalent
reduced, desymmetrized flow, a flow in which each family of
symmetry related states is replaced by a single
representative. The approach leads to the symbolic dynamics
and labeling of all \po s up to a given topological period.
The periodic orbit theory can then yield accurate estimates
of long-time dynamical averages, such as Lyapunov exponents
and escape rates\rf{DasBuch}.

In a series of papers Cvitanovi\'{c}, Putkaradze,
Christiansen and Lan%
\rf{Christiansen97,chfield,LanThesis,CvitLanCrete02,lanVar1,lanCvit07}
have shown that effectively low-dimensional return maps can
be constructed for the high-dimensional (formally infinite
dimensional)  flows described by dissipative partial differential
equations (PDEs) such as the \KSe\ (KS) close to the onset of
`turbulence' or `spatio-temporal chaos.' Such flows have
state-space topology vastly more complicated than the Lorenz
flow, and collections of local Poincar\'e sections together
with maps from a section to a section are required to capture all
of the important asymptotic dynamics.
These KS studies were facilitated by a restriction to the
flow-invariant subspace of odd solutions, but at a price:
elimination of the translational symmetry of the KS system
and with it physically important phenomena, such as traveling
waves. Traveling (or `relative') unstable coherent solutions
are ubiquitous and play a key role in organization of
turbulent hydrodynamic flows, as recently revealed both by
simulations\rf{KawKida01,FE03,WK04,Visw07b,GHCW07} and
experimentation\rf{science04}\ES{update references}.
For KS\rf{SCD07,SiminosThesis}, and even for a relatively
low-dimensional flow such as the
\cLe\rf{GibMcCLE82,FowlerCLE82} used as an example here, with
the simplest continuous (rotational) spatial symmetry
possible, the symmetry induced drifts obscure the underlying
hyperbolic dynamics. The question that we address here is how
is one to construct suitable return maps for arbitrarily
high-dimensional but strongly dissipative flows flows in
presence of continuous symmetries.

In \refsect{s:symDyn} we review the basic notions of symmetry
in dynamics and motivate the need for symmetry reduction.
\refSect{s:introCLE} introduces the \SOn{2}\ equivariant
\cLe\ (CLE), the 5-dimensional set of ODEs that we use
throughout the paper, as a conceptually  simple framework
with which we illustrate the strengths and drawbacks of different
approaches of symmetry reduction.
In \refsect{s:symSol} we describe important classes of
solutions and their symmetries: \eqva, \reqva, \po a and \rpo s,
and use them motivate the need for symmetry reduction.

In \refsect{s:Hilbert} we introduce one of the standard tools
by which the continuous symmetry reduction can be achieved,
projection to a Hilbert basis.
In \refsect{sec:mf} we review the {\mframes}, a direct and
efficient method to compute symmetry invariant bases that
goes back to Cartan, and in \refsect{sec:CLeMovFr} we
illustrate it by applying the method to the \cLe. The method
maps all solutions to a `slice,' a submanifold  of state
space that plays a role for group orbits akin to the role
Poincar\'e sections play in reducing the continuous time
invariance. In contrast to the Hilbert basis approach, slices
are local, and one might need to use more than one slice to
capture the flow globally. In \refsect{s:laserMFnum} we show
that a single local slice suffice for the purpose of reducing
the \cLe\ flow to a return map.
    \ES{Create subsection if that part stays in the paper.
    {\bf PC} dropped:
    We discuss some remedies to the
    situation in \refsect{sec:mf}.}
In \refsect{sec:MovFrameODE} we recast the {\mframes}
into the equivalent, differential \mslices, with the
time integration
restricted to a slice fixed by a given \statesp\ point.
As the slice is local, both methods thus suffer from the same
restrictions, with generic trajectory within slice bound to encounter
singularities.

In \ref{s:StabReq} we derive an expression for the stability
of \reqva\ (traveling waves) in the reduced space
that can be applied without an explicit
calculation of the reduced system.
	\PC{perhaps add ChaosBook.org historical notes here, or a bit earlier in
	the introduction, sis out of the way.}
