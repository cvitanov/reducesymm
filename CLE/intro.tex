% siminos/CLE/intro.tex
% $Author$ $Date$
%
In his seminal paper E. Lorenz\rf{lorenz} reduces both the
continuous time and the spatial symmetries of the
3-dimensional Lorenz equations in order to construct a
1-dimensional return map that yields deep
insights\rf{tucker1-2} into the nature of chaos in this
paradigmatic flow. For strongly contracting, low-dimensional
flows Gilmore, Lefranc and
Letellier\rf{gilmore2003,GL-Gil07b} systematize and
generalize construction of such discrete time return maps,
through use of topological templates, Poincar\'e sections (to
reduce the continuous time invariance) and invariant
polynomial bases (to reduce the spatial symmetries). They
show that in presence of spatial symmetries before return
maps can be constructed one has to  `quotient' the symmetry
and replace the dynamics by a physically equivalent,
desymmetrized flow, a flow in which each family of symmetry
related states is replaced by a single representative. The
approach leads to symbolic dynamics and labeling of all \po
s up to a given topological period. Periodic orbit theory
can then yield accurate estimates of long-time dynamical
averages, such as Lyapunov exponents and escape
rates\rf{DasBuch}.

In a series of papers Cvitanovi\'{c}, Putkaradze,
Christiansen and Lan%
\rf{Christiansen97,chfield,LanThesis,CvitLanCrete02,lanVar1,lanCvit07}
have shown that effectively low-dimensional return maps can
be constructed for the high-dimensional (formally infinite
dimensional)  flows described by dissipative partial differential
equations (PDEs) such as the \KSe\ (KS) close to the onset of
`turbulence' or `spatio-temporal chaos.' Such flows have
vastly more complicated state-space topology than the Lorenz
flow, and collections of local Poincar\'e sections together
with maps from section to section are required to capture all
of the important asymptotic dynamics.

These KS studies were facilitated by a restriction to the
flow-invariant subspace of odd solutions, but at a price:
elimination of the translational symmetry of the KS system
and with it physically important phenomena, such as traveling
waves. Traveling (or `relative') unstable coherent solutions
are ubiquitous and play a key role in organization of
turbulent hydrodynamic flows, as recently revealed both by
simulations\rf{KawKida01,FE03,WK04,Visw07b,GHCW07} and
experimentation\rf{science04}\ES{update references}. As we
shall show here, even for a relatively low-dimensional flow
such as the \cLe\ used as an example in this paper, with the simplest
continuous (rotational) symmetry possible, the symmetry
induced drifts obscure the underlying hyperbolic dynamics.
The question that we address here is how is one to construct
suitable return maps for arbitrarily high-dimensional but
strongly dissipative flows in presence of continuous
symmetries.
\ES{We have to refer to KS paper
and my thesis as they show the need for symmetry reduction in KS.}

In \refsect{s:symDyn} we review the basic notions of symmetry
in dynamics and motivate the need for symmetry reduction.
\refSect{s:introCLE} introduces the \SOn{2}\ equivariant
\cLe\ (CLE), the 5-dimensional set of ODEs that we use
throughout the paper, as a conceptually  simple framework
with which we illustrate the strengths and drawbacks of different
approaches of symmetry reduction.
    \PC{move this to summary:
 Application to
high-dimensional flows is discussed
elsewhere\rf{SiminosThesis,SCD09b}.
    }
In \refsect{s:symSol} we describe important classes of
solutions: \eqva, \reqva, \po s and \rpo s, their symmetries,
and use them motivate the need for symmetry reduction.
In \refsect{s:Hilbert} we introduce one of the standard tools
by which the continuous symmetry reduction can be achieved,
projection to a Hilbert basis. These invariant polynomial
bases can be algorithmically determined
for both Hamiltonian and dissipative systems.
    \PC{move this to summary:
Unfortunately, the computational cost of such algorithms is
at present prohibitive for state space dimensions larger than
ten, but we study fully resolved simulations of PDEs, with
state space dimensions of the order of few tenths to few
hundreds (for \KS\ flow), and well into the tenths of
thousands (for pipe and \pCf s).
    }

In \refsect{sec:mf} we review the \emph{\mframes}, a direct
and efficient method to compute symmetry invariant bases that
goes back to Cartan. The method maps all solutions to a
submanifold or \emph{slice} of state space that plays a role
for group orbits akin to the role Poincar\'e sections play in
reducing the continuous time invariance. In contrast to the
Hilbert basis approach this is a local method for the
symmetry groups we are interested in as the invariant bases
are not defined in subsets of state-space and one is forced
to use more than one slices. We discuss some remedies to the
situation in \refsect{sec:mf}
    \ES{Create subsection if that part stays in the paper.}
and show that local moving frames suffice for the purpose of
reducing the flow to return maps in \refsect{s:laserMFnum}.

In \refsect{sec:MovFrameODE} we show how {\mframes} is
connected with a method for symmetry reduction studied in the
context of PDEs by Rowley and
Marsden\rf{rowley_reconstruction_2000}, see also
\rf{rowley_reduction_2003}, and Beyn and
Th\"ummler\rf{BeTh04,Thum05}.
	\PC{the credits have to be totally rewritten -the relevant literature
	    is in ChaosBook}
Essentially one restricts integration on a slice transverse
to the group action at a given state-space point. We show
that this can be thought of as a differential formulation of
the moving frame method and thus suffers from the same
restrictions, being essentially local and bound to encounter
singularities.

In \ref{s:StabReq} we derive an expression for the stability
of \reqva\ (traveling waves) in the reduced space
that can be applied without an explicit
calculation of the reduced system.
	\PC{perhaps add ChaosBook.org historical notes here, or a bit earlier in
	the introduction, sis out of the way.}
