\PublicPrivate{}{
Implementing symmetry reduction in any of the above ways, the
reward is the same: The dynamics are reduced to a return map
to the Poincar\'e section, which due to the very strong
contraction is approximately $1$-dimensional. The dynamics on
the Poincar\'e section are parametrized by the Euclidean
distance of points along the unstable manifold, as we did for
the Lorenz example. The return map is unimodal and allows for
systematic determination of all cycles of a given length.
Here we were able to determine all cycles up to length $7$,
using the return map of \reffig{fig:CLEinvRM} to generate
guesses, and the multiple-shooting algorithm\ES{reference.}
to refine them to machine accuracy.
}% end Private


In both the finite and infinitesimal time formulations we might
understand the singularities through the analogy between a slice
and a Poincar\'e section. Singularities are introduced in points
in space that are fixed under the group action. Fixed points of time 
evolution are atypical and do not lie on a Poincar\'e section 
unless the section has been specifically designed to contain them.
Therefore one cannot expect, for a flow with many fixed points, to
get away with using just a single Poincar\'e section. The same is
true for fixed points of group actions and their generalization, \fixedsp.
One cannot expect to use a single \slice\ to cover different \fixedsp s.

\CLe\ example is deceptively simple: a single non-trivial \fixedsp\ in combination
with a single equilibrium and a relative equilibrium. Therefore a single Poincar\'e
section and a single slice suffice to reduce the flow to a return map with no
great penalty paid for the exclusion of the, rather boring dynamically, $z$-axis. 
As shown in \refref{siminosThesis} the slight complication that arises by the
coexistense of four equilibria, two relative equilibria and a nice nested \fixedsp\
structure in a effectivelly $8$-dimensional \KS\ system\rf{SCD07} 
suffices to create a situation in which more slices are needed. This will
be the subject of.

In more convoluted systems, such as in \KSe\ with a wealth of equilibria 