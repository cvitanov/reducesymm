% $Author$ $Date$
%
% Predrag created file siminos/CLE/flotsam.tex      Dec 20 2009

\chapter{Flotsam}

This chapter contains material which has not been included in
publication 
{\tt siminos/CLE/CLE.tex}.

{\bf ES 2009-12-19 replaced:}
 On the other hand when one faces
nonlinear field theories\rf{chfield}, either classical or
quantum, the identification existense of order and organization
identified within the bewildering wealth of solutions. Dynamics
within the chaotic attractor of a low-dimensional, continuous
time, (state-space-)volume contracting flow can be in many
cases\rf{gilmore2003} understood by reduction to a discrete time map within a
Poincar\'e section. For sufficiently strong volume contraction
such a mapping provides a complete topological characterization
of the attractor\rf{gilmore2003}. Moreover the set of compact
invariant solutions, equilibria and periodic orbits, organize
the dynamics around them.

{\bf ES 2009-12-19 dropped: }
Nevertheless, many of the early examples of chaotic attractors where
observed in very drastic truncations of PDEs, such as the Lorenz flow\rf{lorenz}.

, see, for example, 
Cushman and Bates\rf{cushman_global_1997} or Marsden and Ratiu\rf{marsden_introduction_1999}


{\bf ES 2009-12-20 dropped}, not sure it is true and does not offer to the discussion: 

When one takes syzygies into account in rewriting the
dynamical system, singularities are introduced. For instance
if we solve \refeq{eq:syzLaser} for $u_2$ and substitute into \refeq{eq:CLEip}
the latter reads
\beq
\begin{split}
  \dot{u}_1 &=2\,\sigma\,(u_4-u_1)\,,\\
  \dot{u}_2 &=-2\left(\,\frac{u_3^2+u_4^2}{u_1} - \rho_2\, u_3 -\,(\rho_1-u_5)\,u_4\right)\,,\\
  \dot{u}_3 &=-(\sigma\, +1)\,u_3+\rho_2\, u_1+e\, u_4\,,\\
  \dot{u}_4 &=-(\sigma\, +1)\,u_4+\,(\rho_1-u_5)\,u_1+\sigma\, \frac{u_3^2+u_4^2}{u_1}-e\,u_3\,,\\
  \dot{u}_5 &=u_4-b\, u_5\,
\end{split}
\label{eq:CLEipSyz}
\eeq
clearly singular as $u_2\rightarrow 0$. 


Moreover when
one \emph{lifts} the dynamics from the quotient space
$\Manif/G$ to the original space $\Manif$ the transformations
have singularities at the \fixedsp s of
the isotropy subgroups in $\Manif$, in the optimal case, \cf
\refref{GL-Gil07b}. Those singularities do not seem to
restrict our ability to use invariant polynomials to obtain
symmetry reduced projections of the dynamics.

