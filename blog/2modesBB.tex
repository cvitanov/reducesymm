% siminos/blog/2modesBB.tex
% $Author$ $Date$

% \chapter{Atlas}
% \label{chap:atlas}

% Predrag 2013-08-10 Burak, git version only
\subsection{Burak's {\twoMode} progress}
\label{chap:2modesBB}

\noindent
{\color{red} Do edit this git file (created 2013-08-10),
it is absent from the original svn version}
\bigskip\bigskip

\begin{description}
\item[2012-05-07  Predrag to Chaos Gang] It's not over until it is over.

\item[2013-07-25  Predrag]
 - instead of computing \cLf\ one more
time, how about giving a try to \twoMode\ $\SOn{2}$-equivariant flow,
defined in the \texttt{reducesymm/cgang/2modes.tex} (you can get it by a click
\HREF{../cgang/2modes.pdf}{here}, provided you had already pdflatex-ed
2modes.tex). Have a look at it, and then meet with Daniel Borrero, 3. floor
Schatz lab, who can walk you through what we had already done and learned
(all in the 2modes.pdf blog). You can play with it for -let's say- two
weeks, see whether you can find an interesting strange attractor worth
slicing. If that does not work out, we'll give up, and go to \KS\ instead,
which is much more important for our overall goals...

\item[2013-08-06 Predrag]
A more precise statement of what we are trying to achieve with this model is
in \emph{siminos/cgang/2modes.tex:}

``For the 4\dmn\ model at hand are using the invariant polynomials
$\{u,v,w,q\}$ dynamics only to develop intuition, but to illustrate the
general \mslices, everything has to do be done in $\pS =
\{x_1,x_2,y_1,y_2\}$ and \slice\ $\pSRed$ as well. You can see that even
for the simplest conceivable $\SOn{2}$ 4-dimensional flow one has to
think about how to construct the invariant polynomial basis, and it is
hard to imagine how anyone could do that for very high\dmn\ flows.''

If we can find a nice strange attractor with comparable amplitudes in the
two modes, and show how to slice it, that would be the simplest example of
power of slicing, as the symmetry-reduced dynamics is 3\dmn\ and something
a human can look at, turn around as 3\dmn\ Mathematica or Matlab figure.

If in addition it turns out that my favorite Sobolev norm (read
\refchap{c-norms}) reduces the number of local slice hyperplanes needed
to cover the attractor, I'll be doubly happy.

\begin{figure}[ht]
\begin{center}
\includegraphics[width=0.9\textwidth]{PKperorb}
\end{center}
\caption{A typical
%Predrag 2013-08-05 removed ``periodic orbit''
\reqv\  of Porter-Knobloch flow.}
\label{fig:PKperorb}
\end{figure}

\item[2013-08-05  Burak] I read the Porter-Knobloch blog and
    \HREF{http://chaosbook.org/paper.shtml\#stability} {Chapter 4} -
    {\em Local stability}, confirmed most of the findings in blog,
    naively experimented on the parameters of the system in $x_i, y_i$
    basis tried to find equilibria, got nothing, then talked to Daniel,
    and re-read the blog and come up with a Monte-Carlo (kind of)
    algorithm hoping that it could get me a strange attractor. So far,
    I only got periodic orbits. \refFig{fig:PKperorb} is a typical one.

\item[2013-08-06 Predrag]
Got worried that there were no updates for 11 days - how about if agree
on a schedule, let's say git pushes
\textcolor[rgb]{1.00,0.00,0.00}{every Monday and Friday}?

                                            \toCB
The \po s (actually, \reqva) that you find are presumably \emph{limit
cycles}. In ChaosBook I define `limit cycles' as \po s which are strictly
exponentially attracting forward in time. Parenthetically, in her thesis
De Witte\rf{DWRGK13,DeWitte13} defines a `limit cycle' as an ``isolated
periodic orbit'' thusly:

``A cycle of a continuous-time dynamical system, in a neighbourhood of
which there are no other cycles, is called a limit cycle.''

That presumably has advantage of being true for both directions of time,
but I do not think we need to get that finicky...


\item[2013-08-05  Burak]
    Here is what I did:
\begin{itemize}
\item
Generate pseudo-random set of parameters ensuring $\mu_1 > -\mu_2 > 0$, $c_1 = 1$ and $c_2 = -1$ as suggested in [2012-04-29 Predrag] and [2012-08-06 Edgar Knobloch]
\item
Numerically compute roots of \refeq{PKinvEqs5a} in $u>0, v>0$ region
starting from pseudo-random pair of points $(u,v)$, to find an
equilibrium in invariant polynomial basis.
\bea
\tilde{f}(\tilde{u},\tilde{v}) &=&
  \tilde{u}\,A_1 - \tilde{v}\,A_2 = 0 %Double checked DB 04-30-2012
\,,\qquad\qquad\qquad\qquad  deg(f) = 2
\continue
\tilde{g}(\tilde{u},\tilde{v}) &=&  %Double checked DB 04-30-2012
 \left(A_1^2
 - c_1\,\tilde{v}\right)
 \left(\tilde{u}+2\,\tilde{v}\right)^2
 +e_2^2\,\tilde{v}^2 = 0
\,,\qquad  deg(g) = 4
\continue
&& \mbox{where }
A_1 = \mu_1+\tilde{a_1}\,\tilde{u}+\tilde{b_1}\,\tilde{v},
\continue
&& \,\,\,\,\qquad A_2 = \mu_2+\tilde{a_2}\,\tilde{u}+\tilde{b_2}\,\tilde{v}
\,.
\label{PKinvEqs5a}
\eea
\item
Calculate corresponding w and q and check if the syzygy holds (it does).
\item
Calculate eigenvalues of the stability matrix at this point.
\item
If the stability matrix has at least one eigenvalue with positive real part (repulsive), at least one eigenvalue with negative real part (attractive) and a complex pair of eigenvalues with non-zero imaginary part (spiral); keep the parameters and the equilibrium point.
\item
Numerically calculate the points $x_i, y_i$ corresponding to the equilibrium in invariant polynomial basis, using following relations:
\bea
  u &=& x_1^2 + x_2^2\,,
\continue
  v &=& y_1^2 + y_2^2\,,
\continue
  w &=& 2x_1^2y_1+4x_1x_2y_2 - 2x_2^2y_1\,,
\continue
  q &=& 2x_1^2y_2+4x_1x_2y_1 + 2x_2^2y_2\,.
\label{eq:PKinvxirels}
\eea
\item
Integrate the Porter - Knobloch velocity function to see time evolution in real coordinates.
\end{itemize}
So far, I got divergent solutions and periodic orbits using parameters that I found this way. My questions:
\begin{itemize}
\item
If I check the eigenvalues of the stability matrix for real coordinates,
I get 3 of the eigenvalues almost same with the ones I get for the
invariant polynomial basis, and one eigenvalue 0 (usually something less
then $10^{-4}$). This gives me the feeling of I am doing things correct,
however, I want to make more sense out of this. Is there a clear
discussion about how these eigenvalues remain unchanged under coordinate
transformations (I saw the discussion about traces in the blog, I
confirmed the result that traces of stability matrices in $u,v,w,q$ basis
and $x_1,x_2,y_1,y_2$ basis are not the same at the origin.).
\item
Is what I did reasonable at all? Is there any obvious wrong logic?
\item
Would you suggest any other restrictive criteria to pick a ``good'' set of parameters, in addition to the ones I force on eigenvalues of the stability matrix? I thought, maybe I should take parameters for which the positive and negative real-part eigenvalues are of the same order.
\item
Is an equilibrium in invariant polynomial basis ($u,v,w,q$) a relative equilirium in real ($x_i,y_i$) basis? If not, what sense I should make out of the fact that the relations \refeq{eq:PKinvxirels} do not provide a unique set of ($x_i,y_i$) for given ($u,v,w,q$).
\end{itemize}
After writing these questions and some more reading, I realized that I did not include anything to eliminate stable limit cycles. I am now starting to read \HREF{http://chaosbook.org/paper.shtml\#invariants} {Chapter 5} - {\em Cycle stability} and then I will try to implement a way of picking equilibria other than attracting limit cycles.

\item[2013-08-06 Predrag]
As we were not successful in finding an interesting strange attractor, probably best
not to be influenced by my (mostly misguided) intuition; keep experimenting, and keep
checking it with Daniel, who remembers what we had tried last time around.
As to our goals, see the ``more precise statement'' above.

My only remark for now is that \reffig{fig:PKperorb} is a \reqv\  of
Porter-Knobloch flow, meaning that the group orbit and time orbit
coincide, it is not a ``periodic orbit''. If you are \emph{on the \reqv}
you should get one of the full \statesp\ Floquet multipliers equal to 1
to machine precision. The reason is why the Floquet exponent is only
$\approx 10^{-4}$ is that you are converging to the \reqv\ forward in
time, and that is only exponential; once you have Newton codes for
\HREF{http://chaosbook.org/paper.shtml\#cycles} {finding \po s} running,
the convergence will be super-exponential.

\item[2013-08-08  Burak] Does \reffig{fig:BBpars3PKflow} look like a strange attractor? It wanders around a relative equilibrium but I'm not sure if it is a periodic orbit. I tried to slice it but my slicing code is buggy. I picked a template point on the relative equilibrium shown with red curve on \reffig{fig:BBpars3PKflow}, the result is \reffig{fig:BBpars3symmred}. \refFig{fig:BBpars3symmred} is a longer run, and it looks more like a periodic orbit when I run it longer. Is there an easy way of telling whether it is a periodic orbit or not?

\begin{figure}[ht]
  \begin{center}
  \includegraphics[width=0.9\textwidth]{BBpars3PKflow}
  \end{center}
  \caption{Projected trajectories of Porter-Knobloch flow in real space for parameters: $\mu_1 = 1.23436,\,a_1=-0.32304,\,b_1=-1.07444,\,c_1=1.00000,\,\mu_2=-0.23149,\,a_2=0.44110,\,b_2=-0.42287,\,c_2=-1.00000,\,e_2=0.67556$}. Blue curve is a trajectory starting from an unstable periodic orbit (or a relative equilibrium), red curve is a relative equilibrium.
  \label{fig:BBpars3PKflow}
\end{figure}

By the way, according to my simulations, an attracting equilibrium in the invariant polynomial basis corresponds to a stable relative equilibrium in the real space, eliminating these gives more interesting dynamics.

\begin{figure}[ht]
  \begin{center}
  \includegraphics[width=0.9\textwidth]{BBpars3symmred}
  \end{center}
  \caption{Projections of Porter-Knobloch dynamics in real space and symmetry reduced space. Parameters: $\mu_1 = 1.23436,\,a_1=-0.32304,\,b_1=-1.07444,\,c_1=1.00000,\,\mu_2=-0.23149,\,a_2=0.44110,\,b_2=-0.42287,\,c_2=-1.00000,\,e_2=0.67556$}.
  \label{fig:BBpars3symmred}
\end{figure}

\item[2013-08-08  Burak] I think this one (\reffig{fig:BBpars4PKflow}) is chaotic.

\begin{figure}[ht]
  \begin{center}
  \includegraphics[width=0.9\textwidth]{BBpars4PKflow}
  \end{center}
  \caption{Projected trajectories of Porter-Knobloch flow in real space for parameters: $\mu_1 = 1.768907,\,a_1=0.406357,\,b_1=-1.660768,\,c_1=1.00000,\,\mu_2=-0.675565,\,a_2=0.083130,\,b_2=-0.047035,\,c_2=-1.00000,\,e_2=-0.455152$}.
  \label{fig:BBpars4PKflow}
\end{figure}

\item[2013-08-10  Predrag] I think you should cheat and find chaos
    first in the invariant polynomials $\{u,v,w,q\}$ representation -
    that is already symmetry reduced. After it looks chaotic in the
    invariant coordinates, plot the same trajectory in the equivariant
 $\pS = \{x_1,x_2,y_1,y_2\}$ coordinates. That should look messy. One
that construct a \slice\ $\pSRed$ as well.

That might sound masochistic (why not slice from the start?), but we
are only learning how to slice, and it is easier when you already have a
symmetry-reduced representation. For very high\dmn\ flows we will
not have the luxury of an invariant polynomial basis...









\item[2013-08-10  Predrag to Chaos Gang] It's not over until it is over.
\end{description}
