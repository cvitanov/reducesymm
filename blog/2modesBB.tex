% siminos/blog/2modesBB.tex
% $Author$ $Date$

% Predrag 2013-08-10 Burak, git version only

\noindent
{\color{red} Do edit this git file (created 2013-08-10),
it is a new addition to the original svn version (which for now
stays untouched, the mother version is still the svn one).
}
\bigskip\bigskip

\subsection{{\twoMode} writeup}
\label{chap:2modesBBproj}

\begin{description}

\item[2013-07-25  Predrag] This is \BBedit{an example of Burak's edit},
and here is an internal footnote\BB{An internal footnote by Burak.}. They
are useful when you change something n the middle of a text, rather then
just append the newest blog post. Please remove the color from my
\PCedit{edits} as you read them and approve them, and I'll do the same
for yours.

% \item[2013-08-13  Predrag to Burak] Please define \eqv, \reqv, \po,
% and \rpo\ here, in form suitable for pasting into the putative
% \twoMode\ slicing paper.

    \PCedit{
\item[2013-08-14  Predrag]
Your definition of a \reqv\ is correct for one-parameter $\Group$, but we
will have to rethink it for $N$\dmn\ Lie group $\Group$. In that case the
time trajectory is a 1\dmn\ curve, generically tracing out the $N$\dmn\
group orbit quasiperiodically. Probably a
\HREF{http://www.streamsound.dk/book1/chaos/chaos.html\#205/z} {better
definition} (click on magenta hyperlinks!) would be that
$\vel({\ssp_{\ssp}}) \in \pS_{EQ}$, and than show that this is then true
true anywhere on the group orbit. Perhaps using
\HREF{http://www.streamsound.dk/book1/chaos/chaos.html\#204/z} {Lie
derivatives}...
    }

\end{description}

% [2013-08-13 Burak]
\noindent{\bf Invariant solutions:}                     \toCB
A point $\ssp_{EQ}$ is an \eqv\ if the velocity function evaluated at this
point is 0, namely, $\dot{\ssp}|_{\ssp=\ssp_{EQ}} = \vel({\ssp_{EQ}}) = 0$.
Orbit of
a point $\ssp$ of a $\Group$-equivariant flow is called a \reqv\ if for any
time evolved point on the orbit $\ssp(\tau) = f^{\tau}(\ssp_{REQ})$ there
exists a parameter $\phi$ such that $\ssp(\tau) = \Group(\phi) \ssp$, in other
words, a time orbit is a \reqv\ if it coincides with the group orbit.
 A
point $\ssp$ is on a \po\ if its trajectory passes through itself after some
certain time \period{}, namely $\ssp = f^{\period{}}(\ssp)$. Finally, a point $\ssp$ of a
$\Group$-equivariant flow is said to be on a \rpo\ if there exists a
point $\ssp(\period{}) = f^\period{}(\ssp)$ on the orbit of $\ssp$, such that it satisfies $\ssp(\period{}) =
\LieEl(\phi) \ssp$ for a certain $\phi$.

\noindent{\bf A truncation of PDE representations:}
The {\twoMode} system is the simplest, coarsest example of a truncation
of a Fourier representation of a PDEe. Consider, for example, the $1D$
\KSe.
%
\PC{2013-08-13 Predrag copied this from \refref{SCD07}, to set up
the conventions for the \KSe\ calculations.}
%
The time evolution of the `flame front velocity'
$u=u(x,t)$ on a periodic domain $u(x,t) = u(x+L,t)$ is given by
\beq
  u_t = F(u) = -{\textstyle\frac{1}{2}}(u^2)_x-u_{xx}-u_{xxxx}
    \,,\qquad   x \in [-L/2,L/2]
    \,.
\ee{BBks}
Here $t \geq 0$ is the time, and $x$ is the spatial coordinate.
The subscripts $x$ and $t$ denote partial derivatives with respect to
$x$ and $t$. In what follows
we shall state results of all calculations either in units of the
`dimensionless system size' $\tildeL$, or the system size $L = 2 \pi
\tildeL$. All numerical results presented in this report
are for $\tildeL=22/2\pi \simeq 3.5014$.
Spatial periodicity $u(x,t)=u(x+L,t)$
makes it convenient to work in the Fourier space,
\beq
  u(x,t)=\sum_{k=-\infty}^{+\infty} a_k (t) e^{ i k x /\tildeL }
\,,
\ee{BBeq:ksexp}
with the $1$-dimensional PDE \refeq{ks}
replaced by an infinite set of
ODEs for the complex Fourier coefficients $a_k(t)$:
\beq
\dot{a}_k= \pVeloc_k(a)
     = ( q_k^2 - q_k^4 )\, a_k
    - i \frac{q_k}{2} \sum_{m=-\infty}^{+\infty} a_m a_{k-m}
\,,
\ee{BBexpan}
where $q_k = k/\tildeL$.
Since $u(x,t)$ is real, $a_k=a_{-k}^\ast$, and we can replace the
sum by a $k > 0$ sum. In the {\twoMode} system we keep only the
$k \in \{\pm 1,pm 2\}$ terms. This is very wrong as an approximation to
the \KSe, but --in the spirit of the Lorentz equations-- OK for hoping
to learn something about the qualitative dynamics of this class of
PDEs.



\subsection{Burak's {\twoMode} blog}
\label{chap:2modesBBblog}

\begin{description}
\item[2012-05-07  Predrag to Chaos Gang] It's not over until it is over.

\item[2013-07-25  Predrag]
 - instead of computing \cLf\ one more
time, how about giving a try to \twoMode\ $\SOn{2}$-equivariant flow,
defined in
\\
\texttt{reducesymm/cgang/2modes.tex}
\\ (you can get it by a click
\HREF{../cgang/2modes.pdf}{here}, provided you had already pdflatex-ed
2modes.tex). Have a look at it, and then meet with Daniel Borrero, 3. floor
Schatz lab, who can walk you through what we had already done and learned
(all in the 2modes.pdf blog). You can play with it for -let's say- two
weeks, see whether you can find an interesting strange attractor worth
slicing. If that does not work out, we'll give up, and go to \KS\ instead,
which is much more important for our overall goals...

\item[2013-08-06 Predrag]
A more precise statement of what we are trying to achieve with this model is
in \emph{reducesymm/cgang/2modes.tex:}

``For the 4\dmn\ model at hand are using the invariant polynomials
$\{u,v,w,q\}$ dynamics only to develop intuition, but to illustrate the
general \mslices, everything has to do be done in $\pS =
\{x_1,x_2,y_1,y_2\}$ and \slice\ $\pSRed$ as well. You can see that even
for the simplest conceivable $\SOn{2}$ 4-dimensional flow one has to
think about how to construct the invariant polynomial basis, and it is
hard to imagine how anyone could do that for very high\dmn\ flows.''

If we can find a nice strange attractor with comparable amplitudes in the
two modes, and show how to slice it, that would be the simplest example of
power of slicing, as the symmetry-reduced dynamics is 3\dmn\ and something
a human can look at, turn around as 3\dmn\ Mathematica or Matlab figure.

If in addition it turns out that my favorite Sobolev norm (read
\texttt{reducesymm/blog/norms.tex}) reduces the number of local slice
hyperplanes needed to cover the attractor, I'll be doubly happy.

\begin{figure}[ht]
\begin{center}
\includegraphics[width=0.9\textwidth]{PKperorb}
\end{center}
\caption{A typical
%Predrag 2013-08-05 removed ``periodic orbit''
attractive \reqv\  of \twoMode\ flow.}
\label{fig:PKperorb}
\end{figure}

\item[2013-08-05  Burak] I read the \twoMode\ blog and
    \HREF{http://chaosbook.org/paper.shtml\#stability} {Chapter 4} -
    {\em Local stability}, confirmed most of the findings in blog,
    naively experimented on the parameters of the system in $x_i, y_i$
    basis tried to find equilibria, got nothing, then talked to Daniel,
    and re-read the blog and come up with a Monte-Carlo (kind of)
    algorithm hoping that it could get me a strange attractor. So far,
    I only got periodic orbits. \refFig{fig:PKperorb} is a typical one.

\item[2013-08-06 Predrag]
Got worried that there were no updates for 11 days - how about if agree
on a schedule, let's say git pushes
\textcolor[rgb]{1.00,0.00,0.00}{every Monday and Friday}?

                                            \toCB
The \po s (actually, \reqva) that you find are presumably \emph{limit
cycles}. In ChaosBook I define `limit cycles' as \po s which are strictly
exponentially attracting forward in time. Parenthetically, in her thesis
De Witte\rf{DWRGK13,DeWitte13} defines a `limit cycle' as an ``isolated
periodic orbit'' thusly:

``A cycle of a continuous-time dynamical system, in a neighbourhood of
which there are no other cycles, is called a limit cycle.''

That presumably has advantage of being true for both directions of time,
but I do not think we need to get that finicky...


\item[2013-08-05  Burak]
    Here is what I did:
\begin{itemize}
\item
Generate pseudo-random set of parameters ensuring $\mu_1 > -\mu_2 > 0$, $c_1 = 1$ and $c_2 = -1$ as suggested in [2012-04-29 Predrag] and [2012-08-06 Edgar Knobloch]
\item
Numerically compute roots of \refeq{PKinvEqs5a} in $u>0, v>0$ region
starting from pseudo-random pair of points $(u,v)$, to find an
equilibrium in invariant polynomial basis.
\bea
\tilde{f}(\tilde{u},\tilde{v}) &=&
  \tilde{u}\,A_1 - \tilde{v}\,A_2 = 0 %Double checked DB 04-30-2012
\,,\qquad\qquad\qquad\qquad  deg(f) = 2
\continue
\tilde{g}(\tilde{u},\tilde{v}) &=&  %Double checked DB 04-30-2012
 \left(A_1^2
 - c_1\,\tilde{v}\right)
 \left(\tilde{u}+2\,\tilde{v}\right)^2
 +e_2^2\,\tilde{v}^2 = 0
\,,\qquad  deg(g) = 4
\continue
&& \mbox{where }
A_1 = \mu_1+\tilde{a_1}\,\tilde{u}+\tilde{b_1}\,\tilde{v},
\continue
&& \,\,\,\,\qquad A_2 = \mu_2+\tilde{a_2}\,\tilde{u}+\tilde{b_2}\,\tilde{v}
\,.
\label{PKinvEqs5b}
\eea
\item
Calculate corresponding w and q and check if the syzygy holds (it does).
\item
Calculate eigenvalues of the stability matrix at this point.
\item
If the stability matrix has at least one eigenvalue with positive real part (repulsive), at least one eigenvalue with negative real part (attractive) and a complex pair of eigenvalues with non-zero imaginary part (spiral); keep the parameters and the equilibrium point.
\item
Numerically calculate the points $x_i, y_i$ corresponding to the equilibrium in invariant polynomial basis, using following relations:
\bea
  u &=& x_1^2 + x_2^2\,,
\continue
  v &=& y_1^2 + y_2^2\,,
\continue
  w &=& 2x_1^2y_1+4x_1x_2y_2 - 2x_2^2y_1\,,
\continue
  q &=& 2x_1^2y_2+4x_1x_2y_1 + 2x_2^2y_2\,.
\label{eq:PKinvxirels}
\eea
\item
Integrate the Porter - Knobloch velocity function to see time evolution in real coordinates.
\end{itemize}
So far, I got divergent solutions and periodic orbits using parameters that I found this way. My questions:
\begin{itemize}
\item
If I check the eigenvalues of the stability matrix for real coordinates,
I get 3 of the eigenvalues almost same with the ones I get for the
invariant polynomial basis, and one eigenvalue 0 (usually something less
then $10^{-4}$). This gives me the feeling of I am doing things correct,
however, I want to make more sense out of this. Is there a clear
discussion about how these eigenvalues remain unchanged under coordinate
transformations (I saw the discussion about traces in the blog, I
confirmed the result that traces of stability matrices in $u,v,w,q$ basis
and $\pS = \{x_1,x_2,y_1,y_2\}$ basis are not the same at the origin.).
\item
Is what I did reasonable at all? Is there any obvious wrong logic?
\item
Would you suggest any other restrictive criteria to pick a ``good'' set of parameters, in addition to the ones I force on eigenvalues of the stability matrix? I thought, maybe I should take parameters for which the positive and negative real-part eigenvalues are of the same order.
\item
Is an equilibrium in invariant polynomial basis ($u,v,w,q$) a \reqv\ in
the full \statesp\ $\pS = \{x_1,x_2,y_1,y_2\}$ basis? ({\bf [2013-08-13 Predrag]} correct, it is.)
If not, what sense I should make out of the fact
that the relations \refeq{eq:PKinvxirels} do not provide a unique point
$\pS = \{x_1,x_2,y_1,y_2\}$ for given ($u,v,w,q$).
({\bf [2013-08-13 Predrag]} You are on a group orbit in $\pS$, to find
out where requires the full reconstruction equations.)
\end{itemize}
After writing these questions and some more reading, I realized that I did not include anything to eliminate stable limit cycles. I am now starting to read \HREF{http://chaosbook.org/paper.shtml\#invariants} {Chapter 5} - {\em Cycle stability} and then I will try to implement a way of picking equilibria other than attracting limit cycles.

\item[2013-08-06 Predrag]
As we were not successful in finding an interesting strange attractor, probably best
not to be influenced by my (mostly misguided) intuition; keep experimenting, and keep
checking it with Daniel, who remembers what we had tried last time around.
As to our goals, see the ``more precise statement'' above.

My only remark for now is that \reffig{fig:PKperorb} is a \reqv\  of
\twoMode\ flow, meaning that the group orbit and time orbit
coincide, it is not a ``periodic orbit''. If you are \emph{on the \reqv}
you should get one of the full \statesp\ Floquet multipliers equal to 1
to machine precision. The reason is why the Floquet exponent is only
$\approx 10^{-4}$ is that you are converging to the \reqv\ forward in
time, and that is only exponential; once you have Newton codes for
\HREF{http://chaosbook.org/paper.shtml\#cycles} {finding \po s} running,
the convergence will be super-exponential.

\item[2013-08-08  Burak] Does \reffig{fig:BBpars3PKflow} look like a
strange attractor? It wanders around a \reqv\ but I'm not
sure if it is a periodic orbit. I tried to slice it but my slicing code
is buggy. I picked a template point on the \reqv\ shown
with red curve on \reffig{fig:BBpars3PKflow}, the result is
\reffig{fig:BBpars3symmred}. \refFig{fig:BBpars3symmred} is a longer run,
and it looks more like a periodic orbit when I run it longer. Is there an
easy way of telling whether it is a periodic orbit or not?

\item[2013-08-13 Predrag]
I do not know what you mean by `\po'. The whole point of this model
is that it has no \po s, only relative invariant solutions, so one
must slice to get any \po s at all?

\item[2013-08-14 Burak] By `\po', I mean an orbit that repeats itself.
\refFig{fig:BBpars3PKflow} is a simulation of \twoMode flow from t = 0 to
100. When I simulated the flow with the same parameters from t=0 to 400 I
got \reffig{fig:BBpars3symmred} and if I run it longer I get a very
similar looking picture. That's what I was guessing that to be a \po.

\item[2013-08-14 Predrag]
There is no `\po' in these figures, in sense of your definitions of
\refsect{chap:2modesBBproj}. I think you found an attractive \rpo\
(attractive torus is the full \statesp) which after symmetry reduction
becomes an attractive \po, AKA a limit cycle. If you plot it in
$\{u,v,w,q\}$, it will be a limit cycle. (Your `Symmetry reduced' frames
in \reffig{fig:BBpars3symmred} look very wrong.)

One way to proceed would be to
change parameters in such a way that this Hopf cycle goes unstable. If it
does it through 2nd Hopf bifurcation, that is a start of a generic
transition to chaos via mode locking and then period doublings.

\item[2013-08-14 Burak]
If you look closely to the upper-left plot in \reffig{fig:BBpars3PKflow},
there is a piece of the blue curve which looks like the red curve
squeezed and turned upside-down ({\bf Predrag} I see it). The starting
point of that simulation corresponds to a point on that piece of the blue
curve. The points on this small piece maps to an \eqv\ in the invariant
polynomial basis, so that's why I was calling that an unstable \reqv\
({\bf Predrag} agreed). After realizing that the actual dynamics (blue
curve) looks like it is making a `wurst' around a group orbit of the
system, I decided to check whether if it is around some \reqv\ of the
system or not. For that reason, I started from other points corresponding
to the relative equilibria of the system, integrated and plotted on top
of the blue curve; one of these curves is the red curve here. Red and
blue curves are time evolutions starting from two different relative
equilibria of the system. While red curve is a stable \reqv, blue curve
starts from an unstable \reqv\ and then starts to rotate and shift around
the stable \reqv\ shown red.

%\PCedit{
%\item[2013-08-13 Predrag] If my \reffig{fig:BBpars3PKflow} caption is now
%correct, comment out this comment :)
%    }


\begin{figure}%[ht]
  \begin{center}
  \includegraphics[width=0.9\textwidth]{BBpars3PKflow}
  \end{center}
  \caption{$3D$ projections of trajectories of \twoMode\ flow in full
  \statesp\ for parameters: $\mu_1 = 1.23436,\, a_1=-0.32304,\,
  b_1=-1.07444,\, c_1=1.00000,\, \mu_2=-0.23149,\, a_2=0.44110,\,
  b_2=-0.42287,\, c_2=-1.00000,\, e_2=0.67556$. Blue curve is a
  trajectory starting close to an unstable \reqv, the blue curve in the
  middle of the top left frame, which converges to a stable \rpo, better
  seen in \reffig{fig:BBpars3symmred}. This \rpo\ originates from a Hopf
  bifurcation of a \reqv, here the coexisting attractive \reqv\ plotted
  as red curve.
  }
  \label{fig:BBpars3PKflow}
\end{figure}

\item[2013-08-08  Burak]
According to my simulations, an attracting \eqv\ in the invariant
polynomial basis corresponds to a stable \reqv\ in the full \statesp.
({\bf [2013-08-13 Predrag]} correct.)
Eliminating these parameter values
gives more interesting dynamics.

\item[2013-08-13 Predrag] \refFig{fig:BBpars3PKflow} does not look like a
strange attractor. You really want to plot these things in $\{u,v,w,q\}$
representation first, and if something looks chaotic, look at it in a
Poincar\'e section; there it is much easier to see whether there is a
stretch \&\ fold unstable manifold with fractal structure.

\item[2013-08-14 Burak] I am going to try to compute Poincar\'e
sections and return maps tomorrow. I have to go back a little bit
since my existing Poincare\'e section codes for ChaosBook exercises
are extremely sloppy.

        \PCedit{
\item[2013-08-20 Predrag] You probably want to implement the
\HREF{http://www.streamsound.dk/book1/chaos/chaos.html\#82} {H\'enon trick},
ChaosBook section {\em 3.2 Computing a Poincar\'e section}.
        }

\item[2013-08-19 Burak] Took longer than I thought. I have the
\twoMode flow in invariant polynomial basis
(\reffig{fig:BBpars4invspacePKflow}), Poincar\'e section and radial
return map (\reffig{fig:BBpars4psectnretmap}). Convergence to a
periodic orbit is clearly seen on these plots, points with $r=6.8$
mapped on themselves.


\SFIG{BBpars4invspacePKflow}{}{
  A projection of \twoMode\ dynamics in the invariant polynomial
  basis $\{u,v,w,q\}$. Parameters: $\mu_1 =
  1.23436,\,a_1=-0.32304,\,b_1=-1.07444,\,c_1=1.00000,\,
  \mu_2=-0.23149,\,a_2=0.44110,\,b_2=-0.42287,\,
  c_2=-1.00000,\,e_2=0.67556\,.$
  }{fig:BBpars4invspacePKflow}

\begin{figure}%[ht]
  \begin{center}
  \includegraphics[width=0.9\textwidth]{BBpars4psectnretmap}
  \end{center}
  \caption{Left: Poincar\'e section for the \twoMode\ flow shown in
  \reffig{fig:BBpars4invspacePKflow}. Section hyperplane passes
  through $\hat{x} = (u=10, v=6, w=0, q=0)$ and the plane normal is
  $\hat{n} = (-6,10,0,0)$ ($\hat{x}$ rotated 90 degrees about $w$
  axis). Right: Poincar\'e return map of radial distance from $w$
  axis.}
  \label{fig:BBpars4psectnretmap}
\end{figure}

        \PCedit{
\item[2013-08-20 Predrag]
Looks right - you are converging to a limit cycle with a negative
least contracting Floquet multiplier. For limit cycles Poincar\'e
sections are boring, they will be useful only one you are looking at
strange attractors.
(I have, experimentally, commented your commit on gitHub, but that
looks useless. Did gitHub alert you to the comment?)
Have you computed Floquet multipliers of this limit cycle? They should
give you the convergence rate exactly.

Now see whether you can pick a convenient parameter that moves one or
a pair of Floquet multipliers through 1 and on to chaos.

As explained in
\HREF{http://www.streamsound.dk/book1/chaos/chaos.html\#78}
{ChaosBook}, 1\dmn\ Poincar\'e return maps, such as of
the radial distance from $w$ axis in \reffig{fig:BBpars4psectnretmap}
are misleading.
        }
\BBedit{
\item[2013-08-21]
I got notification for gitHub comment. I tried to calculate the Floquet multipliers but failed really badly at calculating the Jacobian. I thought that I could estimate the Jacobian (in the sense that it is defined in the ChaosBook, not like in Strogatz's book, I think his definition of Jacobian is the Stability Matrix of the ChaosBook) by multiplying the Jacobians for small time steps in a time ordered manner, but result was horrible. I think the numerical errors build up in an intolerable way. Can you refer me some paper/thesis/book that describes the correct way of computing this? I'm assuming that someone did this before and I'd rather not to re-discover it.
}

\begin{figure}%[ht]
  \begin{center}
  \includegraphics[width=0.9\textwidth]{BBpars3symmred}
  \end{center}
  \caption{Projections of \twoMode\ dynamics in full \statesp\ and symmetry reduced space. Parameters: $\mu_1 = 1.23436,\,a_1=-0.32304,\,b_1=-1.07444,\,c_1=1.00000,\,\mu_2=-0.23149,\,a_2=0.44110,\,b_2=-0.42287,\,c_2=-1.00000,\,e_2=0.67556$}.
  \label{fig:BBpars3symmred}
\end{figure}

\item[2013-08-08  Burak] I think this one (\reffig{fig:BBpars4PKflow}) is chaotic.

\begin{figure}%[ht]
  \begin{center}
  \includegraphics[width=0.9\textwidth]{BBpars4PKflow}
  \end{center}
  \caption{$3D$ projections of trajectories of \twoMode\ flow in full \statesp\ for parameters: $\mu_1 = 1.768907,\,a_1=0.406357,\,b_1=-1.660768,\,c_1=1.00000,\,\mu_2=-0.675565,\,a_2=0.083130,\,b_2=-0.047035,\,c_2=-1.00000,\,e_2=-0.455152$}.
  \label{fig:BBpars4PKflow}
\end{figure}

\item[2013-08-10  Predrag] I think you should cheat and find chaos
    first in the invariant polynomials $\pSRed = \{u,v,w,q\}$ representation -
    that is already symmetry reduced. After it looks chaotic in the
    invariant coordinates, plot the same trajectory in the equivariant
 $\pS = \{x_1,x_2,y_1,y_2\}$ coordinates. That should look messy. After
that construct a \slice\ $\pSRed$.
    \PCedit{
Examples are \reffig{fig:dangelmayrChaos} and \reffig{fig:dangelmayrChaos2}.
    }

That might sound masochistic (why not slice from the start?), but we
are only learning how to slice, and it is easier when you already have a
symmetry-reduced representation. For very high\dmn\ flows we will
not have the luxury of an invariant polynomial basis...

\item[2013-08-12  Burak] I got the parameters that I used in \reffig{fig:BBpars4PKflow} by generating random parameters, discarding if there are attractive equilibria or the time evolution is convergent or divergent in invariant polynomial basis. Unfortunately, this one was a periodic orbit in the invariant polynomial space.

I added another criteria in my parameter generating code on Saturday to eliminate periodic orbits and discovered a bug in it later today. I spent most of today on varying parameters one by one and trying to see if those variations breaks that periodicity. I didn't get anything interesting yet.

I will run another 'new parameter set finder' tonight with the working periodic orbit elimination.

\item[2013-08-13 Predrag] One way to diagnose chaos  is to pick a stable
solution (like \reffig{fig:PKperorb}) and follow it as it undergoes a
Hopf bifurcation into a stable limit cycle. Then one keeps changing
parameters until this \po\ goes unstable and begets chaos, through period
doublings and beyond. To do this, you need to be able to compute Floquet
multipliers of your invariant solutions. For example, does the unstable
\rpo\ in \reffig{fig:BBpars3PKflow} have complex pair of multipliers
(underwent a Hopf bifurcation that turned it unstable) or two real
multipliers (perhaps on the way to period doubling sequence?)

\item[2013-08-13 Predrag] I have asked a Geophysical Fluid Dynamics
(Woods Hole) Fellow Yuuki Yasuda (Earth and Planetary Science, U. Tokyo,
yyuuki@eps.s.u-tokyo.ac.jp) to learn how to use
\HREF{http://sourceforge.net/projects/auto-07p/} {AUTO}, for the same
reason - to follow bifurcations of initially invariant solutions, see how
they go into chaotic behavior. It is well written code which I think is
only good in small dimensions, so we have not used it for our high\dmn\
hydrodynamics calculations. Yuuki will report to me today how it is
working; I'll report whether it might be useful to you.

\item[2013-08-13 Predrag]
I keep getting confused about whether you are plotting a \reqv\ or a \po,
and as we will need this anyhow, please define \eqv, \reqv, \po, and
\rpo\ in \refsect{chap:2modesBBproj}, just to be sure we are on the same
page. (I keep using macros for their names, because depending on the
publication and audience, a `\reqv' might be called a `rotating wave',
etc.)

\item[2013-08-13 Predrag]
A problem with {\twoMode} system is too many parameters (seven! - see
\refeq{PKinvEqs5b}). How about reducing the number of parameters by
demanding that our {\twoMode} system is the $k \in \{\pm 1,\pm 2\}$
truncation of \KS? There is only one parameter left (the system size
$L$), so that is probably too radical -- maybe it will
yield no interesting dynamics at all,
but let something physical like this guide you in reducing the number of
parameters.

The physical setting is that a dissipative turbulent system has a finite
number of Fourier modes nonlinearly coupled and of comparable amplitudes,
while high modes are very strongly suppressed by dissipative terms like
$q_k^4$ in \refeq{BBexpan}. (Xiong Ding can explain the `physical
dimension' to you). So you will be interested in chaotic solutions for
which the two modes $u = {z}_1 \overline{z}_1,\, v = {z}_2
\overline{z}_2$ in \refeq{PKinvEqs1} are of comparable magnitude.

One good exercise is to go through Ruslan's {\em 15.7.1 2009-08-26
Epicycles: 2-Fourier modes} in the \HREF{../blog/blog.pdf} {main blog}.
In this model he looks at a trivial dynamics of two uncoupled modes. I do
not agree with Ruslan pessimistic conclusion - search for `epicycles'
within the blog for more optimistic angles.

\item[2013-08-21 Burak] I varied the parameters I used in \reffig{fig:BBpars4invspacePKflow} and \reffig{fig:BBpars4psectnretmap} and the fixed point in the Poincar\'e section (attracting limit cycle in the flow) got bifurcated to a two cycle. After playing a little bit more, I got \reffig{fig:BBinvspacePKflow130821}. In \reffig{fig:BBinvspacePKflow130821}, every trajectory goes near the origin, make a small turn there and start go away and make another turn at a different location far away from the origin and go back. This behavior is better seen in the Poincar\'e return map on the right hand side of \reffig{fig:BBpsectandretmap130821}. On the return map, different points with larger radial distances are mapped to the neighborhood of the radius 5, and points with radii close to 5 are mapped to different points (without following a regular pattern.). This, I think, satisfies the `extreme dependence on initial state' condition of chaos.

\SFIG{BBinvspacePKflow130821}{}{
  A projection of \twoMode\ dynamics in the invariant polynomial
  basis $\{u,v,w,q\}$. Parameters: $\mu_1 =
  1.23436,\,a_1=-0.32304,\,b_1=-1.07444,\,c_1=1.00000,\,
  \mu_2=-0.23149,\,a_2=0.44110,\,b_2=-0.42287,\,
  c_2=-1.00000,\,e_2=1.8\,.$
  }{fig:BBinvspacePKflow130821}

\begin{figure}%[ht]
  \begin{center}
  \includegraphics[width=0.9\textwidth]{BBpsectandretmap130821}
  \end{center}
  \caption{Left: Poincar\'e section for the \twoMode\ flow shown in
  \reffig{fig:BBinvspacePKflow130821}. Section hyperplane passes
  through $\hat{x} = (u=10, v=4, w=0, q=0)$ and the plane normal is
  $\hat{n} = (-4,10,0,0)$ ($\hat{x}$ rotated 90 degrees about $w$
  axis). Right: Poincar\'e return map of radial distance from $w$
  axis.}
  \label{fig:BBpsectandretmap130821}
\end{figure}

 After liking what I saw in the invariant polynomial space, I went ahead and integrated the system in the full state space and try to \slice\ it using method of moving frames. One projection of the real space flow is in \reffig{fig:BBfullspaceflow130821} and the \reffig{fig:BBsymmredflow130821} is how state space look like after the reduction of symmetry. I tried a few template points (I tried to find fixed points and use them, didn't look good at all) and got the prettiest picture with the template $\slicep = (1,1,0,0)$.

\SFIG{BBfullspaceflow130821}{}{
  A projection of \twoMode\ dynamics in the full state space $\{x_1,x_2,y_1,y_2\}$. Parameters: $\mu_1 =
  1.23436,\,a_1=-0.32304,\,b_1=-1.07444,\,c_1=1.00000,\,
  \mu_2=-0.23149,\,a_2=0.44110,\,b_2=-0.42287,\,
  c_2=-1.00000,\,e_2=1.8\,.$
  }{fig:BBfullspaceflow130821}

 \SFIG{BBsymmredflow130821}{}{
  A projection of \twoMode\ dynamics of \reffig{fig:BBfullspaceflow130821} after symmetry reduction using the method of moving frames (post processing). Slice template: $\slicep = (1,1,0,0)$.
  }{fig:BBsymmredflow130821}

\item[2013-08-22 Predrag] This looks promising. Comparing
\reffig{fig:BBinvspacePKflow130821}
and
\reffig{fig:BBsymmredflow130821}
you can see one of the reasons why I like symmetry reduction by
slices to invariant polynomials, even in a few dimensions, where
syzygies are few: invariant polynomials scrunch everything into the
origin. If you and Daniel settle on this one as an example (I would
prefer an example where you need at least two slice hyper-planes -
presumably when the two modes are of comparable strength; and when
some parameters are eliminated by requiring that the \twoMode\ system
is close to the \KS\ 2-mode truncation), than you should do
Poincar\'e sections more carefully, using points ant not '*',
constructing the unstable manifolds by continuing the Floquet vectors
of the initial (now unstable) Hopf cycle, and constructing the return
map from them, as discussed in
\HREF{http://www.streamsound.dk/book1/chaos/chaos.html\#260/z}
{Chapter 12} {\em Stretch, fold, prune}.

\item[2013-08-21 Daniel] Ok... so while you guys are off doing all the thinking, I brute forced a set of parameters that appear to give chaotic dynamics. Basically, I used the same Monte Carlo-ish technique that I used to find ``good'' templates for slicing \CLf where I let the computer vary parameters by a little bit and see if they give better results. If so they are used a new starting point and the calculation is repeated; if not they are discarded. To measure ``success'' I calculated finite time Lyapunov exponents for the \twoMode\ system in the invariant polynomial basis and required that the leading one be positive (the more positive the better), the second one be zero (guaranteed by autonomous dynamics), and the third one be negative (the more negative the better). The algorithm prints them out in order, so the fourth one is automatically more negative than the third. The code I used is in \texttt{/cgang/Daniel/Matlab/ChaoticExponentsCalculation} and should have enough comments for intelligibility. The algorithm used is based on Wolf et al. \cite{WolfSwift85}. I let the computer run this for a day or so and got the following set of parameters: 

\begin{eqnarray*}
\mu_1&=-2.8023, \mu_2=0.6128, a_1=-0.9146, a_2=-2.6636,... \\
b_1&=-0.6141, b_2=-0.0144, c_1=-4.1122, c_2=1.8854,\, \mathrm{ and }\, e_2=2.1769.
\label{eq:PKChaoticParams}
\end{eqnarray*}

Calculating the Lyapunov exponents for the \twoMode\ system gives,

\begin{equation}
\lambda_1=0.09, \lambda_2=0.00, \lambda_3=-9.14,\, \mathrm{ and }\, \lambda_4=-17.77.
\label{eq:PKLyapunovExp}
\end{equation}

Coincidentally, the first three are almost exactly the Lyapunov exponents for the R\"ossler system ($\lambda_1=0.09, \lambda_2=0.00,\, \mathrm{and}\, \lambda_3=-9.77$). Figure \ref{fig:PKChaoticInv} shows what it ends up looking like in the invariant polynomial basis. If you run it longer it just fills out the attractor. Figure \ref{fig:PKChaoticInv} shows it in Cartesian coordinate. If you run it longer it just fills out the attractor. 

\item[2013-08-23 Burak] I integrated the system with Daniel's parameters and my simulation converged to a fixed point $(0,425.56,0,0)$ in the invariant space (see \reffig{fig:BBPKflowDanielPars} (a)). I found all eigenvalues of the stability matrix at this point to be negative with zero imaginary parts. As expected, this resulted as a relative equilibrium in the full state space (see \reffig{fig:BBPKflowDanielPars} (b)).

I also sliced this solution with the same template that I used for my parameters and got \reffig{fig:BBsymmredPKDanielPars0t12} for the time that the solution looked chaotic (t from 0 to 12), it then makes a jump (see \reffig{fig:BBsymmredPKDanielPars14t20} (a) for evolution from t=14 to t=17) as the solution in invariant polynomial basis does, and then converges to the fixed point (see \reffig{fig:BBsymmredPKDanielPars14t20} (b) for evolution from t=15.5 to t=20) by going through a spiral. I'm blogging this because I found it interesting to see a ``spiral in'' type of convergence on the symmetry reduced manifold while all eigenvalues of the stability matrix of the fixed point in the invariant space has zero imaginary parts.

\begin{figure}[h]
	\centering
		\includegraphics{PKChaotic.pdf}
	\caption{\twoMode\ system in invariant polynomial basis using parameter values from Eq. \ref{eq:PKChaoticParams}}.
	\label{fig:PKChaoticInv}
\end{figure}

\begin{figure}[h]
	\centering
		\includegraphics{PKChaotic2.pdf}
	\caption{\twoMode\ system in Cartesian basis ($x_1,x_2,y_1,y_2$) using parameter values from Eq. \ref{eq:PKChaoticParams}}.
	\label{fig:PKChaotic}
\end{figure}

\begin{figure}[H]
\centering
 (a) \includegraphics[width=0.35\textwidth]{BBinvspacePKflowDanielPars}
 (b) \includegraphics[width=0.35\textwidth]{BBreleqDanielPars}

\caption{(a)A projection of \twoMode\ dynamics in the invariant polynomial
  basis $\{u,v,w,q\}$. Parameters are listed in \refeq{eq:PKChaoticParams}.(b)Relative equilibrium in the real space corresponding to the fixed point in (a).}
\label{fig:BBPKflowDanielPars}
\end{figure}

\SFIG{BBsymmredPKDanielPars0t12}{}{
  A projection of \twoMode\ dynamics of \reffig{fig:BBPKflowDanielPars} after symmetry reduction using the method of moving frames (post processing). Slice template: $\slicep = (1,1,0,0)$.
}{fig:BBsymmredPKDanielPars0t12}

\begin{figure}[H]
\centering
 (a) \includegraphics[width=0.35\textwidth]{BBsymmredPKDanielPars14t17}
 (b) \includegraphics[width=0.35\textwidth]{BBsymmredPKDanielPars15p5t20}

\caption{(a)A projection of \twoMode\ dynamics in the invariant polynomial
  basis $\{u,v,w,q\}$. Parameters are listed in \refeq{eq:PKChaoticParams}.(b)Relative equilibrium in the real space corresponding to the fixed point in (a).}
\label{fig:BBsymmredPKDanielPars14t20}
\end{figure}

\item[2013-08-23 Daniel] Just ran a couple of hundred simulations using the parameters from \refeq{eq:PKChaoticParams} with random initial conditions taken from a Gaussian ball centered at $(u,v,w,q)\, =\,(0,0,0,0)$ with radius of about 8. Each simulation was run for 1000 time units. I found that the trajectories ended up at Burak's fixed point about 70\% of the time, so it's basin of attraction is definitely significant. Then again, 30\% of trajectories fall into the chaotic attractor and stay there for long times, so it's basin of attraction is not tiny either. It also doesn't seem like the chaotic dynamics are transient. I can integrate out to 20000 time units and still be in the attractor if I start in its basin of attraction (e.g., $(u,v,w,q)\, =\,(1,1,1,1)$ . Maybe we can tweek the parameters to make Burak's equilibrium unstable or hyperbolic? Looking at the symbolic expressions for the eigenvalues, I don't see a particularly obvious way to do this. The expressions are a mess! 


\item[2013-08-10  Predrag to Chaos Gang] It's not over until it is over.
\end{description}
