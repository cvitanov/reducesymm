% siminos/blog/Lie.tex
% $Author$ $Date$

\chapter{Lie police}
\label{sect:LiePolice}

\renewcommand{\LieEl}{\ensuremath{g}}  % Predrag Lie group element
\renewcommand{\ssp}{x}

\begin{description}
\item[2011-03-15
\HREF{http://amath.colorado.edu/faculty/jdm}
{Master of Simplecticity}]
%{James Meiss}]
Never did I think I'd read the sentence ``It's a mess'' in a
\emph{Physica D} paper! But cool, nevertheless.

\item[2011-03-25 PC] keep reading:
``looking for periodic orbits in systems with only continuous symmetries
is a fool's errand.'' If you read to the end of Sect. 8 it gets worse:
``a serious nuisance.'' Another first for \emph{Physica D}.

\item[2011-03-17 JM]
In Eq.~(4) of \refref{SiCvi10} you say that every element of a Lie group
that is connected can be written as an exponential. This is false, I
believe. Some Lie groups are connected and some are not. $\SOn{n}$ is
connected, as is $Sp(n)$. But, just like for manifolds (indeed a Lie
group is just a manifold with a group structure), some groups can't be
covered by a single coordinate patch, or in this case by the exponential.
This is also related to the fact that the series for the  Log is not
globally convergent\rf{Hall03}.

\item[2011-03-17 PC]
Of course not {\em every} Lie group element needs to be continuously
connected to identity - that why the finely tuned statement in our humble
paper.
I need a lawyer.

\item[2011-03-17 JM] or a mathematician!

\item[2011-03-17 PC]
no, no, in your case I need a lawyer. We carefully
write above Eq.~(4):

``An element of a compact Lie group
continuously connected to identity can be written as''

We \emph{do not say}
``\emph{\color{red} every} element of a compact Lie group
\emph{\color{red} is} continuously connected to identity...''

% How can it be wrong? Goes back to Peter-Weyl, etc.
We write here
``continuously connected to identity'' so we do not need to point out
that $\On{n}$ has also elements of form (discrete operation $\times$
exponential rep of $\SOn{n}$ group element, where discrete group $Z_2$ is
$\{e,$inversion $\}$ (inversion$)^2=1$, and avoid talking about double
covers, \etc. The paper (as does the discussion of
\HREF{http://chaosbook.org/chapters/continuous.pdf}
{Chapter 10 - Relativity for cyclists}) tries its utmost to minimize the
\HREF{http://chaosbook.org/chapters/appendHist.pdf}{Gruppenpest jargon}
damage, which is a total turnoff to our intended audience of working
plumbers and electricians. If I may be so bold as to cite my foreman
plumber Fabian Waleffe, faculty member of a reputable mathematics
department in a union-busting state, upon my attempt to get him to read
\HREF{http://chaosbook.org/chapters/discrete.pdf}
{Chapter 9 - World in a mirror}, chicken feed in comparison to the
continuous symmetry reduction nightmare:
	\PC{remember to include Barkley's \On{n} as an example of
		semi-direct product.}

``How many Tylenols should I take with this?... (never took group theory,
still need to be convinced that there is any use to this beyond
mind-numbing formalizations.)''

But it is obviously too subtle, will rewrite in the ChaosBook.org.

Now, more seriously - we do not need the explicit exponential form, we
only need to say that for a continuous group (compact or not) we can
linearize the flow in a neighborhood of any group element, and that
the tangent space is spanned by the Lie algebra. How to say it without
too much jargon?

\item[2011-03-17 JM]
Yes, every Lie group has a tangent space (since it is a manifold), and
the tangent space always has a Lie algebra structure. Technically, I
believe that one says that the Lie algebra is the tangent space "at the
identity". So $\SOn{n}$ and $\On{n}$ have the same Lie algebra, the
antisymmetric matrices. The Lie algebra of $Sp(n)$ is the set of
``Hamiltonian matrices.'' More fancily, people talk about the Lie algebra
as the set of left-invariant vector fields... but that always confuses
me.

\item[2011-03-17 PC]
``set of left-invariant vector fields?'' Probably not needed
for compact Lie groups (subgroups of the unitary group). There left
and right vectors are the same thing...

\item[2011-03-17 JM]
For example the symplectic group has elements that are not exponentials.
An example is the Jordan block $B=[[-I, I],[0,I]]$, which I learned from
Alex Dragt. A great reference on Lie groups is Hall\rf{Hall03}.

\item[2011-03-17 PC]
Thanks for introducing me to Dragt; I have now read and like his
\refref{Dragt05} (for CNSers - uploaded to zotero). I prefer the much
deeper \refref{PCgr} to Hall\rf{Hall03}, which I always found rather
pedestrian. \HREF{http://birdtracks.eu}{The Much Deeper} you can have for
a click (by contrast to The Much Cooler and very wonderful
\refref{Meisso7} for which I doled out cool cash). But maybe I should
have a second look... For PDE applications of symmetries
Hoyle\rf{hoyll06} is pretty gentle.


\item[2011-03-17 JM]
$Sp(n) = {\LieEl :  \LieEl^T {\bf \omega} \LieEl = {\bf \omega}}$
where $\LieEl$ is $[2n\!\times\!2n]$.

\item[2011-03-25 PC]
See \refrem{rem:symplectic} concerning notation for $Sp(n)$.

\item[2011-03-17 JM]
So matrix $B \in Sp(n)$, but it is not the exponential of a
``Hamiltonian matrix'' (a matrix of the form ${\bf \omega}S$, where $S$
is symmetric).

\item[2011-03-25 PC]
This is exercise 3.7.12 in in Dragt\rf{Dragt11}
It has not worked for me as yet:
By explicit calculation, $B^T {\bf \omega} B \neq {\bf \omega}$.
I am probably missing something, but it is a worthwhile exercise
to do... (Evangelos? Chao?)

\item[2011-03-25 PC]
Thanks for pointing out that this is called `Hamiltonian matrices.'
Curious: who named this `Hamiltonian matrices'? So far I've tracked it to 1971
(see \refrem{rem:symplectic}).
As a physicist, I hate
this misuse of a term well established in Quantum Mechanics since the
time of Heisenberg (See \refrem{rem:symplectic}). The recovered Irish
alcoholic did quaternions, but I believe that it was the Germans who
formalized the classification of semi-simple Lie algebras towards the end
of 19th century, with Cartan driving in the final stake.

Can you perhaps have a glance at
\HREF{http://chaosbook.org/chapters/continuous.pdf} {Chapter 7 -
Hamiltonian dynamics}, see whether your hair raises? Spurred by you I'm
starting to incorporate more symplectic material, the first draft is
below, \refsect{sect:toCB}.

I care about splitting hairs because I'm trying to collect all possible
names for every single thing in ChaosBook.org commentary, together with
attributions.

\item[2011-03-17 JM]
But it seems you have the group $\SOn{n}$ or perhaps
$\Un{n}$ in mind. Because in the next paragraph
you say that the Lie algebra of your group is the set
of anti-hermitian matrices, and then you say it can be
brought into the anti-symmetric form. I thought this was
true only for the orthogonal group.

\item[2011-03-25 PC]
You are correct, from then on we use only  $\SOn{n}$, as Evangelos
explains here:

\item[2011-03-17 ES]
As we only have applications in mind where $(x_1,x_2,...,x_d)$ are real
coordinates and since any compact Lie group acting on $\reals^n$ can be
identified with a subgroup of $\On{n}$, we only consider $\On{n}$. Then, as
explained above, we only need elements connected to the identity and
therefore restrict attention to $\SOn{n}$.

\item[2011-03-17 JM]
I wasn't aware of this (subgroup) business. I'll have to read about that.

I take it you submitted your most recent paper to the Morrison
festschrift volume? I wrote a paper for that too.

\item[2011-03-17 PC]
Would you mind having a look at it,
\HREF{http://www.cns.gatech.edu/~predrag/papers/preprints.html\#FrCv11}
{\emph{Reduction of continuous}} \emph{symmetries of chaotic flows by the
method of slices}? It is on the same theme, and I'm afraid that until we
reduce KS, pipe flow and plane Couette flow symmetries there will be more
papers like this...

\item[2011-03-17 JM]
Sure, that was how I started, but then I decided I better read
your \emph{Physica D} paper first!

\item[2011-04-23 PC]
Michael Loss says we should look at Fefferman{Feffer83} ``Uncertainty
principle,'' which he thinks attempts to define symplectic distances.
Fefferman's conjectures were proved by Hofer. I have looked at the paper,
and see no connection to what we do; it's about relations between
phase-space volumes and spectra of the Schr\"odinger operator.

\end{description}


\section{Text to return to ChaosBook.org}
\label{sect:toCB}

\subsection{Hamiltonian flows}
\label{sect:HamFlows}


An important class of flows are Hamiltonian flows,
given by a
% time-indepen\-dent
Hamiltonian $H(q,p)$ together with the Hamilton's
equations of motion
\index{flow!Hamiltonian}
%\toRem{rem:statespace}
\beq
\dot { q_i} = {\partial H \over \partial p_i}
    \,, \quad\quad  % \continue %\label{(1)} \\
\dot {p_i} = - {\partial H \over \partial q_i}
\,,
\ee{Ham_eQ}
with the $d= 2\DOF$ \emph{phase space} coordinates $\pSpace$ split into
the configuration space coordinates and the conjugate momenta of a
Hamiltonian system with $\DOF$ degrees of freedom (dof):
\index{degree of freedom}
%    \PublicPrivate{
%    }{% switch \PublicPrivate{
%\toSect{s-HamEqs}
%    }% end \PublicPrivate{
\beq
\pSpace=({\bf q},{\bf p})
\,,\qquad
{\bf q} = (q_1,q_2,\dots,q_\DOF)
\,,\qquad
{\bf p} = (p_1,p_2,\dots,p_\DOF)
\,.
\ee{2.7a}
The equations of motion for a time-independent, $\DOF$-dof Hamiltonian
\refeq{Ham_eQ} can be written compactly as
\index{symplectic!invariance}
\index{invariance!symplectic}
\beq
\dot{x}_i={\omega}_{ij} H_j(\pSpace)
    \,, \quad
    H_j(\pSpace) =
    \frac{\partial~}{\partial\pSpace_j} H(\pSpace)
        \,,
\ee{symplec}
\PC{use O de Almeida book argument here?}
where $\pSpace = ({\bf q},{\bf p}) \in \pS$ is a phase space point,
$H_{k} = \partial_k H$ is the column vector of partial derivatives of
$H$,
\beq
{\omega} = \MatrixII{0}{\bf I}{\bf -I}{~0}
	\,,
\ee{symplNormConv}
and ${\bf I}$ is the [$\DOF\!\times\!\DOF$] unit matrix.
Throughout ChaosBook we reserve the term `phase space' to Hamiltonian
flows, and `\statesp' to general flows with no symplectic symmetry.


\subsection{Symplectic group}
\label{sect:SymplctGroup}

A matrix transformation $\LieEl$ is called \emph{symplectic} if it
preserves the \emph{symplectic bilinear form} $\braket{\sspRed}{\pSpace}
= \sspRed^T {\bf \omega} \pSpace$,
\beq
\LieEl^T {\bf \omega} \LieEl = {\bf \omega}
	\,,
\ee{symlectNorm}
where
$\LieEl^T$ denotes the transpose of $\LieEl$,  and ${\omega}$ is a non-singular
[$2\DOF\!\times\!2\DOF$] antisymmetric matrix that satisfies
\index{symplectic!form}
%\ifOUP
%\notes{
%The term `symplectic' --Greek for twining or plaiting
%together-- was introduced
%into mathematics by Hermann Weyl.
%`Canonical' lineage is church-doctrinal:
%Greek `kanon,' referring to a reed
%used for measurement, came to mean
%in Latin a rule or a standard.
%      }
%\else
%\fi
\beq
{\bf \omega}^T=-{\bf \omega}\,,
\qquad {\bf \omega}^2 =- \matId
\,.
\ee{sympForm}
While these are defining requirements for any {symplectic bilinear form},
${\bf \omega}$ is often conventionally taken to be of form
\refeq{symplNormConv}.


\example{Symplectic form for $\DOF=2$:}{
                    \label{exmp:SymplTwoDof}
												\toCB
For two degrees of freedom the phase space is 4\dmn,
\(
\pSpace=(q_1,q_2,p_1,p_2)
\,,
\)
and the symplectic 2-form is
\beq
{\omega} = \left(
\begin{array}{cccc}
0  &  0 & 1 & 0\\
0  &  0 & 0 & 1\\
-1  &  0 & 0 & 0\\
0  &  -1 & 0 & 0
\end{array} \right)
	\,,
\ee{symlectNorm2D}
The symplectic bilinear form  $\braket{\pSpace^{(1)}}{\pSpace^{(2)}}$
is the sum over the areas of the
parallelepipeds spanned pairwise by the two vectors,
\beq
\braket{\pSpace^{(1)}}{\pSpace^{(2)}} =
(\pSpace^{(1)})^T {\bf \omega} \pSpace^{(2)} =
(q_1^{(1)}p_1^{(2)} -q_1^{(2)}p_1^{(1)})
	+
(q_2^{(1)}p_2^{(2)} -q_2^{(2)}p_2^{(1)})
\,.
\ee{twoFormD2}
It is these oriented areas (not the Euclidean distance between the
two vectors, $\pSpace^{(2)}-\pSpace^{(1)}$) that are preserved by the
symplectic transformations.
    } %end \example{Symplectic form for 2 \dof

If $\LieEl$ is symplectic, so is its inverse $\LieEl^{-1}$, and if
$\LieEl_1$ and $\LieEl_2$ are symplectic, so is their product $\LieEl_2
\LieEl_1$. Symplectic matrices form a group called the \emph{symplectic
group} $Sp(d)$.



\PC{Motivation:
We only need to say that for a continuous group (compact or not) we can
linearize the flow , and that
the tangent space is spanned by the Lie algebra, without
too much jargon aout duble covers and what-not.
	}
%
%\item[2011-03-17 JM]
% + Banks "Modern Quantum Field Theory" 2008 textbook, Appendix E
Use of the symplectic group necessitates a few remarks about
Lie groups in general, a topic that we study in more depth in
Chapter continuous.tex.
A \emph{Lie group} is a group whose elements $\LieEl(\gSpace)$ depend smoothly
on a finite number $N$ of parameters $\gSpace_i$. In a neighborhood of a
given group element the linearization
\[
\LieEl(\gSpace+\delta\gSpace) \simeq \LieEl(\gSpace)
+  \sum_{i=a}^N  \delta\gSpace_a
	\frac{\partial \LieEl(\gSpace)}{\partial_i \gSpace}
\,,
\]
spans the \emph{tangent space} at $\LieEl(\gSpace)$. With an application
of the inverse of a group element $\LieEl^{-1}$, this neighborhood can be
mapped into a neighborhood $\LieEl(\delta\gSpace)$ of the identity
$\LieEl(0)=1$, so the local structure of Lie groups can be understood by
studying the tangent space near the identity element, generated by
\[
\LieEl(\delta\gSpace) \simeq 1
+ \sum_{i=a}^N \delta\gSpace_a \left.
	\LieEl(\gSpace)^{-1}
	\frac{\partial \LieEl(\gSpace)}{\partial_i \gSpace}
		\right|_{\gSpace=0}
\,.
\]
We find it convenient to write this in a specific basis
(repeated indices are summed throughout this
chapter, and the dot product refers to a sum over
Lie algebra generators):
\index{Lie!algebra}\index{generator!Lie algebra}
\beq
\LieEl(\delta\gSpace) \simeq 1 + \delta \gSpace \cdot \Lg
% \LieEl{}_i{}^j \simeq \delta_i^j +  \delta \gSpace_a \, (\Lg_a)_i^j
    \,,\quad
\delta\gSpace \in \reals^N
    \,,\quad
|\delta \gSpace| \ll 1
    \, ,
\ee{InftsmLieTransf}
where $\{\Lg_1,\Lg_2\cdots,\Lg_N\}$, the {\em generators} of
infinitesimal transformations, are a set of $N$ linearly independent
$[d\!\times\!d]$ matrices which span the tangent space and act linearly
on the $d$-dim\-ens\-ion\-al phase space $\pS$. As we shall see later,
they endow the tangent space with a Lie algebra structure. Globally
different Lie groups may have the same local structure: for example, near
the identity $\SOn{n}$ and $\On{n}$ have the same Lie algebra $o(n)$.

The infinitesimal statement of symplectic invariance follows by
substituting \refeq{InftsmLieTransf} into \refeq{symlectNorm} and keeping
the linear terms in $|\delta \gSpace|$
\index{symplectic!transformation}
\index{canonical transformation}
\beq
 \Lg_a^T {\bf \omega} + {\bf \omega} \Lg_a = 0
\,.
\ee{symFunny_L}
This is the defining property for infinitesimal generators of {\em
symplectic} transformations, or the \emph{symplectic
Lie algebra} $sp(d)$. Matrices that satisfy \refeq{symFunny_L} are
sometimes called \emph{Hamiltonian matrices}. A linear combination  of
Hamiltonian matrices is a Hamiltonian matrix, so Hamiltonian matrices
form a linear vector space. By the antisymmetry of $\omega$,
\index{Hamiltonian!matrix}
\beq
({\bf \omega} \Lg_a)^T = {\bf \omega} \Lg_a
\,.
\ee{HamMatr2}
is a symmetric matrix. Its number of independent elements gives the
dimension of the symplectic group $Sp(d)$ is
\beq
N= {d(d+1)}/{2} = \DOF (2 \DOF+1)
\,.
\ee{symplDim}
The lowest-dimensional symplectic group $Sp(2)$ is isomorphic to $SU(2)$
and $\SOn{3}$.

It is easily checked that the exponential of a Hamiltonian matrix
\beq
\LieEl = e^{\gSpace \cdot {\Lg}}
\ee{SymplExcp}
is a symplectic matrix; Lie group elements are obtained from the Lie
algebra elements by exponentiation.


\example{How far are two points from each other in a symplectic vector space?}{
                    \label{exmp:SymplDistance}
												\toCB
Consider the sum of oriented areas (a sophisticate
would say the `action')
\[
S(\pSpace,\slicep) =
\braket{\pSpace}{\slicep} =
		\pSpace^T {\bf \omega} \, \slicep
\]
spanned by a pair of vectors joined by the symplectic 2-form.

Think of one of the patterns
(represented by a point {\slicep} in the phase space  \pS) as a
`template'
%\rf{rowley_reconstruction_2000,%
%rowley_reduction_2003,%
%ahuja_template-based_2007}
or a `reference state' and act with elements of the symplectic group
$Sp(d)$ on it, $\slicep \to \LieEl(\gSpace)\,\slicep$, until
its `distance' to the second pattern (a point $\ssp$ in the phase space),
\beq
S(\ssp,\LieEl(\gSpace)\,\slicep)
    = S(\sspRed,\slicep)
\label{minDistanceSp}
\eeq
is extremized.
Here $\sspRed$ is the point on the group orbit of $\ssp$
(the set of all points that $\ssp$ is mapped to under the group
actions),
\beq
\ssp=\LieEl(\gSpace)\,\sspRed
	\,,\qquad
\LieEl \in \Group
\,.
\ee{sspOrbit}

Unlike the Euclidean length, the symplectic bilinear form is not positive
definite (see \refeq{twoFormD2}), and this `distance' is emphatically not
a Euclidean distance, but it is -by the definition- invariant symplectic
transformations, $S(\LieEl\pSpace,\LieEl\slicep)=S(\pSpace,\slicep)$. If
\slicep\ is on the group orbit of $\ssp$, the form can be made to exactly
vanish by its antisymmetry, $\braket{\slicep}{\slicep} = 0$, so nearby
group orbits do have a small minimal `distance' $|S(\sspRed,\slicep)|$
which satisfies the extremum conditions
\beq
\frac{\partial ~~}{\partial \gSpace_a} \braket{\ssp}{\LieEl(\gSpace)\,\slicep}
   =
\braket{\sspRed}{\sliceTan{a}}
   = 0
    \,,\qquad
\LieEl(\gSpace)\,\sspRed = \ssp
    \,,\qquad
\sliceTan{a} = \Lg_a \slicep
\,.
\ee{SpSlice}
The closest group orbit points thus lie in a $(d\!-\!N)$\dmn\ hyperplane
$\pSRed = \pS/\Group$, the set of vectors $\sspRed \in  \pSRed$
orthogonal to the {\template} tangent space in the
\emph{symmetric} $({\bf \omega} \Lg_a)$ norm
\beq
\sspRed_i({\bf \omega} \Lg_a)_{ij}\slicep_{j} = 0
\,.
\ee{hyperplSp}
In what follows we shall refer to this hyperplane as a \emph{slice,} and
to  \refeq{SpSlice} as the \emph{slice conditions}. The slice
reduces the symplectic symmetry, with the reduced \statesp\ of
(negative !) dimension
$d-N  =  d - {d(d+1)}/{2} = - d(d-1)/2$. Something is amiss here...
    } %end \example{Symplectic form for 2 \dof


\subsection{Canonical transformations}
\label{sect:CanonTransf}

The
evolution of $\jMps^t$ \refeq{hOdes} is again determined by
the {\stabmat} $\Mvar$, \refeq{Bew_Miaw}:
%%RA e
\index{Hessian matrix}
\beq
{d\over dt}\jMps^t(\pSpace) = \Mvar(\pSpace)
\jMps^t(\pSpace)
\,, \qquad
\Mvar_{ij}(\pSpace)={\omega}_{ik}\,S_{kj}(\pSpace)
\,,
\ee{Bew_Mon}
where the symmetric matrix of second derivatives
$S_{kn} = \partial_k \partial_n H$
%index%$S_{kn}=\partial_k \partial_n H$
is called the {\em Hessian matrix}.
From
% \refeq{Bew_Mon} and
the symmetry of $S_{kn}$
%index% $H_{kn}$
it follows that
\index{symplectic!transformation}
\index{canonical transformation}
\beq
 \Mvar^T {\bf \omega} + {\bf \omega} \Mvar = 0
\,.
\ee{Funny_L}
%\label{Sp_L}
This is the defining property \refeq{symFunny_L} for infinitesimal
generators of {\em symplectic} (or canonical) transformations.
%transformations which leave
%the symplectic form ${\bf \omega}$ invariant.

Symplectic matrices are by definition linear transformations that leave
the (antisymmetric) quadratic form $x_i \omega_{ij} y_j$ invariant. This
implies that any symplectic matrix satisfies
\beq
\LieEl^T\omega \LieEl \,=\, \omega
\,,
\label{sympQ}
\eeq
and -- when $\LieEl$ is close to the identity $\LieEl=\matId + \delta t
\Mvar$ -- it follows that that $\Mvar$ must satisfy \refeq{Funny_L}.
\PC{isn't the next paragraph a repeat of this one?}

Consider now a smooth nonlinear change of variables of form $y_i=h_i(x)$,
and define a new function $K(x)=H(h(x))$. Under which conditions does
$K$ generate a Hamiltonian flow? In what follows we will use the notation
$\tilde{\partial}_j=\partial/\partial y_j$,
$s_{ij} =
% (\partial h)_{ji}=
\frac{\partial h_i}{\partial x_j}$
.
By employing the chain rule we
have that
%\GT{It would be nicer, conceptually, to define trafo(?) $x \to y= h(x)$
%and transform $H(x)$  into $K(y) = H(g(y)$ with $g = h^{-1}$. PC: as you
%wish - just make sure that it is consistent with $h$ as defined in conjug.tex}
\beq
\omega_{ij} \partial_j K \,=\, \omega_{ij}
\tilde{\partial}_l H
s_{lj}
% \frac{\partial h_l}{\partial x_j}
\eeq
(Here, as elsewhere in this book, a repeated index implies summation.)
%\GT{Summation over repeated indicees defined?}
By virtue of \refeq{Ham_eQ} $\tilde{\partial}_l H=-\omega_{lm}\dot{y}_m$, so
that, again by employing the chain rule, we obtain
\beq
\omega_{ij} \partial_j K \,=\, -\omega_{ij} s_{jl}
% \frac{\partial h_l}{\partial x_j}
\omega_{lm}
s_{mn}
% \frac{\partial h_m}{\partial x_n}
\dot{x}_n
\eeq
The right hand side simplifies to $\dot{x}_i$ (yielding Hamiltonian
structure) only if
\beq
-\omega_{ij} s_{lj}
% \frac{\partial h_l}{\partial x_j}
\omega_{lm}
s_{mn}
%\frac{\partial h_m}{\partial x_n}
\, =\, \delta_{in}
\eeq
or, in compact notation,
\beq
-\omega (\partial h)^T \omega (\partial h) \,=\, \matId
\eeq
which is equivalent to the requirement \refeq{symlectNorm} that $\partial
h$ is symplectic. h is then called a {\em canonical transformation}.
\index{canonical transformation}
\index{symplectic!map}
We care about canonical transformations for two reasons.
%\toExam{exmp:UnderstFlows}
First (and this is a dark art), if the canonical transformation $h$ is
very cleverly chosen, the flow in new coordinates might be considerably
simpler than the original flow. Second,  Hamiltonian flows themselves are
a prime example of canonical transformations.

  \Remarks
\remark{Symplectic.}{\label{rem:symplectic}
The term symplectic --Greek for twining or plaiting
together-- was introduced
into mathematics by Hermann Weyl.
`Canonical' lineage is church-doctrinal:
Greek `kanon,' referring to a reed
used for measurement, came to mean
in Latin a rule or a standard.

													\toCB
%\item[2011-03-25 PC]
Exposition of \refsect{sect:SymplctGroup} follows Dragt\rf{Dragt05}.
There are two conventions in literature for what the integer argument of
$Sp(\cdots)$ stands for: either $Sp(D)$ or $Sp(d)$ (used, for example, in
\refrefs{Dragt05,PCgr}), where $\DOF$ = \dof, and $d=2\DOF$. As explained
in Chapter 13 of \refref{PCgr}, symplectic groups are the `negative
dimensional,' $d \to -d$ sisters of the orthogonal groups, so only the
second notation makes sense in the grander scheme of things.
Mathematicians can even make sense of the $d= $odd-dimensional case,
see ``Odd symplectic groups'' by Proctor\rf{Proctor88,GeZe84}, by dropping
the requirement that $\omega$ is non-degenerate, and defining
a symplectic group $Sp(\pS,\omega)$ acting on a vector space $\pS$
as a subgroup of $Gl(\pS)$ which preserves a skew-symmetric bilinear form
$\omega$ of \emph{maximal possible rank.} The odd symplectic groups
$Sp(2\DOF+1)$ are not semisimple.
If
you care about group theory for its own sake (our dynamical systems
symmetry reduction techniques are still too primitive to be applicable to
Quantum Field Theory), Chapter 14 is fun, too.

Any finite-dimensional symplectic vector space has a \emph{Darboux basis}
such that $\omega$ takes form \refeq{symlectNorm}.

Referring to the $Sp(d)$ Lie algebra elements as  `Hamiltonian matrices'
as one sometimes does\rf{Dragt05,wikiHamMat} unfortunately conflicts with
what is meant by a `Hamiltonian matrix' in quantum mechanics: the quantum
Hamiltonian sandwiched between vectors taken from any complete set of
quantum states. Not sure where this name comes from; Draft cites
\refrefs{Fulton91,Georgi99}, and  Chapter 17. of his own book in
progress%\rf{Dragt11}.
Fulton and Harris\rf{Fulton91} seem to use it.
Certainly Van Loan\rf{PaiL81} uses in 1981. Taussky in 1972.
Might go all the way back to
Sylvester?

Dragt\rf{Dragt05} convention for phasespace variables is as in
\refeq{2.7a}. He calls the dynamical trajectory $\xInit \to
\pSpace(\xInit,t)$ the `transfer map.' That is really regrettable.

In the mathematics literature the Lie operator $:f:$ is sometimes
referred to as $ad(f)$ where $ad$ is shorthand for adjoint. Dragt uses the
$:f:$ notation instead of $ad(f)$ because it facilitates the writing of
complicated expressions.

I call the matrix of second derivatives of the Hamiltonian, $S_{kn}=
\partial_k \partial_n H$, the `Hessian matrix.'
${\omega}_{ik}\,S_{kj}(\pSpace)$ in \refeq{Bew_Mon} is the (instantaneous)
{\stabmat} of the flow - it is a Hamiltonian matrix. Its integral along
the trajectory, the linearization of the flow ${\jMps}$ that I (as well
as Dragt\rf{Dragt05}) currently call the `\JacobianM' is symplectic.
Coordinate transformations whose gradient is symplectic are called
`canonical transformations'. Hamiltonian flow $\flow{t}{\pSpace}$ is a
canonical transformation, with the linearization $\jMps_{kn} = \partial_n
\flow{t}{\pSpace}_k$  a symplectic transformation.

        } %end \remark{Symplectic.}{

\remark{Sources.}{\label{rem:sources}
													\toCB
Dragt and Habib\rf{DraHab08,Dragt05} have a nice concise discussion of
symplectic Lie operators and their relation to Poisson brackets:
incorporate this into the Liouville operator discussion in ChaosBook.org.

Curiosity: in \refref{Dragt11} Dragt says ``Sometimes Feigenbaum
diagrams are called bifurcation diagrams. However. strictly speaking,
bifurcation diagrams should also display the unstable fixed points, and
Feigenbaum diagrams generally do not.
[news to Predrag]
The use of the term bifurcation in
the context of dynamics is due to Poincar\'e.''
And he does not credit the other guy for the fixed-point equation.
The book is more extreme than even ChaosBook.org: 1872 pages. But
considerably more repetitive.

        } %end \remark{Sources.}

\remark{Killing fields.}{\label{rem:Killing}
													\toCB
The symmetry tangent vector fields discussed above are a special
case of Killing vector fields of Riemannian geometry and special relativity.
From \HREF{http://en.wikipedia.org/wiki/Killing_vector}{wikipedia}
(\HREF{http://en.wikipedia.org/wiki/Spacetime_symmetries}{this wikipedia}
might also be useful:
A Killing vector field is a set of infinitesimal generators of isometries on a Riemannian manifold that preserve. the metric. Flows generated by Killing fields are continuous isometries of the manifold. The flow generates a symmetry, in the sense that moving each point on an object the same distance in the direction of the Killing vector field will not distort distances on the object.

A vector field $X$ is a Killing field if the Lie derivative with respect to
 $X$ of the metric $g$ vanishes:
\beq
    \mathcal{L}_{X} g = 0 \,.
\ee{LieKilling}
\PC{some clip \& paste math gibberish}
Killing vector fields can also be defined on any (possibly nonmetric)
manifold $\pS$ if we take any Lie group $\Group$ acting on it instead of
the group of isometries. In this broader sense, a Killing vector field is
the pushforward of a left invariant vector field on $\Group$ by the group
action.
If the group action is effective, then the space of the Killing vector
fields is isomorphic to the Lie algebra $\mathfrak{g}$ of $\Group$.

If the equations of motion can be cast in
Lagrangian form, with the Lagrangian exhibiting variational
symmetries\rf{Bluman07,BlumanAnco02}, Noether theorem associates
a conserved quantity with each Killing vector.
        } %end \remark{Killing fields.}

  \RemarksEnd

\renewcommand{\LieEl}{\ensuremath{\gamma}}  % also a Siminos Lie group element
\renewcommand{\ssp}{a}
