% siminos/blog/atlas.tex
% $Author$ $Date$

\chapter{Atlas}
\label{chap:atlas}

\begin{description}
\item[2011-11-30 Predrag]
started a new blog, for Predrag's putative letter

\texttt{siminos/atlas/main.tex}

\item[2011-12-06 Roman] However,  ECS can be used to project
trajectories onto a low-dimensional visualization where coordinates $a^k_{i}(t)$
are defined by the inner product
\beq
\label{coordinatesRG}
a^k_{i}(t)= \frac{1}{V} \int_V {\bf e}^k_i(t)^\dagger\cdot({\bf u}(t)-{\bf u}^k(t)) dV,
\eeq
%\pagebreak
where $V$ is the flow domain volume, ${\bf u}^k(t)$ is the ECS closest to
the system state at time $t$, and ${\bf e}^k_i(t)$ are basis functions
representing a set of unstable and weakly stable modes characterizing the
particular ECS.

\item[2011-12-06 Predrag]
Mhm - why are we introducing new notation for \statesp\
coordinates in \refeq{coordinatesRG}? I cannot imagine a situation in which one
would like to make the bases  ${\bf e}^k_i$ time dependent, ${\bf e}^k_i(t)$,
and we mostly use ECS themselves to construct projections, as in
\reffig{f:ssptransient}, not their stability eigenvectors. If you are really
thinking of using ECS's ${\bf u}^k(t)$ as templates whose linearized local
charts compose a global atlas for the flow, they would not be time dependent,
but fixed by a set of Poincar\'e sections to \statesp\ points ${\bf u}^k$, and
the associated set of unstable and weakly stable eigenvectors (what are
"modes"?) ${\bf e}_i^k$ would also be confined to the Poincar\'e section, and
not time dependent. But I think this is way too sophisticated for referees to
wrap their heads around....

I propose you drop time dependence from ${\bf e}^k_i(t)$ and ${\bf
u}^k(t)$ and pass over all that in silence for now. Or perhaps keep the
formula as is, as neither the authors themselves have never sorted out
how the \po\ eigenvectors are to be used in practice.

\begin{figure}
\includegraphics[width=\textwidth]{P97portrait5}
\caption{
{\bf A state-space portrait of turbulent plane Couette flow.}
A turbulent trajectory ${\bf u}_{\text{turb}}$ (solid and dotted black
lines) shadowing the P97 periodic orbit (bold magenta line) and the
unstable manifolds (blue, red, and green lines) of symmetry-related
equilibria (solid blue, red, and green dots; the black dot at the origin
is the laminar flow state). Velocity fields corresponding to the open
magenta dots on P97 are shown in  reffig~{f:ssptransient2}. Dynamic
connections are shown as bold red and blue lines connecting different
equilibria (filled dots).
}
\label{f:ssptransient}
\end{figure}

\item[2011-12-06 John] Predrag's comment on \refeq{coordinatesRG} is on
target: it makes sense with the verbal description attached to it, but
the time-dependence of the basis set $e_i^k(t)$ and the nearest ECS
$u^k(t)$ is pretty confusing. The bigger problem is that the equation as
is reflects what we want to do, but doesn't the projection in
\reffig{f:ssptransient}, which uses a fixed basis based on four
symmetry-related ECS to produce a global portrait.

\item[2012-02-21 PC]  I like the way this chaos class is working, and I
think we can put together some of the best class contributions into a
pedagogical article about sections and slices. And we have a chance to
publish this in a high quality issue of one of the better journals in the
field, Chaos:

\item[2012-02-21 Phil Holmes to PC]
We write in connection with the recent IUTAM Symposium on 50 Years of
Chaos: Applied and Theoretical, held in Kyoto, which you attended. In
addition to the Conference Proceedings to be published by Elsevier as
part of their regular IUTAM series, the American Institute of Physics
(AIP) has agreed to devote a special issue of the AIP Journal
\emph{Chaos} to papers representing a selection of the topics presented.
This is planned to appear as \emph{Vol 22 (4), December 2012}. So far 11
invitees have agreed to contribute papers; only one (David Ruelle) has
declined.


Based on your interesting presentation in Kyoto, we would like to invite
you to submit a paper for consideration for this special issue of Chaos.
Please let us know as soon as possible if you would like to do so, at the
latest, before Monday Jan 16th, 2012. Papers may either review a topic,
preferably including new results, or present entirely new, unpublished
results; papers should also reflect the topics and themes presented at
the symposium. All contributions will be refereed in the usual manner, as
for unsolicited submissions to AIP journals. At present we expect this to
require submission of the paper by \textbf{March 30th, 2012}, with
resubmission of revised papers during June 2012.


\textbf{[NOTE from Phil Holmes]}
Since some of the work you described has appeared in previous papers, in
particular that on channel flows with John Gibson, I would be especially
interested in the new results on symmetry groups hinted at in your
presentation.


\item[2012-02-21 PC to Phil Holmes]
Thank you for your kind invitation to contribute to the special issue of
the AIP Journal Chaos in connection with our "50 Years of Chaos". I am
sorry I am responding past your deadline, but I was not sure we would
have results interesting enough to report in this issue, results that
have not been written up for other publications. Now I am feeling more
confident that we can offer a novel pedagogical review of the symmetry
reduction by the method of slices (that I had described by its
application to one example, the pipe flows, in Kyoto), so I would like to
join this very fine group of contributors to the planned December 2012
Chaos.


\item[2012-02-18 Predrag] clippings

This procedure has been devised by Poincar\'e in the context of celestial
mechanics, in the aim of reducing the analysis of long-term planetary
motion and its dynamic stability [Poincar\'e, Les m\'ethodes nouvelles de
la m\'ecanique c\'eleste 1892].

\begin{figure}
  \includegraphics[width=0.6\textwidth]{AmLeAg06Im1}\\
  \caption{From \refref{AmLeAg06}:
R\"ossler attractor with fixed points and their manifolds. The linear
stable manifold at the lower fixed point is 1-dimensional and the
unstable one is associated with an unstable focus. The stable manifold at
upper fixed point is a stable focus and the unstable manifold is
1-dimensional}
\label{fig:AmLeAg06Im1}
\end{figure}

From \refref{AmLeAg06}:
``A linear system is $\dot{x} =Ax$ and an affine system is $\dot{x}
=Ax+b$, where $A$ is a constant matrix and $b$ is a constant vector.''

                                            \toCB
Draw (un)stable eigenvectors as in \reffig{fig:AmLeAg06Im1}.
``
In the R\"ossler system, the switching is induced by the
nonlinearity which acts when the trajectory is sufficiently far
from lower fixed point, that is, beyond the threshold $x-c$ in
the third equation. The nonlinearity
acts when the trajectory is sufficiently close to upper fixed
points where its converging spiral induces the folding by
sending the trajectory back to the neighborhood of lower fixed
point along its unstable manifold. Thus, the lower fixed point
is mainly responsible for the stretching and upper fixed point for
the folding.
''

\end{description}
