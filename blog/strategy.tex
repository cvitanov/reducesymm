% siminos/blog/strategy.tex
% $Author$ $Date$

\chapter{Strategy, to write up}

\begin{bartlett}{
Someone who makes the same mistake twice is not a wise man.
}
\bauthor{
An ancient Greek saying
    }
\end{bartlett}




\section{How to read me}

Throughout:  {\textdollar} on the margin
{\steady}
indicates that the text has been transferred to
articles and thesis siminos/*/,  or to ChaosBook.org desymmetrization
chapters, such as
\HREF{http://ChaosBook.org/continuous.pdf}
{continuous.pdf}.
%
For those whose Freud needs brushing up:
`Desymmetrization and its discontents' is a pun
on \HREF{http://en.wikipedia.org/wiki/Civilization_and_Its_Discontents}
{Civilization and its discontents}.

Here is a novice's guide to desymmetrization bloggery:
\begin{itemize}
  \item
How to read the running blog: go first to the latest blog post, end
of \refchap{c-DailyBlog}.
  \item
If you are reading an article of common interest (which does not fit into
one of the specialized topics), enter your notes into \refchap{c:lit}.
  \item
\refChap{c-freeze} \emph{Deep freeze} is the blog for the putative
\texttt{siminos/ksReduced/}
article, spearheaded by Evangelos
  \item
Comments to ChaosBook.org go into \refchap{chap:ChaosBook} blog.
  \item
Periodic orbit theory comments belong to \refchap{chap:UPO} {\em
Periodic orbit theory}.
  \item
If Hamiltonian dynamics is your obsession, that's in
\refchap{sect:LiePolice} {\em Lie police}.
  \item
Slicing all things laser should be confined to
\refchap{chap:lasers}{Laser physics: The lingo}
  \item
Guys counting the number of degrees of freedom that capture the physics
of a `turbulent' PDE have gotten a divorce. All Kazzmania now resides in
\texttt{siminos/lyapunov/}.
  \item
Ditto for geophysicists. They reside in
\texttt{siminos/baroclinic/BrCv12.tex}.
  \item
Guys writing the ultimate guide to slicing for the woman on the street,
\texttt{siminos/atlas/}, blog in \refchap{chap:atlas}{\em Atlas}.
  \item
Plumbers who ponder how to slice experimental data blog in
\refchap{c-exp} {\em Symmetry reduction of experimental data}
  \item
Enter your ponderings on all things norm into \refsect{sect:norms}
\emph{Norms, distances}, though some of that is also in \refchap{c:lit}
(for experimental data) and \refchap{sect:LiePolice} (for symplectic
distances).
  \item
Cardiologists (mere electricians, really) have gotten a divorce, too. That
blog's gone to \texttt{DOGS/saldana/excite.tex}.
  \item
All things `{geometric phase}' are in \refchap{c-geometric} {\em
Geometric phase}.

\end{itemize}






\section{Email list}

create email list of our cohort:  Bristol, etc. participants

\subsection{Advertise arXiv rpo paper}

crosslink the paper with nonlin dynamics

email individually arXiv paper link to colleagues who might comment
    on the paper

\section{Classification, keywords}

						\noindent
elsevier (first number is Elsevier only? or? The 2nd is \textbf{PACS})
10.020: 02.20.-a Group theory	\\
10.050: 02.50.-r Probability theory, stochastic processes, and statistics 	\\
10.150: 05.40.Ca Noise	\\
10.180: 05.45.-a Nonlinear dynamics and chaos	\\
10.190: 05.45.Ac Low-dimensional chaos	\\
10.210: 05.45.Gg Control of chaos, applications of chaos	\\
10.220: 05.45.Jn High-dimensional chaos	\\
10.230: 05.45.Mt Quantum chaos - semiclassical methods	\\
10.240: 05.45.Pq Numerical simulations of chaotic systems	\\
10.305: 05.10.Gg Stochastic analysis methods	\\
10.390: 05.70.Ln Nonequilibrium and irreversible thermodynamics	\\
20.030: 05.10.Gg Stochastic analysis methods	\\
20.080: 05.45.-a Nonlinear dynamics and nonlinear dynamical systems	\\
20.090: 05.45.Mt Semiclassical chaos (quantum chaos)	\\
60.090: 46.70.-p Application of continuum mechanics to structures	\\
42.65.Sf Dynamics of nonlinear optical systems; optical instabilities,
         optical chaos and complexity, and optical spatio-temporal dynamics \\
47.10.Fg 	Dynamical systems methods (in Fluid Mechanics)	\\
47.27.ed 	Dynamical systems approaches (turbulent flows)	\\
70.050: 47.27.-i Turbulent flows, convection, and heat transfer	\\
70.110: 47.52.+j Chaos (in fluid dynamics)	\\
70.130: 47.54.+r Pattern selection; pattern formation	\\
70.150: 47.60.+i Flows in ducts, channels, nozzles, and conduits	\\
70.160: 47.62.+q Flow control	\\


						\noindent
\textbf{keywords}	\\
symmetry reduction,	\\
equivariant dynamics,	\\
relative equilibria,	\\
relative periodic orbits,	\\
return maps,	\\
slices,	\\
moving frames,	\\
Hilbert polynomial bases,	\\
invariant polynomials,	\\
Lie groups	\\


\section{Papers to write}

\subsection{\emph{Physica D} ``Continuous symmetry...''}

\begin{description}

\item[2010-05-25 Vaggelis]
Will we follow editor's suggestion and resubmit with minor modifications only?
Deadline for receipt of the {\bf final} manuscripts is July 1st 2010.
Will we go for an arXiv version?

\item[2010-06-07 Predrag]
OK, now we have resubmitted, with minor edits. I think one should always submit
any article that is worth publishing also to arXiv;
it is open to anyone, rich or poor, and it is
more likely to reach the intended audience than only a publication through
any single journal.

\end{description}

\subsection{Reduced trace formulas?}

\begin{description}
 \item[2010-06-17 Vaggelis]
Since I have all rpo's up to level 7 for CLE I think I should try
to apply ``Continuous symmetry reduced trace formulas'' so that I get an incentive
to understand the paper. After all this group theory, it should be easier now.
 \item[2010-06-18 Predrag]
Would be nice if you did - both to understand the group theory better, and
also because I am not sure I have not missed some important detail about
invariant subspaces when I wrote the paper. Would be great to recycle KS
next, if CLE works.
\end{description}


\subsection{\emph{SIAM J. Appl. Dyn. Syst.}}

\begin{description}

\item[2010-06-07 Predrag] Next, the
``Continuous symmetry reduction of Kuramoto-Sivashinsky ...'' paper:
40,000 \rpo s and noplace to go?
Can include movies and more graphics

\end{description}

\subsection{PRL on recycling energy}

\section{Write next NSF proposal }

\section{Spruce up personal websites}

ES homepage with publication list, pdf files of talks, movies

Real ES homepage with links to the publication(s)

\section{Zoteromania}

\begin{description}

\item[2008-07-18 Predrag] about webtools for generating BibTeX:
www.zotero.org
        will pick up most books from Amazon, etc; but
        better to find a book first on
\HREF{http://www.worldcat.org}{www.worldcat.org}
          or
\HREF{http://scholar.google.com}{scholar.google.com}, then zotero it
          in a collection, and export in BibTeX format

\item[2009-12-22 Evangelos]
setting up a cns group at zotero.org

\item[2011-08-16 Predrag] moved the instructions to siminos/bibtex/zotero.txt

\end{description}
