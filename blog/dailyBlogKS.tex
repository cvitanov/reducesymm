% reducesymm/blog/dailyBlogKS.tex
% Predrag  created              Sep 2 2013
% continues siminos/blog/dailyBlog.tex as of that date

\chapter{Kamal's daily blog}
\label{c-dailyBlogKS}

\begin{description}
\item[2013-07-11  Predrag to Kamal] Created this for you to blog
in. Burak will show you the ropes. You need to register on gitHub, let me
know your user ID there.

\item[2013-10-15  Kamal] I am working on symmetry reduction in spiral waves and have studied a few research papers by Barkley, Biktashev. I am trying to find out how to start working on the problem. In the 1994 paper by Barkley and Kevrekidis {\em A dynamical systems approach to spiral waves dynamics} and another paper by Barkley (Phys. Rev. Lett. 72, 1994), they have given an equivalent system of ODE in place of the actual Reaction-Diffusion PDEs. These ODEs also have time shift and $E(2)$ symmetry. Should I apply symmetry reduction on these ODEs or begin with the PDEs?

\item[2013-10-15 Predrag] Our main conceptual confusion (with none of us having
done any actual work so far) is how rotations and translations interact - \ie, if you change the origin and then translate is different from first translating and then rotating. I think you can figure out how that really works in the Barkley ODE system first - that would be a great help. My intuition (usually wrong) is that we want to
quotient the rotational $\SOn{2}$ first.

\item[2013-10-12  Kamal] I will be away during the month of January for my wedding.
I will be working from home and updating through GitHub.
Please let me know if you have anything you want me to take care of.
Further, it will be great for me if you could send me the material
for the QFT final project. As you might have experienced, speed is
an issue with my working on new stuff so I want to start as early
as possible to get something useful out of it given the time constraints.
Also the TA duties are killing me. I am working on it and want to be productive fast.


\item[2013-10-19 Kamal to Predrag] I tried to reduce \SOn{2} symmetry
from the dynamics of the system of five nonlinear ODEs proposed by
Barkley and Kevrekidis\rf{BarKev94}.


The group orbits corresponding to \SOn{2} are non-compact in this
case and form counter-clockwise helices, due to which there is an
infinite number of intersections between a particular group orbit and
the slice. As we discussed, I found the intersection of group orbit
of each point on the original trajectory with the slice which is
closest to the template point. Now the problem is I am getting
discontinuous curves as group orbits.



I am guessing this because I don't have a clever way to find
continuous orbits or because I haven't removed the translational
symmetry yet. Which brings me to the question, how do I remove
translational symmetry? From what I understand since the system is
translationally invariant, I can identify all the points on the plane
to the origin. But this is too trivial or/and too extreme. This
however still doesn't give me a \rpo.
Kindly suggest something.
%\textit{Excuse the format of figures. Still learning LATEX.}

\item[2013-10-20 Predrag] At the moment I have nothing insightful to add,
but Barkley and Kevrekidis\rf{BarKev94} seems to be well written, worth
understanding. My claim is that after symmetry reduction, the dynamics is
(5-1-2) = 2\dmn, so the solutions are either \eqva, circles, or infinite
lines, reconstruction equations yield all regular motions in
their Fig.~4, that's why there is no chaos in the 5\dmn\ ODEs.

\item[2013-10-20 Predrag] Our method of slices is based on the idea that
we look for the point on group orbit of the state $\ssp$ that is closest
to our template. I do not know whether this is a helpful remark, but
I think that is the same as the method of finding a
\HREF{https://inst.eecs.berkeley.edu/~ee127a/book/login/l_svd_lineqs.html}
{pseudo-inverse}.

The Euclidean group $SE(2)$ is the group of rigid motions of the plane.
Our symmetry is probably $E(2)$, with parity also a symmetry? Lawrence Tao
ponders
\HREF{http://terrytao.wordpress.com/2011/03/05/lines-in-the-euclidean-group-se2/}
{this group}, do not know whether we want to understand what he says...
There is lots of stuff on its Lie algebra, for example
\HREF{http://physics.lakeheadu.ca/facNstaff/deGuise/pdffiles/e3.pdf}
{this paper}, or
\HREF{http://link.springer.com/article/10.1007\%2FBF00047669\#page-1}
{this one}.

\item[2013-10-20 Predrag] You might find Saldana's blog of interest, he
has read much of the literature you study in detail; it is
in the svn repository \emph{DOGS}. Your svn ID is \emph{ksharma4},
password \emph{slice!} - Burak or Chris can help you check the repository
out.

\item[2013-10-26 Predrag] You might find
Wulff\rf{Wulff02}, {\em Spiral waves and {Euclidean} symmetries}, of
interest, she reviews Barkley and other papers and puts them into a
larger context that also includes PDEs, I believe. Find it
HREF{http://ChaosBook.org/library/Wulff02.pdf} {here}. Use
\texttt{student} \texttt{Lautrup} to download.

If you search \texttt{reducesymm/bibtex/siminos.bib} for Wulff,
you'll find many such articles.

\item[2013-10-26 Kamal] Thank you for the references. I will soon
update the summary of our discussion on friday.

\item[2013-10-26 Kamal] \textbf{Summary of discussion:} I have practiced the
\emph{method of moving frames} with \cLf\ which had rotational symmetry about
the $z$-axis. But was confused about multiple symmetries, particularly the questions
of commutation and whether I need to take care of infinitesimal and finite transformations.
In the case of multiple
symmetries, for example $SE(2)$ with
rotational and translational symmetry on a 2-dimensional plane, the group manifold
is generated by the union of the actions of all group operations. Any point
lying on a solution of the system can be moved along the group orbits, hopefully
in any order and magnitude, onto a well chosen subspace which is called a slice.
This slice is frequently defined as a hyperplane normal to the tangents defined by
all the symmetry operations at the template point. As a result, a $d$-dimensional
dynamical system with $N$ continuous symmetries could be reduced to dynamics on a submanifold
of $d-N$ dimensions.

\item[2013-10-26 Kamal] \textbf{Symmetry reduction for Barkley ODEs}
\\
Barkley and Kevrekidis, in their 1994 paper\rf{BarKev94}, studied
spiral waves in excitable media whose dynamics are given by the
Fitzhugh-Nagumo reaction-diffusion PDEs. They propose an ODE model
\refeq{BarKev94-6} of five ordinary differential equations based on
numerical studies of the PDEs by Barkley\rf{Barkley92,Barkley94}:
\bea
\dot{x} &=& s \cos{\phi} \,,\qquad
\dot{y} = s \sin{\phi}\,,\qquad
\dot{\phi} =  w \, h(s^2,w^2)     \continue
\dot{s}  &=& s \, f(s^2,w^2) \,,\qquad
\dot{w}= w \, g(s^2,w^2)
\,.
\label{BarKev94-6}
\eea
Here $(x,y)$ is the position of the spiral tip, $s$ is the tip speed, and
$\dot{\phi}= \gamma_0 w$ the instantaneous rotational frequency. The
dynamics of a spiral are independent of its position and these equations are
invariant under translation on the plane and rotations of the x-y plane about the origin.
Generator of infinitesimal transformations for rotation and translations of the plane are given by:
\\
$R({\gamma})
\left(
\begin{array}{c}
x\\y\\\phi\\s\\w
\end{array}
\right) \longrightarrow \left(
\begin{array}{c}
x \cos{\gamma}+y \sin{\gamma}\\
y \cos{\gamma}-x \sin{\gamma}\\
\phi + {\gamma}\\
s\\
w
\end{array}
\right)\,$,\quad $T(\alpha,\beta)
\left(
\begin{array}{c}
x\\
y\\
\phi\\
s\\
w
\end{array}
\right) \longrightarrow \left(
\begin{array}{c}
x +{\alpha} \\
y +{\beta}\\
\phi \\
s\\
w
\end{array} \right)
\,$.
\\
The tangent field for the symmetry transformations of R and T is given by:
$\newline
\left(\begin{array}{c}
y \\-x \\1\\0\\0 \end{array} \right)\, \quad, \quad
\left(\begin{array}{c}
1 \\0 \\0\\0\\0\end{array} \right)\,  \quad \&  \quad
\left(\begin{array}{c}
0 \\1 \\0\\0\\0 \end{array} \right)\,$.

It is clear that the group orbit of any point corresponding to rotation of x-y plane is a
non-compact left-handed spiral. Translation in the x-y plane can be carried out by just
changing x and y as motion along the translational subgroupgroup orbit does not affect any
other dynamical variables.


\textbf{Rotating the dynamics to a slice hyperplane:} Choosing a template point $(\hat{x}_0,\hat{y}_0,\hat{\phi}_0
,\hat{s}_0,\hat{w}_0)$ and applying the three slice conditions (one for each
group tangent) gives us a 2 dimensional manifold $(\hat{x}_0,\hat{y}_0,\hat{\phi}_0
,s,w)$. Rotating all the solutions into the hyperplane $\phi=\hat{\phi}_0$ followed by
translation to the point $(\hat{x}_0,\hat{y}_0,\hat{\phi}_0,s,w)$ leaves us with the quotient space
$[s,w]$. The dynamics thus obtained lies entirely on the slice and has no translational or rotational symmetry.
\begin{figure}[h!tb]
\begin{center}
\includegraphics[width=0.9\textwidth]{ksk_spiral2}
\end{center}
\caption{ Barkley ODE Dynamics.}
\label{fig:ksk_spiral2}
\end{figure}

\emph{Why is this trivial?} In this particular case, the ODEs are
designed such that the s and w components of the dynamics are already
independent of other three variables. Barkley and Kevrekidis
\rf{BarKev94} have solved these for all possible cases. Although in
this simplified model symmetry reduction has not provided us with a
new insight, the \mslices\ can be equally well applied to higher-dimensional
systems.

\item[2013-10-28 Predrag] Have not checked the details, but it looks
good. If you feel confident about it, you can explain it to the
cardiac group on Wed 2pm (this one, or a later one).

Can you write down explicitly the
\HREF{http://www.streamsound.dk/book1/chaos/chaos.html\#216/z}
{reconstruction equations}? They are needed to go from the $[s,w]$ plane
back to the full \statesp\ and generate the motions that one sees
there...

\item[2013-11-06 Kamal] I have used the post-processing method of moving frames. So we have to basically keep track of the original dynamics and going back from slice to full state space dynamics is just applying the same operations in reverse. Did you mean I should do it \emph{"dynamics within the slice"} way? Also I wanted to remind you about my final project for QFT course.
    
\item[2013-10-17, 2013-12-04 Predrag] I was thinking that it might be useful to
learn about the \emph{`Gribov ambiguity'}. My notes (now moved from from svn repository
\\
\texttt{siminos/blog/lit.tex}) are here:
\HREF{../reducesymm/QFT/blog.pdf} {\texttt{reducesymm/QFT}}. Have a look...

\end{description}
\renewcommand{\ssp}{a}
