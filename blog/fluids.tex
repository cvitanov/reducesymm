% siminos/blog/fluid.tex
% $Author$ $Date$

\chapter{Fluids}
\label{chap:fluids}

\begin{description}

\item[2009-12-12 Predrag] Found a whole bunch of symmetry
reduction in fluids papers.

G. Haller and I. Mezi\'c\rf{HaMe98}
``Reduction of three-dimensional, volume-preserving flows
with symmetry''
might apply to our fluid dynamics reductions, both
Eulerian and Lagrangian.
They say ``
We develop a general, coordinate-free theory for the reduction
of volume-preserving flows with a volume-preserving symmetry on
three-manifolds. The reduced flow is generated by a
one-degree-of-freedom Hamiltonian which is the generalization
of the Bernoulli invariant from hydrodynamics. The reduction
procedure also provides global coordinates for the study of
symmetry-breaking perturbations. Our theory gives a unified
geometric treatment of the integrability of three-dimensional,
steady Euler flows and two-dimensional, unsteady Euler flows,
as well as quasigeostrophic and magnetohydrodynamic flows.
''

They reinvent slice, with a new name again:
``
define the `orbit projection map'
$P: \pS \to \pSRed $
''

G. Haller and I. Mezi\'c\rf{HaMe98}, as well as Xia\rf{Xia92} and Cheng
and Sun\rf{CheSu90} might be relevant to Elton
\HREF{http://chaosbook.org/projects/index.shtml\#Elton}
{Lagrangian mixing project}.

\item[2010-05-21 Igor Mezi\'c]
Symmetries in fluids were my first paper\rf{MeWi94}. This is prior to the
one with Haller\rf{HaMe98}.

But Igor was way too busy to discuss the paper with Predrag.

\item[2010-06-01 Predrag]
Might mention these and ``the so-called automorphic and resolvent systems''
both in ChaosBook.org and CLE.tex:

J. Early and J. Pohjanpelto and R. Samelson\rf{EaPoSa10},
\HREF{http://www.aimsciences.org/journals/displayArticles.jsp?paperID=5039},
{``Group foliation of equations in geophysical fluid dynamics,''}
say:\\
``The method of group foliation can be used to construct solutions
to a system of partial differential equations that, as opposed to Lie's method of
symmetry reduction, are not invariant under any symmetry of the equations.
The classical approach is based on foliating the space of solutions into orbits
of the given symmetry group action, resulting in rewriting the equations as a
pair of systems, the so-called automorphic and resolvent systems, involving the
differential invariants of the symmetry group, while a more modern approach
utilizes a reduction process for an exterior differential system associated with
the equations. In each method solutions to the reduced equations are then used
to reconstruct solutions to the original equations.
''

\item[2010-04-06 Bruce Boghosian] % <bruce.boghosian@tufts.edu>

I was thinking more about the matter of reduction of the
state space by symmetries, and one thing occurred to me:  One
very important example of reduction by a symmetry is when you
pass from the Lagrangian description of a fluid to the
Eulerian description.

Of course, the state space of the former is much bigger than
the state space of the latter, since you keep track of labels
of ``fluid particles.''
													\toCB
The so-called Low Lagrangian (named
after Francis Low\rf{Low58}) describes the former; it is canonical in
nature and can be Legendre transformed to a Hamiltonian
description.  A Hamiltonian description in the Eulerian state
space was not known until the 1980s when
\HREF{http://www.ph.utexas.edu/~morrison}{Phil Morrison}\rf{Morr80,Morr98} and
others derived it; it is a singular Lie-Poisson bracket with
large classes of Casimir functionals, etc.
[PC: I have uploaded the above three \refrefs{Low58,Morr80,Morr98}
to Zotero]

Roughly speaking, the Lagrangian description labels all of
the ``fluid particles,'' and the Eulerian description is
obtained by modding out by the symmetry associated with the
interchange of labels.  To the extent that there are an
infinity of ``fluid particles,'' this is reduction by an
infinite-dimensional group.

This must mean that the state space in the Lagrangian
framework has group orbits of states, all of which correspond
to the same Eulerian state.  You would need to locate UPOs by
seeing when time orbits return to the same group orbit,
rather than to the same point.  It should not be difficult to
do this, since we have no trouble at all converting the
Lagrangian description of a fluid to an Eulerian description.
(So-called ALE codes do this routinely, as part of the
evolution.)  So the reduction problem is solved for this very
complicated case.

I am not immediately certain where to go with this
observation.  In a way, it is providing the opposite of what
you are seeking; you want ever smaller reduced state spaces,
and this gives something much bigger, from which our Eulerian
state space is already a substantial reduction.  Still, this
way of looking at things might provide some insights.

On the down side, if $f$ denotes the state in the Lagrangian
description, there will be few or perhaps even no orbits that
satisfy $f(t+T) = f(t)$, since the requirement that every
individual fluid particle return to its initial position is
very severe.  The thing that they will satisfy is $f(t+T) = g
f(t)$ for some $g \in G$, the group of all interchanges of
particle labels.

\item[2010-04-07 Predrag]
You are right - Lagrange to Euler description
is a large state space reduction. I do not see
``symmetry associated with the interchange of labels,''
if that is meant to be a discrete permutation symmetry, as
the labels are continuous, $x=x(x_0,t)$, but I see that
for Euler velocity fields steady solutions each time-labeled
Lagrangian trajectory gets replaced by time-invariant
vector tangent fields, so $1d$ time evolution invariance of each
Lagrangian
orbit is quotiented out. For unsteady flows I do not see any reduction,
Lagrangian particle at $x(x_0,t)$ instant $t$ seems to have as
much information as the as the tangent field $v(x,t)$.


\item[2010-12-31 Philip J. Morrison]
For finite systems symmetry reduction is beautiful and rigorous, and due
to Sourieu. For PDEs Lie's third theorem does not hold (for infinite
parameter Lie groups) and things are basically only formal and despite
the nice language of Arnold and others, these things are not
mathematically any more rigorous than `physicists mathematics'.  I have
been meaning to write a little tutorial on this, explaining how
$T*G/G\equiv g*$ \etc\ is equivalent to the chain rule for functionals,
and might be motivated soon to do this soon.

\item[2011-02-04 ES] This is interesting. For Morrison, Arnold is not rigorous.
For Marsden, Morrison is not rigorous.
For \HREF{http://people.math.gatech.edu/~gangbo/publications/geowassmemoirs_final.pdf}{Gangbo},
Marsden is not rigorous.\\

I vote for \HREF{http://pauli.uni-muenster.de/~munsteg/arnold.html}{Arnold}.

\item[2011-02-04 ES] More serious comment. Morisson talks about
reduction by an infinite dimensional Lee group as being non-rigorous,
and of course I cannot object. But is there a
problem with finite dimensional Lie groups acting on infinite dimensional
spaces?

\item[2011-01-10 Predrag]
Morrison\rf{Morr80,Morr98} says:
``
There are two parts to reduction: kinematic and dynamic. The kinematic
part is concerned with the use of special variables that have a certain
closure property, while the dynamic part refers to a type of symmetry of
the Hamiltonian, viz., that the Hamiltonian be expressible in terms of
the special variables. The symmetry of the Hamiltonian can motivate the
choice of the reduced variables. For example, for the ideal fluid the
form of the Hamiltonian suggests the use of Eulerian variables as a
reduced set.
The symmetry of the Hamiltonian gives rise to one or more constants of
motion (Casimirs) that can, in principle, be used to reduce the order of
the system. However, the term reduction is, in a sense, a misnomer since
in actuality the procedure does not reduce the order of a system, but
splits the system in such a way that it can be integrated in stages.
''

Then he illustrates this by the free rigid body, which, as far as I can
tell is the only actual application of Marsden-Weinstein marsdenia.

``
and symmetries are related to constants of motion, it should come as no
surprise that a general expression for constants of motion, which are the
Casimir invariants, comes along with the reduction framework. A clean way
of seeing this is afforded by triple-bracket dynamics (Bialynicki-Birula
and Morrison, 1991), which is a generalization of a formalism due to
Nambu (1973).
''

``
Reduction for the ideal fluid amounts to the transformation from
Lagrangian to Eulerian variables.
''

I have now read the  Francis Low\rf{Low58} paper and got nothing out of
it. But marsdenites love it. Cendra, Holm, Hoyle, and Marsden\rf{CHHM98}
``The {Maxwell-Vlasov} equations in {Euler-Poincar\'e} form'' might be
worth a read:
``one should do the Legendre transformation slowly and
carefully when there are degeneracies.''
``
In summary, we have taken an existing action, due to Low [1958], for the
Maxwell-Vlasov system of equations and demonstrated how to rederive this system
as Euler-Poincar\'e equations.
''

\item[2010-05-05  Predrag]
Who else refers to Gibbon
\etal\rf{FowlerCLE82,GibMcCLE82,FowlerCLE83,GibMcCLE83}? Might have to
look at \refrefs{ToDe98,VlToDe98a, RaHaAb96}. Not cheerful.

\item[2011-01-15 ES]
I cannot see the symmetry reduction in the original work\rf{Morr80} and
the references in \rf{Morr98} to this particular article do not
illuminate the point. Marsden and Weinstein [Physica 4D, 394 (1982)] are
in general credited with applying reduction to obtain a Hamiltonian
formalism for Vlasov-Poisson system (the difference of their bracket to
that of Morisson being that the latter does not satisfy the Jacobi
identity). I any case you might want to avoid the word fluid when
referring to \rf{Morr80}, as the Vlasov-Maxwell system is a kinetic,
rather than a fluid description of a collisionless plasma.


\item[2011-01-15 PC]
You are right - none of this references illuminate the point or, as is
the case with Marsdenism in general, any point. This is being submitted
to a special issues in honor of Morisson 60th birthday, so something has
to be said about his work on symmetry reduction. Cristel is a big fan of
this work, and always talks about 'casimirs' in ways that do not make
sense to me. Morisson did it first, so they cite him, but I put it here
so I can get Morisson to talk to me about what he really did. Thanks for
'fluid' - I eliminated it now, along with 'symmetry' as in `symmetry
reduction.'
\item[2011-02-04 PC]
Asked Morrison himself. He says his contribution are
Morrison and Greene\rf{MorrGree80} are the
noncanonical Poisson brackets. Marsden picked it up, rewrote it
in his geometrical way. In
the beginning he referred to Morrison, but later on only to
his own earlier papers (which in turn do refer to Morrison).
The symmetry reduction is on the relabeling symmetry.

\item[2011-02-04 ES]
I see. Yet another application of the
\HREF{http://pauli.uni-muenster.de/~munsteg/arnold.html} {Arnold
Principle}. I refer to Morrison but not Marsden in my Vlasov paper, for
what it's worth.


\item[2011-01-15 ES]
Splitting hairs, but since it is for P. Morisson's birthday you might
care about such things: Since the norm is derived from an inner product you
could say ``Hilbert space.'' See Stone and Goldbart, Chapter 2.

\item[2011-01-15 PC]
Not sure - I think I only need a Banach space - we should continue this
in discussing the ChaosBook, more precise write-up. I am avoiding giving
names to this because (a) trying to write this up as simply as possible,
so a physicist can understand it - Antimarsdenism :-), and (b) physicists
associate Hilbert space with quantum mechanics, and this is not the time
and place to reeducate them.

\item[2011-01-15 ES]
Yes, you might be able to get away without deriving the norm through an
inner product and work in a Banach space. Another note: using infimum
does not seem consistent with Antimarsdenism. Do you really need greatest
lower bound or could you do with just saying global minimum?

\item[2011-01-16 PC] Thanks. Purged `the infimum' throughout in the favor of
`the shortest distance.' By the way -to Stefan- it is not spelled
\HREF{http://en.wikipedia.org/wiki/Infimum}{`infinum'}.

\item[2012-04-03 \HREF{http://georgehaller.com}{George Haller}] When I
had just finished my PhD and came to Courant I was invited by fellow
Hungarian Peter Lax to a dinner with Vladimir Igorevich Arnol'd, Louis
Nirenberg and George Zaslavsky. After several failed attempts at small
talk with Arnol'd (all ending with ``Why did you bring this up?'') I
decided to tell him about the core result of my thesis. He cut me off and
said ``Why are you telling me this? I know this since 1966.'' George
asked where was it published, and Arnol'd said in the Proceedings of the
International Congress of Mathematicians. In Russian. So George races to
the library, photocopies the article, races back. It is five pages:
the first two thirds are standard definitions, and the third one is a rather
obvious theorem that has nothing to do with George's thesis. So he goes
back and explains to Arnol'd. Arnol'd listens, understands that George's
result is about something else and say ``This has nothing to do with my
result. Why should I be interested?'' Turns his back to George and that
was the end of it.

\item[2012-04-03 George Haller] learning from the master:       \toCB
Symmetry reduction is trivial in all engineering situations - the only
symmetry of interest is invariance under the abelian circle group $S^1$.
How to deal with it we learn in any introductory mechanics course. If a
Legendre transform of the Lagrangian is performed only with regards to
cyclic coordinates
%(those not appearing explicitly in the Lagrangian),
the result is the Routhian. Marsden complicates it, because he insists on
making the invariance non-abelian, but the only problem of interest is
then the rigid 3D body.

[Predrag: ``This has nothing to do with my result'']

Paper with Mezi\'c\rf{HaMe98} is my only paper on symmetries in fluids.
Igor and I were at CalTech, and Marsden was there for a year. He said
this was trivial, but made an error. That is unusal, Mrsden making an
error. Igor, who worked on this in his thesis and I did it right: we
showed how to reduce volume-preservation for a number of flows.

[Predrag: ``This has nothing to do with my result'']

But George was way too busy to discuss the paper (or Predrag's result)
with Predrag.

\item[2011-07-16 Predrag]
{\em Computation of finite time {Lyapunov} exponents using the
{Perron-Frobenius} operator}, by Phanindra Tallapragada,
\arXiv{1101.4338}. He/she says
``
    The problem of phase space transport which is of interest both
    theoretically and from the point of view of applications has been
    investigated extensively using geometric and probabilistic methods.
    Two of the important tools for this that emerged in the last decade
    are the finite time Lyapunov exponents (FTLE) and the
    Perron-Frobenius operator. The relationship between these approaches
    has not been clearly understood so far. In this paper a methodology
    is presented to compute the FTLE from the Perron-Frobenius operator,
    thus providing a step towards combining both the methods into a
    common framework.
''

                                                    \toCB
Reading it I have learned that the method of Lagrangian coherent
structures studies stretching and contraction around reference
trajectories and is therefore local in nature; it provides information
about invariant manifolds that determine transport in \statesp. The
Cauchy-Green tensor is given by
\[
C(\xInit; t_0; t) = \jMps(\xInit,t)^T \jMps(\xInit,t)
\,.
\]
$\jMps^t$ can be expressed in the singular value decomposition (SVD) form
\beq
\jMps = {U} {D}  {V}^T
\ee{SVD-j}
where ${D}$ is diagonal and real, and ${U}$, ${V}$ are orthogonal
matrices. The diagonal elements
$\sigma_{1}$, $\sigma_{2}$, $\dots$, $\sigma_{d}$ of ${D}$ are called the
\emph{singular values} of $\jMps$, namely the square root of the
eigenvalues of $\jMps^{T}\jMps = {V}{D}^{2}{V}^T$ (or $\jMps\jMps^{T} =
{U}{D}^{2}{U}^T$), which is a symmetric, positive semi-definite matrix
(and thus admits only real, non-negative eigenvalues).
The maximum growth of a
perturbation is given by the maximum principal stretch
\(
\sigma_{max}^2
\), i.e., by the maximum eigenvalue of $C$,
and the finite time Lyapunov exponent is given by
\[
\Lyap(\xInit; t_0; T) = \frac{1}{T} \log \sigma_{max}
\]

He then looks at how Perron-Frobenius operator transports
stability eigenvectors back and forth, and relates this to
the finite-time Lyapunov exponents. Might be worth a closer read.

\item[2012-04-03 George Haller]
Nonlinear seminar at GT was on the unpublished paper G. Haller and F. J.
Beron-Vera, \emph{Geodesic theory of transport barriers in
two-dimensional flows}.

Cauchy-Green geodesics are pull-back (preimages) of straight lines in the future.

{\bf [Predrag]}
I believe I have seen these geodesic equations in Thiffeault 2001 (have
to dig up  the reference).

I believe I have derived these geodesic equations as a variant of our
variational principle for finding periodic orbits and partially
hyperbolic tori\rf{CvitLanCrete02,lanVar1,LCC06} but have not published
them. The idea is to minimize not $(\ssp(t) -\sspRed(t))^2$ but
the error one cycle period later
\[
\left(\jMps(\xInit,t)^\period{}(\ssp(t) -\sspRed(t))\right)^2
= (\ssp(t) -\sspRed(t))^T C(\ssp(t); \period{})(\ssp(t) -\sspRed(t))
\,.
\]
That emphasizes unstable directions, suppresses the unstable ones. I did
not pursue it, because position-dependent metric led to geodesic
equations. We live in $d$\dmn, $d$ large, not $d=2$ of Haller and
Beron-Vera.



\end{description}
