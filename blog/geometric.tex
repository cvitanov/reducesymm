% siminos/blog/geometric.tex
% $Author$ $Date$

% Predrag created this chapter 2012-03-01

\chapter{Geometric phase}
\label{c-geometric}

All things having to do with geometric phase in symmetry reduction.

\begin{description}

\item[2009-05-21 Evangelos talked to Rowley]
I am working on quotienting KS in a some sense. I am in
Snowbird so I got the chance to talk to a few people about
it. I've talked to Gilmore who initially said he cannot help
us with symmetry reduction in thousands of dimensions
(apparently you've talked to him last fall about it) but came
to my talk and got interested I think, asked for the paper
and/or thesis when available. The talk went well I believe.

I've also talked to Rowley (as in Rowley and Marsden). What
they do for ``transverse'' integration is somewhat different to
what we do for CLe. Their transverse integration depends on
the reconstruction equation (while for us it is totally
independent) so that if the reconstruction equation fails
(there is a denominator that can vanish in it) the whole
procedure, not just reconstruction, has to be restarted with
different ``template''. This is what makes their method local.
We don't have such problems I think.

\item[Predrag]
Rowley is right, see \refsect{sect:epyc2Fourier}.

\item[2011-01-24 Predrag]
Ahuja, Kevrekidis and Rowley\rf{ahuja_template-based_2007}
\HREF{http://cwrowley.princeton.edu/}
{``Template-based stabilization} of relative equilibria
              in systems with continuous symmetry'' also
merits a read, it reviews their point of view on the \mslices. Wrote to
Rowley this:

Clancy, you probably do not do this any more, but I liked your papers
with Marsden (much better than Marsden's papers without you), so here it
is:
\\
\HREF{scribd.com/full/47479245?access_key=key-2g9l6dr7a0za0hebb7a2}
{scribd.com/birdtracks}
\\
hope we give you enough credit. We did it as usual; first we did it, then we tracked
down the references - there is an attempt at history in the ChaosBook.org link.
I started out with what you call ``method of connections'' but it induces
geometrical phases that to me seem unphysical, so we switched to
slice \&\ dice approach. Not pretty, but turbulence is not pretty.
This is work in progress - we still have to do man's work, ie chart out
turbulent pipes and planes. If you want me to fix something, let me know.

\item[2011-01-25 Clancy Rowley]
I only skimmed the paper, but I'l look forward
to reading it more closely.  What you found with the geometric phase from
the ``method of connections'' is certainly consistent with what we found
with the Kuramoto-Sivashinsky equation, so that is reassuring!

\item[Evangelos]
Also talked a little bit with Kevin Mitchel about the geometrical origin
of singularities. He is not against the modified invariants and I think
he is right about the impossibility of global reduction with purely
geometrical means.

\item[\bf 2009-08-26 Predrag to Evangelos] When you tried my initial
proposal to project velocity locally (not on a global slice),
the method that when fixed will probably be called ``method
of connections'' - are you sure that the extra geometrical
phases gained by relative periodic orbits were not rational fractions of $\pi$?

\item[2009-11-09 Predrag]
Other setback. I derived a co-moving slice, method of
connections reduced state-space `reconstruction' equation for
Andreas. The advantage would be that it never encounters these
annoying singularities of fixed slices. It is cute, it depends
on $\dot{v}$ (acceleration).

Unfortunately, every time I continue the calculation, three
attempts so far, $\dot{v}$ cancels out (it should do that in
Evangelos thesis calculation for \reqv\ slice too)? and I end
up with the simple `reconstruction,' equation that I derived
earlier, for which Evangelos showed that it accrues
'geometrical phase.' And who the hell needs that.

For the life of me I cannot understand Marsden 'adjoint action'
formulation of the 'method of connections.' Andreas might, he
seems comfortable with the lingo.

\item[2009-12-03 Predrag] Went to Bonn, Andreas would not hear
another word from me about the `method of connections.' Home
alone, again...



\item[2009-08-26 Predrag's `method of connections']

%
%%%%%%%%%%%%%%%%%%%%%%%%%%%%%%%%%%%%%%%%%%%%%%%%%%
% from Rowley-Marsden paper
\SFIG{connections}
{}{
Method of connections, as illustrated by Rowley and
Marsden\rf{rowley_reduction_2003}.
}
{fig:connections}
%%%%%%%%%%%%%%%%%%%%%%%%%%%%%%%%%%%%%%%%%%%%%%%%%%
%


We decompose $\vel(x)$
in a part $\vel_\shortparallel$ parallel
to the group action and a part $\vel_\perp$ transverse to it,
\beq
	\vel(\ssp)=\vel_\shortparallel(\ssp)+\vel_\perp(\ssp)\,,
\ee{flowSplit}
using the projection operator
\beq
 	?? %\PperpOp_{ij}(\ssp)
 =\delta_{ij}-
    \frac{ \groupTan(\ssp)_i \groupTan(\ssp)_j}{\groupTan(\ssp)^2}
\ee{transvProj}
that projects a $d$-dimensional flow $v(\ssp)$ onto
flow
\beq
	\dot{\ssp}_\perp = \vel_\perp(\ssp) = \vel(\ssp)
    - \frac{\groupTan(\ssp) \cdot \vel(\ssp)}{\groupTan(\ssp)^2}
      \, \groupTan(\ssp)
\ee{transvFlow}
in a $(d\!-\!1)$-dimensional {\csection} transverse to the
direction fixed by the point $\ssp$. By ignoring the flow
component that can be compensated for by an $\SOn{2}$
rotation we quotient the flow by $\SOn{2}$.

For an illustration, Rowley and
Marsden\rf{rowley_reduction_2003} \reffig{fig:connections}.


Note, however, that a choice of $\ssp_0$ fixes only a
direction, so the reduced flow is still equivariant under the
action of discrete cyclic group $\Ztwo = \{e,D(\pi)\}$ on
$\ssp$, $\vel(\ssp)$ and the reference point $\ssp_0$, just
as was the case %\refeq{LorenzR}
for the Lorenz flow. %\refeq{Lorenz}.
    \PC{\emph{Mea culpa}: Here I screwed up. I forgot that rotation
    moves $\vel$ and counter-moves $\ssp$ in $\vel(\ssp)$, \ie,
    acts by the Lie derivative \refeq{inftmInv}. I could never
    understand why we do not see a translational zero eigenvalue
    everywhere (the Lie group acts globally and commutatively
    right?), but only on \eqva, \reqva\ and \rpo s. Presumably
    the projection operator \refeq{transvProj} is OK for the
    \reqv\ calculation of \refeq{sect:StabEq},
    as the action of the group on $\ssp_{\REQV{}{1}}$ is trivial?
    Not sure how to rewrite the decomposition induced by
    \refeq{transvProj} correctly, in
    terms of the full Lie derivative action, and not only the $\Lg$
    action.
    }


\item[Predrag's integral of the equivariance condition]
Here is a suspect attempt to derive a connection formula.
Integrate the equivariance condition refeq({inftmInv}):
\bea
{const} &=& (\Lg_a)_{kj}
          \int_0^t d\tau \left(
  \delta_{ik}\vel_j(\ssp) - \Mvar_{ik}(\ssp)\, \ssp_j
           \right)
  \continue
  &=& (\Lg_a)_{ij} (\ssp_j(t)-\ssp_j(0))
    - (\Lg_a)_{kj} \int_0^t d\tau \Mvar_{ik}(\ssp(\tau))\, \ssp_j(\tau)
    \continue
  &=& t(\ssp) - t(\ssp_0) - \int_0^t d\tau \Mvar (\ssp(\tau))\, t(\ssp)
  \,.
\label{integralInv}
\eea
If $const=0$, this is a formula for the transport of the group
tangent field,
\[
t(\ssp) = t(\ssp_0) + \int_0^t d\tau \Mvar (\ssp(\tau)) \, t(\ssp(\tau))
\,,
\]
which is not simply the multiplication by \jacobianM\ $\jMps$.
For a linear flow, this looks a bit unfamiliar:
\[
t(\ssp) = t(\ssp_0) + \Mvar\,\Lg_a \int_0^t d\tau \, \ssp(\tau)
\,.
\]

\item[2010-12-12 PC]
													\toCB
In  Stefan's article\rf{FrCv11} we derive the linear slice by minimizing
the distance to the slice in the Euclidian norm. Marsden\rf{Marsd92} on
p.~52 credits Smale\rf{Smale70I,Smale70II} with doing something
like this in his
\HREF{www.springerlink.com/index/P5835K07WW862766.pdf}
{\emph{Topology and mechanics}}, in the kinetic energy norm.
Marsden\rf{Marsd92} notation is essentially
impenetrable:
`The map' $\alpha(q,v)$ is `a connection on the principal $G$-bundle $Q
\to Q/G$,' and `the 1-form $\alpha_\nu$, defined by
$\braket{\alpha_\nu}{v} = \braket{\nu}{\alpha(q,v)}$,'
is `characterized by'
\beq
K(\alpha_\nu(q))
= \inf \left\{K(q,\beta) \;|\; \beta \in \mathbf{J}^{-1}_q(\nu) \right\}
\,,
\ee{Marsd9:connecPB}
where $K(p,p) = \half \| p^2 \|$ is the kinetic energy function,
and $\nu \in \mathfrak{g}^*$, $\mathfrak{g}^*$ the dual space to
the tangent space (Lie algebra) $T_e(\Group)$.
Note the `$\inf$;' some `distance(?)' is being minimized? Main Gott.
This is supposedly also `explained' in Abraham and Marsden\rf{AbrMars78}.
It intrigues me because it is their (and my own, abandoned) `method of
connections', quotienting out the group direction locally, at each point
along the full state-space trajectory by the usual decomposition into
local `horizontal' slice and `vertical' group tangent space. It
necessitates a computation of a `geometric phase,' and what is that good for?
But anyway, I will keep on trying.



Smale considers a classical mechanical system to be consist of a
configuration space, a smooth ($C^\infty$) manifold $M$ with tangent
bundle $T=T(M)$ as the space of states. The kinetic energy is defined by
a Riemmanian metric on $M$: for each $x\in M$, $K_x$ is the inner product
on the tangent space $T_x(M)$, smooth in $x$. The kinetic energy is the
function $K : T \to \reals$ defined by $K(v) = K_x(v,v)$ where $v \in
T_x(M)$. Lagrange's equations are a ODEs or a smooth tangent vector field
on the `state space' $T=T(M)$. I have not yet found out where he defines the slice
as minimum of distance measured in the kinetic energy norm, but
Marsden\rf{Marsd92} writes:


													\toCB
A Lie group acts on its Lie algebra via the ``adjoint representation.''
Let $\alpha_g : \Group \to \Group$ be the inner automorphism $\alpha_g(h)
= gh g^{-1}$. Then $\alpha_g(e) = e$, and the derivative at $e$ gives us
a linear automorphism of $T_e(\Group)$. Together these define the adjoint
representation of \Group\ on $T_e(\Group)$.


Smale in turn credits Souriau\rf{Sou70,Sou97} with having done related
work on symmetry reduction independently, in his
\HREF{http://www.jmsouriau.com/structure_des_systemes_dynamiques.htm}
{\emph{Structure}} \emph{des syst\`emes dynamiques}. Souriau blessed us forever with
coining the word `moment map' for (whatever it is meant to mean for Marsdenites,
\ie)
the generalization of angular momentum to the case that a group other than
{\SOn{n}} is a symmetry group.
This work is all (and only) in the
context of symplectic geometry, however. As far as the history is concerned,
\HREF{http://math.berkeley.edu/~alanw/Reduction.ps}
{\emph{Some comments}} \emph{on the history, theory, and applications of
symplectic reduction} by J. E. Marsden and A. Weinstein
might be of interest.

\item[2010-05-04 Evangelos] CLe reviewer 2 suggestion:

A.G. Vladimirov, V.L. Derbov, V.Yu. Toronov\rf{VlToDe98}
say:

``It is shown that the phase space of the complex Lorenz
model has the geometric structure associated with a fiber
bundle. Using the equations of motion in the base space of
the fiber bundle the surfaces bounding the attractors in this
space are found. The homoclinic "butterfly" responsible for
the Lorenz-like attractor appearance is shown to correspond
to a codimension-two bifurcation. One-dimensional map
describing bifurcation phenomena in the complex Lorenz model
is constructed.''

Get it from author's
\HREF{http://www.wias-berlin.de/people/vladimir/papers/Vladimirov21.pdf}{homepage}.

Comments before finding the link to the paper: It might be
useful, especially in connection with Kevin Mitchell's
comments. Most probably the base space of the bundle is the
slice, with the fibers being the group orbits. If they show
that the fiber boundle is trivial, that is it has the
structure of a direct product space, this would mean
we can symmetry-reduce globally with a single slice.

\item[2010-05-04 Evangelos] Vladimirov, Toronov and
Derbov\rf{ToDe97a,VlToDe98} use invariant polynomials, without knowing or
refering to Hilbert bases. In fact their change of variables is related
to the Hopf fibration. The difference seems to be that while in the Hopf
fibration the base is the $2$-sphere, the fiber is $S^1$ and the total
space is the $3$ sphere (in $C^2$), here the base is $C\times R$, the
fiber is $U(1)\simeq S^1$ and the total space is $C^2$. According to what
I read in Isham\rf{isham99}, the latter construction should not be called
a fiber bundle as the group orbits do not have the same dimension (the
origin is fixed by the group action.) So they would have to work on
$C^2\backslash \{0\}$, but to me this seems a minor point. Now, the Hopf
fibration forms the base of Kevin Mitchel's argument about frame
singularities, so we might need to have a closer look.

\item[2010-12-03  Predrag] The invariant subspace is the whole $z$ axis,
\ie, so they would have to work on $C^2\backslash \{0,0,z\}$, but to me
this seems a minor point.

\item[2012-03-06 Evangelos to Clancy]
If we follow a relative periodic orbit applying the method of slices, and
change slice a finite number of times (a discrete version of the method
of connections) could we end up with a geometric phase?

\item[2012-03-06 Predrag]
We discussed the method of connections, but Clancy has not thought deeply
about it.  My 5 cents is no - hyperplane slices are fixed in state space
and cannot be made locally orthogonal to the flow. If you are thinking of
a single time trajectory, and every so often rectifying the time
evolution by picking the velocity component normal to the group tangent
at that instant, we will get the geometrical phase only in the
infinitesimal time step limit.

BTW, Clancy said that the for the method of connections the \rpo\ phase
is split into `geometrical phase' and '[something?],'
 forgot what the second adjective was.

\item[2010-05-21 Predrag]
Remember that when we tried my initial proposal to project velocity
locally (not on a global slice), the method that Rowley and
Marsden\rf{rowley_reduction_2003} call ``method of connections'' - we
obtained extra geometrical phases gained by relative periodic orbits? We
gave up on those, but we never checked whether they had any physical
meaning. Is that what these papers are about? That would be very
important for us, as the ``method of connections'' never encounters any
singularities, the denominator in projection transverse to the group flow
is the quadratic casimir and thus strictly positive and non-vanishing
(except in the invariant subspace).

Would be good to understand the ``geometric phase'' of
\refref{ToDe94,ToDe94a}. Toronov and Derbov\rf{ToDe94,ToDe94a} apply
method of connections to CLE and study the geometric phase. I cannot find
Toronov and Derbov\rf{ToDe97}. They say:

``An investigation is made of the geometric phases in a ring laser with
   counterpropagating waves. It is shown that the frequency splitting of
   the counterpropagating waves that appears in the case of radiation
   pulsations or is induced by displacement of an external mirror can be
   regarded as a manifestation of a geometric phase.''

``Pancharatnam's geometric phase is introduced for such nonlinear
   dissipative systems as lasers and liquid flows. Two types of geometric;
   phases are shown to arise in these systems: the phase induced by the
   inner dynamics of the system and the one caused by the cyclic and
   adiabatic variation of the system parameters. A possible generalization
   of the geometric-effects theory in other dissipative systems is
   discussed.''

``We show that such phenomena of laser dynamics as
mean-phase-slope jumps and temporal phase jumps at resonance between the
cavity and spectral line frequencies are intrinsically connected with the
topology of attractors in the space of rays and can be interpreted as the
manifestations of the geometric-phase properties of the evolution
operator.''

``An investigation is made of the geometric phases in a ring laser with
   counterpropagating waves. It is shown that the frequency splitting of
   the counterpropagating waves that appears in the case of radiation
   pulsations or is induced by displacement of an external mirror can be
   regarded as a manifestation of a geometric phase.''

\refRef{VlToDe98}:
``The phase space of the complex Lorenz model has the
   geometric structure associated with a fiber bundle. Using the equations
   of motion in the base space of the fiber bundle the surfaces bounding
   the attractors in this space are found. The homoclinic ``butterffy''
   responsible for the Lorenz-like attractor appearance is shown to
   correspond to a codimension-two bifurcation. One-dimensional map
   describing bifurcation phenomena in the complex Lorenz model is
   constructed.''

\item[2010-05-21 Predrag] 'Someone' should reread
Ning and Haken articles\rf{NiHa91,NiHa92,NiHa92a} who
study geometrical phase
for \rpo s of dissipative systems.
They say things like

``
We show that the geometrical phase defined for the dissipative systems is
invariant under a unitary transformation. [...] the geometrical {(Berry)}
phases discovered in Hamiltonian systems can also be defined as resulting
from parallel transportation of vectors for nonlinear dissipative systems
with cyclic attractors.  [...] Detuned one- and two-photon lasers showing
periodic intensity pulsations are taken as examples of such systems.
''

\item[2012-03-06 Predrag] would be nice to prove, or find the proof that
the geometrical phase is invariant under smooth coordinate
transformations, \ie, coordinate independent, intrinsic property of a
\rpo. That would make the hunt for physical manifestations of it in
fluids all the more pressing. Clancy remarked that he has seen it in swimming
(see \refsect{sec:symmLit}).

\item[2012-05-26 Predrag] read
{\it Geometrical phases from global gauge invariance of nonlinear
classical field theories}, by Garrison and Chiao\rf{GaChi88}.



\end{description}
