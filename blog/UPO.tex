% siminos/blog/UPO.tex
% $Author$ $Date$

\chapter{Periodic orbit theory}
\label{chap:UPO}

This part of the blog collects references to periodic orbit theory.
Some of the articles are in the CNS Zotero collection.

As material is written up, parts of it will migrate from its
current placement into coherent sections, suitable for
inclusion into ChaosBook.org, or, God Forbid, actual {\em
publications}.

\begin{description}

\item[2011-05-15 Predrag]
\HREF{PubMed.gov}{PubMed.gov} is scary. Find a paper you
want, click on "Related citations" on the right, and you get more stuff than you
can ever digest... Zotero helps you generate (imperfect) BibTeX entries.

\item[2011-05-15 Predrag]
Creagh\rf{Creagh94},
{\em Quantum zeta function for perturbed cat maps}, says: ``
The behavior of semiclassical approximations to the spectra of perturbed
quantum cat maps is examined as the perturbation parameter brings the
corresponding classical system into the nonhyperbolic regime. The
approximations are initially accurate but large errors are found to
appear in the traces and in the coefficients of the characteristic
polynomial after nonhyperbolic structures appear. Nevertheless, the
eigenvalues obtained from them remain accurate up to large perturbations.
''
(see CNS Zotero collection)

\item[2011-05-15 Predrag]
Wisniacki \etal\rf{WiVeBeBo04},
{\em Classical invariants and the quantization of chaotic systems}, says: ``
Due to their exponential proliferation, long periodic orbits constitute a
serious drawback in Gutzwiller's theory of chaotic systems. Therefore, it
would be desirable that other classical invariants, not suffering from
the same problem, could be used in alternative semiclassical quantization
schemes. In this Rapid Communication, we demonstrate how a suitable
dynamical analysis of chaotic quantum spectra unveils the role played, in
this respect, by classical invariant areas related to the stable and
unstable manifolds of short periodic orbits.
''

\item[2011-05-15 Predrag]
Heusler \etal\rf{HeMuAlBrHa07},
{\em Periodic-orbit theory of level correlations}, says: ``
We present a semiclassical explanation of the so-called
Bohigas-Giannoni-Schmit conjecture which asserts universality of spectral
fluctuations in chaotic dynamics. We work with a generating function
whose semiclassical limit is determined by quadruplets of sets of
periodic orbits. The asymptotic expansions of both the nonoscillatory and
the oscillatory part of the universal spectral correlator are obtained.
Borel summation of the series reproduces the exact correlator of
random-matrix theory.
''

\item[2011-05-15 Predrag]
Ando \etal\rf{AnBoAi07},
{\em Automatic control and tracking of periodic orbits in chaotic systems}, says: ``
Based on an automatic feedback adjustment of an additional parameter of a
dynamical system, we propose a strategy for controlling periodic orbits
of desired periods in chaotic dynamics and tracking them toward the set
of unstable periodic orbits embedded within the original chaotic
attractor. The method does not require information on the system to be
controlled, nor on any reference states for the targets, and it overcomes
some of the difficulties encountered by other techniques. Assessments of
the method's effectiveness and robustness are given by means of the
application of the technique to the stabilization of unstable periodic
orbits in both discrete- and continuous-time systems.
''

\item[2011-05-15 Predrag]
Brack \etal\rf{BrOgYuRe05},
{\em Uniform semiclassical trace formula for
U(3) $\to$ SO(3) symmetry breaking}, says: ``
We develop a uniform semiclassical trace formula for the density of
states of a three-dimensional isotropic harmonic oscillator (HO),
perturbed by a term which term breaks the U (3) symmetry of the HO,
resulting in a spherical system with SO (3) symmetry.
We obtain an analytical uniform trace formula
which in the limit of strong perturbations (or high energy)
asymptotically goes over into the correct trace formula of the full
anharmonic system with SO (3) symmetry, and in the other limit
restores the HO trace formula with U (3) symmetry.
''

This might be useful to us in leading us to physically interesting systems
with non-abelian symmetries. However, have no time today to continue...
%
%Bengtsson I, Br�annlund J and � Zyczkowski K 2002 Int. J. Mod. Phys. A 17 4675
%give example of

\item[2011-05-16 Predrag]
Mark Srednicki has several interesting recent papers on Riemann zeros.
In {\em Nonclasssical Degrees of Freedom in the Riemann Hamiltonian},
\arXiv{1105.2342},
he says: ``
The Hilbert-Polya conjecture states that the imaginary parts of the zeros of
the Riemann zeta function are eigenvalues of a quantum hamiltonian. If so,
conjectures by Katz and Sarnak put this hamiltonian in Altland and Zirnbauer's
universality class C. This implies that the system must have a nonclassical
two-valued degree of freedom. In such a system, the dominant primitive periodic
orbits contribute to the density of states with a phase factor of -1, which
partially resolves a previously mysterious sign problem for oscillatory
contributions to the density of the Riemann zeros.
''

\item[2011-05-16 Predrag] I believe that Ruelle's linear response
theory is fundamentally wrong, as deterministic diffusion transport
coefficients are nowhere differentiable.
Lucarini \etal\ do like it. In
{\em Relevance of sampling schemes in light of Ruelle's linear response
 theory}, \arXiv{1105.2527}, they say: ``
We reconsider the theory of the linear response of non-equilibrium steady
states to perturbations. We first show that by using a general functional
decomposition for space-time dependent forcings, we can define elementary
susceptibilities that allow to construct the response of the system to general
perturbations. Starting from the definition of SRB measure, we then study the
consequence of taking different sampling schemes for analyzing the response of
the system. We show that only a specific choice of the time horizon for
evaluating the response of the system to a general time-dependent perturbation
allows to obtain the formula first presented by Ruelle. We also discuss the
special case of periodic perturbations, showing that when they are taken into
consideration the sampling can be fine-tuned to make the definition of the
correct time horizon immaterial. Finally, we discuss the implications of our
results in terms of strategies for analyzing the outputs of numerical
experiments by providing a critical review of a formula proposed by Reick.
''

%\item[2011-XX-XX Predrag]
%XXX \etal\rf{XXX},
%{\em XXX}, says: ``
%XXX
%''

\end{description}

%\item[2011-XX-XX Predrag]
%XXX \etal\rf{XXX},
%{\em XXX}, says: ``
%XXX
%''
% (see CNS Zotero collection)
