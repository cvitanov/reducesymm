% siminos/blog/bluesky.tex
% $Author$ $Date$

\chapter{Bluesky research}
\label{bluesky}

\section{Fish}

\subsection{Heteroclinic connections literature}

Do literature review of algorithms to
compute heteroclinic connections.

Are ``Robust Heteroclinic Cycles" relevant to our heteroclinic connections?
\\
http://meetings.siam.org/sess/dsp\_programsess.cfm?SESSIONCODE=6204

\subsection{Find nontrivial stable coherent state}

ChaosBook.org shows that a perturbed H\'enon falls into 13-cycle.
Lan and Cvitanovi{\'c}\rf{lanCvit07} show that ``turbulent" KS has an
unexpected, not trivial {\em stable} \eqv.

Find such by slightly varying cell size,
to cure community from believing that
the laminar state is the only stable attractor.

\subsection{Study bifurcations}

\subsection{Representations of dihedral group}

\noindent {\bf  PC 2007-05-17}:
 refTab~{tab:newfp-symm}
is a nice test of one of the discrete symmetries of \pCf, but
there are 4 discrete symmetries invariant subspaces.
    \PC{copy tab:newfp-symm to here?}
So continuing on
\refeq{projOp1}, I'll construct 4 irreps of
of $C_2 \times C_2 = D_2$ dihedral group generated by 1/2-cell
shifts $\tau_x, \tau_z$:
\begin{align}
P_x^\pm &= \frac{1}{2}(1 \pm \tau_x)
    \continue
P_z^\pm &= \frac{1}{2}(1 \pm \tau_z)
%\label{projOp:tau}
\end{align}
They resolve identity into four irreps
\begin{align}
1 &= ({P}_x^+ + {P}_x^-) ({P}_z^+ + {P}_z^-)
    \continue
  &=  {P}_x^+ {P}_z^+
   +  {P}_x^+ {P}_z^-
   +  {P}_x^- {P}_z^+
   +  {P}_x^- {P}_z^-
    \continue
  &= {P}_1 + {P}_2 + {P}_3 + {P}_4
    \,,
\label{projOp:tau1}
\end{align}
where subscripts follow convention of Gibson geometry.tex
equation ej\_defn.
The 4 orthonormal projection operators, together with
their conventional crystallographic labels, are
\begin{align}
P_1 &= \frac{1}{4}({1} + \tau_x + \tau_z + \tau_{xz})
    & \qquad    A_1
    \continue
P_2 &= \frac{1}{4}({1} + \tau_x - \tau_z - \tau_{xz})
    & \qquad    B_2
    \continue
P_3 &= \frac{1}{4}({1} - \tau_x + \tau_z - \tau_{xz})
    & \qquad    A_1
    \continue
P_4 &= \frac{1}{4}({1} - \tau_x - \tau_z + \tau_{xz})
    & \qquad    B_2
\label{projOp:taus}
\end{align}
I looked up crystallographic labels in a hard-to-find book by
W.~G.~Harter, but they are surely standard. Along with this comes a
4$\times$4 character table which we probably will not use here.

\Eqva\ could belong to some (or none), and from the \ubranch\ and the KS
calculations
we already have examples of their stable/unstable manifolds
having different symmetries from the ``base flow" \eqv.

JH:{
Inversion \refeq{InvOp} $I^2=1$ does not split this further,
since
$s_1*I = s_2$ and $s_2*I=s_1$ closes the group at 4 elements.
   }

Taken together with the continuous symmetries presented above, each \eqva\ may
be taken as a representative of up to 4 tori in \statesp . However, the \eqva\
found so far all possess $s_1$ and $s_2$ symmetries, so there is only
1 torus in those cases.

Some earlier musings, repeated:
\begin{enumerate}
  \item
So, one should always also check the symmetry under
\refeq{projOp1}
projection as well: there are 4 symmetry subspaces,
SS, AS, SA, and AA.
\Eqva\ could belong to some (or none), and from the \ubranch\ and the KS
calculations
we already have examples of their stable/unstable manifolds
having different symmetries from the ``base flow" \eqv.

JFG:{
I don't think the antisymmetric subspaces can be invariant under
Navier-Stokes flow. An antisymmetry is a symmetry with an additional
sign change on all components of $\bu$. The sign change on $\bu$ commutes
through all terms of the Navier-Stokes equations except the nonlinear term.
So if the symmetric subspace is invariant (all terms in NS commute) the
antisymmetric subspace is not (one term doesn't). The equilibria do have
antisymmetric eigenfunctions, but they lead trajectories out of the
symmetric subspace into the general space.

  This is probably confusing with respect to traditional terminology for
KS. There, odd functions $u(x) = -u(-x)$ are called antisymmetric, and KS is
invariant under this antisymmetry. But they aren't invariant under the
corresponding symmetry $u(x) = u(-x)$. Under
$s : u(x) \to u(-x)$, all the linear terms are unchanged, since they're even
derivatives, but the nonlinear term changes sign. So $d s(u)/dt \neq s du/dt$.


PC: agreed
}
\end{enumerate}


\section{Bluesky, completed}

\subsection{Placeholder for completed projects}
