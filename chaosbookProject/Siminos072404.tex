\documentclass[article,preprint]{revtex4}

\input defs

\begin{document}




 Your hunch is we should use the metric $\mathbf{M}=\mathbf{J}_p^T(x_o)
 \mathbf{J}_p(x_o)$ of the central fixed point for the
 unstable manifold and the eigenvectors of $\mathbf{M}$ as the local
 coordinate system. Since $\mathbf{M}$ is obviously a symmetric
 matrix, its eigenvectors form an orthogonal basis.

 I have gathered in the following what I could think on the subject as well as my questions.  

\section{Metric tensor on a periodic orbit}
 
 When we define a coordinate transformation from cartesian to curvilinear
 coordinates:
 \beq
   u^i = f(x^1,x^2\ldots, x^N)\, , \  i=1,\ldots , N
 \eeq
 the metric tensor is determined by the transformation,
 as well the local coordinate system. The metric tensor is given by:
 \beq
  \mathbf{g}=\mathbf{J(x)^T}\mathbf{J(x)}
 \eeq
 where $\mathbf{J}(x)$ the Jacobian of the transformation. The
 covariant and contravariant basis read:
 \beq
  \mathbf{\varepsilon}_i = \frac{\partial\mathbf{r}}{\partial u_i}
 \eeq
 \beq
  \mathbf{\varepsilon}^i =  \mathbf{\nabla} u^i
 \eeq
 Those bases are not in general orthogonal even if $\mathbf{g}$ is a
 symmetric matrix. So, I cannot see the connection between the
 curvilinear coordinate system and the eigenvectors of $\mathbf{M}$. 


 In the case of a flow, the definition of local coordinates that leads
 to $\mathbf{M}$ as the metric tensor seems to be:
 \beq
  u^i(x)=\left(f^{T_p}(x)\right)^i
 \eeq
 Then the contravariant basis vectors read:
 \begin{eqnarray}
  \mathbf{\varepsilon}^i & = & \mathbf{\nabla} u^i \nonumber \\
                         & = & \mathbf{\nabla} \left(f^{T_p}(x)\right)^i
                         \nonumber \\
                         & = & \frac{\partial
                           \left(f^{T_p}(x)\right)^i}{\partial x_k}\,
                         \hat{\mathbf{x}}_k \nonumber \\
                         & = & \left(J_p(x)\right)^{ik}\,
                         \hat{\mathbf{x}}_k 
 \label{eq:contravariant basis}
 \end{eqnarray}
 where $\hat{\mathbf{x}}_j$ the unit vectors of the Cartesian
 coordinate system (in writting \refeq{eq:contravariant basis} we note 
 that there is no difference between contravariant and 
 covariant components in a Cartesian system). Perharps the connection
 with the eigenvectors of $\mathbf{J}$ or $\mathbf{M}$ is obvious, but
 I cannot see it.

\section{Metric tensor on the unstable manifold}

  Although the metric $\mathbf{J}_p^T(x_o)\mathbf{J}_p(x_o)$ on $x_o$ seems to
  be the natural one for a periodic orbit and points in its vicinity,
  something I cannot understand from the beginning is why it should be constant along the unstable
  manifold. In linear order this would be the case, but the actual
  manifold does not lie on the unstable eigendirection of $x_o$,
  rather it bends and changes its orientation. Furthermore, on another
  periodic point, e.g. $\overline{01}$, the natural metric should be
  given by $\mathbf{J}_p^T\mathbf{J}_p$ computed at that point, and
  will be different from the one computed on $x_o$. The metric on non-periodic
  points is then a problem, but perharps $\mathbf{J^T}(x)\mathbf{J}(x)$
  could do the job if computed for the appropriate time interval,
  perharps equal to the number of crossings with the Poincar\'e
  section of a nearby periodic orbit. This is based on the observation
  that $\mathbf{J}^t(x)\rightarrow\mathbf{J}^t(x_o)$, as $x \rightarrow
  x_o$.  

\end{document}