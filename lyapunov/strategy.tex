\ifsvnmulti
 \svnkwsave{$RepoFile: lyapunov/strategy.tex $}
 \svnidlong {$HeadURL$}
 {$LastChangedDate$}
 {$LastChangedRevision$} {$LastChangedBy$}
 \svnid{$Id$}
\fi

\chapter{Strategy, to write up}

% J Hightower http://www.brainyquote.com/quotes/authors/j/jim_hightower.html
% - former texas politician, author, speaker 1943-
% Jamie Waite (?)
% 'Only dead fish swim with the stream' a quote by Malcolm Muggeridge
% from http://www.kittozutto.com/showcase/go-with-the-flow/
\begin{bartlett}{
Even a dead fish can go with the flow.}
\bauthor{
Jim Hightower, Texas politician}
\end{bartlett}


\HREF{http://www.urbandictionary.com/define.php?term=Go\%20with\%20the\%20flow}
{urbandictionary.com}:

\begin{enumerate}
   \item To not push against prevailing behavior/norms/attitudes,
   occasionally including bowing to peer pressure.

   \item To not attempt to exert a large amount of influence on the
   course of events, whether a specific series of events or events in
   general. A person who does this is often referred to as ``laidback''
   or ``easygoing''.

   \item First known to be used by the Roman Emperor Marcus Arelius in
   his `Meditations.' Marcus wrote a lot about the flow of
   thoughts and happiness and concluded that ``most things flow
   naturally'' and that it was better to ``go with the flow''
   than to try to change society.
 \end{enumerate}

\section{How to read me}

Throughout:  {\textdollar} on the margin
{\steady}
indicates that the text has been transferred to
articles siminos/*/,  or to ChaosBook.org
chapters, such as
\HREF{http://ChaosBook.org/continuous.pdf}
{continuous.pdf}.

How to read this blog: go first to the latest blog post, end
of \refchap{c-DailyBlog}. For specialized topics, consult the
Contents.


\section{Papers to write}

See siminos/blog for our most complete listing of
PACS classification, keywords.

\subsection{\emph{Phys. Rev. E} ``The physical dimension of...''}

\begin{description}

\item[2010-05-25 Vaggelis]
Will we go for an arXiv version?

\item[2010-06-07 Predrag]
I think one should always submit
any article that is worth publishing also to arXiv;
it is open to anyone, rich or poor, and it is
more likely to reach the intended audience than only a publication through
any single journal.

\end{description}

\subsection{Reduced trace formulas?}

\begin{description}
 \item[2010-06-17 Vaggelis]
Since I have all rpo's up to level 7 for CLE I think I should try
to apply ``Continuous symmetry reduced trace formulas'' so that I get an incentive
to understand the paper. After all this group theory, it should be easier now.
 \item[2010-06-18 Predrag]
Would be nice if you did - both to understand the group theory better, and
also because I am not sure I have not missed some important detail about
invariant subspaces when I wrote the paper. Would be great to recycle KS
next, if CLE works.
\end{description}


\subsection{\emph{SIAM J. Appl. Dyn. Syst. ?}}

\begin{description}

\item[2010-06-07 Predrag] Next, the
``Dimensional reduction of Kuramoto-Sivashinsky ...'' paper:
40,000 \rpo s and noplace to go?
Can include movies and more graphics

\end{description}
