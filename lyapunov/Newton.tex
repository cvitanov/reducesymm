\ifsvnmulti
 \svnkwsave{$RepoFile: siminos/lyapunov/Newton.tex $}
 \svnidlong {$HeadURL$}
 {$LastChangedDate$}
 {$LastChangedRevision$} {$LastChangedBy$}
 \svnid{$Id$}
\fi

\renewcommand{\ssp}{x}            % state space point

\section{Poincar\'e sections}
\label{s:PoincSect}
% Predrag							 5mar2011
% extracted from ChaosBook.org 		13jun2008
% \Chapter{maps}}{Discrete time dynamics}

\noindent
Successive trajectory intersections with a {\em Poincar\'e
section}, a $(d-1)$-dim\-ens\-ion\-al hypersurface or a set of
hypersurfaces $\PoincS$ embedded in the $d$-dim\-ens\-ion\-al
{\statesp} $\pS$, define the {\em Poincar\'e return map}
$\PoincM({\ssp})$, a $(d-1)$-dim\-ens\-ion\-al map of form
\beq
\ssp' = \PoincM({\ssp})
          =  \flow{\tau(\ssp)}{\ssp}
\,,\qquad
\ssp', \ssp \in \PoincS
\,.
\ee{PoincMap}
Here the {\em first return function} $\tau(\ssp)$--sometimes
referred to as the {\em ceiling function}--is the time of
flight to the next section for a trajectory starting at $\ssp$.
The choice of the section hypersurface $\PoincS$ is altogether
arbitrary, but in practice one
often needs only a local section--a hyperplane of
codimension~1 intersected by a swarm of trajectories near to
the trajectory of interest. This hyperplane can be specified
implicitly through a function $\PoincC(\ssp)$ that is zero
whenever a point $\ssp$ is on the Poincar\'e section,
  \beq
\ssp \in \PoincS \quad \mbox{iff}
\quad \PoincC(\ssp) = 0 \,.
  \ee{PoincU}

First, the flow should pierce the
hypersurface $\PoincS$, rather than being tangent to it. A
nearby point $\ssp + \delta\ssp$ is in the hypersurface
$\PoincS$ if $\PoincC(\ssp+ \delta\ssp)=0$. A nearby point on
the trajectory is given by $\delta\ssp = \vel \delta t$, so a
traversal is ensured by the {\em transversality condition}
\beq
    (\vel \cdot \pde\PoincC) =
    \sum_{j=1}^{d}
    \vel_j(\ssp) \, \pde_j \PoincC(\ssp) \neq 0
\,,\quad
    \pde_j \PoincC(\ssp) =
    \frac{\partial~}{\partial \ssp_j} \PoincC(\ssp)
\,,\quad
    \ssp \in \PoincS
\,.
\ee{transvrCond}
Second, the definition of {Poincar\'e return
map} $\PoincM({\ssp})$ needs to be supplemented
with the orientation condition
\bea
\ssp_{n+1} = \PoincM({\ssp_n}) \,,\qquad &&
\PoincC(\ssp_{n+1}) = \PoincC(\ssp_{n})   = 0 \,,\quad n \in
\integers^{+} \continue
 &&
  \sum_{j=1}^{d} \vel_j(\ssp_{n}) \, \pde_j \PoincC(\ssp_{n}) \,>\, 0
\,.
\label{orientCond}
\eea
In this way the continuous time $t$ flow $
\flow{t}{\ssp}$ is reduced to a discrete time $n$ sequence
$\ssp_n$ of successive {\em oriented} trajectory traversals of
$\PoincS$.


With a sufficiently clever choice of a Poincar\'e section
or a set of sections,
any orbit of interest intersects a section.

\section{Newton method for flows}
\label{s-POs-flows}
% Predrag extracted							05mar2011
% from ChaosBook.org \Chapter{cycles}{Fixed points, and ...}
% Predrag edits  							26sep2008

% \authorRPPC

{\bf Predrag 2011-03-05}
{This appendix is copied from ChaosBook.org - the edits here will
eventually be returned back to ChaosBook.org. Please keep ChaosBook
formatting throughout this chapter. Thanks!}

\noindent
For a continuous time flow the periodic orbit Floquet
multiplier
% \refeq{MargEigParall}
along the flow direction of
necessity equals unity; the separation of any two points along
a cycle remains unchanged after a completion of the cycle.
% \toSect{s:MargEigs}
More unit Floquet multipliers arise if the
flow satisfies conservation laws, such as the symplectic
invariance for Hamiltonian flows, or the dynamics is equivariant
under a continuous symmetry transformation.
% \toSect{s:StabRpo}
% \index{Newton method!flows}

Let us apply the Newton method of
% \refeq{NewtIt}
to search for
periodic orbits with unit Floquet multipliers,
starting with the case of a \emph{continuous time
flow}. Assume that the periodic orbit condition
% \refeq{e:periodic}
holds for $\ssp+\Delta\ssp$ and
$\period{}+\Delta t$, with the initial guesses $\ssp$ and
$\period{}$ close to the desired solution, \ie, with
$|\Delta\ssp|$, $\Delta t$ small. The Newton setup
% \refeq{NewtIt}
\bea
0 &=& \ssp+\Delta\ssp - f^{\period{}+\Delta t}(\ssp+\Delta\ssp)
  \continue
  &\approx&
\ssp - f^{\period{}}(\ssp) + (1 - \jMps(\ssp))\cdot\Delta\ssp
- \vel(f^{\period{}}(\ssp))\Delta t
\label{NewtonVarFlow}
\eea
suffers from two shortcomings. First, we now need to solve not
only for the periodic point $\ssp$, but for the period
$\period{}$ as well. Second, the marginal, unit Floquet
multiplier
% \refeq{MargEigParall}
along the flow direction
(arising from the time-translation invariance of a \po) renders
the factor $(1-\jMps)$ in
\refeq{NewtonVarFlow} % \refeq{NewtItMap}
non-invertible: if $\ssp$ is close to the solution,
$f^\period{}(\ssp)\approx \ssp$, then
$\jMps(\ssp)\cdot\vel(\ssp)=
\vel(f^\period{}(\ssp))\approx \vel(\ssp)$. If $\Delta\ssp$ is
parallel to the velocity vector, the derivative term
$(1-\jMps)\cdot\Delta\ssp\approx 0$, and it becomes harder to invert
$(1-\jMps)$ as the iterations approach the solution.

As a periodic orbit $p$ is a 1\dmn\ set of points invariant
under dynamics, Newton guess is not improved by picking
$\Delta\ssp$ such that the new point lies on the orbit
of the initial one, so we need to constrain the variation
$\Delta\ssp$ to directions transverse to the flow, by requiring,
for example, that
\PC{In remark explain that Davidchack inverts it anyway,
    without a constraint}
\beq
    \vel(\ssp) \cdot \Delta\ssp = 0
\,.
\ee{locTransvVar}
Combining this constraint with the variational condition
\refeq{NewtonVarFlow} we obtain a Newton setup for flows, best
displayed in the matrix form:
	\PC{bottom left corner is dimensionally wrong?
		it should be $1/[\vel]$?}
\beq
\MatrixII{1-\jMps(\ssp)}{\vel(\ssp)}{\vel(\ssp)}{0}
\left(\begin{array}{c}
    \Delta\ssp \\
    \Delta t
  \end{array}\right) =-
\left(\begin{array}{c}
    \ssp-f(\ssp)\\
    0
  \end{array}\right)
\ee{e:veq-1}
This illustrates the general strategy for determining \po s in
presence of continuous symmetries - for each symmetry, break
the invariance by a constraint, and compute the value of the
corresponding continuous parameter (here the period
$\period{}$) by iterating the enlarged set of Newton equations.
Constraining the variations to transverse ones thus fixes both
of Newton's shortcomings: it breaks the time-translation
invariance, and the period $\period{}$ can be read off once the
fixed point has been found (hence we omit the superscript
in $f{}^\period{}$ for the remainder of this discussion).

More generally, the Poincar\'e surface of section technique
% of \refsect{s:PoincSect}
turns the periodic orbit search into a
fixed point search on a suitably defined surface of section,
with a neighboring point variation $\Delta\ssp$ with respect to
a reference point  $\ssp$ constrained to \emph{stay} on the
surface manifold
%  \refeq{PoincU},
\beq
\PoincC( \ssp+\Delta\ssp) = \PoincC (\ssp) =0
\,.
\ee{e:pscond}
The price to pay are constraints imposed by the section: in
order to \emph{stay} on the surface, arbitrary variation
$\Delta\ssp$ is not allowed.
