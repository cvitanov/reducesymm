\svnkwsave{$RepoFile: lyapunov/dailyBlog.tex $}
\svnidlong {$HeadURL$}
{$LastChangedDate$}
{$LastChangedRevision$} {$LastChangedBy$}
\svnid{$Id$}


\chapter{Daily blog}
\label{c-DailyBlog}

\begin{bartlett}{
Je ne veux pas travailler\\
Je ne veux pas dejeuner\\
Je veux seulement oublier\\
Et puis je fume
            }
\bauthor{
\HREF{http://www.youtube.com/watch?v=MBoTRF2aK4s}
{China Forbes - Thomas M. Lauderdale (Pink Martini)}
    }
\end{bartlett}

\renewcommand{\ssp}{x}
\renewcommand{\vel}{\ensuremath{v}}   % state space velocity

\section{Method of slices for \cLf}

\begin{description}

\item[\Slice\ for \cLf.]

\item[2009-10-20 Predrag] Moved Lyapunov stuff to
    \refchap{s:LyapunovVec}


\item[2011-02-04 Evangelos]

\item[2010-04-07 Predrag]

So idea is to coarsely cover the nonlinear
strange attractor with a set of hyperplanes, as in \reffig{fig:Tesselate}.
For any pair, they intersect in a 'boundary' hyperplane, of one less dimension.
So our task is to, for a given strange attractor, pick a set of slice-fixing
points, such that each is approximately tangent to the strange attractor,
and the singularity hyperplanes are eliminated by requiring that they
lie either on the `wrong' side of the slice-slice intersection, or somewhere where
the strange attractor does not tread.

We need to make a global chart by deploying both linear slices and linear
Poincar\'e sections in neighborhoods of the most important (relative)
equilibria and/or (relative) periodic orbits (those are tricky, because
slice fixing points must lie in the full \statesp, and have no symmetry,
so most of the solutions we have are not good as they stand). This is the
periodic-orbit generalization of the idea of
{\statesp\ tessellation}
so dear to professional cyclists, \reffig{fig:Tesselate}.

% In FrCv11.tex replace by Tesselate.png
%%%%%%%%%%%%%%%%%%%%%%%%%%%%%%%%%%%%%%%%%%%%%%%%%%
\SFIG{f_1_08_1}
{}{
Smooth dynamics  (left frame) tesselated by the skeleton of
periodic points, together with their linearized neighborhoods,
(right frame).
Indicated are segments of two 1-cycles and a 2-cycle that
alternates between the neighborhoods of the two 1-cycles,
shadowing first one of the two 1-cycles, and then the other.
}{fig:Tesselate} %{Hyp} %{fig6} and {tr:fig6} in ChaosBook
%
%%%%%%%%%%%%%%%%%%%%%%%%%%%%%%%%%%%%%%%%%%%%%%%%%%
%

\item[2011-02-22 Predrag]
So far the gap between us and Kaz way of thinking is huge.
Maybe start EVO group meetings?

\item[2011-02-23 Evangelos]
I would love to be able to use EVO but I've
    failed, both in the office and at home. I could arrange to meet
    physically with Kazz et al at Paris or Saclay and meet with you
    through EVO?

\item[2011-02-23 Predrag]
There is also a new multi-video service from Skype (paid service), not
sure whether it is comparable to EVO (EVO is real good for projection of
your presentation on other people's seminar screen).

Anyway, go talk to him - they'll have to learn quite a bit before the
collaboration would lead to something, but it would be great if we can
get together - potentially a serious step forward in (confined)
turbulence

\end{description}

\renewcommand{\ssp}{a}
