\ifsvnmulti
 \svnkwsave{$RepoFile: lyapunov/Henon.tex $}
 \svnidlong {$HeadURL$}
 {$LastChangedDate$}
 {$LastChangedRevision$} {$LastChangedBy$}
 \svnid{$Id$}
\fi

\chapter{H\'enon  attractor}
\label{c-Henon}

This section of the blog deals specifically with the
H\'enon attractor nonhyperbolicities.

%
%%%%%%%%%%%%%%%%%%%%%%%%%%%%%%%%%%%%%%%%%%%%%%%%%%
\SFIG{gin99angle}
{}{
(a) Probability distribution of the angle between stable and
unstable manifold for the H\'enon map $x_{n+1} = 1 -1.4\,
x_n^2 + 0.3 x_{n-1}$, and the Lozi map $x_{n+1} = 1
-1.4\,|x_n| + 0.3 x_{n-1}$ (black line, rescaled by a factor
10). (b) Ignore this frame.
}{fig:gin99angle} %{Hyp}
%
%%%%%%%%%%%%%%%%%%%%%%%%%%%%%%%%%%%%%%%%%%%%%%%%%%
%
A dynamical system is said to be \emph{hyperbolic} if its
\statesp\ has no homoclinic tangencies, \ie, the stable and
unstable manifolds are everywhere transversal to each other.
Since covariant Lyapunov vectors correspond to the local
expanding/contracting directions, one can compute their
relative transversality and quantify the degree of hyperbolicity.
The knowledge of the covariant Lyapunov vectors allows testing hyperbolicity by
determining the angle between each pair $(j,k)$ of expanding
($j$) and contracting ($k$) directions
\[
\phi_n^{j,k} = \cos^{-1}(|{\bf v}_n^{(j)} \cdot{\bf v}_n^k|) \in [0,\pi/2]
\,.
\]
As a test, Ginelli \etal\rf{ginelli-2007-99} compute the
probability distribution $P(\phi)$ of $\phi_n^{1,2}$ for the
H\'enon and Lozi two-dimensional maps. Arbitrarily small
angles are found for the H\'enon map, while the distribution
is bounded away from zero in the Lozi map \reffig{fig:gin99angle},
consistent with the common belief that
only the Lozi map is hyperbolic \rf{CoLe84}.

\begin{description}

\item[2011-02-26 Predrag]
{\em Covariant Lyapunov vectors for rigid disk systems}
by Hadrien Bosetti, Harald A. Posch\rf{BoPo10}. They say:

``We study the Lyapunov instability of a two-dimensional hard disk system
in a rectangular box with periodic boundary conditions. The system is
large enough to allow the formation of Lyapunov modes parallel to the $x$
axis of the box. The Oseledec splitting into covariant subspaces of the
tangent space is considered by computing the full set of covariant
perturbation vectors co-moving with the flow in tangent-space. These
vectors are shown to be transversal, but generally not orthogonal to each
other. Only the angle between covariant vectors associated with immediate
adjacent Lyapunov exponents in the Lyapunov spectrum may become small,
but the probability of this angle to vanish approaches zero. The stable
and unstable manifolds are transverse to each other and the system is
hyperbolic.''

%
%%%%%%%%%%%%%%%%%%%%%%%%%%%%%%%%%%%%%%%%%%%%%%%%%%
\SFIG{BoPo10-Fig1small} {}{
The H\'enon attractor (black line) and a finite-length approximation of
its stable manifold (dotted line) are shown. The red vectors are the
covariant vectors at the phase point 0 as explained in the main text. The
blue vectors are Gram-Schmidt vectors.
}{BoPo10-Fig1}
%%%%%%%%%%%%%%%%%%%%%%%%%%%%%%%%%%%%%%%%%%%%%%%%%%
%
I particularly like their H\'enon attractor illustration,
\reffig{BoPo10-Fig1}. [Might use their BoPo10-Fig1.eps in ChaosBook.org;
if so, ask for permission.]

\item[2011-07-08 Predrag]
Hiroki Takahasi, Miki U. Kobayashi, Kazuyuki Aihara,
\emph{How horseshoes are destroyed and what comes afterwards}:

``
We investigate the dynamics of strongly dissipative H\'enon maps at the
first bifurcation parameter at which the uniform hyperbolicity is
destroyed by the formation of tangencies inside the limit set. In a
parameter interval of transition from horseshoes to chaotic attractors,
we prove that the relative frequency of chaotic transient tends to one as
the Jacobian tends to zero. We also present numerical results which
support the conjecture of Lai Y-C, Grebogi C, Yorke J.A., Kan I (1993
Nonlinearity 6 779-797) on the frequency of non-hyperbolic chaotic
transient.
''

\item[2011-07-08 Predrag]
A mathematical paper, perhaps not directly relevant to the Lyapunov
project.
Hiroki Takahasi,
\emph{Prevalent dynamics at the first bifurcation of the H\'enon map},
\arXiv{1011.4200}:

``
We study the dynamics of strongly dissipative H\'enon maps, around the
first bifurcation parameter a* at which the uniform hyperbolicity is
destroyed by the formation of tangencies inside the limit set. We prove
that a* is a full Lebesgue density point of the set of parameters for
which Lebesgue almost every initial point diverges to infinity under
positive iteration. A key ingredient is that a* corresponds to
``non-recurrence of every critical point,'' reminiscent of Misiurewicz
parameters in one-dimensional dynamics. Adapting on the one hand
Benedicks-Carleson's parameter exclusion argument, we construct a set
of ``good parameters'' having a* as a full density point. Adapting
Benedicks-Viana's volume control argument on the other, we analyze
Lebesgue typical dynamics corresponding to these good parameters.
''

\item[2011-06-30 Predrag] \arXiv{1106.4929},
\emph{Simulating rare events in dynamical processes},
by Cristian Giardina, Jorge Kurchan, Vivien Lecomte,
and Julien Tailleur\rf{GiKuLeTa11}. They say:

``
Untypical, rare trajectories of dynamical systems are important: they are
often the paths for chemical reactions, the haven of (relative) stability
of planetary systems, the rogue waves that are detected in oil platforms,
the structures that are responsible for intermittency in a turbulent
liquid, the active regions that allow a supercooled liquid to flow...
Simulating them in an efficient, accelerated way, is in fact quite
simple.
''

This method is of interest to us because I suspect that the
`nonhyperbolicities' that Ginelli\etal\rf{YaTaGiChRa08} find are
localized to a few tangencies in the \statesp. The do not see this,
because they just compute without looking at the attractor, but for
example I expect this will be very clear if one takes a look at the
H\'enon attractor. The flat distribution in stable/unstable angles arise
presumably {\em only} from close passage to the 13-cycle nearly tangent
periodic point discussed in Artuso and Aurell and
Cvitanovi{\'{c}}\rf{AACII} and in ChaosBook version 13 (see exercise
17.1. ``How unstable is the H\'enon attractor?''; sect. 29.1 ``Fictitious
time relaxation''; Table 29.1).

\item[2011-08-25 Hugues]
Kazz and I have decided to try a number of things, including looking at
H\'enon and the famous period-13 orbit. We will first calculate the CLV
along the orbit, to see how "non-hyperbolic" it is, and then extract from
the chaotic trajectory near-tangencies, and compare their location to the
period-13 orbit. {\bf 2011-10-02 Predrag} This discussion is continued on
\refpage{fig:HenonNonHypPoints}.

\item[2011-06-30 Predrag 2 Kazz]
I suspect that the  non-hyperbolicities that
Ginelli\etal\rf{YaTaGiChRa08} find are localized to a few tangencies in
the \statesp. They do not see this, because they just compute without
looking at the attractor, but for example I expect this will be very
clear if one takes a look at the H\'enon attractor. The flat distribution
in stable/unstable angles arise presumably {\em only} from close passage
to the 13-cycle nearly tangent periodic point discussed in Artuso and
Aurell and Cvitanovi{\'{c}}\rf{AACII} and in ChaosBook version 13 (see
exercise 17.1. ``How unstable is the H\'enon attractor?''; sect. 29.1
``Fictitious time relaxation''; Table 29.1).

Kazz, can you color code the small angles while running your code on the
H\'enon attractor? Maybe the 13-cycle will just jump out...

\item[2011-10-02 Kazz]
Here is the first data on the H\'enon map.
\refFig{fig:HenonNonHypPoints}\,(a) shows the angle between the two
Floquet eigenvectors (computed as CLVs) of three \po s, one is hyperbolic
and the others are (almost) non-hyperbolic. Specifically, the former is
the third $\cl{}=10$ orbit $p_{10c} = \cycle{0011111101}$, and the latter
the two $\cl{}=13$ orbits $p_{13a} = \cycle{1110011101000}$ and $p_{13b}
= \cycle{1110011101001}$ in Table 29.1 in
\HREF{http://chaosbook.org/version13/paper.shtml\#relax}{Chaosbook.org},
p.~564, here reproduced as \reftab{t-biham2}. For notation, see
\refappe{s-SymbDynDefs}.

The angle is shown as a function of time over one period. As you
see, while the angle for the hyperbolic orbit is at least 0.4 (i.e.,
indeed hyperbolic), the angle for the $p_{13a}$, $p_{13b}$ orbits reaches 0.04
at $t=11$ (indeed almost non-hyperbolic).

% PC 2011-10-02: generated by Kazz
\begin{figure}
 (a)~\includegraphics[width=0.45\textwidth]{fig1-hyperbolicity}
 (b)~\includegraphics[width=0.45\textwidth]{fig2-nonhyp_points}
\caption{
(a)
(b)
[Kazz 2011-10-02].
}
\label{fig:HenonNonHypPoints}
\end{figure}

Here I use improved initial conditions for the \po s and the results did
not change.

\refFig{fig:HenonNonHypPoints}\,(b) shows, on the top of the H\'enon attractor,
all the points of the above three \po s (filled symbols, not the same ones
as \reffig{fig:HenonNonHypPoints}\,(a)) as well as the 50 most non-hyperbolic
points along a chaotic trajectory of length $10^6$ (plus symbols; most
non-hyperbolic means smallest angle here). We see that some of these
non-hyperbolic points are found very close to the non-hyperbolic \po s
(see inside the light blue rectangles), while none of the non-hyperbolic
points are close to the hyperbolic \po. However, we see that many
non-hyperbolic points are actually far enough from the two non-hyperbolic
\po s, suggesting that there are other non-hyperbolic \po s in this system.

\item[2011-10-03 Predrag] I do not remember other cycles being as
non-hyperbolic as these 13-cycles, but I should have a database of
thousands (?) of  H\'enon cycles somewhere, if you want to have a look...

\item[2011-10-06 Predrag]
Have a look at {\bf 2011-07-01 Predrag} post on Greene and Kim\rf{GreeKim87}
in \refchap{s:LyapunovVec}.

\SFIG{fig3-angle}
{}{
The relation between the non-hyperbolicity
of the chaotic trajectory and \po s. The color code indicates the angle
between the stable and unstable manifolds, and the symbols are the
positions of \po s (black circles = $p_{13a}$, non-hyperbolic; red squares =
$p_{13b}$, non-hyperbolic; green diamonds = $p_{10c}$, hyperbolic). It
shows that the degree of the non-hyperbolicity is a smooth function along
the attractor, and around the most non-hyperbolic point of  $p_{13a}$, $p_{13b}$
(near (0.8,0)) the chaotic trajectory is also
non-hyperbolic. The figure also shows that there are other non-hyperbolic
\po s that are not found, because there are blue regions close to none of
the cycle points of $p_{13a}$, $p_{13b}$ \po s.
(click \HREF{https://www.sugarsync.com/pf/D6368502_0823551_99453}{here} for
the humongous original)
}{fig3-angle}

% PC 2011-10-06: generated by Kazz
\begin{figure}
\includegraphics[width=0.95\textwidth]{fig4-VectorSimilarity}
\caption{
The similarity between the first (second) Lyapunov vectors of the
chaotic trajectory and an \po\ ($p_{13a}$ or $p_{10c}$). Specifically, the color
code indicates the absolute value of the dot product of the two vectors
to compare. Here, I choose one point of the \po s (most non-hyperbolic
point for $p_{13a}$ and most hyperbolic point for $p_{10c}$; shown by filled
symbols) and compare a \po\ vector of that point with the vector of the
chaotic trajectory at all the points along the trajectory. You see that
the two vectors are quite similar if the chaotic trajectory is close to
the reference point (filled symbol), regardless of the index and the
hyperbolicity of the \po s.
(click \HREF{https://www.sugarsync.com/pf/D6368502_0823551_99467}{here} for
the humongous original)
}
\label{fig4-VectorSimilarity}
\end{figure}


\item[2011-10-06 Kazz]
New, improved \reffig{fig3-angle} and \reffig{fig4-VectorSimilarity}.

\item[2011-10-06 Hugues]
It seems pretty clear to me that there exists (infinitely) many "almost
non-hyperbolic" \po s and all the more so than they have longer periods
(remember the plots "minimum angle vs period" of Kobayashi and Saiki).
But, now, how to find them? Would starting points given by points at
which the CLVs of the chaotic trajectory are almost tangent be useful?

\item[2011-10-06 Kazz]
There should be many \po s flowing like the chaotic trajectory. They
therefore have long periods and are non-hyperbolic (almost, always). But,
in my view, it would be interesting to decompose properties of the
chaotic trajectory into those of only a few number of \po s, whose period
is rather short and thus each of which covers only a local region of the
attractor. For the H\'enon map, we are still lacking such a minimal \po,
which accounts for the remaining non-hyperbolic points of the chaotic
trajectory.

\item[2011-10-06 Predrag]
Wow! This comment makes no sense, but it does smack of the famous
Japanese Heresy that Evangelos can explain. There is NO such thing -
instead of this there is perfectly well developed theory that says how
you use \po s and how many do you need to capture the hyperbolic parts of
the {\nws}. It's as elegant and systematic as Stat Mech and Quantum Field
Theory. Read \HREF{http://chaosbook.org/}{The Book}. But who reads books
nowadays? BTW, there is no need to attach prefix U to \po s; there are a
few or no stable orbits in chaotic dynamics, and exponentially many
unstable ones, it's some dumb Soviet style abbreviation that must have
come from Maryland or somewhere.

\item[2011-10-06 Predrag]
Yes, we used to call forward images of the primary H\'enon tangencies
`turnbacks' and such. My theory of
\HREF{http://www.cns.gatech.edu/~predrag/papers/preprints.html\#GeomChaos}{pruning
fronts} says that you only have to identify non-hyperbolicities on the
pruning front, the rest are just forward/backward images of it, and I
like to do it best by sequences of periodic orbits. Grassberger likes to
do it it by stable/unstable manifolds, but I hope that by now you agree
with me that the cycles are the way to go. The brilliant thing about the
pruning front is that you search for non-hyperbolicities systematically,
on a (fractal) line, rather than in the whole plane. This way you
systematically obtain the grammar of admissible itineraries for the
H\'enon-type maps. There are infinitely many nearly non-hyperbolic
cycles, but in practice at most one pair for each time step in the
period. The moment you find \emph{one} cycle that is stable, you have
proven that the H\'enon attractor is not a strange attractor, but rather
a strange repeller: there is roughly 50-50 chance that it is strange /
not strange for \emph{the} H\'enon parameter values. The moment
Grassberger heard my seminar he was able to compute all cycles up to
length 32 or so. Procaccia misunderstood what it was about and just
published it\rf{pre88top}, using partially my notes and adding 1/2
Procaccian gibberish to it, so that pretty much killed the joy of writing
it up for me, it has never been written up well. An attempt is in
\HREF{http://chaosbook.org/paper.shtml\#smale}{ChaosBook.org} chapter
{\em Stretch, fold, prune}.

\item[2011-10-07 Ruslan] Maybe this is not very relevant to your discussion, but the angle between stable and unstable manifolds in 2-D maps varies between $0$ and $\pi$.  The pictures should look something like this:
    
    \includegraphics[width=0.95\textwidth]{henon_su_angle}

    Homoclinic tangencies are places where the angle changes from $0$ to $\pi$.  Anyway, some time ago I have been looking at H\'enon and Ikeda maps trying to identify 'primary' tangencies, but couldn't do it for the latter based on the manifolds (a la Grassberger).  Predrag, are you saying that non-hyperbolicities on the pruning front uniquely define primary tangencies?  If yes, then I better go read this chapter...

\item[2011-10-06 Hugues]
About the apparent smoothness of the angle between stable and unstable
manifolds: can't this information already be seen using (finite-length)
approximations of the stable manifold, as in Posch et al, Fig~3.4 of the
Blog? It seems to me that one can infer from such data how the angle
varies all along the attractor (following a sheet) and in particular one
should be able to locate the main (near-)tangencies with good accuracy.

\item[2011-10-07 Ruslan] In fact, a good approximation to the angle can be found not just on the attractor, but extended across the whole plane by looking at the most contracting directions of the Jacobian matrix of the iterated forward and inverse maps.  This was done by Jaeger and Kantz [Physica D, v. 105, p. 79 (1997)].  The picture looks something like this (for just seven iterations each way):
    
    \includegraphics[width=0.95\textwidth]{henon_su_angle_plane}
    

\item[2011-10-06 Kazz]
Given that the CLVs are local linear approximations for the unstable and
stable manifolds and assuming that these manifolds do not depend on the
position on the chaotic attractor, this should provide the same
information as the angle between the CLVs. Then I think it's easier to
compute the CLV angle directly.

\item[2011-10-06 Predrag]
Bit more subtle than that - turnbacks on long cycles are very sharp and
very small, so things that look smooth are not smooth; they are dense,
and every place on the strange attractor. Turnbacks are visible as
singularities in natural measure (that's always a signature of
non-hyperbolicity, see Figure~16.6 in
\HREF{http://chaosbook.org/paper.shtml\#measure}{ChaosBook.org} chapter
{\em Transporting densities}. The good news is the pruning front theory:
you need to only identify (near-)tangencies on the smooth primary folds,
all the sick stuff then comes for free, by iterating the cycles that
bracket the tangency.

\item[2011-10-06 Kazz]
I have no experience of finding \po s, but I'm sure starting from
non-hyperbolic points is useful. The most primitive way for discrete-time
systems like H\'enon would be simply trying the Newton method for
arbitrarily chosen periods. If it converges to a point near-by, we get
it!

\item[2011-10-06 Predrag]
It's a solved problem - you read chapter
\HREF{http://chaosbook.org/paper.shtml\#cycles}{Fixed points, and how to
get them} first, implement it to get short cycles (or ask Evangelos to
hand you the code), and then you read chapter
\HREF{http://chaosbook.org/paper.shtml\#relax}{Relaxation for cyclists}
to learn how to find ALL cycles up to a given topological length, (or ask
Lan to hand you the code). Hyperbolic cycles are easy - the pain are the
non-hyperbolic ones, that's explained in chapter
\HREF{http://chaosbook.org/paper.shtml\#inter}{Intermittency}, but I do
not think this is useful to us at this point.

\item[2011-10-06 Kazz]
by the way I noticed that the definition of the H\'enon map in ChaosBook
\bea
    x_{n+1}&=&1-ax^2_n+b y_n
        \continue
    y_{n+1}&=& x_n
\label{eq2.1a}
\eea
is different from the ``standard'' one
\bea
    x_{n+1}&=&a - x_n^2 + b y_n
        \continue
    y_{n+1}&=& x_n
\label{eq2.1b}
\eea
but the used parameter values are the same ($a=1.4$, $b=0.3$). I didn't
realize it when I made my code and used ChaosBook's definition. I'd like
to know if you really used the former definition to obtain \po s, for
example, and if the nature of chaos does not change for both definitions.

This is why the shape and the position of the attractor is different
(compare the coordinates of the attractor in my figure and that of
Posch).

\item[2011-10-06 Predrag]
The standard definition is surely H\'enon\rf{henon} own \refeq{eq2.1a}.
Some people chose to rewrite it as \refeq{eq2.1b}, as that is closer to
Fateau' quadratic polynomial, but there is no persuasive reason to mess
with H\'enon's definition, and I find it disrespectful. I even had a bad
experience with that kind of nonsense myself. Some mathematicians
experienced discomfort when period doubling universality was discovered.
When they finally worked it out, did exaaactly what Feigenbaum and I did
to solve the equation and the light went on, convergence of our Newton
iteration became a ``theorem'', they permuted several Greek letters and
signs on my period-doubling universal equation, and since then they
attach names of random mathematicians to it. Not nice.

\item[2011-07-24 Predrag]
                                    \toCB
You might find Demidov's website\rf{DemChaos} helpful, his simulations
are instructive. He uses a different definition for parameters $a$ and
$b$ from H\'enon, but unfortunately uses the same letters. Now, that's in
a really bad taste. His definition is natural if one is interested in
Julia sets, but unfortunately is not the one H\'enon used, and I always
try to follow the foundational papers, rather than confusing everybody
with unnecessary parameter redefinitions. Demidov\rf{DemChaos}
non-H\'enon  parametrization is
\bea
    x_{n+1}' &=& a'+ {x'}{}^2_n + b' y_n'
        \continue
    y_{n+1}' &=& x_n'
\,.
\label{DemidHen}
\eea
(Note yet another screwy sign difference from the H\'enon convention)
Dividing through by $a'$ we get
\(
\frac{x_{n+1}'}{a'} = 1 + a'\left(\frac{x_n'}{a'}\right)^2 + b'\frac{y_n'}{a'}
\,,
\)
so the two parametrizations are related by:
\beq
x={x'}/{a'}
\,,\quad
y={y'}/{a'}
\,;\qquad
a=-{a'}
\,,\quad b= {b'}
\,.
\ee{DemidHenPar}
After a while the light goes on, and you realize that H\'enon was
thinking: in his convention unit square remains unit square, with $a$
telling you how stretched the horseshoe is, and $b$ how compressed it is.
In some other random convention the frame of the map scales with $a$,
which is stupid. Please do me a favor, and do it as in ChaosBook. I've
worked on H\'enon-type maps for years, and there is no reason not to use
the consistent, established notation.

%%%%%%%%%%%%%%%%%%%%%%%%%%%%%%%%%%%%%%%%%%%%%%%%%%%%%%%%%%%%%%%%%%
\SFIG{Demidov_a-6_b-1}
{}{
PC: The Smale backward-forward horseshoe generated by the
Demidov\rf{DemChaos} java applets for the H\'enon parameter values
$(a,b) = (6,-1)$.
    }{Fig:Demidov}
%%%%%%%%%%%%%%%%%%%%%%%%%%%%%%%%%%%%%%%%%%%%%%%%%%%%%%%%%%%%%%%%%


\item[2011-10-06 Predrag]
I can see where this is heading - it will be the same as the CNS kitchen,
grad students and me. I break down first, and then I do their dishes,
because I cannot stand the mess. The rational thing for us would be to
behave like this is 2011, and we just have a brief EVO meeting every few
days, instead of me transcribing your emails. This second I have 6419
emails I'm supposed to deal with (90\% of emails I deal with immediately;
these are emails that require 10 mmin or more of work). As this project
is dear to me, I'm cleaning up after you, but you are not using me in a
good way - you could get more mileage out of my experience than typing.

I never had a touch typing course, but I had some theoretical physics
courses.



\end{description}
