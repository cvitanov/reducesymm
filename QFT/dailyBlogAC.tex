% reducesymm/QFT/dailyBlogAC.tex
% Predrag  created              Jan 31 2018

\section{Chen QFT self-study notes}
\label{c-dailyBlogAC}

%\noindent
%Andy Chen \\final term paper for Spring 2018 QFT self-study course.
%\\

\begin{description}
\item[2013-11-25  Predrag to Andy] Created \refsect{c-dailyBlogAC} for
    you to write your QFT study notes in.

\paragraph{The goal.}
    Take a narrow path, learn enough QFT (and Green's functions and such, but
    no more) to be able to digest \refsect{sect:worldline}~{\em Worldline
    formalism} and check James \refsect{sect:magMomWorldline}~{\em Electron
    magnetic moment in worldline formalism} calculations.


\end{description}


\section{Gauge theories}

 \paragraph{What is a gauge theory?}
According to\rf{belot2002} there are two types of theories
that can be called \lq gauge theories\rq, the Yang-Mills theories and
constrained Hamiltonian theories.

\paragraph{Summary}
In order to get correct predictions from non-Abelian field theories,
which are susceptible to large number of gauge copies, we need to choose
a representative of each gauge orbit.

%\newpage %%%%%%%%%%%%%%%%%%%%%%%%%%%%%%%%%%%%%%%%%%%%%%%%
\printbibliography[heading=subbibintoc,title={References}]


\renewcommand{\ssp}{a}
