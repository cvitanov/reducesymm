% reducesymm/QFT/Gribov.tex
% Predrag  created              Sep 2 2013
% continues siminos/blog/dailyBlog.tex as of that date

\chapter{Gribov ambiguity}
\label{c-Gribov}


% Clipped from \refref{atlas12}:
% \\
Symmetry reduction in dynamics (including classical field theories such
as the \NSe) closely parallels the reduction of gauge symmetry in
quantum field theories. There, the freedom of choosing moving frames
 is called `gauge freedom' and a
particular prescription for choosing a representative from each gauge
orbit is called `gauge fixing'. Just like the slice hyperplanes
may intersect a group orbit many times, a gauge
fixing submanifold may not intersect a gauge orbit, or it may intersect
it more than once (`Gribov ambiguity')\rf{Gribov77,VaZw12} In this
context a chart is called a `Gribov' or `fundamental modular' region and
its border is called a `Gribov horizon' (a convex manifold in the
space of gauge fields). The Gribov region is compact and bounded by the
Gribov horizon. Within a Gribov region the `Faddeev-Popov operator'
(analogue of the group orbit tangent vector) is strictly positive, while on
the Gribov horizon it has at least one vanishing eigenvalue.

% reducesymm/QFT/dailyBlogKS.tex
% Predrag  created              Sep 2 2013
% continues siminos/blog/dailyBlog.tex as of that date

\section{Sharma on Gribov ambiguity}
\label{c-dailyBlogKS}

\noindent
Kamal Sharma\\final term paper for Fall 2013 QFT course.
\\

\begin{description}
\item[2013-11-25  Predrag to Kamal] Created \refsect{c-dailyBlogKS} for
    you to write your QFT final paper.

%\item[2013-10-12  Kamal] I will be away during the month of January
%    for my wedding. I will be working from home and updating through
%    GitHub. Please let me know if you have anything you want me to
%    take care of. Further, it will be great for me if you could send
%    me the material for the QFT final project. As you might have
%    experienced, speed is an issue with my working on new stuff so I
%    want to start as early as possible to get something useful out of
%    it given the time constraints. Also the TA duties are killing me.
%    I am working on it and want to be productive fast.

\item[2013-10-17 Predrag to Kamal] I was thinking that it might be
    useful to learn about the \emph{`Gribov ambiguity'}. My notes are
    in \refchap{c-Gribov}. Have a look... The posts there summarize
    all my notes about the \emph{`Gribov ambiguity'}. I fear it is
    too difficult, starting with your current background in QFT, but
    it is one of directions in which our current symmetry-reduction
    work might develop in the future...

So, instead of the final exam in the Quantum Field Theory Course,
please study some of the references I have listed (or, better still,
find more accessible references by web searches) and write up here,
as you learn, about 10-15 pages of what you have learned. The
delivery deadline is  December 12, 2013 14:20pm.


\end{description}


\subsection{Gauge theories}
All modern theories trying to explain fundamental physics are field theories which describe the configuration of a field and its dynamics as the interaction between fields and their evolution in space and time. These fields may be scalar, vector or tensor fields and they transform under a gauge transformation accordingly to give different configurations of the field. However, for some of these configurations the physical observables, which are in a direct way the reality that we perceive/observe, do not change under gauge transformations. For example, in electromagnetism, the following transformations,
\begin{subequations}\begin{align}
 \bold{A} &\rightarrow \bold{A} + \nabla \psi  \\
 \phi &\rightarrow \phi - \frac{\partial \psi}{\partial t}\end{align} \end{subequations}
keep the observables $\bold{E}$ and $\bold{B}$ (electric and magnetic
fields respectively) unchanged. This is an example of a gauge
transformation.

\paragraph{What is gauge freedom and gauge fixing?}
The invariance of physically observables quantities with respect to a
gauge transformation implies that the system has redundant degrees of
freedom in field variables. All field configurations that transform into
one another through gauge transformations are physically equivalent and,
therefore, for correct predictions, should be counted as one. Gauge
fixing is the mathematical procedure of selecting an equivalence class
for each set of physically identical field configurations. A coherent and
consistent prescription of selecting the representative configurations
(also known as gauge fixing) out of all possible detailed configurations
is required to make any meaningful predictions using the given theory.


 \paragraph{What is a gauge theory?}
According to\rf{belot2002} there are two types of theories
that can be called \lq gauge theories\rq, the Yang-Mills theories and
constrained Hamiltonian theories. The Hamiltonian theories subsume the
Yang-Mills theories. Such theories have a common striking feature known
as gauge freedom. Gauge freedom is a fancy way of saying that the
theory has two kinds of variables -- physical and unphysical-- due to
which the initial value problem in such theories is ill-defined. This
means that a set of initial conditions does not uniquely determine the
evolution of all the dynamical variables of the theory. The set of
`physical' variables will evolve the same way but the `unphysical'
variables can evolve in infinite number of arbitrary ways thus allowing
infinite number of solutions for the same initial conditions. It is
possible to interpret classical gauge theories as deterministic only if
we consider the physical dynamical variables as the complete description
of physical reality. If two states differ only in their unphysical
variables, they represent the same physical configuration of the field.

Speaking in mathematical dialect, a gauge theory is a field theory in
which the Lagrangian is invariant under a Lie group of continuous
transformations. Such a group of transformations is called a symmetry
group of the theory. A gauge field is a vector field associated with the
generators of the symmetry group and the particles that arise due to
quantization of the gauge field are called gauge bosons.

\paragraph{Consequences of gauge symmetry}
The presence of gauge symmetry has significant consequences on the
results of the theory. Let us consider a free non-relativistic particle
in one dimensional space. The Hamiltonian of such a system is given just
by the kinetic energy of the particle $ H= -\frac{1}{2}
\frac{d^{2}}{dx^{2}} $. The energy eigenvalue spectrum of this problem is
continuous in the absence of symmetry. In the case of a periodic boundary
condition when the points $x$ and $ x+pL, p \epsilon I $ are physically
identified with each other, the wavefunction has a gauge symmetry
$\Psi(x)=\Psi(x+pL)$. The spectrum now becomes discrete and this is how
gauge symmetry affects the physical configuration space\rf{VaZw12}. This
is a simple example which illustrates that a gauge theory must obey some
constraints that identify the physically identical configurations and
that it makes the spectrum more restrictive.

\paragraph{Abelian and non-Abelian gauge theories}
The continuous symmetry operations on the Lagrangian which keep the
action invariant constitute a Lie group. The generators of infinitesimal
transformations of such a Lie group define the algebra of such a symmetry
group. If the generators of the gauge symmetry group commute then the
theory is said to be Abelian, otherwise non-Abelian.

\paragraph{Gribov copies}
In 1978, Gribov\rf{Gribov77},\rf{VaZw12} showed that for a non-Abelian
theory, for example SU(2) and SU(3), the local gauge group imposes more
stringent constraints than it does on an Abelian gauge theory. The
\emph{transversality condition} $ \partial \cdot A=0$ fixes the gauge
uniquely for Abelian gauge theories but in the case of non-Abelian gauge
theories, there exist distinct phase space configurations $A$ and $A'$
related by a finite gauge transformations $A' = U A$ such that
$\partial\cdot A=0$ and $\partial\cdot A'=0$ where $A \neq A'$. These
distinct configurations are called Gribov copies and in a non-Abelian
gauge theory they have an additional constraint of being physically
identical. The subspace of the full state space that contains only
physically distinct configurations of the field is called a fundamental
modular region and is free of any Gribov copies. In 1978,
Singer\rf{Singer78} showed that in non-Abelian theories Gribov copies are
unavoidable and that the physical configuration space is topologically
non-trivial. On the other hand, for an Abelian theory the physical
configuration space is a linear vector space.

\paragraph{Quantum ChromoDynamics and confinement}
Quantum ChromoDynamics is a theory which describes the strong
interactions at sub-atomic level. Strong interaction is the force between
quarks and gluons. At very high energies these sub-atomic particles are
asymptotically free which means that they behave like free particles. But
these particles have not been observed. This is because in low energy
conditions they interact with each other and form bound states called
hadrons, such as proton and neutron. This phenomenon is called
\emph{confinement}.

In spite of QCD being qualitatively similar to QED, the problem of
confinement is not very well understood. In QED, the method of
perturbative expansion using Feynman diagrams has proved very successful
because the small value of the QED coupling constant makes the
contribution of higher terms more and more negligible. In QCD the value
of the coupling constant is a function of energy and becomes larger as
the energy is lowered. This phenomena called `infrared slavery" is
responsible for the failure of perturbation theory for low energy
phenomena. For understanding the low energy behaviour of the theory,
various non-perturbative methods have been developed to describe
confinement. Different methods work well in different conditions and so
QCD is a patchwork of different methods that work in different
conditions.

\paragraph{Dynamical implications of non-Abelian nature of gauge symmetry
group}
The existence of Gribov copies and suppression of infrared modes due to
their closeness to Gribov horizon leads to an interesting feature: The
calculation of the gluon propagator under a non-Abelian theory leads to
expulsion of the gluon from the physical spectrum of the solution. This
is seen as non-existence of a physical pole in the gluon propagator.

\paragraph{Gauge orbit and physical configuration}
Each physical configuration of the field $ A_{phys} $ is associated with
a corresponding gauge orbit which is collection of all physically
identical configurations. The physicalconfiguration space is the space of
all gauge orbits modulo the group of local gauge transformations
$\mathcal{G}={U}$, $$P=\mathcal{A}/\mathcal{G}.$$


\subsection{Yang-Mills theory}
As QCD is a specific case of a general Yang-Mills theory, it is a good
example of the general theory. Consider the compact group $SU(N)$ of
[$N\!\times\!N$]
unitary matrices $U$ of determinant one. These matrices can be expressed
as
\[
U = \exp(-ig\theta_{a}X_{a})
\,,
\]
where $X^{a}$ are the generators of
SU(N) group and satisfy commutation relations
\[
[X^{a},X^{b}]=i f_{abc}X^{c}
\,.
\]
These generators are defined to be hermitian and normalizable
as follows: $$X^{\dagger}=X,$$ $$Tr[X_{a}X_{b}]=\frac{\delta_{ab}}{2}.$$
Now, the generators $X_{a}$ belong to the adjoint representation of the
group $SU(N)$, i.e. $$U X_{a}U^{\dagger}=X_{b} (D^{A})_{ba},$$ with
$(D^{A}(X_{a}))_{bc}=-if_{abc}$. Here $f^{abc}$ are the structure
constants of $SU(N)$ and have the following property, $$f^{abc}f^{dbc}=N
\delta^{ad}.$$

We can construct a Lagrangian which by design would be invariant under
the above defined $SU(N)$ group. The Yang-Mills action for this
Lagrangian would be
$$S_{YM}=\int d^{4}x \frac{1}{2} Tr F_{\mu\nu}F_{\mu\nu},$$
whereby $F_{\mu\nu}$ is the field strength
$$ F_{\mu\nu}=\partial_{\mu}A_{\nu}-\partial_{\nu}A_{\mu}-ig[A_{\mu},A_{\nu}],$$
and $A_{\mu}$ are the gluon fields that belong to adjoint representation
of $SU(N)$ symmetry, i.e.
$$A_{\mu}=A^{a}_{\mu}X^{a}.$$
The field strength is given by
$$F_{\mu\nu}=\partial_{\mu}A^{a}_{\mu}+gf_{akl}A^{k}_{\mu}A^{l}_{\nu}.$$
$A_{\mu}$ under the $SU(N)$ symmetry transforms as
$$ A'_{\mu}=UA_{\mu}U^{\dagger}-\frac{i}{g}(\partial_{\mu}U)U^{\dagger}.$$
We find that
$$F'_{\mu\nu}=U F_{\mu\nu}U^{\dagger},$$
and can now see that Yang-Mills action is invariant under $SU(N)$ symmetry.
The infinitesimal transformations can thus be written as
$$\delta A^{a}_{\mu}=-D^{ab}_{\mu}\theta^{b},$$
with $D^{ab}_{\mu}$ the covariant derivative in the adjoint representation
$$ D^{ab}_{\mu}=\partial_{\mu}\delta^{ab}-g f^{abc}A^{c}_{\mu}.$$
There is also a matter part of the action but we will work with pure Yang-Mills action.

\paragraph{Faddeev-Popov Ghosts}
Faddeev-Popov ghosts or ghost fields are additional fields which are
introduced into gauge field theories to maintain the consistency of path
integral formulation. In order for the quantum field theories to deliver
unambiguous and sensible results we need to avoid overcounting of Feynman
diagrams that correspond to physically equivalent processes. In a gauge
field theory, each physical configuration has infinite number of full
state space configurations all of which lie on a gauge orbit. Selecting a
representative configuration from this equivalence class is required in
order for path integral method to work. But usually there is no such
prescription of selecting such a representative configurations. However,
it is possible to modify the action by adding extra terms called
\emph{ghost-fields} that break gauge symmetry. In general, ghost fields
can add or break gauge symmetry in a field theory. This method is called
Faddeev-Popov procedure. Ghost fields are mathematical tools and
represent virtual particles in Feynman diagrams. They are also essential
for unitarity.

\paragraph{Faddeev-Popov Quantization}
To understand the Gribov problem we first need to have a look at
Faddeev-Popov quantization \PCedit{(reference 53)}. The Yang-Mills action
 is given by
\[
S_{YM}= \frac{1}{4}\int d^{d}x
F^{a}_{\mu\nu}F^{a}_{\mu\nu}
\,.
\]
Our naive assumption that the
generating functional $Z(J)$ would be given by $$Z(J)=\int
[dA]exp[-S_{YM}+\int dx J^{a}_{\mu}A^{a}_{\mu}].$$ But this functional is
not well defined. We can look at the quadratic part of the action,
$$Z(J)_{quadr}=\int[dA] exp[-\frac{1}{4}\int dx
(\partial_{\mu}A_{\nu}(x)-\partial_{\nu}A_{\mu}(x))^{2}+\int dx
J^{a}_{\mu}(x)A^{a}_{\mu}(x)$$ $$ =\int[dA] exp[-\frac{1}{2}\int dx dy
A^{a}_{\nu}(x)[\delta^{ab} \delta (x-y)(\partial^{2}
\delta_{\mu\nu}-\partial_{\mu}\partial_{\nu})A^{b}_{\mu}(y))+\int dx
J^{a}_{\mu}(x)A^{a}_{\mu}(x) $$ which after a gaussian integration gives
$$ Z(J)_{quadr}=(detA)^{-\frac{1}{2}} \int[dA] e^{-\frac{1}{2}\int dx dy
J^{a}_{\nu}(x) A_{\mu\nu}(x,y)^{-1} J^{a}_{\mu}(y)},$$ with
$A_{\mu\nu}(x,y)=
\delta(x-y)(\partial^{2}\delta{\mu\nu}-\partial_{\mu}\partial_{\nu})$,
is ill defined because $A_{\mu\nu}(x,y)$ is not invertible. There is
something wrong with the generating functional.

Following the derivation of gauge fixed action given in
\PCedit{(ref 16)}:
\[
S=S_{YM}+\int dx (\bar{c}\partial_{\mu}D^{ab}_{\mu}c^{b}
 -\frac{1}{2 \alpha}(\partial_{\mu}A^{a}_{\mu})^{2})
\,.
\]
If we take the limit $\alpha \rightarrow 0$, we have the Landau Gauge
which stays a fixed point under normalization. If $\alpha=1$, we have the
Feynman gauge in which the gluon propagator is the simplest.

\paragraph{What is BRST symmetry?}
Fixing the gauge cause the local gauge symmetry to break. However, after
fixing the gauge, a new symmetry called the BRST symmetry appears which
is basically the symmetry of the ghost fields. For example, inserting
a \emph{b-field}
\[
S=S_{YM} + \int d^{d}x (
b^{a} \partial_{\mu} A_{\mu}
+\alpha \frac{b^{a})^2}{2}
+ \bar{c}^{a}\partial_{\mu}D^{ab}_{\mu}c^{b}
                    )
\,,
\]
where Z(J) is now given as
$$Z(J)=\int[dA][dc][d\bar{c}][db]
e^{[-S+\int dx J^{a}_{\mu{a}=0}A^{a}_{\mu}]}.$$
Here \emph{b} is the
bosonic field. The action for this theory has a new symmetry called the
BRST symmetry, $$sS=0,$$ with $$s A^{a}_{\mu}=-(D_{\mu}c)^{a},
sc^{a}=\frac{1}{2}gf^{abc}c^{b}c^{c},$$\\ $$s\bar{c}^{a}=b^{a}, sb^{a}=0
$$ \\
$$s\bar{\psi}_{\alpha}=-i g c^{a} (X^{a})^{ij} \psi^{j}_{\alpha}, s\bar{\psi}^{i}_{\alpha}=-i g \psi^{j}_{\alpha} c^{a} (X^{a})^{ji}.$$
BRST symmetry guarantees that Yang-Mills theory is unitary in
perturbation theory. Introduction of BRST symmetry introduces extra
particles called ghost particles $c$ and $\bar{c}$. Like other ghost
particles, these particles too violate spin-statistics thorem.

\subsection{Gribov ambiguity}

For any kind of gauge orbit the gauge fixing condition might have one, more or no solutions, i.e. the slice might intersect the gauge orbit once, more than once or never. Consider two Gribov copies $A_{\mu}$ and $A'_{\mu}$ related by a gauge transformation $$ A'_{\mu}=U A_{\mu} U^{\dagger}-\frac{i}{g}(\partial_{\mu}U)U^{\dagger},$$ which obviously satisfy the transversality condition $$\partial_{\mu}A_{\mu}=0 \ \partial_{\mu}A'_{\mu}=0. $$ These equations when combined and expanded to first order gives $$ -\partial_{\mu}(\partial_{\mu}\alpha+i g [\alpha,\partial_{\mu}])=0 $$ which is equivalent to $$-\partial_{\mu}D_{\mu} \alpha  =0.$$ This means that the relevant Gribov copies are in a space orthogonal to the trivial null space. The transversality condition, $\partial_{\mu}A_{\mu}$, also implies that Faddeev-Popov operator is Hermitian, $$-\partial_{\mu}D_{\mu}=\partial_{\mu}D_{\mu}.$$ Thus existence of Gribov copies are connected to zero eigenvalues of the Faddeev-Popov operator.

\paragraph{Important observation}
For small $A_{\mu}$, the equation reduces to the eigenvalue equation $$-\partial^{2}_{\mu}\psi=\epsilon \psi,$$ has positive eigenvalues but this cannot be gauranteed for large $A_{\mu}$. This means that for large $A_{\mu}$ the eigenvalues of Fadeev-Popov operator are zero.


\paragraph{Possible Solutions?}
\textbf{Gribov Region and Gribov Horizon}\\
We need to improve the gauge fixing for non-Abelian theories. This can
be done by finding the Gribov region $\Omega$ which is defined as
subspaces with positive eigenvalues of Faddeev-Popov operator, $$\Omega
= {A^{a}_{\mu},\partial_{\mu}A^{a}_{\mu}=0, \mathcal{M}^{ab}>0},$$
where $\mathcal{M}$ is the Faddeev-Popov operator,
$$\mathcal{M}^{ab}(x,y)=-\partial_{\mu}D^{ab}_{\mu}\delta(x-y).$$ This is
the region which obeys Landau gauge and where the FP operator is positive
definite. The border of the Gribov region is the manifold where the first
eigenvalue of the FP operator becomes zero. This is known as Gribov
horizon. The eigenvalues become negative on the other side of the Gribov
horizon.

Another way of choosing a Gribov region is to select those points on the gauge orbit which have minimum $A^{2}$. This definition also agrees with our previous definiton on Gribov region.

\paragraph{Properties of the Gribov region}
\begin{enumerate}
  \item
It is important that each group orbit passes through the Gribov region
because we want to take into account all possible physical
configurations. Gribov showed that for every configuration
infinitesimally close to the Gribov horizon, there exists a Gribov copy
on the other side of the horizon infinitesimally close the horizon. It
has been rigorously proved that every gauge orbit passes through the
Gribov region.
  \item
Gribov region is a convex manifold.\PCedit{(reference 12)}
  \item
Gribov region is bounded in every direction.
\end{enumerate}
Unfortunately, despite of all the nice properties, it has been shown
\PCedit{(reference 86)} that Gribov region still contains Gribov copies.

\paragraph{Another possible solution: Fundamental Modular Region}
Now when even the Gribov region has Gribov copies, let us define a
fundamental modular region as a more restrictive subspace of all the
configurations which have all absolute minima of the functional. We
shall select only the configurations closest to the region. This is
called the fundamental modular region or the minimal Landau gauge.

\paragraph{Properties of FMR}
\begin{enumerate}
  \item
All gauge orbits intersect with FMR
  \item
$ A_{\mu}=0 $ belongs to FMR as 0 is the smallest norm.
  \item
FMR is convex and bounded in every direction.
  \item
The boundary of $\Lambda$, $\delta \Lambda$ has some points common with
the Gribov horizon.
  \item Gribov copies exist on the boundary.
\end{enumerate}

\paragraph{Other attempts at gauge fixing}
Singer showed that suitable regularity conditions at infinity does not
leave any continuous gauge choices. This means that there is no unique
representative of the gauge orbit that is continuous in the space of
gauge orbits. A gauge free of Gribov copies is a singular gauge and very
difficult to handle in computations and also violates Lorentz
invariance. In 2005, Ghiotti, Kalloniatis, and Williams tried to
improve the Fadeev-Popov gauge fixing by including the determinant into
the action but in such a method the number of Gribov copies is not
accounted for and no further calculations have been done along these
lines. Slavnov, in 2008 and 2010, pointed out that if we do not take
the absolute value of the determinant of the Faddeev-Popov operator and
integrate over all Gribov copies, their effects will cancel out. The
disadvantage is that if we make approximations then the errors can get
very large. Stochastic quantization with stochastic gauge fixing
introduces a gauge-fixing force which is tangent to the gauge orbit. More
work needs to be done in this method.

\paragraph{Summary}
In order to get correct predictions from non-Abelian field theories,
which are susceptible to large number of gauge copies, we need to choose
a representative of each gauge orbit. Some new methods and mathematical
tricks have been explored but none have given a consistent recipe for
selection of the representatives of these equivalence classes. However,
if approximations are made in a clever manner, some of the methods can
give us practically usable results. There are also other methods, such as
the semi-classical approach by Gribov \PCedit{(reference ?)} which I have
not discussed here.



\renewcommand{\ssp}{a}


\section{Daily Gribov blog}
\label{s-DailyGBlog}



\begin{description}
\item[2013-12-07  Predrag]
here are a few things I would like to understand in dynamics language:
\begin{enumerate}
  \item All Gribov copies papers say that electromagnetism $U(1)$ is trivial;
  Landau gauge defines a slice normal to all gauge orbits. Why is it that we find
  \SOn{2}\ not trivial?
  \item All Gribov copies papers focus on the copy that includes the
  origin. Are they not automatically excluding the `non-perturbative' solutions,
  \ie\ neighborhoods of turbulent \rpo s that we care about?
  \item What does `gauge potential' connection $A^\mu$ mean in the dynamical
  systems language? How does it get promoted to a dynamical field?
  \item Landau gauge seems to correspond to template's group tangent vector.
  Does any commonly used gauge correspond to `co-moving frame'?
\end{enumerate}


\item[2013-11-23  Predrag] Moved all matters Gribov from
\texttt{siminos/blog/lit.tex} to here.

\item[2011-07-08 Predrag] from \refref{FrCv11}:                     \toCB
\\
% At this point
It is worth noting that imposing the global and fixed slice
%\refeq{PCsectQ}
is not the only way to separate equivariant dynamics into `group
dynamics' and `shape' dynamics\rf{BeTh04}. In modern mechanics and even
field theory (where elimination of group-directions is called
`gauge-fixing') it is natural to separate the flow {\em locally} into
group dynamics and a transverse, `horizontal'
flow\rf{Smale70I,AbrMars78}, by the `method of
connections'\rf{rowley_reduction_2003}. From our point of view, such
approaches are not useful, as they do not reduce the dynamics to a
lower-dimensional \reducedsp\ $\pS/\Group$.


\item[2012-05-20 Jeff Greensite] has written a book\rf{Greensite11} of
possible interest, \emph{An introduction to the confinement problem}.
I have put a GaTech eBook copy
\HREF{http://ChaosBook.org/library/Greensite11.pdf}{Greensite11.pdf} into
ChaosBook.org/library.

\HREF{http://en.wikipedia.org/wiki/Gribov_ambiguity}{Gribov ambiguity wiki}
(edits by Predrag):

Gauge fixing means choosing a representative from each gauge orbit. The
space of representatives is a submanifold and represents the gauge fixing
condition. Ideally, every gauge orbit will intersect this submanifold
once and only once. This is generally impossible globally, especially for
non-abelian gauge theories, because of topological obstructions and the
best that can be done is make this condition true locally. A gauge fixing
submanifold may not intersect a gauge orbit at all or it may intersect it
more than once. This is called a Gribov\rf{Gribov77} ambiguity.

Gribov ambiguities lead to a nonperturbative failure of the BRST
symmetry, among other things.

A way to resolve the problem of Gribov ambiguity is to restrict the
relevant functional integrals to a single \emph{Gribov region} or {\em
fundamental modular region} whose boundary is called a \emph{Gribov
horizon}.

{\em Gribov copies} play a crucial role in the infrared (IR) regime while
it can be neglected in the perturbative ultraviolet (UV)
regime\rf{Gribov77,Zwanz89,Zwanz93}. The restriction to the Gribov region
(defined in such a way that the Faddeev-Popov operator is strictly
positive) can be achieved by adding a nonlocal term, commonly known as
`horizon term', to the YM action\rf{Zwanz89,Zwanz93,Zwanz92}. This is a
nonlocal term in the 4-dimensional Euclidean space, written as an
integral over the `horizon function.'

Greensite: ``In non-Abelian theories, there are many gauge copies -
Gribov copies - that satisfy the Coulomb gauge condition. The Gribov
region is the space of all Gribov copies with positive Faddeev-Popov
eigenvalues. Configurations of the Gribov horizon have at least one FP
eigenvalue $\lambda =0$. What counts for confinement is the density of
eigenvalues $\rho(\lambda)$ near $\lambda =0$, and the `smoothness' of
these near-zero eigenvalues.

The Gribov horizon is a convex manifold in the space of gauge fields,
both in the continuum and on the lattice. The Gribov region, bounded by
that manifold, is compact.
''

Amusingly, they can find the first 200 eigenstates of the lattice
Faddeev-Popov operator on each time-slice of each lattice configuration
by the Arnoldi algorithm.

See also Heinzl\rf{Heinzl96,HeRuSch08}, as well as
\refrefs{RuSchVo02,MaScha94,vanBaal91,vanBaal97,DellAnZwan91,Cutkosky84,Singer78}.
Review of \refref{VaZw12} promises ``to give a pedagogic review of the
ideas of Gribov and the subsequent construction of the GZ action,
including many other topics related to the Gribov region.''


\item[2012-06-15 Daniel]
Wow! Kinda lost me here... How does this apply exactly? Is the
point to make an analogy between our slices and these Gribov regions?
All this QFT stuff kind of comes out of left field. Discussing
it with Predrag convinced me that it may be interesting to have this here
because there appear to be some strong parallels with what we're doing. I
may need a little massaging to make it more palatable, though.

\item[2012-06-15 Predrag]
Laufer and Orland\rf{LauOrl12} say in
{\em The geometry of {Yang-Mills} orbit space on the lattice}: ``
We find coordinates, the metric tensor, the inverse metric tensor and the
Laplace-Beltrami operator for the orbit space of Hamiltonian SU(2) gauge
theory on a finite, rectangular lattice. This is done using a complete
axial gauge fixing. The Gribov problem can be completely solved, with no
remaining gauge ambiguities.
''

\item[2012-06-15 Evangelos]
This seems very interesting and I have to read it. It will most probably
confuse people, if we are really addressing fluid-dynamicists.

{\bf [2012-06-15 Predrag]} I am trying to reach out to quantum filed
theorists, that's why these facts are made explicit here - otherwise they
think it is just plumbing, nothing to do with Fundamental Physics..

Maybe shorten it? For instance Faddeev-Popov operator is not introduced
here, so we might avoid reference to it altogether. In the last sentence,
is the Gribov region or the Gribov horizon a convex manifold?

\item[2010-09-28 ES: Faddeev-Popov ghosts]                    \toCB
(moved to here from froehlich/blog)
\\
From my random readings, supposedly making up for my inability to attend
colloquia in French: Faddeev in a
\HREF{http://www.scholarpedia.org/article/Faddeev-Popov_ghosts}{scholarpedia
article} discusses the difficulties in Yang-Mills quantization that led
him and Popov to introduce fictitious fields, now known as
\emph{Faddeev-Popov ghosts}. The problem was that of gauge fixing,
essentially of working on a slice. Faddeev says:

`` %\begin{ttfamily}
It was clear that the equivalence principle had to be taken into account.
In the functional integral framework, the equivalence principle implies
that one has to integrate over classes of gauge equivalent fields instead
of integrating over all fields $A_\mu^a$.

The choice of the representatives in the classes of equivalent fields is
realized by means of a gauge condition (gauge fixing), for instance,
\[
    \partial_{\mu} A_{\mu}^{a} = 0 .
\]
This condition defines a plane in the set of all fields, which is
intersected by the gauge orbits defined by
\[
    A_{\mu} = A_{\mu}^{a}t_{a} \to A_{\mu}^{\Omega}
            = \Omega A_{\mu} \Omega^{-1} + \partial_{\mu} \Omega \Omega^{-1} .
\]
In this context, the difference among abelian and non-abelian cases
becomes clear. In the abelian case, we take $\Omega(x) =
\exp{i\Lambda(x)}$ and a gauge orbit is defined by
\[
    A_{\mu} \to A_{\mu} + \partial_{\mu} \Lambda ,
\]
which is just a linear shift. Thus all the abelian orbits intersect the
gauge surface at the same angle.

In the non-abelian case, the gauge orbit equations are non-linear and the
intersection angle depends on the field parameterizing the orbit. It is
clear that this must be taken into account in the functional integral.
'' %\end{ttfamily}

See the wikipedia link to put this in the proper context. As an abelian
case Faddeev lists quantum electrodynamics where the group is $U(1)$ (the
same as in \cLe\ if we think in terms of complex variables). As a
non-abelian example he lists the Standard Model: $U(1)\times SU(2) \times
SU(3)$.

Faddeev relies on the group being connected to write group action in
exponential form, as Stefan does. $\On{2}$ in Kuramoto-Sivashinsky is
non-abelian so I am worried that there might be more work required. Are
all non-connected compact Lie groups non-abelian and vice versa?

The weight factor Faddeev and Popov introduced might be helpful in
trace-formulas for non-abelian groups.

\item[2012-06-15 Evangelos]
Something that still confuses me: Is `Faddeev-Popov operator' really the
gauge orbit tangent vector, or is it the group generator? {\bf
[2012-06-15 Predrag]} I think `group generator' would not have enough
information (why should it be position dependent?), so it should be
something like our group orbit tangent vector, except positivity of
eigenvalues sounds like a projection on a given direction across the
slice.

\item[2013-11-27  Predrag]
Williams\rf{Williams02} {\em {QCD}, gauge-fixing, and the {Gribov}
problem}seems like a nice reference.

 	Privatdozent Dr. habil.
 \HREF{http://www.tpi.uni-jena.de/~axm/} {Axel Maas} is cute - has a blog.
He also finds gauge invariance a \HREF{http://axelmaas.blogspot.com/2013/09/blessing-and-bane-redundancy.html}
{blessing and bane}.
In it, \arXiv{1309.1957}, a paper with one of the worst abstracts
ever written, is explained in gentle terms. Some snippets from the article:

``
The minimal Landau gauge\rf{Maas11se} is obtained by minimizing the Hilbert square norm
\beq
|| A ||^2 = \int | A_\mu^b(x) |^2 d^4x,
\eeq
to some local minimum (in general not an absolute minimum) with respect to local gauge transformations $g(x)$.  These act according to ${^g}A_\mu = g^{-1} A_\mu g + g^{-1} \p_\mu g$.  At a local minimum, the functional $F_A(g) \equiv ||{^g}A||^2$ is stationary and its second variation is positive.  It is well known that these two properties imply respectively that the Landau gauge (transversality) condition is satisfied, $\p \cdot A = 0$, and that the Faddeev-Popov operator is positive {\it i. e.} $(\omega, M(A) \omega) \geq 0$ for all $\omega$.  Here the Faddeev-Popov operator acts according to
\beq
\label{Macts}
M^{ac}(A) \omega^c = -  D_\mu^{ac}(A) \p_\mu \omega^c,
\eeq
where the gauge covariant derivative is defined by $D_\mu^{ac}(A) \omega^c = \p_\mu \omega^a + f^{abc} A_\mu^b \omega^c$, and the coupling constant has been absorbed into $A$.  Configurations $A$ that satisfy these two conditions are said to be in the (first) Gribov region\rf{Gribov77} which we designate by $\Omega$.  It is known\rf{Maas11se} that in general there are more than one local minimum of $F_A(g)$, and we do not specify which local minimum is achieved.  This gauge is realized numerically by minimizing a lattice analog of $F_A(g)$ by some algorithm, and the local minimum achieved is in general algorithm dependent. However, for all commonly employed algorithms this does not yield different expectation values, as they all are equivalent to an averaging over the first Gribov region\rf{Maas11se} {Maas:2013vd}.
''

{Maas:2013vd}
  A.~Maas,
  %``Local and global gauge-fixing,''
  PoS ConfinementX {\bf }, 034 (2012)
  [arXiv:1301.2965 [hep-th]].


Some snippets from
\HREF{http://www.tpi.uni-jena.de/~axm/southampton12.pdf} {Maas (2012)}:

QCD is a gauge theory
with a Lie-algebra-element valued gauge-field
$A_\mu = \Lg_aA_\mu^a$.
Gauge transformations change the gauge fields, but leave physics
invariant: gauge orbits. Set of all gauge-equivalent copies is a gauge
orbit. Gauge fields not unique. Quantities that change along a gauge
orbit are gauge-dependent. Gauge transformations do not change physics.

Gauge freedom: Choice of
internal coordinate system.

Path integral averages with flat weight over all gauge copies.
Gauge-fixing is the introduction of a non-flat weight, such that all
copy- independent quantities remain unchanged.
Two possibilities
\\
(1)
Averaging over all or some
copies - must be normalized
\\
(2)
Single out one copy as representative.

Any gauge fixing yields a residual set of gauge copies: Residual gauge
orbit.

Residual copies are related by a residual gauge symmetry.

Gribov copies if isolated.

An example of a Gribov copy: Instanton
[Maas, EPJC 06].

Gribov-Singer ambiguity.

Landau-gauge hypersurface can be decomposed into (Gribov) regions.

Innermost Gribov region is convex and bounded and pierced at least once
by all gauge orbits: Contains all information.

In perturbation theory: Local gauge condition in Landau gauge
$\partial^\mu A_\mu=0$
Sufficient for perturbation theory. Insufficient beyond perturbation theory.
There are gauge-equivalent configurations which obey the same
local gauge-condition: Gribov copies. There are no known local gauge conditions, which lead to a
unique gauge configuration.

\item[2012-06-14 Predrag]
\HREF{http://marcofrasca.wordpress.com/about/}{Frasca} is either insane
or just yet another ignorant field theorist: ``I have worked on almost
all fields of physics'' (???). I checked the publication list, and it is
no  L.D. Landau. But the blog is informative:

{\bf [2012-06-05 Frasca]} (edited by Predrag):

``The answer to the question of the mass gap in Yang-Mills theory has
made enormous progress mostly by the use of lattice computations and,
quite recently, with the support of theoretical analysis. Contrarily to
common wisdom, the most fruitful attack to this problem is using Green
functions. The reason why this was not a greatly appreciated approach
relies on the fact that Green functions are gauge dependent.
Nevertheless, they contain physical information that is gauge independent
and this is exactly what we are looking for: The mass gap.''

(Predrag: this I interpret in the spirit of Gutzwiller - the Green function
is coordinate dependent, but it's trace - which yield the spectrum - is
coordinate invariant.)

``\HREF{http://marcofrasca.wordpress.com/2011/01/28/the-saga-of-landau-gauge-progators-a-short-history/}
{The Saga of Landau-Gauge Propagators: A Short History}'' is a good read:

``We cannot forever ignore the low energy behavior of QCD as its complete
understanding could have impact at unexpected large scales.''


\item[2013-12-07  Predrag]
About the
\HREF{http://marcofrasca.wordpress.com/2011/02/03/the-gribov-obsession/}
{Gribov obsession/}: Frasca says ``Gribov copies should be renamed Gribov
obsession as I did. If you want a fine description of the problem you can
read appendix H in the paper by Alfred Actor\rf{Actor79}, or the
beautiful lectures by Silvio Sorella and Rodrigo Sobreiro\rf{SobSor05a}
{\em Introduction to the {Gribov} ambiguities} are pedagogical.
See also
\refref{SobSor05}.

``Gribov himself proposed a solution limiting solutions to the so called
first Gribov horizon as Gribov pointed out that the set of gauge orbits
can be subdivided in regions with the first one having the Fadeed-Popov
determinant with all positive eigenvalues and the next ones with
eigenvalues becoming zero and then going to negative. ''

Residual freedom: impose Landau gauge condition. It reduces configuration
space to a hypersurface. Leaves the non-perturbative gauge freedom to
choose between gauge-equivalent Gribov copies: Residual gauge orbit.

There is no possibility using a local condition to restrict to one
representative of the residual gauge orbit. So, leave the global color
symmetry unfixed.

Possibilities to complete the Landau gauge:
\\
1) Take the representative inside the fundamental
modular region
\\
2) Average over (part of) the residual gauge orbit
\\
3) Average over the complete residual orbit: Hirschfeld-like
\\
4)
Average over the elements in the first Gribov region:
Minimal Landau gauge.

A local minimum of $\int d^dx A_\mu^a A^\mu_a$
defines the first Gribov region.
The Gribov region $\Omega$ is  defined as the set of
field configurations fulfilling the Landau gauge condition and for
which the Faddeev-Popov operator,
\begin{eqnarray}
\mathcal{M}^{ab} &=&  -\partial_{\mu}\left( \p_{\mu} \delta^{ab}
    + g f^{acb} A^c_{\mu} \right) \;,
\end{eqnarray}
is strictly positive, namely
\begin{eqnarray}
\Omega &\equiv &\{ A^a_{\mu}, \; \partial_{\mu} A^a_{\mu}=0,
    \; \mathcal{M}^{ab}  >0\} \;.
\end{eqnarray}
The boundary, $\partial \Omega$, of the region $\Omega$ is called the
(first) Gribov horizon. However, this region $\Omega$ still
contains a number of Gribov copies and is therefore still ``larger'' than
the fundamental modular region (FMR), which is completely free of Gribov
copies. Unfortunately, it is unknown how to treat the FMR analytically
~~{Semenov,Dell'Antonio:1991xt,Dell'Antonio2,vanBaal:1991zw}.

Nele Vandersickel PhD thesis\rf{Vandersickel11} {\em A study of the
Gribov-Zwanziger action: from propagators to glueballs} might be
pedagogical. Her
\HREF{http://physik.uni-graz.at/~dk-user/talks/Vandersickel20100225.pdf}
{talk} gives a compact overview, might be useful for writing this up.
Chapter 3 of her thesis\rf{Vandersickel11}, \arXiv{1104.1315}, gives a
pedagogic overview of the Gribov-Zwanziger framework, not available yet
in the literature.

This 2010 \HREF{https://sites.google.com/site/facingqcd/} {workshop}
gives a pretty good idea of who is who in the field. I am curious what
Nele is doing today - \HREF{https://biblio.ugent.be/person/802000264158}
{her papers} stop in 2012.

Dudal~\etal\rf{DGSVV08} is a very impressive, huge paper.


\item[2016-02-17  Predrag]
van den Heuvel and van Baal\rf{vdHvB95} in {\em Dynamics on the {SU(2)}
fundamental domain} mean \slice\ when the say ``fundamental domain''.
They use our `nearest point on the orbit' slice condition.

``
Let ${\cal A}$ be the set of all gauge fields ${\cal A}  : S^3 \to
su(2)$. The \emph{physical configuration space} is the space of gauge
orbits ${\cal A}/\Group$. Let $\Group$ be the set of all gauge
transformations $g : S^3 \to SU(2)$. We would like to have a
\emph{fundamental domain}, that is, a set of gauge fields which is in
one-to-one correspondence with the physical configuration space.
''

``
As a representative of an orbit, we choose the gauge field that has
lowest norm.'' \ie, the same \slice\ condition as what we do.

This work is continued by Cucchieri\rf{Cucchieri98} {\em Numerical study
of the fundamental modular region in the minimal {Landau} gauge}, who
studies the so-called \emph{fundamental modular region}, a
region free of Gribov copies, in the minimal Landau gauge for pure $SU(2)$
lattice gauge theory.

\item[2017-05-26 Predrag]
Astorino, Canfora and Zanelli\rf{AsCaZa12}
{\em Gribov copies, BRST-exactness and Chern-Simons theory}

\item[2017-06-16 Predrag]
Capri \etal\rf{CDPFGMPS17}
{\em Nonperturbative aspects of {Euclidean Yang-Mills} theories in linear
covariant gauges: {Nielsen} identities and a {BRST}-invariant two-point
correlation function}

Capri \etal\rf{CFPSST17}
{\em Aspects of the refined {Gribov–Zwanziger} action in linear covariant gauges}

\item[2017-06-18 Predrag]
Zhou, Yan and Zhang\rf{ZhYaZh17}
{\em Quantization of {Yang–Mills} theory without the {Gribov} ambiguity}

\end{description}
\renewcommand{\ssp}{a}

%\newpage %%%%%%%%%%%%%%%%%%%%%%%%%%%%%%%%%%%%%%%%%%%%%%%%
\printbibliography[heading=subbibintoc,title={References}]
