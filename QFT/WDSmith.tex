% reducesymm/QFT/WDSmith.tex
% $Author$ $Date$
% Predrag  switched to github.com               jul  8 2013

\section{Warren D Smith correspondence}
\label{sect:WDSmith}
% former siminos/blog/dailyBlog.tex

\begin{description}
\item[2013-10-22 Nio Makiko to Warren]  nio@riken.jp  wrote:\\
  Thank you for useful information on knots theory and muon $(g-2)$.
 We will carefully examine them.

 As for numerical values of Set V, we are reluctant to open these values
 before publication.  Once we finish writing up the detailed paper and
 posting it on arXiv,  I will let you know.

\item[2013-10-23  Warren D Smith] warren.wds@gmail.com\\
well, I don't have any choice :).  However I point out perhaps you could send
me the numbers for $\alpha^4$ diagrams even if you want to wait on the
$\alpha^5$  diagrams?

Meanwhile I found out more interesting info which the textbooks do not know!

Cvitanovi\'c 1977 conjectured that gauge-invariant quenched diagram sets
always have small sum of diagram values, indeed small enough that he
thought quenched QED series
would always converge\rf{Cvit77b}
    \PC{`quenched' = no internal electron loops, \ie, $m_e \to \infty$
    approximation}.
He apparently originally thought this even
for unquenched but Lautrup\rf{Lautrup77} disproved it and Dyson had long had a
highly convincing argument\rf{Dyson52} (Reprinted p.255-6 in Selected papers
of Freeman Dyson with
commentary, AMS 1996)
-- and I now have even more convincing
arguments -- that generic QED series diverge for any nonzero $\alpha$.)
This was due to Cvitanovi\'c's empirical observation of amazingly huge
cancellations within gauge-invariant diagram sets, especially in
quenched QED.
    \PC{Suslov\rf{Suslov99} argues that t'Hooft and Lautrup
     renormalons are not significant? I have not studied it.}

However, what Cvitanovi\'c did not know, was that
\HREF{http://www.itp.ac.ru/en/persons/bogomolny-evgeny-borisovich/}
{Bogomolny} and Kubyshin 1981-1982
found estimates of the growth rate of generic QED series, and also for the
quenched QED subseries,
and indeed for QED for diagrams with k electron loops only (k fixed)\rf{BogKub81}.
I only found this out the other day, but I had long known about
estimates due to Dyson and others predicting divergence for QED
series.  It is just that this B+K work by permitting arbitrary fixed
k, directly addresses Cvitanovi\'c's quenched-convergence conjecture and
massively conflicts with it -- for quenched QED the prediction is that
at Nth order we get a quenched diagram sum growing factorially with N:

Evgeny B Bogomolny and Yu A Kubyshin:
Asymptotic estimates for graphs with a fixed number of fermion loops
in quantum electrodynamics.
\\
1. The choice of the form of the
steepest-descent solutions,
Soviet J. Nucl. Phys. 34,6 (1981) 853-858.
Asymptotic estimates for diagrams with a fixed number of fermion loops
in quantum electrodynamics.
\\
2. The extremal configurations with the
symmetry group $O(2)\times O(3)$,
Soviet J. Nuclear Phys. 35 ,1 (1982) 114-119.
    \PC{there is a hard copy at GT library, 4th Floor East
Call Number: 	QC173.I252X (or microfilm?)}

I believe in the sort of arguments Bo+Ku are making, albeit the
details are questionable.
(E.g. saddlepoint asymptotic estimates are not rigorous unless you prove stuff
about the saddlepoints and about tail estimates, which they never proved, and
probably nobody can prove.)

Note that this believed N! growth for fixed-k (including quenched) QED diagrams
(Lautrup's diagrams also feature N! growth) is far faster than the
believed growth -- more like (N/2)! -- for the \emph{full} QED series!!
This fact that subseries diverge far more rapidly than full series
can only be explained by presuming that
the values at different k cancel each other amazingly well when we sum
over all k.
This indicates Cvitanovi\'c was extremely wrong asymptotically, and that
cancellations
quite different than his observations ultimately occur which seem even
more dramatic.

However, Cvitanovi\'c was correct that amazingly large cancellations
(also) occur, empirically, within the quenched diagrams alone, at least
for the small N that have been reached by computer.   And I presume that
Kreiman's knot ideas\rf{Kreimer00} and my ``fractal distribution''
empirical observations are a partial explanation of why.

So a consistent picture is now developing about how QED perturbative series (and
various interesting sub-series, such as quenched) allegedly behave
asymptotically
(Although I never saw anything saying all that I just said in one place...)

\item[2013-10-23  Warren to Predrag]  do you have a graph-theoretic
characterization of `gauge invariant diagram set' or know how many such
(minimal) sets there are at order N?   E.g. does their count grow
exponentially, super-exponentially, polynomially or what?  You gave a
formula for the count of quenched GI-sets which grows only polynomially
but I suspect exponential or faster growth for unquenched QED.  Even an
incomplete graph-theory understanding might be adequate to get good growth
bounds.

\item[2013-10-23 Warren]
Although I do not fully understand ``gauge invariant classes of Feynman diagram''
I have figured out enough now to prove that their count grows
ultimately superexponentially in unquenched QED.  Specifically I now
claim to have a proof that

  Number of GI classes of Feynman diagrams in QED at $\alpha^N$ order
\[
  \to 0.01 *  96^{-N/4} * N! / [ (N/2)! * (N/4)! ]
\]
for an infinite set of integers $N>0$.
And this clearly grows superexponentially.
(My bound is very unlikely to be optimal.)

\item[2013-10-23 Predrag to Warren]
of course, I'm very interested in this discussion, but can you do me a
favor and actually read my paper, and edit your initial email
accordingly, before we wrangle with further details?

Many of your statements are addressed in my paper, and I can answer thyem
more efficiently if you go through them first. I love Dyson dearly, but
his statement is an elementary statement about asymptotic series for
factorials. Mine is about mass-shell gauge invariant quantities (perhaps
planar?), a topic that is gaining some traction now, unfortunately not
applicable to QED. Ain't they weird? They are perfectly happy developing
methods for gauge theories that do not work on the 'trivial' U(1) case?
Go figure...

If you can get gauge sets added up, that would be useful, but they
probably do not have them: I assume they  use the second method of
computing magnetic moment,
\HREF{http://www.cns.gatech.edu/\%7Epredrag/papers/preprints.html\#g-2}
{eq (6.22)}  [that I believe I post-invented
after Schwinger? but probably I'm deluding myself..] I think that mixes up
gauge sets.

\item[2013-10-23 Warren]
Actually, I already had read it
\HREF{http://www.cns.gatech.edu/~predrag/papers/NPB77.pdf}
{here}.
The problem is not me not reading it; it is me being an idiot.  And/or, you.
Actually, it is quite likely that I am more of an idiot now than you
were in 1977, but I suspect we both are capable of considerable
idiocy...

[there is much more still on 2013-10-23, but I better go to bed...]

\item[2013-11-23  Predrag]
\HREF{WarrenSmithQED131123.html} {Warren Smith} draft.

\end{description}
