% reducesymm/QFT/WDSmith.tex
% $Author$ $Date$
% Predrag  switched to github.com               jul  8 2013

\chapter{Warren D Smith correspondence}
\label{c-WDSmith}

\begin{description}

\item[2013-10-23  Warren D. Smith] warren.wds@gmail.com
\\
(Predrag: I do not know
Dr. Smith, but for his life until 2009, see his resume
\HREF{http://scorevoting.net/WarrenSmithPages/homepage/}
{here}.)

I found out interesting info which the textbooks do not know!

Cvitanovi\'c 1977 conjectured that gauge-invariant quenched diagram sets
always have small sum of diagram values, indeed small enough that he
thought quenched QED series
would always converge\rf{Cvit77b}
    \PC{`quenched' = no internal electron loops, \ie, $m_e \to \infty$
    approximation}.
He apparently originally thought this even
for unquenched but Lautrup\rf{Lautrup77} disproved it and Dyson had long had a
highly convincing argument\rf{Dyson52} (Reprinted p.255-6 in Selected papers
of Freeman Dyson with
commentary, AMS 1996)
-- and I now have even more convincing
arguments -- that generic QED series diverge for any nonzero $\alpha$.)
This was due to Cvitanovi\'c's empirical observation of amazingly huge
cancellations within gauge-invariant diagram sets, especially in
quenched QED.

However, what Cvitanovi\'c did not know, was that
\HREF{http://www.itp.ac.ru/en/persons/bogomolny-evgeny-borisovich/}
{Bogomolny} and Kubyshin 1981-1982
found estimates of the growth rate of generic QED series, and also for the
quenched QED subseries,
and indeed for QED for diagrams with k electron loops only (k fixed)\rf{BogKub81}.
I only found this out the other day, but I had long known about
estimates due to Dyson and others predicting divergence for QED
series.  It is just that this B+K work by permitting arbitrary fixed
k, directly addresses Cvitanovi\'c's quenched-convergence conjecture and
massively conflicts with it -- for quenched QED the prediction is that
at Nth order we get a quenched diagram sum growing factorially with N:

Evgeny B Bogomolny and Yu A Kubyshin:
\\
1. The choice of the form of the
steepest-descent solutions\rf{BogKub81}.
Asymptotic estimates for diagrams with a fixed number of fermion loops
in quantum electrodynamics.
\\
2. The extremal configurations with the
symmetry group $O(2)\times O(3)$,
Soviet J. Nuclear Phys. 35 ,1 (1982) 114-119.
    \PC{there is a hard copy at GT library, 4th Floor East
Call Number: 	QC173.I252X (or microfilm?)}

I believe in the sort of arguments Bo+Ku are making, albeit the
details are questionable.
(E.g. saddlepoint asymptotic estimates are not rigorous unless you prove stuff
about the saddlepoints and about tail estimates, which they never proved, and
probably nobody can prove.)

Note that this believed N! growth for fixed-k (including quenched) QED diagrams
(Lautrup's diagrams also feature N! growth) is far faster than the
believed growth -- more like (N/2)! -- for the \emph{full} QED series!!
This fact that subseries diverge far more rapidly than full series
can only be explained by presuming that
the values at different k cancel each other amazingly well when we sum
over all k.
This indicates Cvitanovi\'c was extremely wrong asymptotically, and that
cancellations
quite different than his observations ultimately occur which seem even
more dramatic.

However, Cvitanovi\'c was correct that amazingly large cancellations
(also) occur, empirically, within the quenched diagrams alone, at least
for the small N that have been reached by computer.   And I presume that
Kreiman's knot ideas\rf{Kreimer00} and my ``fractal distribution''
empirical observations are a partial explanation of why.

So a consistent picture is now developing about how QED perturbative series (and
various interesting sub-series, such as quenched) allegedly behave
asymptotically
(Although I never saw anything saying all that I just said in one place...)

\item[2013-10-23  Warren to Predrag]  do you have a graph-theoretic
characterization of `gauge invariant diagram set' or know how many such
(minimal) sets there are at order N?   E.g. does their count grow
exponentially, super-exponentially, polynomially or what?  You gave a
formula for the count of quenched gauge sets which grows only polynomially
but I suspect exponential or faster growth for unquenched QED.  Even an
incomplete graph-theory understanding might be adequate to get good growth
bounds.

\item[2013-10-23 Warren]
Although I do not understand ``gauge sets of Feynman diagrams''
I have figured out enough now to prove that their count grows ultimately
superexponentially in unquenched QED.  Specifically I now claim to have a
proof that the number of gauge sets of Feynman diagrams in QED
at $\alpha^n$ order is
\[
  \approx 0.01\,(96)^{-n/4}\frac{n!}{(n/2)!  (n/4)!}
\]
for any integers $n>0$.
And this clearly grows superexponentially.
(My bound is very unlikely to be optimal.)

\item[2013-10-23 Predrag to Warren]:
I'm interested in this discussion, but can you do me a
favor and actually read my paper, and edit your initial email
accordingly, before we wrangle with further details?

Many of your statements are addressed in my paper, and I can answer them
more efficiently if you go through them first. I love Dyson dearly, but
his statement is an elementary statement about asymptotic series for
factorials. Mine is about mass-shell gauge invariant quantities (perhaps
planar?), a topic that is gaining some traction now, unfortunately not
applicable to QED.
%Aren't my colleagues weird? They are perfectly happy developing methods
%for gauge theories that do not work on the `trivial' U(1) case? Go
%figure...

If you can get gauge sets added up, that would be useful, but Aoyama
\etal\rf{AoHaKiNi12,AoHaKiNi15,AoKiNi18} do not have them: they  use the second
method of computing magnetic moment,
\HREF{http://www.cns.gatech.edu/\%7Epredrag/papers/preprints.html\#g-2}
{eq (6.22)} (see \refsect{sect:selfEnergy} above).
That mixes up the gauge sets.

%\item[2013-10-23 Warren]
%Actually, I already had read you paper (found
%\HREF{http://www.cns.gatech.edu/~predrag/papers/NPB77.pdf}
%{here}).
%The problem is not me not reading it; it is me being an idiot.  And/or, you.
%Actually, it is quite likely that I am more of an idiot now than you
%were in 1977, but I suspect we both are capable of considerable
%idiocy...

%[there is much more still in my 2013-10-23 email, but I better go to bed...]

\item[2013-11-23 Warren Smith]
\HREF{WarrenSmithQED131123.html} {draft of my notes} .

\item[2013-10-24 Warren]
\begin{quote} Predrag:
   Bogomolny and Kubyshin\rf{BogKub81} should find combinatorial growth
   in all sectors - a perturbation expansion around a saddle point is
   always asymptotic. For example, a path integral over a non-linear
   oscillator with quartic potential can be well defined as integral, but
   saddle point expansion is asymptotic - a simple example is worked out
   in Sect 3.3 {\em Saddle-point expansions are asymptotic}
   \HREF{http://chaosbook.org/FieldTheory/QMlectures/lectQM.pdf} {(click
   here)}.

   I agree with Dyson (and Bogomolny \&\ Y. Kubyshin, which is more
   modern version of the argument).
\end{quote}

Essentially, what they did is this.  ``Instanton" solutions of quantum
field theory stationarize the action and hence are ``saddlepoints."  The
full path integral over all quantum field histories, is hopefully
dominated by this saddlepoint and nearby configurations.  It seems there
are more than one kind of instanton, and you need to know which one is
the ``dominant" saddlepoint, and they may have the wrong one, but if so
it hopefully does not matter much (changes some constants, but not the
qualitative behavior, of their results, one hopes).  So anyhow, under
this assumption they actually are able to work out the asymptotics of the
Nth term in QFT perturbative series, in the limit where N is large.  They
find $N!^P$ style growth for positive powers P.  Furthermore, for the
``quenched" QED sub-series, one might naively think that approach would
say nothing about it, but they have further generating functionology
tricks which enables them to get conjectured asymptotics for that, and
lots of other sub-series, too.

The result is $N!^P$ style growth in all cases.
Indicating divergency and with radius of convergence zero.

Suslov\rf{Suslov99} and other Russians claim the original idea for this
was due to Lev Lipatov, but it has been explored by a fairly large number
of papers \&\ authors now, I think mostly French \&\ Russian.  My tome
has 2 chapters 5 \&\ 6 on Dysonian \&\ other divergence arguments which
give pointers into the literature.  The Suslov paper you cite also
has many such pointers.

\begin{quote} Predrag:
   What I claim/hope is that gauge invariance + mass-shell condition (neither
   accounted for in the above asymptotic estimates) induce cancellations that
   make the theory convergent.
\end{quote}

--I think this literature already had invented some way of handling
gauge invariance within their saddle pointage.   They had thought of
it and figured out a way to deal with it.
(Mind you, all of this stuff is horribly nonrigorous.)
And if by ``mass shell" you mean what I am calling ``quenched QED"
(kind of a battle over which name is worse...) then as I said, Bogol
already had a trick
for obtaining that from the unquenched analysis.

\begin{quote} Predrag:
Mass shell means that all external legs of Feynman diagrams are the
physical, asymptotic states satisfying $E = mc^2$. Intermediate virtual
states do not do that. Gauge invariance cancellations kick in for the
mass shell states, not for the off-mass shell amplitudes. Saddle point
estimates do not use mass-shell conditions, to the best of my knowledge.

Suslov\rf{Suslov99}
   {\em High orders of perturbation theory. {Are} renormalons significant?},
    \arXiv{hep-ph/0002051},
    seems to be about an unphysical but comparatively well behaved
$\phi^4$ quantum field theory.
It seems to argue that renormalons of t'Hooft and Lautrup do not matter. I
   have not studied Suslov's papers (or the ones that cite
   them \HREF{http://inspirehep.net/record/510344/citations}
   {inspirehep.net/record/510344/citations}).
\end{quote}
The Lautrup renormalon does not contradict the instanton-based results.
And renormalons also lead to yet more kinds of series divergency.
By the way my tome also discusses Lautrup renormalon.
Suslov in his footnote 4 and nearby points out that the whole
saddlepoint approach of Lipatov is nonrigorous (like I didn't know)
and in fact may be bogus.   Physicists often
use saddlepoint analysis but rarely do what it takes to do it
rigorously, for example they
rarely prove ``tail estimates."  (And of course when everything is in
an infinite dimensional space like it is here, that adds a whole new
level of difficulty in trying
to get any rigor.)  They generally instead just hope for the best.
Suslov suggests that the renormalon may be a clue that in this case, the
tail may be big enough that the whole Lipatov technique is wrong.

I also think everything Suslov does re Borel transforms, is wrong for QED since
QED is not going to be Borel summable.  However for $\phi^4$ theory
Borel might be OK... yes, it is, see table 11 in my tome.


\begin{quote} Predrag:
   You have to learn all this stuff (including
   \HREF{http://www.cns.gatech.edu/~predrag/papers/preprints.html\#PlanFieldThe}
   {planar field theory}) to get people to read you.
\end{quote}

I don't know whether anybody will read me, and I've had plenty of
trouble learning what i did learn, and I'm not willing to do too much
work to overcome that.
 I am certainly quite ignorant in many respects.  I draw some comfort
from the fact that many leaders in this area, including many Nobel
prize winners, have made plenty of errors, some of which I detected
for the first time, and some of which were real howlers.   The moral
of that for me is, nobody really knows what they are doing about this
stuff.  You've got to know the right stuff, not all stuff.  I hope if
my ideas are valid they will eventually gain some attention. If they
are wrong I hope this will become clear as soon as possible...

\end{description}

%\newpage %%%%%%%%%%%%%%%%%%%%%%%%%%%%%%%%%%%%%%%%%%%%%%%%
\printbibliography[heading=subbibintoc,title={References}]
