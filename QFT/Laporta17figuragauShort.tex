%% Laporta17figuragauShort.tex %%%%%
%% compiled by  reducesymm/QFT/blog.tex
%%%%%%%%%%%%%%%%%%%%%%%%%%%%%%%%%%%%%%%%
\begin{figure}[t]
\begin{center}
\begin{picture}(125,50)(70,350)
%\begin{picture}(125,30)(70,350)
\thicklines
%%%
{
\put(+000.0,+200.0){\makebox(0,0)[lb]{
% fotone obliquo (0.000000,200.000000) (0.000000,205.000000)
   \qbezier(+000.0,+200.0)(+001.2,+201.2)(+000.0,+202.5)
   \qbezier(+000.0,+202.5)(-001.2,+203.8)(+000.0,+205.0)
% elettrone curvo (0.000000,200.000000) (-20.000000,160.000000) 1000.000000
% cerchio r=44721.365140 c=(-40010.000000,20180.000000) ang=(-26.536403,-26.593699)
   \qbezier(+000.0,+200.0)(-010.0,+180.0)(-020.0,+160.0)
% elettrone curvo (0.000000,200.000000) (20.000000,160.000000) 1000.000000
% cerchio r=44721.365140 c=(-39990.000000,-19820.000000) ang=(26.593699,26.536403)
   \qbezier(+000.0,+200.0)(+010.0,+180.0)(+020.0,+160.0)
% fotone obliquo (-14.142136,171.715729) (14.142136,171.715729)
   \qbezier(-014.1,+171.7)(-013.3,+170.8)(-012.4,+171.7)
   \qbezier(-012.4,+171.7)(-011.5,+172.6)(-010.6,+171.7)
   \qbezier(-010.6,+171.7)(-009.7,+170.8)(-008.8,+171.7)
   \qbezier(-008.8,+171.7)(-008.0,+172.6)(-007.1,+171.7)
   \qbezier(-007.1,+171.7)(-006.2,+170.8)(-005.3,+171.7)
   \qbezier(-005.3,+171.7)(-004.4,+172.6)(-003.5,+171.7)
   \qbezier(-003.5,+171.7)(-002.7,+170.8)(-001.8,+171.7)
   \qbezier(-001.8,+171.7)(-000.9,+172.6)(+000.0,+171.7)
   \qbezier(+000.0,+171.7)(+000.9,+170.8)(+001.8,+171.7)
   \qbezier(+001.8,+171.7)(+002.7,+172.6)(+003.5,+171.7)
   \qbezier(+003.5,+171.7)(+004.4,+170.8)(+005.3,+171.7)
   \qbezier(+005.3,+171.7)(+006.2,+172.6)(+007.1,+171.7)
   \qbezier(+007.1,+171.7)(+008.0,+170.8)(+008.8,+171.7)
   \qbezier(+008.8,+171.7)(+009.7,+172.6)(+010.6,+171.7)
   \qbezier(+010.6,+171.7)(+011.5,+170.8)(+012.4,+171.7)
   \qbezier(+012.4,+171.7)(+013.3,+172.6)(+014.1,+171.7)
% fotone curvo (5.303301,189.393398) (10.000000,180.000000) 0.100000
% fotone semicircolare r=5.355061 c=(6.712311,184.227029) ang=(105.255119,-52.125016)
% n=8
   \qbezier(+005.3,+189.4)(+006.3,+188.7)(+007.1,+189.6)
   \qbezier(+007.1,+189.6)(+008.3,+190.3)(+008.9,+189.1)
   \qbezier(+008.9,+189.1)(+009.2,+187.9)(+010.4,+188.1)
   \qbezier(+010.4,+188.1)(+011.7,+187.9)(+011.5,+186.6)
   \qbezier(+011.5,+186.6)(+011.0,+185.5)(+012.0,+184.9)
   \qbezier(+012.0,+184.9)(+013.0,+183.9)(+011.9,+183.0)
   \qbezier(+011.9,+183.0)(+010.8,+182.5)(+011.2,+181.4)
   \qbezier(+011.2,+181.4)(+011.3,+180.0)(+010.0,+180.0)
% fotone curvo (12.247449,175.505103) (17.320508,165.358984) 0.100000
% fotone semicircolare r=5.784178 c=(13.769367,169.924737) ang=(105.255119,-52.125016)
% n=9
   \qbezier(+012.2,+175.5)(+013.2,+174.8)(+014.0,+175.7)
   \qbezier(+014.0,+175.7)(+015.0,+176.5)(+015.7,+175.4)
   \qbezier(+015.7,+175.4)(+016.1,+174.2)(+017.3,+174.5)
   \qbezier(+017.3,+174.5)(+018.6,+174.6)(+018.5,+173.3)
   \qbezier(+018.5,+173.3)(+018.2,+172.1)(+019.3,+171.7)
   \qbezier(+019.3,+171.7)(+020.3,+171.0)(+019.6,+170.0)
   \qbezier(+019.6,+170.0)(+018.6,+169.2)(+019.3,+168.2)
   \qbezier(+019.3,+168.2)(+019.8,+167.0)(+018.5,+166.6)
   \qbezier(+018.5,+166.6)(+017.3,+166.5)(+017.3,+165.4)
% fotone curvo (15.811388,168.377223) (18.708287,162.583426) -0.100000
% fotone semicircolare r=3.302973 c=(17.839217,165.770015) ang=(127.874984,285.255119)
% n=5
   \qbezier(+015.8,+168.4)(+014.5,+168.2)(+014.7,+166.9)
   \qbezier(+014.7,+166.9)(+015.4,+166.0)(+014.6,+165.1)
   \qbezier(+014.6,+165.1)(+014.1,+163.9)(+015.4,+163.5)
   \qbezier(+015.4,+163.5)(+016.5,+163.7)(+016.9,+162.6)
   \qbezier(+016.9,+162.6)(+017.8,+161.6)(+018.7,+162.6)
\put(+000.0,+152.0){\makebox(0,0){$(1)$}}
}}
\put(+025.0,+200.0){\makebox(0,0)[lb]{
% fotone obliquo (25.000000,200.000000) (25.000000,205.000000)
   \qbezier(+025.0,+200.0)(+026.2,+201.2)(+025.0,+202.5)
   \qbezier(+025.0,+202.5)(+023.8,+203.8)(+025.0,+205.0)
% elettrone curvo (25.000000,200.000000) (5.000000,160.000000) 1000.000000
% cerchio r=44721.365140 c=(-39985.000000,20180.000000) ang=(-26.536403,-26.593699)
   \qbezier(+025.0,+200.0)(+015.0,+180.0)(+005.0,+160.0)
% elettrone curvo (25.000000,200.000000) (45.000000,160.000000) 1000.000000
% cerchio r=44721.365140 c=(-39965.000000,-19820.000000) ang=(26.593699,26.536403)
   \qbezier(+025.0,+200.0)(+035.0,+180.0)(+045.0,+160.0)
% fotone obliquo (14.309550,178.619101) (35.690450,178.619101)
   \qbezier(+014.3,+178.6)(+015.2,+177.7)(+016.1,+178.6)
   \qbezier(+016.1,+178.6)(+017.0,+179.5)(+017.9,+178.6)
   \qbezier(+017.9,+178.6)(+018.8,+177.7)(+019.7,+178.6)
   \qbezier(+019.7,+178.6)(+020.5,+179.5)(+021.4,+178.6)
   \qbezier(+021.4,+178.6)(+022.3,+177.7)(+023.2,+178.6)
   \qbezier(+023.2,+178.6)(+024.1,+179.5)(+025.0,+178.6)
   \qbezier(+025.0,+178.6)(+025.9,+177.7)(+026.8,+178.6)
   \qbezier(+026.8,+178.6)(+027.7,+179.5)(+028.6,+178.6)
   \qbezier(+028.6,+178.6)(+029.5,+177.7)(+030.3,+178.6)
   \qbezier(+030.3,+178.6)(+031.2,+179.5)(+032.1,+178.6)
   \qbezier(+032.1,+178.6)(+033.0,+177.7)(+033.9,+178.6)
   \qbezier(+033.9,+178.6)(+034.8,+179.5)(+035.7,+178.6)
% fotone obliquo (8.096915,166.193830) (41.903085,166.193830)
   \qbezier(+008.1,+166.2)(+009.0,+165.3)(+009.9,+166.2)
   \qbezier(+009.9,+166.2)(+010.8,+167.1)(+011.7,+166.2)
   \qbezier(+011.7,+166.2)(+012.5,+165.3)(+013.4,+166.2)
   \qbezier(+013.4,+166.2)(+014.3,+167.1)(+015.2,+166.2)
   \qbezier(+015.2,+166.2)(+016.1,+165.3)(+017.0,+166.2)
   \qbezier(+017.0,+166.2)(+017.9,+167.1)(+018.8,+166.2)
   \qbezier(+018.8,+166.2)(+019.7,+165.3)(+020.6,+166.2)
   \qbezier(+020.6,+166.2)(+021.4,+167.1)(+022.3,+166.2)
   \qbezier(+022.3,+166.2)(+023.2,+165.3)(+024.1,+166.2)
   \qbezier(+024.1,+166.2)(+025.0,+167.1)(+025.9,+166.2)
   \qbezier(+025.9,+166.2)(+026.8,+165.3)(+027.7,+166.2)
   \qbezier(+027.7,+166.2)(+028.6,+167.1)(+029.4,+166.2)
   \qbezier(+029.4,+166.2)(+030.3,+165.3)(+031.2,+166.2)
   \qbezier(+031.2,+166.2)(+032.1,+167.1)(+033.0,+166.2)
   \qbezier(+033.0,+166.2)(+033.9,+165.3)(+034.8,+166.2)
   \qbezier(+034.8,+166.2)(+035.7,+167.1)(+036.6,+166.2)
   \qbezier(+036.6,+166.2)(+037.5,+165.3)(+038.3,+166.2)
   \qbezier(+038.3,+166.2)(+039.2,+167.1)(+040.1,+166.2)
   \qbezier(+040.1,+166.2)(+041.0,+165.3)(+041.9,+166.2)
% fotone curvo (32.559289,184.881421) (38.093073,173.813853) 0.100000
% fotone semicircolare r=6.309484 c=(34.219425,178.794259) ang=(105.255119,-52.125016)
% n=9
   \qbezier(+032.6,+184.9)(+033.6,+184.1)(+034.5,+185.1)
   \qbezier(+034.5,+185.1)(+035.6,+185.9)(+036.3,+184.7)
   \qbezier(+036.3,+184.7)(+036.8,+183.5)(+038.0,+183.8)
   \qbezier(+038.0,+183.8)(+039.4,+183.8)(+039.4,+182.4)
   \qbezier(+039.4,+182.4)(+039.0,+181.2)(+040.2,+180.7)
   \qbezier(+040.2,+180.7)(+041.4,+179.9)(+040.5,+178.8)
   \qbezier(+040.5,+178.8)(+039.5,+178.0)(+040.2,+176.9)
   \qbezier(+040.2,+176.9)(+040.7,+175.6)(+039.4,+175.2)
   \qbezier(+039.4,+175.2)(+038.1,+175.1)(+038.1,+173.8)
% fotone curvo (40.118579,169.762842) (43.516402,162.967196) 0.100000
% fotone semicircolare r=3.874114 c=(41.137926,166.025237) ang=(105.255119,-52.125016)
% n=6
   \qbezier(+040.1,+169.8)(+041.0,+169.0)(+041.9,+169.8)
   \qbezier(+041.9,+169.8)(+043.1,+170.3)(+043.5,+169.1)
   \qbezier(+043.5,+169.1)(+043.5,+168.0)(+044.6,+167.8)
   \qbezier(+044.6,+167.8)(+045.7,+167.1)(+045.0,+166.0)
   \qbezier(+045.0,+166.0)(+044.1,+165.4)(+044.6,+164.3)
   \qbezier(+044.6,+164.3)(+044.8,+163.0)(+043.5,+163.0)
\put(+025.0,+152.0){\makebox(0,0){$(2)$}}
}}
\put(+050.0,+200.0){\makebox(0,0)[lb]{
% fotone obliquo (50.000000,200.000000) (50.000000,205.000000)
   \qbezier(+050.0,+200.0)(+051.2,+201.2)(+050.0,+202.5)
   \qbezier(+050.0,+202.5)(+048.8,+203.8)(+050.0,+205.0)
% elettrone curvo (50.000000,200.000000) (30.000000,160.000000) 1000.000000
% cerchio r=44721.365140 c=(-39960.000000,20180.000000) ang=(-26.536403,-26.593699)
   \qbezier(+050.0,+200.0)(+040.0,+180.0)(+030.0,+160.0)
% elettrone curvo (50.000000,200.000000) (70.000000,160.000000) 1000.000000
% cerchio r=44721.365140 c=(-39940.000000,-19820.000000) ang=(26.593699,26.536403)
   \qbezier(+050.0,+200.0)(+060.0,+180.0)(+070.0,+160.0)
% fotone obliquo (33.670068,167.340137) (66.329932,167.340137)
   \qbezier(+033.7,+167.3)(+034.5,+166.5)(+035.4,+167.3)
   \qbezier(+035.4,+167.3)(+036.2,+168.2)(+037.1,+167.3)
   \qbezier(+037.1,+167.3)(+038.0,+166.5)(+038.8,+167.3)
   \qbezier(+038.8,+167.3)(+039.7,+168.2)(+040.5,+167.3)
   \qbezier(+040.5,+167.3)(+041.4,+166.5)(+042.3,+167.3)
   \qbezier(+042.3,+167.3)(+043.1,+168.2)(+044.0,+167.3)
   \qbezier(+044.0,+167.3)(+044.8,+166.5)(+045.7,+167.3)
   \qbezier(+045.7,+167.3)(+046.6,+168.2)(+047.4,+167.3)
   \qbezier(+047.4,+167.3)(+048.3,+166.5)(+049.1,+167.3)
   \qbezier(+049.1,+167.3)(+050.0,+168.2)(+050.9,+167.3)
   \qbezier(+050.9,+167.3)(+051.7,+166.5)(+052.6,+167.3)
   \qbezier(+052.6,+167.3)(+053.4,+168.2)(+054.3,+167.3)
   \qbezier(+054.3,+167.3)(+055.2,+166.5)(+056.0,+167.3)
   \qbezier(+056.0,+167.3)(+056.9,+168.2)(+057.7,+167.3)
   \qbezier(+057.7,+167.3)(+058.6,+166.5)(+059.5,+167.3)
   \qbezier(+059.5,+167.3)(+060.3,+168.2)(+061.2,+167.3)
   \qbezier(+061.2,+167.3)(+062.0,+166.5)(+062.9,+167.3)
   \qbezier(+062.9,+167.3)(+063.8,+168.2)(+064.6,+167.3)
   \qbezier(+064.6,+167.3)(+065.5,+166.5)(+066.3,+167.3)
% fotone curvo (35.857864,171.715729) (31.742581,163.485163) -0.100000
% fotone semicircolare r=4.692144 c=(34.623279,167.188917) ang=(74.744881,232.125016)
% n=7
   \qbezier(+035.9,+171.7)(+035.0,+172.8)(+034.0,+171.8)
   \qbezier(+034.0,+171.8)(+033.4,+170.8)(+032.3,+171.3)
   \qbezier(+032.3,+171.3)(+031.0,+171.4)(+030.9,+170.1)
   \qbezier(+030.9,+170.1)(+031.2,+168.9)(+030.1,+168.4)
   \qbezier(+030.1,+168.4)(+029.0,+167.6)(+030.0,+166.6)
   \qbezier(+030.0,+166.6)(+031.0,+166.0)(+030.5,+164.9)
   \qbezier(+030.5,+164.9)(+030.4,+163.5)(+031.7,+163.5)
% fotone curvo (64.142136,171.715729) (68.257419,163.485163) 0.100000
% fotone semicircolare r=4.692144 c=(65.376721,167.188917) ang=(105.255119,-52.125016)
% n=7
   \qbezier(+064.1,+171.7)(+065.1,+171.0)(+066.0,+171.8)
   \qbezier(+066.0,+171.8)(+067.1,+172.5)(+067.7,+171.3)
   \qbezier(+067.7,+171.3)(+067.9,+170.1)(+069.1,+170.1)
   \qbezier(+069.1,+170.1)(+070.4,+169.7)(+069.9,+168.4)
   \qbezier(+069.9,+168.4)(+069.2,+167.5)(+070.0,+166.6)
   \qbezier(+070.0,+166.6)(+070.7,+165.4)(+069.5,+164.9)
   \qbezier(+069.5,+164.9)(+068.2,+164.7)(+068.3,+163.5)
% fotone curvo (56.123724,187.752551) (61.547005,176.905989) 0.100000
% fotone semicircolare r=6.183492 c=(57.750709,181.786942) ang=(105.255119,-52.125016)
% n=9
   \qbezier(+056.1,+187.8)(+057.2,+187.0)(+058.0,+188.0)
   \qbezier(+058.0,+188.0)(+059.1,+188.8)(+059.8,+187.6)
   \qbezier(+059.8,+187.6)(+060.3,+186.4)(+061.5,+186.7)
   \qbezier(+061.5,+186.7)(+062.9,+186.7)(+062.8,+185.4)
   \qbezier(+062.8,+185.4)(+062.5,+184.1)(+063.6,+183.7)
   \qbezier(+063.6,+183.7)(+064.8,+182.9)(+063.9,+181.8)
   \qbezier(+063.9,+181.8)(+062.9,+181.0)(+063.7,+180.0)
   \qbezier(+063.7,+180.0)(+064.1,+178.7)(+062.8,+178.3)
   \qbezier(+062.8,+178.3)(+061.6,+178.2)(+061.5,+176.9)
\put(+050.0,+152.0){\makebox(0,0){$(3)$}}
}}
\put(+075.0,+200.0){\makebox(0,0)[lb]{
% fotone obliquo (75.000000,200.000000) (75.000000,205.000000)
   \qbezier(+075.0,+200.0)(+076.2,+201.2)(+075.0,+202.5)
   \qbezier(+075.0,+202.5)(+073.8,+203.8)(+075.0,+205.0)
% elettrone curvo (75.000000,200.000000) (55.000000,160.000000) 1000.000000
% cerchio r=44721.365140 c=(-39935.000000,20180.000000) ang=(-26.536403,-26.593699)
   \qbezier(+075.0,+200.0)(+065.0,+180.0)(+055.0,+160.0)
% elettrone curvo (75.000000,200.000000) (95.000000,160.000000) 1000.000000
% cerchio r=44721.365140 c=(-39915.000000,-19820.000000) ang=(26.593699,26.536403)
   \qbezier(+075.0,+200.0)(+085.0,+180.0)(+095.0,+160.0)
% fotone obliquo (66.835034,183.670068) (91.329932,167.340137)
   \qbezier(+066.8,+183.7)(+067.1,+182.5)(+068.3,+182.7)
   \qbezier(+068.3,+182.7)(+069.5,+182.9)(+069.7,+181.7)
   \qbezier(+069.7,+181.7)(+070.0,+180.5)(+071.2,+180.8)
   \qbezier(+071.2,+180.8)(+072.4,+181.0)(+072.6,+179.8)
   \qbezier(+072.6,+179.8)(+072.8,+178.6)(+074.0,+178.9)
   \qbezier(+074.0,+178.9)(+075.2,+179.1)(+075.5,+177.9)
   \qbezier(+075.5,+177.9)(+075.7,+176.7)(+076.9,+176.9)
   \qbezier(+076.9,+176.9)(+078.1,+177.2)(+078.4,+176.0)
   \qbezier(+078.4,+176.0)(+078.6,+174.8)(+079.8,+175.0)
   \qbezier(+079.8,+175.0)(+081.0,+175.3)(+081.2,+174.1)
   \qbezier(+081.2,+174.1)(+081.5,+172.9)(+082.7,+173.1)
   \qbezier(+082.7,+173.1)(+083.9,+173.3)(+084.1,+172.1)
   \qbezier(+084.1,+172.1)(+084.4,+170.9)(+085.6,+171.2)
   \qbezier(+085.6,+171.2)(+086.8,+171.4)(+087.0,+170.2)
   \qbezier(+087.0,+170.2)(+087.2,+169.0)(+088.4,+169.3)
   \qbezier(+088.4,+169.3)(+089.6,+169.5)(+089.9,+168.3)
   \qbezier(+089.9,+168.3)(+090.1,+167.1)(+091.3,+167.3)
% fotone obliquo (58.670068,167.340137) (83.164966,183.670068)
   \qbezier(+058.7,+167.3)(+059.9,+167.1)(+060.1,+168.3)
   \qbezier(+060.1,+168.3)(+060.4,+169.5)(+061.6,+169.3)
   \qbezier(+061.6,+169.3)(+062.8,+169.0)(+063.0,+170.2)
   \qbezier(+063.0,+170.2)(+063.2,+171.4)(+064.4,+171.2)
   \qbezier(+064.4,+171.2)(+065.6,+170.9)(+065.9,+172.1)
   \qbezier(+065.9,+172.1)(+066.1,+173.3)(+067.3,+173.1)
   \qbezier(+067.3,+173.1)(+068.5,+172.9)(+068.8,+174.1)
   \qbezier(+068.8,+174.1)(+069.0,+175.3)(+070.2,+175.0)
   \qbezier(+070.2,+175.0)(+071.4,+174.8)(+071.6,+176.0)
   \qbezier(+071.6,+176.0)(+071.9,+177.2)(+073.1,+176.9)
   \qbezier(+073.1,+176.9)(+074.3,+176.7)(+074.5,+177.9)
   \qbezier(+074.5,+177.9)(+074.8,+179.1)(+076.0,+178.9)
   \qbezier(+076.0,+178.9)(+077.2,+178.6)(+077.4,+179.8)
   \qbezier(+077.4,+179.8)(+077.6,+181.0)(+078.8,+180.8)
   \qbezier(+078.8,+180.8)(+080.0,+180.5)(+080.3,+181.7)
   \qbezier(+080.3,+181.7)(+080.5,+182.9)(+081.7,+182.7)
   \qbezier(+081.7,+182.7)(+082.9,+182.5)(+083.2,+183.7)
% fotone curvo (60.857864,171.715729) (56.742581,163.485163) -0.100000
% fotone semicircolare r=4.692144 c=(59.623279,167.188917) ang=(74.744881,232.125016)
% n=7
   \qbezier(+060.9,+171.7)(+060.0,+172.8)(+059.0,+171.8)
   \qbezier(+059.0,+171.8)(+058.4,+170.8)(+057.3,+171.3)
   \qbezier(+057.3,+171.3)(+056.0,+171.4)(+055.9,+170.1)
   \qbezier(+055.9,+170.1)(+056.2,+168.9)(+055.1,+168.4)
   \qbezier(+055.1,+168.4)(+054.0,+167.6)(+055.0,+166.6)
   \qbezier(+055.0,+166.6)(+056.0,+166.0)(+055.5,+164.9)
   \qbezier(+055.5,+164.9)(+055.4,+163.5)(+056.7,+163.5)
% fotone curvo (85.526385,178.947231) (89.142136,171.715729) 0.100000
% fotone semicircolare r=4.122590 c=(86.611110,174.969905) ang=(105.255119,-52.125016)
% n=6
   \qbezier(+085.5,+178.9)(+086.5,+178.2)(+087.4,+179.0)
   \qbezier(+087.4,+179.0)(+088.7,+179.6)(+089.1,+178.3)
   \qbezier(+089.1,+178.3)(+089.1,+177.0)(+090.3,+176.8)
   \qbezier(+090.3,+176.8)(+091.5,+176.1)(+090.7,+175.0)
   \qbezier(+090.7,+175.0)(+089.7,+174.3)(+090.3,+173.2)
   \qbezier(+090.3,+173.2)(+090.5,+171.8)(+089.1,+171.7)
\put(+075.0,+152.0){\makebox(0,0){$(4)$}}
}}
\put(+100.0,+200.0){\makebox(0,0)[lb]{
% fotone obliquo (100.000000,200.000000) (100.000000,205.000000)
   \qbezier(+100.0,+200.0)(+101.2,+201.2)(+100.0,+202.5)
   \qbezier(+100.0,+202.5)(+098.8,+203.8)(+100.0,+205.0)
% elettrone curvo (100.000000,200.000000) (80.000000,160.000000) 1000.000000
% cerchio r=44721.365140 c=(-39910.000000,20180.000000) ang=(-26.536403,-26.593699)
   \qbezier(+100.0,+200.0)(+090.0,+180.0)(+080.0,+160.0)
% elettrone curvo (100.000000,200.000000) (120.000000,160.000000) 1000.000000
% cerchio r=44721.365140 c=(-39890.000000,-19820.000000) ang=(26.593699,26.536403)
   \qbezier(+100.0,+200.0)(+110.0,+180.0)(+120.0,+160.0)
% fotone obliquo (91.835034,183.670068) (114.142136,171.715729)
   \qbezier(+091.8,+183.7)(+092.2,+182.4)(+093.4,+182.8)
   \qbezier(+093.4,+182.8)(+094.7,+183.2)(+095.0,+182.0)
   \qbezier(+095.0,+182.0)(+095.4,+180.7)(+096.6,+181.1)
   \qbezier(+096.6,+181.1)(+097.8,+181.5)(+098.2,+180.3)
   \qbezier(+098.2,+180.3)(+098.6,+179.0)(+099.8,+179.4)
   \qbezier(+099.8,+179.4)(+101.0,+179.8)(+101.4,+178.5)
   \qbezier(+101.4,+178.5)(+101.8,+177.3)(+103.0,+177.7)
   \qbezier(+103.0,+177.7)(+104.2,+178.1)(+104.6,+176.8)
   \qbezier(+104.6,+176.8)(+105.0,+175.6)(+106.2,+176.0)
   \qbezier(+106.2,+176.0)(+107.4,+176.4)(+107.8,+175.1)
   \qbezier(+107.8,+175.1)(+108.1,+173.9)(+109.4,+174.3)
   \qbezier(+109.4,+174.3)(+110.6,+174.6)(+111.0,+173.4)
   \qbezier(+111.0,+173.4)(+111.3,+172.2)(+112.5,+172.6)
   \qbezier(+112.5,+172.6)(+113.8,+172.9)(+114.1,+171.7)
% fotone obliquo (85.857864,171.715729) (108.164966,183.670068)
   \qbezier(+085.9,+171.7)(+087.1,+171.3)(+087.5,+172.6)
   \qbezier(+087.5,+172.6)(+087.8,+173.8)(+089.0,+173.4)
   \qbezier(+089.0,+173.4)(+090.3,+173.1)(+090.6,+174.3)
   \qbezier(+090.6,+174.3)(+091.0,+175.5)(+092.2,+175.1)
   \qbezier(+092.2,+175.1)(+093.5,+174.8)(+093.8,+176.0)
   \qbezier(+093.8,+176.0)(+094.2,+177.2)(+095.4,+176.8)
   \qbezier(+095.4,+176.8)(+096.6,+176.5)(+097.0,+177.7)
   \qbezier(+097.0,+177.7)(+097.4,+178.9)(+098.6,+178.5)
   \qbezier(+098.6,+178.5)(+099.8,+178.2)(+100.2,+179.4)
   \qbezier(+100.2,+179.4)(+100.6,+180.6)(+101.8,+180.3)
   \qbezier(+101.8,+180.3)(+103.0,+179.9)(+103.4,+181.1)
   \qbezier(+103.4,+181.1)(+103.8,+182.3)(+105.0,+182.0)
   \qbezier(+105.0,+182.0)(+106.2,+181.6)(+106.6,+182.8)
   \qbezier(+106.6,+182.8)(+106.9,+184.0)(+108.2,+183.7)
% fotone obliquo (81.742581,163.485163) (118.257419,163.485163)
   \qbezier(+081.7,+163.5)(+082.6,+162.6)(+083.5,+163.5)
   \qbezier(+083.5,+163.5)(+084.4,+164.4)(+085.2,+163.5)
   \qbezier(+085.2,+163.5)(+086.1,+162.6)(+087.0,+163.5)
   \qbezier(+087.0,+163.5)(+087.8,+164.4)(+088.7,+163.5)
   \qbezier(+088.7,+163.5)(+089.6,+162.6)(+090.4,+163.5)
   \qbezier(+090.4,+163.5)(+091.3,+164.4)(+092.2,+163.5)
   \qbezier(+092.2,+163.5)(+093.0,+162.6)(+093.9,+163.5)
   \qbezier(+093.9,+163.5)(+094.8,+164.4)(+095.7,+163.5)
   \qbezier(+095.7,+163.5)(+096.5,+162.6)(+097.4,+163.5)
   \qbezier(+097.4,+163.5)(+098.3,+164.4)(+099.1,+163.5)
   \qbezier(+099.1,+163.5)(+100.0,+162.6)(+100.9,+163.5)
   \qbezier(+100.9,+163.5)(+101.7,+164.4)(+102.6,+163.5)
   \qbezier(+102.6,+163.5)(+103.5,+162.6)(+104.3,+163.5)
   \qbezier(+104.3,+163.5)(+105.2,+164.4)(+106.1,+163.5)
   \qbezier(+106.1,+163.5)(+107.0,+162.6)(+107.8,+163.5)
   \qbezier(+107.8,+163.5)(+108.7,+164.4)(+109.6,+163.5)
   \qbezier(+109.6,+163.5)(+110.4,+162.6)(+111.3,+163.5)
   \qbezier(+111.3,+163.5)(+112.2,+164.4)(+113.0,+163.5)
   \qbezier(+113.0,+163.5)(+113.9,+162.6)(+114.8,+163.5)
   \qbezier(+114.8,+163.5)(+115.6,+164.4)(+116.5,+163.5)
   \qbezier(+116.5,+163.5)(+117.4,+162.6)(+118.3,+163.5)
% fotone curvo (111.547005,176.905989) (116.329932,167.340137) 0.100000
% fotone semicircolare r=5.453375 c=(112.981883,171.644770) ang=(105.255119,-52.125016)
% n=8
   \qbezier(+111.5,+176.9)(+112.6,+176.2)(+113.4,+177.1)
   \qbezier(+113.4,+177.1)(+114.6,+177.8)(+115.2,+176.6)
   \qbezier(+115.2,+176.6)(+115.5,+175.4)(+116.8,+175.6)
   \qbezier(+116.8,+175.6)(+118.1,+175.4)(+117.9,+174.1)
   \qbezier(+117.9,+174.1)(+117.3,+173.0)(+118.4,+172.3)
   \qbezier(+118.4,+172.3)(+119.3,+171.3)(+118.3,+170.4)
   \qbezier(+118.3,+170.4)(+117.2,+169.9)(+117.6,+168.7)
   \qbezier(+117.6,+168.7)(+117.7,+167.4)(+116.3,+167.3)
\put(+100.0,+152.0){\makebox(0,0){$(5)$}}
}}
\put(+250.0,+235.0){\makebox(0,0)[lb]{
% fotone obliquo (0.000000,165.000000) (0.000000,170.000000)
   \qbezier(+000.0,+165.0)(+001.2,+166.2)(+000.0,+167.5)
   \qbezier(+000.0,+167.5)(-001.2,+168.8)(+000.0,+170.0)
% elettrone curvo (0.000000,165.000000) (-20.000000,125.000000) 1000.000000
% cerchio r=44721.365140 c=(-40010.000000,20145.000000) ang=(-26.536403,-26.593699)
   \qbezier(+000.0,+165.0)(-010.0,+145.0)(-020.0,+125.0)
% elettrone curvo (0.000000,165.000000) (20.000000,125.000000) 1000.000000
% cerchio r=44721.365140 c=(-39990.000000,-19855.000000) ang=(26.593699,26.536403)
   \qbezier(+000.0,+165.0)(+010.0,+145.0)(+020.0,+125.0)
% fotone obliquo (-8.944272,147.111456) (12.649111,139.701779)
   \qbezier(-008.9,+147.1)(-008.4,+146.0)(-007.3,+146.5)
   \qbezier(-007.3,+146.5)(-006.2,+147.1)(-005.6,+146.0)
   \qbezier(-005.6,+146.0)(-005.1,+144.9)(-004.0,+145.4)
   \qbezier(-004.0,+145.4)(-002.8,+145.9)(-002.3,+144.8)
   \qbezier(-002.3,+144.8)(-001.8,+143.7)(-000.6,+144.3)
   \qbezier(-000.6,+144.3)(+000.5,+144.8)(+001.0,+143.7)
   \qbezier(+001.0,+143.7)(+001.6,+142.6)(+002.7,+143.1)
   \qbezier(+002.7,+143.1)(+003.8,+143.7)(+004.3,+142.6)
   \qbezier(+004.3,+142.6)(+004.9,+141.4)(+006.0,+142.0)
   \qbezier(+006.0,+142.0)(+007.1,+142.5)(+007.7,+141.4)
   \qbezier(+007.7,+141.4)(+008.2,+140.3)(+009.3,+140.8)
   \qbezier(+009.3,+140.8)(+010.4,+141.4)(+011.0,+140.3)
   \qbezier(+011.0,+140.3)(+011.5,+139.2)(+012.6,+139.7)
% fotone obliquo (-12.649111,139.701779) (15.491933,134.016133)
   \qbezier(-012.6,+139.7)(-011.9,+138.6)(-010.9,+139.3)
   \qbezier(-010.9,+139.3)(-009.8,+140.0)(-009.1,+139.0)
   \qbezier(-009.1,+139.0)(-008.4,+137.9)(-007.4,+138.6)
   \qbezier(-007.4,+138.6)(-006.3,+139.3)(-005.6,+138.3)
   \qbezier(-005.6,+138.3)(-004.9,+137.2)(-003.9,+137.9)
   \qbezier(-003.9,+137.9)(-002.8,+138.6)(-002.1,+137.6)
   \qbezier(-002.1,+137.6)(-001.4,+136.5)(-000.3,+137.2)
   \qbezier(-000.3,+137.2)(+000.7,+137.9)(+001.4,+136.9)
   \qbezier(+001.4,+136.9)(+002.1,+135.8)(+003.2,+136.5)
   \qbezier(+003.2,+136.5)(+004.2,+137.2)(+004.9,+136.1)
   \qbezier(+004.9,+136.1)(+005.6,+135.1)(+006.7,+135.8)
   \qbezier(+006.7,+135.8)(+007.8,+136.5)(+008.5,+135.4)
   \qbezier(+008.5,+135.4)(+009.2,+134.4)(+010.2,+135.1)
   \qbezier(+010.2,+135.1)(+011.3,+135.8)(+012.0,+134.7)
   \qbezier(+012.0,+134.7)(+012.7,+133.7)(+013.7,+134.4)
   \qbezier(+013.7,+134.4)(+014.8,+135.1)(+015.5,+134.0)
% fotone obliquo (-15.491933,134.016133) (17.888544,129.222912)
   \qbezier(-015.5,+134.0)(-014.7,+133.0)(-013.7,+133.8)
   \qbezier(-013.7,+133.8)(-012.7,+134.5)(-012.0,+133.5)
   \qbezier(-012.0,+133.5)(-011.2,+132.5)(-010.2,+133.3)
   \qbezier(-010.2,+133.3)(-009.2,+134.0)(-008.5,+133.0)
   \qbezier(-008.5,+133.0)(-007.7,+132.0)(-006.7,+132.8)
   \qbezier(-006.7,+132.8)(-005.7,+133.5)(-005.0,+132.5)
   \qbezier(-005.0,+132.5)(-004.2,+131.5)(-003.2,+132.3)
   \qbezier(-003.2,+132.3)(-002.2,+133.0)(-001.4,+132.0)
   \qbezier(-001.4,+132.0)(-000.7,+131.0)(+000.3,+131.7)
   \qbezier(+000.3,+131.7)(+001.3,+132.5)(+002.1,+131.5)
   \qbezier(+002.1,+131.5)(+002.8,+130.5)(+003.8,+131.2)
   \qbezier(+003.8,+131.2)(+004.8,+132.0)(+005.6,+131.0)
   \qbezier(+005.6,+131.0)(+006.3,+130.0)(+007.3,+130.7)
   \qbezier(+007.3,+130.7)(+008.4,+131.5)(+009.1,+130.5)
   \qbezier(+009.1,+130.5)(+009.9,+129.5)(+010.9,+130.2)
   \qbezier(+010.9,+130.2)(+011.9,+131.0)(+012.6,+130.0)
   \qbezier(+012.6,+130.0)(+013.4,+129.0)(+014.4,+129.7)
   \qbezier(+014.4,+129.7)(+015.4,+130.5)(+016.1,+129.5)
   \qbezier(+016.1,+129.5)(+016.9,+128.5)(+017.9,+129.2)
% fotone obliquo (-17.888544,129.222912) (8.944272,147.111456)
   \qbezier(-017.9,+129.2)(-016.6,+129.0)(-016.4,+130.2)
   \qbezier(-016.4,+130.2)(-016.1,+131.5)(-014.9,+131.2)
   \qbezier(-014.9,+131.2)(-013.7,+131.0)(-013.4,+132.2)
   \qbezier(-013.4,+132.2)(-013.2,+133.4)(-011.9,+133.2)
   \qbezier(-011.9,+133.2)(-010.7,+132.9)(-010.4,+134.2)
   \qbezier(-010.4,+134.2)(-010.2,+135.4)(-008.9,+135.2)
   \qbezier(-008.9,+135.2)(-007.7,+134.9)(-007.5,+136.2)
   \qbezier(-007.5,+136.2)(-007.2,+137.4)(-006.0,+137.2)
   \qbezier(-006.0,+137.2)(-004.7,+136.9)(-004.5,+138.2)
   \qbezier(-004.5,+138.2)(-004.2,+139.4)(-003.0,+139.2)
   \qbezier(-003.0,+139.2)(-001.7,+138.9)(-001.5,+140.2)
   \qbezier(-001.5,+140.2)(-001.2,+141.4)(-000.0,+141.1)
   \qbezier(+000.0,+141.1)(+001.2,+140.9)(+001.5,+142.1)
   \qbezier(+001.5,+142.1)(+001.7,+143.4)(+003.0,+143.1)
   \qbezier(+003.0,+143.1)(+004.2,+142.9)(+004.5,+144.1)
   \qbezier(+004.5,+144.1)(+004.7,+145.4)(+006.0,+145.1)
   \qbezier(+006.0,+145.1)(+007.2,+144.9)(+007.5,+146.1)
   \qbezier(+007.5,+146.1)(+007.7,+147.4)(+008.9,+147.1)
\put(+000.0,+117.0){\makebox(0,0){$(6)$}}
}}
}
%%%
 \end{picture}
  \\[3ex]
\begin{tabular}{ccrrr}
gauge set &$(k,m,m)$& value~~ & ansatz \\
\hline
   (1)  & (1,3,0) & - 1.9710    & - 1/2  \\%01
   (2)  & (2,2,0) &{\color{red}-0.1424}  & \phantom{+} 1/2 &~(!)\\%02
   (3)  & (1,2,1) &  - 0.6219   & - 1/2  \\%03
   (4)  & (2,1,1) &  \phantom{+} 1.0867  & \phantom{+} 1/2  \\%04
   (5)  & (3,1,0) &  - 1.0405   & - 1/2  \\%05
   (6)  & (4,0,0) &  \phantom{+} 0.5125  & \phantom{+} 1/2  \\%06
\hline
%  1+2+{\ldots}+6 & - 2.176866027739540077443259355895893938670  \\
% -2*1.9710-2*0.14249-2*0.6219+1.0867-2*1.0405+0.5125 =
% -1.9710-0.14249-0.6219+1.0867-1.0405+0.5125 =  -2.17669
\end{tabular}
 \end{center}
%%/ \phantom{ }\vspace{0truecm}\phantom{ }
\caption{\label{Laporta17figuragauShort}
(top)
Examples of 4-loop vertex diagrams belonging to Laporta\rf{Laporta17} gauge sets
(1) to (6).
%(1) = $(1,3,0)$,
%(2) = $(2,2,0)$,
%(3) = $(1,2,1)$,
%(4) = $(2,1,1)$,
%(5) = $(3,1,0)$,
%(6) = $(4,0,0)$.
The remaining diagrams in the set can be obtained by permuting separately
the vertices on the left and right side of the electron line, and
considering also the mirror images of the diagrams. For all 25
gauge sets, see \reffig{Laporta17figuragau}.
The table:
Gauge-set contributions $a^{(8)}_{kmm'}$, see \refeq{quenchAnom}, as
reported by Laporta\rf{Laporta17}
(for the full 25 gauge sets, see \reftab{Laporta17:tableset}).
The last column: 1977 Cvitanovi\'c predictions\rf{Cvit77b}.
Signs are right, except for the set (2) = $(2,2,0)$, which is anomalously
small, and the remaining sets are surprisingly close to multiples of 1/2.
There might be factors of 2 having to do with symmetries, missing from
the guesses of \refref{Cvit77b}, but I cannot see how that would work.
Only (4) = $(2,1,1)$ and  (6) = $(4,0,0)$ are symmetric,
but (1) = $(1,3,0)$, (4) and (5) = $(3,1,0)$ seem to
have an extra factor of 2 or 4.
}
 \end{figure}
%%%%%%%%%%%%%%%%%%%%%%%%%%%%%%%%%%%%%%%%
