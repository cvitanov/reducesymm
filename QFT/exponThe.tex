% GitHub cvitanov/reducesymm/QFT/exponThe.tex
% to compile:  cd cvitanov/reducesymm/QFT; pdflatex blog; biber blog

% Predrag  created              Dec 23 2018

\section{Non-Abelian exponentiation theorem}
\label{s:FiMaFu18}
\newcommand{\vev}[1]{{\left< {#1} \right>}}

{\bf [2018-12-22 Predrag]}

% Bartomeu Fiol <bfiol@ub.edu>,  jmartinez@icc.ub.edu, ariosfukelman@icc.ub.edu
Fiol, Martínez-Montoya and Fukelman\rf{FiMaFu18} {\em Wilson loops in
terms of color invariants} addresses the question of whether one can
compute directly the logarithm of $\vev{W}_R$, the vacuum expectation
value (vev) of a Wilson loop.

The reason I'm intrigued by this paper is that their `$n$-gluon chord
diagrams' are also the $n$-photon, no-fermion loop `quenched-'
% , or `q-type'
diagrams
% (`quenched', as this corresponds to the $N_f$-independent part
% of the vertex amplitude in QED with $N_f$ flavors)
of the quenched QED in the worldline formalism. So far we are looking at
what corresponds to $\vev{W}_R$ chord diagrams, but we should really be
looking at the $\ln \vev{W}_R$ connected diagrams, and really at their
Legendre transform, the 1pI diagrams. I would generate $\ln \vev{W}_R$
using Dyson-Scwinger equations for connected correlation functions, see
the relations between the full generating function $Z[J]$,
the connected generating function $W[J]$, and
the 1pI generating function $\Gamma[\phi]$ in
\refref{FieldThe}.

The \emph{non-Abelian exponentiation theorem} implies that certain color
invariants present in $\vev{W}_R$ are absent in $\ln \vev{W}_R$.
To me this looks like the connection between the full and connected
partition functions, except that here quark lines are not providing
connections, only the crossed gluon lines are.

The perturbative expansion of vevs can be written in terms of
color invariants built from contractions of the
fully symmetrized traces
\beq
d_R^{a_1\dots a_n}
  =\frac{1}{n!} \sum_{\sigma \in {\cal S}_n} \hbox{ tr }
        T_R^{a_{\sigma(1)}}\dots T_R^{a_{\sigma (n)}}
\ee{symtraces}
where $T^a_R$ are the generators of the Lie algebra of the group G, in
the representation R.
$d_R$ with no superscript is the trace of the identity matrix,
$d_R= \text {\tr} \, {\bf 1}= \text{dim } R$.

The color invariants $d_R^{a_1 a_1\dots a_k a_k}$ in (\ref{introexact})
can be reduced to lower order color invariants.
Examples of color invariants: $d_R^{aabb}$ or $d_R^{abcd}d_A^{abcd}$.

Fiol \etal\rf{FiMaFu18} conventions for color invariants are largely
those of van Ritbergen, Schellekens and Vermaseren {\em Group theory
factors for {Feynman} diagrams}\rf{RiScVe99},
\arXiv{hep-ph/9802376}.
Some of the invariants are given by Okubo and Patera {\em General indices
of simple {Lie} algebras and symmetrized product
representations}\rf{OkuPat83}. Many are presumably in Chaper~7 of
\refref{PCgr} and some in \refref{NPB81}; perhaps re-expressing color
invariants in terms of orthogonal Dynkin indices might yield some extra
insights.

They show that the vev of $W_R$ can be written in term of symmetrized
traces (\ref{symtraces}),
with pairwise contracted indices,
\beq
\vev{W}_R = \frac{1}{d_R}\sum_{k=0}^\infty \frac{1}{k!}
d_R^{a_1 a_1\dots a_k a_k} g^k
\,,
\ee{introexact}
where $g=g^2_\text{YM}/4$ is a Yang-Mills ``fine structure'' constant,
up to $\pi$ here and there.
This expression gives the vev of 1/2 BPS circular Wilson loop for any
representation R of a gauge group G. It allows to discuss exact relations
among vevs in different representations. For instance, if $R^t$ is the
transpose representation of $R$ (in the sense of having Young diagrams
transpose to each other), then, following \refrefs{NegDimE7,CK82,PCgr}
\[
\vev{W}_{R^t}(\lambda,N)=\vev{W}_R(\lambda,-N)
\]
thus relating, for instance, vevs in the symmetric and the antisymmetric
representations of \SUn{n}.

They carry out the integrals over the full Lie algebra
(in ${\cal N}=4$ super Yang Mills matrix model),
\beq
\vev{W}_R = \frac{1}{d_R}\vev{\hbox{\tr} \, e^{2\pi M}}=
 \frac{1}{d_R} \frac{\int_{\mathfrak{g}} dM    \,
\text{\tr} e^{2\pi M} \,
e^{-\frac{2\pi^2}{g}\text{tr }M^2}}{\int_{\mathfrak{g}}dM \,
e^{-\frac{2\pi^2}{g}\text{tr }M^2}}
\ee{winmm}
Denoting by $m^a$ the coefficients of the matrix $M$ in the Lie algebra,
the two-point function in this Gaussian matrix model is
\beq
\vev{m^a m^b}= \frac{g}{2\pi^2}\,\delta^{ab} \hspace{1cm} a,b=1,\dots, N
\ee{atwopoint}
To compute the vev of the normalized Wilson loop, we expand the exponent
insertion in (\ref{winmm}), use the two-point function (\ref{atwopoint})
and apply Wick's theorem,
\beq
\vev{W}_R =\frac{1}{d_R}\sum_{k=0}^\infty
\frac{(2\pi)^{2k}}{(2k)!} \vev{m^{a_1}\dots m^{a_{2k}}} \hbox{tr } T^{a_1}_R \dots T^{a_{2k}}_R=\frac{1}{d_R}\sum_{k=0}^\infty d_R^{a_1a_1\dots a_k a_k} \frac{g^k}{k!}
\ee{exactvev}
where $d_R^{a_1\dots a_k}$ are the symmetrized
traces (\ref{symtraces}).

At every order evaluation of the fully symmetrized traces
(\ref{symtraces}) with pairwise contracted indices $d_R^{a_1 a_1\dots a_k
a_k}$. yields a combination of lower order color invariants. At low
orders they evaluate them by hand, using the methods of van Ritbergen,
Schellekens and Vermaseren\rf{RiScVe99},
\begin{align*}
d_R^{aa} & =  \text{tr } T_R^a T_R^a= c_R d_R \\
d_R^{aabb} & =  \frac{1}{3}\text{ tr}
\left(2T_R^aT_R^aT_R^bT_R^b+T_R^aT_R^bT_R^aT_R^b\right)
=(c_R^2-\frac{1}{6}c_A c_R)d_R
\end{align*}
Higher orders, up to order $g^{7}$, they evaluate using
FormTracer\rf{CyMiSt16}. They are not pretty, but the perturbative
expansion of $\ln \vev{W}_R$ is considerably simpler.

For \SUn{n} fully (anti)symmetric irrep
there is an intriguing factorization, their eq.~(2.11).


The theorem offer some support to the conjecture of
Fiol, Gerchkovitz and Komargodski\rf{FiGeKo16}
{\em Exact bremsstrahlung function in {N}=2 superconformal field theories}.
A closed formula for the bremsstrahlung function is given in terms of a
derivative with the respect to the coupling,
\beq
B_R(\lambda,N)
=\frac{1}{2\pi^2}\lambda \frac{\partial \ln \vev{W}_R}{\partial \lambda}
= \cdots
\ee{closedb}
Only a subset of the most general color invariants appears in the
expansion of the bremsstrahlung function $B$.

This formula is worth remembering when thinking of the self-energy vs.
vertex diagram computation of the anomalous magnetic moment.

Computation of $\ln \vev{W}_R$ up to order $g^{7}$ demonstrates $\ln
\vev{W}_R$ is simpler than that of $\vev{W}_R$.
Many color invariants present in the expansion of $\vev{W}_R$ are absent
in the expansion of $\ln \vev{W}_R$. For instance, there are no color
invariants involving the quadratic casimir $c_R^k$ with $k\geq 2$.

This expansion is simpler
for $\ln \vev{W}_R$ than for $\vev{W}_R$ itself: the only color
invariants that appear in the perturbative expansion of $\ln \vev{W}_R$
at a given order are those that cannot be written as products of color
invariants that appear at lower orders of the perturbative expansion,
thus providing an illustration of the non-Abelian exponentiation theorem.

They
assert that this
simpler structure is a consequence of the non-Abelian exponentiation
theorem%\cite{Gatheral:1983cz, Frenkel:1984pz}:
; at every order in
perturbation theory, the only color invariants that can appear in $\ln
\vev{W}_R$ are the ones that cannot be written as products of color
invariants that appear at lower orders in the perturbative expansion of
$\vev{W}_R$.

Very interesting is their diagrammatic interpretation of the perturbative
expansion of $\ln \vev{W}_R$. According to the literature,
in the Feynman gauge, the only Feynman diagrams that contribute to
$\vev{W}_R$ involve gluon propagators starting and ending on the Wilson
line. Such diagrams are called \emph{chord diagrams}. %\cite{Touchard}.
At order $2n$
there are $(2n-1)!!$ of them. On the other hand, by virtue of the
non-Abelian exponentiation theorem,
%  \cite{Gatheral:1983cz,Frenkel:1984pz},
to compute $\ln \vev{W}_R$ one only needs to take into
account a subset of them, the so-called \emph{connected chord diagrams}:
diagrams where all gluon lines overlap with some other gluon line, see
\reffig{f:FiMaFu181f}.

\begin{figure}
\centering
\includegraphics[width=.8\textwidth]{FiMaFu181f}
\caption{
Examples of chord diagrams. For $k$ gluons, by Wick's
theorem there are (2k-1)!! such diagrams. The second one is a connected
chord diagram. They contribute to $\ln \vev{W}_R$.
% Their number is given by the recursion relation (\ref{connectedchord}).
The last diagram is a
fully disconnected chord diagram.
For $k$ gluons, there are ${\cal C}_k= {(2k)!}/{(k+1)! k!}$ of them.
(From Fiol \etal\rf{FiMaFu18})
        }
\label{f:FiMaFu181f}
\end{figure}

The number of connected chord diagrams with $n$ chords satisfies the
recursion relation % \cite{stein, nijenwilf}
\beq
a_1=1 \hspace{1cm} a_n =(n-1)\sum_{k=1}^{n-1} a_k a_{n-k}
\,,
\ee{connectedchord}
so up to seven loops the numbers are
\[
a_n=1,1,4,27,248,2830,38232,\dots
\]
%It can be proven \cite{stein2} that
Asymptotically the ratio of the number of connected chord diagrams to the
total number of chord diagrams with $n$ gluons (or photons)
is $e$ times less than the total number of Feynman diagrams,
\[
\lim_{n\to \infty} \frac{a_n}{(2n-1)!!}=\frac{1}{e}
\,.
\]
To compute $\ln \vev{W}_R$ by evaluating just the connected gluon
diagrams, we have to take into account that according to the non-Abelian
exponentiation theorem,
% \cite{Gatheral:1983cz, Frenkel:1984pz},
the color factor of each diagram is a modified color factor $\bar c_i$.
To compute $\bar c_i$ of a given connected gluon diagram, one has
subtract the color factors of all possible gluon insertions of the diagram,
as illustrated in \reffig{f:FiMaFu18modCFac}.

\begin{figure}
  \centering
  \includegraphics[width=1\textwidth]{FiMaFu18modCFac}
  \put(-263,2){\Large$-$\Huge$($}
  \put(-245,2){\LARGE$2$}
  \put(-132,2){\Large$+$}
  \put(0,2){\Huge$)$}
  \put(-290,-30){\large$\bar c= c_R \left( c_R - \frac{1}{2} c_A \right)^2 -  \left( 2c_R(-\frac{1}{2}c_R c_A)+c_R^3 \right)= \frac{1}{4} c_R c_A^3$}
  \caption{
The modified color factor of this connected 3-gluon diagram is obtained
by subtracting from its color factor all other gluon insertions.
(From Fiol \etal\rf{FiMaFu18})
    }
\label{f:FiMaFu18modCFac}
\end{figure}

Two chord diagrams have the same reduced color factor if their
intersection graphs are isomorphic. The intersection graph is defined as
follows: % \cite{Bouchet}:
for each chord introduce a point on the plane; if two chords cross, draw
an edge between the two points, see \reffig{f:FiMaFu186c}.
Since only connected chord diagrams contribute to $\ln \vev{W}_R$, we can
restrict our attention to connected intersection graphs. The numbers of
non-isomorphic connected intersection graphs for chord diagrams
% has been discussed in \cite{arratia}. Their numbers
are
\[
1,1,2,6,21,110,789,8336,117283,\dots
\]
At order $g^4$, there are $7!!=105$ 4-gluon chord diagrams, 27 connected
chord diagrams, and only 6 connected intersection graphs. In detail,
the 27 connected chord diagrams are grouped according to
six intersection graphs as $27=8+4+8+2+4+1$.
The reduced color factors that they list up to four loops, with
intersection graph up to four dots, are surprisingly simple.

\begin{figure}
\centering
\includegraphics[width=.8\textwidth]{FiMaFu186c}
\caption{
The intersection graph of a 4-gluon Feynman diagram. For each gluon, draw
a dot on the plane; each time two gluon lines intersect, draw a link
between the corresponding dots.
(From Fiol \etal\rf{FiMaFu18})
    }
\label{f:FiMaFu186c}
\end{figure}

If $n$ is the number of gluon propagators, they have
\[
\ln \vev{W}_R= \sum_{n=1}^\infty \frac{1}{2n!}
\left(2 g^2\right)^n \sum_{conn} \bar c_i
\]
where the sum $\sum_{conn}$ runs over connected chord diagrams with $n$
gluon propagators.

Starting at seventh order, color invariants are not all independent; the
first identity they satisfy is\rf{RiScVe99}
\beq
d_A^{abcdef}d_A^{abcdef}-\frac{5}{8}d_A^{abcd}d_A^{cdef}d_A^{efab}
+\frac{7}{240}c_A^2 d_A^{abcd}d_A^{abcd}+\frac{1}{864}c_A^6 d_A
=0
\,.
\ee{adjointrelation}
They indulge in a bit of intriguing but so far inconclusive numerology.
