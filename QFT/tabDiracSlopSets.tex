%%%%%%%%%%%%%%%%%%%%%%%%%%%%%%%%%%%%%%%%%%%%%%%%%%%%%
% tabDiracSlopSets.tex    2017-06-02
% compiled by  reducesymm/QFT/blog.tex
% needs \usepackage{booktabs}\usepackage{amsmath}
\begin{table}
\centering
{\small
\begin{tabular}{r@{~~~~}ccccc@{~~~~}l}
$2n$ & \multicolumn{5}{c}{$(k,m,m')$} & anomaly \\
    \toprule[1.5pt]\\[-1.0em]
% Entering  row 2
 & $\bf (1,0,0)$
 \\[-1ex]
\raisebox{1.5ex}{2}
 & IR div.           &&&&& \raisebox{1.5ex}{}
  \\[1ex]
 \cmidrule(lr){2-3}\\[-0.8em]
% Entering  row 4
 & $\bf (1,1,0)$  &  $\bf (2,0,0)$
 \\[-1ex]
\raisebox{1.5ex}{4}
 & -${\gamma}$ (-.XX)&  ${\gamma}$  (.XX) &&&& \raisebox{1.5ex}{0 (.XX)}
  \\[1ex]
 \cmidrule(lr){2-4}\\[-0.8em]
% Entering  row 6
 & $\bf (1,2,0)$ & $\bf (2,1,0)$   & $\bf (3,0,0)$
 \\[0.1ex]
 & ${\gamma}$ (.XX) & -${\gamma}$ (-.XX) &  ${\gamma}$ (.XX)
 \\%[-1ex]
\raisebox{1.5ex}{6}
 & $\bf (1,1,1)$ &&&&&          \raisebox{1.5ex}{1 (.XX)}\\
 & ${\gamma}$ (.XX)
  \\[1ex]
 \cmidrule(lr){2-5}\\[-0.8em]
% Entering  row 8
 & $\bf (1,3,0)$     & $\bf (2,2,0)$  & $\bf (3,1,0)$  & $\bf (4,0,0)$
 \\[0.1ex]
 &  -${\gamma}${\color{red}$\cdot$?} ({\color{red}+}.135)
                     & ${\gamma}$$\cdot?$ (.380)
                                      & -${\gamma}${\color{red}$\cdot$?} (-1.098)
                                                        &  ${\gamma}$ (.647)
 \\%[-1ex]
\raisebox{1.5ex}{8}
 & $\bf (1,2,1)$  & $\bf (2,1,1)$ &&&& \raisebox{1.5ex}{0 (.351)}\\
 & -${\gamma}$ (-.079)    &   ${\gamma}${\color{red}$\cdot$?} (.366)
  \\[1ex]
% \cmidrule(lr){2-6}
%% Entering  row 10
% & $\bf (1,4,0)$ & $\bf (2,3,0)$  & $\bf (3,2,0)$
%                                        & $\bf (4,1,0)$
%                                            & $\bf (5,0,0)$
% \\[0.1ex]
% &    $\frac{1}{2}${\color{red}$\cdot$12} (6.2)
%                 & -$\frac{1}{2}$ (-0.72)   & $\frac{1}{2}$  {\color{red} $\cdot 0$ (-0.40)}
%                                        & -$\frac{1}{2}${\color{red}$\cdot$2} (-1.02)
%                                             &  $\frac{1}{2}${\color{red}$\cdot$2} (1.09)
% \\%[-1ex]
%\raisebox{1.5ex}{10}
% & $\bf (1,3,1)$  & $\bf (2,2,1)$ & $\bf (3,1,1)$ &&&
%        \raisebox{1.5ex}{$\frac{3}{2} {\color{red} \cdot 4}\,(6.78)$}\\
% &  $\frac{1}{2}$ (0.90)    & -$\frac{1}{2}${\color{red}$\cdot$4} (-2.16)
%                                  & $\frac{1}{2}${\color{red}$\cdot$5} (2.62)
%  \\[1ex]
% & $\bf (1,2,2)$ \\
% & $\frac{1}{2}$ (0.30)
%  \\[1ex]
\bottomrule
\end{tabular}
} %end {\small
\caption{\label{tabGaugeSets}
% Updated \reffig{Cvit77bFig3} comparison of
Comparison of the $\pm\frac{1}{2}$ gauge-set $(k,m,m')$ sign ansatz
\refeq{Cvit77b(5)} with the numerical value of the corresponding slope of
the {Dirac} form factor gauge set, stated in $(\cdots)$ bracket.
Starting with 4-loops, the gauge-set ansatz \refeq{Cvit77b(5)} fails.
%All gauge sets are surprisingly close to integer multiples of 1/2;
%the ones differing from multiple 1 are marked in red.
The signs predictions are correct, except for the two anomalously small gauge sets
?$(2,2,0)$, and its ``descendent'' ?$(3,2,0)$, which might be approximately zero.
}
\end{table}
%%%%%%%%%%%%%%%%%%%%%%%%%%%%%%%%%%%%%%%%%%%%%%%%%%%%%
