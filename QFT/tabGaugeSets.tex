%%%%%%%%%%%%%%%%%%%%%%%%%%%%%%%%%%%%%%%%%%%%%%%%%%%%%
% tabGaugeSets.tex    2017-06-02
% compiled by  reducesymm/QFT/blog.tex
% needs \usepackage{booktabs}\usepackage{amsmath}
\begin{figure}
\centering
\begin{tabular}{r@{~~~~}ccccc@{~~~~}l}
$2n$ & \multicolumn{5}{c}{$km'm$} & anomaly \\
    \toprule[1.5pt]\\[-1.0em]
% Entering  row 2
 & $\bf (1,0,0)$
 \\[-1ex]
\raisebox{1.5ex}{2}
 & 1/2            &&&&& \raisebox{1.5ex}{$\frac{1}{2}$}
  \\[1ex]
 \cmidrule(lr){2-3}\\[-0.8em]
% Entering  row 4
 & $\bf (1,1,0)$  &  $\bf (2,0,0)$
 \\[-1ex]
\raisebox{1.5ex}{4}
 & -1/2 (-.65)&  1/2  (.31) &&&& \raisebox{1.5ex}{0 (-.33)}
  \\[1ex]
 \cmidrule(lr){2-4}\\[-0.8em]
% Entering  row 6
 & $\bf (1,2,0)$ & $\bf (2,1,0)$   & $\bf (3,0,0)$
 \\[0.1ex]
 & 1/2 (.56) & -1/2 (-.47) &  1/2 (.44)
 \\%[-1ex]
\raisebox{1.5ex}{6}
 & $\bf (1,1,1)$ &&&&&          \raisebox{1.5ex}{1 (.93)}\\
 & 1/2 (.43)
  \\[1ex]
 \cmidrule(lr){2-5}\\[-0.8em]
% Entering  row 8
 & $\bf (1,3,0)$ & $\bf (2,2,0)$ & $\bf (3,1,0)$  & $\bf (4,0,0)$
 \\[0.1ex]
 &   -1/2 (-1.97) & 1/2 (-.14)   &  -1/2 (-1.04)  &  1/2 (.51)
 \\%[-1ex]
\raisebox{1.5ex}{8}
 & $\bf (1,2,1)$  & $\bf (2,1,1)$ &&&& \raisebox{1.5ex}{0 (-2.17)}\\
 & -1/2 (-.62)    &   1/2 (1.08)
  \\[1ex]
 \cmidrule(lr){2-6}
% Entering  row 10
 & $\bf (1,4,0)$ & $\bf (2,3,0)$  & $\bf (3,2,0)$  & $\bf (4,1,0)$  & $\bf (5,0,0)$
 \\[0.1ex]
 &       1/2 (?) &     -1/2 (?) &   1/2 (?)     &  -1/2 (?)    &  1/2 (?)
 \\%[-1ex]
\raisebox{1.5ex}{10}
 & $\bf (1,3,1)$  & $\bf (2,2,1)$ & $\bf (3,1,1)$ &&& \raisebox{1.5ex}{$\frac{3}{2}$ (7.60)}\\
 &  1/2 (?)      &      -1/2 (?)&       1/2 (?)
  \\[1ex]
 & $\bf (1,2,2)$ \\
 & 1/2 (?)
  \\[1ex]
\bottomrule
\end{tabular}
\caption{\label{tabGaugeSets}
Updated \reffig{Cvit77bFig3}
comparison of the gauge-set approximation \refeq{Cvit77b(1)} and the actual
numerical values of corresponding gauge sets, together with the
5-loop prediction. Starting with 4-loops, the gauge-set approximation
fails in detail. Still,
the signs are right, except for the anomalously small set $(2,2,0)$,
and the remaining sets are surprisingly close to multiples of 1/2.
}
\end{figure}
%%%%%%%%%%%%%%%%%%%%%%%%%%%%%%%%%%%%%%%%%%%%%%%%%%%%%
