%%%%%%%%%%%%%%%%%%%%%%%%%%%%%%%%%%%%%%%%%%%%%%%%%%%%%
% tabGaugeSets.tex    2017-06-02
% compiled by  reducesymm/QFT/blog.tex
% needs \usepackage{booktabs}\usepackage{amsmath}
\begin{figure}
\centering
{\small
\begin{tabular}{r@{~~~~}ccccc@{~~~~}l}
$2n$ & \multicolumn{5}{c}{$\bf (k,m',m)$} & anomaly \\
    \toprule[1.5pt]\\[-1.0em]
% Entering  row 2
 & $\bf (1,0,0)$
 \\[-1ex]
\raisebox{1.5ex}{2}
 & $\frac{1}{2}$            &&&&& \raisebox{1.5ex}{$\frac{1}{2}$}
  \\[1ex]
 \cmidrule(lr){2-3}\\[-0.8em]
% Entering  row 4
 & $\bf (1,1,0)$  &  $\bf (2,0,0)$
 \\[-1ex]
\raisebox{1.5ex}{4}
 & -$\frac{1}{2}$ (-.65)&  $\frac{1}{2}$  (.31) &&&& \raisebox{1.5ex}{0 (-.33)}
  \\[1ex]
 \cmidrule(lr){2-4}\\[-0.8em]
% Entering  row 6
 & $\bf (1,2,0)$ & $\bf (2,1,0)$   & $\bf (3,0,0)$
 \\[0.1ex]
 & $\frac{1}{2}$ (.56) & -$\frac{1}{2}$ (-.47) &  $\frac{1}{2}$ (.44)
 \\%[-1ex]
\raisebox{1.5ex}{6}
 & $\bf (1,1,1)$ &&&&&          \raisebox{1.5ex}{1 (.93)}\\
 & $\frac{1}{2}$ (.43)
  \\[1ex]
 \cmidrule(lr){2-5}\\[-0.8em]
% Entering  row 8
 & $\bf (1,3,0)$     & $\bf (2,2,0)$  & $\bf (3,1,0)$  & $\bf (4,0,0)$
 \\[0.1ex]
 &  -$\frac{1}{2}${\color{red}$\cdot$4} (-1.97)
                     & $\frac{1}{2}${\color{red}$\cdot$0 (-.14)}
                                      & -$\frac{1}{2}${\color{red}$\cdot$2} (-1.04)
                                                        &  $\frac{1}{2}$ (.51)
 \\%[-1ex]
\raisebox{1.5ex}{8}
 & $\bf (1,2,1)$  & $\bf (2,1,1)$ &&&& \raisebox{1.5ex}{0 (-2.17)}\\
 & -$\frac{1}{2}$ (-.62)    &   $\frac{1}{2}${\color{red}$\cdot$2} (1.08)
  \\[1ex]
 \cmidrule(lr){2-6}
% Entering  row 10
 & $\bf (1,4,0)$ & $\bf (2,3,0)$  & $\bf (3,2,0)$
                                        & $\bf (4,1,0)$
                                            & $\bf (5,0,0)$
 \\[0.1ex]
 &    $\frac{1}{2}${\color{red}$\cdot$12} (6.2)
                 & -$\frac{1}{2}$ (-0.72)   & $\frac{1}{2}$ {\color{red}(-0.40)}
                                        & -$\frac{1}{2}${\color{red}$\cdot$2} (-1.02)
                                             &  $\frac{1}{2}${\color{red}$\cdot$2} (1.09)
 \\%[-1ex]
\raisebox{1.5ex}{10}
 & $\bf (1,3,1)$  & $\bf (2,2,1)$ & $\bf (3,1,1)$ &&& \raisebox{1.5ex}{$\frac{3}{2} {\color{red} \cdot 4}(6.78)$}\\
 &  $\frac{1}{2}$ (0.90)    & -$\frac{1}{2}${\color{red}$\cdot$4} (-2.16)
                                  & $\frac{1}{2}${\color{red}$\cdot$5} (2.62)
  \\[1ex]
 & $\bf (1,2,2)$ \\
 & $\frac{1}{2}$ (0.30)
  \\[1ex]
\bottomrule
\end{tabular}
} %end {\small
\caption{\label{tabGaugeSets}
Updated \reffig{Cvit77bFig3} comparison of the gauge-set approximation
\refeq{Cvit77b(1)} and the actual numerical values of corresponding gauge
sets, together with the 5-loop prediction. Starting with 4-loops, the
gauge-set approximation $\pm1/2$ fails in detail. Still,
the signs are right (except for the anomalously small set $(2,2,0)$, and its
``descendent'' $(3,2,0)$),
and the remaining sets are surprisingly close to multiples of 1/2.
The 5-loop gauge sets are the Volvkov\rf{Volkov19} results,
the sum of which disagrees with Aoyama \etal\rf{AoKiNi18}.
Looking only at the 4- and 5-loop results, Volvkov surmises that (1,m,0)
is approximately the final value, and the other values are approximately
zero with random errors.
They do not look random to me; all signs except one follow my prediction, and the
numbers are (unexplained) integers multiples of 1/2.
}
\end{figure}
%%%%%%%%%%%%%%%%%%%%%%%%%%%%%%%%%%%%%%%%%%%%%%%%%%%%%
