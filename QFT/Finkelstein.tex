% reducesymm/QFT/Finkelstein.tex

\chapter{David Finkelstein}
\label{c-Finkelstein}

David Ritz Finkelstein (July 19, 1929 -- January 24, 2016)

For a bibliography, see
\HREF{http://www.physics.gatech.edu/user/david-finkelstein}
{GaTech} homepage,
his
\HREF{https://www.davidritzfinkelstein.com/publications.html}
{selected publications}, and
\HREF{https://inspirehep.net/search?ln=en&ln=en&p=exactauthor\%3AD.Finkelstein.1&of=hb&action_search=Search&sf=earliestdate&so=d&rm=&rg=25&sc=0}
{inSpire} listing.

\section{Finkelstein notes}
\label{sect:Finkelstein}

\begin{description}
\item[1958-05-15 David Finkelstein]

Finkelstein\rf{Finkel55}
{\em Internal structure of spinning particles}

\item[1958-05-15 The Paper]
Finkelstein\rf{Finkel58}
{\em Past-future asymmetry of the gravitational field of a point particle}.
See \refsect{sect:Finkel69a} for a scan, still to be proofread.

\item[1971-01-13 Jim Anderson] David was very hard to understand. In
colloquia first 15 minutes one could follow, then it got fuzzy. They
tried to write a book on general relativity, but could not agree on
anything beyond the first chapter. Eventually Anderson wrote the book by
himself.

They wrote a paper on the cosmological constant. As metric is irreducible,
determinant is geometric object. Imposing $\det g = 1$ as a Lagrange
multiplier adds the cosmological term as a Lagrange multiplier.

Anderson and Finkelstein\rf{AndFin71}
{\em Cosmological constant and fundamental length}: ``
In usual formulations of general relativity, the cosmological constant
$\lambda$
appears as an inelegant ambiguity in the fundamental action principle.
With a slight reformulation, $\lambda$ appears as an unavoidable Lagrange
multiplier, belonging to a constraint. The constraint expresses the
existence of a fundamental element of space-time hypervolume at every
point. The fundamental scale of length in atomic physics provides such a
hypervolume element. In this sense, the presence in relativity of an
undetermined cosmological length is a direct consequence of the existence
of a fundamental atomic length.''

In 2016 the paper has some 10 references (about 90 overall).

This work was continued in
Finkelstein, Galiautdinov and Baugh\rf{FiGaBa01}
{\em Unimodular relativity and cosmological constant}:

`Unimodular gravity', or `the Weyl-transverse gravity', is an alternative
theory of gravity considered by Einstein in 1919 without a Lagrangian and
put into Lagrangian form by Anderson and Finkelstein\rf{AndFin71}. It is
a restricted version of General Relativity in which the determinant of
the metric $-\sqrt{g}$ is a fixed function and the field equations are
given by the trace-free part of the full Einstein equations. Or,
unimodular gravity is just a gauge  fixed version of Einstein gravity,
with unimodularity can be imposed with a Langrange multiplier.  The
Bianchi identity forces the Lagrange multiplier to take the role of a
cosmological constant which now appears as an integration constant.


\item[1990-05-24 David Finkelstein] \texttt{900524PhysToday.pdf}, a
letter to Physics Today (unpublished) in response to Khurana\rf{Khurana89a}.

Wilczek\rf{Wilczek82} {\em Quantum mechanics of fractional-spin
particles} names `anyons' (since interchange of two of these particles
can give any phase), credits  Leinaas and Myrheim (1977) etc., but not
Finkelstein.

Wilczek and Zee\rf{WilZee83}
{\em Linking numbers, spin, and statistics of solitons}:
``Some aspects of this subject appear to have been anticipated in the
remarkable paper of D. Finkelstein and J. Rubinstein\rf{FinRub68}."

Misner, Thorne and Zurek\rf{MiThZu09}
{\em {John Wheeler}, relativity, and quantum information}:
``The physics becomes lucid in 1959-62, through insights by David
Finkelstein"
`` which dealt with only half of the Schwartzfield geometry and revealed
its horizon and thence its role as a black hole"

Finkelstein\rf{Finkel87} {\em The Quantum Paradox},
in {Encyclopaedia Britannica} {Yearbook of Science and the Future}
is a very helpful popularization of his view of quantum mechanics.

Finkelstein \etal\rf{FJSS62}
{\em Foundations of quaternion quantum mechanics}

Adler\rf{Adler95} seems to credit them at length. In his 1996 review of
Adler's book, Finkelstein writes diplomatically\rf{Finkel96}: ``Adler
does cite a gauge-invariant quaternionic theory.Its imaginary operator
I varies in spacetime and is then a natural Higgs field. But this
theory does not account for color SU(3), the one-handed neutrino and
several other peculiarities of nature that seem to belong together.
Adler quite reasonably shelves this theory. Indeed, the gaps in this
theory led me to shelve the whole quaternionic project in the 1960s.
Adler suggests a quaternionic route that might close these gaps.''

\item[2016-02-21 Predrag]
A man wrote a whole PhD thesis\rf{Familton15}
    on {\em Quaternions: A History of Complex Noncommutative Rotation
    Groups in Theoretical Physics}, without mentioning  Finkelstein
    \etal\rf{FJSS62,FJS63,FJSS63} once. How is that possible? But the
    thesis looks iffy to me (PhD in Mathematics Education), so we'll
    let that one pass.

Hestenes\rf{Hestenes66} {\em Space-time Algebra}, not Jauch or
Finkelstein\rf{FJSS62,FJS63,FJSS63}, is apparently credited with the
invention of quaternionic quantum mechanics. They published first, but
there are no mutual citations - it's only much later Finkelstein cites
Hestenes, but no citations in the reverse direction.

Weird thing is that I have never heard of Professor Hestenes until
today, when I checked him out to find why he does not credit
Finkelstein \etal\rf{FJSS62,FJS63,FJSS63} for their quaternion quantum
mechanics. Neither does De Leo\rf{DeLeo96}-- but Finkelstein had cited
him in more recent papers.

Now, when I think of it, I think we do have a grad student who insists
computing everything in terms of bi-vectors, driving Zangwill to
distraction. Hestenes, of course, has yet another interpretation of
Quantum Mechanics\rf{Hestenes86}.

The idea of Euclidean / Minkowski spaces as limits of compact groups is
- actually - something Finkelstein strongly felt was the right thing to
do.

\item[2016-06-19 Predrag]
Finkelstein\rf{Finkelstein16} 2016
{{The Melencolia Manifesto}} ebook is available through Georgia Tech Library
for free. The blurb:

``Few artworks have been the subject of more extensive modern
interpretation than Melencolia I by renowned artist, mathematician, and
scientist Albrecht Dürer (1514). And yet, did each of these art experts
and historians miss a secret manifesto that Dürer included within the
engraving? This is the first work to decrypt secrets within Melencolia I
based not on guesswork, but D\"urer's own writings, other subliminal
artists that inspired him (i.e., Leonardo da Vinci), the Jewish and
Christian Bibles, and books that inspired D\"urer (De Occulta Philosophia
and the Hieorglyphica). To read the covert message of Melencolia I is to
understand that D\"urer was a humanist in his interests in mathematics,
science, poetry, and antiquity. This book recognizes his unparalleled
power with the burin, his mathematical skill in perspective, his
dedication to precise language, and his acute observation of nature.
Melencolia I may also be one of the most controversial (and at the time
most criminal) pieces of art as it hid D\"urer's disdain for the hierarchy
of the Catholic Church, the Kaiser, and the Holy Roman Empire from the
general public for centuries. This book closely ties the origins of
philosophy (science) and the work of a Renaissance master together, and
will be of interest for anyone who loves scientific history, art
interpretation, and secret manifestos.''

\item[2017-03-27 Dean Rickles]
author of {\em {David Finkelstein} interview}\rf{Rickles17} writes:

I just found
\HREF{https://drive.google.com/file/d/0B4hs0-ue2dXQdm14R0JqNmdzQVU/view?usp=sharing}
{this 1980-06-13 report} of Wheeler's for NSF funding for David. Wheeler
writes ``A country that cannot afford a Finkelstein is a poor country,''
and that if ``the flag is to be kept flying'' the NSF should fund his
work.

[Predrag: \HREF{http://publish.uwo.ca/~valluri/aboutme.html}
{Sree Ram Valluri} seems to have prospered as Assistant Professor in
Canada, notwithstanding Wheeler's assessment.]




\end{description}

\section{Plasma physics}
\label{sect:plasma}

\begin{description}

\item[1962-??-??]
November 1962 Plasma lab meeting photo (Shlomit has it)

\HREF{https://en.wikipedia.org/wiki/James_R._Powell} {James R. Powell}
I sent email to info@maglev2000.com, no response.

\item[1964-01-09 Finkelstein and Rubinstein] {\em Ball lightning}. 	David
considered their paper\rf{FinRub68} the first calculations of ball
lightning, though consensus is that it remains an unexplained
\HREF{https://en.wikipedia.org/wiki/Atmospheric_electricity} {atmospheric
electrical phenomenon}.

The long lifetime of ball lightning suggested that perhaps there is a
confinement mechanism at work. He was in plasma, so he had to look into
it. He generalized the virial theorem to relativistic theory and showed
the pressure inside a configuration of plasma and electromagnetic fields
couldn't exceed the pressure outside. Russian physicists thought that
ball lightning might be a natural fusion reactor held together by its own
magnetic field. There was no significant confinement: ``If this model is
appropriate, then ball lightning has no relevance to controlled-fusion
plasma research.'''

He officially worked on plasma physics, at Yeshiva at a laboratory,
trying to make a relativistic pinch. A Budker machine, a linear one to
begin with.

\HREF{https://revolvy.com/main/index.php?s=David\%20Finkelstein}
{revolvy.com obit}:
``He investigated ball lightning with Julio Rubinstein and James R.
Powell. They concluded that ball lightning is most likely a wandering St.
Elmo's fire, a low-temperature soliton in the atmospheric electric
current flow.''

\item[2016-08-08 Susskind]
David had an experimental side. He was interested in the phenomenon of
ball lightning and built an apparatus in Yeshiva to demonstrate some
plasma effect. He ordered a bank of capacitors from some supplier. It
came packaged but the leads were exposed and without opening it he
charged it up. Unknown to him, the switches were bolted down and when he
tried to discharge it --BANG--. It blew a large hole in the ceiling.
Fortunately he used a long pole and was far enough away that he didn't
get blown up.
David also worked on solar physics and magnetic storms. He applied
topology to the structure of the magnetic fields. I don't know how
influential it was, but it sounded right to me when he explained it.

\item[1967-06-01 Finkelstein, Goldberg and  Shuchatowitz]
 {High voltage impulse system}\rf{FiGoSh66}

\item[1967-08-02 Finkelstein]
Woods Hole seminar with title ``Topology of Magnetized Fluids'' was
unforgettable. He gave Ed A. Spiegel a stack of transparencies, and told
him - just put them on the overhead projector in any order, one by one,
and I will give my talk accordingly. OK, thought Ed, and inserted some of
his own transparencies into the stack. And so it went, with Ed putting on
random transparencies and David delivering a talk on ball lightning.
\HREF{https://www.iau.org/administration/membership/individual/3485/}
{William C. Saslaw} credits
\HREF{https://darchive.mblwhoilibrary.org/bitstream/handle/1912/2927/WHOI-66-46_v2.pdf}
{conversations} with David.

\item[1967-06-02 Presby and Finkelstein]
{\em Plasma Phasography}\rf{PreFin67}

\item[1970-06-05 Finkelstein and Powell]
{\em Earthquake lightning}\rf{FinPow70}.

\item[2017-01-05]
Barry\rf{Barry13} {\em Ball Lightning and Bead Lightning: Extreme Forms
of Atmospheric Electricity}: ``It is the opinion of many
investigators that the experimental results of Powell and
Finkelstein\rf{PowFin69,PowFin70} were the most significant in the
history of this type of investigation.''

\item[1977-05-02]
Finkelstein and Weil\rf{FinWei78}
{\em Magnetohydrodynamic kinks in astrophysics},
``Three-dimensional magnetic kinks in nonresistive plasmas may be created and
annihilated in pairs and conserve their homotopy properties during their
lifetime. Such kinks could prove relevant to astrophysical, geophysical, or
laboratory plasma problems...'' is a basis for a thermonuclear fusion
patent\rf{Weil92}. The research was funded by the est foundation.

\item[2017-01-05]
Emailed to Michael Creutz at BML about Powell? No response...


\end{description}

\section{Quantum computing}
\label{sect:Qcomput}

\begin{description}

\item[1968-03-25 Finkelstein] {\em Space-time code}\rf{Finkel69,Finkel72}:
``Richard P. Feynman said space-time points were like digital computers. Peter G.
Bergmann, in a lecture, said the signature of spacetime must come from
[2$\times$2] Hermitian matrices. Roger Penrose said all along that everything
must be made of spinors. I came to believe them. It is not their
responsibility.''

\item[1969-01-22 Finkelstein]
{\em Space-time structure in high energy interactions}\rf{Finkel69a}
is the paper where David introduces quantum computers, as an extension of
his {space-time code}\rf{Finkel69,Finkel72}. See \refsect{sect:Finkel69a}.

\item[1981-05-06 Benioff]
 {\em Quantum mechanical {Hamiltonian} models of discrete processes that erase
 their own histories: {Application to Turing} machines}\rf{Benioff82}

\item[1981-05-07 Feynman]
{\em Simulating physics with computers}\rf{Feynman82}.

\item[1984-07-13 Deutsch and Penrose]
{\em Quantum theory, the {Church-Turing} principle and the universal quantum
computer}\rf{DeuPen85}.



\item[2016-12-20 Lawrence Schulman]  <lschulma@clarkson.edu> to his son
{\bf Leonard Schulman} <schulman@caltech.edu>:
When you were at GaTech you had
co-taught a course with Finkelstein (and had not been happy
with how the burden had been ``shared'').
Did anything Finkelstein said or did suggested that many of the
ideas of quantum computing had originated with him? I do know that Finkelstein
had gone into notions of quantum logic and had been a pioneer in many
areas. His ideas
often were way ahead of their time. So it's quite possible that he anticipated
ideas of quantum computing.

PS. Mom recalls his sitting on the floor guru-like in our house in Potsdam when
we had a dinner --with company-- for him. But when we went to his house he was
extremely gracious and actually behaved like a normal person.

\item[2016-12-20 Predrag]
Maybe you got to Yeshiva a bit late - the Physics Department started in an open
space above a garage (before the fancy Balfour building was built) and there
David always officiated from the floor, lotus-position. His followers did the
same - in particular a woman graduate student whose name I cannot remember.

\item[2016-12-20 Leonard Schulman]
Well, it was maybe not so much the burden that was not shared,
as the actual course... It was supposed to be (and was titled) a course
on quantum computing. But David had his interests (outlined in his
papers from the 60s and or so on quantum logic) and I had mine (which
was the then-new field of quantum computing) and I'm sure we didn't get
as much out of the interaction as we could have. If memory serves, we
eventually just alternated lectures, each proceeding on his own plan.

His quantum logic ideas did not have much to do with (modern) quantum computing,
at least, not that I know of.

The Feynman paper\rf{Feynman82,Feynman85} (which was from some keynote address) was
visionary in grasping the computational complexity aspect.

I don't remember encountering any of that in David's work.

That doesn't mean that the focus on the noncommutative aspects of quantum logic
wasn't a wise focus, or that he didn't say interesting things...it does mean that
to the best of my knowledge, the current river of research on those things has
not been much influenced by those papers. My knowledge is limited. Incidentally
some noncommutative models of computation have been studied since the early 90s
(people like Nisan and Wigderson), completely apart from any quantum motivation,
rather to do with circuit complexity.

\item[2018-12-04 Shlomit]
In \HREF{https://www.youtube.com/watch?v=o3hHO3S8Unk}
   {On Quantum Computing} interview
Preskill quotes Feynman, but no word of Finkelstein. This is what the Giuseppe
Castagnoli\rf{CasFin01,CasFin03,FinCas07,Castagnoli08} keeps complaining to me
that David's original and early contribution to quantum computing is not
acknowledged. Well... Paul A. Benioff\rf{Benioff82} gave David such
credit but never in writing.
In 1982 David organized a special conference at MIT about quantum computation.
Part I of it is covered in the 1982 issue of IJTP
\HREF{https://link.springer.com/journal/10773/21/3} {issue 3-4}. The other parts
were also published in IJTP but later.

BTW, this special issue might also suggest additional ``Bold Ideas'' speakers.

What about one of the authors in the
\HREF{https://link.springer.com/journal/10773/56/1} {memorial volume} edited by
Heinrich Saller\rf{Saller16}?

\item[2018-12-08 Predrag]
If Castagnoli keeps complaining, why doesn't he ever in his own papers
acknowledge the alleged David's original and early contribution to quantum
computing?
He credits\rf{Castagnoli08} David for ``introduction of the notion of
quantum bit and identification of computation in the quantum
framework\rf{Finkel69a}''.

\item[2019-02-04 Shlomit]
I am not sure of the quality of 2001 {\em  Theory of the quantum speedup}
paper by Castagnoli and Finkelstein\rf{CasFin01}.
David supported Castagnoli's work. % but I am not sure how highly thought of it:
The paper is behind Springer paywall (as seen from Georgia Tech), but the
abstract and the first 2 pages are visible: ``
Insofar as quantum computation is faster than classical, it appears to be
irreversible. In all quantum algorithms found so far the speed-up depends
on the extra-dynamical irreversible projection representing quantum
measurement. Quantum measurement performs a computation that dynamical
computation cannot accomplish as efficiently.
''

\item[2019-02-14 Daniel Murphy]
I got so side-tracked reading into David's discussions and works on
quantum logic and quantum information that I have not yet had time to
write, but I will draft the revised panel tomorrow. It should not take
long. I have read more than I probably need to in order to write it.

Never-the-less, I am very enthusiastic about this. I think this is a very
good opportunity to highlight his older works that seem to have gotten
swept under his legacy of quantum spacetime ventures.

Given the word ``quantum algorithm" was not used before the 1960's he was
surely one of the first people to use it in his lectures on space time
code and space time structure. He discussed quantum information
processing nearly 15 years before Feynman in 1983 and Deutsch in 1985,
which are some of the earliest quantum computation works generally cited
in the modern QC literature,  besides Von Neumann's contributions.

His ``space-time code" ideas discuss spacetime resulting from
quantum-logical structures, and are therefore very different than the hot
topic that CFTs are embedded in an ``error-correcting code''  in AdS.
However, I am pretty sure he seems to really brings forth the ``it from
bit'' idea of John Wheeler nearly 20 years before anything was Wheeler
mentioned it in 1989.

He seems to have been a man ahead of his time when it came to some of
these ideas, and I think a quantum information lecturer is very
appropriate for a talk series commemorating him.

\item[2019-03-17 Shlomit to Colin Parker]
A few bullets, lots of homework:
\begin{itemize}
  \item
Read
1968 paper
Finkelstein {\em Space-time structure in high energy interactions},
% In Gudehus, T., Kaiser, G., Perlmutter, A. editors. Conference on high
% energy interactions, Coral Gables (1968)
\HREF{http://homepages.math.uic.edu/~kauffman/FinkQuant.pdf}
{(click here)}. His collaborator Castagnoli wrote to me:
''This is a seminal paper. It shows for the first
time that computation is feasible in the (reversible) quantum framework
and, most importantly, introduces the notion of quantum bit and quantum
information, encoded in the state of respectively a spin 1/2 particle and
a chain thereof. The very foundational notions of quantum information are
in it, for the first time.''

Castagnoli also wrote passionately that the positioning of
the seminal work of David's 1968 work would be well described by the
following paragraph, from Paul Benioff
\HREF{https://en.wikipedia.org/wiki/Paul_Benioff} {Wikipedia page}, if it
were not for the fact the priorities are wrongly ascribed to Paul
Benioff:

\begin{quote}
In the 1970s, Benioff began to research the theoretical
feasibility of quantum computing. His early research culminated in a
paper, published in 1980 that described a quantum mechanical model of
Turing Machines\rf{Benioff80}. This work was based on a classical
description in 1973 of reversible Turing machines by
Bennett\rf{Bennett73}. Benioff's model of a quantum computer was
reversible, and did not dissipate energy\rf{Benioff82a}. At the time,
there were several papers arguing that the creation of a reversible model
of quantum computing was impossible. Benioff's paper\rf{Benioff82} was
the first to show that reversible quantum computing was theoretically
possible. This work and other later work by several authors initiated the
field of quantum computing.
\end{quote}
I have the following comments about this Wikipedia page:
\begin{itemize}
   \item[(A)]  In it,
Benioff is considered to be the first to have introduced the notion of
quantum computation, which is not correct: Benioff's paper is
1980, that of David is eleven years earlier.
   \item[(B)]  David told me that he
had met Benioff at a congress, I do not remember
it could have been around 2010.
David told me that, in that occasion, Benioff explicitly told him to have
been inspired by his 1968 paper. However,
David's 1968 paper is not in the references of Benioff's 1980 paper.
   \item[(C)]
In spite of this, the Wikipedia page pinpoints two important things.
Before Benioff (a fortiori before David's 1968 paper), people thought
that quantum computation was impossible in the reversible quantum
framework. That thinking was due to a wrong
theorem by Brillouin (otherwise a great physicist) that
computation is essentially irreversible. This changed after the
works on reversible computation by Landauer, Bennett, Fredkin, and
Toffoli, but these works came after David's 1968 paper. Thus, David
was much ahead of his time in seeing the feasibility of computation in
the reversible quantum framework.
    \end{itemize}
   \item	
I think you will enjoy the easy and pleasant reading that gives some
sense of David's insistence on logic as a physical variable and his early
thoughts about computing in an interview that took place only two years
before his death. You can access it
\HREF{https://www.aip.org/history-programs/niels-bohr-library/oral-histories/40665}
{here}
or Rickles (2017) {\em David Finkelstein interview}\rf{Rickles17} in the
memorial issue of IJTP.
% International Journal of Theoretical Physics 56(1). Pp 40-87.
% I will be happy to meet with you and bring you a hard copy of this issue.
   \item	
A book chapter Finkelstein (1988) {\em Finite physics}\rf{Finkel88}. I
could not find a copy of this chapter and I do not have the book.
% \HREF{https://dl.acm.org/citation.cfm?id=213990&picked=prox} {TOC} of the
% book. David's chapter is pp. 323-347.
It is referenced in, for example,
Peres and Terno {\em Quantum Information and Relativity Theory}\rf{PerTer04},
Sorkin {\em Finitary substitute for continuous topology}\rf{Sorkin91},
and
Sorkin
{\em Forks in the road, on the way to quantum gravity}\rf{Sorkin97}.
   \item	
A very early work on quaternions with colleagues (1962), highly cited,
prior even to his ``Space Time Code'' series might also have some seeds
about computation. I have attached it and I find myself too remote from
physics to judge its relevance.
   \item	
I cannot judge the relevance of his very early insistence on logic as a
physical variable to quantum computation. The very early publication
Finkelstein (1965)
{\em The logic of quantum physics}\rf{Finkel63}
%Trans. New York Acad. Sci. 25, 621.
is behind a paywall.
   \item	
I also cannot judge the relevance of David's work on spin and statistics
to quantum computation. Here are several references to David's work on
this topic:
\begin{itemize}
   \item	
Duck and Sudarshan\rf{DucSud98}
{\em Toward an understanding of the spin-statistics theorem}
where we learn that Feynman learned the famed belt trick (illustrating
$\pi$ spin 1/2 rotations by twisting a belt) from
Finkelstein\rf{Finkel55,FinRub68,Schulman68}.
   \item	
Finkelstein (1955)
{\em Internal structure of spinning particles}\rf{Finkel55}.
%Physical Review,100, 924-931.
   \item	
Finkelstein (1966) {\em Kinks}\rf{Finkel66}.
%Journal of Mathematical Physics,7, 1218–1225.
   \item	
Finkelstein and Misner (1959)
{\em Some new conservation laws}\rf{FinMis59}.
% Annals of Physics, 6, 230–243.
   \item	
Finkelstein and Rubinstein (1968)
{\em Connection between spin, statistics, and kinks}\rf{FinRub68}.
% Journal of Mathematical Physics,9, 1762–1779.
    \end{itemize}
\end{itemize}



\end{description}

\newpage
\section{Space-time structure in high energy interactions}
\label{sect:Finkel69a}

\noindent
%{\em Space-time structure in high energy interactions}\rf{Finkel69a} \\
D. Finkelstein$^*$
\\
Belfer Graduate School
of Science, Yeshiva University, New York, New York
\\{\footnotesize
* Young Men's Philanthropic League Professor of Physics.
  Supported in part by the National Science Foundation.
  }


\medskip

\hfill{\footnotesize
{\bf [2018-12-09 Predrag]} A scan of Finkelstein \refref{Finkel69a},
not proofread yet.
}

\bigskip\bigskip

The preceding two talks were motivated by the courageous faith that our present
ideas of the continuum and the gravitational field extend into the range of
elementary particle sizes and far below. Equally interesting to high-energy
physicists is the possibility that these ideas of space and time are already at
the very edge of their domain and are wrong for shorter distances and times, and
that high-energy experiments are a probe through which departure from the
classical continuum can be discovered. The present talk is devoted to this
alternative. The difficulty is that all our present theoretical work is based on
a microscopic continuum and one is faced by the rather formidable problem of
re-doing all physics in a continuum-free manner. Yet I think those who cope with
the conceptual problems of quantum field theory for enough years eventually get
sick of the ambiguities and divergencies that seem to derive from the continuum
and are driven to seek some way out of this intellectual impasse. I would like to
describe a program of this kind that I have been led into after vainly trying to
extract some information.· about the small from extremely non-linear field
theories.

The starting point here also is Riemann, who explicitly poses the question of
whether the world is a continuous or a discrete manifold in his famous inaugural
lecture. He points out certain philosophical advantages of the discrete manifold,
in fact argues for it more strongly than for the continuous, and yet devotes his
life to the continuous. Why?

I think primarily on grounds of simplicity. Why is the continuous simpler than
the discrete manifold? For one thing because it has more symmetries available to
it. A continuum can have rotational symmetry, the world has rotational symmetry,
a checkerboard cannot have rotational symmetry. Today, however, there are more
options open to us than there were to Riemann. Faced with the apparent dilemma of
the discrete and the continuum, our experience with quantum theory leads us to
plunge boldly between these two with the synthesis called quantum. By a quantum I
mean an object whose propositional calculus is isomorphic to the lattice of
subspaces of a Hilbert space$^1$. In present day quantum field theory, quantum
concepts are injected at the top. We make up a complete classical picture of the
world, a geometrical and dynamical structure which could in principle describe a
real world, and then we take this beautiful theory and amend it by quantization.
Is it not possible that in fact quantum concepts belong in the foundation? In
particular, that instead of taking geometry and quantizing it, so to speak, we
should take quanta and geometrize them? Should we not try to make a space-time
theory in which from the start the elementary objects which make up the
space-time are described by quantum laws and the space-time itself is assembled
out of these by the quantum-logical procedures that we have mastered already in
the quantum many-body problem?

I was forcefully introduced to this whole idea by Feynman some eight years ago.
He didn't believe in continuum then. (He still doesn't believe in the continuum.
See his {\em Character of Physical Laws})$^2$. He suggested that a reasonable
model for the world is a computer, a giant digital computer. The things we call
events are processes of computation, and the fundamental fields represent stored
information. The continuum theory of the world is totally absurd from this point
of view. It imparts to each point of space-time an infinite memory capacity, in
that an infinite number of bits are required to define a fundamental field like
the electric vector potential. It takes an infinite channel capacity to
communicate these numbers from one point of space to another. An infinite number
of computations must be done by the computing element at each point to work out
the field equations and pass on the output to the future.

The idea that the world is some kind of computer is not far removed,
incidentally, from the idea that the world is some kind of brain; which is not
far removed from the idea that the world is the mind of God; which stems back to
the hermetic doctrines of the second century B. C. (I am indebted to Professor
Jauch for this reference) so we are working in an ancient and honorable
tradition.

The question is, however, can we make a computer which is compatible from the
start with the principles of Quantum Theory and Relativity; and the first step to
do this is to set up a little dictionary in which we translate the basic terms of
physics into discrete or computer terms, instead of into continuum terms as we
have had so much practice in doing. It will take a little more than half an hour
to present a detailed dictionary of this kind. Let me simply indicate the spirit
in which this can be done and the fact that to my surprise there seems to be no
definite obstacle. I will indicate just a few of the possible models toward the
end of my talk but I have the impression that I have wandered into open meadows
where there are as yet no fences and unlimitable expanses of grass in which to
graze.

To begin the dictionary, it's helpful to organize the structure of physics into
three groups of concepts:

At the top, dynamics -concepts clustering primarily about action, in which things
like charge and mass figure.

In the level of dynamics we move freely with geometric concepts which make up the
next lower level, in which the fundamental single quantity is probably the notion
of time, and in which we also have such basic things as space-time events, a
relation of causality, and the speed of light.

In dealing with both the dynamic and geometrical levels of physics we work with
the tools of logic, which make up the deepest level. In a sense this program is
an attempt to reduce all physics to logic. There is not that much difference
between a logical recursion procedure and a finitary computer.

Now I indicate briefly the manner of translation at each of these levels, the
logical, geometrical and the dynamical, in a way which I think makes the
procedure fairly easy to follow for anyone else who cares to retrace my steps.

The first thing is to make our computer out of quantum elements. Computers are
ordinarily made out of binary elements, bits, zeros and ones. Let us simply
suppose that instead of dealing with these by the tools of ordinary Boolean
algebra we use the algebra of subspaces of Hilbert space. In more familiar terms,
if a bit can take on the values O and 1 it can also take on coherent
superpositions of these states. In brief, the natural building block for a
quantum computer is the spin-$\half$ theory, as a natural building block for a
classical computer is the set with two elements$^3$.

Table I indicates in fair detail how the processes of propositional calculus which figure in the synthesis of computers in modern automata theory are to be translated from Boolean algebra into Hilbert space theory. Perhaps the only element of novelty involved is the need to go beyond the lowest-order propositional calculus. We make up computers out of things like words, sequences of bits, and we must look forward to making a path out of sequences of quantum elements for example, so we need the prescriptions for making such assemblages out of individuals.

Putting it differently, we need to go deeper into the propositional calculus and deal with systems having internal structure.

This is dealt with in a higher order or predicate calculus in ordinary logic, but in quantum mechanics we encounter the same problems every time we do many-body theory. We know how, given a theory of two objects, that is two Hilbert spaces, to make a theory of the pair, which is one of the basic steps in the synthesis of computers. Another of the important operations is going from an object to a new one called an arbitrary set of such objects. This is sometimes called the star process in automata synthesis.
The corresponding procedure in set theory is going from a set S to $2^S$. In a case where an object is described by a Hilbert space J1 the immediate obstacle is: what do we mean by two to the power of a Hilbert space? In fact, there is a beautiful correspondence between the laws of set theory and the laws of the exterior algebra over a Hilbert space. This is the algebra that comes in whenever we do the theory of many fermions.

One way to express what I'm saying now is: when we learn that, say, the electron obeys Fermi-Dirac statistics, we can express this knowledge by saying the fundamental object of electron. theory is a set of electrons; not for example a sequence of electrons, a basically different logical construct in which order is important, nor what I call a series of electrons, in which multiple membership is admitted. For a set, an element is either in it or isn't, and can't be in it twice, and this is reflected in the Fermi­Dirac statistics of electron theory.

Another important ingredient in the synthesis of automata is the theory of relations. A relation is a property of several objects. Having understood properties as subspaces of Hilbert space, and having understood several objects as meaning multiplied Hilbert spaces, there is no difficulty in formulating a quantum theory of relations. Some concepts of the classical theories do not admit direct quantum translations because of complimentarity. In particular the notion of a partial ordering, which is fundamental to the usual theory of automata, does not translate directly and I found it convenient to replace it by the notion of a precedence relation, an anti-symmetric transitive relation. In classical logic there is no particular advantage to using one rather than the other, but the definition I've just given of a precedence relation (the anti-symmetry and the transitivity) admit immediate formal extension into the theory of Hilbert spaces. By sticking to the things that possess such immediate translation, we guarantee a kind of correspondence. We know that when we go to a classical limit we will reconstruct the concepts of classical logic: it is qnly a question of neglecting commutators.

So much for the bottom leve1$^4$. The idea in each level is to reduce the concepts of the level to the smallest number of most operational concepts and translate them, in the hope that if we get their formal properties right, all else will follow. For example, instead of dealing with the whole continuum structure of wave functions, probability amplitudes, inner products, and the metric structure of Hilbert space, it was the yes-or-no structure, the implication relations PC:Q, the relations defining the lattice of a Hilbert space
that we singled out at the logical level. What are the
corresponding things at the geometric level?

All the concepts of space and time can be expressed also in a theory of a
partially ordered set. (It's odd that the same tool should work twice.)
Space-time can be regarded as a causal measure space: that is to say, all the
metric concepts can be expressed in terms of two things: the measure of a
space-time volume, and the relation of causality. For example, the distance
between two events can be defined in terms of the measure of the causal interval
between the two events. The topology of space-time can be defined in terms of a
system of neighborhoods consisting of causal intervals. In fact, I would say that
relativity gains by this translation in that the theories of its entire
structure, from the topological level up to the level of causality and geodesics,
can all be expressed in terms of these two things in analogy with the way we
develop the corresponding theories of the one-dimensional time axis; whereas in
the ordinary development the topology and the metric structure are somehow
divorced in that we do not use the things corresponding to spheres in general
relativity to define this topology. It would give rather odd results if we said
the distant stars are in topological contact with us just because we receive
null rays from them.

So: Space-time is a causal measure space$^5$. Where in the theory of quantum automata
do we find the corresponding concepts? Right on the surface, waiting for us to
grasp them.

First, the rule for translating measure is derived from the lower level. Measure
is basically a logical concept in the theory of automata: you just count
processes. So I will consider that our causal measure space is to have a measure
derived by counting. In the quantum theory counting is done by the trace
operation. Statements of location in space-time in a theory of this kind are
represented in Hilbert space, just as statements of location in phase-space of
classical mechanics are represented in Hilbert space in quantum mechanics. The
only thing that remains is to specify a causal structure.

Where in automata do we find the causal relation but in the fact that some
computations must take place before other computations, in the relation of
\emph{logical dependence}. In fact, in von Neumann's beautiful comparison of the
computer and the brain, one comes very close to discussing the geometry of the
brain or of the automaton in terms of just the one notion of what I will call
logical precedence. The measurements that von Neumann makes on computers and
brains he expresses in terms of arithmetic depth and arithmetic breadth.
Arithmetic depth is the maximum number of logically dependent processes.
Arithmetic breadth is the maximum number of logically independent, concurrent
processes in the chain. For the computers of his day, these integer measurements
of time and space were "'108 for time and ~1 for space. Today computers are still
essentially one dimensional, having perhaps a logical breadth of $\approx lOXXX$. For the
brain it's about 10 in both directions. You might say that man is a
two­dimensional creature. If we wish to model space-time we will have to think
even more in terms of such highly parallel or asynchronous computational models.
Each event in space-time is somehow a calculation going on independently of those
that occur in spacelike surfaces relative to it, and if you push the duration of
the fundamental step down below 10XXX-cm for safety, then the arithmetic depth of
the universe is at least 1040 at present; and the arithmetic breadth of the
universe at least 10120 at present, the cube of the former number of course,
expressing the four-dimensional nature of the computational process that we must
seek to model. The point is, given these two basic notions of cause and measure
for automata a complete logical theory of the geometry of automata can be worked
out.

Let me just mention two examples, one motivated entirely by the idea of a logical
model without any consideration of relativistic invariance, and then a
modification of this to make it exactly Lorentz invariant.

Let's consider the simplest kind of serial computation, which I call the binary
code. Suppose we start from a single kind of binary digit representing two
alternatives which you can think of as O, a move forward in time and a step to
the right or 1, a move forward in time and a step to the left in a kind of
checkerboard diagram. We build a path as a sequence of such things, so let me
call this basic thing out of which we will assemble space-time a link. Let us
pass to the quantum .·· theory by describing the link A by a two-dimensional
Hilbert space. The notion of a path TI is then a well-defined Hilbert-space
concept, a quantum sequence of such links: $TI= seq A$ For the end-point of the
path, it suffices simply to ignore order, to say that two paths correspond to the
same point in space-time, have the same end-point, if they differ only by a
permutation of their links. The Hilbert space describing the point of space-time
p is then gotten from the Hilbert space describing the path TI in this primitive
space-time by a symmetrization procedure that leads from the usual direct product
to the familiar Bose-Einstein quantization. A point of space-time of this model
is a Bose-Einstein ensemble or series of two-state objects, links described by a
two-dimensional Hilbert space: p = ser A.

I've given the lowest level of this model. The next thing is the causal structure: what does it mean to say one such ensemble p comes before another p in the assembly
l 2
process? The simplest procedure is to say that p C p if
l 2
the number of links of each kind in p is smaller, so that
l
you can get to p , intuitively speaking, by just adding
2
more links and not subtracting. This can be expressed in
terms of quantum symbolic logic quite trivially. Considering
two points p, p and a kind of link 6 , for each point
l 2
there is a number operator n (1), n (2). Then the primitive
model for the causal relation is
\[formula\]

If we go over to the classical limit of this quantum theory, which I do very
childishly just by dropping the commutators, we obtain a classical causal measure
space S, therefore a classical geometry which we can then look at as a
geometrical object in itself. What is its structure? Its structure is the future
null cone N+ of special relativity with exactly the familiar Minkowsky measure
and causal ordering together with what has to be counted as an internal
coordinate, a single angle:
\(
S=N^+ \times S^l
\,.
\)

From the future null cone $N^+$ it is of course a triviality to assemble all of
space-time by very simple formal procedures. In particular, two words in the
binary code provide us with enough material to flesh out the convex closure of
the null cone. The theory is not exactly covariant because the commutation
relations of two harmonic oscillators, which is what we are dealing with here are
not invariant under $SL_2$ even though the causal ordering C is invariant under
$SL_2$ The non-covariance of the commutation relations disappears in the
classical limit. That's why we end up with a Lorentz invariant space. We then are
faced with a decision between two models; one covariant only in the classical
limit, which suggests that if we look at fine enough regions in space-time it is
not inconsistent to imagine that departures from special relativistic covariance
show up; the other, exactly covariant at all distances, obtained by replacing the
commutation relations for two harmonic oscillators by the commutation relations
of the Majorana representation. If you like you can say that what we have here is
a new interpretation for the Majorana representation and algebra. The idempotents
in it can be regarded as statements of location in a space which could be
regarded as a quantized null cone in which the covariant relation $p_1 Cp_2$
within the Majorana algebra plays the role of the causal order. So I call this
exactly invariant theory the Majorana space-time. Two of these models suffice to
make up a four-dimensional space-time, but when you double up on the number of
external coordinates, you also multiply the number of internal coordinates. It
turns out that the full structure of the algebra of the automaton· generating
what I would call two words in the binary code is that of the solid future cone
c+ multiplied not into $S^1$ but into
$U(2,C)=S^1 \times SU_2$.
These are internal coordinates in the simple sense that changing the values of
these coordinates does not produce any causal separation.

Let's ascend to the dynamical level, where the action is. What is the form in
which the laws of nature should be expressed in such a space-time? Guided by our
success in dealing with partial orderings of the two lower levels, it is
suggested that we try and express the laws of mechanics also by the theory of a
partially ordered set. (It is encouraging that in the theory of thermodynamics,
which really belongs to the same level as dynamics, such a formulation in terms
of partial ordering is actually more unified and beautiful than that, say, in
terms of an entropy principle, or the more classical ones. All of the
thermodynamics has been expressed as a theory of a partially ordered set in which
the objects are the states and the partial ordering is the relation of the
existence of a natural process going from one state to another. The entropy
appears as a valuation of this partial ordering in the same way that time appears
as a valuation of the causal ordering and measure appears as a valuation of the
implicative ordering of the previous two levels$)^6$. Feynman has exhibited a
model of quantum theory in a discrete space, which is a big step towards a
quantum space. It is in fact a two-dimensional model of the Dirac equation. If
you think of a two-dimensional space-time in the form of a checkerboard, if you
suppose that only the black squares can be occupied as in the game of checkers
and that a man can step forward to the right or step forward to the left, then
again a path is obviously a binary sequence, and the elementary link is a binary
digit. Feynman pointed out that the Dirac equation on this simple model could be
derived from the law that the transition amplitude for a path is $(im)^R$ where R is
the number of reflections of motion along the path$^7$. The difference form of
this law is the statement that the amplitude $\Psi$ for the pattern

\bigskip\bigskip

is a superposition of the amplitudes for two preceding patterns with a certain
coherent phase:

\bigskip\bigskip

This is in fact the Dirac equation in this space, although it is not entirely
familiar looking. In the case where we go over to wave-functions which change
slightly over a single square, so that their changes can be accurately
represented in terms of derivatives, the first two amplitudes give
\(
\alpha^\mu\partial_\mu
\)
and the third is just the mass term in the Dirac equation. The mass is here given
an immediate numerical interpretation as the probability amplitude of jitter
(zitter) per chronon, the probability amplitude for reflection of motion in one
unit of time on this lattice. The argument in the amplitude $\Psi$ is not merely a
point in space, but a point and a direction, which is sufficient to define a path
in the classical motion.

The space of paths of this model is the configuration space of the linear Ising
model and the quantity R in Feynman's transition amplitude is the pair
Hamiltonian for the linear Ising model. The generalization from two discrete
dimensions to a covariant model is simply to replace the Ising model by the
linear Heisenberg model, the Ising Hamiltonian operator by the Heisenberg pair
Hamiltonian.

But I must stop now.

\bigskip\bigskip

TABLE I. CONCEPTS OF QUANTUM LOGIC
\\
I. Propositional System
Concept System, Object Proposition Implies Identity proposition
Null proposition
Or, adjunction And, conjunction Not Disjoint Measure Quantity, coordinate,
variable Function of a quantity
Propositional function of a quantity
Point
Notation a,b, ... , A,B ...
P,Q,

Representation
subspace inclusion the Hilbert space
the zero vector
span intersection orthocomplement orthogonal dimension
operator
cf. functional calculus
"predicate" singlet, pure state


II. Calculus of Propositional Systems Sum, disjunction
direct sum Product
direct product Theory of Binary Relations
Relation Transpose Antisymmetry Transitive Function, Mapping

Assemblies Set of a's
Series of a's
Sequence of a I s R, aRb RT,aRTb:bRa exchange subspace

R.LRT aRb nbRcC aRc
F: (PC: Q) c:: (F (P)c CF(Q)),F(P) l
set a, 2a (F .D) ser a (B. E.) seq a (M.B.)
linear transformation
exterior algebra on
Ia
symmetric tensors on
Ia
tensors on
Ia

Concept  Notation  Representation
\\
Quantifiers

Numerical Universal Existential  NP, Na Pa nP, na Pa,Va Pa UP, Ua Pa,3a. Pa
\\

TABLE II, CONCEPTS OF GEOMETRY OF LOGICAL NETS
\\
Concept Notation Representation
\\
Point (Event) p Computational step \\
Measure (of point set) Pl Cardinality
\\
Causal precedence pCp' Logical precedence \\
Time-like path 7T Maximal well-ordered set \\
Space-like surface L Maximal non-ordered set
\\
Metric p(p,p')  P" | pep' cp''
\\

REFERENCES
1.
J.M. Jauch, Foundations of Quantum Mechanics, Addison­Wesley, 1967.

2.
R.P. Feynman, The Character of Physical Laws, M.I.T. Press, 1967.

3.
See J. Hartmanis Lectures on Automata Theory, Tata Institute, Bombay, 1968 and J. von Neumann, The Computer and the Brain, Yale UP, 1958.

4.
For further references see D. Finkelstein, The Physics of Logic, IC/68/35, International Center for Theoretical Physics, Trieste, 1968.

5.
For related studies of quantized geometry see especially

H. Snyder, Quantized Space-Time, Phys. Rev. 79,38 (1947);
C.F. v. Weizsacker, E. Scheibe and G. Sussman, Komplementaritat und Logik III, Zeits, f. Naturforschung 13a, 705 (1958); and C.F. Weizsacker, Quantum Theory and Beyond (preprint). For further references see D. Finkelstein, The Space-Time Code, IC/68/19, International Center for Theoretical Physics, Trieste, 1968.

6.
R. Giles, Mathematical Foundations of Thermodynamics, Pergamon Press (1964).

7.
See R. P. Feynman and A. R. Hibbs\rf{Feynman65},
{\em Quantum Mechanics and Path Integrals}, McGraw-Hill, 1965.

\bigskip\bigskip

DISCUSSION

JAUCH The exposition of Professor Finkelstein offers such dazzling possibilities
that if I make a simple question or two it will not do justice to all the
richness of what he has presented us; but I would like to make one statement and
one question.

First, the motivation of all this was given in terms of Riemann's question about
the replacement of the continuum by discreteness. However, when you start out you
throw continuity out the front door and it comes in the back door again, namely
through Hilbert space. Hilbert space, of course, is constructed with coefficients
from the continuum, and so continuity comes in; at the lower level to be sure,
but still it is there, somehow you will not be able to get rid of it, and you
have to live with it.

The second is a question, simply a technical question. did not quite understand:
did you get the three-plus-one dimensionality of space, or is that an input that
you have to put in?

FINKELSTEIN It was put in by hand when I said the binary code. If you take the
singulary code, you get a one­dimensional space-time consisting of nothing but a
time axis, plus a single internal degree of freedom having nothing to do with
causality, a kind of Newtonian world. If you wanted, say, a nine-dimensional
world you'd only have to use a ternary code with three basic characters.

As for the continuity, remember it is not discreteness which is the goal, but
finiteness. I don't ever want to have to do an integral again as long as I live.
I want to do nothing but finite sums, and if we work with finite dimensional
Hilbert spaces, we find nothing but finite sums to be computed, even though they
possess the full continuous symmetry group of, in this case, the Lorentz group.
Incidentally, I should mention that in this kind of a model, and in fact all the
ones I've exhibited, time is the number of chronons. This is also the
dimensionality of the Hilbert space in which one need work up to a given time. At
the present epoch there is no evidence then that one needs a Hilbert space of
dimension more than 1042 to describe all statements about location in space-time;
and if you want, say, to discuss sets of points in space-time as in el4~tron
theory, you're still down to a dimension of only 2(1o ) ' which is considerably
less than infinity.

COLEMAN It is not clear in the scheme precisely what one means when talking about
the present epoch. Clearly to speak of this being a certain time rather than some
other time is rather like a statement saying that the value of the electric field
at this point is a certain value rather than another value, since you have, so to
speak, not only quantized the field but quantized the argument of the field. Now
we know when discussing the electric field in conventional quantum mechanical
theories even though the current classical value of the electric field is some
certain value, we have to include the possibility of the value of the electric
field being arbitrarily large. So, I wonder if this happens in your scheme, and
if so, if you might not be led back to an infinite dimensional Hilbert space
after all.

FINKELSTEIN And the answer is, I don't fully know. I 1ve been worried about the
fact that the most primitive sorts of models that I've made up all possess a
cosmological origin. That's why I could speak of the time being finite, and so
forth. It's possible with some sweat to make up models which lack this and
then,not surprisingly, they operate in infinite­dimensional Hilbert spaces, so
again one is confronted with the danger of having to do at least an infinite sum.
Right now Iain more interested in finitenes.s than even in preserving
time-translation invariance. There's much more evidence for one than the other.

COLEMAN So your models have possessed Lorentz but not Poincar\'e covariance. Is
that correct?

FINKELSTEIN Right. But in the continuum limit, in one model, you do have the
solid future light cone, and as long as you translate within it, you have exact
Poincar\'e invariance -translation as well as Lorentz -but the continuum limit
breaks down.

COLEMAN No, but truly the Poincar\'e things are not unitarily implementable.

FINKELSTEIN Exactly. Back in the quantum theory there must be some remnants of
this invariance and I have not fully explored it yet. There is always available
something like the future time-translation. This is the semi-group, rather than
group, of future-time-like-translations by discrete amounts, multiples of course
of the fundamental constant T. The spectra that one gets for the coordinates in
the quantum theory before going to the classical limits varies slightly from
model to model, but typically, in spite of the exact Lorentz covariance, one
might have that t is an integer multiple of the fundamental constant, whereas x,y
and z possess purely continuous spectra.

WIGNER I am a good deal confused by a number of things that you said,
particularly about the invariance of the theory. You don't postulate Poincar\'e
invariance, but you do postulate Lorentz invariance and time-displacement
invariance, or did I mishear that?

FINKELSTEIN In fact, I've described several theories so that it's understandable
that the hypotheses could get garbled. In my initial work I postulated no
invariance at all. I simply looked for quantum models of binary computation
procedures, and was rather shocked to discover that the simplest non­trivial
model possessed Lorentz invariance in the classical limit. Then I noticed that
one could restore full Lorentz invariance in the quantum theory by a slight
change in the commutation relations and exhibit a whole class of other models.
These models still lack time-transitional invariance as unitary transformations,
as one might expect for a theory which contains only the future light cone. A
time-translation does exist, which is not unitary but an isometric linear
transformation in the quantum theory.

WIGNER I see, but you don't have time-translation invariance.

FINKELSTEIN Time-translation is not represented by a unitary transformation.
Time-translation by \emph{discrete} quantities is represented by an operation
something like the excitation operator for a harmonic oscillator; it doesn't have
an inverse.

WIGNER The question which is not terribly clear is simply this: if you have
time-translation invariance and Lorentz invariance, by the combination of the two
you have also invariance with respect to every other translation. Now, if one
assumes invariance in this way, unless·one restricts the group terribly strongly,
one obtains a dense manifold of Poincar\'e transformations. And this is almost
the same thing as true Poincar\'e invariance. You see, if I may talk about
distant past, the Poincar\'e group's representations were investigated. The
assumptions were not that there is space­time, only that there is invariance, and
that led somewhat disappointingly to space-time. Now naturally, one wants to
restrict somewhat the group, but there is no restriction of the Poincar\'e group
which is not everywhere dense and which contains ...

FINKELSTEIN Right. And of course this question of symmetry is crucial to any
model of this kind. It is the first thing one has to rub one's nose in. All that
I can say is that, no, we do not have the group of time translational invariance.
We have only a semi-group, in the models I've exhibited here. We don't have
inverses. We can go into the future, but not into the past. The representation of
the step into the future is also not by a unitary operator. So these are not even
unitary representations of semi-groups that we have here, but isometries, which
preserve the length but have no inverse. You can't go home again.

COLEMAN I can give an example that may clarify Professor Wigner's problem
although it does not have the full complexity of structure of Dr. Finkelstein's
model. If you consider the Hilbert space of all square-integrable functions whose
support is the interior of the forward light cone, that is a legitimate Hilbert
space. On that Hilbert space, in the natural way, Lorentz transormations act as
unitary transformations. They transform the points and induce a change in the
functions that doesn't change the measure. However if you consider translations
in the forward direction, these don't change the norm of functions, but they map
the full space into a subspace, and therefore do not have an everywhere defined
inverse. So therefore you have the Lorentz Group represented in the usual way,
but a subset of the Poincar\'e group, to wit translations with vectors that lie
in the forward light cone, is represented not by unitary transformations but by
isometries from the whole space into a smaller space. And that's sort of
group-theoretical structure that comes up here.

FINKLESTEIN Incidentally, it's typical of automata that for them the passage of
time is a semi-group rather than a group of transformations.


\newpage
\section{Bellissard notes}
\label{sect:Bellissard}

Jean notes, 1 May 2016

About Predrag's remark concerning the proposal of Finkelstein to define a
new logic, I may have a clue. Just a clue.

I am not sure that Finkelstein was correct, but that is how I understand
the claim. I just spent three weeks teaching the Section of the book of
Cover and Thomas on {\em Information Theory}, concerning the Kolmogorov
Complexity. I used the notes I prepared for the Fall semester, based on
the book by Feynman of the Theory of Computation, to also describe in
detail the concept and the functioning of Turing Machines, the
non-solvability of the halting problem. What I did not understand in the
Fall and I understood very clearly this time is that there is a compete
equivalence between the point of view of Turing, namely designing a
machine (how inefficient it can be) to execute the usual task of
computing, and the point of view of G\"odel, namely establishing the
rules of logic. A Turing Machine is nothing but executing all possible
rules of logic, not more but not less. As a consequence, the halting
problem and the incompleteness theorems of G\"odel are exactly the same
claim, the first one expressed in the language of machines, appealing to
engineers, the other one expressed in the language of logic, appealing to
(arrogant) mathematicians.


    Then comes a problem: Quantum Mechanics is expressible in terms of the
    mathematical of "Set Theory", and obey all rules of logic, as established
    over the centuries by logician since Aristotle. At this point, the formalism
    can be complicated, using Hilbert spaces and operator algebras, using the
    quantum operations to describe the measurement or the dissipative phenomena,
    but all these mathematical tools are just a consequence of the axioms of Set
    Theory.


    So, if Turing is equivalent to G\"odel, any quantum machine performing quantum
    computation should be implementable on a Turing Machine (!!), including the
    key aspect of quantum computers namely the "entanglement". The entanglement
    is indeed, in principle, the main and only advantage of quantum computers
    over classical ones. It provides a way to make massive parallel computation,
    in particular this is exactly why the Quantum Fourier Transform (QFT) is
    much faster than the FFT: while the later requires $O(N \ln N)$ operations to
    compute the Fourier transform of a function on the integers modulo(N), the
    QFT requires only $O(\ln^2(N))$ operations !! One of the most striking
    consequences is that factorizing a very large integer of of order N requires
    $O(\ln^3(N))$ operations only while the best algorithm known so far on
    classical machines requires $O(\exp{ \ln^3(N)})$ (may be I am wrong on this last
    estimate).


    If a quantum computer is implementable on a Turing machine, then we should
    expect dramatic consequences. One of them could be that apart from improving
    the speed of the calculation, there are problems impossible to solve on a
    quantum computer as well, and that the class of such problems might not be
    much larger than the unsolvable ones on a classical computer.


    Similar concerns have been expressed recently Gil Kalai\rf{Kalai06}
    (Mathematics Department, Hebrew University of Jerusalem) in {\em The
    quantum computer puzzle}. The argument being that quantum systems are
    fundamentally "noisy", a statement that has been expressed many times
    by physicists in a different way: they talk of the fundamentally
    probabilistic character of Quantum Mechanics.


    In other words, discretizing quantum information in terms of the concept of
    "qbits" or "qubits", is fine, but controlling these units of quantum
    information might appear problematic, especially when it comes to
    engineer logical quantum gates. When I started teaching Quantum Computing in
    2003, I was expecting this problem to be solved soon. But in 2016, no
    qualitative progress big enough has been made in the technology
    to manipulate more than 12-15 qubits, which was what people knew to do 15
    years ago already. It may appear more and more in the near future that this
    task is hopeless.


    This is where the proposal of David Finkelstein may come about: by
    expressing the rules of logic through Boolean Algebra, Shannon, in his PhD
    Thesis, made the breakthrough that was required to implements the idea of
    Turing and G\"odel. So, I suspect Finkelstein to have though that something
    equivalent should be done to express or to encode quantum information.  He
    is even more radical, since he would like to propose enlarging the rules of
    logic to include the concepts that should be at the basis of Quantum
    Mechanics.


    May be he was dreaming, may be he was right, who knows ? But there are
    several facts that are hardly accepted by physicists, in particular the ones
    claiming the existence of a "Theory of Everything". Namely the Kolmogorov
    approach t complexity implies in particular that a very limited amount of
    data produced by Nature will ever be accessible to science. These data must
    have a finite Kolmogorov complexity, namely they can be reproduced by a
    finite algorithm, expressed by a finite string of 0's and 1's. Chaitin
    showed that, using the Halting Problem, or the G\"odel Incompleteness Theorem,
    it becomes easy to build numbers (the famous number $\Omega$) that can be defined
    logically without ambiguity, but cannot be computed (if one except a small
    number of digits). He showed that the family of computable numbers is
    countable: hence this set is tiny. The same statement, expressed in terms of
    logic, means that among all logical claims that can be expressed through an
    infinite sequence of 0's and 1's, only a tiny proportion can be actually
    observed.  The tremendous success of theories produced by physicists over
    the centuries, such as Newton's Laws for Classical Mechanics, comes from the
    expression of these Laws into a very small set of equations, with
    consequences going way beyond the few examples that Newton himself, or the
    creators of other laws, investigated. Nevertheless, interpreting such laws
    as algorithms liable to be fed to a Turing Machine, the family of computable
    outcomes is automatically "small". Hence the family of data amenable to a
    physical interpretation is also automatically "small" compare to the
    information produced by Nature.


    To follow up on Finkelstein's claim, would require to check that the axioms of
    Quantum Mechanics are logically complete. This is what happened in the
    beginning of the 20th century: someone proved that the axioms of Arithmetic
    are logically complete. Then G\"odel used Arithmetic, the set of natural
    integers to reduce logic to a coding by integers and to prove the
    incompleteness of a set of axioms. If we could do something similar with
    either Quantum Mechanics, or with a simplified version sufficient to built
    the theory of quantum computation, then we could build a new logic with
    different rules, which could be implemented on a machine, that would allows
    to reach a much bigger set of data provided by Nature.


This is not the route followed by the physicists working on Quantum
Gravity.
\HREF{https://scholar.google.com/citations?hl=en&user=xmOSptwAAAAJ&view_op=list_works&sortby=pubdate}
{Preskill} is writing a book on Quantum Computing including topological
quantum computation and also a very new chapter on quantum information,
following the path similar to the one proposed by Shannon in 1948.

The axioms of Arithmetic, namely the axioms describing the natural
integers where proposed by Peano in 1889, after several previous
proposals. Russell accepted them as suitable indeed to describe the set
of natural integers. Poincar\'e argued that before accepting them they need
to be proved to be consistent. By consistent, he meant that there are no
theorem based on this family of axioms leading to prove that 0=1. The
first proof of G\"odel incompleteness theorem in 1931, is based on Peano's
axioms. But the first proof of their consistency was provided by Gerhard
Gentzen in 1936, using a method called "transfinite induction" which is
supposed to be encodable by a Turing Machine. G\"odel himself gave a
different proof in 1958.

If you think just a little, we just cannot do anything without the set of
natural integers. So, at least for them, a proof of the consistence of
the axiom describing the integers is required.

It is unclear whether such a work could be done for the part of Quantum
Mechanics that is relevant for quantum computing. We do not need the
whole formalism for quantum computing, since quantum information is
reduced to qubits. It is sufficient to describe axioms using the qubits
only and the corresponding quantum gates. Since I haven't followed the
literature in this direction, it may happen that such axioms are already
available. I have no idea whether such axioms have been proved to be
consistent. But if they are, they might be a way to describe quantum
computing in terms of a quantum analog of Turing Machines (many proposals
can be found on the web, but I have no idea whether they are as radical
as David Finkelstein had in mind).

The length of this note: This is what happens when you learn a topic and
you haven't yet mastered it. The longer the explanation the more likely
the author is to be confused with the topic.

\section{Finkelstein life}
\label{sect:FinkelsteinLife}

\begin{description}
\item[1964-08-28 John B. Garner to DF] UWiscArchive/640828Garner2DF.pdf

``
Last night The Northside Reporter, one of Hazel Brannon Smith's newspapers
was bombed. Damage is estimated at \$1,000 to \$1,500. I wonder how long she
will have to stop publishing . This completely cancels the \$1000 Pulitzer
Prize she received earlier this year. The paper is a small weekly and has a
liberal editorial policy and we desperately need it. This is the third
bombing in Jackson recently--the past six months I think. The first was a
barber shop that was charging below union rates and the second was a
partially finished Negro motel.
''

\item[1964-09-15  DF to John B. Garner] UWiscArchive/640915DF2Garner.pdf

``
Morton Schiff and I are flying down the 19th to start the term off
''

\item[1964-10-01  DF to Robert W. Morse] UWiscArchive/641001DF2Morse.pdf

``
Tougaloo College has a special significance for Mississippi and the entire
south. One recent example of this, among many, is the role of its faculty in
the formation of the Freedom Democratic Party.
''


\item[1964-10-29  DF to AIP] UWiscArchive/641029DFreport.pdf

seems to be a fun read

\item[1965-12-22]
In (1966) {\em Kinks} paper\rf{Finkel66} David writes: ``I am also grateful
to Tougaloo College for the hospitality afforded me during part of this
work.''


\item[1999-03-04] In the {\em The Life of the Cosmos} Lee Smolin\rf{Smolin99}
who met David Finkelstein by accident at a relativity conference
Smolin dropped in on in New York City during his first year of college. A few
minutes conversation served to incubate a lifetime of inspiration: ``
I struck up a conversation and asked him what he was working on. He replied that his
approach to physics was to imagine how God might have made the world, and then
to try to emulate Him. Having come to the conclusion that ``God cannot integrate, but
most likely he can count,'', he had constructed a game which described an electron
moving in a discrete world. I went away perplexed, without getting his name. It was
only many years later that I had met David Finkelstein, a man whose diverse contributions
to physics include the discovery of the meaning of black-hole horizons and the
notion of solutions in statistical physics. I think that I have never met a purer spirit in
my science work, and his lifelong search for a description of a world simple enough
for God to have made as been an inspiration to many seekers in the field of quantum
gravity.
''

\item[2017-01-04]
This is a total riot: poor David has been laboring on his Space Time Code
since something like 1947 and Googlette don't care:)


\HREF{https://www.linkedin.com/in/amrouhi} {Maureen Rouhi},
GaTech CoS Director of Communications (University of London
Ph.D., Chemistry, University of the Philippines
B.S and M.S., Agricultural chemistry; speaks Tagalog and Farsi)
wrote:
``FYI, I purposely omitted the terminology Space Time Code, because a
Google search brings this top response:

\begin{quote}
{\em A space–time code (STC) is a method employed to improve the
reliability of data transmission in wireless communication systems using
multiple transmit antennas.}
\end{quote}

\item[2016-09-13]
Alexander\rf{Alexander16}
{\em My journey into the physics of {David Finkelstein}}



\end{description}

%\newpage %%%%%%%%%%%%%%%%%%%%%%%%%%%%%%%%%%%%%%%%%%%%%%%%
\printbibliography[heading=subbibintoc,title={References}]
