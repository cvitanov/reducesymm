% reducesymm/QFT/Finkelstein.tex

\chapter{David Finkelstein}
\label{c-Finkelstein}

David Ritz Finkelstein (July 19, 1929 -- January 24, 2016)



\section{Finkelstein notes}
\label{sect:Finkelstein}

\begin{description}
\item[1958-05-15 David Finkelstein]

Finkelstein\rf{Finkel55}
{\em Internal structure of spinning particles}

\item[1958-05-15 The Paper]
Finkelstein\rf{Finkel58}
{\em Past-future asymmetry of the gravitational field of a point particle}

\item[1971-01-13 Jim Anderson] David was very hard to understand. In
colloquia first 15 minutes one could follow, then it got fuzzy. They
tried to write a book on general relativity, but could not agree on
anything beyond the first chapter. Eventually Anderson wrote the book by
himself.

They wrote a paper on the cosmological constant. As metric is irreducible,
determinant is geometric object. Imposing $\det g = 1$ as a Lagrange
multiplier adds the cosmological term as a Lagrange multiplier.

Anderson and Finkelstein\rf{AndFin71}
{\em Cosmological constant and fundamental length}: ``
In usual formulations of general relativity, the cosmological constant
$\lambda$
appears as an inelegant ambiguity in the fundamental action principle.
With a slight reformulation, $\lambda$ appears as an unavoidable Lagrange
multiplier, belonging to a constraint. The constraint expresses the
existence of a fundamental element of space-time hypervolume at every
point. The fundamental scale of length in atomic physics provides such a
hypervolume element. In this sense, the presence in relativity of an
undetermined cosmological length is a direct consequence of the existence
of a fundamental atomic length.''

In 2016 the paper has some 10 references (about 90 overall).

This work was continued in
Finkelstein, Galiautdinov and Baugh\rf{FiGaBa01}
{\em Unimodular relativity and cosmological constant}:

`Unimodular gravity', or `the Weyl-transverse gravity', is an alternative
theory of gravity considered by Einstein in 1919 without a Lagrangian and
put into Lagrangian form by Anderson and Finkelstein\rf{AndFin71}. It is
a restricted version of General Relativity in which the determinant of
the metric $-\sqrt{g}$ is a fixed function and the field equations are
given by the trace-free part of the full Einstein equations. Or,
unimodular gravity is just a gauge  fixed version of Einstein gravity,
with unimodularity can be imposed with a Langrange multiplier.  The
Bianchi identity forces the Lagrange multiplier to take the role of a
cosmological constant which now appears as an integration constant.


\item[1990-05-24 David Finkelstein] \texttt{900524PhysToday.pdf}, a
letter to Physics Today (unpublished) in response to Khurana\rf{Khurana89a}.

Wilczek\rf{Wilczek82} {\em Quantum mechanics of fractional-spin
particles} names `anyons' (since interchange of two of these particles
can give any phase), credits  Leinaas and Myrheim (1977) etc., but not
Finkelstein.

Wilczek and Zee\rf{WilZee83}
{\em Linking numbers, spin, and statistics of solitons}:
``Some aspects of this subject appear to have been anticipated in the
remarkable paper of D. Finkelstein and J. Rubinstein\rf{FinRub68}."

Misner, Thorne and Zurek\rf{MiThZu09}
{\em {John Wheeler}, relativity, and quantum information}:
``The physics becomes lucid in 1959-62, through insights by David
Finkelstein"
`` which dealt with only half of the Schwartzfield geometry and revealed
its horizon and thence its role as a black hole"

Finkelstein\rf{Finkel87} {\em The Quantum Paradox},
in {Encyclopaedia Britannica} {Yearbook of Science and the Future}
is a very helpful popularization of his view of quantum mechanics.

Finkelstein \etal\rf{FJSS62}
{\em Foundations of quaternion quantum mechanics}

Adler\rf{Adler95} seems to credit them at length. In his 1996 review of
Adler's book, Finkelstein writes diplomatically\rf{Finkel96}: ``Adler
does cite a gauge-invariant quaternionic theory.Its imaginary operator
I varies in spacetime and is then a natural Higgs field. But this
theory does not account for color SU(3), the one-handed neutrino and
several other peculiarities of nature that seem to belong together.
Adler quite reasonably shelves this theory. Indeed, the gaps in this
theory led me to shelve the whole quaternionic project in the 1960s.
Adler suggests a quaternionic route that might close these gaps.''

\item[2016-02-21 Predrag]
A man wrote a whole PhD thesis\rf{Familton15}
    on {\em Quaternions: A History of Complex Noncommutative Rotation
    Groups in Theoretical Physics}, without mentioning  Finkelstein
    \etal\rf{FJSS62,FJS63,FJSS63} once. How is that possible? But the
    thesis looks iffy to me (PhD in Mathematics Education), so we'll
    let that one pass.

Hestenes\rf{Hestenes66} {\em Space-time Algebra}, not Jauch or
Finkelstein\rf{FJSS62,FJS63,FJSS63}, is apparently credited with the
invention of quaternionic quantum mechanics. They published first, but
there are no mutual citations - it's only much later Finkelstein cites
Hestenes, but no citations in the reverse direction.

Weird thing is that I have never heard of Professor Hestenes until
today, when I checked him out to find why he does not credit
Finkelstein \etal\rf{FJSS62,FJS63,FJSS63} for their quaternion quantum
mechanics. Neither does De Leo\rf{DeLeo96}-- but Finkelstein had cited
him in more recent papers.

Now, when I think of it, I think we do have a grad student who insists
computing everything in terms of bi-vectors, driving Zangwill to
distraction. Hestenes, of course, has yet another interpretation of
Quantum Mechanics\rf{Hestenes86}.

The idea of Euclidean / Minkowski spaces as limits of compact groups is
- actually - something Finkelstein strongly felt was the right thing to
do.





\end{description}

\section{Plasma physics}
\label{sect:plasma}

\begin{description}

\item[1962-??-??]
November 1962 Plasma lab meeting photo (Shlomit has it)

\HREF{https://en.wikipedia.org/wiki/James_R._Powell} {James R. Powell}
I sent email to info@maglev2000.com, no response.

\item[1964-01-09 Finkelstein and Rubinstein] {\em Ball lightning}. 	David
considered their paper\rf{FinRub68} the first calculations of ball
lightning, though consensus is that it remains an unexplained
\HREF{https://en.wikipedia.org/wiki/Atmospheric_electricity} {atmospheric
electrical phenomenon}.

The long lifetime of ball lightning suggested that perhaps there is a
confinement mechanism at work. He was in plasma, so he had to look into
it. He generalized the virial theorem to relativistic theory and showed
the pressure inside a configuration of plasma and electromagnetic fields
couldn't exceed the pressure outside. Russian physicists thought that
ball lightning might be a natural fusion reactor held together by its own
magnetic field. There was no significant confinement: ``If this model is
appropriate, then ball lightning has no relevance to controlled-fusion
plasma research.'''

He officially worked on plasma physics, at Yeshiva at a laboratory,
trying to make a relativistic pinch. A Budker machine, a linear one to
begin with.

\HREF{https://revolvy.com/main/index.php?s=David\%20Finkelstein}
{revolvy.com obit}:
``He investigated ball lightning with Julio Rubinstein and James R.
Powell. They concluded that ball lightning is most likely a wandering St.
Elmo's fire, a low-temperature soliton in the atmospheric electric
current flow.''

\item[2016-08-08 Susskind]
David had an experimental side. He was interested in the phenomenon of
ball lightning and built an apparatus in Yeshiva to demonstrate some
plasma effect. He ordered a bank of capacitors from some supplier. It
came packaged but the leads were exposed and without opening it he
charged it up. Unknown to him, the switches were bolted down and when he
tried to discharge it --BANG--. It blew a large hole in the ceiling.
Fortunately he used a long pole and was far enough away that he didn't
get blown up.
David also worked on solar physics and magnetic storms. He applied
topology to the structure of the magnetic fields. I don't know how
influential it was, but it sounded right to me when he explained it.

\item[1967-06-01 Finkelstein, Goldberg and  Shuchatowitz]
 {High voltage impulse system}\rf{FiGoSh66}

\item[1967-08-02 Finkelstein]
Woods Hole seminar with title ``Topology of Magnetized Fluids'' was
unforgettable. He gave Ed A. Spiegel a stack of transparencies, and told
him - just put them on the overhead projector in any order, one by one,
and I will give my talk accordingly. OK, thought Ed, and inserted some of
his own transparencies into the stack. And so it went, with Ed putting on
random transparencies and David delivering a talk on ball lightning.
\HREF{https://www.iau.org/administration/membership/individual/3485/}
{William C. Saslaw} credits
\HREF{https://darchive.mblwhoilibrary.org/bitstream/handle/1912/2927/WHOI-66-46_v2.pdf}
{conversations} with David.

\item[1967-06-02 Presby and Finkelstein]
{\em Plasma Phasography}\rf{PreFin67}

\item[1970-06-05 Finkelstein and Powell]
{\em Earthquake lightning}\rf{FinPow70}.

\item[2017-01-05]
Barry\rf{Barry13} {\em Ball Lightning and Bead Lightning: Extreme Forms
of Atmospheric Electricity}: ``It is the opinion of many
investigators that the experimental results of Powell and
Finkelstein\rf{PowFin69,PowFin70} were the most significant in the
history of this type of investigation.''

\item[1977-05-02]
Finkelstein and Weil\rf{FinWei78}
{\em Magnetohydrodynamic kinks in astrophysics},
``Three-dimensional magnetic kinks in nonresistive plasmas may be created
and annihilated in pairs and conserve their homotopy properties during
their lifetime. Such kinks could prove relevant to astrophysical,
geophysical, or laboratory plasma problems...'' is a basis for a
thermonuclear fusion patent\rf{Weil92}. The research was funded by the est fundation.

\item[2017-01-05]
Emailed to Michael Creutz at BML about Powell?


\end{description}



\section{Bellissard notes}
\label{sect:Bellissard}

Jean notes, 1 May 2016

About Predrag's remark concerning the proposal of Finkelstein to define a
new logic, I may have a clue. Just a clue.

I am not sure that Finkelstein was correct, but that is how I understand
the claim. I just spent three weeks teaching the Section of the book of
Cover and Thomas on {\em Information Theory}, concerning the Kolmogorov
Complexity. I used the notes I prepared for the Fall semester, based on
the book by Feynman of the Theory of Computation, to also describe in
detail the concept and the functioning of Turing Machines, the
non-solvability of the halting problem. What I did not understand in the
Fall and I understood very clearly this time is that there is a compete
equivalence between the point of view of Turing, namely designing a
machine (how inefficient it can be) to execute the usual task of
computing, and the point of view of G\"odel, namely establishing the
rules of logic. A Turing Machine is nothing but executing all possible
rules of logic, not more but not less. As a consequence, the halting
problem and the incompleteness theorems of G\"odel are exactly the same
claim, the first one expressed in the language of machines, appealing to
engineers, the other one expressed in the language of logic, appealing to
(arrogant) mathematicians.


    Then comes a problem: Quantum Mechanics is expressible in terms of the
    mathematical of "Set Theory", and obey all rules of logic, as established
    over the centuries by logician since Aristotle. At this point, the formalism
    can be complicated, using Hilbert spaces and operator algebras, using the
    quantum operations to describe the measurement or the dissipative phenomena,
    but all these mathematical tools are just a consequence of the axioms of Set
    Theory.


    So, if Turing is equivalent to G\"odel, any quantum machine performing quantum
    computation should be implementable on a Turing Machine (!!), including the
    key aspect of quantum computers namely the "entanglement". The entanglement
    is indeed, in principle, the main and only advantage of quantum computers
    over classical ones. It provides a way to make massive parallel computation,
    in particular this is exactly why the Quantum Fourier Transform (QFT) is
    much faster than the FFT: while the later requires $O(N \ln N)$ operations to
    compute the Fourier transform of a function on the integers modulo(N), the
    QFT requires only $O(\ln^2(N))$ operations !! One of the most striking
    consequences is that factorizing a very large integer of of order N requires
    $O(\ln^3(N))$ operations only while the best algorithm known so far on
    classical machines requires $O(\exp{ \ln^3(N)})$ (may be I am wrong on this last
    estimate).


    If a quantum computer is implementable on a Turing machine, then we should
    expect dramatic consequences. One of them could be that apart from improving
    the speed of the calculation, there are problems impossible to solve on a
    quantum computer as well, and that the class of such problems might not be
    much larger than the unsolvable ones on a classical computer.


    Similar concerns have been expressed recently Gil Kalai\rf{Kalai06}
    (Mathematics Department, Hebrew University of Jerusalem) in {\em The
    quantum computer puzzle}. The argument being that quantum systems are
    fundamentally "noisy", a statement that has been expressed many times
    by physicists in a different way: they talk of the fundamentally
    probabilistic character of Quantum Mechanics.


    In other words, discretizing quantum information in terms of the concept of
    "qbits" or "qubits", is fine, but controlling these units of quantum
    information might appear problematic, especially when it comes to
    engineer logical quantum gates. When I started teaching Quantum Computing in
    2003, I was expecting this problem to be solved soon. But in 2016, no
    qualitative progress big enough has been made in the technology
    to manipulate more than 12-15 qubits, which was what people knew to do 15
    years ago already. It may appear more and more in the near future that this
    task is hopeless.


    This is where the proposal of David Finkelstein may come about: by
    expressing the rules of logic through Boolean Algebra, Shannon, in his PhD
    Thesis, made the breakthrough that was required to implements the idea of
    Turing and G\"odel. So, I suspect Finkelstein to have though that something
    equivalent should be done to express or to encode quantum information.  He
    is even more radical, since he would like to propose enlarging the rules of
    logic to include the concepts that should be at the basis of Quantum
    Mechanics.


    May be he was dreaming, may be he was right, who knows ? But there are
    several facts that are hardly accepted by physicists, in particular the ones
    claiming the existence of a "Theory of Everything". Namely the Kolmogorov
    approach t complexity implies in particular that a very limited amount of
    data produced by Nature will ever be accessible to science. These data must
    have a finite Kolmogorov complexity, namely they can be reproduced by a
    finite algorithm, expressed by a finite string of 0's and 1's. Chaitin
    showed that, using the Halting Problem, or the G\"odel Incompleteness Theorem,
    it becomes easy to build numbers (the famous number $\Omega$) that can be defined
    logically without ambiguity, but cannot be computed (if one except a small
    number of digits). He showed that the family of computable numbers is
    countable: hence this set is tiny. The same statement, expressed in terms of
    logic, means that among all logical claims that can be expressed through an
    infinite sequence of 0's and 1's, only a tiny proportion can be actually
    observed.  The tremendous success of theories produced by physicists over
    the centuries, such as Newton's Laws for Classical Mechanics, comes from the
    expression of these Laws into a very small set of equations, with
    consequences going way beyond the few examples that Newton himself, or the
    creators of other laws, investigated. Nevertheless, interpreting such laws
    as algorithms liable to be fed to a Turing Machine, the family of computable
    outcomes is automatically "small". Hence the family of data amenable to a
    physical interpretation is also automatically "small" compare to the
    information produced by Nature.


    To follow up on Finkelstein's claim, would require to check that the axioms of
    Quantum Mechanics are logically complete. This is what happened in the
    beginning of the 20th century: someone proved that the axioms of Arithmetic
    are logically complete. Then G\"odel used Arithmetic, the set of natural
    integers to reduce logic to a coding by integers and to prove the
    incompleteness of a set of axioms. If we could do something similar with
    either Quantum Mechanics, or with a simplified version sufficient to built
    the theory of quantum computation, then we could build a new logic with
    different rules, which could be implemented on a machine, that would allows
    to reach a much bigger set of data provided by Nature.


This is not the route followed by the physicists working on Quantum
Gravity.
\HREF{https://scholar.google.com/citations?hl=en&user=xmOSptwAAAAJ&view_op=list_works&sortby=pubdate}
{Preskill} is writing a book on Quantum Computing including topological
quantum computation and also a very new chapter on quantum information,
following the path similar to the one proposed by Shannon in 1948.

The axioms of Arithmetic, namely the axioms describing the natural
integers where proposed by Peano in 1889, after several previous
proposals. Russell accepted them as suitable indeed to describe the set
of natural integers. Poincar\'e argued that before accepting them they need
to be proved to be consistent. By consistent, he meant that there are no
theorem based on this family of axioms leading to prove that 0=1. The
first proof of G\"odel incompleteness theorem in 1931, is based on Peano's
axioms. But the first proof of their consistency was provided by Gerhard
Gentzen in 1936, using a method called "transfinite induction" which is
supposed to be encodable by a Turing Machine. G\"odel himself gave a
different proof in 1958.

If you think just a little, we just cannot do anything without the set of
natural integers. So, at least for them, a proof of the consistence of
the axiom describing the integers is required.

It is unclear whether such a work could be done for the part of Quantum
Mechanics that is relevant for quantum computing. We do not need the
whole formalism for quantum computing, since quantum information is
reduced to qubits. It is sufficient to describe axioms using the qubits
only and the corresponding quantum gates. Since I haven't followed the
literature in this direction, it may happen that such axioms are already
available. I have no idea whether such axioms have been proved to be
consistent. But if they are, they might be a way to describe quantum
computing in terms of a quantum analog of Turing Machines (many proposals
can be found on the web, but I have no idea whether they are as radical
as David Finkelstein had in mind).

The length of this note: This is what happens when you learn a topic and
you haven't yet mastered it. The longer the explanation the more likely
the author is to be confused with the topic.

\section{Finkelstein life}
\label{sect:FinkelsteinLife}

\begin{description}
\item[1964-08-28 John B. Garner to DF] UWiscArchive/640828Garner2DF.pdf

``
Last night The Northside Reporter, one of Hazel Brannon Smith's newspapers
was bombed. Damage is estimated at \$1,000 to \$1,500. I wonder how long she
will have to stop publishing . This completely cancels the \$1000 Pulitzer
Prize she received earlier this year. The paper is a small weekly and has a
liberal editorial policy and we desperately need it. This is the third
bombing in Jackson recently--the past six months I think. The first was a
barber shop that was charging below union rates and the second was a
partially finished Negro motel.
''

\item[1964-09-15  DF to John B. Garner] UWiscArchive/640915DF2Garner.pdf

``
Morton Schiff and I are flying down the 19th to start the term off
''

\item[1964-10-01  DF to Robert W. Morse] UWiscArchive/641001DF2Morse.pdf

``
Tougaloo College has a special significance for Mississippi and the entire
south. One recent example of this, among many, is the role of its faculty in
the formation of the Freedom Democratic Party.
''


\item[1964-10-29  DF to AIP] UWiscArchive/641029DFreport.pdf

seems to be a fun read

\item[1965-12-22]
In (1966) {\em Kinks} paper\rf{Finkel66} David writes: ``I am also grateful
to Tougaloo College for the hospitality afforded me during part of this
work.''

\item[2017-01-04]
This is a total riot: poor David has been laboring on his Space Time Code
since something like 1947 and Googlette don't care:)


\HREF{https://www.linkedin.com/in/amrouhi} {Maureen Rouhi},
GaTech CoS Director of Communications (University of London
Ph.D., Chemistry, University of the Philippines
B.S and M.S., Agricultural chemistry; speaks Tagalog and Farsi)
wrote:
``FYI, I purposely omitted the terminology Space Time Code, because a
Google search brings this top response:

\begin{quote}
{\em A space–time code (STC) is a method employed to improve the
reliability of data transmission in wireless communication systems using
multiple transmit antennas.}
\end{quote}


\end{description}

%\newpage %%%%%%%%%%%%%%%%%%%%%%%%%%%%%%%%%%%%%%%%%%%%%%%%
\printbibliography[heading=subbibintoc,title={References}]
