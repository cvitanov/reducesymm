% reducesymm/QFT/Finkelstein.tex

\chapter{David Finkelstein}
\label{c-Finkelstein}


\section{Finkelstein notes}
\label{sect:Finkelstein}

\begin{description}
\item[1958-05-15 David Finkelstein]

David Ritz Finkelstein (July 19, 1929 -- January 24, 2016)

Finkelstein\rf{Finkel55}
{\em Internal structure of spinning particles}

MAY 15, 1958:
Finkelstein\rf{Finkel58}
{\em Past-future asymmetry of the gravitational field of a point particle}

\item[1990-05-24 David Finkelstein] \texttt{900524PhysToday.pdf}, a
letter to Physics Today (unpublished) in response to Khurana\rf{Khurana89a}.

Wilczek\rf{Wilczek82} {\em Quantum mechanics of fractional-spin
particles} names `anyons' (since interchange of two of these particles
can give any phase), credits  Leinaas and Myrheim (1977) etc., but not
Finkelstein.

Wilczek and Zee\rf{WilZee83}
{\em Linking numbers, spin, and statistics of solitons}:
``Some aspects of this subject appear to have been anticipated in the
remarkable paper of D. Finkelstein and J. Rubinstein\rf{FinRub68}."

Misner, Thorne and Zurek\rf{MiThZu09}
{\em {John Wheeler}, relativity, and quantum information}:
``The physics becomes lucid in 1959-62, through insights by David
Finkelstein"
`` which dealt with only half of the Schwartzfield geometry and revealed
its horizon and thence its role as a black hole"

Finkelstein\rf{Finkel87} {\em The Quantum Paradox},
in {Encyclopaedia Britannica} {Yearbook of Science and the Future}
is a very helpful popularization of his view of quantum mechanics.

Finkelstein \etal\rf{FJSS62}
{\em Foundations of quaternion quantum mechanics}

Adler\rf{Adler95} seems to credit them at length. In his 1996 review of
Adler's book, Finkelstein writes diplomatically\rf{Finkel96}: ``Adler
does cite a gauge-invariant quaternionic theory.Its imaginary operator
I varies in spacetime and is then a natural Higgs field. But this
theory does not account for color SU(3), the one-handed neutrino and
several other peculiarities of nature that seem to belong together.
Adler quite reasonably shelves this theory. Indeed, the gaps in this
theory led me to shelve the whole quaternionic project in the 1960s.
Adler suggests a quaternionic route that might close these gaps.''

\item[2016-02-21 Predrag]
A man wrote a whole PhD thesis\rf{Familton15}
    on {\em Quaternions: A History of Complex Noncommutative Rotation
    Groups in Theoretical Physics}, without mentioning  Finkelstein
    \etal\rf{FJSS62,FJS63,FJSS63} once. How is that possible? But the
    thesis looks iffy to me (PhD in Mathematics Education), so we'll
    let that one pass.

Hestenes\rf{Hestenes66} {\em Space-time Algebra}, not Jauch or
Finkelstein\rf{FJSS62,FJS63,FJSS63}, is apparently credited with the
invention of quaternionic quantum mechanics. They published first, but
there are no mutual citations - it's only much later Finkelstein cites
Hestenes, but no citations in the reverse direction.

Weird thing is that I have never heard of Professor Hestenes until
today, when I checked him out to find why he does not credit
Finkelstein \etal\rf{FJSS62,FJS63,FJSS63} for their quaternion quantum
mechanics. Neither does De Leo\rf{DeLeo96}-- but Finkelstein had cited
him in more recent papers.

Now, when I think of it, I think we do have a grad student who insists
computing everything in terms of bi-vectors, driving Zangwill to
distraction. Hestenes, of course, has yet another interpretation of
Quantum Mechanics\rf{Hestenes86}.

The idea of Euclidean / Minkowski spaces as limits of compact groups is
- actually - something Finkelstein strongly felt was the right thing to
do.





\end{description}
