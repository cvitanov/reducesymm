% reducesymm/QFT/dailyBlog.tex
% $Author$ $Date$
% Predrag  switched to github.com               jul  8 2013
% former siminos/blog/dailyBlog.tex

\chapter{Daily QFT blog}
\label{c-DailyBlog}


\section{Is QED finite?}
\label{sect:finiteQED}

\begin{description}

\item[2012-07-13 Aoyama] \etal\rf{AoHaKiNi12}
summarize the results of numerical work on the complete determination of
the 10th-order contribution to the anomalous magnetic moment.

\item[2013-11-23  Predrag] The surprise about Aoyama \etal\rf{AoHaKiNi12}
is that they have completely ignored my paper\rf{Cvit77b}, and for
them the set of all diagrams without a fermion loop (their `q-type',
presumably for `quenched-type') is one `gauge-invariant set'. So their sets
are the crudest possible, counting only different lepton loop insertions.
Their `q-type' set is a sum of my gauge-invariant sets. For example, for 4-loop
contributions I have 8 gauge-invariant sets (where time-reversed pairs count as
one set).

Their $A^{(6)}_1 = 1.18\dots $ includes the one electron
loop correction (so it is not `quenched'), while I wrote down in Fig.~3
that the sum is $A^{(6)}_1 = 0.93\dots $, not sure where the difference
comes from. It does not matter as far as my finiteness conjecture is
concerned.

Their quenched set $V$ gives $A^{(8)}_{1,V}= -2.17\dots $, while I
predict $A^{(8)}_1 \approx 0 $, which is not too far off for a sum of 518
diagrams.

Their quenched set $V$ gives $A^{(10)}_{1,V} = 10\dots $, while I
predict $A^{(10)}_1 \approx 3/2 $, which is not too far off for a sum of 6354
diagrams.

\item[2013-10-22 Nio Makiko to Warren]  nio@riken.jp  wrote:\\
  Thank you for useful information on knots theory and muon $(g-2)$.
 We will carefully examine them.

 As for numerical values of Set V, we are reluctant to open these values
 before publication.  Once we finish writing up the detailed paper and
 posting it on arXiv,  I will let you know.

\item[2013-10-23  Warren D Smith] warren.wds@gmail.com\\
well, I don't have any choice :).  However I point out perhaps you could send
me the numbers for $\alpha^4$ diagrams even if you want to wait on the
$\alpha^5$  diagrams?

Meanwhile I found out more interesting info which the textbooks do not know!

Cvitanovi\'c 1977 conjectured that gauge-invariant quenched diagram sets
always have small sum of diagram values, indeed small enough that he
thought quenched QED series
would always converge\rf{Cvit77b}
    \PC{`quenched' = no internal electron loops, \ie, $m_e \to \infty$
    approximation}.
He apparently originally thought this even
for unquenched but Lautrup\rf{Lautrup77} disproved it and Dyson had long had a
highly convincing argument\rf{Dyson52} (Reprinted p.255-6 in Selected papers
of Freeman Dyson with
commentary, AMS 1996)
-- and I now have even more convincing
arguments -- that generic QED series diverge for any nonzero $\alpha$.)
This was due to Cvitanovi\'c's empirical observation of amazingly huge
cancellations within gauge-invariant diagram sets, especially in
quenched QED.
    \PC{Suslov\rf{Suslov99} argues that t'Hooft and Lautrup
     renormalons are not significant? I have not studied it.}

However, what Cvitanovi\'c did not know, was that
\HREF{http://www.itp.ac.ru/en/persons/bogomolny-evgeny-borisovich/}
{Bogomolny} and Kubyshin 1981-1982
found estimates of the growth rate of generic QED series, and also for the
quenched QED subseries,
and indeed for QED for diagrams with k electron loops only (k fixed)\rf{BogKub81}.
I only found this out the other day, but I had long known about
estimates due to Dyson and others predicting divergence for QED
series.  It is just that this B+K work by permitting arbitrary fixed
k, directly addresses Cvitanovi\'c's quenched-convergence conjecture and
massively conflicts with it -- for quenched QED the prediction is that
at Nth order we get a quenched diagram sum growing factorially with N:

Evgeny B Bogomolny and Yu A Kubyshin:
Asymptotic estimates for graphs with a fixed number of fermion loops
in quantum electrodynamics.
\\
1. The choice of the form of the
steepest-descent solutions,
Soviet J. Nucl. Phys. 34,6 (1981) 853-858.
Asymptotic estimates for diagrams with a fixed number of fermion loops
in quantum electrodynamics.
\\
2. The extremal configurations with the
symmetry group $O(2)\times O(3)$,
Soviet J. Nuclear Phys. 35 ,1 (1982) 114-119.
    \PC{there is a hard copy at GT library, 4th Floor East
Call Number: 	QC173.I252X (or microfilm?)}

I believe in the sort of arguments Bo+Ku are making, albeit the
details are questionable.
(E.g. saddlepoint asymptotic estimates are not rigorous unless you prove stuff
about the saddlepoints and about tail estimates, which they never proved, and
probably nobody can prove.)

Note that this believed N! growth for fixed-k (including quenched) QED diagrams
(Lautrup's diagrams also feature N! growth) is far faster than the
believed growth -- more like (N/2)! -- for the \emph{full} QED series!!
This fact that subseries diverge far more rapidly than full series
can only be explained by presuming that
the values at different k cancel each other amazingly well when we sum
over all k.
This indicates Cvitanovi\'c was extremely wrong asymptotically, and that
cancellations
quite different than his observations ultimately occur which seem even
more dramatic.

However, Cvitanovi\'c was correct that amazingly large cancellations
(also) occur, empirically, within the quenched diagrams alone, at
least for the small N that have been reached by computer.   And I
presume that Kreiman's knot
ideas\rf{Kreimer00} and my ``fractal distribution'' empirical observations are a
partial explanation of why.

So a consistent picture is now developing about how QED perturbative series (and
various interesting sub-series, such as quenched) allegedly behave
asymptotically
(Although I never saw anything saying all that I just said in one place...)

\item[2013-10-23  Warren to Predrag]  do you have a graph-theoretic
characterization of `gauge invariant diagram set' or know how many such
(minimal) sets there are at order N?   E.g. does their count grow
exponentially, super-exponentially, polynomially or what?  You gave a
formula for the count of quenched GI-sets which grows only polynomially
but I suspect exponential or faster growth for unquenched QED.  Even an
incomplete graph-theory understanding might be adequate to get good growth
bounds.

\item[2013-10-23 Warren]
Although I do not fully understand ``gauge invariant classes of Feynman diagram''
I have figured out enough now to prove that their count grows
ultimately superexponentially in unquenched QED.  Specifically I now
claim to have a proof that

  Number of GI classes of Feynman diagrams in QED at $\alpha^N$ order
\[
  \to 0.01 *  96^{-N/4} * N! / [ (N/2)! * (N/4)! ]
\]
for an infinite set of integers $N>0$.
And this clearly grows superexponentially.
(My bound is very unlikely to be optimal.)

\item[2013-10-23 Predrag to Warren]
of course, I'm very interested in this discussion, but can you do me a
favor and actually read my paper, and edit your initial email
accordingly, before we wrangle with further details?

Many of your statements are addressed in my paper, and I can answer thyem
more efficiently if you go through them first. I love Dyson dearly, but
his statement is an elementary statement about asymptotic series for
factorials. Mine is about mass-shell gauge invariant quantities (perhaps
planar?), a topic that is gaining some traction now, unfortunately not
applicable to QED. Ain't they weird? They are perfectly happy developing
methods for gauge theories that do not work on the 'trivial' U(1) case?
Go figure...

If you can get gauge sets added up, that would be useful, but they
probably do not have them: I assume they  use the second method of
computing magnetic moment,
\HREF{http://www.cns.gatech.edu/\%7Epredrag/papers/preprints.html\#g-2}
{eq (6.22)}  [that I believe I post-invented
after Schwinger? but probably I'm deluding myself..] I think that mixes up
gauge sets.

\item[2013-10-23 Warren]
Actually, I already had read it
\HREF{http://www.cns.gatech.edu/~predrag/papers/NPB77.pdf}
{here}.
The problem is not me not reading it; it is me being an idiot.  And/or, you.
Actually, it is quite likely that I am more of an idiot now than you
were in 1977, but I suspect we both are capable of considerable
idiocy...

[there is much more still on 2013-10-23, but I better go to bed...]

\item[2013-11-23  Predrag]
\HREF{WarrenSmithQED131123.html} {Warren Smith} draft.

\item[2013-12-08  Predrag] to Piotr, Wanda and Andrea
(Piotr Czerski <piotr.czerski@ifj.edu.pl>,
 wanda.alberico@to.infn.it,
 andrea.prunotto@gmail.com):

I'm no fan of Feynman diagrams, and I'm always looking
for other ways to look at perturbative expansions. So just a little email
- if you have a new angle\rf{PrAlCz13} on subsets of diagrams which are gauge
invariant sets, I would be curious to learn how you look at that.

Just something to keep in mind :)

PS to Andrea: I realize you might rather forget this stuff (takes you a
decade to write a paper?) but at least I got a ringtone out of you. The
only problem is, I do not have a cell phone, so I do not know how to make
it ring. At least I'm more technologically savvy than
\HREF{http://www.theguardian.com/science/2013/dec/06/peter-higgs-boson-academic-system?CMP=twt_gu}
{Peter Higgs}.

\item[2013-12-10  \HREF{https://sites.google.com/site/andreaprunotto/} {Andrea}]
Sorry for late reply (well, we're used to longer gaps). Yes! I actually
took 10 years to write this paper out of my master thesis, but I have
some excuses: I did my PhD in Biochemistry (Z\''urich) and now I work on
Genetics (Lausanne). This summer my ``old'' professor Wanda
 found my work in a drawer
and then contacted me, telling me that it would be a good idea to publish
it.

About your request: I'm really interested in seeing if the rooted-map
approach to Feynman diagrams can address the problem you've risen. But I
have no idea what the ''subsets of diagrams which are gauge invariant
sets'' are. I've checked a bit on the web but I'm sure you can give me
better indications (the works I found were too technical: I need to know
the basis of the problem). Can you send me some specific link at freshman
level, in particular where I can see the geometry of these subclasses of
diagrams?

\item[2013-12-11  Predrag]
Why Google when you can click on the link in my email?

    ... I'm no fan of Feynman diagrams (my rant is
\HREF{http://www.cns.gatech.edu/~predrag/papers/preprints.html\#FiniteFieldTheo}
{here}) ... The article defines the gauge invariant sets.



\end{description}


\section{Quantum Field Theory}
\label{sect:QFT}

\begin{description}
\item[2013-11-23  Predrag] Moved all matters QFT from
\texttt{siminos/blog/Lie.txt} to here.

\item[2010-03-04 Predrag]
Kevin Mitchell is here, says we should study Littlejohn and
M. Reinsch\rf{LiRe97}: ``Gauge fields in the separation of
rotations and internal motions in the $n$-body problem.'' I
will put it into \wwwcb{/library}. Read 3-body problem
sections. See p. 14 for a discussion of the three-body
coordinates and p. 25 for a discussion of the three-body
section (gauge).



\item[2011-07-27 PC]
What follows is casting eye far ahead - to the role of gauge invariance
in Quantum Field Theories.
Following articles seem of interest as follow-ups on
Cvitanovi\'c\rf{PCar}, {\em Group theory for {Feynman} diagrams in
non-{Abelian} gauge theories}:

Should add this article to Birdtracks.eu/refs: Astorino\rf{Astor10}
writes ``Jones polynomial arises as special cases: Sp(2), SO(-2), and
SL(2,R). These results are confirmed and extended up to the second order,
by means of perturbation theory, which moreover let us establish a
duality relation between $SO(\pm N)$ and $Sp(\mp N)$ invariants. A
correspondence between the first orders in perturbation theory of SO(-2),
Sp(2) or SU(2) Chern-Simons quantum holonomy's traces and the partition
function of the Q=4 Potts model is built.''

Khellat\rf{Khel10} strikes me as dubious...

Martens\rf{Mart11} writes: ``We calculate the two-loop matching corrections for the
   gauge couplings at the Grand Unification scale in a general framework
   that aims at making as few assumptions on the underlying Grand Unified
   Theory (GUT) as possible. In this paper we present an intermediate
   result that is general enough to be applied to the Georgi-Glashow
   SU(5) as a ``toy model''. The numerical effects in this theory are
   found to be larger than the current experimental uncertainty on $\alpha$s .
   Furthermore, we give many technical details regarding renormalization
   procedure, tadpole terms, gauge fixing and the treatment of group
   theory factors, which is useful preparative work for the extension of
   the calculation to supersymmetric GUTs.
   ''

Tye and Zhang\rf{TyZh10} write: ``
Bern, Carrasco and Johansson have conjectured dual
   identities inside the gluon tree scattering amplitudes.
   We use the properties of the heterotic string and open string tree
   scattering amplitudes to refine and derive these dual identities.
   These identities can be carried over to loop amplitudes using the
   unitarity method. Furthermore, given the $M$-gluon (as well as
   gluon-gluino) tree amplitudes, $M$-graviton (as well as
   graviton-gravitino) tree scattering amplitudes can be written down
   immediately, avoiding the derivation of Feynman rules and the
   evaluation of Feynman diagrams for graviton scattering amplitudes

Eto \etal~rf~{EFGKNOV08} write:
   ``We construct the general vortex solution in the color-flavor-locked
   vacuum of a non-Abelian gauge theory, where the gauge group is taken
   to be the product of an arbitrary simple group and U(1). Use of the
   holomorphic invariants allows us to extend the moduli-matrix method
   and to determine the vortex moduli space in all cases. Our approach
   provides a new framework for studying solitons of non-Abelian
   varieties with various possible applications in physics.''

and there is much much more...; will continue some other time.

\item[2011-11-03 PC] Today is that time. I'm sitting in
\HREF{http://intractability.princeton.edu/blog/2011/05/workshop-counting-inference-and-optimization-on-graphs/}
     {Intractability Workshop:}
     \emph{Counting, Inference and Optimization on Graphs}
with a bunch of high-level computer nerds, and I almost afraid to say
what I'll say next (plumbers avoid physicists that say such things): In
constructing our atlas of inertial manifold of turbulent pipe flow, we
fix the $SO(2) \times O(2)$ phase separately on each local chart. The
freedom of doing that is called ``local gauge invariance'' (blame Hermann
Weyl for the ugly word) and in the limit of $\infty$ period cycles, cycle
points are dense and their local charts are infinitesimal, so this is
really local gauge invariance. In the world of computer science they use
this freedom profitably, to reduce the number of terms they use in their
computations. That suggests that there might be a (variational?)
principle that selects an optimal choice of (relative) template phases
(\ie, gauge transformations that connect a chart to the next chart).

Nerds call this 'reparametrization' - it supposedly speeds up calculations.
Have not really seen that in quantum field theory, with exception of light
cone gauges and their relatively recent applications by the Witten cult.

Literature: \refref{CheChe08,YeChe11} and stuff on
\HREF{http://www.hpl.hp.com/personal/Pascal_Vontobel/ciog2011/reading_list_web.html}{this
site} (if you can understand any of it).

Feel free to ignore this remark. It's future research.

\item[2013-02-22 PC] Fomin and Pylyavskyy\rf{FomPyl12}
{\em Tensor diagrams and cluster algebras}, {\arXiv{1210.1888}},
is a major orgy in birdtracking. Should study it some day.

\item[2014-07-20 PC] More birdtracking - they construct orthogonal
projection operators: Keppeler and
    Sj{\"o}dahl\rf{KeppSjo14}, {\em Hermitian {Young} operators},
    Sj{\"o}dahl\rf{Sjodahl13,Sjodahl13a} {\em Tools for calculations in
    color space}, and his student Thor{\'e}n\rf{Thoren14}.

Gu, J. and Jockers\rf{GuJock14}.

More birdtracking still:
Kol and Shir\rf{KolShir14} {\em Color structures and permutations}.

\item[2014-12-02 PC] More birdtracking:
Costa and Hansen\rf{CosHan14},
{\em Conformal correlators of mixed-symmetry tensors}

\item[2012-05-16  Parameswaran Nair]  vpnair@optonline.net writes
on saddle solutions of Yang-Mills:

I attributed the conjecture to Hitchin; it was actually due to Atiyah and
Jones. ''The only finite action solutions of the YM equations are
instantons, either self-dual or antiself-dual.'' This was the conjecture
for which the refs provide counter examples.

\HREF{http://ChaosBook.org/library/Schiff91.pdf}{Here is} the paper by my
student Schiff\rf{Schiff91}, who writes:
``
Following a proposal of Burzlaff (Phys.Rev.D 24 (1981) 546), we find
solutions of the classical equations of motion of an abelian Higgs model
on hyperbolic space, and thereby obtain a series of non-self-dual
classical solutions of four-dimensional SU(3) gauge theory. The lowest
value of the action for these solutions is roughly 3.3 times the standard
instanton action.
''

``
In physics, despite the fact that the non-self-dual solutions correspond
to saddle points, and not minima, of the Yang-Mills functional, to do a
correct semiclassical approximation by a saddle-point evaluation of the
path integral, it is certainly necessary to include a contribution due to
nonself-dual solutions, and if it should be the case that there is a
non-self-dual solution with action lower than the instanton action (this
question is currently open, and of substantial importance), then such a
contribution would even dominate. Unfortunately, it is questionable
whether the semiclassical approximation can give a reliable picture of
quantized gauge theories; it has been argued that in four-dimensional
gauge theory small quantum fluctuations
around classical solutions cannot be responsible for
confinement, unlike in certain lower-dimensional
theories. But it may still be possible to extract some
physics from the semiclassical approach. A first step in
such a direction would be to obtain a good understanding
of the full set of non-self-dual solutions and their properties.
[...] We pursue an old idea, due to Burzlaft
[10], for obtaining a non-self-dual, ''cylindrically symmetric''
solution for gauge group SU(3). If we write
$\reals^4 = \reals \times \reals^3$, and identify some SU(2) [or SO(3)] subgroup
of SU(3), with generators that we will denote T', then we
can look at the set of SU(3) gauge potentials which are invariant
under the action of the group generated by the
sum of the T''s and the generators of rotations on the IR
factor of E (we choose the T''s and the IR rotation generators
to satisfy the same commutation relations). We
call such potentials ''cylindrically symmetric'' (in analogy
to the standard notion of cylindrical symmetry in IR,
which involves writing R =RXIR and requiring rotational
symmetry on the E factor). Such potentials will
be specified by a number of functions of two variables:
the coordinate on the IR factor of E (which we will
denote x), and the radial coordinate of the E factor
(which we will denote y). Clearly the equations of motion
for such cylindrically symmetric potentials (if they are
consistent) will reduce to equations on the space
I (x,y ):y ~ 0].
,,

The earlier work is by Sibner, Sibner, Uhlenbeck\rf{SiSiUhl89}. They
write ``
The Yang-Mills functional for connections on principle SU(2) bundles over
S 4 is studied. Critical points of the functional satisfy a system of
second-order partial differential equations, the Yang-Mills equations. If,
in particular, the critical point is a minimum, it satisfies a
first-order system, the self-dual or anti-self-dual equations. Here, we
exhibit an infinite number of finite-action non-minimal unstable critical
points. They are obtained by constructing a topologically nontrivial loop
of connections to which min-max theory is applied. The construction
exploits the fundamental relationship between certain invariant
instantons on S 4 and magnetic monopoles on H 3. This result settles a
question in gauge field theory that has been open for many years.
''

% 2012-05-16 fund this:
%@article{
%author = {Gil Bor and Richard Montgomery},
%title = {{SO(3)} invariant {Yang-Mills} fields which are not self-dual},
%}

Bor\rf{Bor92} writes
``
We prove the existence of a new family of non-self-dual finite-energy
solutions to the Yang-Mills equations on Euclidean four-space, with SU(2)
as a gauge group. The approach is that of ``equivariant geometry:''
attention is restricted to a special class of fields, those that satisfy
a certain kind of rotational symmetry, for which it is proved that (1) a
solution to the Yang-Mills equations exists among them; and (2) no
solution to the self-duality equations exists among them. The first
assertion is proved by an application of the direct method of the
calculus of variations (existence and regularity of minimizers), and the
second assertion by studying the symmetry properties of the linearized
self-duality equations. The same technique yields a new family of
non-self-dual solutions on the complex projective plane.
''

\item[2012-04-19 Daniel] ``graveyard of obvious ideas'' rings a little
    aggressive, no? ``if you are a master of quantum-mechanics or QFT symmetries
and their linear irreducible representations,\rf{PCgr} you may leave your
baggage at the door'' rings a little aggressive, too.
\\{\bf Predrag:} Get's edgy. In ``master of their linear irreducible
    representations'' I make fun of myself. Let the referee object to
    that?


{\bf [2012-06-14 Predrag]} Grin and bear it. Pulling my Senior discount
card here.

\item[2013-07-15 Predrag] I've collected a bunch of QFT e-books, saved
them in \wwwcb{/library}:

5449Grigorenko06.pdf

Abarbanel 2013.pdf

CoKaWa04.pdf

DasFerbel03.pdf

Zee03.pdf

hao89.pdf

Nichkawde13.pdf

Das06.pdf

Milton01.pdf

SeoSan12.pdf

NagashimaI10.pdf


\item[2013-01-20  Predrag]
This really belongs to planar field theory, but for time being I note
it here: Lucini and Panero\rf{LucPan13} (in Chaosbook.org/library)
might be of interest. All I get is one sentence and a reference only
to \refref{PlanFieldThe}.

I should also read Kang and Loebl\rf{KanLoe09}
{\em The enumeration of planar graphs via {Wick}'s theorem}.

\item[2013-03-27  Predrag] Do not understand this article:
Jim\'enez-Lara and J. Llibre\rf{JimLli11},
{\em Periodic orbits and nonintegrability of generalized
classical {Yang--Mills Hamiltonian} systems}.

                                                \toCB
D. Biswas \etal\rf{BALL92} {\bf Existence of stable periodic orbits
in the $x^2y^2$ potential: a semiclassical approach}, re-derives the
Dahlqvist and G. Russberg\rf{DR_prl} result. Also read Nip
\etal\rf{NTOD92} {\em Search for regular orbits in the $x^2y^2$
potential problem}

Hu\rf{HJLW01}
{\em General initial value form of the semiclassical propagator},
write: ``
We show a general initial value form of the semiclassical propagator.
Similar to cellular dynamics, this formulation involves only the
nearby orbits approximation: the evolution of nearby orbits is
approximated by linearized dynamics. This phase space smearing
formulation keeps the accuracy of the original Van Vleck-Gutzwiller
propagator. As an illustration, we present a simple initial value
form of the semiclassical propagator. It is nonsingular everywhere
and is efficient for numeric implementation.
''

\item[2013-04-16  Predrag] There seems to be whole literature on
classical Yang-Mills (CYM). In
{\em Entropy production in classical Yang-Mills theory from Glasma
initial conditions} Hideaki Iida,  Teiji Kunihiro,  Berndt M\''uller,
Akira Ohnishi,  Andreas Sch\''afer,  and Toru T. Takahashi,
\arXiv{1304.1807}, % \rf{IKMOST13}
write:

Pure Yang-Mills theory in temporal gauge with the Hamiltonian in the
noncompact (A, E) scheme on a cubic spatial lattice. The initial
condition satisfies Gauss' law; check its validity as well as Energy
conservation carefully at every time step. Define distance (6), (7) that
is gauge invariant under residual (time independent) gauge transformations.

                                                    \inCB
They call the stability matrix `Hessian', and its eigenvalues at time
slice the `local Lyapunov exponents (LLEs)'\rf{KMOSTY10}: LLE plays the
role of a ``temporally local'' Lyapunov exponent, which specifies the
departure rate of two trajectories in a short time period. Then they say
this (?): ``For a system where stable and unstable modes couple with each
other as in the present case, an LLE does not generally agree with the
Lyapunov exponent in a long time period.'' ``\refRef{KMOSTY10} introduced
another kind of Lyapunov exponent called the intermediate Lyapunov
exponent (ILE), which is an ``averaged Lyapunov exponent'' for an
intermediate time period; i.e., a time period which is sufficiently small
compared to the thermalization time but large enough to sample a
significant fraction of phase space. By its definition (13) it
is the set of stability exponents for a finite time \jacobianM.

``Two comments are in order, here: A Lyapunov exponent [PC: not the
Lyapunov exponent, they mean the stability exponent] can be (real)
positive, negative, zero or purely imaginary. Liouville's theorem tells
us that the determinant of the time evolution matrix U is unity, implying
that the sum of all positive and negative ILEs is zero. The KS entropy is
given as a sum of positive Lyapunov exponents. The second comment
concerns gauge invariance of the Lyapunov exponents. In the Appendix we
show that LLE and ILE are indeed gauge invariant under time-independent
gauge transformations in the temporal gauge.''

\item[2013-11-27  Predrag] One way to reduce symmetry seems obvious;
\textbf{average over the group orbit}. That reduces the dimension of the \statesp\
by the dimension of the group orbit; in the reduced \statesp\ each group
orbit is replaced by a point, it's average value. It is a natural construct
in the theory of linear
representations of groups, very important, where it is called a
`character' of the representation; I use it in my
\HREF{http://chaosbook.org/~predrag/papers/Cvi07.pdf} {trace formula} for
systems with continuous symmetries. But it is a trace of a linear evolution
operator.

Still, I do not seem to know how to do this for (1) nonlinear systems,
(2) QFT gauge fixing. Barth and Christensen\rf{BarChr83}  is an example
where this is done - perhaps a way too complicated example... (here are
my notes \HREF{http://chaosbook.org/FieldTheory/extras/BarChr83note.pdf}
{on it}, which I no longer understand myself :)

\item[2014-01-11 Predrag]
I always wonder whether we should be reducing symmetries
by averaging over group orbits (method of characters, used in
my derivation of the {\Fd} in presence of continuous symmetries).
Churchill, Kummer and Rod\rf{ChKuRo83} write in
{\em On averaging, reduction, and symmetry in {Hamiltonian} systems}:
The existence of periodic orbits for Hamiltonian systems at
low positive energies can be deduced from the existence of nondegenerate
critical points of an averaged Hamiltonian on an associated ``reduced
space.'' The paper exploits discrete symmetries, including reversing
diffeomorphisms, that occur in a given system. The symmetries are used to
locate the periodic orbits in the averaged Hamiltonian, and thence in the
original Hamiltonian when the periodic orbits are continued under
perturbations admitting the same symmetries.''

\item[2014-01-11 Predrag]
Kummer\rf{Kummer81}, {\em On the construction of the reduced phase space
of a {Hamiltonian} system with symmetry} writes:
``
Weinstein [...] uses this correspondence between connections
and lifts in his construction of Sternberg's phase space for a
particle in a Yang-Mills field.
''

l. A. Weinstein, A universal phase space for particles in Yang-Mills fields, Lett. Math. Phys.
2 (1978), 417-420.
2. S. STERNBERG, Minimal coupling and the symplectic mechanics of a classical particle in
the presence of a Yang-Mills field, Proc. Nat. Acad. Sci. 74 (1977),5253-5254.
3. S. STERNBERG, On the role offield theories in our physical conception of geometry in Differential
geometric methods in mathematical physics II, Springer Lecture Notes in Mathematics,
676 (1977), 1-80.

\item[2014-07-18 Predrag]
Mari\~no\rf{Marino14} lectures on non-perturbative effects are perhaps of
interest:
 ``a review of non-perturbative
instanton effects in quantum theories, with a focus on large N gauge
theories and matrix models. I first consider the structure of these
effects in the case of ordinary differential equations, which provide a
model for more complicated theories, and I introduce in a pedagogical way
some technology from resurgent analysis, like trans-series and the
resurgent version of the Stokes phenomenon. After reviewing instanton
effects in quantum mechanics and quantum field theory, I address general
aspects of large N instantons, and then present a detailed review of
non-perturbative effects in matrix models.''

\item[2014-07-30 Predrag]
Paul Hoyer recent talk at
\HREF{http://www.physics.ncsu.edu/LC2014/LC2014-Friday/PaulHoyer.pdf}
{Light Cone 2014} is too packed to be useful, but maybe has some pointers
to recent interesting non-perturbative QCD results.

\item[2014-08-01 Burak]
Sydney Coleman\rf{Col77} gave a lecture on semi-classical formulations of QFT in 1975.
He has several semi-classical treatment methods with names
``Zeroth-Order Weak-Coupling Expansion",
``Coherent-State Variation",
``First-Order Weak-Coupling Expansion",
``Bohr-Sommerfeld Quantization", and
``DHN Formula". I didn't yet understand what any of these means, I'm just starting
to read it, but if we can find a way of adapting these methods to the numerical
calculations that might be a good starting point.

\item[2014-08-01 Predrag]
Coleman was a great expositor - on a level of a Nobelist. Would be cute
if we found something in his lectures we can use today. ``DHN Formula" (I
knew all three - the first two authors are currently dead) should be the
Gutzwiller formula for QFT. Paolo Muratore-Ginanneschi, a former student
in our Copenhagen group, wrote something on that\rf{Murat03} that I still
have not studied in depth. But we maybe should.

\item[2014-08-01 Burak]
Coleman refers to the pedagogical overview by Rajaraman \rf{Raj75} for the derivation
of ``DHN Formula'' and I think this will be a good starting point for me. So far,
keywords are familiar, in the introduction, he talks about periodic orbits and
their stability and how hard it is to find them for realistic cases. A lot of work
apparently in this period has been done using Sine-Gordon equation because they
had analytical solutions. I looked for articles which cite \rf{Raj75} but didn't
see anything that relies on numerical solutions of field equations.

\item[2014-08-13 Predrag]
In my notes it says that Dyson himself told me to read {\em Drawing
theories apart: the dispersion of {Feynman} diagrams in postwar physics},
by Kaiser\rf{Kaiser09}. So we better read it - I will put a pilfered
eBook copy \HREF{http://ChaosBook.org/library/Kaiser09.pdf}{Kaiser09.pdf}
into ChaosBook.org/library.

\item[2014-11-05 Predrag]
Do not know if this is something for us, but worth having a look at:

Luca Salasnich,
 \emph{Discrete bright solitons in Bose-Einstein condensates and dimensional
  reduction in quantum field theory}, \arXiv{1411.0160}:
`` We  review the derivation of an effective one-dimensional (1D) discrete
nonpolynomial Schr\"odinger equation from the continuous 3D Gross-Pitaevskii
equation with transverse harmonic confinement and axial periodic potential.
Then we study the bright solitons obtained from this discrete nonpolynomial
equation showing that they give rise to the collapse of the condensate above a
critical attractive strength. We also investigate the dimensional reduction of
a bosonic quantum field theory, deriving an effective 1D nonpolynomial
Heisenberg equation from the 3D Heisenberg equation of the continuous bosonic
field operator under the action of transverse harmonic confinement. Moreover,
by taking into account the presence of an axial periodic potential we find a
generalized Bose-Hubbard model which reduces to the familiar 1D Bose-Hubbard
Hamiltonian only if a strong inequality is satisfied. Remarkably, in the
absence of axial periodic potential our 1D nonpolynomial Heisenberg equation
gives the generalized Lieb-Liniger theory we obtained some years ago.''

\item[2014-11-23 Predrag]
Zvonkin\rf{Zvonkin97}
(\HREF{http://ChaosBook.org/library/Zvonkin97.pdf}{click here})
writes in
{\em Matrix integrals and map enumeration: {An} accessible introduction}:
``Physicists working in two-dimensional quantum gravity invented a new
method of map enumeration based on computation of Gaussian integrals over
the space of Hermitian matrices. This paper explains the basic facts of
the method and provides an accessible introduction to the subject.''

\item[2015-01-09 Predrag]

Tudor Dimofte, gave a talk in Skiles on
{\em Geometric representation theory, symplectic duality, and 3d
supersymmetric gauge theory}

Abstract: Recently, a ``symplectic duality" between D-modules on certain
pairs of algebraic symplectic manifolds was discovered, generalizing
classic work of Beilinson-Ginzburg-Soergel in geometric representation
theory. I will discuss how such dual spaces (some known and some new) arise
naturally in supersymmetric gauge theory in three dimensions.

Tudor is a mathematical physicist at the IAS,
School of Physics, Princeton.

I went to the talk, and - wow! You would think I know something about a
gauge theory but is is like it was in Lithuanian: I understood individual
words, and the alphabet seemed to be Latin - there were things that
looked like letter G or letter H and what we call quotient M/G is
apparently called `resolution'. The foundational paper is Braden, Licata,
Proudfoot and Webster\rf{BrLiPrWe12}, and its followups on
``Quantizations of conical symplectic resolutions II: category O and
symplectic duality''. Good luck reading these...

and of course, it was emphatically N=4 and not N=2, so now I'm at peace

:)

\item[2015-02-04 Predrag] Stephan Stetina
    <stetina@hep.itp.tuwien.ac.at> thesis \\
    on \arXiv{1502.00122} uses my
    \HREF{http://chaosbook.org/fieldtheory/} {Field Theory}, and says
    that 2PI graphs are no sweat (for me they were). Wrote to him:

You seem to have
\HREF{http://chaosbook.blogspot.com/1985/05/feynmans-review-of-my-field-theory-book.html}
{proven Feynman wrong} :) That's no mean achievement. Congrats!

Me and my friends have been studying turbulence in fluid dynamics as a
warmup for doing the same in Yang-Mills. If you see some interesting
turbulence in relativistic fluid dynamics, we are always willing to have
a look at things more field-theoretical.



\item[2015-02-09 Stephan Stetina]
If you are referring to Feynman comments on your book, I definitely disagree
with them - I found your book on field theory more than helpful!

It is very difficult to study (quantum) turbulence within our approach -
however it would be very interesting to do so! The original idea was to
derive the two-fluid hydrodynamics of superfluids from an underlying
field theory. To be able to obtain analytical results, we had to apply
some rather drastic simplifications:

We assumed the superfluid condensate to be uniform and homogeneous (which
translate in a homogeneous superflow in the hydro picture). Further more
we used imaginary time formalism which strictly limits us to study
systems in equilibrium. It would most likely be very challenging (in
particular numerically) to introduce a condensate with arbitrary space
and time dependence. In the current calculations, a probable onset of
turbulence manifests itself as "something going wrong" - for example
above certain velocities of the superfluid it is no longer possible to
construct a stable and homogeneous superfluid phase. Another example is
the appearance of the ``two-stream instability" which can also be detected
in our approach (see for instance \arXiv{1312.5993}).
I am not sure yet how much this approach has in common with the one you
have cited.

\item[2015-08-20 Predrag]
Strauss, Horwitz, Levitan, and Yahalom\rf{SHLY15}
{\em Quantum field theory of classically unstable {Hamiltonian} dynamics}
might be a good starting point to learn about dynamical systems for which
the motions can be described in terms of geodesics on a manifold. They
say : ``
... ordinary potential models can be cast into this form by means of a
conformal map. The geodesic deviation equation of Jacobi, constructed
with a second covariant derivative, is unitarily equivalent to that of a
parametric harmonic oscillator, and we study the second quantization of
this oscillator. The excitations of the Fock space modes correspond to
the emission and absorption of quanta into the dynamical medium, thus
associating unstable behavior of the dynamical system with calculable
fluctuations in an ensemble with possible thermodynamic consequences.
''

\item[2015-09-15 Predrag]
Dashen, Hasslacher and Neveu\rf{DaHaNeI74},
\emph{Nonperturbative methods and extended-hadron models in field theory.
{I}. {Semiclassical} functional methods},
are reputed to be the first people to use WKB methods in field theory.


\end{description}







\section{General Relativity}
\label{sect:GR}

\begin{description}
\item[2014-06-06 Deirdre Shoemaker]
A new article\rf{YaZiLe14} out by
\HREF{http://perimeterinstitute.ca/people/luis-lehner}
{Lehner},
\HREF{https://sites.google.com/site/aaronbzimmerman/} {Zimmerman}
and Huan Yang,
\arXiv{1402.4859},
explores turbulence in gravity - hitherto thought to be unlikely or
impossible. They write:

``
We show that rapidly-spinning black holes can display turbulent
gravitational behavior which is mediated by a new type of parametric
instability. This instability transfers energy from higher temporal and
azimuthal spatial frequencies to lower frequencies--- a phenomenon
reminiscent of the inverse energy cascade displayed by 2+1-dimensional
turbulent fluids. Our finding reveals a path towards gravitational
turbulence for perturbations of rapidly-spinning black holes, and
provides the first evidence for gravitational turbulence in an
asymptotically flat spacetime. Interestingly, this finding predicts
observable gravitational wave signatures from such phenomena in black
hole binaries with high spins and gives a gravitational description of
turbulence relevant to the fluid-gravity duality.
''


Lorena <lm\_zertuche@yahoo.com> writes to Deirdre:
Below is an excerpt from
\HREF{http://phys.org/news/2014-06-gravitational-fields-black-holes-eddy.html}
{this article}. Do you find this surprising?

``The team decided to study fast-spinning black holes, because a
fluid-dynamics description of such holes hints that the spacetime around
them is less viscous than the spacetime around other kinds of black
holes. Low viscosity increases the chance of turbulence - think of the
way water is more swirly than molasses.

The team also decided to study non-linear perturbations of the black
holes. Gravitational systems are rarely analyzed at this level of detail,
as the equations are fiendishly complex. But, knowing that turbulence is
fundamentally non-linear, the team decided a non-linear perturbation
analysis was exactly what was called for.

They were stunned when their analysis showed that spacetime did become
turbulent.
``I was quite surprised,'' says Yang, who has been studying general
relativity (GR) - Einstein's theory of gravity - since his PhD. ``I never
believed in turbulent behaviour in GR, and for good reason. No one had
ever seen it in numerical simulations, even of dramatic things like
binary black holes.''

``Over the past few years, we have gone from a serious doubt about
whether gravity can ever go turbulent, to pretty high confidence that it
can,'' says Lehner.
How did this behaviour hide until now? ``It was hidden because the
analysis needed to see it has to go to non-linear orders,'' says Yang.

This
\HREF{http://phys.org/news/2014-04-liquid-spacetime-slippery-superfluid.html}
{blurb} might also be of interest.

\item[2014-06-07 Predrag to]
Huan Yang <huan.yang07@gmail.com>,
Aaron Zimmerman <azimmer@cita.utoronto.ca>,
Luis Lehner <llehner@perimeterinstitute.ca>

Dear Luis, Huan and Aaron,

like the elephant, turbulence is many things to many people. In the past
decade a quiet revolution has been in progress on one front, determining
sets of exact numerical solutions of the PDEs of motion (no modeling, no
statistical assumptions) that shape and chaperon turbulent motions. So
far it works for 1-spatial dimension Kuramoto-Sivashinski system, and the
full 3-dimensional Navier-Stokes equations in transitional turbulence
regime. We are having very hard time trying to make it work for cardiac
dynamics (model PDEs for 2-dimensional cardiac tissue).

The news from the front have not yet reached physicists who labor outside
our little community, so we have prepared an introduction (click on
ChaosBook.org/tutorials/) for non-specialists. It should be a
self-tutorial, though I find that it seems to work only one-on-one,
either in person, or on Skype :)

Briefly, this description of turbulence works as long as the scales of
coherent structures of interest are not very much smaller than the domain
size. What one needs are the exact equations of motion, accurate
numerical codes, and wisely picked boundary conditions; in your case,
starting with a black hole in a box might be a good idea. The work is
hard - it took 3 years from starting to code Navier-Stokes to the first
of the unstable periodic solutions that you see in the tutorial, so one
does not set out on these computations lightly.

I have always believed that gravity should be turbulent (since my PhD :),
and am very interested to see where your work will lead you. The purpose
of this email is to share a bit of what we have learned: basically forget
statistics and what Kolmogorov could do in 1942. Today we can solve the
exact equations and understand the coherent structures we see in
turbulence.

\item[2016-01-08 Predrag to]
Bogomolny wants us to study Englert and
Schwinger\rf{EnglSchw85a,EnglSchw85b,EnglSchw85c}. Why?
\refRef{EnglSchw85a} {\em Semiclassical atom} seems to be reinventing
Gutzwiller, without citing him.

\end{description}
