% reducesymm/QFT/dailyBlogGX.tex
% Predrag  created              Jan 31 2018

\section{Quantum Field Theory Spring 2018 Report}

\noindent
%Andy Chen (dropped out)
Guopeng Xu <guopengxu@gatech.edu> \\
term paper for Spring / Summer 2018 QFT self-study course.


\paragraph{The goal.}
    Take a narrow path, learn enough Quantum Field Theory (QFT) (and
    Green's functions and such, but no more) to be able to digest
    \refsect{sect:worldline}~{\em Worldline formalism} and check James
    \refsect{sect:magMomWorldline}~{\em Electron magnetic moment in
    worldline formalism} calculations.

 \paragraph{What is a gauge theory?}
According to\rf{belot2002} there are two types of theories
that can be called \lq gauge theories\rq, the Yang-Mills theories and
constrained Hamiltonian theories.

\paragraph{Summary}
In order to get correct predictions from non-Abelian field theories,
which are susceptible to large number of gauge copies, we need to choose
a representative of each gauge orbit.

\section{Morelia Worldline Formalism course}
\label{c-MoreliaCourse}

\begin{description}

\item[2017-07-03 Predrag] First Morelia discussion (all errors are mine):

Christian is inclined to compute (g-2) starting with their two-field
spinor QED Bern-Kosower formula. In that formulation all photons are born equal;
one of them is kept as the external field (the seagull vertex, or the $k^\mu$
coefficient) is the
magnetic moment $\sigma_{\mu\nu}$), and the rest are contracted pairwise in
all possible ways. In the quenched case, this yields one gauge invariant set,
the self-energy set of \refsect{sect:selfEnergy}.

James would like to start with their electron propagator in constant
external field, and keep the term linear in the external field. I vastly
prefer that, because it should be possible to distinguisn in- and
out-legs, and the three kinds of $N$-photon propagators that yield the
minimal gauge sets.

I probably need to got through the proof of gauge invariance with them.

\item[2017-07-04 Predrag] Morelia Schubertiad day 1:
the lecture written up in Schubert\rf{Schubert12} 2012 {\em Lectures on
the worldline formalism}, sects. 1.4~{\em Gaussian integrals} and
1.5~{\em The N-photon amplitude}.

Christian was right. One has to start with scalar QED one-loop effective
action to understand the Bern-Kosowar type master formulas. That yields a
loop with any number of photons attached, each photon vertex carrying a
1D proper time Green's function. This could be computed by usual math
methods techniques for computing Green's functions, but they find it
useful for reasons that will be understood later to compute it as a sum
of Fourier modes. The marginal modes (4 space-time translations) are fixed
in the Gauss way, by shifting the origin to loops center of mass (\ie,
different symmetry reduction for each loop).

We then separate the integration over $x_0$, thus reducing the path integral
to an integral over the relative coordinate $q$:
\beq
x^{\mu}(\tau) = x_0^{\mu}(\tau) + q^{\mu}(\tau),
\ee{Schubert12(1.31)}
with  the relative coordinate $q$ periodic and satisfying constraint
\beq
\int_0^T d\tau\, q^{\mu}(\tau) = 0
\,.
\ee{Schubert12(1.32)}
In the symmetry-reduce$q$-space the zero-mode integral then yields
the energy-momentum conservation $\delta$ function. The 1D Laplacian
$M=-d^2/d\tau^2$ has positive eigenvalues (the usual $k^2$ Fourier modes),
(do the exercise!)
\beq
\det M = (4T)^D
\,,
\ee{Schubert12(1.34)}
and the bosonic Green's function of $-\frac{1}{2}\frac{d^2~}{d\tau^2}$ in
the symmetry-reduced space is
(this ${}^{-2}$ should presumably be ${}^{-1}$?)
\beq
G^c_B(\tau,\tau')
= 2 \bra{\tau}\left(\frac{d^2~}{d\tau^2}\right)^{-2}\ket{\tau'}
= |\tau-\tau'| - \frac{(\tau,\tau')^2}{T} - \frac{T}{6}
\,.
\ee{Schubert12(1.35)}
The first derivative $\dot{G}$  has a sign function, and $\ddot{G}$ has a
$\delta(\tau-\tau')$ (because of the translation invariance, $d/d\tau$
can always e taken to act on the left variable $\tau$).

This results in a Bern-Kosower\rf{BerKos91} type master formula
\begin{eqnarray}
&&\!\!\!\!\!\!\!
\Gamma_{\rm scal}[k_1,\varepsilon_1;\ldots;k_N,\varepsilon_N]
=
\label{Schubert12(1.43)}\\
&&\quad{(-ie)}^N
{(2\pi )}^D\delta (\sum k_i)
{\displaystyle\int_{0}^{\infty}}{dT\over T}
{(4\pi T)}^{-{D\over 2}} e^{-m^2T}
\prod_{i=1}^N \int_0^T\!\!\!d\tau_i
\continue
&&
\quad %\!\!\!\!\!\!\!
\times
\exp\biggl\lbrace\sum_{i,j=1}^N
\Bigl\lbrack  \half G_{Bij} k_i\cdot k_j
-i\dot G_{Bij}\varepsilon_i\cdot k_j
+\half\ddot G_{Bij}\varepsilon_i\cdot\varepsilon_j
\Bigr\rbrack\biggr\rbrace
\mid_{\rm {\rm lin}(\varepsilon_i)}
\nonumber
\end{eqnarray}
for the one-loop
$N$-photon amplitude in scalar QED, with photon
momenta $k_i$ and polarization vectors
$\varepsilon_i$. $m$ denotes the mass, $e$ the charge and $T$ the total proper time
of the scalar loop particle.


This method is not applicable to open
fermion lines.

A scary realisation. This is a special case of our general, nonlinear,
arbitrary order interaction vertex smooth conjugacy calculation of
\refsect{sect:scfpo}. In other words, our calculation is not just for
scalar theory in vacuum, it is for any nonconstant background, and any
order of interaction, \ie, inter alia general relativity. As we start from
a saddlepoint (classical periodic solution) that is not translation
invariant, we do not have to worry about fixing the marginal modes -
there are none.

I fear having to explain the smooth conjugacy method to civilians.

\item[2017-07-05 Predrag] Morelia Schubertiad day 2:
the $N=2$ photon legs case is worked out in Schubert\rf{Schubert12} 2012
lectures sect. 1.6~{\em The vacuum polarization}.

Even though the result is strictly zero (by Furry theorem, or by time
reversal of odd number of $\dot{G}$ functions),
$N=3$ is useful to start understanding how integrations by parts work.

For $N$ photon legs, see sect. 2.5~{\em Integration-by-parts and the
replacement rule} and Ahmadiniaz, Schubert and Villanueva\rf{AhScVi13}
{\em String-inspired representations of photon/gluon amplitudes},
\arXiv{1211.1821}:
``The Bern-Kosower rules provide an efficient way for obtaining parameter
integral representations of the one-loop $N$-photon/gluon amplitudes
involving a scalar, spinor or gluon loop, starting from a master formula
and using a certain integration-by-parts (``IBP'') procedure. Strassler
observed that this algorithm also relates to gauge invariance, since it
leads to the absorption of polarization vectors into field strength
tensors. Here we present a systematic IBP algorithm that works for
arbitrary $N$ and leads to an integrand that is not only suitable for the
application of the Bern-Kosower rules but also optimized with respect to
gauge invariance. In the photon case this means manifest transversality
at the integrand level, in the gluon case that a form factor
decomposition of the amplitude into transversal and longitudinal parts is
generated naturally by the IBP, without the necessity to consider the
nonabelian Ward identities. Our algorithm is valid off-shell, and
provides an extremely efficient way of calculating the one-loop
one-particle-irreducible off-shell Green's functions (``vertices'') in
QCD. In the abelian case, we study the systematics of the IBP also for
the practically important case of the one-loop $N$-photon amplitudes in a
constant field.''

\item[2017-07-06 Predrag] Morelia Schubertiad day 3:

Idrish Huet explained to me how the numerical Monte-Carlos of
worldline path integrals work.
% his partner Christina actually spent a year in Atlanta, working

\item[2017-07-05 Christian] says there is a relevant new arXiv from some Vietnamese
authors, but I couldn't find it.

\item[2017-07-07 Predrag] Morelia Schubertiad day 4:

\item[2017-07-09 5:22 am Predrag]
had a panic attack that we'll never get started on (g-2). Forgot all
about going to Patzquaro, spent entire Sunday writing up the new
\refsect{sect:magMom}~{\em Electron magnetic moment} and
\refsect{sect:magMomWorldline}~{\em Electron magnetic moment in worldline
formalism}.

\item[2017-07-10 Predrag] Morelia Schubertiad day 5:
handwritten notes only.

\item[2017-07-11 Predrag] Morelia Schubertiad day 6:

Christian assigned homework: reformulate the magnetic moment vertex
operator \refeq{BAGTB17(35-1)}, projections \refeq{PRD10-74-III(2.3)} and
\refeq{PRD10-74-III(2.2)} in configuration coordinates.

\item[2017-07-12 Predrag] Morelia Edwardsiad day 7: {\bf Scalar QED open lines}.
Following Ahmadiniaz, Bashir and Schubert\rf{AhBaSc16} 2016
{\em Multiphoton amplitudes and generalized {Landau-Khalatnikov-Fradkin}
transformation in scalar {QED}},  	\arXiv{1511.05087}, and
Ahmadiniaz, Bastianelli and Corradini\rf{AhBaCo16} {\em Dressed scalar
propagator in a non-Abelian background from the worldline formalism},
\arXiv{1508.05144}.

Worldline is sum of N! photon insertions, spinor indices and gauge
transformations at the endpoints. In 1950 Feynman\rf{Feynman50} gave the
scalar QED worldline integral \refeq{AhBaSc16(1)}. Note $dT$ for the open
line, as opposed to $dT/T$ for the closed loop: due to einbein gauge
fixing that leads to different Fadeev-Popov for the loop than for the
line. The worldine Green's function is now
\bea
\Delta(\tau_1,\tau_2) &=& \frac{1}{2}|\tau_1-\tau_2|
          - \frac{1}{2}(\tau_1-\tau_2) + \frac{\tau_1\tau_2}{T}
\label{EdwardsScProp}\\
                     &=&
\frac{1}{2}\left(G_B(\tau_1,\tau_2)
-G_B(\tau_1,0)-G_B(0,\tau_2)+G_B(0,0)
            \right)
\,.
\nnu
\eea

\item[2017-07-13 Predrag] Morelia Edwardsiad day 8:
handwritten notes only.

\item[2017-07-14 Predrag] Morelia Edwardsiad day 9:

Wrote down the master formula for $S^{xx'}_{(N)}$ and its kernel
$K^{xx'}_{(N)}$. Rewrote it in momentum space. Verified that $N=0$
generates the free propagator. Checked the $N=1$ vertex. The hardest was
computing the fermion self-energy, in terms of 2 dimensional
regularization hypergeometric functions. Currently can compare to
Davydychev\rf{DaOsSa00} only numerically.

Read also
Davydychev\rf{Davydychev06}
{\em Geometrical methods in loop calculations and the three-point function}

\item[2018-06-10 Predrag] Les Houches Edwardsiad day 10:

James has an open-line Dirichlet Green's function that should suffice
to compute (g-2). Predrag is sceptical, thinks that Dirichlet Green's function
is a bad choice, as it breaks the translational invariant. Would prefer
some periodic formulation, where a periodic space box is taken off to
infinity, than put on the mass-shell either by some absorptive cut
(yielding \refeq{PRD10-74-III(2.2)}) or amputated and renormalized by
$\sqrt{Z_2}$ as in LSZ formulation.

I thought that in the quenched QED worldline formulation, the
one-particle reducible graphs renormalize (g-2) contributions (my talk\\
\texttt{reducesymm/presentations/LesHouch18/finiteQED.tex}) \\
but my $N$-photon formulation of the QED vertex overcounts counter-terms,
or, as Magnea says in his handwritten
\HREF{http://personalpages.to.infn.it/~magnea/} {lecture 2},
``one-particle reducible graphs should be included only once.''


\end{description}


\section{Xu QFT self-study notes}
\label{c-dailyBlogGX}

\begin{description}

\item[2018-01-25  Predrag to Guopeng]
    Created \refsect{c-dailyBlogGX} for you to write your QFT study notes
    in (to James: they are both undergraduate students, and they are
    expected to form a study group.
    % Andy is from Georgia Tech, has no prior exposure to QFT.
    Guopeng is from Jilin University, has gone
    through the first part of Peskin and Schroeder\rf{PesSch95}.

\item[2018-01-08 Christian Schubert]
that's great the Guopeng wants to learn the worldline formalism. The problem
is that on the part that interests you most, the open fermion line, we
still have not written up anything intelligible. But James is in
the process of writing up his part of the lectures that we gave for you,
which is going to be incorporated into the lecture notes that I have on
the web with Olindo Corradini. So maybe for the time being Guopeng might
want to work through the lectures as they are, and hopefully by the time
he is through with this, James' fermion line part may already be in shape.
Any questions, we are available.

\item[2018-01-09 James Edwards ]
It's great to hear that Guopeng is getting involved in the worldline
formalism. Christian's suggests that you begin with the Olindo Corradini
and Christian Schubert and
 notes\rf{CorSch15}, \arXiv{1512.08694}. 2012 notes by
Schubert\rf{Schubert12},
\HREF{https://indico.cern.ch/event/206621/attachments/317309/442801/lectures_morelia_CS.pdf}
{(click here)} may be also helpful
(Predrag: I think the 2012 notes are included in the entirety into the
Corradini and Schubert  notes\rf{CorSch15}. However, you might find parts of
Corradini 2012 lectures useful,
\HREF{https://indico.cern.ch/event/206621/attachments/317310/442802/LecturesMorelia_OC.pdf}
{(click here)}).

In the meantime, I will write up my notes on open fermion lines and share
them with you as soon as possible. Needless to say, if Guopeng has any
questions he is most welcome to get in touch with us to discuss things in
more detail.

\item[2018-01-16 James]
Would Guopeng like to get involved in some ongoing calculations pertaining
to g-2? Our worldline representation of the fermion propagator provides a
new technique for attacking this problem and there has already been some
progress in calculating some of the ingredients for the 3 loop
contribution, but we would also like to look at 1+ loop rainbow diagrams.

To achieve this, we would study the open fermion line with a low energy
photon attached along with a certain number of virtual photon loops.
Now we know how to deal with the fermion line we have a good idea
of how to calculate these quantities in a particularly efficient way
(that automatically picks out the term linear in $k$ with the right gamma
matrix structure to allow for extraction of the structure constant we
seek). We expect to be able to streamline the calculation  and maybe even
earn some advantages in the limit of large numbers of virtual photons. In
principle, this can be extended to the case of a constant electromagnetic
background too.

We can begin working by checking the one loop contribution as a warm-up
(I have notes on this already) before looking at the 2-loop rainbow. This
requires familiarity with the scalar propagator in the worldline approach
before looking at open spinor lines and the coupling to external photons.
The latter remain as my job to write up our notes from Predrag's visit,
and I will strive to do this asap.

\item[2018-02-13 Predrag]
    I find discrete lattice problems very useful in understanding
    concepts in QFT. Study
    \HREF{http://chaosbook.org/FieldTheory/postscript.html}
    {ChaosBook.org/FieldTheory/postscript.html} Chapter~1 {Lattice field
    theory}. You will learn that Laplacian generates all walks on a
    lattice (making it easier to understand path integrals), that a
    Green's function is the propagator (sum over all walks), that the
    mysterious $e^{iET/\hbar}$ and $e^{ip\cdot{x}/\hbar}$'s of Quantum
    Mechanics are roots of unity (a consequence of time and space
    translation invariance), and how that enables you to compute the
    propagators / Green's functions.

    All of the above can be then continued in the small lattice spacing
    limit into things you accept on faith when you read QFT textbook.
    But here it is just matrices and vectors, and you can check every
    step.

    Do not use much time on the discrete lattice formalism, as we shall
    not use it in the current project.

\item[2018-02-18 Predrag] To see whether you have actually understood
Green's function on the level needed for our term's goals, see whether
you can follow worldline Green's function calculations following
eq.~\refeq{AhBaSc16(1)i}, eq.~\refeq{EdwardsScProp} above, eqs.~(2.35)
and (2.41) in Corradini and Schubert notes\rf{CorSch15}.

%\item[2018-02-23 Andy] I have showed Guopeng how to compile tex files and
%this is a test push commit.

\item[2018-03-27 Guopeng to James]
I think I have understood the material ``Spinning particle in QM and QFT.'' I
skipped the 1.4 section and 2.5. If you could give me the calculation,  I can
go through it.

\item[2018-03-28 James]
Great to hear that you have gone through the material from the notes. At this
stage, I don't think you need to worry about the gravitational case in section
1.4 and it's true that initially we are more concerned with Abelian QED (that
said, I would recommend that you do take a look at section 2.5.1 when you have
a chance, just because it gives the simplest extension of the worldline
formalism to the non-Abelian case). These days we have some more powerful
techniques to deal with colour degrees of freedom, but it's good to see the
simpler presentation in that section).

I understand that you were referring to the material on open lines in the
worldline approach, which is of course relevant to the g-2 calculation. I have
almost written up my notes from Predrag's visit and hope to send them to you by
the end of the week; I hope you don't mind being an ``alpha tester'' for these
notes as one of the first people to go through them! I would encourage you to
go through them critically in case any typos have crept in. At the very least,
I will send you the vacuum propagators for the scalar and spinor particle this
week.

In the meantime, do let us know if you have any questions about the notes or
exercises so far. If you're eager to get started, I would suggest that you
review the path integral calculation of quantum mechanical kernels (\ie\ matrix
elements of the form $K(x, y; T) = \bra{y} \exp[-i T H] \ket{x}$ including
electromagnetic interactions that will be used repeatedly in the open line
calculations. For example, sects.~2.18 to 2.20 of Kleinert\rf{Kleinert09} path
integral textbook has full details.
    \PC{2018-04-16
    I have a copy of Kleinert\rf{Kleinert09}}

\item[2018-03-31 Predrag to Guopeng - time to start LaTexing]
Skip section 2.5.1 (extension of the worldline formalism to the non-Abelian
case) in Corradini and Schubert notes\rf{CorSch15}, we are very short on time.
Focus on QED. I believe you have gone through the path integral calculation of
quantum mechanical kernel.

Instead, please write up your derivation of the Green's functions (1.35) and
(2.15), then (1.41), here, in this text, in LaTeX, with all signs and
prefactors correct. If you have time before James' notes arrive, going through
sect.~2.4 {\em The vacuum polarization} might be good - that connects these
mysterious worldline Green's functions to the standard Feynman integrals like
hose you see in Peskin and Schroeder.

\item[2018-04-16 Predrag to Guopeng]
Go through
Stone and Goldbart\rf{StGo09}, {\em Mathematics for Physics: A Guided Tour
for Graduate Students}, Chapter~5  \emph{Green Functions}.
A pre-publication draft can be found
\HREF{http://www.goldbart.gatech.edu/PostScript/MS\_PG\_book/bookmaster.pdf}
{here}.

It is all about 1\dmn\ Green's functions - understanding material related to
kind of function you need for worldline formalism should be helpfull to you.

Stone and Goldbart\rf{StGo09} is an advanced summary where you will find
almost everything one needs to know. More pedestrian and perhaps easier to
read is Arfken and Weber\rf{ArWe05}.
% {\em Mathematical Methods for Physicists: A Comprehensive Guide},
% \HREF{http://birdtracks.eu/courses/PHYS-7143-17/ArWe05chap3.pdf}
% {Chapter 3}.

I like Mathews and Walker\rf{MathWalk73}, based on lectures by Richard
Feynman at Cornell University. You can download it from
\HREF{https://www.scribd.com/doc/225369262/Mathematical-Methods-of-Physics-2nd-Edition-Mathews-Walker}
{here}. However, I'm not sure it will help you understand the problem at
hand.

The above examples are the usual way students are taught Green's functions. I
personally find deriving them as continuum limits of lattice formulations
(\HREF{http://chaosbook.org/FieldTheory/QMlectures/lectQM.pdf\#chapter.1}
{click here}) the most insightful and easiest to understand. My derivation
is for periodic boundary conditions. You will have to also understand other kinds of
boundary conditions that arise in your project.

\item[2018-07-12 Guopeng]
I added the Edwards reference
Ahmadiniaz \etal\rf{ABCES17}
{\em One-particle reducible contribution to the one-loop spinor propagator in a constant field}

\item[2018-07-12 Guopeng]
I added the Edwards / Cvitanovi\'c June 2018 whiteboard notes, \emph{reducesymm/guopeng}:

\begin{description}
  \item[Mrenorm.pdf]
The perturbative expansions for the magnetic moment anomaly \refeq{IRstruct(1)}.
  \item[M4.pdf]
  \item[GammaVertex.pdf]
  \item[WardIds.pdf]
  \item[worldLprop.pdf]
  \item[M4.pdf]
  \item[aSums.pdf]
The next five files are gauge sets $a_{km'm}$ defined in \refeq{quenchAnom}:
  \item[a111.pdf]
  \item[a111Gset.pdf]
  \item[a111old.pdf]
  \item[a120.pdf]
  \item[a210.pdf]
  \item[1loopPhi3.pdf] is an outline of Edwards purported 1-loop calculation.
\end{description}

\item[2018-07-13 Guopeng]
working through Edward's {\bf 1loopPhi3.pdf} calculation sketch:

The note starts out with the $N$ photon insertions propagator for a
charged scalar field in external field \refeq{Nprop}
\bea
\mbox{$N$ photon ins. fig}
%\langle 0|T \phi (x) \phi (y) |0\rangle_{(N)}
&=&
 \int_0^\infty \! dT \,e^{-m^2 T}
  \continue
&=& (ie)^N
 \int_0^\infty \! dT \,e^{-m^2 T}
 \int_0^Td\tau_1 \cdots \int_0^Td\tau_N
 \nonumber\\ &&
\times  \int_{_{x(0)=y}}^{^{\, x(T)=x}}
\!\!\!\!\!\!\!\!\!\!\!\! {\cal D}x
\,e^{i\sum_{i=1}^Nk_i\cdot x(\tau_i)}
e^{ -\int_0^Td\tau\, {1\over 4} \dot x^2}
\,.
\label{Nprop1}
\eea


The expression
$\#/[(x_a-x_b)^2]^{D/2-1}$ can be obtained by change of variable
$u=1/T$, and evaluating the Gaussian integral
\[
\int_{0}^{\infty} \!\frac{dT}{(4\pi T)^{D/2}} e^{-(x_a-x_b)^2/4T}
\,.
\]
{\bf The Ingredient \#1}: I wonder if there should be a $T$ in the denominator
of the derivative term, because I can't find the extra $T$ in the  eq.~(2.11)
of ``Spinning particles in the QM and QFT"\rf{CorSch15}. I calculate it is by
factorizing the Laplacian,
\bea
\Det\!\left(-\frac{1}{4}\frac{\partial^2}{\partial\tau^2}+\frac{B_{ab}}{4T}\right)
&=& \Det\!\left[-\frac{1}{4}\frac{\partial^2}{\partial\tau^2}
            \left(1-(\frac{\partial^2}{\partial\tau^2})^{-1}\frac{B_{ab}}{T}\right)
        \right]
\label{GXgreen1}\\
&=& \Det\!\left(-\frac{1}{4}\frac{\partial^2}{\partial\tau^2}\right)
    \,
    \Det\!\left(1-(\frac{\partial^2}{\partial\tau^2})^{-1}\frac{B_{ab}}{T}\right)
\,.
\nnu
\eea
where
$\Det\!\left(-\frac{1}{4}\frac{\partial^2}{\partial\tau^2}\right)=(4\pi T)^{-\frac{D}{2}}$.
As for the
$\Det\!\left(1-(\frac{\partial^2}{\partial\tau^2})^{-1}\frac{B_{ab}}{T}\right)$,
it would appear that
\[
-(\frac{\partial^2}{\partial\tau^2})^{-1}\frac{B_{ab}}{T}=\frac{G_{Bab}}{T}
\,,
\]
so that the determinant could be $(1+\frac{G_{Bab}}{T})^{-\frac{D}{2}}$,
where $G_{Bab}$ is the loop Green's function.
Note that
\[
B_{ab}(\tau_1,\tau_2)
=[\delta(\tau_a-\tau_1)-\delta(\tau_b-\tau_1)][\delta(\tau_a-\tau_2)-\delta(\tau_b-\tau_2)]
\,,
\]
thus
$-(\frac{\partial^2}{\partial\tau_1^2})^{-1}\frac{B_{ab}}{T}$
is proportional to
$\tau-\frac{1}{2}(\tau_a+\tau_b)$,
since
\bea
\frac{\partial^2}{\partial\tau_1^2}|\tau_1-\tau_2|
&=& 2\delta(\tau_1-\tau_2)
    \continue
|\tau_a-\tau|-|\tau_b-\tau|
&=& 2\tau-(\tau_a+\tau_b)
\,.
\label{GXgreen2}
\eea

I don't know why this is equal to the loop Green's function.

\item[2018-07-16 Guopeng]
{\bf The Ingredient \#2}:
\[
\Delta_{(1)}(\tau_1,\tau_2|\tau_a,\tau_b)
:=\bra{\tau_1}-\frac{1}{4T}\frac{\partial^2~}{\partial\tau^2}
    +\frac{B_{ab}}{4\bar{T}}\ket{\tau_2}
=\bra{\tau_1}-\frac{1}{4T}\frac{\partial^2~}{\partial\tau^2}\ket{\tau_2}
+\bra{\tau_1}\frac{B_{ab}}{4\bar{T}}\ket{\tau_2}
\]
 The first term,it's the Green function with DBC.According to James's calculation,the second term should be
 \[
 \frac{[\Delta_{1a}-\Delta_{1b}][\Delta_{a2}-\Delta_{b2}]}{\bar{T}+G_{Bab}}
 \]
 But actually I don't quite understand this result,if James could explain more to me, I think I could catch its point.
 In my opinion, If we extract $\bar{T}$ in the denominator, then the denominator becomes $1+\frac{G_{Bab}}{\bar{T}}$,which
 is relevant to the ingredient \#1

An extra confusion about the master formula in the top of the 1loopPhi3  calculation.It seems like there is no linear part
of the polarization vector,so where are those vector going?







% enter blog posts before here
\end{description}

%\newpage %%%%%%%%%%%%%%%%%%%%%%%%%%%%%%%%%%%%%%%%%%%%%%%%
\printbibliography[heading=subbibintoc,title={References}]


\renewcommand{\ssp}{a}
