% reducesymm/QFT/dailyBlog.tex
% $Author$ $Date$
% Predrag  switched to github.com               jul  8 2013
% former siminos/blog/dailyBlog.tex

\chapter{General Relativity}
\label{c-GR}



\section{Daily blog}
\label{sect:GRblog}

\begin{description}
\item[2014-06-06 Deirdre Shoemaker]
A new article\rf{YaZiLe14} out by
\HREF{http://perimeterinstitute.ca/people/luis-lehner}
{Lehner},
\HREF{https://sites.google.com/site/aaronbzimmerman/} {Zimmerman}
and Huan Yang,
\arXiv{1402.4859},
explores turbulence in gravity - hitherto thought to be unlikely or
impossible. They write:

``
We show that rapidly-spinning black holes can display turbulent
gravitational behavior which is mediated by a new type of parametric
instability. This instability transfers energy from higher temporal and
azimuthal spatial frequencies to lower frequencies--- a phenomenon
reminiscent of the inverse energy cascade displayed by 2+1-dimensional
turbulent fluids. Our finding reveals a path towards gravitational
turbulence for perturbations of rapidly-spinning black holes, and
provides the first evidence for gravitational turbulence in an
asymptotically flat spacetime. Interestingly, this finding predicts
observable gravitational wave signatures from such phenomena in black
hole binaries with high spins and gives a gravitational description of
turbulence relevant to the fluid-gravity duality.
''


Lorena <lm\_zertuche@yahoo.com> writes to Deirdre:
Below is an excerpt from
\HREF{http://phys.org/news/2014-06-gravitational-fields-black-holes-eddy.html}
{this article}. Do you find this surprising?

``The team decided to study fast-spinning black holes, because a
fluid-dynamics description of such holes hints that the spacetime around
them is less viscous than the spacetime around other kinds of black
holes. Low viscosity increases the chance of turbulence - think of the
way water is more swirly than molasses.

The team also decided to study non-linear perturbations of the black
holes. Gravitational systems are rarely analyzed at this level of detail,
as the equations are fiendishly complex. But, knowing that turbulence is
fundamentally non-linear, the team decided a non-linear perturbation
analysis was exactly what was called for.

They were stunned when their analysis showed that spacetime did become
turbulent.
``I was quite surprised,'' says Yang, who has been studying general
relativity (GR) - Einstein's theory of gravity - since his PhD. ``I never
believed in turbulent behaviour in GR, and for good reason. No one had
ever seen it in numerical simulations, even of dramatic things like
binary black holes.''

``Over the past few years, we have gone from a serious doubt about
whether gravity can ever go turbulent, to pretty high confidence that it
can,'' says Lehner.
How did this behaviour hide until now? ``It was hidden because the
analysis needed to see it has to go to non-linear orders,'' says Yang.

This
\HREF{http://phys.org/news/2014-04-liquid-spacetime-slippery-superfluid.html}
{blurb} might also be of interest.

\item[2014-06-07 Predrag to]
Huan Yang <huan.yang07@gmail.com>,
Aaron Zimmerman <azimmer@cita.utoronto.ca>,
Luis Lehner <llehner@perimeterinstitute.ca>

Dear Luis, Huan and Aaron,

like the elephant, turbulence is many things to many people. In the past
decade a quiet revolution has been in progress on one front, determining
sets of exact numerical solutions of the PDEs of motion (no modeling, no
statistical assumptions) that shape and chaperon turbulent motions. So
far it works for 1-spatial dimension Kuramoto-Sivashinski system, and the
full 3-dimensional Navier-Stokes equations in transitional turbulence
regime. We are having very hard time trying to make it work for cardiac
dynamics (model PDEs for 2-dimensional cardiac tissue).

The news from the front have not yet reached physicists who labor outside
our little community, so we have prepared an introduction (click on
ChaosBook.org/tutorials/) for non-specialists. It should be a
self-tutorial, though I find that it seems to work only one-on-one,
either in person, or on Skype :)

Briefly, this description of turbulence works as long as the scales of
coherent structures of interest are not very much smaller than the domain
size. What one needs are the exact equations of motion, accurate
numerical codes, and wisely picked boundary conditions; in your case,
starting with a black hole in a box might be a good idea. The work is
hard - it took 3 years from starting to code Navier-Stokes to the first
of the unstable periodic solutions that you see in the tutorial, so one
does not set out on these computations lightly.

I have always believed that gravity should be turbulent (since my PhD :),
and am very interested to see where your work will lead you. The purpose
of this email is to share a bit of what we have learned: basically forget
statistics and what Kolmogorov could do in 1942. Today we can solve the
exact equations and understand the coherent structures we see in
turbulence.

\item[2016-08-20 Predrag]
Shipley and Dolan\rf{ShiDol16}
{\em Binary black hole shadows, chaotic scattering and the {Cantor} set}
uses ChaosBook.org 3-disk billiard: ``
We investigate the qualitative features of binary black hole shadows
using the model of two extremally charged black holes in static
equilibrium (a Majumdar-Papapetrou solution). Our perspective is that
binary spacetimes are natural exemplars of chaotic scattering , because
they admit more than one fundamental null orbit, and thus an uncountably
infinite set of perpetual null orbits which generate scattering
singularities in initial data. Inspired by the three-disc model, we
develop an appropriate symbolic dynamics to describe planar null
geodesics on the double black hole spacetime. We show that a
one-dimensional (1D) black hole shadow may be constructed through an
iterative procedure akin to the construction of the Cantor set; thus the
1D shadow is self-similar. Next, we study non-planar rays, to understand
how angular momentum affects the existence and properties of the
fundamental null orbits. Taking slices through 2D shadows, we observe
three types of 1D shadow: regular, Cantor-like, and highly chaotic. The
switch from Cantor-like to regular occurs where outer fundamental orbits
are forbidden by angular momentum. The highly chaotic part is associated
with an unexpected feature: stable and bounded null orbits, which exist
around two black holes of equal mass M. To show how this possibility
arises, we define a certain potential function and classify its
stationary points. We conjecture that the highly chaotic parts of the 2D
shadow possess the Wada property. Finally, we consider the possibility of
following null geodesics through event horizons, and chaos in the
maximally extended spacetime.
''

\item[2017-05-03 Predrag]
If I get serious about learning something about GR, this looks like a good
overview:
J. Isenberg
\HREF{http://www.iamp.org/bulletins/Bulletin-Apr2017-print.pdf}
{The Mathematical Side of General Relativity: Part I}

\item[2017-05-22 Predrag]
Jai Grover and Alexander Wittig\rf{GroWit17}
{\em Black hole shadows and invariant phase space structures}:

``Utilizing concepts from dynamical systems theory, we demonstrate how
the existence of light rings, or fixed points, in a spacetime will give
rise to families of periodic orbits and invariant manifolds in phase
space. It is shown that these structures define the shape of the black
hole shadow as well as a number of salient features of the spacetime
lensing. We illustrate this through the analysis of lensing by a hairy
black hole. ''

The paper looks elementary enough that one could learn something about
chaos in GR from it.


\end{description}

%\newpage %%%%%%%%%%%%%%%%%%%%%%%%%%%%%%%%%%%%%%%%%%%%%%%%
\printbibliography[heading=subbibintoc,title={References}]
