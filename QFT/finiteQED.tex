% reducesymm/QFT/finiteQED.tex
% $Author$ $Date$
% Predrag  switched to github.com               jul  8 2013

\begin{bartlett}{
Is there any method of computing the anomalous moment of the
electron which, on first approximation, gives a fair approximation to the
$\alpha$ term and a crude one to $\alpha^2$; and when improved, increases
the accuracy of the $\alpha^2$ term, yielding a rough estimate to
$\alpha^3$ and beyond?
        }
\bauthor{
Feynman's challenge, 12th Solvay Conference\rf{Feynman62}
    }
\end{bartlett}



\begin{description}

\item[2017-05-21 Predrag] to
\\
Sergey  A. Volkov <volkoff\_sergey@mail.ru>
\\
Skobeltsyn Institute of Nuclear Physics,
\\
Moscow State University, Moscow, 119234 Russia

Dear Sergey

I have just read your
{\em New method of computing the contributions of graphs without lepton
loops to the electron anomalous magnetic moment in {QED}}\rf{Volkov17}
(and the earlier \refref{Volkov16}) with great interest, and I can see
that you are set to compute the 5-loop correction to the electron
$(g-2)$. This is a very hard calculation, and approaching it
strategically is a necessity.

May I suggests that you order your calculation by gauge sets of
\reffig{tabGaugeSets}, as illustrated (for the 4-loop case) in Stefano
Laporta's \reffig{Laporta17figuragauShort}? My hunch is that the gauge
set (6) = $(4,0,0)$ would be the most interesting, though Stefano thinks
it too hard, and suggests starting with a 5-loop relative $(1,3,1)$ (or
$(1,2,2)$?) of the set (3) = $(1,2,1)$ instead. While the contributions
of individual vertex graphs (and self-energy sets\rf{AoHaKiNi15}) are all
over the place, all gauge-invariant sets are insanely small up to order
8, and it would be very sweet to see that this continues through order 10
(at least for the 5-loop graphs with no electron loops).

By the way, to check my conjecture one needs the gauge sets only to two
significant digits or so, no high accuracy is needed.

best regards
\\
Predrag


\end{description}

\newpage

As a prelude to these notes, you might enjoy the
introduction to Dunne and Schubert 2005 paper\rf{DunSch06}
%{\em Multiloop information from the {QED} effective lagrangian}
which starts with a nice historical review of ideas about the the
perturbation series in QED. They note:
``Despite of the many insights which have been gained along these lines,
a point which remains poorly understood is the influence of gauge
cancellations on the divergence structure of a gauge theory.''

In this spirit, I start here by explaining the numerics that motivates
the QED finiteness conjecture, and then review the approaches that offer
a promise of establishing it. A proof of the conjecture might be within
reach; so might be methods for computing gauge
invariant QFT sets, without recourse to traditional Feynman diagrams.

\section{Gauge sets}
\label{sect:finitness}

%%%%%%%%%%%%%%%%%%%%%%%%%%%%%%%%%%%%%%%%
\begin{figure}
\begin{center}
\includegraphics[width=1.00\textwidth]{JegNif09fig10}
\end{center}
\caption{\label{JegNif09fig10}
The three-loop vertex diagrams contributing to $A^{(6)}_1$
magnetic moment
(from Jegerlehner and Nyffeler\rf{JegNif09}).
Lautrup \etal\rf{LaPeRa72} were the first to note that
subsets
% $(k,m,m')$ =
$(3,0,0)$ = $\{23,24,25,26,28\}$;
$(2,1,0)$ = $\{29,31,33,35,37,39,41,43,45,47\}$ and its time-reversal
$(2,0,1)$ = $\{30,32,34,36,38,40,42,44,46,48\}$;
$(1,2,0)$ = $\{49,51,53,55,57,59,61,63,65,67\}$ and its time-reversal
$(1,0,2)$ = $\{50,52,54,56,58,60,62,64,66,68\}$;
and
$(1,1,1)$ = $\{69,70,71,72\}$
are the minimal gauge sets, see \reffig{Cvit77bFig2}.
}
 \end{figure}
%%%%%%%%%%%%%%%%%%%%%%%%%%%%%%%%%%%%%%%%

%%%%%%%%%%%%%%%%%%%%%%%%%%%%%%%%%%%%%%%%%%
\begin{table}
\begin{center}
\includegraphics[width=0.80\textwidth]{Cvit77bTab1}
\end{center}
\caption{\label{Cvit77bTab1}
Comparison of the number of vertex diagrams without fermion loops, gauge
sets, and the gauge-set approximation \refeq{Cvit77b(1)} for the magnetic
moment in $2n$th order.
From \refref{Cvit77b}.
}
\end{table}
%%%%%%%%%%%%%%%%%%%%%%%%%%%%%%%%%%%%%%%%%%%%%%%%%%%%%%%%%%%

% finiteQEDins.tex
% PC created from lectures/talks/DFS_pris.tex       2017-05-25
% PC with edits March 2003
% PC with edits July 1997
%Acceptance speech - 1993 NKT Research Prize in Physics
%Dansk Fysisk Selskab \AA rsm\o de, Lalandia, R\o dby, 18 maj 1993
In 1972 Toichiro Kinoshita and I had completed computing a large number
of 3-loop anomalous magnetic moment Feynman diagrams and regularization
counterterms\rf{CviKin72}, \reffig{JegNif09fig10}.
The subsequent 4- and 5-loop numerical and analytic calculations
are nothing short of heroic\rf{KinLin90,Laporta17,AoHaKiNi15}.
We had used the quantum field theory in the standard way, by expanding
the magnetic moment into combinatorially many Feynman diagrams (see the
numbers of vertex graphs in \reftab{Cvit77bTab1}). Each Feynman diagram
corresponds to an integral in many dimensions, with oscillatory integrand
with thousands of terms, each integral separately UV divergent, IR
divergent, and unphysical, as its value depends on the definition of
counterterms and the choice of gauge. The numerical values of these
integrals are all over the place, in our case $\pm 10$ to $\pm 100$ was a
typical range.

We added up hundreds of such contributions, of wildly fluctuating values,
and obtained (for the no-fermion loops subset $V$, in the notation of
\refref{AoHaKiNi15})
\[
 \aql{6}\,=\, +  (0.92 \pm 0.02) \left(\frac{\alpha}{\pi}\right)^3.
\]
But why ``+'' and not ``-''? Why so small? Why does a sum of hundreds of
diagrams and counterterms yield a number of order of unity, and not 10 or
100 or any other number?

%%%%%%%%%%%%%%%%%%%%%%%%%%%%%%%%%%%%%%%%
\begin{figure}
\begin{center}
\includegraphics[width=0.50\textwidth]{Cvit77bFig1}
\end{center}
\caption{\label{Cvit77bFig1}
Rows: the fourth-order gauge sets
$(k,m,m')$: (1) = $(1,1,0)$,
(2) = $(2,0,0)$
and
(3) = $(1,0,1)$.
Columns: external field insertions into the two self-energy sets.
For diagrams related by time
reversal (here (1) and (3))
the value listed under the first diagram of the pair is
the total contribution of the pair. Contributions seem to be of order
$\pm\frac{1}{3}\left(\frac{\alpha}{\pi}\right)^2$, and suggest that
a set and its time-reversed partner should be counted separately.
From \refref{Cvit77b}.
}
 \end{figure}
%%%%%%%%%%%%%%%%%%%%%%%%%%%%%%%%%%%%%%%%

If gauge invariance of QED guarantees that all UV and on-mass shell IR
divergences cancel, could it be that it also enforces cancellations among
the finite on-mass shell contributions?

As first noted by Lautrup, Peterman and de Rafael\rf{LaPeRa72}, the
renormalized on-mass shell QED vertex diagrams separate into a sum of
minimal gauge-invariant subsets, each subset separately UV and IR finite.
To simplify matters, in what follows we shall consider only the
no-fermion loop diagrams, or `quenched-', or `q-type' diagrams
(`quenched', as this corresponds to the $O(N_f)$ part of the amplitude in
QED with $N_f$ flavors).
The minimal gauge-invariant subsets without electron loops (see
\reffig{JegNif09fig10} diagrams $\{23-72\}$; \reffig{Cvit77bFig1};
\ref{Cvit77bFig2}; \ref{Cvit77bFig3}; and \ref{Laporta17figuragauShort})
will be hereafter be referred to as \emph{gauge sets}.

A gauge set
$(k,m,m')$ consists of all 1-particle irreducible vertex diagrams
without electron loops, with $k$ photons crossing the external vertex
(cross-photons) and $m [m']$ photons originating and terminating on the
incoming [outgoing] electron leg (leg-photons), where $m\geq m'$. For
asymmetric pairs of sets, with $m\neq m'$, the contribution to the
anomaly $a_{kmm'}$ is, in my definition, the sum of the set and its
mirror (time-reversed) image,
\beq
a[V]=\left.\frac{1}{2}(g-2)\right|_V
       =  \sum_{k=1}^\infty\sum_{m=0}^\infty\sum_{m'=0}^m
          a_{kmm'}\left(\frac{\alpha}{\pi}\right)^{k+m+m'}
\,.
\ee{quenchAnom}

When the diagrams that we had computed\rf{CviKin74c} are grouped into
gauge sets, \reffig{Cvit77bFig1} to \reffig{Laporta17figuragauShort},
a surprising thing happens; while the
finite part of each Feynman diagram is of order of 10 to 100, every
known gauge set adds up to approximately
$$
		   \pm {1 \over 2} \left(\frac{\alpha}{\pi}\right)^n
\,,
$$
with the sign given by a simple empirical rule
\beq
a_{kmm'} = (-1)^{m+m'}\frac{1}{2}
\,.
\ee{Cvit77b(5)}
The sign rule is further corroborated by sets with photon
self-energy insertions (but with the absolute size scaled down to
$3-15\%$ of \refeq{Cvit77b(5)}).
In \reffig{Cvit77bFig3} I compared this rule with the actual numbers and
made my 1977 four-loop prediction\rf{Cvit77b}.

The ``zeroth'' order
estimate of the electron magnetic moment anomaly $a$ is now given by
the ``gauge-set approximation,'' convergent and summable to all orders
\beq
a=\frac{1}{2}(g-2) =  \frac{1}{2} \frac{\alpha}{\pi}
                     \frac{1}
           {\left( 1 - \left(\frac{\alpha}{\pi}\right)^2
			\right)^2
		      } + \mbox{``corrections"}
\,.
\ee{Cvit77b(1)}
This is not how one usually thinks of perturbation theory. Most of our
colleagues believe that in 1952 Dyson\rf{Dyson52} had  shown that the
perturbation expansion is an asymptotic series (for a discussion, see
Dunne and Schubert\rf{DunSch06}), in the sense that the $n$-th order
contribution should be exploding combinatorially
$$
{1 \over 2} (g-2) \approx
\cdots + n^n \left(\frac{\alpha}{\pi}\right)^n + \cdots
\,,
$$
and not growing slowly like my estimate
\[
{1 \over 2} (g-2) \approx
\cdots + \frac{n}{2}\left(\frac{\alpha}{\pi}\right)^{2n} + \cdots
\,.
\]
For me, the above is the most intriguing hint that something deeper than
what we know today underlies quantum field theory, and the most suggestive
lesson of our calculation.

%%%%%%%%%%%%%%%%%%%%%%%%%%%%%%%%%%%%%%%%%%%%%%%%%%%%%%%%
\begin{sidewaysfigure}[p]
\thisfloatpagestyle{empty} % Lucy Day Apr 18 2008
\center{
\includegraphics[width=0.45\textwidth,angle=-90]{Cvit77bFig2}
%\end{center}
        } %end \center{
\caption{\label{Cvit77bFig2}
Every vertex diagram belongs both to a `gauge set' and to a `self-energy set'.
This table illustrates the two kinds of sets.
The 3-loop gauge sets $(k,m,m')$ are arranged in the rows, and the
self-energy sets (or the `externally gauge-invariant' sets, vertex diagrams
obtained by inserting an extra vertex into a self-energy diagram) in the
columns, labeled as in Fig.~3 of \refref{CviKin74c}. The values are
finite parts in the $\ln\lambda$ IR cut-off approach, such as those
listed in \refref{LevWri73}. For different IR separation methods (such as
in \refref{CviKin74c}) and different gauges, individual diagrams have
different values. The gauge sets, however, are separately gauge invariant.
The self-energy sets (whose number grows combinatorially with the order
in perturbation theory) are not, only their sum is gauge invariant.
From \refref{Cvit77b}.
}
 \end{sidewaysfigure}
%%%%%%%%%%%%%%%%%%%%%%%%%%%%%%%%%%%%%%%%

%%%%%%%%%%%%%%%%%%%%%%%%%%%%%%%%%%%%%%%%
\begin{figure}
\begin{center}
\includegraphics[width=0.90\textwidth]{Cvit77bFig3}
\end{center}
\caption{\label{Cvit77bFig3}
Comparison of the 1977 gauge-set approximation to the anomaly $a$
and the actual numerical
values of corresponding gauge sets, together with the  1977 eighth-order
prediction of \refref{Cvit77b}. For the updated listing, see
\reffig{tabGaugeSets}.
}
 \end{figure}
%%%%%%%%%%%%%%%%%%%%%%%%%%%%%%%%%%%%%%%%

%%%%%%%%%%%%%%%%%%%%%%%%%%%%%%%%%%%%%%%%%%
%%%%%%%%%%%%%%%%%%%%%%%%%%%%%%%%%%%%%%%%%%%%%%%%%%%%%
% tabGaugeSets.tex    2017-06-02
% compiled by  reducesymm/QFT/blog.tex
% needs \usepackage{booktabs}\usepackage{amsmath}
\begin{figure}
\centering
\begin{tabular}{r@{~~~~}ccccc@{~~~~}l}
$2n$ & \multicolumn{5}{c}{$km'm$} & anomaly \\
    \toprule[1.5pt]\\[-1.0em]
% Entering  row 2
 & $\bf (1,0,0)$
 \\[-1ex]
\raisebox{1.5ex}{2}
 & 1/2            &&&&& \raisebox{1.5ex}{$\frac{1}{2}$}
  \\[1ex]
 \cmidrule(lr){2-3}\\[-0.8em]
% Entering  row 4
 & $\bf (1,1,0)$  &  $\bf (2,0,0)$
 \\[-1ex]
\raisebox{1.5ex}{4}
 & -1/2 (-.65)&  1/2  (.31) &&&& \raisebox{1.5ex}{0 (-.33)}
  \\[1ex]
 \cmidrule(lr){2-4}\\[-0.8em]
% Entering  row 6
 & $\bf (1,2,0)$ & $\bf (2,1,0)$   & $\bf (3,0,0)$
 \\[0.1ex]
 & 1/2 (.56) & -1/2 (-.47) &  1/2 (.44)
 \\%[-1ex]
\raisebox{1.5ex}{6}
 & $\bf (1,1,1)$ &&&&&          \raisebox{1.5ex}{1 (.93)}\\
 & 1/2 (.43)
  \\[1ex]
 \cmidrule(lr){2-5}\\[-0.8em]
% Entering  row 8
 & $\bf (1,3,0)$     & $\bf (2,2,0)$  & $\bf (3,1,0)$  & $\bf (4,0,0)$
 \\[0.1ex]
 &  -1/2{\color{red}$\cdot$4} (-1.97)
                     & 1/2{\color{red}$\cdot$0 (-.14)}
                                      & -1/2{\color{red}$\cdot$2} (-1.04)
                                                        &  1/2 (.51)
 \\%[-1ex]
\raisebox{1.5ex}{8}
 & $\bf (1,2,1)$  & $\bf (2,1,1)$ &&&& \raisebox{1.5ex}{0 (-2.17)}\\
 & -1/2 (-.62)    &   1/2{\color{red}$\cdot$2} (1.08)
  \\[1ex]
 \cmidrule(lr){2-6}
% Entering  row 10
 & $\bf (1,4,0)$ & $\bf (2,3,0)$  & $\bf (3,2,0)$
                                        & $\bf (4,1,0)$
                                            & $\bf (5,0,0)$
 \\[0.1ex]
 &    1/2{\color{red}$\cdot$12} (6.2)
                 & -1/2 (-0.81)   & 1/2 {\color{red}(-0.41)}
                                        & -1/2{\color{red}$\cdot$2} (-1.06)
                                             &  1/2{\color{red}$\cdot$2} (1.09)
 \\%[-1ex]
\raisebox{1.5ex}{10}
 & $\bf (1,3,1)$  & $\bf (2,2,1)$ & $\bf (3,1,1)$ &&& \raisebox{1.5ex}{$\frac{3}{2}$ (7.60)}\\
 &  1/2 (0.75)    & -1/2{\color{red}$\cdot$4} (-2.1)
                                  & 1/2{\color{red}$\cdot$5} (2.64)
  \\[1ex]
 & $\bf (1,2,2)$ \\
 & 1/2 (0.36)
  \\[1ex]
\bottomrule
\end{tabular}
\caption{\label{tabGaugeSets}
Updated \reffig{Cvit77bFig3} comparison of the gauge-set approximation
\refeq{Cvit77b(1)} and the actual numerical values of corresponding gauge
sets, together with the 5-loop prediction. Starting with 4-loops, the
gauge-set approximation $\pm1/2$ fails in detail. Still,
the signs are right (except for the anomalously small set $(2,2,0)$, and its
``descendent'' $(3,2,0)$),
and the remaining sets are surprisingly close to multiples of 1/2.
The 5-loop gauge sets are the preliminary Volvkov\rf{Volkov18} results,
currently in disagreement with Aoyama \etal\rf{AoKiNi18}.
}
\end{figure}
%%%%%%%%%%%%%%%%%%%%%%%%%%%%%%%%%%%%%%%%%%%%%%%%%%%%%

%%%%%%%%%%%%%%%%%%%%%%%%%%%%%%%%%%%%%%%%%%


%%%%%%%%%%%%%%%%%%%%%%%%%%%%%%%%%%%%%%%%%%
\input{Laporta17figuragauShort}
%%%%%%%%%%%%%%%%%%%%%%%%%%%%%%%%%%%%%%%%%%

%\item[2017-04-28
In 1977 Laporta\rf{Laporta17} published the individual contribution
of the 891 4-loop vertex diagrams contributing to the electron $(g-2)$
(evaluated up to 1100 digits of precision). The vertex diagrams separate
in 25 gauge-invariant sets (\reffig{Laporta17figuragau}).
%classifying them according to the number of photon corrections on the
%same side of the main electron line and the insertions of electron loops (see
%Cvitanovi\'c\rf{Cvit77b} for more details on the 3-loop classification).
The numerical contribution of each set, truncated to 40 digits, is
listed in the \reftab{Laporta17:tableset}.
Adding only the diagrams without closed electron loops (see
\reffig{Laporta17figuragauShort} and \ref{tabGaugeSets}), one finds for
the quenched set (in Aoyama \etal\rf{AoHaKiNi12} nomenclature, the set
of diagrams with no lepton loops):
\bea
 \aql{8}&=& -2.176866027739540077443259355895893938670
\continue
        &=& -2.17\dots \,\qquad \mbox{ Aoyama \etal\ 2012\rf{AoHaKiNi12}}
\continue
        &\approx& 0 \,\qquad\qquad\quad\quad \mbox{Cvitanovi\'c 1977\rf{Cvit77b}}
\continue
 \aql{10}&=& 8.726(336)\dots \,\; \mbox{   Aoyama \etal\ 2015\rf{AoHaKiNi15}}
\continue
        &\approx& 3/2  \,\qquad\qquad\quad \mbox{Cvitanovi\'c 1977\rf{Cvit77b}}
\,.
\label{anomalValues}
\eea
%Closed electron loops only:
%\begin{align}
% \aql{4}&=  \phantom{+}0.264620262813094503290612188456063884609 \,.
%\end{align}
While my 1977 prediction $\aql{8} \approx 0$ (instead of the correct
$\aql{8} \approx -2$) does not pan out, the difference is small,
considering that this is a sum of 518 vertex diagrams (or 47 self-energy
diagrams)\rf{KinLin90}.
Likewise, my prediction for $\aql{10}$ is not too far off, considering
this is a sum of 6354 vertex diagrams of \reftab{Cvit77bTab1} (or 389
self-energy diagrams).


\subsection{Self-energy sets}
\label{sect:selfEnergy}

There are two ways of grouping vertex diagrams, into \emph{gauge sets},
and into \emph{self-energy sets} (or the ``externally gauge-invariant''
sets). Every vertex diagram belongs both to a gauge set and to a
self-energy set, as illustrated by \reffig{Cvit77bFig2}.
Formulation of the $(g-2)$ computation directly from self-energy graphs
is due to
\HREF{http://chaosbook.org/~predrag/papers/preprints.html\#PerturbativeQED}
{Cvitanovi\'{c} and Kinoshita}, see the ``new formula'' (6.22) in
\refref{CviKin74c}, based on the Ward-Takahashi identity. Not only does
the calculation use fewer Feynman graphs, but it was very important for
us, as it enabled us to calculate the 3-loop electron magnetic moment by
two independent methods (and in this way we did actually identify and
eliminate a subtle numerical error in one of the graphs).
Parenthetically, Carroll papers\rf{CarYao74,Carroll75} puzzle me. He
gives the credit to the mass-operator formalism of
Schwinger\rf{Schwinger51a,Schwinger51b,Sommerfield58}, but no credit to
us\rf{CviKin74c}, even though his papers look closer to ours than to
Schwinger and Sommerfield; he cites us, he also uses the Ward-Takahashi
identity. Our formulation might be equivalent to Schwinger's, but it
looks quite different in detail, and I was not aware of Schwinger
mass-operator when we derived it.

The gauge sets are minimal, and separately gauge invariant (for a proof,
see \refref{Cvit77b}). The self-energy sets are not, only their sum is
gauge invariant.
Unlike gauge sets, whose number grows polynomially, the number of
self-energy sets grows combinatorially - they save significant amount
of computing for few-loops computations, but cannot be used to argue the
finiteness of QED.
That is the reason why Aoyama \etal\rf{AoHaKiNi12,AoHaKiNi15}
calculations have nothing to say about my 1977 paper\rf{Cvit77b}: they do
not compute individual vertex diagrams, but only the self-energy sets,
and for them the set of all diagrams without a fermion loop (`quenched-'
or `q-type' diagrams) is a single `gauge-invariant set'  $V$. For
example, for 5-loops the set $V$ is a sum 9 vertex gauge sets (where
time-reversed pairs count as one set, see \reffig{tabGaugeSets}), but
Aoyama \etal\rf{AoHaKiNi15} only give their sum \refeq{anomalValues}.

\section{Where do we go from here?}
\label{sect:future}

Aoyama \etal\ 5-loop calculations already push the envelope of what is
numerically attainable, they would be loath to switch from self-energy
diagrams formulation to the vertex diagrams formulation, it would mean
(for the quenched set $V$) going from 389 self-energy graphs to 6354
vertex diagrams. Stefano Laporta deserves a bit of well earned rest. So
what is ahead?

%\item[2017-05-25 Predrag] to Stefano

I've now reread much of the relevant literature known to me. There might
be much more - people who do things related to my 1977 paper\rf{Cvit77b}
never alert me to their papers, presumably because the reports of my
death have been greatly exaggerated. Sergey  A. Volkov appears to be the
only person set up do the requisite 5-loop calculations.

There seem to be 2 approaches that might be relevant to
establishing bounds on, and perhaps even the direct computation
of gauge sets (ignoring the $N\!=\!2$ and $N\!=\!4$
supersymmetric models):
(1) \emph{Hopf algebraic approach} of Kreimer and collaborators,
and
(2)
\emph{worldline formalism} pursued by
\HREF{http://www.ifm.umich.mx/ifm/index.php/fisca/academicos/schubert/}
{Schubert} and collaborators.

\subsection{Volkov method}
\label{sect:Volkov}

In \refref{Volkov16} Volkov explains that $A_1^{(2n)}$ is free from
infrared divergences since they are removed by the on-shell
renormalization.
However, Volkov also states that there is no universal
method in QED for canceling IR divergences in the
Feynman graphs analogous to the R operation, and
that the standard subtractive on-shell renormalization cannot remove
IR divergences point-by-point in Feynman-parametric space, as it
does for UV divergences. Moreover, it can generate additional
IR-divergences.

That QED on-mass shell amplitudes are IR-free must be an old result; even
I have several papers generalizing that to
QCD\rf{MassShell,IRstruct,QCDmshell,NPB81}. Tom Kinoshita and I solved
the problem of point-by-point removal of IR divergences in
Feynman-parametric space in my thesis\rf{CviKin74b}. I have a bright
memory of figuring out how to do it one quiet evening in Ithaca,
babysitting for a friend's toddler. But
our approach was apparently not general enough to deal with 4-loop and
higher order corrections.

Volkov's algorithm is developed in {\em New method of computing the
contributions of graphs without lepton loops to the electron anomalous
magnetic moment in {QED}}\rf{Volkov17}. It is based on the ideas used for
proving UV-finiteness of renormalized Feynman
amplitudes\rf{Speer68,AnZaPo73}. He focuses on $n$-loop graphs with no
lepton loops, or, in Aoyama \etal\rf{AoHaKiNi12} nomenclature,
$\aql{2n}$.

While Volkov organizes Feynman graphs by self-energy graph families, in
contrast to \refrefs{CviKin74c,CarYao74,Carroll75,AoHaKiNi15} he does
not evaluate these self-energy graphs directly;
all his calculations are performed with vertex graphs. Individual  vertex
graphs are precisely what is  necessary for evaluating gauge sets\rf{Cvit77b}
(discussed in \refsect{sect:finitness} above).

So far Volkov  has evaluated the ladder graph
%(Figure \ref{fig_ladder})
and the fully crossed graph
%(Figure \ref{fig_crosses})
up to 5 loops. The cross graphs are of interest because they do not
contain divergent subgraphs, so their contributions only depend on the
gauge, but not on the choice of subtraction procedure.

While the contributions of individual vertex graphs (and self-energy
sets\rf{AoHaKiNi15}) are all over the place, all gauge-invariant sets are
insanely small up to order 8, and it would be very sweet to see that this
continues through order 10 (at least for the 5-loop graphs with no
electron loops). My hunch is that starting with the gauge set
(6) = $(4,0,0)$
in \reffig{Laporta17figuragauShort} would be the most
rewarding, though Stefano Laporta thinks it too hard, and suggests
starting with a 5-loop relative
$(1,3,1)$ (or $(1,2,2)$?) of the set (3) = $(1,2,1)$ instead.
A numerical check of the QED finiteness conjecture would require the
gauge sets evaluated only to two significant digits or so, no high
accuracy is needed.


\subsection{Hopf algebraic approach}
\label{sect:HopfAlgebra}

Hopf algebraic approach of Kreimer and collaborators\rf{BrDeKr96,
Kreimer00, KreYea08, KisKre16} is very appealing - it is just that I
personally have no clue how to turn it into a direct $(g-2)$ gauge set
calculation. In the 2008 paper\rf{KreYea08} Dirk Kreimer and
\HREF{https://arxiv.org/find/math-ph,math/1/au:+Yeats_K/0/1/0/all/0/1}
{Karen Yeats} write: ``
One case where there is a natural interpretation is QED with a linear number of
generators, namely
\beq
X_1 = 1 + \sum_{k \geq 1}p(k)x^k
      \frac{X_1^{2k+1}}
           {(1-X_2)^{2k}(1-X_3)^{2k}}
\,,
\ee{KreYea08p413}
with $X_2$ and $X_3$ as before and with $p(k)$ linear, which corresponds
to counting with Cvitanovi\'c's gauge invariant sectors\rf{Cvit77b}.
''
%{\bf 2017-05-31 Predrag}
Even in this simple case I do not see how this counts the gauge sets. My
generating function for $G_{2n}$, the number of gauge sets
(eq.~(7) in \refref{Cvit77b})  is
\beq
\sum_{n=1}^\infty G_{2n}
    =
      \frac{X}
           {(1+X)(1-X)^{3}}
\,.
\ee{Cvit77b(7)}



\subsection{Worldline formalism}
\label{sect:worldline}

How and why Feynman in 1950 introduced `worldline formalism' (initially
for scalar QED, appendix to \refref{Feynman50}, then for spinor QED,
appendix to \refref{Feynman51}) is explained in Schubert\rf{Schubert01}
(which also has an extensive bibliography up to 2001).
In 1982 Affleck, Alvarez, and Manton\rf{AffAlMa82} used the Feynman
worldline path integral representation of the quenched effective action
for scalar QED in the constant electric field, and, independently,
Lebedev and Ritus\rf{LebRit84} did the same for the spinor QED in 1984.

A formula for the charged scalar propagator to emit and reabsorb $N$
photons along the way of its propagation from $x'$ to $x$ can be derived
as follows\rf{AhBaSc16}.
The free scalar propagator for the Euclidean
Klein-Gordon equation\rf{Schubert12,AhBaSc16} is
\beq
D_0(x,x')=\bra{x}\frac{1}{-\Box +m^2}\ket{x'}
\,,
\ee{Schubert12(1.1)}
where $\Box$ is the $D$-dimensional Laplacian. Exponentiate the
denominator following Schwinger,
% proper-time parameter $T$
\beq
D_0(x,x')=\int_0^\infty\!\!dT\,{e}^{-m^2T}\bra{x}e^{-T(-\Box)}\ket{x'}
\,,
\ee{Schubert12(1.4)}
Replace the operator in the exponent by a path integral
\beq
D_0(x,x')=\int_0^\infty\!\!dT\,e^{-m^2T}
\int_{x(0)=x'}^{x(T)=x}\!\!\!\!\mathcal{D}x(\tau)\,
    e^{-\int_0^T\!\!d\tau \frac{1}{4}\dot{x}^2}
\,,
\ee{Schubert12(1.7)}
where $\tau$ is a proper-time parameter. This is the worldline path
integral representation of the relativistic propagator of a scalar
particle in Euclidean spacetime. It\, is easily evaluated and leads to
the usual space and momentum space free propagators. Adding the QED
interaction terms leads to the Feynman's path integral
representation\rf{Feynman50} of the charged scalar propagator  of mass
$m$ in the presence of a background field $A(x)$,
\beq
D(x,x')=\int_0^\infty\!\!dT\,e^{-m^2T}
    \int_{x(0)=x'}^{x(T)=x}\!\!\mathcal{D}x(\tau)\,
            {e}^{-S_0-S_e-S_i}
\,,
\ee{AhBaSc16(1)}
where (0) is the free propagation
\beq
S_0 = \int_0^T\!\!d\tau \frac{1}{4}\dot{x}^2
\,,
\ee{Ahmadiniaz1}
(e) is the interaction of the charged scalar with the external field
\beq
S_e = -ie\int_0^T\!\!d\tau\,\dot{x}^\mu A_\mu(x(\tau))
\,,
\ee{Ahmadiniaz2}
and (i) are the virtual photons exchanged along the charged particle's
trajectory
\beq
S_i = \frac{e^2}{2}\int_0^T\!\!d\tau_1\int_0^T\!\!d\tau_2\,
      \dot{x}_1^\mu\,D_{\mu\nu}(x_1-x_2)\,\dot{x}_2^\nu
\,,
\ee{Ahmadiniaz3}
where $D_{\mu\nu} $ is the $x$-space photon propagator.
% In $D$ dimensions and arbitrary covariant gauge

Consider first the charged scalar field in external field, neglecting
internal photon loops. By taking the constant external field $A(x)$ to be
a sum of $N$ plane waves, one obtains the rule for inserting $N$ external
photons:
\bea
D_{(N)}(x,x')
%\langle 0|T \phi (x) \phi (y) |0\rangle_{(N)}
&=& (-\lambda)^N
 \int_0^\infty \! dT \,e^{-m^2 T}
 \int_0^Td\tau_1 \cdots \int_0^Td\tau_N
 \nonumber\\ &&
\times  \int_{_{x(0)=y}}^{^{\, x(T)=x}}
\!\!\!\!\!\!\!\!\!\!\!\! {\cal D}x
\,e^{i\sum_{i=1}^Nk_i\cdot x(\tau_i)}
e^{ -\int_0^Td\tau\, {1\over 4} \dot x^2}
\,.
\label{Nprop}
\eea
For the spinor case, the magnetic moment will be given by the term linear
in a constant external field $A(x)$, and in order to define gauge sets,
one will have to distinguish the in- and out-electron lines.

The object of great interest to us is the quenched internal virtual
photons term \refeq{Ahmadiniaz3}:
\beq
\int_{x(0)=x'}^{x(T)=x}\!\!\mathcal{D}x(\tau)\,
            {e}^{-S_i}
=
\int_{x(0)=x'}^{x(T)=x}\!\!\mathcal{D}x(\tau)\,
            {e}^{-\frac{e^2}{2}\int_0^T\!\!d\tau_1\int_0^T\!\!d\tau_2\,
      \dot{x}_1^\mu\,D_{\mu\nu}(x_1-x_2)\,\dot{x}_2^\nu}
\,.
\ee{AhBaSc16(1)i}
Expanded perturbatively in $\alpha/\pi$, this yields the usual
Feynman-parametric vertex diagrams. However, it is Gaussian in
$\dot{x}^\mu$, and if by integration by parts, $\dot{x}^\mu$ are
eliminated in favor of $x^\mu$, internal photons can be integrated over
directly, prior to an expansion in $(\alpha/\pi)^n$, and one gets
integrals in terms of \emph{$N$-photon propagators}, and not the usual vertex
graphs. Each usual vertex graph corresponds to a permutation of internal
photon insertions, and from that
comes the factorial growth in the number of graphs.

These integrations by parts lead to the first and second proper-time
derivatives of the Green's function, worked out in the literature (for
example, in \refrefs{Strassler92,Schubert12}), the details would take too
much space to recap here. My notes on these papers are below,
around \refpage{sect:SchSch96}. I find Bastianelli, Huet, Schubert, Thakur and
Weber 2014 paper\rf{BHSTW14} (the {2017-05-24} blog entry below) quite
inspirational. Apologies, my notes are just a jumble, jottings taken as I
try to understand this literature.

%%%%%%%%%%%%%%%%%%%%%%%%%%%%%%%%%%%%%%%%%%%%%%%%%
\begin{figure}[h]
\includegraphics[width=1\textwidth]{BHSTW143loopphotonprop}
 \caption{
 Quenched diagrams contributing to the three loop QED photon propagator.
 From \refref{BHSTW14}.
 }
 \label{BHSTW143loopphotonprop}
\end{figure}
%%%%%%%%%%%%%%%%%%%%%%%%%%%%%%%%%%%%%%%%%%%%%%%%%

Thus, for the quenched scalar QED, the worldline integrals are expressed
in terms of $N$-photon propagators, the central ingredient that defines
the quenched gauge sets \refeq{quenchAnom}.
Unlike the Feynman parameter integrals for individual vertex graphs, they are
independent of the ordering of the momenta $k_1,\ldots,k_N$; the formula
\refeq{AhBaSc16(1)i} contains all $\approx N!$ ways of attaching the $N$
photons to the charged particle propagator.
The formulation combines combinatorially many Feynman diagrams into a single integral.
An example are the quenched contributions to the
three-loop photon propagator shown in \reffig{BHSTW143loopphotonprop}.

In QED the $N$-photon propagator formulation combines into one integral
all Feynman graphs related by permutations of photon legs along fermion
lines, that is, it should yield \emph{one} integral for a gauge set
$(k,m,m')$ defined in \refeq{quenchAnom}.
%, provided one may distinguish the leg-photons from the cross-photons.

\subsection{High-orders QED in worldline formalism}
\label{sect:highQEDworldline}

A non-perturbative formula for QED in a constant field, given for scalar
QED in 1982 by Affleck, Alvarez, and Manton\rf{AffAlMa82}, and for spinor
QED by Lebedev and Ritus\rf{LebRit84} in 1984, is an example how the
worldline formalism can yield high-order information on QED amplitudes.
Huet, McKeon, and Schubert\rf{HuMcSc10} continue this in their 2010
%{\em {Euler-Heisenberg} lagrangians and asymptotic analysis in 1+1 {QED.
% Part I: Two}-loop}
%(no GaTech online access, \arXiv{1010.5315}):``
study of the l-electron loop, $N$-photon amplitudes in the limit of large
photon numbers and low photon energies, this time for 1+1 dimensional
scalar QED, in order to illustrate the large cancellations inside gauge
invariant classes of graphs.

Affleck \etal\rf{AffAlMa82} use the Feynman\rf{Feynman50} `worldline
path integral' representation of the quenched effective action for scalar
QED in the constant electric field, and calculate the amplitude in a
stationary path approximation. The stationary trajectory so obtained is a
circle with a field dependent radius, called ``instanton''  in this
context. The worldline action on this trajectory yields the correct
exponent, and the second variation determinant yields the correct  prefactor.
Using Borel analysis, they obtain non-perturbative information on the
on-shell renormalized $N$-photon amplitudes at large $N$ and low energies.

For the quenched spinor QED (fermion lines decorated by photon exchanges)
closed-form expressions for general $N$ require the worldline
super-\,formalism\rf{Schubert01}, at the cost of introducing Fradkin
1966\rf{Fradkin66} Grassmann path integral.

Dunne and Schubert\rf{DunSch06} study $N$-photon amplitudes, in scalar
and spinor QED, in the quenched approximation, \ie, taking only the
diagrams with one electron loop and are led to ``the following
generalization of Cvitanovi\'c's conjecture: the perturbation series
converges for all on-shell renormalized QED amplitudes at leading order
in $N_f$. It must be emphasized that the on-shell renormalization is
essential in all of the above.''
Unlike Cvitanovi\'c\rf{Cvit77b} purely numerical conjecture, theirs is a
sophisticated argument, buttressed by Borel dispersion relations.


\section{Summary}
\label{sect:Summary}

Currently Sergey  A. Volkov is in the best position to check the QED
finiteness conjecture numerically, by computing the 5-loops gauge sets.

Worldline formalism could be useful on a qualitative level, as a way of
proving the finiteness of QED conjecture,
    \begin{enumerate}
  \item
Develop a saddle point expansion for the $N$-photon propagator
integrals, such that the
leading term explains the apparent $\approx \pm 1/2$ (or a multiple
thereof) size of each quenched gauge set. Affleck \etal\rf{AffAlMa82}
show they way.
  \item
Use that to establish bounds on gauge sets for large orders, prove
finiteness of quenched QED. If that works, I trust electron loop
insertions will be next, and thereafter renormalons\rf{Lautrup77}, \etc,
will go \HREF{https://soundcloud.com/poets-org/notgogentle-mp3-5/s-2o7zI}
{gently into that good night}.
    \end{enumerate}
and in a precise way, as a new computational tool:
    \begin{enumerate}
  \item
Develop a new worldline formulation of QED in which each gauge set is
given by a computable integral. Parenthetically, a reformulation of the
self-energy diagrams magnetic moment calculation would be an even greater
saving - all quenched diagrams contributions calculated at one go. A new
worldline formulation might make it possible to evaluate orders beyond
5-loops, as the number of gauge sets grows only polynomially. A win-win.
  \item
A renormalization question:
A gauge-invariant set is by definition UV and IR finite. Does that mean
that its worldline formalism integrand is pointwise finite, no need for
counterterms?
    \end{enumerate}
My main problem at the moment (well, there are many:) is that nobody
seems to have written an explicit formula for the spinor QED anomalous
magnetic moment in the worldline formalism.
