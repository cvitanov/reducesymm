% reducesymm/QFT/finiteQED.tex
% $Author$ $Date$
% Predrag  switched to github.com               jul  8 2013

\begin{bartlett}{
Is there any method of computing the anomalous moment of the
electron which, on first approximation, gives a fair approximation to the
$\alpha$ term and a crude one to $\alpha^2$; and when improved, increases
the accuracy of the $\alpha^2$ term, yielding a rough estimate to
$\alpha^3$ and beyond?
        }
\bauthor{
Feynman's challenge, 12th Solvay Conference\rf{Feynman62}
    }
\end{bartlett}



\begin{description}

\item[2017-05-21 Predrag] to
\\
Sergey  A. Volkov <volkoff\_sergey@mail.ru>
\\
Skobeltsyn Institute of Nuclear Physics,
\\
Moscow State University, Moscow, 119234 Russia

Dear Sergey

I have just read your
{\em New method of computing the contributions of graphs without lepton
loops to the electron anomalous magnetic moment in {QED}}\rf{Volkov17}
(and the earlier \refref{Volkov16}) with great interest, and I can see
that you are set to compute the 5-loop correction to the electron
$(g-2)$. This is a very hard calculation, and approaching it
strategically is a necessity.

May I suggests that you order your calculation by gauge sets,
as listed (for the 4-loop case) in Stefano Laporta's
\reftab{Laporta17:tablesetShort} and illustrated in
\reffig{Laporta17figuragau}? My hunch is that the gauge set (6) = $(4,0,0)$ would be
the most interesting, though Stefano thinks it too hard, and suggests
starting with a 5-loop relative $(1,3,1)$ (or $(1,2,2)$?)
of the set (3) = $(1,2,1)$ instead. While the
contributions of individual vertex graphs (and self-energy sets\rf{AoHaKiNi15})
are all over the place, all gauge-invariant sets are insanely small up to
order 8, and it would be very sweet to see that this continues through
order 10 (at least for the 5-loop graphs with no electron loops).

By the way, to check my conjecture one needs the gauge sets only to two
significant digits or so, no high accuracy is needed.

best regards
\\
Predrag


\end{description}

\newpage

Dunne and Schubert 2005 paper\rf{DunSch06}
%{\em Multiloop information from the {QED} effective lagrangian}
starts with a nice historical review of ideas about the the perturbation
series in QED. They then note that,
``Despite of the many insights which have been gained along these lines,
a point which remains poorly understood is the influence of gauge
cancellations on the divergence structure of a gauge theory.''

\section{Gauge sets}
\label{sect:finitness}

%%%%%%%%%%%%%%%%%%%%%%%%%%%%%%%%%%%%%%%%
\begin{figure}
\begin{center}
\includegraphics[width=1.00\textwidth]{JegNif09fig10}
\end{center}
\caption{\label{JegNif09fig10}
The third order contributions $A^{(6)}_1$ to $(g-2)$.
From Jegerlehner and Nyffeler\rf{JegNif09}.
Lautrup \etal\rf{LaPeRa72} where the first to note that
subsets $(k,m,m')$ =
$(3,0,0)$ = $\{23,24,25,26,28\}$;
$(2,1,0)$ = $\{29,31,33,35,37,39,41,43,45,47\}$ and its time-reversal
$(2,0,1)$ = $\{30,32,34,36,38,40,42,44,46,48\}$;
$(1,2,0)$ = $\{49,51,53,55,57,59,61,63,65,67\}$ and its time-reversal
$(1,0,2)$ = $\{50,52,54,56,58,60,62,64,66,68\}$;
and
$(1,1,1)$ = $\{69,70,71,72\}$
are the minimal gauge sets.
}
 \end{figure}
%%%%%%%%%%%%%%%%%%%%%%%%%%%%%%%%%%%%%%%%

%%%%%%%%%%%%%%%%%%%%%%%%%%%%%%%%%%%%%%%%%%
\begin{table}
\begin{center}
\includegraphics[width=0.80\textwidth]{Cvit77bTab1}
\end{center}
\caption{\label{Cvit77bTab1}
Comparison of the number of vertex diagrams without fermion loops, gauge
sets and the ``gauge-set approximation'' for the magnetic moment in
$2n$th order.
From \refref{Cvit77b}.
}
\end{table}
%%%%%%%%%%%%%%%%%%%%%%%%%%%%%%%%%%%%%%%%%%%%%%%%%%%%%%%%%%%

% finiteQEDins.tex
% PC created from lectures/talks/DFS_pris.tex       2017-05-25
% PC with edits March 2003
% PC with edits July 1997
%Acceptance speech - 1993 NKT Research Prize in Physics\\
%Dansk Fysisk Selskab \AA rsm\o de, Lalandia, R\o dby, 18 maj 1993
In 1972 Toichiro Kinoshita and I had completed computing thousands of
3-loop $(g-2)$ diagrams and counterterms\rf{CviKin72},
\reffig{JegNif09fig10}
(the subsequent 4- and 5- loop numerical and analytic calculations
are nothing short of heroic\rf{KinLin90,Laporta17,AoHaKiNi15}).
We had used the quantum field theory in the standard way, by expanding
the magnetic moment into combinatorially many Feynman diagrams (see the
numbers of vertex graphs in \reftab{Cvit77bTab1}). Each Feynman diagrams
corresponds to an integral in many dimensions with integrand with
thousands of terms, each integral UV divergent, IR divergent, and
unphysical, as its value depends on the definition of counterterms and
the choice of gauge. Each integral, with its dreadfully oscillatory
integrand,  is evaluated by Monte Carlo methods in 10-20 dimensions with
no hint of what the answer should be; in our case $\pm 10$ to $\pm 100$
was a typical range.

We added up hundreds of such apparently random contributions and obtained
(for the no-fermion loops subset $V$, in the notation of
\refref{AoHaKiNi15})
\[
 \aql{6}\,=\, +  (0.92 \pm 0.02) \left(\frac{\alpha}{\pi}\right)^3.
\]
But why ``+'' and not ``-''? Why so small? Why does a sum of hundreds of
diagrams / counterterms yield a number of order of unity, and not 10 or
100 or any other number?

%%%%%%%%%%%%%%%%%%%%%%%%%%%%%%%%%%%%%%%%
\begin{figure}
\begin{center}
\includegraphics[width=0.50\textwidth]{Cvit77bFig1}
\end{center}
\caption{\label{Cvit77bFig1}
Rows: the fourth-order gauge sets
$(k,m,m')$: (1) = $(1,1,0)$,
(2) = $(2,0,0)$
and
(3) = $(1,0,1)$.
Columns: the two self-energy sets.
For diagrams related by time
reversal (here (1) and (3))
the value listed under the first diagram of the pair is
the total contribution of the pair. Contributions seem to be of order
$\pm\frac{1}{3}\left(\frac{\alpha}{\pi}\right)^2$.
From \refref{Cvit77b}.
}
 \end{figure}
%%%%%%%%%%%%%%%%%%%%%%%%%%%%%%%%%%%%%%%%

If gauge invariance of QED guarantees that all UV and IR divergences
cancel, could it be that it also enforces cancellations among the finite
parts?

As first noted by Lautrup, Peterman and de Rafael\rf{LaPeRa72}, the
renormalized on-mass shell QED vertex diagrams separate into a sum of
minimal gauge-invariant subsets, each subset separately UV and IR finite.
To simplify matters, in what follows we shall consider only the
no-fermion loop diagrams, or `q-type' or `quenched-type' diagrams
(`quenched', as this corresponds to the $O(N_f)$ part of the amplitude in
QED with $N_f$ flavors).
The minimal
gauge-invariant subsets without electron loops (for example,
\reffig{JegNif09fig10} diagrams $\{23-72\}$ and \reffig{Cvit77bFig2};
\ref{Cvit77bFig1}; \ref{Cvit77bFig3}; and \ref{Laporta17figuragauShort})
will be hereafter be referred to as \emph{gauge sets}. A gauge set
$(k,m,m')$ consists of all proper, 1-particle irreducible vertex diagrams
without electron loops with $k$ photons crossing the external vertex
(cross-photons) and $m [m']$ photons originating and terminating on the
incoming [outgoing] electron leg (leg-photons), where $m\geq m'$. For
asymmetric pairs of sets, with $m\neq m'$, the contribution to the
anomaly $a_{kmm'}$ is, in my definition, the sum of the sets and its
mirror (time-reversed) image,
\beq
a^{(V)}=\frac{1}{2}(g-2)^{(V)}
       =  \sum_{k=1}^\infty\sum_{m=0}^\infty\sum_{m'=0}^m
          a_{kmm'}\left(\frac{\alpha}{\pi}\right)^{k+m+m'}
\,.
\ee{quenchAnom}

When the diagrams that we had computed\rf{CviKin74c} are grouped into
gauge sets, \reffig{Cvit77bFig1} to \reffig{Cvit77bFig3},
a surprising thing happens; while the
finite part of each Feynman diagram is of order of 10 to 100, every
known gauge set adds up to approximately
$$
		   \pm {1 \over 2} \left(\frac{\alpha}{\pi}\right)^n
\,,
$$
with the sign given by a simple empirical rule
\beq
a_{kmm'} = (-1)^{m+m'}\frac{1}{2}
\ee{Cvit77b(5)}
The sign rule is further corroborated by sets with photon
self-energy insertions (but with the absolute size scaled down to
$3-15\%$ of \refeq{Cvit77b(5)}).
In \reffig{Cvit77bFig3} I compared this rule with the actual numbers and
made my 1977 four-loop prediction\rf{Cvit77b}.

The ``zeroth'' order
estimate of the electron magnetic moment anomaly $a$ is now given by
the ``gauge set approximation,'' convergent and summable to all orders
\beq
a=\frac{1}{2}(g-2) =  \frac{1}{2} \frac{\alpha}{\pi}
                     \frac{1}
           {\left( 1 - \left(\frac{\alpha}{\pi}\right)^2
			\right)^2
		      } + \mbox{``corrections"}
\,.
\ee{Cvit77b(1)}
This is not how one usually thinks of perturbation theory.
Most of our colleagues believe that in 1952
Dyson\rf{Dyson52} has shown that the perturbation expansion is an
asymptotic series, in the sense that the $n$-th order contribution should
be exploding combinatorially
$$
{1 \over 2} (g-2) \approx
\cdots + n^n \left(\frac{\alpha}{\pi}\right)^n + \cdots
\,,
$$
and not growing slowly like my estimate
\[
{1 \over 2} (g-2) \approx
\cdots + \frac{n}{2}\left(\frac{\alpha}{\pi}\right)^{2n} + \cdots
\,.
\]
For me, the above is the most intriguing hint that something deeper than
what we know underlies quantum field theory, and the most suggestive
lesson of our calculation.




%%%%%%%%%%%%%%%%%%%%%%%%%%%%%%%%%%%%%%%%
%\begin{figure}
%\begin{center}
%%%%%%%%%%%%%%%%%%%%%%%%%%%%%%%%%%%%%%%%%%%%%%%%%%%%%%%%
\begin{sidewaysfigure}[p]
\thisfloatpagestyle{empty} % Lucy Day Apr 18 2008
\center{
\includegraphics[width=0.45\textwidth,angle=-90]{Cvit77bFig2}
%\end{center}
        } %end \center{
\caption{\label{Cvit77bFig2}
Every vertex diagram belongs both to a `gauge set' and to a `self-energy set'.
This table illustrates the two kinds of sets.
The sixth-order gauge sets $(k,m,m')$ are arranged in the rows, and the
self-energy sets (or the `externally gauge-invariant' sets) in the
columns, labeled as in Fig.~3 of \refref{CviKin74c}. The values are
finite parts in the $\ln\lambda$ IR cut-off approach, such as those
listed in \refref{LevWri73}. For different IR separation methods (such as
in \refref{CviKin74c} and different gauges, individual diagrams have
different values. The gauge sets are separately gauge invariant; the
self-energy sets (whose number grows combinatorially with the order in
perturbation theory) are not, only their sum is gauge invariant. From
\refref{Cvit77b}.
}
 \end{sidewaysfigure}
%%%%%%%%%%%%%%%%%%%%%%%%%%%%%%%%%%%%%%%%

%%%%%%%%%%%%%%%%%%%%%%%%%%%%%%%%%%%%%%%%
\begin{figure}
\begin{center}
\includegraphics[width=0.90\textwidth]{Cvit77bFig3}
\end{center}
\caption{\label{Cvit77bFig3}
Comparison of the ``gauge-set approximation'' and the actual
numerical values of corresponding gauge sets, together with an
eighth-order prediction.
From \refref{Cvit77b}.
}
 \end{figure}
%%%%%%%%%%%%%%%%%%%%%%%%%%%%%%%%%%%%%%%%

%%%%%%%%%%%%%%%%%%%%%%%%%%%%%%%%%%%%%%%%%%
\input{Laporta17figuragauShort}
%%%%%%%%%%%%%%%%%%%%%%%%%%%%%%%%%%%%%%%%%%

%\item[2017-04-28
In 1977 Laporta\rf{Laporta17}
has published the individual contribution of the 891
4-loop vertex diagrams contributing to the electron $(g-2)$
(evaluated up to 1100 digits of precision).
The vertex diagrams separate in 25
gauge-invariant sets (\reffig{Laporta17figuragau}),
%classifying them according to the number of photon corrections on the
%same side of the main electron line and the insertions of electron loops (see
%Cvitanovi\'c\rf{Cvit77b} for more details on the 3-loop classification).
The numerical contribution of each set, truncated to 40 digits, is
listed in the \reftab{Laporta17:tableset}.
Adding only the diagrams without
closed electron loops (see \reffig{Laporta17figuragauShort} and
\reftab{Laporta17:tablesetShort}), one finds for the quenched set
(in Aoyama \etal\rf{AoHaKiNi12} nomenclature, the set $V$ diagrams with
no lepton loops):
\bea
 \aql{8}&=& -2.176866027739540077443259355895893938670
\continue
        &=& -2.17\dots \,\qquad \mbox{Aoyama \etal\rf{AoHaKiNi12} (2012)}
\continue
        &\approx& 0 \,\qquad\qquad\quad \mbox{Cvitanovi\'c\rf{Cvit77b} (1977)}
\continue
 \aql{10}&=& 8.726(336)\dots \,\qquad \mbox{Aoyama \etal\rf{AoHaKiNi15} (2015)}
\continue
        &\approx& 3/2  \,\qquad\qquad \mbox{Cvitanovi\'c\rf{Cvit77b} (1977)}
\,.
\eea
%Closed electron loops only:
%\begin{align}
% \aql{4}&=  \phantom{+}0.264620262813094503290612188456063884609 \,.
%\end{align}
While my 1977 predictions $\aql{8} \approx 0$ (instead of the correct
$\aql{8} \approx -2$) do no pan out, the difference is small,
considering that this is a sum of sum of 518 vertex diagrams
(or 47 self-energy diagrams)\rf{KinLin90}. Likewise,
my prediction for $\aql{10}$ is not too far off for a sum of 6354
independent vertex diagrams of \reftab{Cvit77bTab1} (belonging to 389
self-energy sets).

%%%%%%%%%%%%%%%%%%%%%%%%%%%%%%%%%%%%%%%%%%
\begin{table}
\small
\begin{center}
\begin{tabular}{rrrrr}
\hline
   1  & (1,3,0) & - 1.9710    & - 1/2  \\%01
   2  & (2,2,0) &  - 0.1424   & \phantom{+} 1/2 &~(!)\\%02
   3  & (1,2,1) &  - 0.6219   & - 1/2  \\%03
   4  & (2,1,1) &  \phantom{+} 1.0867  & \phantom{+} 1/2  \\%04
   5  & (3,1,0) &  - 1.0405   & - 1/2  \\%05
   6  & (4,0,0) &  \phantom{+} 0.5125  & \phantom{+} 1/2  \\%06
\hline
%  1+2+{\ldots}+6 & - 2.176866027739540077443259355895893938670  \\
% -2*1.9710-2*0.14249-2*0.6219+1.0867-2*1.0405+0.5125 =
% -1.9710-0.14249-0.6219+1.0867-1.0405+0.5125 =  -2.17669
\end{tabular}
\end{center}
\caption{
Contribution to $A^{(8)}_1$ of the 6 gauge sets of
\reffig{Laporta17figuragauShort}, as reported by Laporta\rf{Laporta17}
(for the full 25 gauge-invariant sets, see \reftab{Laporta17:tableset}).
The last column: 1977 Cvitanovi\'c predictions\rf{Cvit77b}
for the six no-electron loops gauge sets.
Signs are right, except for the set (2) = $(2,2,0)$, which is anomalously small,
and the remaining sets are surprisingly close to multiples of 1/2.
There might be factors of 2 having to do with symmetries, missing from the
guesses of \refref{Cvit77b}, but I cannot see how that would work.
Only (4) = $(2,1,1)$ and  (6) = $(4,0,0)$ are symmetric,
but (1) = $(1,3,0)$, (4) and (5) = $(3,1,0)$ seem to
have an extra factor of 2 or 4.
}
\label{Laporta17:tablesetShort}
\end{table}
%%%%%%%%%%%%%%%%%%%%%%%%%%%%%%%%%%%%%%%%%%%%%%%%%%%%%%%%%%%
%\newpage

\subsection{Self-energy sets}
\label{sect:selfEnergy}

There are two ways of grouping vertex diagrams, into `gauge sets' and
into `self-energy sets' (or the `externally gauge-invariant' sets). Every
vertex diagram belongs both to a `gauge set' and to a `self-energy set',
as illustrated by \reffig{Cvit77bFig2}.
Reformulation of the $(g-2)$ computation directly from self-energy graphs is due to
\HREF{http://chaosbook.org/~predrag/papers/preprints.html\#PerturbativeQED}
{Cvitanovi\'{c} and Kinoshita},
see the ``new formula'' (6.22) in \refref{CviKin74c}, based on the
Ward-Takahashi identity. Not only does the calculation use fewer Feynman
graphs, but it was very important for us, as it enabled us to calculate
the 3-loop electron magnetic moment by two wholly independent methods
(and in this way we did actually identify and eliminate a subtle
numerical error in one of the graphs).
Parenthetically, Carroll papers\rf{CarYao74,Carroll75} puzzle me. He
gives the credit to the mass-operator formalism of
Schwinger\rf{Schwinger51a,Schwinger51b,Sommerfield58}, but no credit to
us\rf{CviKin74c}, even though his papers look closer to ours than to
Schwinger and Sommerfield - he also uses the Ward-Takahashi identity. Our
formulation might be equivalent to Schwinger's, but it looks quite
different in detail, and I was not aware of Schwinger mass-operator when
we derived it.

The gauge sets are minimal, and separately gauge invariant (for a proof,
see \refref{Cvit77b}). The self-energy sets are not, only their sum is
gauge invariant.
Unlike gauge-sets, whose number grows polynomially, the number of
self-energy graphs grows combinatorially - they save significant amount
of computing for few-loops computations, but cannot be used to establish
finiteness of QED.
That is the reason why Aoyama \etal\rf{AoHaKiNi12,AoHaKiNi15}
calculations have nothing to
say about my 1977 paper\rf{Cvit77b}: they do not compute individual
vertex diagrams, but only the self-energy sets, and for them the set of
all diagrams without a fermion loop (`q-type', or `quenched-type') is their single
`gauge-invariant set'  $V$,  the sum of the vertex
gauge sets (sketched in \reffig{Laporta17figuragauShort}). For
example, for 4-loops the set $V$ is a sum 9 vertex gauge
sets (where time-reversed pairs count as one set).

\section{Where do we go from here?}
\label{sect:future}

Aoyama \etal\ 5-loop calculations push the envelope of what is
numerically attainable - they will not switch from self-energy diagrams
formulation to the traditional vertex graphs formulation, it would mean
(for the quenched set $V$) going from 389 self-energy graphs to 6354
vertex diagrams. Stefano Laporta deserves a bit of well earned rest.
What is ahead?

%\item[2017-05-25 Predrag] to Stefano

I've now searched through the literature, at least the literature
familiar to me (people who do things related to my 1977
paper\rf{Cvit77b} never alert me to their papers, presumably
because the news of my death are somewhat exaggerated).
Sergey  A. Volkov is the only person set up do the requisite 5-loop
calculations.

There seem to be 2 approaches that might be relevant to
establishing bounds on, and actual computation
of gauge sets (I am ignoring various $N=2$ and $N=4$
supersymmetric models):
(1) \emph{Hopf algebraic approach} of Kreimer and collaborators,
and
(2)
\emph{worldline formalism} pursued by Schubert and collaborators.

\subsection{Volkov method}
\label{sect:Volkov}

%    \PC{2017-05-20
That QED on-mass shell amplitudes are IR-free must be an old result; even
I have several papers generalizing that to
QCD\rf{MassShell,IRstruct,QCDmshell,NPB81}. I and Tom Kinoshita solved
the problem of point-by-point removal of IR divergences in
Feynman-parametric space in my thesis\rf{CviKin74b}. I have a bright
memory of figuring out how to do it one quiet evening in Ithaca,
babysitting for a friend's toddler. But
our approach was apparently not general enough to deal with 4-loop and
higher order corrections.
In \refref{Volkov16} Volkov explains that $A_1^{(2n)}$ is free from
infrared divergences since they are removed by the on-shell
renormalization as well as the ultraviolet ones.
However, Volkov also states that there is no universal
method in QED for canceling IR divergences in the
Feynman graphs analogous to the R operation, and
that the standard subtractive on-shell renormalization cannot remove
IR divergences point-by-point in Feynman-parametric space as well as it
does for UV divergences. Moreover, it can generate additional
IR-divergences.
% \item[2017-05-19 Volkov]
Volkov's algorithm is developed in {\em New method of computing the
contributions of graphs without lepton loops to the electron anomalous
magnetic moment in {QED}}\rf{Volkov17}. It is based on the ideas used for
proving UV-finiteness of renormalized Feynman
amplitudes\rf{Speer68,AnZaPo73}.
He focuses on $n$-loop graphs with no lepton loops, or, in Aoyama
\etal\rf{AoHaKiNi12} nomenclature,
$\aql{2n}$.

While Volkov organizes Feynman graphs by self-energy graph families, in
contrast to \refrefs{CviKin74c,CarYao74,Carroll75,AoHaKiNi15} he does
not evaluate these self-energy graphs directly;
all his calculations are performed with vertex graphs. Individual  vertex
graphs are precisely what is  necessary for evaluating gauge sets\rf{Cvit77b}.

So far Volkov  has evaluated the ladder graph
%(Figure \ref{fig_ladder})
and the fully crossed graph
%(Figure \ref{fig_crosses})
up to 5 loops. The cross graphs are of interest because they do not
contain divergent subgraphs, so their contributions only depend on the
gauge, but not on the choice of subtraction procedure.

\subsection{Hopf algebraic approach}
\label{sect:HopfAlgebra}

Hopf algebraic approach of Kreimer and collaborators\rf{BrDeKr96,
Kreimer00, KreYea08, KisKre16} is very appealing - it is just that I
personally have no clue how to turn it into a $(g-2)$ gauge-set
calculation. In the 2008 paper\rf{KreYea08} Dirk Kreimer and
\HREF{https://arxiv.org/find/math-ph,math/1/au:+Yeats_K/0/1/0/all/0/1}
{Karen Yeats} write: ``
One case where there is a natural interpretation is QED with a linear number of
generators, namely
\beq
X_1 = 1 + \sum_{k \geq 1}p(k)x^k
      \frac{X_1^{2k+1}}
           {(1-X_2)^{2k}(1-X_3)^{2k}}
\,,
\ee{KreYea08p413}
with $X_2$ and $X_3$ as before and with $p(k)$ linear, which corresponds
to counting with Cvitanovi\'c's gauge invariant sectors\rf{Cvit77b}.
''
%{\bf 2017-05-31 Predrag}
but I do not see how this counts the gauge sets. My
generating function for $G_{2n}$, the number of gauge sets
(eq.~(7) in \refref{Cvit77b})  is
\beq
\sum_{n=1}^\infty G_{2n}
    =
      \frac{X}
           {(1+X)(1-X)^{3}}
\,.
\ee{Cvit77b(7)}



\subsection{Worldline formalism}
\label{sect:worldline}

How and why Feynman\rf{Feynman50} 1950 introduced `worldline formalism'
is nicely explained in Schubert\rf{Schubert01}, which also has an
extensive bibliography up to 2001.
In 1982 Affleck, Alvarez, and Manton\rf{AffAlMa82} used the
Feynman\rf{Feynman50} worldline path integral representation of the
quenched effective action in scalar QED in the constant electric field,
and, independently, Lebedev and Ritus\rf{LebRit84} in 1984 did the same
for the spinor QED. Using Borel analysis, they obtained non-perturbative
information on the on-shell renormalized $N$-photon amplitudes at large
$N$ and low energy. In the stationary path approximation, the stationary
trajectory is a loop, or what they call ``instanton''  in this context.
The worldline action on this trajectory yielded the correct exponent, and
the second variation determinant gave precisely the same prefactor.
Worldline formalism operates with $N$-photon propagators, the ingredient
that defines the quenched gauge sets for $(g-2)$.

\begin{figure}[h]
\includegraphics[width=1\textwidth]{BHSTW143loopphotonprop}
 \caption{Quenched diagrams contributing to the three loop QED photon propagator.
 From \refref{BHSTW14}.
 }
 \label{BHSTW143loopphotonprop}
\end{figure}


For the scalar field theory amplitudes the
 ``worldline integrals''
\bea
\langle 0|T \phi (x) \phi (y) |0\rangle_{(N)} &=& (-\lambda)^N
 \int_0^\infty \! dT \,{\rm e}^{-m^2 T}
 \int_0^Td\tau_1 \cdots \int_0^Td\tau_N
 \nonumber\\ &&
\times  \int_{_{x(0)=y}}^{^{\, x(T)=x}}
\!\!\!\!\!\!\!\!\!\!\!\! {\cal D}x
\,{\rm e}^{i\sum_{i=1}^Nk_i\cdot x(\tau_i)}
{\rm e}^{ -\int_0^Td\tau\, {1\over 4} \dot x^2} \ .
\label{Nprop}
\eea
operate with $N$-photon propagators, the ingredient that defines the
quenched gauge sets for $(g-2)$.
Their advantage over Feynman parameter integrals is that they are
valid independently of the ordering of the momenta $k_1,\ldots,k_N$; the
rhs of \refeq{Nprop} contains all $N!$ possibilities of attaching the $N$
momenta to the propagator. %, as shown in figure \ref{fig:Nprop}. The
formulation combines many Feynman diagrams into a single integral. In QED
it combines into one integral all Feynman graphs related by permutations
of photon legs along fermion lines, that is, one integral for a gauge set
$(k,m,m')$ defined in \refeq{quenchAnom}, if on can distinguish the
leg-photons from the cross-photons.
An example are the quenched contributions to the three-loop photon
propagator shown in \reffig{BHSTW143loopphotonprop}. My main problem
(well, there are many:) is that nobody seems to have written a formula
for QED $(g-2)$ in the worldline formalism.
Also, for the spinor QED case (fermion lines and photon exchanges) closed-form
expressions for general $N$ requive the worldline
super-\,formalism\rf{Schubert01}, at the cost of introducing
additional multiple Grassmann integrals.


%\item[2017-05-23 Predrag]
Dunne and Schubert\rf{DunSch06} study $N$-photon amplitudes, in scalar
and spinor QED, in the quenched approximation, \ie, taking only the
diagrams with one electron loop
and are led to ``the following generalization of Cvitanovi\'c's
conjecture: the perturbation series converges for all on-shell
renormalized QED amplitudes at leading order in $N_f$.
It must be emphasized that on-shell renormalization is essential in all
of the above.''
Unlike Cvitanovi\'c\rf{Cvit77b} purely numerical conjecture, theirs is a
much more sophisticated argument, buttressed by Borel dispersion
relations.


 I find
Bastianelli, Huet, Schubert, Thakur and Weber 2014 paper\rf{BHSTW14} (the
{2017-05-24} blog entry below) the most inspirational.
My notes on these papers are below (\refpage{sect:SchSch96}).
Apologies, they are just a jumble, jottings taken as I try to understand
this literature.


Worldline formalism could be useful in a crude way, as a way of proving
my finiteness of QED conjecture, and in a precise way, as a new
computational tool:
    \begin{enumerate}
  \item
Develop a saddle point expansion for $N$-photon propagator such that the
leading term explains the apparent $\approx \pm 1/2$ (or a multiple
thereof) size of each quenched gauge-set.
  \item
Use that to establish bounds on gauge-sets for large orders, prove
finiteness of quenched QED. If that works, I trust electron loop
insertions will be next, and thereafter renormalons\rf{Lautrup77} \etc\ will go
\HREF{https://soundcloud.com/poets-org/notgogentle-mp3-5/s-2o7zI}
{gently into that good night}.
  \item
Develop a new formulation of QED in which each gauge set is given by a
computable integral. That might make it possible to evaluate orders
beyond 5-loops, as the number of gauge-set grows only polynomially.
  \item
A gauge-invariant set is by definition UV and IR finite. Does that mean
that the integrand is pointwise finite, no need for counterterms?
    \end{enumerate}
