% GitHub/reducesymm/QFT/planar.tex     %   compile by  pdflatex blog; biber blog

\chapter{Planar field theory}
\label{c-planar}

\begin{description}

% \item[2020-10-09 Predrag]   2023-03-17 did this, finalized:
% Search also for `planar' in \emph{reducesymm/dasgroup/blog.tex} files.

\item[2012-09-14 Ionel Popescu] visited Georgia Tech and gave a seminar
about Garoufalidis and Popescu\rf{GarPop12} {\em Analyticity of the
planar limit of a matrix model}. This is a very impressive paper that
makes Br{\'{e}}zin, Itzykson, Parisi and Zuber\rf{BIPZ78} rigorous, and I
have not studied it into any significant depth.

\item[2012-09-14 Predrag to Ionel Popescu] <ioionel@gmail.com>
I did not work out many of these planar diagram countings (I might have
something in unpublished notes, but it is most likely of no interest
today), but do we agree on free planar diagrams counting  Eq. (6.5) and the
planar cubic vertex (valence 3) diagram counting Eq. (6.8)
in our \HREF{http://www.cns.gatech.edu/~predrag/papers/NPB82.pdf}
{{\em The planar sector of field theories}}\rf{NPB82}?

It has to be $1/n$ in expansions, as planar n-leg Green's functions are
cyclically symmetric. If you ever care, I also have Feynman-diagram
counting formulas for full (not planar) theories. My problem is that Feynman
diagram expansions are
\HREF{http://www.cns.gatech.edu/~predrag/papers/preprints.html\#FiniteFieldTheo}
{stupid},
and I never found any use for my planar field
theory
\HREF{http://www.cns.gatech.edu/~predrag/papers/preprints.html\#PlanFieldThe}
{either}.
We were lucky with QED, as it is so weakly nonlinear, but QCD has
to be explored by its nonperturbative solutions, and that's real numerical
work and not pretty. Probably like turbulence exhibited by Navier-Stokes,
not instantons, merons, morons, or other analytic solutions.

Perturbative General Relativity has vertices/nodes of all orders, so if one
cars about planar perturbative GR, your counting might be of interest for
that case... Your Sect. 10.~{\em Other examples of planar limits} 3- and 4-
vertex counting should be of interest for planar QCD.

I enjoyed the seminar and am totally dazzled.

\item[2012-09-15 Ionel]
I will check that the 3-valent generating series in your paper matches
the one we can get with our formulae.   I think I did that calculation
but I do not have it now.

On the other hand, I agree that the matrix models are fading out to some
extent, though I saw last year some talks by physicists trying to resurrect
this area which are related to QCD.  They proposed several other matrix
models which are good for something.  Since I do not know that much
physics, I can not tell more.  One such particular talk was by Marcos
Marinio.

I will get back to you, with the calculations.

\item[2012-09-15 Predrag]
BTW, I have a citation gripe:
\begin{enumerate}

  \item
G. 't Hooft\rf{tHooft1974jz}
\HREF{http://igitur-archive.library.uu.nl/phys/2005-0622-152933/UUindex.html}
{\emph{A planar diagram theory for strong interactions}}
Nucl. Phys. B 72, 461 (1974).\\
 A fundamental `birdtracks' paper.	Introduced diagrammatic double-line
 notation for the adjoint representation of U(n). Does not introduce
 birdtrack projection operators for SU(n) and other groups. That is done
 in my 1976 paper.

  \item
G. 't Hooft\rf{tHooft78}
\HREF{http://igitur-archive.library.uu.nl/phys/2005-0622-153832/UUindex.html}
{\emph{On the phase transition towards permanent quark confinement}}
 Nucl. Phys. B 138, 1 (1978).\\
No `birdtracks.'	The projection operator for the adjoint
representation of SU(n) given in (2.18) is the same as in the above 1976
paper and in \HREF{http://birdtracks.eu/} {birdtracks.eu}\rf{PCgr}, up to
a normalization.	

  \item
G. 't Hooft\rf{tHooft82}
Comm. Math. Phys. {86}, {449--464} (1982)
\HREF{https://doi.org/10.1007/bf01214881}
{\emph{On the convergence of planar diagram expansions}},

  \item
 G. 't Hooft\rf{tHooft83}
\HREF{https://doi.org/10.1007/bf01206876}
{\emph{Rigorous construction of planar diagram field theories in four
       dimensional {Euclidean} space}}
 Comm. Math. Phys. {88}, 1 (1983). %\\

  \item
 G. 't Hooft\rf{tHooft84}
\HREF{https://doi.org/10.1007/978-1-4757-0280-4_10}
{{\em Planar diagram field theories}}

  \item
 G. 't Hooft\rf{tHooft99}
\arXiv{hep-th/9808113}
 {\emph{Counting planar diagrams with various restrictions}}\\
A very nice paper. Inter alia, rederives my 1981 planar field theory
 equations, without a citation.
\end{enumerate}

\begin{verbatim}
@Article{Cvit81planar,
  author       = {Cvitanovi{\'c}, P.},
  title        = {Planar perturbation expansion},
  year         = {1981},
  volume       = {99},
  pages        = {49},
  doi          = {10.1016/0370-2693(81)90801-7},
  journal = {Phys. Lett. B},
}

@Article{NPB82,
  author  = {Cvitanovi{\'c}, P. and Lauwers, P. G.
            and Scharbach, P. N.},
  title   = {The planar sector of field theories},
  journal = {Nucl. Phys. B},
  year    = {1982},
  volume  = {203},
  pages   = {385--412},
  doi     = {10.1016/0550-3213(82)90320-0},
}
\end{verbatim}

\item[2020-10-18 Predrag]
The above email was never sent. Popescu does not work on this any more.

\item[2014-10-01 Predrag to Ionel]
I'm deeply shocked that you would introduce the Mickey Mouse diagram with
no reference to my early work (when you were toiling through the
kindergarden), see bottom p. 118
\HREF{http://www.cns.gatech.edu/~predrag/papers/preprints.html\#PlanFieldThe}
{here}. Also, it is crazy to write all
those dummy indices - we have not done that  since 1975.

If you really want to write planar theory as the leading 1/N limit of
$tr(M^k)$ vertices, never denote hermitian matrices by two indices
downstairs; use up-down indices, see eq. (3.22). Then, as fast as a
lightning, switch to birdtracks, where $M^a_b$ is a directed line (with
an arrow indicated). If you like HeHa'd t'Hooft (pronounced just as it is
spelled), cite no paper written in any year other than 1974. But there is
lots of other good stuff out there - see see 4.9 A BRIEF HISTORY OF
BIRDTRACKS.

The planar filed theory is given in the unsang papers here. Yes, it
should be analytic - everything is growing as powers, there are no
combinatorials anywhere, there is no Dyson argument, perturbation series
is not asymptotic and should have a finite radius of convergence.

There are no exponentials in the partition functions - at best it is some
kind of continued fraction. Wherever the full field theory has fully
symmetrized weight 1/k!, planar field theory has the cyclic permutation
weight 1/k, where k is the number of external legs (fixing a special
point as Jones does strikes me as a very bad idea.. I'm not sure that
$N\to\infty \log (\exp(...))$ is good way to think about planar field theory, but
it will pull out the planar term, and by the time you account for
surfaces of every genus, combinatorials have to come back, so maybe it's
a good way to think.

Your analytic result (for k-vertices with every k) is very pretty.
General relativity has vertices of every k, there might be some geometric
interpretation to such a theory.

For the real challenge: I believe that the loop-insertion free part of
QED is also convergent, with a finite radius of convergence. Go for it!

\item[2014-10-02 Ionel]
I have to admit that this presentation of planar diagram and random
matrices I learned partially from Stavros, whom I consider to be closer
to math physics community, and with paper and pencil from the paper by
Zvonkin\rf{Zvonkin97}.  Again, this may be ancient history and
kindergarten for physicists but I was literally fascinated by this and
able to understand it.  Even though I tried to get more from the physics
community I was not able to follow the language and put this on hold.

I looked into your monograph and it is more readable than random
references I got before. I will start reading more.

I agree that writing the indices up-down instead of both down makes the
whole thing clearer.  I was plagued by Zvonkin's notation.

You are right, there are no exponentials in the partition function. I was
stressed to get to the analytic part, which in the end I did not have
time to do, and I was inexact with the expressions.
My intention with this talk was to cover more on the analytical part but
in the end I ended up with only the combinatorial aspect.  I could have
rushed a bit, but looking at the reaction of the students, I wanted to
get in the end something in one piece rather than two broken pieces.

Can you point out which of your paper has a clear statement of the what
you mention as ``loop-insertion free part of QED'' you refer to?  I want to
take a look.

\item[2014-10-02 Predrag to Ionel]
 You just had the mother of planar field theory (father unknown)
 parachute into a cloud of grad students. You did a great job and I was
 amazed by how many students showed up (even though the pizza is
 atrocious).

\rf{Zvonkin97} notes are fun. After reading them, you might be receptive to my
\HREF{http://chaosbook.org/FieldTheory/} {field theory notes}.

The only paper known to me that claims that QED perturbation expansion
might be analytic is the one I cited.

By the way, your grown-up colleagues have been bugging me for a month
that all math is pure and cannot be subdivided into areas (I was not
allowed to write that SoM is Balkans, only with 52 small countries) - so
there is no border between physics and math either. It's linguistical
only.

\item[2013-01-20 Predrag]
Lucini and Panero\rf{LucPan13} \CBlibrary{LucPan13}
might be of interest. All I get is one sentence and a reference only
to \refref{PlanFieldThe}.

I should also read Kang and Loebl\rf{KanLoe09}
{\em The enumeration of planar graphs via {Wick}'s theorem}.

\item[2018-04-28 Predrag]
Cicuta\rf{Cicuta79} {\em Vanishing graphs, planarity, and {Reggeization}},
and
Butera, Cicuta and Enriotti\rf{BuCiEn80} {\em Group weight and vanishing graphs}
should be included in the next update of birdtracks.eu.

\item[2018-06-06 Predrag]

\HREF{https://www.hep.phys.soton.ac.uk/content/james-drummond}
{James M. Drummond} {\em Cluster algebras and scattering amplitudes},
\arXiv{1710.10953}, was a stunning performance, a picture of which I'll
post on Flickr.com. Analytic relations for scattering amplitudes
expressed the cluster algebras for planar $N=4$ super Yang-Mills theory
that generalize the Steinmann relations.  I still do not know what hit
me.

\item[2018-08-11 Predrag]
Khovanov\rf{Khovanov14} {\em Heisenberg algebra and a graphical calculus},
\arXiv{1009.3295} is the foundational paper in this field.
He gives a categorification of the Heisenberg algebra that acts naturally
on the graphical category of representations  of  symmetric  group.
Abstract: ``
A new calculus of planar diagrams involving diagrammatics for biadjoint
functors and degenerate affine Hecke algebras is introduced. The calculus
leads to an additive monoidal category whose Grothendieck ring contains
an integral form of the Heisenberg algebra in infinitely many variables.
We construct bases of vector spaces of morphisms between products of
generating objects in this category.
''

\item[2019-12-06 Predrag]
\phantomsection\label{post:PSSY19}
Penington, Shenker, Stanford and Yang\rf{PSSY19}
{\em Replica wormholes and the black hole interior},
\arXiv{1911.11977}:
Their analysis is inspired by the ``free probability'' results discussed
in 2009 Speicher\rf{Speicher09}, {\em Free probability theory},
\arXiv{0911.0087} and \arXiv{1404.3393}, and e.g. figure 1 of the (Predrag's)
earlier\rf{PlanFieldThe}, from 1981.

   \item[2023-03-23 Predrag]
Voiculescu\rf{Voiculescu86}
{\em Addition of certain non-commuting random variables} (1986).

Speicher\rf{Speicher94} {\em Multiplicative functions on the lattice of
non-crossing partitions and free convolution} (1994)

Speicher {\em Free probability (FP) theory and non-crossing partitions},
S{\'e}minaire Lotharingien de Combinatoire
\HREF{https://www.emis.de/journals/SLC/wpapers/s39speicher.pdf}
     {{\bf 39}, B39c} (1997),
built upon fundamental work of Rota.

Mingo and Speicher\rf{MinSpe17}
{\em Free Probability and Random Matrices} (2017).

In general, the $n$-th moment is a polynomial in the cumulants $c_1, \cdots ,
c_n$, but it is very hard to write down a concrete formula for this.
Nevertheless there is a very nice way to understand the combinatorics
behind this connection, and this is given by the concept of multiplicative
functions on the lattice of all partitions.

Introduces zeta function and its inverse, M{\"o}bius function. Has nice
graphical illustration of 4th cumulant on p.~14; FP is related to my planar
field theory in which external legs cannot be crossed (but my traces have
cyclic symmetry).


\item[2019-12-06 Predrag]
See also MathOverflow
\HREF{https://mathoverflow.net/questions/94028/classical-convolution-vs-free-convolution}
{Classical convolution vs. free convolution}
and Terrence Tao's
\HREF{https://terrytao.wordpress.com/2010/02/10/245a-notes-5-free-probability/\#more-3466}
{non-commutative probability} notes.

Smith\rf{Smith92} {\em Planar version of {Baym-Kadanoff} theory}

Djordje Mini\'c, {\em Remarks on Large N Coherent States}
\arXiv{hep-th/9502117}: `` Recently it has become apparent that another
algebraic structure seems to be natural for the large N matrix models,
the so called free Fock space. This concept naturally appears within the
context of non-commutative probability theory, developed by Voiculescu
and collaborators\rf{Voiculescu91}, and it has been used in the physics
literature in the analysis of large N matrix models by Haan\rf{Haan80} and
Cvitanovi\'c and co-workers\emph{\rf{NPB82}}, and more recently by
Douglas\rf{Douglas95} and Gopakumar and Gross\rf{GopGro95} {\em Mastering
the master field}. (Similar ideas have been exploited in the study of so
called infinite statistics by Greenberg\rf{Greenberg91}).
''

Haan writes `` In 1980 I used these operators to define ``planar fields''
for the non-interacting field theory\rf{Haan80}'' (received 16 June
1980), which puts him three months ahead of me\rf{PlanFieldThe} (received
21 October 1980).

Douglas\rf{Douglas95} {\em Stochastic master fields} writes: ``We treat
the stochastic equation for large N master fields proposed by Greensite
and Halpern using a construction of master fields modelled after work of
Voiculescu, and show that it contains the same information as the usual
factorized Schwinger-Dyson equations. We comment on the relation to
earlier work of Haan\rf{Haan80} and of Cvitanovi\'c, Lauwers and
Scharbach\rf{NPB82}.

``The problem of the construction of the master field has been discussed in
many works [...] by using methods of non-commutative
probability theory. In \refref{AreVol96}
it was shown that the masterfields satisfy to standard equations of
relativistic field theory but fields are quantized according to a new
rule. These fields have a realization in the free (Boltzmannian)
Fock space.''

Araki and Tanii\rf{AraTan96}
{\em {Ward-Takahashi} identities in large {N} field theories}
use my formalism to derive the {Ward-Takahashi} identity,
their eq.~(24).

Ebrahimi-Fard and Patras\rf{EbrPat16}
{\em The combinatorics of {Green}'s functions in planar field theories}:
`` In the early 1980s, Cvitanovi\'c \etal\rf{PlanFieldThe,NPB82} proposed
a perturbative approach to quantum field theories in the planar setting.
This was largely motivated by a desire to properly encode the behavior of
the planar sector of quantum chromodynamics (QCD), based on 't~Hooft's
seminal 1974 paper\rf{tHooft:1974jz}. An interesting feature of planar
field theories is the manner in which the calculus of symmetry factors
differs (and becomes simpler) compared to classical field theories.
Planarity is reflected in the strictly non-commutative nature of the
theory, which results in a rather substantial deviation from the
classical description of the relations between different types of Green's
functions. Cvitanovi\'c \etal\ observed that the functional relation
between the generating functionals of the full and connected planar
Green's functions is encoded by a fixed point type equation, which is
solved by the generating functionals. This fixed point equation replaces
the common exponential map that relates the generating functionals of the
full and connected Green's functions in classical theories (in this
article, classical will refer to non-planar field theories and their
associated objects, such as Green's functions, Feynman diagrams, and
amplitudes).

The exponential relation between classical generating functionals is
analogous to the moments-cumulants relation in classical probability
theory\rf{Speed83}. Singer realized the existence of a similar connection
between planar field theories and Voiculescu's theory of free
probability\rf{Voiculescu91}. It turns out that the description given by
Cvitanovi\'c \etal\  of the relations between planar Green's functions is
closely related to Speicher's combinatorial approach to the relations
between moments and cumulants in free probability\rf{Speicher09}.

The description of the relations between planar Green's functions
presented in these notes is based on our recent work on the algebraic and
combinatorial structures underlying the relations between moments and
cumulants in free and classical probability theory\rf{EbrPat15}
[...]
It turns out that in both cases the linear fixed point equation has a
proper exponential solution. In the classical case, this exponential
solution coincides with the standard exponential that relates classical
moments and cumulants. In the non-classical setting, the relation between
free cumulants and moments is also portrayed by an exponential, which is
defined with respect to a non-commutative shuffle product. The difference
between these two exponentials is analogous to the difference between
exponential solutions of scalar- and matrix-valued non-autonomous linear
differential equations. Here, we propose a similar approach to
the–Hopf–algebraic understanding of the relations between full and
connected Green's functions in planar QFT.

[...] the polynomial expressions giving full Green's functions in terms
of connected ones constitute a multivariate generalization of the
classical Bell polynomials that relate, among others, moments and
cumulants in classical probability\rf{EbLuMa14}.

[...]  A precise description of the combinatorial nature of the recursive
structure that is on display here will be elaborated on below, in terms
of a double tensor Hopf algebra equipped with a non-cocommutative
unshuffle coproduct.

''
\item[2020-10-18 Predrag]
I might not have noticed this paper (still have to read it):
Koplik, Neveu and Nussinov\rf{KoNeNu77}
\HREF{https://doi.org/10.1016/0550-3213(77)90344-3}
{{\em Some aspects of the planar perturbation series}}:\\
We discuss several features of the class of planar Feynman diagrams in
perturbation theory. We show that the number of planar n-loop diagrams
grows exponentially in n in any field theory, and calculate the detailed
asymptotic behavior for $\phi^3$ and $\phi^4$. We discuss the
implications of this result for the convergence of the series and its
relation to various semiclassical instabilities, with special reference
to non-Abelian gauge theories. We also derive a functional integral
equation for planar graphs.

\item[2020-10-18 Predrag]
I have totally forgotten that I had read this paper in 1980, have to
reread it:
Br{\'{e}}zin, Itzykson, Parisi and Zuber\rf{BIPZ78}
\HREF{https://doi.org/10.1007/bf01614153}
{{\em Planar diagrams}}:\\
We investigate the planar approximation to field theory through the limit
of a large internal symmetry group. This yields an alternative and
powerful method to count planar diagrams. Results are presented for cubic
and quartic vertices, some of which appear to be new. Quantum mechanics
treated in this approximation is shown to be equivalent to a free Fermi
gas system.

Also
Itzykson and Zuber\rf{ItzZub80a} {\em The planar approximation. {II}},
(1980);
\HREF{https://www.lpthe.jussieu.fr/~zuber/MesPapiers/iz_JMP80.pdf}
{click here},
\HREF{https://www.lpthe.jussieu.fr/~zuber/erratum_IZ80.pdf} {errata}.

\item[2020-10-18 Predrag]
I have never noticed this paper (still have to read it):
Parisi\rf{Parisi82}
\HREF{https://doi.org/10.1016/0370-2693(82)90849-8}
{{\em A simple expression for planar field theories}}:
In a $U(N)$ invariant theory in presence of a random background gauge
field, the quenched expectation values of $U(N)$ invariant objects are
volume independent in the limit $N\to\infty$.

\item[2022-06-08 Predrag to Maxim Kontsevich]
- sent as pdf attachment to a 15 year old email:)

Thanks for the `Rutgers' talk today - fascinating!
Some comments to myself, feel free to ignore:

For me the simplest example of `over-determined' matrix system is the
Hamilton-Cayley equation satisfied by any $[n\times{n}]$ matrix (see
birdtracks.eu
\HREF{https://birdtracks.eu/version9.0/GroupTheory.pdf\#section.6.5}
{sect.~6.5 {\em Characteristic equation}}). In the free algebra basis of
cyclic words (traces of products of matrices) and their products, the problem
is `over-determined' in the sense that there are infinitely many equations,
one for each $p>n$ in my eq.~(6.48).

Your `over-determined' examples are much more sophisticated, I have no
suggestion as how to prove them.

For my papers that I think are related to your work, see
\HREF{https://cns.gatech.edu/~predrag/papers/preprints.html\#PlanFieldThe}
{Planar field theory} on my homepages.

Your loop equations look to me similar to my \refref{PlanFieldThe}
Dyson-Schwinger equations
\HREF{https://cns.gatech.edu/~predrag/papers/PlanFieldThe.pdf} {eq.~(19)},
and I think your Catalan numbers show up in eq.~(21).
(A longer version, our \refref{NPB82}, is
\HREF{https://cns.gatech.edu/~predrag/papers/NPB82.pdf} {here}).
Catalan number counting also shows up in my \refref{PCar}
\HREF{https://cns.gatech.edu/~predrag/papers/PCar.pdf} {eq.~(B2)},
see also eq.~(B5).

The Feynman graph counting in our \refref{CvLaPe78}, like
\HREF{https://cns.gatech.edu/~predrag/papers/PlanFieldThe.pdf} {eq~(3.14)},
could (maybe?) also be of interest.

   \item[2023-02-20 Predrag]
Titouan Carette, Etienne Moutot, Thomas Perez and Renaud Vilmart
{\em Compositionality of planar perfect matchings},
\arXiv{2302.08767}:

Computing scalars of the planar W-calculus corresponds to counting perfect
matchings of planar graphs, and so can be carried in polynomial time using
the FKT algorithm, making the planar W-calculus an efficiently simulable
fragment of the ZW-calculus, in a similar way that the Clifford fragment is
for ZX-calculus.

The main idea behind the algorithm is that for planar graphs, it is possible
to find a good orientation of the edges (called a Pfaffian orientation) in
polynomial time such that the number of perfect matchings is the Pfaffian of
the adjacency matrix A (actually its skew-symmetric version, called Tutte
matrix) of the oriented graph. A result due to Cayley then shows that the
Pfaffian is the square root of the determinant of A.

In particular, all planar graphs are Pafaffian, says a very unhelpful
(\HREF{https://en.wikipedia.org/wiki/Pfaffian_orientation} {wiki}).

   \item[2023-03-02 Predrag]
Notes on `free probability'.

In 1980 I formulated {\em planar field theory}\rf{PlanFieldThe,NPB82};
the two published papers are
\HREF{https://cns.gatech.edu/~predrag/papers/preprints.html\#PlanFieldThe} {here}).
A bosonic field theory an $n$-point Green function is fully symmetric, so its
symmetry factor is $1/n!$.
In planar field theory sources $J$ do not commute (we called them
'non-commuting sources' or 'non-$c$'; probabilists seem to call them free `free',
which -for a physicist- is confusing, as we already have `free' or
non-interacting field theories), and an
$n$-point Green function is only cyclically symmetric, with symmetry factor
$1/n$ (as in Ruelle's dynamical zeta function, as periodic orbits
are generally only cyclically symmetric). I define the theory by Dyson-Schwinger
equation, e.g. figure 1 of my
\refref{PlanFieldThe}.

Speicher explains the 1991  work of Voiculescu\rf{Voiculescu91}: ``Free probability theory was
created by Dan Voiculescu around 1985, motivated by his efforts to
understand special classes of von Neumann algebras.'' He writes about
Cvitanovi{\'c}, Lauwers and Scharbach\rf{NPB82} {\em The planar sector of
field theories}: ``This description of freeness in terms of free
cumulants is related to the planar approximations in random matrix theory.
In a sense some aspects of this theory of freeness were anticipated (but
mostly neglected) in the physics community in the \refref{NPB82} paper.''
Voiculescu does not seem to cite us at all.

%(see also {\bf 2019-12-06} post on \refpage{post:PSSY19})

Laura Foini and Jorge Kurchan\rf{FoiKur19}
{\em The Eigenstate Thermalization Hypothesis and Out of Time Order Correlators},
\arXiv{1803.10658} (2019),
and
Silvia Pappalardi, Felix Fritzsch and Toma{\v{z}} Prosen\rf{PaFrPr23}
{\em General eigenstate thermalization via free cumulants in quantum lattice systems},
\arXiv{2303.00713} (2023)
emphasize the role that free cumulants play in `eigenstate thermalization'.
Pappalardi \etal\rf{PaFrPr23} eq.~(S3d) for the 4th cumulant
\begin{subequations}
	\begin{align}
		k_1(A) & = \langle A \rangle\ ,
		\\
		\label{k2}
		k_2(A) & = \langle A^2 \rangle - \langle A \rangle^2\ ,
		\\
		k_3(A) & = \langle A^3 \rangle - 3 \langle A \rangle \langle A^2 \rangle +2 \langle A \rangle^3\ ,
		\\
		k_4(A) & = \langle A^4 \rangle - 2\langle A^2 \rangle^2 - 4 \langle A \rangle\langle A^3\rangle + 10\langle A \rangle^2\langle A^2 \rangle - 5 \langle A \rangle^4
\label{PaFrPr23(S3d)}
	\end{align}
\end{subequations}
differs from ChaosBook
\HREF{https://chaosbook.org/chapters/ChaosBook.pdf\#equation.T.1.10}{(A20.6)}.
That is because their formula is for \emph{free} cumulants.

For comparisonm here is the relevant clipping from ChaosBook Appendix A20:

    \begin{quote}
Moments can be collected into the (exponential) moment-generating function
\index{moment!-generating function}
\index{generating!function, exponential}
\index{exponential!generating function}
\beq
\expct{e^{\beta\obser}}
    = 1 + \sum_{k=1}^\infty \frac{\beta^k}{k!}\expct{\obser^k}
\,.
\ee{MomGenFct}
Why the prefactor $1/k!$ (a Taylor series), and not $1/k$ (a logarithmic
series), or $1$ (discrete Laplace transform or $Z$-transform)?
In statistical,
stochastic and quantum mechanics / quantum field theory applications one
is solving linear ODEs or PDEs, and their solutions are always
exponential in form.

Hardly any experiment measures $\obser^k$ for $k>2$
and raising approximate numbers to high powers is not smart:
if $|\obser| <1$, $\obser^k$ gets very small very fast, and conversely if
$|\obser| >1$, $\obser^k$ gets very big. Still, with a bit of hindsight,
one finds that moments do play a natural,
fundamental role if folded into
the \emph{cumulant-generating function}
\index{cumulant} \index{cumulant!generating function}
\beq
\ln \expct{e^{\beta\obser}}
  = \sum_{k=1}^\infty \frac{\beta^k}{k!}\expct{\obser^k}_c
\,,
\ee{MomGenFct1}
where the subscript $c$ indicates a \emph{cumulant}, or, in statistical
mechanics and quantum field theory contexts, the `connected Green's
function'. Were $\expct{\obser^k}=\expct{\obser}^k$, we would have only
one term in the series \refeq{MomGenFct1},
\(
\ln \expct{e^{\beta\obser}}
  = \ln e^{\beta\expct{\obser}}
  = \beta\expct{\obser}
\,,
\)
and that would be that. So cumulants $\expct{\obser^k}_c$ measure
fluctuations about the mean $\expct{\obser}$. Indeed, expanding the
logarithm of the series \refeq{MomGenFct}, it is easy to check that the
first cumulant is the {mean}, the second is the {variance},
\beq
\expct{\obser^2}_c
   = \expct{(\obser-\expct{\obser})^2}
   = \expct{{\obser}^2} - \expct{\obser}^2
   = \sigma^2
\,,
\ee{variance}
and $\expct{\obser^3}_c$ is the third central moment, or the
\emph{skewness},
\index{skewness}
\beq
\expct{\obser^3}_c
   = \expct{(\obser-\expct{\obser})^3}
   = \expct{{\obser}^3}
    -3 \expct{\obser^2} \expct{\obser}
    +2 \expct{\obser}^3
\,.
\ee{skewness}
% The higher cumulants, however, are \emph{not} central moments.
The fourth cumulant,
\bea
\expct{\obser^4}_c
  &=& \expct{(\obser-\expct{\obser})^4}
      -3 \expct{(\obser-\expct{\obser})^2}^2
  \continue
  &=& \expct{{\obser}^4}
     -4 \expct{\obser^3} \expct{\obser}
     -3 \expct{{\obser}^2}^2
   +12 \expct{\obser^2} \expct{\obser}^2
    -6 \expct{\obser}^4
\label{4thCumulant}
\,,
\eea
rewritten in terms of standardized moments, is known as the
\emph{kurtosis}:
\index{kurtosis}
\beq
\frac{1}{\sigma^4}\expct{\obser^4}_c
  = \frac{1}{\sigma^4}\expct{(\obser-\expct{\obser})^4} -3
\,.
\ee{kurtosis}

The reason why cumulants are preferable to moments is that
for a normalized Gaussian distribution all cumulants beyond the
second one vanish, so they are a measure of deviation of statistics
from Gaussian.
For a `free' or `Gaussian' field theory the only non-vanishing cumulant
is the second one; for field theories with interactions the derivatives
of
\(
\ln \expct{\exp(\beta\obser)}
\)
 with respect to $\beta$ then yield cumulants, or the
Burnett coefficients, % \refeq{trans-k},
or `effective' $n$-point Green's functions or $n$-point correlations.
    \end{quote}

   \item[2023-03-23 Predrag]
Celestino, Ebrahimi-Fard, Nica, Perales and Witzman\rf{CENPW23}
{\em Multiplicative and semi-multiplicative functions on non-crossing
partitions, and relations to cumulants}
(2023)

   \item[2023-03-17 Jean-Bernard]
\HREF{https://www.lpthe.jussieu.fr/~zuber/index_en.html} {Zuber}
zuber@lpthe.jussieu.fr
\\
I posted two days ago \arXiv{2303.05875}. What will please you, I hope,
it makes a crucial use of an old work of yours, on planar diagrams,
namely the graphical interpretation of the functional relation
$Z(j)=1+W(j Z(j))$ between generating functions of full and connected
Green functions. What I did was to generalize  that relation to higher
genus. Why did I do that? I was interested in a  purely combinatorial
problem, namely the counting of partitions (of the set ${1,\cdots,n}$,
say), while taking into account their genus (defined in the usual way,
through their diagrammatic representation as a map).

What I found curious is that your own paper seems to have remained
unnoticed, at least in the combinatorics community. Only a few weeks ago
did I hear a mathematician, Fr{\'e}d{\'e}ric Patras, from Nice, mention
it.

   \item[2023-03-17 Jean-Bernard]
BTW, did you know that the counting of \emph{planar Green functions}
(or non crossing cumulants, as people call it also), that is encoded in
the equation above, had been achieved in 1972, 6 years before our
Br{\'e}zin, Itzykson, Parisi and Zuber\rf{BIPZ78}
{\em Planar diagrams} (1978),
 by a certain Kreweras\rf{Kreweras72}, in
{\em Sur les partitions non croisees d'un cycle} (1972).

   \item[2023-03-02 Predrag]
Wow! Never heard of
\HREF{http://www.numdam.org/item/MSH_1999__145__103_0.pdf} {Germain
Kreweras}, Professeur de Math{\'e}matiques à l'Universit{\'e} Pierre et
Marie Curie (Paris VI), originaire de la Lithuanie:) The paper is
constantly cited, now at 400 citations, and getting into these citations
one descends into a rabbit hole too vast for me. I give up.

   \item[2023-03-17, 2023-03-23 Predrag] Notes on \\
Zuber\rf{Zuber23} {\em Counting partitions by genus. {I. Genus} 0 to 2},
\arXiv{2303.05875} (2023). Very clear and pedagogical. To talk to many-body
quantum chaos physicist, needs a presentation which is less systematic but
more diagramatical and examples-driven.

[...] the formula given by
Kreweras\rf{Kreweras72} on the census of non crossing partitions may be
conveniently encoded in the following functional relation between the
generating function of moments $Z[j]$ and that of cumulants $W[j]$
\beq
 Z[j] = 1+ W[jZ[j]]
\,.
\ee{Z0W}
This relation is equivalent to the functional identity
$P\circ G=\mathrm{id}$, where $G(u):= u^{-1} Z^{(0)}(u^{-1})$ and
$P(z):= z^{-1} W(z)$, and $R(z)=P(z)-\frac{1}{z}$ is the celebrated
Voiculescu $R$ function\rf{Voiculescu86,Speicher94}.
%\cite{V86, Speicher}

There is a simple diagrammatical interpretation of the relation \refeq{Z0W}
due to Cvitanovi\'c\rf{PlanFieldThe},
% see Fig.~\ref{cvitanovic1},
which reads: {\it in {an arbitrary} planar (i.e., non-crossing)
diagram, the marked point 1 on the exterior circle is necessarily
connected to a $n$-vertex, $n\ge 1$, between the $n$ edges of which lie
arbitrary insertions of other (linear) diagrams of $Z[j]$}.

   \item[2023-03-22 Predrag]
If one thinks visually, planar field theory Dyson-Schwinger equations\rf{PlanFieldThe}
are actually easier to draw and understand than the full ``crossing''
field theory. The idea that there could be such a theory
came from 1974 't~Hooft\rf{tHooft1974jz}
\HREF{http://igitur-archive.library.uu.nl/phys/2005-0622-152933/UUindex.html}
{\emph{A planar diagram theory for strong interactions}}
$1/N$ expansion.

In \HREF{http://www.cns.gatech.edu/~predrag/papers/NPB82.pdf}
{{\em The planar sector of field theories}}\rf{NPB82}
we made it into a full-fledged, stand-alone planar QCD,
including the planar QCD Ward identities,
as I needed something like that to prove my
\HREF{https://ChaosBook.org/~predrag/papers/preprints.html\#FiniteFieldTheo}
{\em QED finiteness conjecture}. I did not succeed in proving the
conjecture.

I set up my formulation of planar field theory paralleling the
full theory functional formalism of
Itzykson and Zuber\rf{ItzZub80} {\em Quantum Field Theory} (1980).
Parenthetically, this malloppo has  705 pages, and after watching our
graduate students' study group suffer through endless $\mu\nu$'s, I wrote
for them a 118 pages version with only the essential insights, nary a
$\mu\nu$:

Cvitanovi{\'c}\rf{FieldThe}  {\em Field Theory} (1983),
\HREF{http://ChaosBook.org/FieldTheory} {website}.

Turns out I was wrong: the graduate students do not want to the beauty of
$Z[j]\Rightarrow W[j] \Rightarrow \Gamma[\Phi]$: they \emph{want} to
suffer, and they continue with their $\mu\nu$ litanies to this day,
understanding nothing.

   \item[2023-03-22 Predrag to Jean-Bernard] I believe that `free
       cumulants' are given by my connected ('Helmholtz free energy')
       generating function $W[J]$.

Question: do you\rf{Zuber23} consider the 1PI, 'Gibbs free energy', or
effective action $\Gamma[\Phi]$,
\beq
 W[J] = \Gamma[\Phi] +  J_i \Phi_i + \Phi_i J_i
\,,\qquad
 J_i = j_i Z[j]
\,,
\ee{NPB82(4.17)}
our \refref{NPB82} eq.~(4.17)? What is that in the language of planar
(AKA `free') cumulants?

I can elaborate, in way that I fear most people find hard to follow. In
\wwwcb{} I develop, in the language of dynamical systems, periodic orbit
theory that even people who believe that they understand it, don't (at
this meeting, Bogomolny).

In this, dynamical system's context, field theorist's
partition function $Z[j]$ is called a `trace formula', easy to derive,
but not smart.

The smarter formulation corresponds to field theorist's cumulants $W[j]$,
versions of which are known as dynamical zeta functions and Fredholm
determinants. They are smart, because they automatically account for
ergodic systems shadowing of long orbits by shorter ones, and have much
better convergence (my\rf{inv} `cycle expansions') than the trace formulas.

Essential to these formulations is to express the set of all periodic
orbits in terms of the smallest parts that I call `prime' orbits from
whose repeats under time translations and discrete symmetries build all
periodic orbits. Periodic orbits of length $n$ consist of $n$  temporally
ordered periodic points, so their combinatorial weight in generating
functions is $1/n$, rather than field theorist's $1/n!$. Suggestive, no?

Indeed, the \emph{prime} orbits can be generated systematically by a
`planar' (`free' is a really confusing term for a field theorist) version
of M{\"o}bius function, or what I call\rf{skeleton} the ordered
concatenations of primes, see
\toChaosBook{section.R.2} {Appendix~A18.2}
{{\em Prime factorization for dynamical itineraries}}.

(According to Jon Keating, these are the `Beurling primes', the 1937
theory that establishes the prime number theorem for ordered sets, but I
have not tracked that reference down.)

My ordered concatenations of primes is from when chaos was all the rage:
my 11 June 1990 contribution to the Los Alamos conference known for -at
least to me- Smale's list of
\HREF{https://en.wikipedia.org/wiki/Smale\%27s_problems}
{eighteen unsolved problems}. I found playing Hilbert somewhat pompous,
so my talk\rf{skeleton} ends with 10 homework problems.
In this workshop I am trying to explain my solution to homework problem
5.~\emph{Field theory}.

My question to Jean-Bernard, rephrased:
Is the ordered concatenations of primes the planar version of `effective
action' or `Gibbs free energy' $\Gamma[\Phi]$?

   \item[2023-03-27 Jean-Bernard]
— yes, free cumulants have a generating function given by the function
called W[J] in your notes. In our 1978 paper\rf{BIPZ78}, sect 3.3, we
had found the remarkable relation between that W[J] called there $\psi(j)$
and that of moments ({\em Green functions})  called $\phi(j)$, namely
$\phi(j)=\psi(j \phi(j))$. (More standard notations for a field theorist are
indeed W(j) and Z(j) and this is what I have been using in my latest
paper). And we established using Lagrange inversion formula the
equivalence between  that functional relation and the explicit counting
of free/planar cumulants/Green functions, rediscovering Kreweras'
formulae established 6 years before… But we missed the very nice
diagrammatic interpretation of that $Z(j)=1+W[j Z[j]]$  that you gave
soon after.

Now your next question is about 1PI functions/diagrams. That too we
discussed in our sect 3.4 in the planar limit, I don't know if there is
any nice diagrammatic interpretation behind it. Maybe you know ? And do
these 1PI simplify anything in our current discussion ?

Sorry for these lengthy reminders. Our mistake was to present this
discussion of the relation between W and Z after starting computing
things by the saddle point method, the two things being obviously
independent. As the latter is non rigorous, most mathematicians stopped
before that point and don't seem to be aware of that relation… And
needless to say, they didn't read either your Physics Letter of 1981...

— as for the possible connection of partitions with periodic orbits and
their factorization, I don't know. I know very little about these orbits
and I have no intuition. Is there a diagrammatic representation of these
things that would parallel that of maps, partitions etc ?

   \item[2023-03-27 Jean-Bernard]
A question to Laura and Silvia. Did you eventually manage to get in touch
with Beno{\^{\i}}t Collins and to ask him about this Theorem 2.6 of their
paper\rf{CMSS07} \arXiv{math/0606431}? This is, I find, quite intriguing,
because it establishes a link between free cumulants and what people call
Weingarten calculus, in the large $N$ limit, namely formulae involving
characters of the \SUn{N} and symmetric groups $S_N$. I checked it in
simple cases, but I still miss a general argument.
Maybe Predrag's birdtracks would help?

I also looked at the subleading term in that large N limit. The bad (or
stimulating?) news is that it is not the $1/N^2$ term that I would
attribute to genus 1 corrections... Strange. This seems to confirm that
there are several extrapolations of free cumulants...

   \item[2023-03-27 Laura Foini]  laura.foini@gmail.com\\
Speicher told me that theorem 2.6 is a particular case of formula 28
which is proven for a more general case which I still have to understand.
The simple formula of the product of matrix elements being equal to free
cumulants he didn't know where to find it.
Maybe Predrag knows better.
But as we show the formula holds also for the orthogonal group so in my
opinion there should be a way to get the asymptotic for both U(N) and
O(N) which of course should be consistent with Weingerten calculus. Maybe
it is already known.

   \item[2023-03-27 Silvia Pappalardi] <pappalardi.ga@gmail.com>\\
There is something (probably more) I am missing.

At the beginning I thought that the [Wrong! Missing] diagram was coming from
the difference between commutative sources and not c ones (full cumulants or
planar ones). But now I am confused and I think I am wrong.

I tried to repeat the exercise and compute the fourth order contribution by
pulling legs on the ones you sent us (as attached). From a first attempt
(without keeping tracks of the indices) I think I got too many diagrams,
which also include the non crossing ones. I get 3  [O - - ] x [O - -], while
the non-crossing are only 2  [O - - ] x [O - -]. I repeated the exercise
keeping track of the indices, and I now can identify the ``crossing partition''
in the dashed box of my notes. But I am now wondering, is there a way to get
rid of it diagrammatically? Or shall this come from an entropic counting like
in random matrix models?

   \item[2023-03-29 Predrag]
I iterated \refeq{Z0W} to order 4 on this
\HREF{https://flic.kr/p/2opXGnr} {blackboard}
and planar field theory moments generating function $Z[j]$ expressed in
terms of (connected) cumulants $W[j]$ agrees with \refeq{PaFrPr23(S3d)}
from Pappalardi \etal\rf{PaFrPr23}.


   \item[2023-03-30 Predrag]  Collins \etal\rf{CMSS07}\\

Within free probability theory of first order, machinery was provided by
Voiculescu with the concept of the R-transform, and by Speicher with the
concept of free cumulants; see, e.g., [VDN92, NSp06].
% D. V. Voiculescu, K. J. Dykema, and A. Nica. Free random variables.
% American Mathematical Society, Providence, RI, 1992.
%
% A. Nica and R. Speicher: Lectures on the Combinatorics of Free Probability,
% London Mathematical Society Lecture Note Series, New York :
% Cambridge University Press, to appear.

   \item[2023-03-31 Jean-Bernard]
2 \textbf{Laura}, re the funny diagram of our discussion : there is no contradiction
with the kind of arguments and power counting of $N$ that I was recalling yesterday.
These arguments apply to ($U(N)$ or $O(N)$) invariant objects, typically products of traces
of powers of the matrix. Hence in which each matrix index appears twice and only twice,
with implicit summation. And the double line notation of Mr. 't~Hooft  is fully relevant,
and the double lines keep parallel.
The quantity you were considering, on the other hand,
$< A_{ij}^2 A_{ji}^2>$ or the like, is not of that type. Whence the difficulties you have
in drawing the lines and counting powers of N : it's just another game!

   \item[2023-03-27 Laura]
For what concerns lower order diagrams I am quite confident that in the 4th
moment where you start to see differences between cumulants and free
cumulants you neglect $A_{ij}^4$. And that the diagrams that we retain are in one
to one correspondence with non-crossing partitions.

The proof of the formula you are talking about based on low rank HCIZ integral is in our paper
\arXiv{1906.08479} section 5.1.

   \item[2023-03-31 Jean-Bernard]
2 \textbf{Predrag}, I regret we didn't have more time to discuss. In particular I would have been
happy to ask you about this amazing occurrence of free cumulants that I was
mentioning at the end, namely  this computation of $< A_{i1 i2} A_{i2 i3}… A_{in i1}>$,
$i1, i2, \cdots$ in all distinct and not summed over, averaged over the unitary or orthogonal
group acting by conjugation on A. In the large N limit, this gives
\[
\lim   < A_{i1 i2} A_{i2 i3}… A_{in i1}> = N^{1-n} \kappa_n(A).  \qquad\qquad  (1)
\]
I suspect that the proof may be hidden in an early paper by B. Collins, his thesis or
else. I checked it for low values of n but I lack a general and intrinsic proof of it.
Denis Bernard showed me a proof which is quite elegant but extrinsic, since it uses
yet another unexpected occurrence of free cumulants, namely the large N, finite rank
limit of the HCIZ integral $I(A,B) =\int DU  \exp N Tr A U B U^*$  : if B is of finite rank, then
\[
\log I(A,B) = N \sum_n     \kappa_n(A) (\tr B^n)  /n    \qquad\qquad  (2)
\]
Using this for B a cyclic permutation on $\{1,2,\cdots n\}$, you get   (1).
Nice ... but {\em extrinsic}, it seems to me.

My question is : are your famous birdtracks of any help in proving that relation?

   \item[2023-04-17 Laura]
I was discussing with Dario Benedetti (Ecole Polytecnique) and we agreed that in the
't~Hooft expansion and in our diagrams you have the same subleading terms probably.
I was saying that my subleading term goes as $1/N$ and Dario explained
me that you have it with symmetric matrices, with non orientable surfaces,
and it is a diagram which correspond to the Moebius strip.

The term $1/N^2$ corresponds to $A_{ii}^4$ which is generically subleading only if\\
$\frac{1}{N}\tr A=0$ as in the Gaussian case.

I thought better to today discussion and I think it is not quite correct.
In fact my subleading term
$\frac{1}{N} \sum_{i \ne j} (A_{ij} A_{ji})^2$ is there also for hermitian models.

I think that the difference between the two approaches is that in the
"standard approach" you don't care of distinguishing if two indices are equal
or not and you sum over. So that term that I single out is a term out of many
in your leading diagram
$A_{ij} A_{ji} A_{ik} A_{ki}$ when $j=k$.
In our approach we care of keeping track of repetition of indices.

   \item[2023-04-18 Jean-Bernard]
your subleading term being in $1/N$ rather than $1/N^2$
has nothing to do with real symmetric vs Hermitian matrices.
As I said in a previous mail, the argument à la 't Hooft with a counting
of powers of N in terms of faces of a surface and its genus
holds true for an invariant quantity, like the expectation value of a
trace or multitrace of powers of the matrix. As you say, then all indices
are repeated twice and you can follow their flow, defining faces etc.
You don't have that in your case, and the counting is different…

You might discuss
with
\HREF{https://scholar.google.fr/citations?hl=en&user=w5XOZQoAAAAJ&view_op=list_works&sortby=pubdate}
{Ivan Kostov} at IPhT, who is quite knowledgeable about this matrix stuff.

   \item[2023-04-21 Ruilin Shi]
 <rshi49@gatech.edu> GaTech math oral comprehensive exam:

Tur{\'a}n problem on planar graphs was first studied by Dowden\rf{Dowden15},
\arXiv{1512.04385},
in 2015. Dowden defined the planar Tur\'an number $\ex_{\mathcal{P}}(n,H)$ of
a graph $H$ to be the maximum number of edges in an $n$-vertex planar graph
without $H$ as a subgraph. It follows from Euler's formula that
$\ex_{\mathcal{P}}(n, C_3) = 2n - 4$ for all $n \ge 3$. Dowden proved that
$\ex_{\mathcal{P}}(n, C_4) \le \frac{15(n-2)}{7}$ for all $n \ge 4$ and
$\ex_{\mathcal{P}}(n, C_5) \le \frac{12n - 33}{5}$ for all $n \ge 11$, and
showed that in both cases equality holds infinitely often. In 2020, Ghosh,
Gy\H ori, Martin, Paulos, and Xiao\rf{GGMPX20} \arXiv{2004.14094} proved that
$\ex_{\mathcal{P}}(n, C_6) \le \frac{5n}{2} - 7$ with equality holding
infinitely often. They further conjectured that $\ex_{\mathcal{P}}(n,
C_{\ell}) \le \frac{3(\ell - 1)}{\ell}n - \frac{6(\ell + 1)}{\ell}$ for all
$\ell \ge 7$ and all sufficiently large $n$. We prove that
$\ex_{\mathcal{P}}(n,C_7) \le \frac{18n}{7} - \frac{48}{7}$ for all $n > 38$,
and show that equality holds for infinitely many integers $n$.

$k$-cycle $C_{k}$ is a cycle of length $k$
\\
$k$-face is a face bounded by a $k$-cycle
\\
$k$-path is a path with $k$ edges

\arXiv{2212.02668} says: In 1884 Tait conjectured that every cubic
3-connected planar graph is hamiltonian.
The Petersen graph is the smallest
non-planar 3-connected cubic graph which is not hamiltonian.
Not every cubic 3-connected bipartite graph is hamiltonian,
the Horton graph is a counterexample.

Barnette's Conjecture:
Every 3-connected cubic planar bipartite graph is hamiltonian.

\arXiv{2209.14799} counts random cubic planar graphs.

\arXiv{2202.00592}:
Cubic planar graphs are 3-regular graph that admit a crossing-free embedding in
the plane.

   \item[2023-05-19 Silvia Pappalardi]
I was reading a review \arXiv{0911.0087} and ...
see comment to Theorem 22.4.2, p.~7 ... Speicher knew! (at least
in 2009), but maybe you were already aware:)

\begin{quote}
``Free probability theory was created by Dan Voiculescu around 1985,
motivated by his efforts to understand special classes of von Neumann
algebras. His dis- covery in 1991 that also random matrices satisfy
asymptotically the freeness relation transformed the theory
dramatically.''

``In a sense some aspects of this theory of freeness were anticipated
(but mostly neglected) in the physics community in the Cvitanovi\'c 1982
paper\rf{NPB82}.''
\end{quote}

I hope your cumulants were able to be free.


   \item[2024-02-11 Predrag]
\HREF{https://www.epflepf.com/adam/}
{Adam W. Marcus} (no longer active?):
\HREF{https://www.epflepf.com/adam/talks/ff_talk.pdf}
{{\em Polynomials and (finite) free probability}}.

A. W. Marcus, {\em Finite free point processes} (2022),
\arXiv{2205.00495}.


\end{description}


%\newpage %%%%%%%%%%%%%%%%%%%%%%%%%%%%%%%%%%%%%%%%%%%%%%%%
\printbibliography[heading=subbibintoc,title={References}]
