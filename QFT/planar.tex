% reducesymm/QFT/planar.tex
% Predrag  switched to github.com               jul  8 2013

\chapter{Planar field theory}
\label{c-planar}

\begin{description}

\item[2013-01-20  Predrag]
Lucini and Panero\rf{LucPan13} \CBlibrary{LucPan13}
might be of interest. All I get is one sentence and a reference only
to \refref{PlanFieldThe}.

I should also read Kang and Loebl\rf{KanLoe09}
{\em The enumeration of planar graphs via {Wick}'s theorem}.

\item[2018-06-06 Predrag]

\HREF{https://www.hep.phys.soton.ac.uk/content/james-drummond}
{James M. Drummond} {\em Cluster algebras and scattering amplitudes},
\arXiv{1710.10953}, was a stunning performance, a picture of which I'll
post on Flickr.com. Analytic relations for scattering amplitudes
expressed the cluster algebras for planar $N=4$ super Yang-Mills theory
that generalize the Steinmann relations.  I still do not know what hit
me.

\item[2019-12-06 Predrag]
Penington, Shenker, Stanford and Yang\rf{PSSY19}
{\em Replica wormholes and the black hole interior},
\arXiv{1911.11977}:
Their analysis is inspired by the ``free probability'' results discussed
in 2009 Speicher\rf{Speicher09}, {\em Free probability theory},
\arXiv{0911.0087} and \arXiv{1404.3393}, and e.g. figure 1 of the (Predrag's)
earlier\rf{PlanFieldThe}, from 1981.
Speicher explain the work of Voiculescu\rf{Voiculescu91}: ``Free probability theory was
created by Dan Voiculescu around 1985, motivated by his efforts to
understand special classes of von Neumann algebras.'' He writes about
Cvitanovi{\'c}, Lauwers and Scharbach\rf{NPB82} {\em The planar sector of
field theories}: ``This description of freeness in terms of free
cumulants is related to the planar approximations in random matrix theory.
In a sense some aspects of this theory of freeness were anticipated (but
mostly neglected) in the physics community in the \refref{NPB82} paper.''
Voiculescu does not seem to cite me.

Se also mathoverflow
\HREF{https://mathoverflow.net/questions/94028/classical-convolution-vs-free-convolution}
{Classical convolution vs. free convolution}
and Terrence Tao's
\HREF{https://terrytao.wordpress.com/2010/02/10/245a-notes-5-free-probability/\#more-3466}
{non-commutative probability} notes.

Smith\rf{Smith92} {\em Planar version of {Baym-Kadanoff} theory}

Djordje Mini\'c, {\em Remarks on Large N Coherent States}
\arXiv{hep-th/9502117}: `` Recently it has become apparent that another
algebraic structure seems to be natural for the large N matrix models,
the so called free Fock space. This concept naturally appears within the
context of non-commutative probability theory, developed by Voiculescu
and collaborators\rf{Voiculescu91}, and it has been used in the physics
literature in the analysis of large N matrix models by Haan\rf{Haan80} and
Cvitanovi\'c and co-workers\rf{NPB82}, and more recently by
Douglas\rf{Douglas95} and Gopakumar and Gross\rf{GopGro95} {\em Mastering
the master field}. (Similar ideas have been exploited in the study of so
called infinite statistics by Greenberg\rf{Greenberg91}).
''

Haan writes `` In 1980 I used these operators to define ``planar fields''
for the non-interacting field theory\rf{Haan80}'' (received 16 June
1980), which puts him three months ahead of me\rf{PlanFieldThe} (received
21 October 1980).

Douglas\rf{Douglas95} {\em Stochastic master fields} writes: ``We treat
the stochastic equation for large N master fields proposed by Greensite
and Halpern using a construction of master fields modelled after work of
Voiculescu, and show that it contains the same information as the usual
factorized Schwinger-Dyson equations. We comment on the relation to
earlier work of \rf{Haan80} and of Cvitanovi\'c, Lauwers and
Scharbach\rf{NPB82}.

``The problem of the construction of the master field has been discussed in
many works [...] by using methods of non-commutative
probability theory. In \refref{AreVol96}
it was shown that the masterfields satisfy to standard equations of
relativistic field theory but fields are quantized according to a new
rule. These fields have a realization in the free (Boltzmannian)
Fock space.''

Araki and Tanii\rf{AraTan96}
{\em {Ward-Takahashi} identities in large {N} field theories}
use my formalism to derive the {Ward-Takahashi} identity,
their eq.~(24)

Ebrahimi-Fard and Patras\rf{EbrPat16}
{\em The combinatorics of {Green}'s functions in planar field theories}:
`` In the early 1980s, Cvitanovi\'c \etal\rf{PlanFieldThe,NPB82} proposed
a perturbative approach to quantum field theories in the planar setting.
This was largely motivated by a desire to properly encode the behavior of
the planar sector of quantum chromodynamics (QCD), based on 't Hooft's
seminal 1974 paper\rf{tHooft:1974jz}. An interesting feature of planar
field theories is the manner in which the calculus of symmetry factors
differs (and becomes simpler) compared to classical field theories.
Planarity is reflected in the strictly non-commutative nature of the
theory, which results in a rather substantial deviation from the
classical description of the relations between different types of Green's
functions. Cvitanovi\'c \etal\ observed that the functional relation
between the generating functionals of the full and connected planar
Green's functions is encoded by a fixed point type equation, which is
solved by the generating functionals. This fixed point equation replaces
the common exponential map that relates the generating functionals of the
full and connected Green's functions in classical theories (in this
article, classical will refer to non-planar field theories and their
associated objects, such as Green's functions, Feynman diagrams, and
amplitudes).

The exponential relation between classical generating functionals is
analogous to the moments-cumulants relation in classical probability
theory\rf{Speed83}. Singer realized the existence of a similar connection
between planar field theories and Voiculescu's theory of free
probability\rf{Voiculescu91}. It turns out that the description given by
Cvitanovi\'c \etal\  of the relations between planar Green's functions is
closely related to Speicher's combinatorial approach to the relations
between moments and cumulants in free probability\rf{Speicher09}.

The description of the relations between planar Green's functions
presented in these notes is based on our recent work on the algebraic and
combinatorial structures underlying the relations between moments and
cumulants in free and classical probability theory\rf{EbrPat15}
[...]
It turns out that in both cases the linear fixed point equation has a
proper exponential solution. In the classical case, this exponential
solution coincides with the standard exponential that relates classical
moments and cumulants. In the non-classical setting, the relation between
free cumulants and moments is also portrayed by an exponential, which is
defined with respect to a non-commutative shuffle product. The difference
between these two exponentials is analogous to the difference between
exponential solutions of scalar- and matrix-valued non-autonomous linear
differential equations. Here, we propose a similar approach to
the–Hopf–algebraic understanding of the relations between full and
connected Green’s functions in planar QFT.

[...] the polynomial expressions giving full Green's functions in terms
of connected ones constitute a multivariate generalization of the
classical Bell polynomials that relate, among others, moments and
cumulants in classical probability\rf{EbLuMa14}.

[...]  A precise description of the combinatorial nature of the recursive
structure that is on display here will be elaborated on below, in terms
of a double tensor Hopf algebra equipped with a non-cocommutative
unshuffle coproduct.

''

\end{description}


%\newpage %%%%%%%%%%%%%%%%%%%%%%%%%%%%%%%%%%%%%%%%%%%%%%%%
\printbibliography[heading=subbibintoc,title={References}]
