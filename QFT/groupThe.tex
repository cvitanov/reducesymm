% reducesymm/QFT/groupThe.tex
% Predrag  created              Aug 7 2014
% notes for birdtracks.eu

\chapter{Group theory blog}
\label{c-groupThe}


Enter here notes of general group-theoretic interest, perhaps
for inclusion into revisions of \wwwgt.

\section{Notes on Tai PhD thesis}
\label{s-groupTheBlog}

\HREF{http://www.math.upenn.edu/~mtai/}
{Matthew Tai}'s 2014 PhD thesis\rf{TaiThesis,Tai13}
{\em Family algebras and the isotypic components of $g \bigotimes g$}
(PhD adviser
\HREF{http://www.math.upenn.edu/~kirillov/}
{Alexandre A. Kirillov}\rf{Ki00,Ki01}, of
 Institute for Information Transmission Problems, Russian Academy of Sciences)
appears to supersede the Casimir and many other discussions of {\wwwgt}.
My 2014-10-17 letter to Tai, mtai@math.upenn.edu:

Dear Matthew

Rumors of my death are exaggerated, so I always wonder about why nobody
tells me anything about advances related to my work? Here you are, my
best birdtracks student, and we have not even been introduced?

Anyway, I've started writing down some notes on your thesis in GitHub,
\\
\HREF{https://github.com/cvitanov/reducesymm}
{GitHub.com/cvitanov/reducesymm}
sect.~{\em Notes on Tai PhD thesis}. For anything technical, please do
not email me, but edit directly into the GitHub version, and let me know
by email to \texttt{dasgroup@mail.gatech.edu} when you have \texttt{git
push}ed something new to the server. Here are a few notes, from the first
superficial reading. We can meet to discuss face to face anything any
time on Skype or Google Hangouts.

\begin{enumerate}
  \item
Should I write in {\wwwgt} that chapter ? is superseded by your thesis?
  \item
With an eye on revising {\wwwgt}:
which sections of the thesis in particular I should I study?
  \item
Do you have some clever way of generating your diagrams?
  Mine were all drawn by hand, using xfig.
  Do you want to contribute any of the scripts/programs to {\wwwgt} 'extras'?
  \item
why no link to {\wwwgt}?
  \item
ending lines with white dots rather than symmetrizers on external lines
is clever. (but I would not know how to do that if there are internal
symmetrizers and or several symmetrizer in the same diagram)
  \item
any errors, typos, etc. in {\wwwgt} I should fix?
  \item
I wonder where I got the `Pfaffian' from (in your discussion of $D_r /
SO(2k)$). I have no recollection - you happen to know a good reference?
  I should add Pfaffian to the index.
  \item
`The degrees of the primitive Casimir operators' or
`exponents' are the (Betti numbers-1). Compare
my  {\em Table 7.1 Betti numbers for the simple Lie groups}
with  Tai {\em Table 10.1 Exponents for the exceptional Lie algebras}.
``The name `exponents' comes from the exponents of the hyperplane
arrangement corresponding to the simple reflection planes of the Weyl
Group of the Lie algebra. The exponents can also be considered
topologically [...] also have representation-theoretic interpretations''
  \item
Can you contribute your thesis \texttt{*.bib} to {\wwwgt}?
  \item
for $G_2$, should I check Pieter Mostert unpublished paper?
  \item
for $F_4$, I should check 'Albert algebra' (related to
\HREF{http://www.ams.org/journals/bull/1974-80-06/S0002-9904-1974-13622-0/}
{Albert} of
{\wwwgt} ref.~[70] C. W. Curtis\rf{Curtis1963} ...?)
  \item
My 'defining rep', 'fundamental 1-box Young tableaux representation'
or `defining $n$-dimensional rep' is 'reference representation'
or `standard representation'.
  \item
  \item typos
  \begin{itemize}
    \item[p. 23] Clebsche vertices
%    \item[p. ?]
%    \item[p. ?]
  \end{itemize}
\end{enumerate}



\section{Daily group theory blog}
\label{s-groupTheBlog}



\begin{description}

\item[2013-02-22 PC] Fomin and Pylyavskyy\rf{FomPyl12}
{\em Tensor diagrams and cluster algebras}, {\arXiv{1210.1888}},
is a major orgy in birdtracking. Should study it some day.

\item[2014-07-20 PC] More birdtracking - they construct orthogonal
projection operators: Keppeler and
    Sj{\"o}dahl\rf{KeppSjo14}, {\em Hermitian {Young} operators},
    Sj{\"o}dahl\rf{Sjodahl13,Sjodahl13a} {\em Tools for calculations in
    color space}, and his student Thor{\'e}n\rf{Thoren14}.

Gu, J. and Jockers\rf{GuJock14}.

More birdtracking still:
Kol and Shir\rf{KolShir14} {\em Color structures and permutations}.

\item[2014-12-02 PC] More birdtracking:
Costa and Hansen\rf{CosHan14},
{\em Conformal correlators of mixed-symmetry tensors}

\item[2014-08-07  Predrag]
Phil Morrison told me that Birkhoff discovered $G_2$ while looking for
hidden symmetries of the heat kernel. Did not find the reference (it is
in his thin book on statistical mechanics?) but this is of interest:

Ilka Agricola\rf{Agricola08} writes:
``
In a talk delivered in Leipzig on June 11, 1900, Friedrich
Engel\rf{Engel1900} gave the first public account of his newly discovered
description of the smallest exceptional Lie group $G_2$, and he wrote in
the corresponding note to the Royal Saxonian Academy of Sciences:
Moreover, we hereby obtain a direct definition of our 14-dimensional
simple group $G_2$ which is as elegant as one can wish for. Indeed,
Engel's definition of $G_2$ as the isotropy group of a generic 3-form in
7 dimensions is at the basis of a rich geometry that exists only on
7-dimensional manifolds,
''

This is precisely how I think of $G_2$, so should give Engel and
Reichel credit, and
cite Agricola\rf{Agricola08}.

One immediately discovers that Baez and Huerta\rf{BaeHue14} think of
$G_2$ as a small ball rolling on a big ball, and so on; with this one can
be infinitely distracted.

\item[2015-12-02  Predrag]
Should one study
Klink and Wickramasekara\rf{KliWic15},
{\em Relativity, Symmetry and the Structure of Quantum Theory I} ?
They say: ``The history of how quantum mechanics was developed is a
fascinating one and underlies the focus of this book; namely, given the
rules that the founders of quantum mechanics developed, is it possible to
find principles that lead to the structure of quantum mechanics as it was
historically formulated? This is the first book in a series of works
considering what particular relativity is applicable to a given dynamical
theory. The series considers Newton, Einstein, and de Sitter
relativities, while this book examines the unitary irreducible
representations of the Galilei group and see how they provide the
framework for Galilean quantum theory.
''

\item[2015-09-14 BJ]
I at last have maybe some hint on the difference and on the common origin
of our two distinct magic triangles. I asked P. Deligne for advice but not
very diplomatically I enquired about his relation with your work so total
silence was the answer.
I discussed since with P. Vogel and was wondering whether you had made
progress. My breakthrough comes out of nowhere so it takes time to
decipher.
I suggested to Vogel he should go beyond powers of the adjoint
representation (your idea for the triangle) I probably should do it if it
has not been done before...

Answered 2016-02-06 by dasgroup@gatech.edu email,
complaining that `` I do not know why Deligne is so unwilling to cite my
work that precedes his - he has known about it forever
(\HREF{http://birdtracks.eu/extras/Deligne96.pdf} {click here}). I
have never met him. I had once been in Vogel's office (about 10 years
ago?). The desk was covered by birdtracks :) I'm a bit annoyed that that
things that I did decades before Vogel and are now credited to them, and
I get cited only for Vogletracks, not Vogelsong. But I do have a tenured
job and freedom to do anything, so who cares. I too find Deligne's paper
more beautiful than my own formulation. It's just that I did it first,
but apparently if you are not a "mathematician" it does not count.

BJ response:
I visited Vogel about 9 months ago about his (super)group manifold and
his version of the E line! He is trying to disprove a conjecture of
Deligne in this field.

Here is the general idea of my program, nothing has been published.
The theory of second order Painlev\'e differential ODE is classical. The
definition of difference Painlev\'e equations is not uniformly agreed upon
however but the main one due to Sakai does indeed suggest that my En
series (equivalently that of the Del Pezzo's) correspond to so called
q-difference equations whereas yours (I mean the A0 A1 A2 end) is closer to
those diagrams appearing in additive difference Painlev\'e equations but F4
and G2 are missing today. I urged the experts to look for them and one more
branch appears with A3 symmetry as an important piece. Furthermore I am
trying to clarify  the delicate connection(s) to (continuous) differential
equations and c-Painlev\'e VI takes a leading role there only.
A very small workshop: One day would cover discretizations of
symmetries and of spaces, one day for the characteristic properties of
Painlev\'es (what makes them polyquitous) and one day for algebraic geometry
(curves appear in string theory but also surfaces ...Del Pezzo's...),
Calabi-Yau's etc... but confinement of singularities and the algebraic
entropy criterion of Viallet et al. is very relevant, you may know about
it from another side, QRT maps etc...)

\item[2016-02-07  Predrag]
Never heard of them until today, but there is huge literature on
\HREF{https://en.wikipedia.org/wiki/Del_Pezzo_surface} {Del Pezzo
surfaces}.


\item[2015-11-11 Bruce Westbury]
I know you no longer work on ``birdtracks" but I thought you might be
interested in my preprint\rf{Westbury15} {\em Extending and quantising
the Vogel plane}. \\
(Answered 2016-02-06 by dasgroup@gatech.edu email,
complaining that ``I'm a bit annoyed that that things that I did decades
before Vogel and Deligne are now credited to them, and I get cited only
for Vogletracks, not Vogelsong.'').

\item[2016-02-08 Bruce Westbury]
I did try to understand spinors but felt I did not succeed. I posted my
attempt in \arXiv{1007.2579}.

I think the problem with the E7 (and also the E6, F4 D4 series) is that
it does not actually form a series. There is nothing written on this but
this does seem to be the consensus. When I say it is not a series, I
mean you end up with a finite list (so no parameter). This is known to
happen for the G2 series. I wrote an account of this in
\arXiv{1011.6197}

However, I still believe the E8 series exists as Deligne conjectured and
am currently working on proving this. This seems to me to be crucial.
\\
(Answered 2016-02-08 by dasgroup@gatech.edu email,
complaining that ``
My only bone is that I do not get how the $E_8$ series that I computed in
a 1977 preprint and officially published in 1981 gets to be called
``Deligne conjecture'' from 1996 on. I explain the history in detail in
birdtracks.eu section 21.2 A BRIEF HISTORY OF EXCEPTIONAL MAGIC. They
have recomputed the same formulas, they know of my work, they mention
that they this ``well known to physicists (cf. Cvitanovi´c [83]),'' and
that's it. The total credit for what was a wonderful breakthrough for me.
That's neither nice nor professional. Oh, screw them :)
'')

\item[2016-02-06  Predrag] Khudaverdian and Ruben Mkrtchyan\rf{KhuMkr16},
{\em Universal volume of groups and anomaly of {Vogel}'s symmetry},
write: ``
We show that integral representation of universal
volume function of compact simple Lie groups gives rise to six analytic
functions on $CP^2$, which transform as two triplets under group of
permutations of Vogel's projective parameters.  This substitutes
expected invariance under permutations of universal parameters by more
complicated covariance.

 We provide an analytical continuation of these functions and particularly
calculate their change  under  permutations of parameters.  This last
relation is universal generalization, for an arbitrary simple Lie group
and an arbitrary point in Vogel's plane, of the Kinkelin's reflection
relation on Barnes' $G(1+N)$ function. Kinkelin's relation gives asymmetry
of the $G(1+N)$ function (which is essentially the volume function for $SU(N)$
groups)  under $N\leftrightarrow -N$ transformation (which is  equivalent
of the permutation of parameters, for $SU(N)$ groups), and coincides with
universal relation on permutations at the $SU(N)$ line on Vogel's plane.
These results are also applicable to universal partition function of
Chern-Simons theory on three-dimensional sphere.

This effect is  analogous to modular covariance, instead of invariance,
of partition functions of appropriate gauge theories under modular
transformation of couplings.
''

Mkrtchyan is not spring chicken. The funny thing is that, while there are
legions of young Witteninos, all this work seem to be carried out by old
men.

\item[2015-12-02  Predrag]
Email to Ruben:

I have not been working on these problems for a while, so apologies if
everything I write about here is something you already know. We have
uncovered more $N \to -N$ relationships than just $SO(-n)=Sp(n)$
(\HREF{http://birdtracks.eu/refs/index.html}{click here}).

In {\em Spinors in negative dimensions},
Phys. Scripta 26, 5 (1982) Tony Kennedy and I did it for ``spinsters".

In {\em Negative dimensions and $E_7$ symmetry}, Nucl. Phys. B188, 373
(1981) (as well as in Chapter 20. {\em E7 family and its
negative-dimensional cousins} of the birdtracks book) I have a
negative-dimension mapping where $E_7$  appears as a
negative-dimensional relative of $SO(4)$.

That might be of interest for your Vogel song:)

BTW, `Parizi' $\to$ `Parisi'


\end{description}
\renewcommand{\ssp}{a}
