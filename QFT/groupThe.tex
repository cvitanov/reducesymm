% reducesymm/QFT/groupThe.tex
% Predrag  created              Aug 7 2014
% notes for birdtracks.eu

\chapter{Group theory blog}
\label{c-groupThe}


Enter here notes of general group-theoretic interest, perhaps
for inclusion into revisions of \wwwgt.

\section{Notes on Tai PhD thesis}
\label{s-groupTheBlog}

\HREF{http://www.math.upenn.edu/~mtai/}
{Matthew Tai}'s 2014 PhD thesis\rf{TaiThesis,Tai13}
{\em Family algebras and the isotypic components of $g \bigotimes g$}
(PhD adviser
\HREF{http://www.math.upenn.edu/~kirillov/}
{Alexandre A. Kirillov}\rf{Ki00,Ki01}, of
 Institute for Information Transmission Problems, Russian Academy of Sciences)
appears to supersede the Casimir and many other discussions of {\wwwgt}.
My 2014-10-17 letter to Tai, mtai@math.upenn.edu:

Dear Matthew

Rumors of my death are exaggerated, so I always wonder about why nobody
tells me anything about advances related to my work? Here you are, my
best birdtracks student, and we have not even been introduced?

Anyway, I've started writing down some notes on your thesis in GitHub,
\\
\HREF{https://github.com/cvitanov/reducesymm}
{GitHub.com/cvitanov/reducesymm}
sect.~{\em Notes on Tai PhD thesis}. For anything technical, please do
not email me, but edit directly into the GitHub version, and let me know
by email to \texttt{dasgroup@mail.gatech.edu} when you have \texttt{git
push}ed something new to the server. Here are a few notes, from the first
superficial reading. We can meet to discuss face to face anything any
time on Skype or Google Hangouts.

\begin{enumerate}
  \item
Should I write in {\wwwgt} that chapter ? is superseded by your thesis?
  \item
With an eye on revising {\wwwgt}:
which sections of the thesis in particular I should I study?
  \item
Do you have some clever way of generating your diagrams?
  Mine were all drawn by hand, using xfig.
  Do you want to contribute any of the scripts/programs to {\wwwgt} 'extras'?
  \item
why no link to {\wwwgt}?
  \item
ending lines with white dots rather than symmetrizers on external lines
is clever. (but I would not know how to do that if there are internal
symmetrizers and or several symmetrizer in the same diagram)
  \item
any errors, typos, etc. in {\wwwgt} I should fix?
  \item
I wonder where I got the `Pfaffian' from (in your discussion of $D_r /
SO(2k)$). I have no recollection - you happen to know a good reference?
  I should add Pfaffian to the index.
  \item
`The degrees of the primitive Casimir operators' or
`exponents' are the (Betti numbers-1). Compare
my  {\em Table 7.1 Betti numbers for the simple Lie groups}
with  Tai {\em Table 10.1 Exponents for the exceptional Lie algebras}.
``The name `exponents' comes from the exponents of the hyperplane
arrangement corresponding to the simple reflection planes of the Weyl
Group of the Lie algebra. The exponents can also be considered
topologically [...] also have representation-theoretic interpretations''
  \item
Can you contribute your thesis \texttt{*.bib} to {\wwwgt}?
  \item
for $G_2$, should I check Pieter Mostert unpublished paper?
  \item
for $F_4$, I should check 'Albert algebra' (related to
\HREF{http://www.ams.org/journals/bull/1974-80-06/S0002-9904-1974-13622-0/}
{Albert} of
{\wwwgt} ref.~[70] C. W. Curtis\rf{Curtis1963} ...?)
  \item
My 'defining rep', 'fundamental 1-box Young tableaux representation'
or `defining $n$-dimensional rep' is 'reference representation'
or `standard representation'.
  \item
  \item typos
  \begin{itemize}
    \item[p. 23] Clebsche vertices
%    \item[p. ?]
%    \item[p. ?]
  \end{itemize}
\end{enumerate}



\section{Daily group theory blog}
\label{s-groupTheBlog}



\begin{description}
\item[2014-08-07  Predrag]
Phil Morrison told me that Birkhoff discovered $G_2$ while looking for
hidden symmetries of the heat kernel. Did not find the reference (it is
in his thin book on statistical mechanics?) but this is of interest:

Ilka Agricola\rf{Agricola08} writes:
``
In a talk delivered in Leipzig on June 11, 1900, Friedrich
Engel\rf{Engel1900} gave the first public account of his newly discovered
description of the smallest exceptional Lie group $G_2$, and he wrote in
the corresponding note to the Royal Saxonian Academy of Sciences:
Moreover, we hereby obtain a direct definition of our 14-dimensional
simple group $G_2$ which is as elegant as one can wish for. Indeed,
Engel's definition of $G_2$ as the isotropy group of a generic 3-form in
7 dimensions is at the basis of a rich geometry that exists only on
7-dimensional manifolds,
''

This is precisely how I think of $G_2$, so should give Engel and
Reichel credit, and
cite Agricola\rf{Agricola08}.

One immediately discovers that Baez and Huerta\rf{BaeHue14} think of
$G_2$ as a small ball rolling on a big ball, and so on; with this one can
be infinitely distracted.

\item[2015-12-02  Predrag]
Should one study
Klink and Wickramasekara\rf{KliWic15},
{\em Relativity, Symmetry and the Structure of Quantum Theory I} ?
They say: ``The history of how quantum mechanics was developed is a
fascinating one and underlies the focus of this book; namely, given the
rules that the founders of quantum mechanics developed, is it possible to
find principles that lead to the structure of quantum mechanics as it was
historically formulated? This is the first book in a series of works
considering what particular relativity is applicable to a given dynamical
theory. The series considers Newton, Einstein, and de Sitter
relativities, while this book examines the unitary irreducible
representations of the Galilei group and see how they provide the
framework for Galilean quantum theory.
''



\end{description}
\renewcommand{\ssp}{a}
