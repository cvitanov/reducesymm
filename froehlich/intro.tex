%           %experimenting with svn-multi

\svnkwsave{$RepoFile: siminos/froehlich/intro.tex $}
\svnidlong {$HeadURL$}
{$LastChangedDate$}
{$LastChangedRevision$} {$LastChangedBy$}
\svnid{$Id$}

    \ifarticle
\begin{abstract}
    \else
\chapter{Continuous symmetry reduction by the \mslices}
\label{chap:CLF}
\begin{quote}
{\bf Abstract:}

    \fi %end of article switch

When a dynamical system has a continuous symmetry, it is possible to exploit this symmetry to reduce the system to an equivalent simpler system. One method for doing this is \mslices. In this paper we investigate how the \mslices\ can be applied to linear subspaces. There are two main obstacles to using a subspace for the \mslices: the slice has to intersect every group orbit in order to be valid for the entire {\statesp} and the \mslices\ potentially introduces singularities into the flow. We show that any point in the {\statesp} can be rotated into these linear subspaces, guaranteeing they can be used for the entire {\statesp} and that singularities introduced into the system by the method of slices correspond to simple jumps in the reduced space and do not cause any actual difficulties. Throughout this paper we focus on $\SOn{2}$ symmetries, using the \cLe\ as a simple example. In addition we show that if the symmetry is a product of $\SOn{2}$ symmetries acting on distinct coordinates of the {\statesp}, then it is sufficient to consider each $\SOn{2}$ action independently.
    \PC{\color{red}
Stefan, write this: often! this might be the only part
of this text that most people glance at.
%PC 2010-09-30: planted an error into the abstract, just to see how
%   often do you edit it.
	}
     \PC{
   When you write
   a project report or a research article, you always write abstract, introduction
   and conclusions first, and then keep rewriting them often.
   They are the most important parts of the text, as that is
   for most people only parts they will look at.
   }
    %
    \Private{ % subversion label pages
$\footnotemark\footnotetext{{\tt \svnkw{RepoFile}}, rev. \svnfilerev:
 last edit by \svnFullAuthor{\svnfileauthor},
 \svnfilemonth/\svnfileday/\svnfileyear}$
    } % end \Private{

    \ifarticle
\end{abstract}
    \else
\end{quote}
    \fi %end of article switch



\section{Introduction}
\label{sect:intro}

Many papers have been written about using symmetries to help understand different fluid flows. This has been done for fluid flows such as the \KS\ flow\rf{ku,siv}, the {\pCf}\rf{Visw07b,GHCW07,HGC08,HalcrowThesis}, and a pressure driven flow through a cylindrical pipe\rf{Wk04,Kerswell05}. These are  that experience turbulent flow. The hope is that understanding these relatively `simple' systems will provide a better understanding of turbulence in general.

Rotational and translational symmetries appear in many fluid flows. An example is a cylindrical pipe\rf{Wk04,Kerswell05}. If you rotate the pipe around its axis or translate it, the shifted and rotated state of the fluid is also a solution of the Navier-Stokes equations. These rotations and translations of the pipe are an example of a continuous symmetry.

Several methods exist for exploiting a continuous symmetry of a system. The most prominent method is to rewrite the system in terms of a Hilbert basis for the symmetry. The {\statesp} coordinates are replaced by polynomials that are invariant under the action of the symmetry group. The polynomials are chosen to form a basis for the space of all invariant polynomials but are related by relations called syzygies\rf{DasBuch}. Hilbert bases work very well for low dimensional systems, but the number of basis polynomials and the difficulty of determining them greatly increase with the dimension of the system, making it infeasible to calculate them for the high (possibly infinite) dimensional flows encountered in fluid systems. The \mslices\rf{CartanMF} provides a less computationally intensive method for these high dimensional systems. In the \mslices\ the system is replaced with an equivalent system on a subspace of the \statesp. This method is already being employed to reduce symmetries in various fluid flows\rf{SiminosThesis,rowley_reconstruction_2000,rowley_reduction_2003}.

One simple choice of the subspace is a hyperplane. This paper will investigate the use of hyperplanes for the \mslices\ with an emphasis on $\SOn{2}$ symmetries since these appear in many fluid flows (specifically the \KS, {\pCf}, and pipe flows mentioned earlier). These linear slices have already been used in \refref{SiminosThesis,rowley_reconstruction_2000,rowley_reduction_2003} to reduce the dynamics of fluid flows. There are two main obstacles to being able to use a subspace for the \mslices: every point must be rotatable into the slice and the \mslices\ introduces singularities into the flow. Locally any point is rotatable into a linear slice, and \refref{rowley_reconstruction_2000,rowley_reduction_2003} demonstrate that this is true globally for certain {\statesp}s. This paper provides a more general proof that is applicable to more {\statesp}s than that of \refref{rowley_reconstruction_2000,rowley_reduction_2003}.

The second difficulty with the \mslices\ is that it can introduce singularities into the flow. One method of handling this is to proceed as in \refref{SiminosThesis} and try to place a slice in such a manner that these singularities do not occur. This can lead to complicated subspaces being used for the symmetry reduction. While linear slices will in general experience these singularities, we show that in the case of a $\SOn{2}$ symmetry, and in general for `well-behaved' symmetries, these singularities correspond to nothing more than an instantaneous jump in the trajectory in the linear slice that is easily calculated.

Many fluid flows have symmetry groups which are the product of $\SOn{2}$ symmetry groups (in particular the \KS, {\pCf}, and cylindrical pipe flows) acting on distinct coordinates of the {\statesp}. We demonstrate that when using linear slices for these symmetry groups it is sufficient to consider each $\SOn{2}$ action separately.

Throughout this paper the \cLe\rf{GibMcCLE82} will be used as an example because they posses a simple $\SOn{2}$ symmetry for which the \mslices\ is easily implemented. The {\statesp} of the \cLe\ is only 5-dimensional, but the method is the same regardless of dimension and to gain an understanding of how it works it suffices to use this simple system.

Working with symmetries requires some background and the beginning part of this paper is devoted to building up the knowledge necessary to understand what a symmetry is and how it can be used. We follow this by proving some important results for linear slices and then close with a short section on how the \mslices\ can be used to find invariants of a symmetry. This paper draws heavily from the teachings of Chaosbook.org\rf{DasBuch} and makes use of its notation.

\bigskip
\noindent {\bf Acknowledgments.}
%This report is written in collaboration with
% % E.~Siminos and
%P.~Cvitanovi\'c.
S.F. work was supported by the National Science Foundation
grant DMR~0820054 and a Georgia Tech President's Undergraduate
Research Award.
P.C. thanks Glen Robinson Jr. for support. 	



