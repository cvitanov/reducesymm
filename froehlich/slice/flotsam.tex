% siminos/froehlich/slice/flotsam.tex  called by FrCv11.tex
% $Author$ $Date$

\section{Flotsam}
\label{sec:flotsam}

\noindent \textbf{PC 2010-12-11~~}
This text is clipped from 2010-06-07 siminos/CLE/CLE.tex as is. Use parts of it:

``
An inconvenience inherent in the linear slices formulation is that they are
local, and the reduced flow encounters singularities in
subsets of the \reducedsp, with the reduced trajectory exhibiting
large, slice-induced jumps.
This \sset\ is introduced by and
depends on the \slice-fixing condition. We have shown,
in the $5$-dimensional \cLe\ example, that the location of
the \sset\ can be manipulated by judicious choice of the slice
fixing point, and geometrical information about the dynamics can
be extracted by constructing a return map through a
\Poincare\ section that does not intersect the singular set.
The trick is to
construct a good set of symmetry in\-vari\-ant Poincar\'e
sections, and that is a dark art \edit{for systems of
dimension higher than three}, with or without a
symmetry.
In higher-dimensional flows, with more involved symmetry
group actions and larger sets of stationary solutions, where a
single slice and \Poincare\ section will not suffice, we can
still expect to cover the \reducedsp\ with multiple slices,
obtaining a set of discrete maps involving multiple
\Poincare\ sections. As to high-dimensional applications,
it was shown in \refref{SiminosThesis} that the coexistence
of four equilibria, two \reqva\ and a nested \fixedsp\
structure in an effectively $8$-dimensional \KS\
system\rf{SCD07} complicates matters considerably. This
application of symmetry reduction to a spatially extended,
PDE system is the
subject of a forthcoming publication\rf{SCD09b}.
''
