% file siminos/froehlich/slice/frame.tex
% $Author$ $Date$

% \section{\Mframes}
%       \label{sec:frame}

Suppose you are observing turbulence in a pipe flow, or your
defibrillator has a mesh of sensors measuring electrical currents that
cross your heart, or you have a precomputed pattern, and are sifting
through the data set of observed patterns for something like it. Here you
see a pattern, and there you see a pattern that seems much like the first
one. How `much like the first one?' Think of the first pattern
(represented by a point {\slicep} in the \statesp\  \pS) as a
`template'\rf{rowley_reconstruction_2000,rowley_reduction_2003} or a
`reference state' and use the symmetries of the flow to slide and rotate
the `{\template}' until it overlays the second pattern (a point $\ssp$ in
the \statesp), \ie, act with elements of the symmetry group \Group\ on
the {\template} ${\slicep} \to {\LieEl}{\slicep}$ until the distance between
the two patterns
\beq
|\ssp - {\LieEl}{\slicep}|
    = |\sspRed - \slicep|
\label{minDistance0}
\eeq
is minimized. Here $\sspRed$ is the point on the group orbit of $\ssp$
(the set of all points that $\ssp$ is mapped to under the groups
actions),
\beq
\ssp=\LieEl \sspRed
	\,,\qquad
\LieEl \in \Group
\,,
\ee{sspOrbit}
closest to the {\template} {\slicep}.
This distance needs to be an invariant of the symmetry group: here
we shall assume \Group\ is a subgroup of the group of Euclidean
transformations, and measure distance
$|\ssp|^2=\braket{\ssp}{\ssp}$ in terms of the Euclidean inner product
\( %beq
\braket{x}{y} = \sum_i^d {x}_i y_i
%    \,,\; %\qquad
% x, y \in \pS \subset \reals^d
	\,,
\) %\ee{innerR}
or, if the \statesp\ is a normed function space,
\( %\beq
\braket{g}{f} = \int dx \, {g}(x) f(x)
\,,
\) %\ee{innerL2}
in terms of the $L^2$ norm $|f|^2 = \braket{f}{f}$. In practice, we
represent such spatially extended functions by discrete meshes or finite
basis sets, within a (possibly large) finite-dimensional \statesp\  $\pS
\subset \reals^d$. An example is a representation of a dissipative PDE by
truncating the Fourier basis \refeq{FourierExp} to a finite number of
modes.

As \Group\ is a symmetry of the system and the length $|\ssp|$ is
invariant under symmetry operations, $\Group$ is a subgroup of the group
of orthogonal transformations $\On{d}$, and its Lie algebra {generators}
$\Lg_a$ (see \refeq{FiniteRot}) are a set of $N$ linearly independent
$[d\!\times\!d]$ antisymmetric matrices acting linearly on the {\statesp}
vectors $\ssp \in \pS \subset \reals^d$. If we parameterize a Lie group
element $\LieEl=\LieEl(\gSpace)\propto\exp{({\gSpace} \cdot \Lg)}$ by
parameters $\gSpace = (\gSpace_1,\gSpace_2,\cdots\gSpace_N)$, the minimal
distance satisfies the extremum conditions
\beq
\frac{\partial ~~}{\partial \gSpace_a} |\ssp - \LieEl\slicep|^2
   =
\braket{\sspRed - \slicep}{\sliceTan{a}}
   = 0
    \,,\qquad
	  \sliceTan{a} = \Lg_a \slicep
\,.
\label{PCsectQ}
\eeq
Thus the minimum distance condition, combined with the Euclidean norm, says
that the point $\sspRed$ in the group orbit of \statesp\ point $\ssp$
closest to the {\template} $\slicep$ lies in a hyperplane
through the point $\slicep$, normal to the group action tangent space
$\sliceTan{}$.

By the antisymmetry of the Lie algebra generators we have
$\braket{\slicep}{\sliceTan{a}} = \braket{\slicep}{\Lg_{a}\slicep}=0$, so
we can replace $\sspRed - \slicep \to \slicep$ in \refeq{PCsectQ}, and
the transformation parameters $\gSpace$ which map the state $\ssp$ to
$\sspRed$, the group orbit point
closest to the {\template} $\slicep$, are determined solely by being
normal to the $N$ {\template} tangent vectors $\sliceTan{a}$,
\beq
\braket{\sspRed}{\sliceTan{a}} =0
    \,,\qquad
\sspRed = \LieEl(\gSpace) \ssp
\,.
\ee{PCsectQ0}
The group orbit points closest to the {\template} thus lie in a
$(d\!-\!N)$\dmn\ hyperplane $\pSRed = \pS/\Group$, the set of vectors
$\sspRed \in  \pSRed$ orthogonal to the {\template} tangent space spanned
by tangent vectors $\{\sliceTan{1},\cdots,\sliceTan{N}\}$
\beq
\sspRed_1\sliceTan{a,1} + \sspRed_2\sliceTan{a,2}
  + \cdots + \sspRed_d\sliceTan{a,d} = 0
\,.
\ee{hyperpl}
In what follows we shall refer to this hyperplane as a
\emph{slice,} and to  \refeq{PCsectQ0} as the \emph{slice conditions}.
Slice so defined is a particular case of symmetry reduction
by transverse sections of group
orbits\rf{FelsOlver98,FelsOlver99,OlverInv} that can be traced back to
Cartan's \mframes\rf{CartanMF}. `Moving frame' refers to the action
$\LieEl(\gSpace)$ that
brings a \statesp\ point \ssp\ into the slice.

In the choice of the {\template} $\slicep$ one should avoid solutions
that belong to the invariant or partially symmetric subspaces; for such
choices $\Lg_{a}\slicep=0$, and some or all $\sliceTan{a}$ vanish
identically and impose no slice conditions. The {\template} $\slicep$
should be a generic \statesp\ point in the sense that its group orbit has
the full $N$ dimensions of the group \Group. In particular, even though
the simplest solutions (laminar, \etc) often capture important physical
features of a flow, most \eqva\ and short \po s have nontrivial
symmetries and thus are not suited as choices of symmetry-reducing
templates.

It should be emphasized is that a {\template} is \emph{not} a
{localized} structure. We are not using translations / rotations to
superimpose a localize, `solitonic' solution over a solitonic {\template}. In
a strongly nonlinear, turbulent flow a good {\template} is typically a
nontrivial global solution.

\subsection{Computing the moving frame rotation angle}
\label{exam:CLErotAngle}

															\toCB
The idea of reducing a flow with Lie-group structure to a system of
a smaller dimension dates back to Lie, Poincar\'e, and Cartan.
Time-evolution and symmetry group actions foliate the \statesp\ into
$(N\!+\!1)$\dmn\ submanifolds: Given a state (a \statesp\ point $\ssp(0)$
at time $\tau=0$), we can trace its  1\dmn\ trajectory $\ssp(\tau)$ by
integrating its equations of motion, and its $N$\dmn\ group orbit by
acting on it with the symmetry group \Group. Locally, a continuous time
flow can be reduced to the set of neighboring points $\ssp(0)$ by a
\PoincSec; a slice does the same for local neighborhoods of group orbits.
{\em Symmetry reduction} by the \mframes\ is a precise rule for
how to pick a unique point \sspRed\ for each such equivalence class, and
compute the transformation $\ssp =\LieEl \sspRed$ that relates the full
\statesp\ point  $\ssp \in \pS$, to its symmetry reduced representative
$\sspRed \in \pSRed$.

To show how the rotation into the \slice\ is computed, consider first the
\cLe. The \reducedsp\ point $\sspRed$ is given by
$\ssp=\LieEl(\gSpace) \sspRed$ where $\gSpace$ is such that
$\braket{\sspRed}{\sliceTan{}}=0$. Substituting the \SOn{2}\ Lie algebra
generator and a finite angle \SOn{2} rotation \refeq{CLfRots} acting on a
5\dmn\ \statesp\ into the slice condition \refeq{PCsectQ0}
yields
\(\braket{\ssp}{\sliceTan{}}\cos\gSpace
    +\braket{\groupTan_{}(\ssp)}{\sliceTan{}} \sin\gSpace
= 0
\,,
\)
and the explicit formula for $\gSpace$:
\bea
\tan\gSpace &=&
   - \, {\braket{\ssp}{\sliceTan{}}}/
          {\braket{\groupTan_{}(\ssp)}{\sliceTan{}}}
\,.
\label{SL:CLEsliceRot}
\eea
The dot product of two tangent fields in \refeq{SL:CLEsliceRot} is a
sum of inner products weighted by Casimirs \refeq{QuadCasimir},
\beq
\braket{\groupTan(\sspRed)}{\groupTan(\slicep)}
   = \sum_m C_2^{(m)} {\sspRed}_i\, \delta_{ij}^{(m)} \slicep_j
\,.
\ee{braket}
For \cLe\
$\ssp = (x_1,x_2,y_1,y_2,z)$,
$\sspRed = (x_1',x_2',y_1',y_2',z)$,
and the moving frame condition \refeq{SL:CLEsliceRot} yields
\beq
\tan\gSpace =
- \, \frac{x_1 x_2'-x_2 x_1'+y_1 y_2' -y_2 y_1'}
       {x_1 x'_1+x_2 x'_2+y_1 y'_1+y_2 y'_2}
\,.
\ee{braketCL}
This formula is particularly simple, as in the \cLe\
example the group acts only through $m=0$ and $m=1$ representations
(in the Fourier mode labeling of \refeq{SO2irrepAlg-Lg}).

Consider next the general form \refeq{SO2irrepAlg-m} of action of an
$\SOn{2}$ symmetry on arbitrary Fourier coefficients of a spatially
periodic function \refeq{FourierExp}. Substituting this into the slice
condition \refeq{PCsectQ0} and using \refeq{SO2irrepAlg-m} we find that
    \PC{RECHECK! and prettify this formula}
\bea
\braket{e^{\gSpace \Lg}\ssp}{\groupTan(\slicep)}
=\braket{\ssp}{\sum\limits_m \left(\cos(-m\gSpace) \id^{(m)}
     +\sin(-m\gSpace) \frac{1}{m}\Lg^{(m)}\right) \sliceTan{}}
\continue
=\sum\limits_m
   \left(
    \braket{\ssp}{\Lg^{(m)} \slicep} \cos(m\gSpace)
  - \braket{\ssp}{\id^{(m)} \slicep} \sin(m\gSpace)
   \right)
   =0
\,.
\label{eq:so2sing}
\eea
This is a polynomial equation, with coefficients determined by
$\braket{\ssp}{\Lg^{(m)} \slicep}$ and $\braket{\ssp}{\id^{(m)}\slicep}$,
as we can see by rewriting $\cos(m\gSpace)$, $\sin(m\gSpace)$ as
polynomials of degree $m$ in $\sin(\gSpace)$ and $\cos(\gSpace)$. The
phase $\gSpace$ that rotates $\ssp$ into any of the group-orbit
traversals of the slice hyperplane corresponds to a real root of this
polynomial. In general these phases have to be computed numerically.

As a generic group orbit is a smooth $N$\dmn\ manifold embedded in the
$d$\dmn\ \statesp, several values of $\gSpace$ might be local extrema of
the distance function \refeq{PCsectQ}.
Our prescription is to pick the closest \reducedsp\ point as the unique
representative of the entire group orbit. \ie, determine the absolute
minimum or the \emph{infimum} of \refeq{minDistance0}.
For example, group orbits of
\SOn{2}\ are topologically circles, and the distance function
\refeq{minDistance0} has maxima, minima and inflection points as extrema:
if \gSpace\ is a solution of the slice condition \refeq{SL:CLEsliceRot},
so is $\gSpace+\pi$. We can pick the closest by noting that
the local minima have positive curvature,
\beq
\frac{\partial^2}
     {\partial \gSpace^2}
        |\sspRed - \slicep|^2
    =
-  \braket{\sspRed}{\Lg^2\slicep}
\,.
\ee{SO2inflPoint}
It does not matter
whether the group is compact, for example $\SOn{n}$, or noncompact, for
example the Euclidean group $E_2$ that underlies the generation of spiral
patterns\rf{Barkley94}; in either case any group orbit has one or several
locally closest passages to the {\template} state, and generically only
one that is the closest one.
(Here we focus only on continuous symmetries - discrete symmetries that
flows such as the \KS\ and {\pCf} exhibit will also have to be taken into
account\rf{SCD07,HGC08,DasBuchMirror}.)

So we do not have to compute all zeros of the slice conditions
\refeq{eq:so2sing} - all we care about is the infimum, the zero that
corresponds to the shortest distance \refeq{minDistance0}.
While post-processing of a full \statesp\ trajectory $\ssp(\tau_j)$
requires a numerical (Newton method) determination of the
`moving frame' rotation
$\gSpace(\tau_j)$ at each time step $\tau_j$, the computation is not
as onerous as it might seem, as the knowledge of $\gSpace(\tau_j)$ and
$\groupTan(\sspRed(\tau_j))$
gives us a very good guess for $\gSpace(\tau_{j+1})$. We
can go a step further, and write the equations for the flow restricted to
the \reducedsp\ \pSRed.

\subsection{Dynamics within a slice}
\label{sec:mslices}

Any \statesp\ trajectory can be written in a factorized
form $\ssp(\tau)=\LieEl(\tau)
\sspRed(\tau)$. Differentiating both sides with respect to time and
setting $\velRed={d\sspRed}/{d\tau}$ we find
\(
\vel(\ssp)=\dot{\LieEl} \, \sspRed+\LieEl \, \velRed(\sspRed)
\,.
\)
By the equivariance \refeq{eq:FiniteRot}
\[
\vel=\velRed + \LieEl^{-1} \, \dot{\LieEl} \, \sspRed
\,.
\]
Noting that $\LieEl^{-1}\dot{\LieEl}=e^{-\gSpace \cdot \Lg} \,
\frac{d ~~}{d \, \tau} e^{\gSpace \cdot \Lg}=\dot{\gSpace}\cdot \Lg$,
we obtain the equation for the velocity of the reduced flow:
\beq
\velRed(\sspRed)=\vel(\sspRed)-\dot{\gSpace}(\sspRed)\cdot \groupTan(\sspRed)
\,.
\ee{eq:redVel}
The velocity $\vel$ in the full \statesp\ is thus the sum of the
`angular' velocity \refeq{PC:groupTan1} along the group orbit,
$\dot{\gSpace} \cdot \groupTan(\ssp)$, and the remainder $\velRed$.

Eq. \refeq{eq:redVel} is true for any factorization
$\ssp=\LieEl \sspRed$, and by itself provides no
information on how to calculate $\dot{\gSpace}$. That is attained by
demanding that the reduced trajectory stays within a slice. Let
$\sliceTan{a}$ be the {\template} tangents.
Slice conditions \refeq{PCsectQ0} that the flow is confined to the slice
yield
\beq
\braket{\vel(\sspRed)}{\sliceTan{a}}
 -\braket{\dot{\gSpace}\cdot \groupTan(\sspRed)}{\sliceTan{a}}=0
\,.
\label{eq:slicecondition}
\eeq
This is a matrix equation in
$\braket{\groupTan_b(\sspRed)}{\sliceTan{a}}$ that one can in
principle\rf{FiSaScWu96} solve for $\dot{\gSpace}_a$. Here we shall
consider only the $\SOn{2}$ case, which has a single group tangent:
\bea
\velRed(\sspRed) &=& \vel(\sspRed)
   -\dot{\gSpace}(\sspRed) \groupTan(\sspRed)
\continue
\dot{\gSpace}(\sspRed) &=& {\braket{\vel(\sspRed)}{\sliceTan{}}}/
               {\braket{\groupTan(\sspRed)}{\sliceTan{}}}
\,.
\label{eq:so2reduced}
\eea
One way to think about this reduction of a flow to a slice is in terms of
Lagrange multipliers (see {Stone and Goldbart}\rf{StGo09}, Sect 1.5 for
intuitive, geometrical interpretation of Lagrange multipliers). The first
equation defines the flow confined to the slice (see
\reffig{fig:Fullspace}\,(b)), and integration of the second,
`reconstruction' equation\rf{Marsd92,MarsdRat94} enables us to track the
corresponding trajectory in the full \statesp. For invariant subspaces
$\dot{\gSpace}=0$, so they are always included within the slice. No
information is lost about the physical flow: if we know one point on a
trajectory, we can hop at will back and forth between the reduced
and the full \statesp\ trajectories, just as we can reconstruct a
continuous trajectory from its \PoincSec s.

%
% ****** End of file frame.tex ******
