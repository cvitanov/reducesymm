% file siminos/froehlich/slice/frame.tex
% $Author$ $Date$

% \section{\Mframes}
%       \label{sec:frame}

Suppose you are observing turbulence in a pipe flow, or your
defibrillator has a mesh of sensors measuring electrical currents that
cross your heart, or you have a precomputed pattern, and are sifting
through the data set of observed patterns for something like it. Here you
see a pattern, and there you see a pattern that seems much like the first
one. How `much like the first one?' Think of the first pattern
(represented by a point {\slicep} in the \statesp\  \pS) as a
`template'\rf{rowley_reconstruction_2000,rowley_reduction_2003} or a
`reference state' and use the symmetries of the flow to slide and rotate
the `{\template}' until it overlays the second pattern (a point $\ssp$ in
the \statesp), \ie, act with elements of the symmetry group \Group\ on
the {\template} ${\slicep} \to {\LieEl}{\slicep}$ until the distance between
the two patterns
\beq
|\ssp - {\LieEl}{\slicep}|
    = |\sspRed - \slicep|
\label{minDistance0}
\eeq
is minimized. Here $\sspRed$ is the point on the group orbit of $\ssp$
(the set of all points that $\ssp$ is mapped to under the groups
actions),
\beq
\ssp=\LieEl \sspRed
	\,,\qquad
\LieEl \in \Group
\,,
\ee{sspOrbit}
closest to the {\template} {\slicep}.
This distance needs to be an invariant of the symmetry group: here
we shall assume \Group\ is a subgroup of the group of Euclidean
transformations, and measure distance
$|\ssp|^2=\braket{\ssp}{\ssp}$ in terms of the Euclidean inner product
\( %beq
\braket{x}{y} = \sum_i^d {x}_i y_i
%    \,,\; %\qquad
% x, y \in \pS \subset \reals^d
	\,,
\) %\ee{innerR}
or, if the \statesp\ is a normed function space,
\( %\beq
\braket{g}{f} = \int dx \, {g}(x) f(x)
\,,
\) %\ee{innerL2}
in terms of the $L^2$ norm $|f|^2 = \braket{f}{f}$.
In practice, we represent such spatially extended functions by discrete
meshes or truncated basis sets, with a (possibly large)
finite-dimensional \statesp\  $\pS \subset \reals^d$.

If we parameterize a Lie group element $\LieEl=\LieEl(\gSpace)$ by
parameters $\gSpace = (\gSpace_1,\gSpace_2,\cdots\gSpace_N)$, the minimal
distance satisfies the extremum conditions
\beq
\frac{\partial ~~}{\partial \gSpace_a} |\ssp - \LieEl\slicep|^2
   =
\braket{\sspRed - \slicep}{\sliceTan{a}}
   = 0
    \,,\qquad
	  \sliceTan{a} = \Lg_a \slicep
\,.
\label{PCsectQ}
\eeq
The minimum distance condition, combined with the Euclidean norm says
that the point $\sspRed$ in the group orbit of \statesp\ point $\ssp$
closest to the {\template} $\slicep$ lies in a $(d\!-\!N)$\dmn\ hyperplane
through the point $\slicep$, normal to the group action tangent space
$\sliceTan{}$. In what follows we shall refer to this hyperplane as a
\emph{slice.} Slice so defined is a particular case of symmetry reduction
by transverse sections of group
orbits\rf{FelsOlver98,FelsOlver99,OlverInv} that can be traced back to
Cartan's \mframes\rf{CartanMF}.

As \Group\ is a symmetry of the system and the
length $|\ssp|$ is invariant under symmetry operations, $\Group \subset \On{d}$,
the Lie algebra {generators} $\Lg_a$ are a set of $N$ linearly
independent $[d\!\times\!d]$ antisymmetric matrices acting linearly on
the {\statesp} vectors $\ssp \in \pS \subset \reals^d$.
By the antisymmetry of the Lie algebra generators we have
$\braket{\slicep}{\sliceTan{a}}
 = \braket{\slicep}{\Lg_{a}\slicep}=0$, and the transformation parameters
$\gSpace$ for which the state $\ssp$ is closest to the {\template}
$\slicep$ are fixed by $N$ slice conditions \refeq{PCsectQ},
\beq
\braket{\sspRed}{\sliceTan{a}} =0
    \,,\qquad
\sspRed = \LieEl(\gSpace) \ssp
\,.
\ee{PCsectQ0}


A transformation induced by infinitesimal
time-dependent variations \refeq{PC:groupTan0} of group `phases'
$\delta \gSpace_a = \timeStep \, \dot{\gSpace_a}$ is
\beq
\dot{\ssp} = \dot{\gSpace} \cdot \groupTan(\ssp)
\,.
\ee{PC:groupTan1}
So $\dot{\gSpace} \cdot \groupTan(\ssp)$ is the velocity
of the flow along the group orbit of \ssp.

As a generic group orbit is a smooth curved $N$\dmn\ manifold embedded in
the $d$\dmn\ \statesp, several values of $\gSpace$ might be local extrema
of the distance function \refeq{PCsectQ}. For example, group orbits of
\SOn{2}\ are topologically circles, and the distance function
\refeq{minDistance0} has maxima, minima and inflection points as extrema.
  %
The physically most interesting extremum is the closest one, or the
\emph{infimum}, the absolute minimum of \refeq{minDistance0}. It does not
matter whether the group is compact, for example $\SOn{n}$, or
noncompact, for example the Euclidean group $E_2$ that underlies the
generation of spiral patterns\rf{Barkley94}; in either case any group
orbit has one or several locally closest passages to the {\template}
state, and generically only one that is the closest one.

A given compact group orbit intersects a slice at least twice, and
possibly many times. Our prescription is to pick the infimum distance
\reducedsp\ point as the unique representative of the entire group orbit.

In what follows we denote by `\mframes' the post-processing of the full
\statesp\ flow (\refsect{exam:CLErotAngle}), and by `\mslices' the
integration of flow confined to the \reducedsp\ (\refsect{sec:mslices}).
 The symmetry-reduction induced singularities are
artifacts of the reduction method, more tractable numerically if given
the full \statesp\ trajectory.
In practice, symmetry reduction is best carried out as post-processing,
after the numerical trajectory is obtained by integrating the full
\statesp\ flow.

\subsection{Computing the moving frame rotation angle}
\label{exam:CLErotAngle}

To show how the rotation into the \slice\ is computed, consider first the
\cLe. By \refeq{PCsectQ0} the \reducedsp\ trajectory is given by
$\sspRed=\LieEl(\gSpace) \ssp$ where $\gSpace$ is such that
$\braket{\sspRed}{\sliceTan{}}=0$. Substituting the \SOn{2}\ Lie algebra
generator \refeq{CLfLieGen} and  a finite angle \SOn{2} rotation
\refeq{CLfRots} acting on a 5\dmn\ space \refeq{eq:CLeR} into the slice
condition \refeq{PCsectQ0} yields the explicit formula for $\gSpace$:
\bea
0 &=&
    \braket{\ssp}{\sliceTan{}}\cos\gSpace
    +\braket{\groupTan_{}(\ssp)}{\sliceTan{}} \sin\gSpace
\label{SL:CLEsliceRot0}\\
\tan\gSpace &=&
   - \, {\braket{\ssp}/{\sliceTan{}}}
          {\braket{\groupTan_{}(\ssp)}{\sliceTan{}}}
\,.
\label{SL:CLEsliceRot}
\eea
The dot product of two tangent fields in \refeq{PCinflPoint} is thus a
sum of inner products weighted by Casimirs,
\beq
\braket{\groupTan(\sspRed)}{\groupTan(\slicep)}
   = \sum_m C_2^{(m)} {\sspRed}_i\, \delta_{ij}^{(m)} \slicep_j
\,.
\ee{braket}
An example is the Fourier series \refeq{tangL2norm}.

For \cLe\ this yields
\[
\tan\gSpace =
- \, \frac{x_1 x_2'-x_2 x_1'+y_1 y_2' -y_2 y_1'}
       {x_1 x'_1+x_2 x'_2+y_1 y'_1+y_2 y'_2}
\,.
\]
%    \PC{Stefan, I think you have to explain that, just as in the
%    case of Poincar\'e sections, one keeps track of only oriented
%    crossings, \ie, a circle has only on section in a \slice,
%    not two. That should give you a precise $\pi$ rotation rule
%    for singularity crossing.}
% PC 2010-12-16 - just pick the infimum.
Note that if \gSpace\ is a solution, so is $\gSpace+\pi$.
This formula is particularly simple, as in the \cLe\
example the group acts only through $m=0$ and $m=1$ representations.

%\noindent
%{\bf $\SOn{2}$ singularities.}
%\label{ex:so2singularities}
Consider next the general form \refeq{SO2irrepAlg-m} of action
of an $\SOn{2}$ symmetry on arbitrary Fourier coefficients of a smooth
function.
Substituting this into the slice condition \refeq{PCsectQ0} we find that
    \PC{prettify this formula}
\bea
\braket{e^{\gSpace \Lg}\ssp}{\groupTan(\slicep)}
=\braket{\ssp}{(\sum\limits_m (\cos(-m\gSpace) \id^{(m)}
     +\sin(-m\gSpace) \frac{1}{m}\Lg^{(m)})) \sliceTan{}}
\continue
=\sum\limits_m(\cos(m\gSpace) \braket{\ssp}{\Lg^{(m)} \slicep}-\sin(m\gSpace) \braket{\ssp}{\id^{(m)} \slicep}).
\label{eq:so2sing}
\eea
Rewriting $\cos(m\gSpace)$, $\sin(m\gSpace)$ as
polynomials of degree $m$ in $\sin(\gSpace)$ and $\cos(\gSpace)$, so
\refeq{eq:so2sing} is a polynomial whose coefficients are
determined by $\braket{\ssp}{\Lg^{(m)} \slicep}$ and
$\braket{\ssp}{\id^{(m)}\slicep}$. The phase $\gSpace$ corresponds to a root of
this polynomial and in general the `phases' $\gSpace_a$ have to be computed
numerically.
We do not have to compute all the roots of
this polynomial - all we care about is the infimum, or the root that
corresponds to the shortest distance \refeq{minDistance0}.


\ifboyscout
\subsubsection{Subsubsection header}
[Just checking what the header looks like]
\else
\fi



%
% ****** End of file frame.tex ******
