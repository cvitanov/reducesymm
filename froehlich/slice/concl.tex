% siminos/froehlich/slice/concl.tex  called by FrCv11.tex
% $Author$ $Date$

% \section{Conclusion}

If a physical problem has a symmetry, one should use it.
Unattended to, symmetry can be a great nuisance; deftly deployed
it can be a powerful tool in simplifying physical problems.
One does not want to compute the same solution over and over, all
one needs is one representative per the equivalence class (the group orbit).
    \PC{symmetry reduction = lowering the order (Arn Kozl Neist)}

Symmetry strongly constrains the form of solutions and their bifurcations,
if appropriately implemented, it can significantly accelerate
convergence of numerical algorithms,
it splits the dynamical \statesp\ into chains of lower-dimensional flow-invariant
subspaces, dictates its invariant partitions and the symbolic dynamics that dictates its coarse geometry.

The theory of \emph{linear} representations of symmetry groups is
a well developed, essentially completed subject. Role of symmetries in
\emph{nonlinear} settings is altogether another story.

Shift in time to the closest passage: ``close recurrence.''
    \PC{cite myself, Gibson + myself}

Many spatially extended and fluid dynamics systems exhibit continuous symmetries:
Excitable media\rf{ZaZha70,Winfree73,Winfree1980,BaKnTu90,Barkley94},
the \KS\ flow\rf{ku,siv,SCD07},
{\pCf}\rf{Visw07b,GHCW07,HGC08,HalcrowThesis}, and flow through an
cylindrical pipe\rf{Wk04,Kerswell05} are invariant under
combinations of translational (Euclidean), rotational
$\SOn{2}$ and discrete symmetries.

In this paper we investigated symmetry reduction by the
\mslices\ and answer affirmatively the main about the method:
(1) does a slice cut the group orbit of every point in the dynamical \statesp?
(2) can one deal with the global singularities that the method necessarily
induces? Locally linear slices intersect each group orbit only once,
but extended globally, a slice can intersect a group orbit multiple times.
We shown that the \reducedsp\ trajectory
passing through a {\sset} does so as a simple shift,
causing no difficulties (other than to sloppy numerics). In addition we
demonstrated that the problem of dealing with singularities of a product
of $\SOn{2}$ groups acting on different coordinates of the {\statesp}
(as is the case for the \KS\rf{ku,siv},
{\pCf}\rf{Visw07b,GHCW07,HGC08,HalcrowThesis}, and
pipe flows\rf{Wk04,Kerswell05}) is equivalent to dealing with the
symmetries of each \SOn{2}\ symmetry independently.



Even though every slice cuts all group orbits, it makes no sense physically to
use one slice
(a set of all group orbit points that are closest to a given `template')
globally. Instead we should do what we already do for KS Poincar\'e sections.
We need to make a global chart by deploying both linear slices and linear
Poincar\'e sections in neighborhoods of the most important (relative)
equilibria and/or (relative) periodic orbits (those are tricky, because
slice fixing points must lie in the full \statesp, and have no symmetry,
so most of the solutions we have are not good as they stand). This is the
periodic-orbit generalization of the idea of
\HREF{http://chaosbook.org/overheads/trace/Tesselate.jpg}{\statesp\ tessellation}
so dear to professional cyclist(s).


Boundaries
between hyperplanes are themselves hyperplanes of one dimension less,
easy to compute once we have decided on the set of slices. To find
what slice a given full \statesp\ trajectory point is in, one rotates
with respect to each slice, and checks whether the given group orbit
belongs to it. In the \reducedsp\ the trajectory is integrated within a
given slice until it hits a hyperplane boundary - then one switches to
the next slice across the boundary. Boundary corners are measure zero, no
way you would hit them.

Global chart should be sufficiently fine-grained that we never hit any
slice {\sset}. That means that the neighborhood - bounded by
intersections with neighboring slices is sufficiently small that group
tangent space is nowhere within the slice - works in smooth flows
for sufficiently small neighborhoods.


More has to be done to reduce a system with a discrete
symmetry before the \mslices\ can be used for the continuous symmetry.
