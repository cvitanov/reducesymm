% master file siminos/froehlich/slice/intro.tex 
% $Author$ $Date$

% \section{Introduction}
%    \label{sec:intro}

Suppose you are observing turbulence in a pipe flow. Here you see a
pattern, and there you see a pattern that seems much like the first one.
How ``much like the first one?'' If the dynamics of the system under study is invariant under
a group of continuous symmetries, one way of answering the question is by
measuring distances between different states in the
symmetry-reduced \statesp\ $\pS/\Group$, a space in which each group orbit (class
of physically equivalent states) is represented by a single point.
But there exist no preferred, one-size fits all
symmetry-reduction method. The literature
(see \refrefs{CBcontinuous,SiCvi10,SiminosThesis} for a review) broadly
offers two approaches (a) Hilbert invariant polynomial bases, and (b) methods which
slice group orbits much in the way that \Poincare\ sections cut across
time-evolving trajectories. The invariant polynomial approaches can use
the symmetries to recast

This paper describes  the question in a


\Mslices\ as applied to chaotic/turbulent dynamics has been
studied in \refrefs{SiminosThesis,SiCvi10,Wilczak09,CBcontinuous}.

The new results of this paper are:


%
% ****** End of file intro.tex ******
