% master file siminos/froehlich/slice/intro.tex
% $Author$ $Date$

% \section{Introduction}
%    \label{sec:intro}

In spatially extended, turbulent flows one observes
similar patterns at different spatial positions and different times.
How `similar?' If the dynamics of the system under study is invariant under
a group of continuous symmetries, one way of answering this question is by
measuring distances between different states in the
symmetry-reduced \statesp\ $\pS/\Group$, a space in which each group orbit (class
of physically equivalent states) is represented by a single point.
But there exist no preferred, one-size fits all
symmetry-reduction method. The literature
(see \refrefs{CBcontinuous,SiCvi10} for a review) broadly
offers two approaches (a) Hilbert invariant polynomial bases, and (b) methods which
slice group orbits, generalizing the way that \Poincare\ sections cut
time-evolving trajectories.
%
Our task is to formulate a computationally straightforward method of reducing
the dynamics to a lower-dimensional \statesp, where each group orbit of
the full system (\ie, set of physically equivalent states) is
represented by a single point. If successful, the methods that we develop
should be applicable to very high-dimensional flows, such as
translationally equivariant fluid flows bounded by pipes or
planes\rf{GHCW07,GibsonMovies}.
%
In this paper we argue that for high-dimensional chaotic/turbulent
nonlinear flows the \mslices\ studied in
\refrefs{SiCvi10,Wilczak09,CBcontinuous} is the only feasible,
implementable and practical technique.
Here the method is rederived as a distance minimization problem in the
space of patterns. The new results reported in this this paper are:
    (a) Any generic (linear) slice cuts across group orbits of {\em all}
        states in the \statesp.
    (b) Every slice carries along with it a {\sset}. We show how to
        compute the jump of the \reducedsp\ trajectory whenever it crosses
        through such singularity.
    (c) We avoid these singularities (artifacts of the symmetry
        reduction by linear slices) by tiling the \statesp\ with an atlas
        constructed from a set of local slices.
	(d) We show that if a continuous symmetry has a product structure
	   (such as $\SOn{2} \times \SOn{2}$ symmetries of pipe and plane
	   fluid flows), each symmetry induces its own {\sset}.

\subsection{Symmetries of dynamics}
\label{sect:SymmDyn}
	
The \cLe\
\beq
\begin{split}
	\dot{x}_1 &= -\sigma x_1 + \sigma y_1\\
	\dot{x}_2 &= -\sigma x_2 + \sigma y_2\\
	\dot{y}_1 &= (r_1-z) x_1 - r_2 x_2 -y_1-e y_2 \\
	\dot{y}_2 &= (r_1-z) x_2 + e y_1- y_2\\
	\dot{z} &= -b z + x_1 y_1 + x_2 y_2\,.
	\label{eq:CLeR}
\end{split}
\eeq
were introduced by Gibbon and McGuinness\rf{GibMcCLE82}
as a 5-dimensional model of baroclinic instability in the
atmosphere.
In all numerical calculations that follow we shall set the
parameters to \refref{SiCvi10} values,
\beq
r_1=28,\; b={8}/{3},\;
\sigma=10,\quad \mbox{and}  \quad e={1}/{10}
\,.
\ee{SiminosPrmts}
Here we are not interested in the physical applications of these
equations; rather, we use them as a simple example of a dynamical system
with a continuous (but no discrete) symmetry.

A flow $\dot{x}= \vel(x)$ is \emph{equivariant} under an operation $\LieEl$ if
\beq
\vel(x)=\LieEl^{-1}\vel(\LieEl \, x)
\,.
\ee{eq:FiniteRot}
If $\ssp(\tau)$ is a solution to the dynamical equations, then this
implies $\LieEl\,\ssp(\tau)$ is also a solution. The totality of group
elements $\LieEl$ forms  \Group\, the {\em symmetry group} of the flow.
Most of the Lie groups that we will only be considering here are
compact.
For example, in studies of the \KS\ equations\rf{ku,siv,SCD07},
{\pCf}\rf{Visw07b,GHCW07,HGC08}, and cylindrical pipe
flow\rf{Wk04,Kerswell05} the imposition of periodic boundary conditions
renders the symmetry group compact.

The \CLe\ are equivariant under rotations around the $z$-axis, with a
finite angle \SOn{2} rotation given by
\beq
\LieEl(\gSpace) \,=\,  \left(\barr{ccccc}
  \cos \gSpace  & \sin \gSpace  & 0 & 0 & 0 \\
 -\sin \gSpace  & \cos \gSpace  & 0 & 0 & 0 \\
 0 & 0 &  \cos \gSpace & \sin \gSpace   & 0 \\
 0 & 0 & -\sin \gSpace & \cos \gSpace   & 0 \\
 0 & 0 & 0             & 0              & 1
    \earr\right)
\,,
\ee{CLfRots}
and the corresponding Lie algebra generator
\beq
 \Lg \,=\,   \left(\barr{ccccc}
    0  &  1 & 0  &  0 & 0  \\
   -1  &  0 & 0  &  0 & 0 \\
    0  &  0 & 0  &  1 & 0  \\
    0  &  0 &-1  &  0 & 0 \\
    0  &  0 & 0  &  0 & 0
    \earr\right)
\,.
\ee{CLfLieGen}
The action of \SOn{2}\ thus decomposes the  \statesp\ into $m=0$
\SOn{2}-invariant subspace ($z$-axis) and  $m=1$ subspace of
multiplicity 2.
The generator $\Lg$ is anti-hermitian,
$\dual{\Lg} = - \Lg$, and the group is compact, its
elements parameterized by $\gSpace \mbox{ mod } 2\pi$. Locally, at
$\ssp \in \pS$, the infinitesimal action of the group is
given by the group tangent field $\groupTan(\ssp) = \Lg \ssp
= (x_2,-x_1,y_2,-y_1,0)$, with the flow induced by
the group action normal to the radial direction in the
$(x_1,x_2)$ and $(y_1,y_2)$ planes, while the $z$-axis is left
invariant.

\subsection{\SOn{2} irreducible representations.}
    \label{exam:SO2irrepst}
Expand a smooth periodic function $u(\gSpace + 2\pi) =
u(\gSpace)$ as a Fourier series
\beq
u(\gSpace) = \frac{a_0}{2} + \sum_{m=1}^\infty \left(
a_m \cos m \gSpace + b_m \sin m \gSpace
                               \right)
\,.
\ee{FourierExp}
The matrix representation of the \SOn{2}\ action
%DB% \refeq{SO2LieFunct}
on the $m$th Fourier coefficient pair
$(a_m,b_m)$ is
\bea
\LieEl^{(m)}(\gSpace') &=&  \left(\barr{cc}
 ~\cos m \gSpace'  & \sin m \gSpace' \\
 -\sin m \gSpace'  & \cos m \gSpace'
    \earr\right)
\continue
&=& \cos m \gSpace' \id^{(m)}
  + \sin m \gSpace'\, \frac{1}{m} \Lg^{(m)}
\,,
\label{SO2irrepAlg-m}
\eea
with the Lie group generator
\beq
 \Lg^{(m)} \,=\,   \left(\barr{cc}
    0  &  m  \\
   -m  &  0
    \earr\right)
\,.
\ee{SO2irrepAlg-Lg}
$\Lg^{(m)}$ is the Lie
algebra generator and $\id^{(m)}$ is the identity on the $m$-irreducible
subspace, 0 elsewhere.
The \SOn{2}\ group tangent $\groupTan(u)$
to \statesp\ point $u(\gSpace)$ is the sum over invariant subspace
contributions                            \toCB
\beq
 \groupTan(u) = \sum_{m=1}^\infty \groupTan^{(m)}(u)
    \,,\qquad
 \groupTan^{(m)}(u)
\,=\, m \,\left(\barr{c}
   ~b_m  \\
   -a_m
    \earr\right)
\,.
\ee{u:x:tang}
The $L^2$ norm of $\groupTan(u)$ is weighted by
the quadratic Casimir \refeq{QuadCasimir}. For \SOn{2} this is
$C_2^{(m)} = m^2$,
\beq
\oint \frac{d\gSpace}{2\pi}
     \, (\Lg u(\gSpace))^T \Lg u(2\pi-\gSpace)
= \sum_{m=1}^\infty m^2 \left(a_m^2 + b_m^2\right)
\,.
\ee{tangL2norm}
It converges only for sufficiently smooth $u(\gSpace)$.
$\Lg$ generates translations, and by \refeq{SO2irrepAlg-Lg} the
angular velocity
of the $m$th Fourier mode is $m$ times higher than for the $m=1$
component. If $| u^{(m)}| = (a_m^2+b^2_m)^{1/2}$ does not fall off faster
than $1/m$, the action of \SOn{2}\ is overwhelmed by the high Fourier
modes.

%
% ****** End of file intro.tex ******
