% master file siminos/froehlich/slice/chart.tex
% $Author$ $Date$

% \section{Charting the \reducedsp}
%       \label{sec:chart}

{\bf Tessellation of the \reducedsp\ by two or many slices:
how to implement it?}
We have to study Roweis%
	\PC{
A sobering fact: Rowais, assistant professor at N.Y.U., a young
star in the field and universally liked, jumped out of his Washington
Square apartment earlier this year.
	}
 and Saul\rf{RoSa00}
\emph{``Nonlinear dimensionality reduction by locally linear embedding.''}
Our proposal is essentially to approximately
\HREF{http://en.wikipedia.org/wiki/Voronoi_tessellation} {Voronoi
tessellate} (see \HREF{http://en.wikipedia.org/wiki/Voronoi_diagram}
{wiki on Voronoi diagrams}) the curved manifold in which the reduced
strange attractor is embedded by a finite set of hyperplanes, each a
slice tangential to one of a finite number of `reference states' or
`templates.' In computer science and linear programming a related method
is called \HREF{http://en.wikipedia.org/wiki/Vector_quantization}{`vector
quantization,'} (also called `block quantization' or `pattern matching
quantization'), a lossy data compression method where sets of points are
clustered by their distance to `prototype' or `centroid' points. The
method is `lossy,' as in the replacement of a point by its `prototype,'
one drops the residual, \ie, the Euclidean distance between the two. It
works by encoding values from a multidimensional vector space into a `a
codebook,' a finite set of values from a discrete subspace of lower
dimension. Only the index of the codeword in the codebook is sent, thus
conserving memory and increasing compression. Vector quantization
numerical algorithms are unlikely to be useful to us.

So how do we propose to implement this tessellation?



% in part clipped from
% www.cs.unm.edu/~terran/downloads/classes/cs529-s10/docs/pretest_soln.pdf
An (affine) hyperplane is the locus of points obeying the equation:
\[
\sliceTan{1}\sspRed_1 + \sliceTan{2}\sspRed_2 + \cdots + \sliceTan{d}\sspRed_d = c
\,.
\]
In vector notation $\braket{\sspRed}{\sliceTan{}{}^{(j)}}=0$. In the case
of a slice the extremal distance condition (??) sets $c=0$, so all our slices
include the point at the origin, $\sspRed=0$, and are not affine. A
$(d\!-\!1)$-dimensional hyperplane embedded in a $d$-dimensional space
and passing through the origin is defined by a normal vector \sliceTan{},
a vector orthogonal to every vector in the hyperplane (\sliceTan{} is the
primary object, the template \slicep\ secondary in this way of thinking).
The intersection of two hyperplanes is a $(d\!-\!2)$-dimensional
hyperplane - the `ridge, `boundary,' or `edge' where two hyperplanes
meet. The intersection is $(d\!-\!2)$-dimensional because every vector in
the intersection must be orthogonal to the normals of both hyperplanes.
It is a $(d\!-\!2)$-dimensional hyperplane which also goes through the
origin. If you are an ant crawling along the trajectory $\sspRed(\tau)$
symmetry-reduced with respect to the first slice, you do not need to
compute this hyperplane. Every so often you compute
$\braket{\sspRed(\tau)}{\sliceTan{}{}^{(2)}}$. As long as this number is
negative, you have not gone too far. The moment it changes sign, you the
ant have crossed ridge, we symmetry-reduce with respect to the second
slice, and the ant continues its merry stroll along the second slice. We
do not need to be very precise about the instant where we switch, as long
as we are far away form wither slice's singularity subspace, so the ridge
can be `fuzzy,' and the numerical check infrequent and cheap.

There is a rub, though - you have to figure out how to pick the phases of
different templates $\sliceTan{}{}^{(j)}$ in such way that you somehow
minimize the distance from one to the next as you cross the ridge. We
proposed in \refref{SCD07} to use heteroclinic connections for that - they
fix the relative phases of different \eqva.

I'm guided by \KS\ work with Lan, where \Poincare\ sections
are picked physically,
by neighborhoods of important solutions (2-rolls states, 3-rolls
states). The choice should be `physical,'
dictated by the dominant patterns seen in the solutions of
nonlinear PDEs.
I'll write up the proposal that follows
in the paper with Stefan\rf{FrCvi11}.

Even though every generic
slice cuts all group orbits, it makes no sense physically to
use one slice
(a set of all group orbit points that are closest to a given `template')
globally. Instead we should do what we
already do for \KS\ Poincar\'e sections. While \refref{Christiansen97} demonstrated
that \po\ theory can be applied to spatially extended systems, the
main advance of
\refref{lanCvit07} was to show that for more turbulent/chaotic systems a set
of Poincar\'e sections is needed to capture the dynamics.
A slice defined here is a purely group-theoretic, linear construct, with no reference
to dynamics; a given slice-fixing point (template) \slicep\ defines
the associated slice, a ($d\!-\!N$)-dimensional hyperplane.
Within it, there is a ($d\!-\!N\!-1$)-dimensional
singularity hyperplane
\[
\braket{\groupTan(\sspRed)}{\sliceTan{}}=0
\]
where the group tangent of a point
$\sspRed=\LieEl^{-1} \ssp$ lies in the slice. The singularity hyperplane
is also purely group-theoretical,
determined by the slice-fixing point (template) \slicep,
that pays no heed to nonlinear dynamics. If we pick another
slice-fixing point $\slicep'$, it comes along with its own slice
and singularity hyperplane. So idea is to coarsely cover the nonlinear
strange attractor with a set of hyperplanes, as in \reffig{fig:Tesselate}.
For any pair, they intersect in a 'boundary' hyperplane, of one less dimension.
So our task is to, for a given strange attractor, pick a set of slice-fixing
points, such that each is approximately tangent to the strange attractor,
and the singularity hyperplanes are eliminated by requiring that they
lie either on the `wrong' side of the slice-slice intersection, or somewhere where
the strange attractor does not tread.

We need to make a global chart by deploying both linear slices and linear
Poincar\'e sections in neighborhoods of the most important (relative)
equilibria and/or (relative) periodic orbits (those are tricky, because
slice fixing points must lie in the full \statesp, and have no symmetry,
so most of the solutions we have are not good as they stand). This is the
periodic-orbit generalization of the idea of
% \HREF{http://chaosbook.org/overheads/trace/Tesselate.jpg}
{\statesp\ tessellation}
so dear to professional cyclists, \reffig{fig:Tesselate}.

% In FrCv11.tex replace by Tesselate.png
%{Hyp} %{fig6} and {tr:fig6} in ChaosBook
%%%%%%%%%%%%%%%%%%%%%%%%%%%%%%%%%%%%%%%%%%%%%%%%%%
 \begin{figure}
 \includegraphics[width=0.35\textwidth]{f_1_08_1}
 \caption{\label{fig:Tesselate}
Smooth dynamics  (left frame) tesselated by the skeleton of
periodic points, together with their linearized neighborhoods,
(right frame).
Indicated are segments of two 1-cycles and a 2-cycle that
alternates between the neighborhoods of the two 1-cycles,
shadowing first one of the two 1-cycles, and then the other.
(From \wwwcb{}.)
  }\end{figure}
%%%%%%%%%%%%%%%%%%%%%%%%%%%%%%%%%%%%%%%%%%%%%%%%%%
%


Boundaries
between hyperplanes are themselves hyperplanes of one dimension less and
should be easy compute once we have decided on the set of slices. To find
what slice a given full \statesp\ trajectory point is in, one group-rotates
with respect to each slice, and checks whether the given group orbit
belong to it. In the \reducedsp\ the trajectory is integrated within a
given slice until it hits a hyperplane boundary - then one switches to
the next slice across the boundary. This passage does not need to
be computed very precisely, as long as the singularity
hyperplanes are kept a safe distance away. Boundary corners are measure zero, no
way you would hit them.

Global chart should be sufficiently fine-grained that we never hit any
slice singularities. That means that the neighborhood - bounded by
intersections with neighboring slices is sufficiently small that group
tangent space is nowhere within the part of the slice explored by
the strange attractor - works for smooth flows
with sufficiently small neighborhoods.


%
% ****** End of file chart.tex ******
