% master file siminos/froehlich/slice/chart.tex
% $Author$ $Date$

% \section{Charting the \reducedsp}
%       \label{sec:chart}


So far, the good news is that for a generic {\template} $\slicep$ (\ie,
any $\slicep$ whose group orbit has the full $N$-dimensions of the
symmetry group \Group), the slice hyperplane \refeq{PCsectQ} cuts across
the group orbit of {\em every} point in the full \statesp\ \pS. But is
this a useful symmetry reduction of the full \statesp? A distant pattern
that is a bad match to a given {\template} will have any number of
locally `minimal' distances, each yet another bad match. Physically it
makes no sense to use a single slice (a set of all group orbit points
that are closest to one given {\template}) globally.

Work on \KS\ and the work of Rowley and
Marsden\rf{rowley_reconstruction_2000} suggests how to proceed: it was
shown in \refrefs{lanCvit07,SCD07} that for turbulent/chaotic systems a
set of Poincar\'e sections is needed to capture the dynamics. The choice
of sections should reflect the dynamically dominant patterns seen in the
solutions of nonlinear PDEs. We propose to construct a global atlas of
the symmetry \reducedsp\ $\pS/\Group$ by deploying both linear slices and
linear Poincar\'e sections across neighborhoods of the qualitatively most
important patterns, taking care that the {\template s} chosen have no
symmetry. Each slice $\pSRed{}^{(j)}$, tangential to one of a finite
number of {\template s}  $\slicep{}^{(j)}$, provides a local chart for a
neighborhood of an important, qualitatively distinct class of solutions
(2-rolls states, 3-rolls states, \etc); together they `Voronoi'
tessellate  the curved manifold in which the reduced strange attractor is
embedded by a finite set of hyperplane
tiles\rf{rowley_reconstruction_2000,RoSa00}. This is the symmetry-reduced
generalization of the idea of {\statesp\ tessellation} by a set of
periodic-orbits, so dear to a professional cyclist,
\reffig{fig:singpass}\,(b).

So how do we propose to implement this tessellation?

The physical task is to, for a given dynamical flow, pick a set of
qualitatively distinct {\template s} whose slices are locally tangent to
the strange attractor. A `slice' is a purely group-theoretic, linear
construct, with no reference to dynamics; a given {\template}
$\slicep{}^{(1)}$ defines the associated slice $\pSRed$, a
($d\!-\!1$)\dmn\ tangent hyperplane (for simplicity, in this section we
specialize to the $\SOn{2}$ case). Within it, there is a ($d\!-\!2$)\dmn\
{\sset} \refeq{sliceSingl}. If we pick another {\template} point
$\slicep{}^{(2)}$, it comes along with its own slice and {\sset}. Any
neighboring pair of $(d\!-\!1)$\dmn\ slices intersects in a `ridge'
(`boundary,' `edge'), a $(d\!-\!2)$\dmn\ hyperplane, easy to compute. All
intersections of slices, ridges and {\sset s} contain the fixed-point
subspace $\pS_\Group$. Global atlas so constructed should be sufficiently
fine-grained that we never hit any {\sset} singularities. The {\sset}s
should be eliminated by requiring that they lie either on the far sides
of the slice-slice intersections, or elsewhere where the strange
attractor does not tread. Each `chart' or `tile,' bounded by ridges to
neighboring slices, should be sufficiently small so that {\sset} is
nowhere within the part of the slice explored by the strange attractor.

Follow an ant as it traces out a symmetry-reduced trajectory
$\sspRed{}^{(1)}(\tau)$, confined to the slice $\pSRed{}^{(1)}$. The
moment $\braket{\sspRed{}^{(1)}(\tau)}{\sliceTan{}{}^{(2)}}$ changes
sign, the ant has crossed the ridge, we symmetry-reduce with respect to
the second slice, and the ant continues its merry stroll within
$\pSRed{}^{(2)}$ slice. Or, if you prefer to track the  given full
\statesp\ trajectory $\ssp(\tau)$, you compute the moving-frame angle
with respect to each (global) slice, and check to which tile does the
given group orbit belong.

There is a rub, though - one needs to pick the phases of neighboring
{\template s} in such way that you minimize the distance from one to the
next as the ant crosses the ridge. This a reflection of the flaw inherent in use
of a linear slice globally: a slice is derived from the Euclidean
notion of distance, but for nonlinear flows the distance has to be
measured curvilinearly, along unstable
manifolds\rf{Christiansen97,DasBuch}. We nevertheless have to stick with tessellation by
linearized tangent spaces, as curvilinear charts seem computationally too
prohibitive. The {\em relative phase} between two
different \reqva\ can be fixed, as proposed in \refref{SCD07}, by the
shortest heteroclinic connection, a rigid bridge from one
neighborhood to the next.



%
% ****** End of file chart.tex ******
