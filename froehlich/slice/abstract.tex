% master file siminos/froehlich/slice/abstract.tex
% $Author$ $Date$

We study symmetry reduction of dynamical systems with
continuous symmetries by the \mslices\ (\mframes) and show that a `slice'
defined by minimizing the distance to a single generic `{\template}'
intersects the group orbit of every point in the full {\statesp}. Global
symmetry reduction by a single slice is, however, not natural for a
chaotic / turbulent flow; it is better to cover the \reducedsp\ by a set
of slices, one for each dynamically prominent unstable pattern.
Judiciously chosen, such tessellation eliminates the dynamical traversals
of the \sset\ that comes along with each slice, an artifact of using the
{\template}'s local group linearization globally. We compute the jump in
the \reducedsp\ induced by crossing a \sset. As an illustration of the
method, we reduce the $\SOn{2}$ symmetry of the \cLe.
  %
  %
\PC{{\bf to Stefan}:
write this  often! this might be the only part of this text that most
people glance at.
%PC 2010-09-30: planted an error into the abstract, just to see how
%   often do you edit it.
%
%
   When you write a project report or a research article, you always
   write abstract, introduction and conclusions first, and then keep
   rewriting them often. They are the most important parts of the text,
   as that is for most people only parts they will look at.
   }

%
% ****** End of file abstract.tex ******
