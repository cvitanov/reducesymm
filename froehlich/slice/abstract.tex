% master file siminos/froehlich/slice/abstract.tex
% $Author$ $Date$

When a dynamical system has a continuous symmetry, it is possible to
exploit this symmetry to reduce the system to an equivalent simpler
system. One method for doing this is \mslices. In this paper we
investigate how the \mslices\ can be applied to linear subspaces. There
are two main obstacles to using a subspace for the \mslices: the slice
has to intersect every group orbit in order to be valid for the entire
{\statesp} and the \mslices\ potentially introduces singularities into
the flow. We show that any point in the {\statesp} can be rotated into
these linear subspaces, guaranteeing they can be used for the entire
{\statesp} and that singularities introduced into the system by the
method of slices correspond to simple jumps in the reduced space and do
not cause any actual difficulties. Throughout this paper we focus on
$\SOn{2}$ symmetries, using the \cLe\ as a simple example. In addition we
show that if the symmetry is a product of $\SOn{2}$ symmetries acting on
distinct coordinates of the {\statesp}, then it is sufficient to consider
each $\SOn{2}$ action independently.


\ifboyscout
    {\color{red} {\bf PC to Stefan}:
write this  often! this might be the only part of this text that most
people glance at.
%PC 2010-09-30: planted an error into the abstract, just to see how
%   often do you edit it.
%
%
   When you write a project report or a research article, you always
   write abstract, introduction and conclusions first, and then keep
   rewriting them often. They are the most important parts of the text,
   as that is for most people only parts they will look at.
   }
\else
\fi


%
% ****** End of file abstract.tex ******
