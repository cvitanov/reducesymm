% master file siminos/froehlich/slice/FrCv11.tex    pdflatex FrCv11
% $Author: predrag $ $Date: 2010-11-17 08:30:53 -0500 (Wed, 17 Nov 2010) $
%%   based on edits of aiptemplate.tex AIP REVTeX4 Ver. 4.1, 9 Oct 2009.
% Predrag, created the first draft				2010-11-21

                        %% logical setup, no need to edit %%%%%%%%%%
                        \newif\ifpaper \newif\ifPDF               %%
                        \newif\ifboyscout  \newif\ifarticle       %%
                        \boyscouttrue %% commented, WWW/boyscouts %%
                        \articletrue 							  %%
                        \paperfalse\PDFtrue %% hyperlinked    %%%%%%
    % Toggle between draft and non-draft versions
%\boyscoutfalse                 % public, for hyperlinked ChaosBook/projects
%\papertrue\boyscoutfalse     % article for submission


\documentclass[aip,cha,graphicx]{revtex4-1}
%\documentclass[aip,reprint]{revtex4-1}
	% \documentclass[aip,cha, refers to Chaos journal
%\draft %obsolete, invoke option instead
% marks overfull lines with a black rule on the right
\usepackage[pdftex]{graphicx}
\usepackage{array}
\usepackage[pdftex,colorlinks]{hyperref}
\input ../../inputs/editsDasbuch   %% editing comments, DasBuch style
\input def            %% edited, initially from dasbuch/book/inputs/def.tex
\input ../../inputs/defsFroehlich     %% all Stefan edits: \renewcommand, etc
\graphicspath{{../../figs/}{../../Fig/}}  %% directories with color graphics files
\hypersetup{
   pdfauthor=Stefan Froehlich and Predrag Cvitanovic,
   pdfkeywords=complex Lorenz flow,
   pdftitle=Reducing continuous symmetries}

\begin{document}
\title{Desymmetrization and its discontents}

% repeat the \author .. \affiliation  etc. as needed
% \email, \thanks, \homepage, \altaffiliation all apply to the current author.
% Explanatory text should go in the []'s,
% actual e-mail address or url should go in the {}'s for \email and \homepage.
% Please use the appropriate macro for the type of information

% \affiliation command applies to all authors since the last \affiliation command.
% The \affiliation command should follow the other information.

\author{Stefan Froehlich}
%\email[]{Your e-mail address}
%\thanks{}
%\altaffiliation{}
%\affiliation{}

\author{Predrag Cvitanovi\'{c}}
\email[]{predrag@gatech.edu}
%\homepage[]{Your web page}
%\thanks{}
%\altaffiliation{}
\affiliation{Center for Nonlinear Science,
        School of Physics, Georgia Institute of Technology,
        Atlanta, GA 30332-0430}

\date{\today}

\begin{abstract}
% insert abstract here
\end{abstract}

\pacs{
02.20.-a, 05.45.-a, 05.45.Jn, 47.27.ed
% 02.20.-a  Group theory, mathematics
% 05.45.-a 	Nonlinear dynamics and chaos
% 05.45.Jn 	High-dimensional chaos
% 47.10.Fg 	Dynamical systems methods (in Fluid Mechanics)
% 47.27.ed 	Dynamical systems approaches (turbulent flows)
% 47.52.+j 	Chaos in fluid dynamics
	}

\maketitle %must follow title, authors, abstract and \pacs

% Body of paper goes here.
\section{Introduction}
\label{sec:intro}

Even every slice cuts all group orbits, it makes no sense physically to
use one slice
(a set of all group orbit points that are closest to a given `template')
globally. Instead we should do what we already do for KS Poincar\'e sections.
We need to make a global chart by deploying both linear slices and linear
Poincar\'e sections in neighborhoods of the most important (relative)
equilibria and/or (relative) periodic orbits (those are tricky, because
slice fixing points must lie in the full \statesp, and have no symmetry,
so most of the solutions we have are not good as they stand). This is the
periodic-orbit generalization of the idea of
\HREF{http://chaosbook.org/overheads/trace/Tesselate.jpg}{\statesp\ tessellation}
so dear to professional cyclist(s).


Boundaries
between hyperplanes are themselves hyperplanes of one dimension less and
should be easy compute once we have decided on the set of slices. To find
what slice a given full \statesp\ trajectory point is in, one rotates
with respect to each slice, and checks whether the given group orbit
belong to it. In the \reducedsp\ the trajectory is integrated within a
given slice until it hits a hyperplane boundary - then one switches to
the next slice across the boundary. Boundary corners are measure zero, no
way you would hit them.

Global chart should be sufficiently fine-grained that we never hit any
slice singularities. That means that the neighborhood - bounded by
intersections with neighboring slices is sufficiently small that group
tangent space is nowhere within the slice - works in smooth flows
for sufficiently small neighborhoods.

\subsection{}
\subsubsection{}

% If in two-column mode, this environment will change to single-column format so that long equations can be displayed.
% Use only when necessary.
%\begin{widetext}
%$$\mbox{put long equation here}$$
%\end{widetext}

% Use the graphics or graphicx packages.
%
% Here is an example of the general form of a figure:
% \begin{figure}
% \includegraphics{}%
% \caption{\label{}}%
% \end{figure}

% Tables may be be put in the text as floats.
% Here is an example of the general form of a table:
% Fill in the caption in the braces of the \caption{} command. Put the label
% that you will use with \ref{} command in the braces of the \label{} command.
% Insert the column specifiers (l, r, c, d, etc.) in the empty braces of the
% \begin{tabular}{} command.
%
% \begin{table}
% \caption{\label{} }
% \begin{tabular}{}
% \end{tabular}
% \end{table}

\section{Conclusion}
\input concl

\begin{acknowledgments}
Authors are grateful to
D.~Barkley,
W.-J.~Beyn,
K.A.~Mitchell,
B.~Sandstede,
R.~Wilczak,
and in particular E.~Siminos and R.L.~Davidchack
for many spirited exchanges.
S.F. work was supported by the National Science Foundation
grant DMR~0820054 and a Georgia Tech President's Undergraduate
Research Award.
P.C. thanks Glen Robinson Jr. for support. 	
\end{acknowledgments}

% Create the reference section using BibTeX:
\bibliography{../../bibtex/siminos}

\end{document}
%
% ****** End of file aiptemplate.tex ******
