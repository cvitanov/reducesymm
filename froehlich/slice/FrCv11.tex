% master file siminos/froehlich/slice/FrCv11.tex    pdflatex FrCv11
% $Author: predrag $ $Date: 2010-11-17 08:30:53 -0500 (Wed, 17 Nov 2010) $
%%   based on edits of aiptemplate.tex AIP REVTeX4 Ver. 4.1, 9 Oct 2009.
% Predrag, created the first draft				2010-11-21

                        %% logical setup, no need to edit %%%%%%%%%%
                        \newif\ifpaper \newif\ifPDF               %%
                        \newif\ifboyscout  \newif\ifarticle       %%
                        \boyscouttrue %% commented, WWW/boyscouts %%
                        \articletrue 							  %%
                        \paperfalse\PDFtrue %% hyperlinked    %%%%%%
    % Toggle between draft and non-draft versions
%\boyscoutfalse                 % public, for hyperlinked ChaosBook/projects
%\papertrue\boyscoutfalse     % article for submission


\documentclass[aip,cha,graphicx]{revtex4-1}
%\documentclass[aip,reprint]{revtex4-1}
	% \documentclass[aip,cha, refers to Chaos journal
%\draft %obsolete, invoke option instead
% marks overfull lines with a black rule on the right
\usepackage[pdftex]{graphicx}
\usepackage{array}
\usepackage[pdftex,colorlinks]{hyperref}
\input ../../inputs/editsDasbuch   %% editing comments, DasBuch style
\input def            %% edited, initially from dasbuch/book/inputs/def.tex
\input ../../inputs/defsFroehlich     %% all Stefan edits: \renewcommand, etc
\graphicspath{{../../figs/}{../../Fig/}}  %% directories with color graphics files
\hypersetup{
   pdfauthor=Stefan Froehlich and Predrag Cvitanovic,
   pdfkeywords=complex Lorenz flow,
   pdftitle=Reducing continuous symmetries}

\begin{document}
\title{Desymmetrization and its discontents}

% repeat the \author .. \affiliation  etc. as needed
% \email, \thanks, \homepage, \altaffiliation all apply to the current author.
% Explanatory text should go in the []'s,
% actual e-mail address or url should go in the {}'s for \email and \homepage.
% Please use the appropriate macro for the type of information

% \affiliation command applies to all authors since the last \affiliation command.
% The \affiliation command should follow the other information.

\author{Stefan Froehlich}
%\email[]{Your e-mail address}
%\thanks{}
%\altaffiliation{}
%\affiliation{}

\author{Predrag Cvitanovi\'{c}}
\email[]{predrag@gatech.edu}
%\homepage[]{Your web page}
%\thanks{}
%\altaffiliation{}
\affiliation{Center for Nonlinear Science,
        School of Physics, Georgia Institute of Technology,
        Atlanta, GA 30332-0430}

\date{\today}

\begin{abstract}
% insert abstract here
\end{abstract}

\pacs{
02.20.-a, 05.45.-a, 05.45.Jn, 47.27.ed
% 02.20.-a  Group theory, mathematics
% 05.45.-a 	Nonlinear dynamics and chaos
% 05.45.Jn 	High-dimensional chaos
% 47.10.Fg 	Dynamical systems methods (in Fluid Mechanics)
% 47.27.ed 	Dynamical systems approaches (turbulent flows)
% 47.52.+j 	Chaos in fluid dynamics
	}

\maketitle %must follow title, authors, abstract and \pacs

\section{Introduction}
\label{sec:intro}

Suppose you are observing turbulence in a pipe flow, or your
defibrillator has a mesh of sensors measuring electrical currents that
cross your heart. Here you see a pattern, and there you see a pattern
that seems much like the first one. How ``much like the first one?'' Or
you have a precomputed pattern, and are sifting through the turbulent
data set for something like it. Think of the first pattern (represented
by a point {\slicep} in the \statesp) as a
`template'\rf{rowley_reconstruction_2000,rowley_reduction_2003} or a
`reference state' and use the symmetries of the flow to slide and rotate
the `template' until it overlays the second pattern, \ie, act with elements of
the symmetry group \Group\ on the template \statesp\ point ${\slicep} \to
{\LieEl}{\slicep}$ until the distance between the two patterns, measured
in a bilinear norm of \refsect{def:innerProduct},
%    \PC{recheck, is the unitary case different,
%	(one does need $2 \Re \,\braket{\sspRed}{\slicep}$).
%	If so, restrict consideration to subgroups of \SOn{n}.}
\bea
|\ssp - {\LieEl}{\slicep}|^2
    &=& |\sspRed - \slicep|^2
		\,=\,
   \braket{\sspRed - \slicep}{\sspRed - \slicep}
%\\ |\ssp|^2 - 2\,\Re\braket{\sspRed}{\slicep} +|\slicep|^2
\,,
    \label{minDistance}
\eea
is minimized. Here $\sspRed$ is the point on the group orbit
of $\ssp$,
\beq
\ssp=\LieEl \sspRed
\,,
\ee{sspOrbit}
closest to the template. Rather than blindly sliding one pattern over the
other, we determine the group rotation `angles' $\gSpace =
(\gSpace_1,\gSpace_2,\cdots\gSpace_N)$ by solving the extremum condition
	\PC{to select minima, need 2nd partials}
\bea
0 &=&
\frac{\partial ~~}{\partial \gSpace_a} |\ssp - \LieEl\slicep|^2
	\continue
  &=& \braket{\sspRed - \slicep}{\sliceTan{a}}
    \,,\qquad
	  \sliceTan{a} = \Lg_a \slicep
\,,
\label{PCsectQ}
\eea
where $\ssp \in \pS$ is a point in the full \statesp, and  the group
elements \refeq{FiniteRot} of Lie group $\Group$ are represented  by
$\LieEl=\exp(\gSpace \cdot \Lg)$.
By the antihermiticity of $\Lg$, \refeq{antiHerm},  we have
$\braket{\slicep}{\sliceTan{a}}=0$, and the transformation parameters
$\gSpace$ for which the state $\ssp$ is closest to the template
$\slicep$ are fixed by $N$ slice conditions
\beq
\braket{\sspRed}{\sliceTan{a}} =0
    \,,\qquad
\sspRed = \LieEl(\gSpace) \ssp
\,.
\ee{PCsectQ0}

A given compact group orbit intersects a slice at least twice, and
possibly many times, so we need a prescription for how to
pick a unique \reducedsp\ point as the representative of the entire group
orbit.

The extremal distance condition \refeq{PCsectQ1} is a condition that
closest point $\sspRed$ in the group orbit of $\ssp$ lies in a
$(d\!-\!N)$-dimensional hyperplane normal to the group action tangent
space $\sliceTan{}$ at template point $\slicep$. If the two patterns are
close, their group orbits will be nearly parallel, and for a smooth flow
this tangent plane will be transverse to all group orbits of $\ssp$ in a
neighborhood of \slicep.

However, we will be bolder, and show next that for a generic template $\slicep$ (\ie, any $\slicep$ whose group orbit is $N$-dimensional), the slice hyperplane \refeq{PCsectQ1} cuts across the group orbit of {\em every} point in the full \statesp\ \pS.

    \PC{reuse this somewhere: "
If $\ssp$ is the invariant subspace \refeq{def:centralizer}, its group
orbit is itself and the distance $|\sspRed - \slicep|$ is fixed.
    "}

As a generic group orbit is a curved $N$-dimensional manifold embedded in
the \statesp, several values of $\gSpace$ might be local extrema of the
distance function \refeq{minDistance}; we only care about those that are
local {\em minima}, for which all the eigenvalues of tensor
\refeq{PCinflPoint} are positive. The physically most interesting one is
presumably the closest one, the absolute minimum of \refeq{minDistance}.
It does not matter whether the group is compact, for example $\On{n}$, or
noncompact, for example the Euclidean group that underlies the generation
of spiral patterns\rf{Barkley94}; in either case any group orbit has
one or several locally closest passages to the template state.
    \PC{recycle this:
Here we describe symmetry reduction by the
{\em {\mslices}} of
Cartan\rf{CartanMF,FelsOlver98,FelsOlver99,OlverInv}.
    }

For
example, group orbits of \SOn{2}\ are topologically circles, and their
projections have maxima, minima and inflection points as extrema.
In order to ensure that we are looking at local minima, we will have to
check the local curvature, \ie, the eigenvalues and eigenvectors of the
symmetric matrix $[N\!\times\!N]$ matrix of second derivatives
of distance,
\beq
\frac{\partial^2}
     {\partial \gSpace_a \partial \gSpace_b}
        |\sspRed - \slicep|^2
    =
%  - \braket{\Lg_a e^{\gSpace \cdot \Lg} \ssp}{\sliceTan{b}}=
  - \braket{\groupTan_a(\sspRed)}{\sliceTan{b}}=
  \braket{\sspRed}{\dual{\Lg}{}_a {\Lg}{}_b\slicep}
\ee{PCinflPoint}


The distance surface $|\sspRed - \slicep|$ can have inflection points,
What role do they play? They are non-generic, but if we consider distance
to local minima at successive instants of a time-evolving trajectory,
coalescence of
nearby minima, maxima pairs cannot be avoided. At the instant of
coalescence the denominator in \refeq{SF:sliceEas} goes through a simple
pole, and the integrated trajectory within slice might jump.



\subsection{}
\subsubsection{}

% If in two-column mode, this environment will change to single-column format so that long equations can be displayed.
% Use only when necessary.
%\begin{widetext}
%$$\mbox{put long equation here}$$
%\end{widetext}

% Use the graphics or graphicx packages.
%
% Here is an example of the general form of a figure:
% \begin{figure}
% \includegraphics{}%
% \caption{\label{}}%
% \end{figure}

% Tables may be be put in the text as floats.
% Here is an example of the general form of a table:
% Fill in the caption in the braces of the \caption{} command. Put the label
% that you will use with \ref{} command in the braces of the \label{} command.
% Insert the column specifiers (l, r, c, d, etc.) in the empty braces of the
% \begin{tabular}{} command.
%
% \begin{table}
% \caption{\label{} }
% \begin{tabular}{}
% \end{tabular}
% \end{table}

\section{Conclusion}
\label{sec:intro}

\input concl

\begin{acknowledgments}
Authors are grateful to
D.~Barkley,
W.-J.~Beyn,
K.A.~Mitchell,
B.~Sandstede,
R.~Wilczak,
and in particular E.~Siminos and R.L.~Davidchack
for many spirited exchanges.
S.F. work was supported by the National Science Foundation
grant DMR~0820054 and a Georgia Tech President's Undergraduate
Research Award.
P.C. thanks Glen Robinson Jr. for support. 	
\end{acknowledgments}

% Create the reference section using BibTeX:
\bibliography{../../bibtex/siminos}

\end{document}
%
% ****** End of file aiptemplate.tex ******
