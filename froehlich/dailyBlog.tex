\svnkwsave{$RepoFile: froehlich/dailyBlog.tex $}
\svnidlong {$HeadURL: svn://zero.physics.gatech.edu/froehlich/blog/dailyBlog.tex $}
{$LastChangedDate: 2009-11-21 20:25:56 -0500 (Sat, 21 Nov 2009) $}
{$LastChangedRevision: 29 $} {$LastChangedBy: froehlich $}
\svnid{$Id: dailyBlog.tex 29 2009-11-22 01:25:56Z froehlich $}


\chapter{Research blog on symmetry reduction}
\label{chap:blog}

$\footnotemark\footnotetext{{\tt \svnkw{RepoFile}}, rev. \svnfilerev:
 last edit by \svnFullAuthor{\svnfileauthor},
 \svnfilemonth/\svnfileday/\svnfileyear}$

% J Hightower - former texas politician, author, speaker 1943-
\begin{bartlett}{
Only dead fish go with the flow}
\bauthor{
\HREF{http://www.brainyquote.com/quotes/authors/j/jim_hightower.html}
     {J Hightower}, Texas politician}
\end{bartlett}


\begin{description}
\item[2010/05/10 SF] Definition (9.9) of \emph{stabilizer}
in ChaosBook.org version13, \HREF{http://chaosbook.org/version13/chapters/discrete.pdf}
{Chapter 9 - World in a mirror} seems wrong.

\item[2010/05/10 PC] You are right - we have now
\HREF{http://www.flickr.com/photos/birdtracks/4259634492/}
{replaced ``stabilizer''} by $G_p$-\emph{symmetric} throughout the next chapter.
Please alert me to its occurrences in this chapter, suggest how to
fix them.

BTW, I referenced equation number in your remark with respect to
the stable ChaosBook.org version13 in order that it always links
 correctly: equation numbers etc. keep changing in the unstable,
 currently edited version.


\item[2010/05/11 SF] I started reading chapter 10 on my own.
I do not feel like I have a good grasp of Lie groups and algebras,
do you have any suggestions on what I should read to learn more
or is it not important for the research?.

\item[2010/05/12 PC] En attendant Godot (by the time KGB geniuses have
 your computer configured, the summer might be over), so I moved the
 computer that you were using on Monday to your desk for this summer.

As far as Lie groups and algebras are concerned, ignore them from now - lets
just understand what $SO(2) / U(1)$ invariance does to complex Lorenz equations.
Physicists usually learn $SOn{3}$ next and not the general theory of Lie groups, so lets
start small: invariance on a circle.

\item[2010/05/13 SF] I finished looking through chapters 9 and 10 and the paper. I am going to start looking at the problems from chapter 10 to make sure I understood the material. If any of the problems give me trouble or I'm not sure of the solution I'll post it here.

\item[2010-05-13 PC] That's good, but work within
siminos/froehlich/exerFlow.tex in your blog. Most of the exercises are
already there. Edit them as you see fit, add new ones and their solutions
 if you find that there are missing steps that you need to do in order
to simulate/analyze \cLe.

\item[2010-05-13 PC] You might want to join
\HREF{http://www.zotero.org/groups/cns}
{http://www.zotero.org/groups/cns}
in order to be able to access the papers we are saving there. Nothing
urgent, for now. Search for zotero in siminos/blog/blog.pdf to find
a bit more info about it.

\item[2010/05/24] I started thinking about the problem of when the group tangent at the point is perpendicular to the group tangent at the slice point. I think I understand what is happening but would like you to check my work to make sure I actually do.

    For the $x_1 = 0$ slice for the \cLe\ I think I can show that it is only possible for a trajectory to touch the slice instantaneously unless it is an equilibrium solution: Suppose a trajectory stays in the slice for some interval of time, this means $x_1 = \dot x_1 = 0$ in this interval, which gives us that $y_1 = 0$ during this time. So $y_1$ is constant for some interval of time, so $\dot y_1=0$ here also. Looking at $\dot y_1 = 0$ when $x_1=y_1 =0$ gives $x_2 = -(e/\rho_2)y_2$. Plugging these three restrictions into the equations for $\dot x_2$ and $\dot y_2$ yield the equations $\dot y_2=-\sigma y_2 (1+(\rho_2 / e))$ and $\dot y_2 = (\rho_1 - z)(-e/\rho_2)y_2-y_2$ this means that $-\sigma y_2 (1+(\rho_2 / e))=(\rho_1 - z)(-e/\rho_2)y_2-y_2$ provided $y_2$ is nonzero, then $-\sigma (1+(\rho_2 / e))=(\rho_1-z)(-e/\rho_2)-1$. z is the only variable in that equation, so it must be constant. This means that $\dot z = -b z + x_2 y_2 + x_1 y_1=-b z+ (-e/\rho_2)y_2^2=0$. Again $y_2$ is the only variable so it too must be constant, so $x_2$ is also constant. This means the solution can stay in the slice only if all its values are constant, so it is an equilibrium solution. There are possibly some divide by zero difficulties, but hopefully a similar argument will hold for these cases.

    So we do not run into this problem for the \cLf.

    As I understand it, the reason the group tangent along the trajectory being perpendicular to the group tangent of the slice causes problem is because this means that doing an infinitesimal rotation at the point keeps it inside the slice so the point is not a unique representative (at least when doing the infinitesimal rotations), is this correct?

    When choosing the slice to be a hyperplane through a point, it doesn't have to be perpendicular to the group tangent at some point does it? As long as the group tangent at any point isn't contained inside the hyperplane there shouldn't be a problem locally.

    I thought about trying to show that this condition being true for an interval of time implies it is always true and I have no clue how to go about it so far. I think I see why it is true if the ODE is linear like in QM, but my argument is not quite complete.


\item[2010-05-25 ES] Linear slices are indeed not flow invariant, so only equilibria (or any points in $\Fix{\SOn{2}}$) stay on the linear slice
under equivariant dynamics. Another way to show this would be

\exercise{Linear slices are not flow invariant}{\label{exer:SliceInvCLf}
(a) By forming the inner product of the group tangent direction with the velocity field of \cLe\ show that the linear slice $x_1=0$ is not invariant under \cLf.
(b) Can we construct a flow invariant linear slice? (c) Repeat the exercise using as less information as possible about \cLe.
\authorES{2010-05-25}
}
\solution{exer:SliceInvCLf}{Linear slices are not flow invariant}{(a) The inner product is eq. (71) in internal version of my thesis, in case you need to double check your result. The rest is left to you. (c) Predrag finds such warnings useless, but I have not tried to solve this part. You have of course to make use of equivariance
and perharps of the group representation. Flow contraction of \cLf\ might come in handy, too.
}

The trouble is that \refexer{exer:SliceInvCLf}, or what you have shown above does not help us, I think, to conclude that the singularity at $x_1=x_2=0$ is not reached by the dynamics. What would help, would be

\exercise{$x=0$ singularity in \cLf}{\label{exer:SingOnFixedspCLf}
(a) Show that for \cLf\ $x_1=x_2=0$ implies that $y_1=y_2=0$, that is we can only run into $x_1=0$ slice singularity in $\Fix{\SOn{2}}$ (the $z$ axis).
Since $\Fix{\SOn{2}}$ is flow invariant, general dynamics cannot reach it and we can safely apply the method of moving frames/slices. (b) Is this
result \cLf\ or group representation specific? Can you prove it using as less information for \cLf\ as possible?
\authorES{2010-05-25}
}
\solution{exer:SingOnFixedspCLf}{$x=0$ singularity in \cLf}{Sadly, I do not know how to solve this.}

Note that one might be even more lucky than \refexer{exer:SingOnFixedspCLf} suggests and be able to show that we can only run into a singularity at the origin.

\item[2010-05-28 ES] I am sure you've noticed that when you try to integrate in invariant polynomials basis
you run to some singularity - I had never tried that before. I've only looked at your Mathematica files,
do you get the same behavior with Matlab as well? I've tried playing a bit with the integrator,
for instance using implicit Runge-Kutta should work if stiffness and not a singularity was the problem.
Any intuition on what happens and why? If you have access to Gilmore and Letellier book\rf{GL-Gil07b},
the section \emph{Tips for Integration} might be helpful.

\item[2010-05-28 PC] We have the book in \wwwcb{/tutorials}.

\item[2010-06-01 SF] I checked the differential equations for the Hilbert basis from chapter 10 of the chaos book and got a different set of a equations:
\bea
    \dot u_1 &=& 2\sigma (u_4-u_1)
        \continue
     \dot u_2 &=& 2 u_4 (\rho_1-u_5)+2 \rho_2 u_3 -2 u_2
        \continue
     \dot u_3 &=& -(\sigma+1) u_3 +\rho_2 u_1 +e u_4
        \continue
     \dot u_4 &=& \sigma u_2-(\sigma +1) u_4 + (\rho_1-u_5) u_1 - e u_3
        \continue
     \dot u_5 &=& u_4 - b u_5
\,.
\label{SF:HilbertBasEqs}
\eea

    The equations in the book are very similar to these, except $u_3$ and $u_4$ are switched, the $\rho_2$ terms are dropped (I guess since it is usually taken to be zero), and it has $\sigma-1$ instead of $\sigma+1$ in the equation for $\dot u_4$ (formerly $\dot u_3$).

    When I use this system of equations I don't have any difficulties with stiffness. I also checked the two sets of equilibrium points for when $e+\rho_2 = 0$. They are equilibrium points for this new system, but they were not for the old. So I think this new system is the correct system, and the $\sigma-1$ was somehow causing the stiffness. I can change the formula in the book if you want, but I wanted this system to be checked before I did that.

\item[2010-06-01 ES] Stefan, thanks for catching this. The funny thing is that I had found this error in the
    past, corrected it in CLE paper, I copy here from \emph{revision 1449}:

==========================
    \beq
    \begin{split}
	    u_1 &= x_1^2+x_2^2 \cont
	    u_2 &= y_1^2+y_2^2 \cont
	    u_3 &= x_1 y_2-x_2 y_1\cont
	    u_4 &= x_1 y_1+x_2 y_2\cont
	    u_5 &= z\,.
	    \label{eq:ipLaser}
    \end{split}
    \eeq

    \ES{mathematica notebook:CLEtransfJac.nb
    {\bf PC:} Changed $z$ to $u_5$; have not checked the algebra}\
    \ES{Rechecked algebra by hand, keeping $\rho_2\neq 0$. $u_3$ and $u_4$ where permuted. I will have to check
		the figures for consistency.}
\beq
\begin{split}
  \dot{u}_1 &=2\,\sigma\,(u_4-u_1)\,,\\
  \dot{u}_2 &=-2\left(\,u_2 - \rho_2\, u_3 -\,(\rho_1-u_5)\,u_4\right)\,,\\
  \dot{u}_3 &=-(\sigma\, +1)\,u_3+\rho_2\, u_1+e\, u_4\,,\\
  \dot{u}_4 &=-(\sigma\, +1)\,u_4+\,(\rho_1-u_5)\,u_1+\sigma\, u_2-e\,u_3\,,\\
  \dot{u}_5 &=u_4-b\, u_5\,.
\end{split}
\label{eq:CLEip}
\eeq
===================================

I think it was Predrag who didn't like the formatting and copied
the Chaosbook expressions back (interchanging $u_3$ and $u_4$ one
can fit two lines in one). Anyway, my mistake, I should have immediately
corrected Chaosbook as well.

Stefan, you do use basis \refeq{eq:ipLaser}, right?

\item[2010-06-02 PC] Apologies! Evangelos, are you going to fix it in ChaosBook?

\item[2010-06-02 SF]
I've been doing the computer simulations for the method of moving frames
and using a Hilbert basis on the CLE. I also started reading the paper from
Spieker's blog on plane-Couette and pipe flows, but I'm not sure what to do
after that.

\item[2010-06-03 PC]
One thing to do now is take possession of the blog,
and to purge from exerFlow.tex and flow.tex the stuff
that is a leftover from Rebecca and will not go into your own report,
add your own text and figures. In particular,

(a) have you computed the
stability eigenvalues and eigenvectors of the \reqv\ in
the full space, Hilbert-basis reduced space and within the $x_1=0$ and
general linear slice reduced space. Evangelos has a computation in his thesis
that can probably be made shorter.

(b) would love to see how you cross the singularities when integrating
within (or rotating back into) the \reducedsp. Do a right version
of \refexer{exer:PCsectionCLe}. Discuss that in the appropriate
place in flow.tex, with your own figures. As Evangelos mentions, we
did not look at that carefully, and we do not know whether it is
possible or not to find a slice for which the strange attractor
of \cLe\ avoids the singularities altogether.
This might help you test and generalize the ideas
of your proposal of 2010/05/24 above.

Have more stuff we need to do, but have to run to the conference now....

\item[2010-06-04 SF]
I do use the basis \refeq{eq:ipLaser} for the Hilbert polynomial calculation (Sorry about just now responding, I apparently forgot to commit my changes on Wednesday).

\item[2010-06-11 SF]
When I have been calculating the eigenvalues and vectors at the relative equilibrium in the full space I have not been getting 0 as an eigenvalue or the direction tangent to the group action as an eigenvector. The eigenvalue that I thought was 0 was just really small at the normal parameter values, but can be made larger. Also, the determinant of the stability matrix at the relative equilibrium isn't 0 in general.

\item[2010-06-12 PC]
I cannot quite tell what you are computing, as you are not
checking in the updated programs.
How about reworking \refsect{sect:SymmDyn},
understanding Lie derivatives,
doing \refexer{exer:InfinRotInvari},
and rewriting this material in your own, more precise way?
For a \reqv\ flow and group tangent vectors coincide.
Dotting by the velocity $c$ (\ie, summing over $\groupTan_a$)
the
equivariance condition \refeq{eq:InfnmslRot},
$
\groupTan_a(\vel)  - \Mvar(\ssp) \, \groupTan_a(\ssp) =0
$,
I get
\beq
(\velRel \cdot \Lg - \Mvar ) \vel =0
\,.
\ee{ReqvMargEig}
In other words, you have to be in the co-rotating
frame for eigenvalue to be marginal, and the velocity
be the corresponding right eigenvector. That is even
clearer when you start computing stability of \rpo s.
See whether there is anything smart to be said for the
corresponding left eigenvectors - in the theory of
rotating spirals they call them
``Response Functions'' (and capitalize them), while the
right group tangent vectors they call ``Goldstone
modes'' and they make a big deal out of them
(I hope to blog my reading report into
the Siminos blog soon). I still need to write this
up in the ChaosBook sect. 10.3 Stability, would like
to see how you think about it.

\item[2010-06-14 SF]
That's what I was doing wrong (not computing it in the co-moving frame). I still need to start looking at the left eigenvectors, and I am going to finish rewriting \refsect{sect:SymmDyn} tomorrow.


\end{description}
