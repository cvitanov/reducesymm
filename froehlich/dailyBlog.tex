\svnkwsave{$RepoFile: froehlich/dailyBlog.tex $}
\svnidlong {$HeadURL: svn://zero.physics.gatech.edu/froehlich/blog/dailyBlog.tex $}
{$LastChangedDate: 2009-11-21 20:25:56 -0500 (Sat, 21 Nov 2009) $}
{$LastChangedRevision: 29 $} {$LastChangedBy: froehlich $}
\svnid{$Id: dailyBlog.tex 29 2009-11-22 01:25:56Z froehlich $}


\chapter{Research blog on symmetry reduction}
\label{chap:blog}

$\footnotemark\footnotetext{{\tt \svnkw{RepoFile}}, rev. \svnfilerev:
 last edit by \svnFullAuthor{\svnfileauthor},
 \svnfilemonth/\svnfileday/\svnfileyear}$

% J Hightower - former texas politician, author, speaker 1943-
\begin{bartlett}{
Only dead fish go with the flow}
\bauthor{
\HREF{http://www.brainyquote.com/quotes/authors/j/jim_hightower.html}
     {J Hightower}, Texas politician}
\end{bartlett}


\begin{description}
\item[2010/05/10 SF] Definition (9.9) of \emph{stabilizer}
in ChaosBook.org version13, \HREF{http://chaosbook.org/version13/chapters/discrete.pdf}
{Chapter 9 - World in a mirror} seems wrong.

\item[2010/05/10 PC] You are right - we have now
\HREF{http://www.flickr.com/photos/birdtracks/4259634492/}
{replaced ``stabilizer''} by $G_p$-\emph{symmetric} throughout the next chapter.
Please alert me to its occurrences in this chapter, suggest how to
fix them.

BTW, I referenced equation number in your remark with respect to
the stable ChaosBook.org version13 in order that it always links
 correctly: equation numbers etc. keep changing in the unstable,
 currently edited version.


\item[2010/05/11 SF] I started reading chapter 10 on my own.
I do not feel like I have a good grasp of Lie groups and algebras,
do you have any suggestions on what I should read to learn more
or is it not important for the research?.

\item[2010/05/12 PC] En attendant Godot (by the time KGB geniuses have
 your computer configured, the summer might be over), so I moved the
 computer that you were using on Monday to your desk for this summer.

As far as Lie groups and algebras are concerned, ignore them from now - lets
just understand what $SO(2) / U(1)$ invariance does to complex Lorenz equations.
Physicists usually learn $SOn{3}$ next and not the general theory of Lie groups, so lets
start small: invariance on a circle.

\item[2010/05/13 SF] I finished looking through chapters 9 and 10 and the paper. I am going to start looking at the problems from chapter 10 to make sure I understood the material. If any of the problems give me trouble or I'm not sure of the solution I'll post it here.

\item[2010-05-13 PC] That's good, but work within 
siminos/froehlich/exerFlow.tex in your blog. Most of the exercises are
already there. Edit them as you see fit, add new ones and their solutions
 if you find that there are missing steps that you need to do in order
to simulate/analyze \cLe.

\item[2010-05-13 PC] You might want to join
\HREF{http://www.zotero.org/groups/cns}
{http://www.zotero.org/groups/cns}
in order to be able to access the papers we are saving there. Nothing
urgent, for now. Search for zotero in siminos/blog/blog.pdf to find 
a bit more info about it.


=======
\end{description}
