\svnkwsave{$RepoFile: siminos/froehlich/open.tex $}
\svnidlong {$HeadURL$}
{$LastChangedDate$}
{$LastChangedRevision$} {$LastChangedBy$}
\svnid{$Id$}


\chapter{Open problems}
\label{chap:open}

Here we ponder where to go from
Siminos thesis\rf{SiminosThesis}, investigate various ways
of `quotienting' the \SOn{2} symmetry in more general settings than
the \cLe.
    %
    \Private{ % subversion label pages
$\footnotemark\footnotetext{{\tt \svnkw{RepoFile}}, rev. \svnfilerev:
 last edit by \svnFullAuthor{\svnfileauthor},
 \svnfilemonth/\svnfileday/\svnfileyear}$
    } % end \Private{

\section{\CLe\ odds \& ends}

\exercise{Probability of hitting $(x_1,x_2) =(0,0)$:}{
\label{exer:CLEsmall-x1x2}
Investigate by simulation (and - that is extra cost -
perhaps by thinking) how close does a strange attractor
trajectory of \cLf\ come to
hitting $(x_1,x_2) =(0,0)$ and/or  $(y_1,y_2) =(0,0)$?
%    \authorPC{Jul 14 2009}
    } %end\exercise{Probability of hitting $(x_1,x_2) =(0,0)$

\solution{exer:CLEsmall-x1x2}
{Probability of hitting $(x_1,x_2) =(0,0)$.}{
%
%%%%%%%%%%%%% PC generated by PCsimul.nb
\SFIG{ProblemsPill} %CLEsmall-x1x2}
{}{
A very long time ($t=5000$) simulation of \cLf. $\{x_1,x_2\}$ plot indicates
that probability of hitting $(x_1,x_2) =(0,0)$ is for all practical purposes
equal to zero, and $\sqrt{x_1^2+x_2^2} > 0.02$.
\authorPC{Jul 14 2009}
%(The initial point is on the strange attractor).
}
{fig:CLEsmall-x1x2}
%%%%%%%%%%%%%%%%%%%%%%%%%%%%%%%%%%%%%%%%%%%%%
%
A very long time ($t=5000$) simulation of \cLf, plot of
$\{x_1,x_2\}$  neighborhood in \reffig{fig:CLEsmall-x1x2},
indicates that probability of hitting $(x_1,x_2) =(0,0)$ is
for all practical purposes equal to zero, and
$\rho_1(t)=\sqrt{x_1^2+x_2^2} > 0.02$. Plots of $\{y_1,y_2\}$
neighborhood look similar.
Hence reduction of \statesp\ to 4-dimensional $x_2=0$ \reducedsp\ by rotating successive trajectory increments back
to the $x_1>0$ semi-axis should not run into an $x_1 =0$
singularity in determining the rotation angle $\cos\theta$.
\ES{I certainly agree with this and I think it is clear in
section 4.1.4.2 of the \emph{thesis no one wants to read} that
the singularity of the transformations does not pose a problem
for reduction in the case of \cLe\ as we do not encounter it
numerically (I've tried to pay the extra price and produce a proof
that the denominator cannot vanish but failed).
The reason that the singularity is annoying for me (beyond
visualization problems) is that in the \KS\ case dynamics goes
through zeros in the denominator.}
    \authorPC{Jul 14 2009}
    } %end \solution{Probability of hitting $(x_1,x_2) =(0,0)$.}{


\subsection{$\SOn{2}$ invariants of \cLe}
\label{sect:invariants}

\noindent{\bf Predrag -
July 6 2009}:\\
The generators (Lie algebra elements) of $\SOn{n}$ rotations
are antisymmetric (\cf\ \refeq{SO2generCLe}),
$v(x)\cdot \Lg \cdot v(x)=0$, so from \refeq{eq:InfnmslRot}
it follows that
\beq
0=v(x) \cdot \frac{dv}{dx} \cdot \Lg \cdot x
\,.
\label{eq:ObscurIdnt}
\eeq
This would appear to be a nontrivial multinomial relation between
the 5 coordinates of \cLe, but \texttt{Mathematica}
evaluation shows that it is identically satisfied by the
dynamical equations, yielding no constraint on dynamics.

Noether's theorem suggests that a conserved (invariant)
quantity should be associated with each 1-parameter
continuous invariance\rf{arnold89}, \ie, it should be
possible to write a ``Hamiltonian,'' in terms of ``momentum,
angle'' variables such that that the ``momentum'' variable
(radius $r$ in the harmonic oscillator of
\refexam{exer:HarmOscPolar}) is conserved, and the conjugate
angle variable has trivial dynamics.

We do not know how to construct such invariant function,
but due to the length conservation under rotations
(antisymmetry of the Lie algebra generators such as \refeq{SO2generCLe}),
functions such as
$R^2_{\EQB{0}} = (\ssp-\ssp_{\EQB{0}}) \cdot (\ssp-\ssp_{\EQB{0}}) $
\beq
\frac{d~}{d\theta} R^2_{\EQB{0}} = (\ssp-\ssp_{\EQB{0}}) \cdot \Lg \cdot (\ssp-\ssp_{\EQB{0}})
= 0
\ee{lengthInv}
are invariant.

\section{A flow with two Fourier modes}

\noindent{\bf Predrag -
Jul 9 2009}:\\
\CLe\ \refeq{eq:CLe} of Gibbon and McGuinness\rf{GibMcCLE82} have a
degenerate 4-dimensional subspace consisting of has two linear $m=1$
representations, with \SOn{2} acting only in its lowest non-trivial
representation. If you write it in polar coordinates, terms like
$\cos(\theta)$ but no $\cos(2\theta)$. If there are no higher $m$
components, group orbits look like circles, and that is misleading -
higher Fourier components would be more typical of terms you get in PDEs.

Here is a possible model, still 5-dimensional, but with \SOn{2} acting in
the two lowest representations. Such models arise as truncations of
Fourier-basis representations of PDEs on periodic domains. In the complex
form, the simplest such modification of \cLe\ may be the `2-mode' system
\beq
\begin{split}
 \dot{x} &=-\sigma x+ \sigma x^* y  \,,\\
 \dot{y} &=(r-z)x^2-a y \,,\\
 \dot{z} &= \frac{1}{2}\left(x^2 y^*+x^{*2} y\right)-b z\,,
 \label{eq:2me}
\end{split}
\eeq
where $x,y$, $r=r_1+ i\,r_2$, $a=1+i\,e$ are complex and $z$,
$b$, $\sigma$ are real. Rewritten in terms of real variables
$x=x_1+ i\, x_2\,,\ y=y_1+i\, y_2$ this is a 5-dimensional
first order ODE system
%\PC{complete the rewrite here!}
%    2012-02-20 Bryce alerted me - now using his solution
%    2012-03-02 Borrero alerted me
% x1dot = - sigma*x1 + sigma*(x1*y1 + x2*y2)
% x2dot = -sigma*(x2) + sigma*(- x1*y2 + x2*y1)
\bea
	\dot{x}_1 &=& -\sigma x_1 + \sigma (x_1 y_1 - x_2 y_2)\continue
	\dot{x}_2 &=& -\sigma x_2 + \sigma (x_1 y_2 - x_2 y_1)\continue
	\dot{y}_1 &=& -y_1 +e y_2 +(r_1-z)(x_1^2-x_2^2)
                  -r_2 x_1 x_2\continue
	\dot{y}_2 &=& -y_2 +e y_1 +r_2 (x_1^2-x_2^2)
                  +r_1 x_1 x_2\continue
	\dot{z} &=& -b z + x_1^2 y_1 - x_2^2 y_1 + x_1 x_2 y_2
\,.
\label{eq:2meR}
\eea


\subsection{Rotational equivariance of 2-mode system}

\exercise{Discover the equivariance of a given flow:}{ \label{exer:discvInvar}
Suppose you were given \cLe, but nobody told you they are \SOn{2} equivariant.
More generally, you might encounter a flow without realizing that it
has a continuous symmetry - how would you test for it?
    }

\solution{exer:discvInvar}{Discover the equivariance of a given flow?}{
If $M$ is in the Lie algebra, by the equivariance condition \refeq{eq:InfnmslRot}
the Lie derivative
$
M \cdot v(x)-\Mvar \cdot M \cdot \ssp
$
vanishes. You have $v(x)$ and the \stabmat\ $\Mvar$, hence try finding whether
a $M$ can be found such that the Lie derivative vanishes. You know that if the
symmetry group is a subgroup of \SOn{d}, the Lie algebra elements $M$ can be
taken antisymmetric. (Have not tried to solve
this problem, so let us know if you succeed)
\authorPC{Aug 25 2009}
    }

\section{\Eqva}

An \eqv\ is any point for which the velocity
field of an ordinary differential equation is zero.

\exercise{\Eqva\ of 2-mode system:}{\label{exer:2mEquiCLe}
Find all \eqva\ of 2-mode system.
    }

\solution{exer:2mEquiCLe}{\Eqva\ of 2-mode system.}{
(solution not available)
    } %end \solution{exer:2mEquiCLe}{\Eqva\ of 2-mode system.}


\subsection{Eigenvalues and eigenvectors of the \stabmat}

\exercise{Eigenvalues and eigenvectors of $\EQV{0}$ \stabmat:}
         {\label{exer:2mEigenE0}
Find the eigenvalues and the eigenvectors of the \stabmat\ $\Mvar$ at $\EQV{0}$.
}

\solution{exer:2mEigenE0}{Eigenvalues and eigenvectors of $\EQV{0}$ \stabmat.}{
(solution not available)
} % end \solution{exer:2mEigenE0}{eigenvectors of $\EQV{0}$ \stabmat

\exercise{Plotting the eigenvalues and eigenvectors of the \stabmat\ at $\EQV{0}$:}{\label{exer:2mPlotEigenE0}
Plot the eigenvectors of $\Mvar$ at $\EQV{0}$ and the 2-mode
system at values very close to $\EQV{0}$.
}

\solution{exer:2mPlotEigenE0}
{Plotting the eigenvalues and eigenvectors of the \stabmat\ at $\EQV{0}$:}{
(solution not available)
} % end \solution{exer:2mPlotEigenE0}{Plotting the eigenvalues and eigenvectors of the \stabmat\ at $\EQV{0}$

\exercise{Finding the eigenvalues and eigenvectors of  $\REQB{1}$ \stabmat:}{\label{exer:2mEigenE1}
Compute the eigenvalues and eigenvectors of the \stabmat\ evaluated at $\REQB{1}$.
}

\solution{exer:2mEigenE1}{Finding the eigenvalues and eigenvectors of  $\REQB{1}$ \stabmat.}{
(solution not available)
} % end \solution{exer:2mEigenE1}{Finding the eigenvalues and eigenvec
