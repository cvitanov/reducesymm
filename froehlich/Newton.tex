%           %experimenting with svn-multi

\svnkwsave{$RepoFile: siminos/froehlich/Newton.tex $}
\svnidlong {$HeadURL$}
{$LastChangedDate$}
{$LastChangedRevision$} {$LastChangedBy$}
\svnid{$Id$}

\section{Newton searches for \po s}

We have already seen how knowing periodic orbits and their
stability can help us to understand the dynamics of the
system, but there still is the question of how to actually
find periodic orbits.

Suppose we are trying to find the periodic orbits of a system. This means
we want to find $x$ and $T$ such that $\ssp = f^T\left(\ssp\right)$,
where $f^T$ is the trajectory at time $T$. We shall use Newton's method
to search zeros of the function $F\left(\ssp,T\right) = \ssp -
f^T\left(\ssp\right)$. Let $\left(\ssp,T\right)$ be an initial guess for
Newton's method close to an actual solution $\left(\ssp + \Delta x,T +
\Delta T\right)$, \ie\ $0 = \ssp +\Delta \ssp - f^{T+\Delta T}\left(\ssp
+ \Delta \ssp\right)$. Keeping the Taylor series to linear order
\[
0 \approx \ssp-
f^T\left(\ssp\right) + \left(I -
\jMps\left(\ssp\right)\right)\Delta \ssp -
v\left(f^T(\ssp)\right)\Delta T
\]
we get the Newton iteration estimate for
$(\Delta \ssp,\Delta T)$

We already know that if $\ssp$ is on a periodic orbit then
$
% \jMps\left(\ssp\right) \vel(\ssp) = v\left(\ssp\right) \Rightarrow
\left(I-\jMps\left(\ssp\right)\right) v\left(\ssp\right) =
0$, so $I-\jMps\left(\ssp\right)$ becomes
non-invertible as $\ssp$ approaches a periodic orbit.
%$\left(I-\jMps\left(\ssp\right)\right) v\left(\ssp\right)\approx 0$.
If $\Delta x$ is in the direction of
$v\left(\ssp\right)$ then
$\left(I-\jMps\left(\ssp\right)\right) \Delta \ssp \approx 0$
so the equation becomes very costly to solve accurately. This
means that we need to constrain the $\Delta \ssp$ to be
transverse to the flow during a Newton's iteration. We do this
by requiring $\Delta x$ satisfy
\beq
v (\ssp)^T \Delta \ssp = 0
\,.
\ee{SF:NewtonTransv}
We then get the system
\beq
    \begin{pmatrix}
        I-\jMps(\ssp)& \partial \vel(\ssp)\\
        \vel(\ssp)& 0
    \end{pmatrix}
    \begin{pmatrix}
        \Delta \ssp\\
        \Delta t
    \end{pmatrix}
    =-
    \begin{pmatrix}
        \ssp - f(\ssp)\\
        0
    \end{pmatrix}
\eeq
This problem was caused by the velocity vector being an
eigenvector of the {\jacobianM} with unit Floquet
multiplier. Thus for every unit Floquet multiplier we
need to add a constraint. It was
already shown that for every dimension of a continuous
symmetry there is a unit Floquet multiplier. For each of
these we need to add a constraint
similar to \refeq{SF:NewtonTransv}.

Next consider the same situation, but instead we will use a
cost (or error) function $I\left(\Delta \ssp\right) =
\left(\ssp + \Delta \ssp - f\left(\ssp + \Delta
\ssp\right)\right)^2/2$. If $\ssp+\Delta \ssp$ is on a
periodic orbit then $I\left(\delta \ssp\right)$ and its
linear in $\Delta x$ approximation  $\left(\ssp + \Delta \ssp
- f\left(\ssp+ \delta
\ssp\right)\right)\left(I-\jMps\left(\ssp\right)\right)$ are
both equal to 0. This cost function measures how
far the point $\ssp+ \Delta \ssp$ is from being a zero of the
function $F(\ssp, \Delta \ssp) = \ssp - f(\ssp)$. The cost
function gives a non-negative value, and its zeros are the
points on periodic orbits, so if we can find a way to
minimize the function then we would either find a periodic
orbit or learn that there are not any near our initial guess
(if the local minimum is not zero then there is no zero
nearby).

The linear approximation of the cost function is zero at a
periodic orbit, so the cost function is dominated by the
second order term near the orbits. Expanding $I(\Delta \ssp)$
to the second order, we get that $I \approx \tilde{\Delta
\ssp}^2/2 + (\ssp - f(\ssp)) \tilde{\Delta \ssp} +
(\ssp-f(\ssp))^2/2$ where $\tilde{\Delta \ssp} =
(I-\jMps(\ssp))\Delta \ssp$.
