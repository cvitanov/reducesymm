% defsBlog.tex
% $Author$ $Date$

% includes experimental definitions, to be transferred to ChaosBook.org
% lines prefaced by  %DB: are already included in ChaosBook.org macros

%%%%%%%%%%%%%%% VAGGELIS MACROS %%%%%%%%%%%%%%%%%%%%%%
\newcommand{\refFigToFig}[2]{Figures~\ref{#1} to~\ref{#2}}

    \ifboyscout
\newcommand{\toCB}{\marginpar{\footnotesize 2CB}}  % to compare with ChaosBook
\newcommand{\inCB}{\marginpar{\footnotesize now in CB}} % entered in ChaosBook
    \else
\newcommand{\toCB}{}
\newcommand{\inCB}{}
    \fi %end of internal draft switch

\newcommand{\bseq}{\begin{subequations}}
\newcommand{\eseq}{\end{subequations}}

\newcommand{\chebT}{\mathrm{T}}
\newcommand{\chebU}{\mathrm{U}}

\newcommand{\normVec}{\ensuremath{\mathbf{n}}}    % group orbit curvature normal

\newcommand{\bCell}{\ensuremath{\Omega}}
\newcommand{\NS}{Navier-Stokes}
\newcommand{\NSE}{Navier-Stokes Equations}
\newcommand{\NSe}{Navier-Stokes equations}
\newcommand{\stateDsp}{state-space}
\newcommand{\StateDsp}{State-space}
\newcommand{\Template}{Template}
\newcommand{\ii}{\ensuremath{\mathrm{i}}} % sqrt{-1}
\newcommand{\wurst}{wurst}
\newcommand{\Wurst}{Wurst}
\newcommand{\twoMode}{Porter-Knobloch}
\newcommand{\templates}{templates} % {slice-fixing point} % {reference state}

%%%%%%%%%%%%%%% DasBuch MACROS %%%%%%%%%%%%%%%%%%%%%%

\renewcommand{\arXiv}[1]{
              {\href{http://arXiv.org/abs/#1}{arXiv:#1}}}
%  \SFIG{#1}    % f_name.png (or .pdf)
%       {#2}    % short caption text
%       {#3}    % full caption text
%       {#4}    % f-figure-label
\newcommand{\SFIG}[4]{\begin{figure}[h]
              %\hspace*{-0.10\textwidth}
              \hspace*{0.10\textwidth}
              \begin{minipage}[b]{0.55\textwidth}
                      \caption[#2]{#3}
                      \label{#4}
              \end{minipage}~~~~~%
              \begin{minipage}[b]{0.40\textwidth}
                      \includegraphics[width=1.00\textwidth]{#1}
              \end{minipage}
              %\hfill
              \end{figure} }

% \FIG{#1}    % \includegraphics[width=0.40\textwidth]{Fig/f_name.ps}  ...
%     {#2}    % short caption text omitted here
%     {#3}    % full caption text
%     {#4}    % f-figure-label
\newcommand{\FIG}[4]{\begin{figure}[h]
            \begin{minipage}[b]{0.98\textwidth}
                    #1\end{minipage}
                      \caption[#2]{#3}
                      \label{#4}\end{figure}}

  %% Predrag 26aug2007: this supposedly speeds up compilation
  %% of frequently used object(Lipkin, "Latex" p. 362)
%%%%  mark section as `cyclist':
            \newsavebox{\bike}\sbox{\bike}
            {\raisebox{-5.0ex}[4.5ex][3.5ex]
             {\includegraphics[width=1.6cm]{w11-1bicycle}}}
\newcommand{\cyclist}{\marginpar[\hfill\usebox{\bike}]
        {\usebox{\bike}\hfill}}

%%%% mark section as `pedestrian':
           \newsavebox{\trot}\sbox{\trot}
           {\raisebox{-5.0ex}[4.5ex][3.5ex]
            {\includegraphics[width=1.6cm]{i-4-pedestrian}}}
\newcommand{\pedestrian}{\marginpar[\hfill\usebox{\trot}]
        {\usebox{\trot}\hfill}}


%%%%%%%%%%%%%%% FORMATING: PAGINATION FOR bin/mkbook %%%%%%%%%%%%%%%%%%%%%%%

\newcommand{\Remarks}{\section*{\textsf{\textbf{Commentary}}}}
\newcommand{\RemarksEnd}{}

\newtheorem{rmark}{{\small\textsf{\textbf{Remark}}}}[chapter]
\newcommand{\remark}[2]{
        % \begin{quotation}
        \begin{rmark}
        {\small\em\noindent {\small\sf \underline{ #1} ~} #2 }
    \end{rmark}
    	}

\newtheorem{exmple}{\noindent\small\textsf{\textbf{Example}}}[chapter]
\newcommand{\example}[2]{
    \vskip -13mm
        %\begin{offset}
        \begin{exmple}
           \noindent\small
           \textsf{\textbf{#1}} ~
       \slshape\sffamily{#2}
       % \textsl would not work...
    \end{exmple}
    %\end{offset}
        \vskip -1mm
             }

\newcommand{\Resume}{\section*{\textsf{\textbf{R\'esum\'e}}}
        \addcontentsline{toc}{section}{{~~~~Resum\'e}}
    %            \addtocontents{toc}{{\small r\'esum\'e \thepage ~}}
    }
\newcommand{\ResumeEnd}{}

\newtheorem{exerc}{\textsf{\textbf{Exercise}}}[chapter]
% \newtheorem{exerc}{}[chapter]
% \newtheorem{exerc}{{$\bullet$}}[chapter]
 \newcommand{\exercise}[2]{
        %\vskip -13mm
         \noindent
         \begin{exerc}{
\renewcommand{\theenumi}{\alph{enumi}}
\renewcommand{\labelenumi}{\textbf{(\alph{enumi})\ }}
    {\noindent\small
         ~~\textsf{\textbf{#1}} ~
           \slshape\sffamily{#2}  } % \textsl would not work...
    }
         \vskip -1mm
% removed the line: % \noindent\rule[.1mm]{\linewidth}{.5mm}
         \end{exerc}
                          }

\newcommand{\Exercise}[2]{      %environment for obligatory problems
        %\vskip -13mm
        \noindent
        \begin{exerc}{
\renewcommand{\theenumi}{\alph{enumi}}
\renewcommand{\labelenumi}
    {\textsf{\textbf{ (\alph{enumi})\ }}}
        {\noindent
         ~~\textsf{\textbf{\underline{#1}}} ~
           \slshape\sffamily{#2}  } % \textsl would not work...
        }
         \vskip -1mm
        \end{exerc}
                          }

\newcommand{\solution}[3]{
        {\noindent\small
         \textsf{\textbf{Solution \ref{#1}~-~#2}} %LABEL - TITLE
         \slshape\sffamily{#3}                    %TEXT
         }
         \vskip  1ex  %4mm
% removed the line: % \noindent\rule[.1mm]{\linewidth}{.5mm}
                        }
%%%%%%%%%%%%%%% WALLY's FAVORITE MACROS %%%%%%%%%%%%%%%%%%%%%%
\def\xh{\mathbf{\hat{x}}}  \def\yh{\mathbf{\hat{y}}}  \def\zh{\mathbf{\hat{z}}}

\def\Pv{{\mathcal{P}_v}}  \def\Pe{{\mathcal{P}_\eta}}

\newcommand{\bvec}[1]{\boldsymbol{#1}}

\def\vphi{\bvec{\phi}}
\def\vpsi{\bvec{\psi}}
\def\grad{\bvec{\nabla}}
\def\vv{\bvec{v}}      % \def\vv{{\mbox {\boldmath $v$}}}
\def\vx{\bvec{x}}      % \def\vx{{\mbox {\boldmath $x$}}}
\def\vk{\bvec{k}}      % \def\vk{{\mbox {\boldmath $k$}}}
\def\uh{\hat{u}} \def\vh{\hat{v}} \def\wh{\hat{w}} \def\eh{\hat{\eta}}
\def\ub{\overline{u}} \def\wb{\overline{w}}
% \def\eg{{\it e.g.\ }} \def\ie{{\it i.e.\ }}
% \def\Real{{\mathbb R}}	% this is redundant, ChaosBook defines this instead:
% \newcommand{\reals}{\mathbb{R}}


%%%%%%%%%%%%%%% GIBSON FAVORITE MACROS %%%%%%%%%%%%%%%%%%%%%%

\newcommand{\bu}{\ensuremath{{\bf u}}}
\newcommand{\bv}{\ensuremath{{\bf v}}}
\newcommand{\bff}{\ensuremath{{\bf f}}}
\newcommand{\dbu}{\delta {\bf u}}
\newcommand{\dbv}{\delta {\bf v}}
\newcommand{\hbu}{\hat{{\bf u}}}
\newcommand{\hbv}{\hat{{\bf v}}}
\newcommand{\hu}{\hat{u}}
\newcommand{\hv}{\hat{v}}
\newcommand{\hw}{\hat{w}}
%\newcommand{\bnabla}{{\boldmath \nabla}} % what's wrong with this?
\newcommand{\be}{{\bf e}}
\newcommand{\bx}{{\bf x}}
\newcommand{\ex}{{\hat{\bf x}}} % unit vectors
\newcommand{\ey}{{\hat{\bf y}}}
\newcommand{\ez}{{\hat{\bf z}}}
\newcommand{\Om}{\Omega}    % JFG mantra

\newcommand{\bPhi}{{\bf \Phi}}
\newcommand{\bphi}{{\bf \phi}}
\newcommand{\bhphi}{{\bf \hat{\phi}}}
\newcommand{\bU}{{\bf U}}
\newcommand{\bW}{{\bf W}}
\newcommand{\lapl}{\nabla^2}
% \newcommand{\tEQ}{\ensuremath{{\text{EQ}}}}
\newcommand{\tNB}{\ensuremath{{\text{NB}}}}
\newcommand{\tLB}{\ensuremath{{\text{LB}}}}
\newcommand{\tUB}{\ensuremath{{\text{UB}}}}
\newcommand{\tLM}{\ensuremath{{\text{LM}}}}
\newcommand{\tNS}{\ensuremath{{\text{NS}}}}
\newcommand{\tCFD}{\ensuremath{{\text{CFD}}}}
\newcommand{\uEQ}{\ensuremath{\bu_{\text{\tiny EQ}}}}
\newcommand{\vEQ}{\ensuremath{\bv_{\text{\tiny EQ}}}}
\newcommand{\uLM}{\ensuremath{\bu_{\text{\tiny LM}}}}
\newcommand{\uLB}{\ensuremath{\bu_{\text{\tiny LB}}}}
\newcommand{\uNB}{\ensuremath{\bu_{\text{\tiny NB}}}}
\newcommand{\uNBtwo}{\ensuremath{\bu_{\text{\tiny NB2}}}}
\newcommand{\uUB}{\ensuremath{\bu_{\text{\tiny UB}}}}
\newcommand{\vLM}{\ensuremath{\bv_{\text{\tiny LM}}}}
\newcommand{\vLB}{\ensuremath{\bv_{\text{\tiny LB}}}}
\newcommand{\vNB}{\ensuremath{\bv_{\text{\tiny NB}}}}
\newcommand{\vUB}{\ensuremath{\bv_{\text{\tiny UB}}}}
\newcommand{\bbR}{\mathbb{R}}
\newcommand{\bbU}{\mathbb{U}}
\newcommand{\bbUsymm}{\bbU_{S}}
%\newcommand{\half}{\frac{1}{2}}
\newcommand{\pd}[2]{\frac{\partial #1}{\partial #2}}
\newcommand{\Norm}[1]{\|{#1}\|}
%\newcommand{\grad}{\boldsymbol{\nabla}}

%\newcommand{\Refl}{\ensuremath{R}}
\newcommand{\Shift}{\ensuremath{\tau}}
% \newcommand{\Shift}{\ensuremath{\mathbf{S}}}
\renewcommand{\shift}{\ensuremath{\ell}}
\newcommand{\nameit}{\ensuremath{w-}}

%%%%%%%%%%%% Abbreviations, Siminos thesis specific %%%%%%

\newcommand{\cont}{\,, \\ }
% PC Jul 3 2009: removed Siminos redefinions:
%\renewcommand{\statesp}{phase space}
%\renewcommand{\Statesp}{Phase space}
% PC Aug 20 2009: temporarily disabled ChaosBook definions:
\renewcommand{\reducedsp}{reduced state space}
\renewcommand{\Reducedsp}{Reduced state  space}
\newcommand{\Manif}{\ensuremath{\mathcal{M}}}
\newcommand{\bbUplus}{\Fix{\Dn{1}}}
\newcommand{\Lint}[1]{\frac{1}{L}\!\oint d#1\,}

\newcommand{\Gelement}{\ensuremath{g}}         % group element in \Group
\newcommand{\LGelement}[1]{\ensuremath{g(#1)}} % Lie group element in \Group
\newcommand{\slicepComp}[2]{{\ensuremath{\overline{#1}'_{#2}}}}   % slice-fixing point component
\newcommand{\Subgroup}{H}

\newcommand{\BCs}{boundary conditions}
\newcommand{\eps}{\epsilon}
\newcommand{\fh}{\hat{f}}
\newcommand{\Reynolds}{\textit{Re}}  % Reynolds number
\newcommand{\Rey}{\text{Re}}
\newcommand{\dt}{\Delta t}
\newcommand{\Dx}{\xi}
\newcommand{\ignore}[1]{}
\newcommand{\bnabla}{{\boldmath \nabla}}

\newcommand{\bg}{{\bf g}}
\newcommand{\bt}{{$\bullet$}}

%%%%%%%%%%%% MACROS, CLe paper specific %%%%%%%%%%

\newcommand{\vf}{v}	%%% keep notation for vector field flexible.
\newcommand{\REQB}[1]{\ensuremath{\mathrm{Q}_{#1}}} % For ODE's, use REQV from chaosbook for PDE's

%%%%%%%%%%%%%%% TEXT MACROS %%%%%%%%%%%%%%%%%%%%%%

\newcommand{\ew}{eigen\-value}
\newcommand{\Ew}{eigen\-value}
\newcommand{\ev}{eigen\-vector}
\newcommand{\Ev}{eigen\-value}
\newcommand{\ef}{eigen\-function}
\newcommand{\Ef}{eigen\-value}
\newcommand{\steady}{\textdollar~}

\newcommand{\ubranch}{upper branch}
\newcommand{\Ubranch}{Upper branch}
\newcommand{\lbranch}{lower branch}
\newcommand{\Lbranch}{Lower branch}
\newcommand{\newfp}{NU {\eqb}}

%%%%%%%%%%%%%%% Vlasov %%%%%%%%%%%%%%%%%%%%%%%%%%%%%%%%%%%%%%%%%%%%%%%%

\newcommand{\lambdaV}{\ensuremath{\lambda}} % Vlasov eigenvalue
\newcommand{\VlasovLrz}{\ensuremath{\mathcal{A}}} % Linearized Vlasov operator
\newcommand{\VlasovLrzSD}{\ensuremath{\mathcal{B}(\theta)}}
\newcommand{\SDop}[1]{U(#1)}
\newcommand{\SDopI}[1]{U^{-1}(#1)}

%%%%%%%%%% Focus macros %%%%%%%%%%%%%%%%%%%%%%
\newcommand{\recFlow}{recurrent flow}
\newcommand{\RecFlow}{Recurrent flow}

\ifdasbuch
\newcommand{\Points}[1]{}
\else
\newcommand{\Points}[1]{\marginpar{\fbox{\sf
                % $\bullet$
                    #1~point
                % $\bullet$
                  } } }
\fi

%%%%%%%%%%%%%%% EXPERIMENTAL EXERCISES IN NOTATION %%%%%%%%%%%%%%%%%%%%

%%%%%%%%%% FLOWS: %%%%%%%%%%%%%%%%%%%%%%%%%%%%

\renewcommand\velField[1]{{F(#1)}}  % ODE velocity field
\renewcommand{\vel}{F}      % state space velocity vector
\renewcommand{\ssp}{a}             % state space point

\renewcommand{\pSRed}{\ensuremath{\hat{\cal M}}} % reduced state space
\renewcommand{\sspRed}{\ensuremath{\hat{\ssp}}}    % reduced state space point, experiment
\renewcommand{\velRed}{\ensuremath{\hat{\vel}}}    % ES reduced state space velocity
\renewcommand{\slicep}{\ensuremath{\hat{\ssp}'}}   % slice-fixing point, experimental

\newcommand{\mapRed}{\ensuremath{\hat{\map{f}}}}    % PC reduced state space map


%%%%%%%%%% LINEARIZED FLOWS: %%%%%%%%%%%%%%%%%%%%%%%%%%%%

\renewcommand{\deltaX}{{\delta a}}  %trajectory displacement
\renewcommand{\derF}[1]{{DF |_{#1}}}    % Gibson stability matrix
\renewcommand{\derf}[2]{{Df^{#1}|_{#2}}}    % Gibson jacobiam  matrix


\newcommand{\LieElrep}{\ensuremath{\mathbf{G}}}

%%%%%%%%%%%% REMOVE THIS EVENTUALLY %%%%%%%%%%%
