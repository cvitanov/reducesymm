% elton/inputs/defsElton.tex
% $Author: predrag $ $Date: 2015-09-26 19:42:18 -0400 (Sat, 26 Sep 2015) $

%%%%%%%%%%%% MACROS, noisy project specific %%%%%%%%%%

    \ifboyscout
\newcommand{\toCB}{\marginpar{\footnotesize 2CB}}  % to compare with ChaosBook
\newcommand{\inCB}{\marginpar{\footnotesize now in CB}} % entered in ChaosBook
\newcommand{\JRE}[1]{$\footnotemark\footnotetext{JRE: #1}$}
\newcommand{\JREedit}[1]{{\color{red}#1}}
\renewcommand{\authorPC}[1]{\hfill (P. Cvitanovi\'c, #1)}
\renewcommand{\authorJMH}[1]{\hfill (J.M. Heninger, #1)}
\renewcommand{\authorMMFPC}[1]
     {\hfill (M.M. Farazmand and P. Cvitanovi\'c, #1)}
\renewcommand{\authorMMF}[1]{\hfill (M.M. Farazmand, #1)}
    \else
\newcommand{\toCB}{}
\newcommand{\inCB}{}
\newcommand{\JRE}[1]{}
\newcommand{\JREedit}[1]{#1}
\renewcommand{\authorPC}[1]{\hfill (P. Cvitanovi\'c)}
\renewcommand{\authorJMH}[1]{\hfill (J.M. Heninger)}
\renewcommand{\authorMMFPC}[1]
     {\hfill (M.M. Farazmand and P. Cvitanovi\'c)}
\renewcommand{\authorMMF}[1]{\hfill (M.M. Farazmand)}
   \fi %end of internal draft switch

\newcommand{\PCpost}[2]{\item[#1 Predrag] {#2}}
\newcommand{\APWpost}[2]{\item[#1 Ashley] {#2}}
\newcommand{\KYSpost}[2]{\item[#1 Kimberly] {#2}}
\newcommand{\BEpost}[2]{\item[#1 Bruno] {#2}}
\newcommand{\FFpost}[2]{\item[#1 Franco] {#2}}
\newcommand{\MMFpost}[2]{\item[#1 Mohammad] {#2}}
\newcommand{\AFpost}[2]{\item[#1 Adam] {#2}}
\newcommand{\YLpost}[2]{\item[#1 Lan] {#2}}
\newcommand{\NBBpost}[2]{\item[#1 Burak] {#2}}

%%%%%%%%%%%% MACROS, project specific %%%%%%%%%%


\newcommand{\NS}{Navier-Stokes}
\newcommand{\NSe}{Navier-Stokes equation}
\newcommand{\Reynolds}{\ensuremath{\textit{Re}}}  % Reynolds number
\newcommand{\Velgradmat}{Matrix of velocity gradients}

\newcommand{\steady}{\marginpar{{\color{green}\textdollar}}}

%%%%%%%%%%%%%%% Sundry symbols within math eviron.: %%%%%%%%%%%%
\newcommand{\pd}[2]{\frac{\partial #1}{\partial #2}}
\newcommand{\grad}{{\bf \nabla}}
\newcommand{\trHalf}[1]{\tau_{#1}}    % 1/2 cell translation
\newcommand{\DiffC}{\ensuremath{D}}     % diffusion constant

%%% 3D physical flow naming conventions
% PC \teq is the name, \seq is the  symbol
\newcommand{\teq}[1]{\ensuremath{{\text{eq#1}}}}
\newcommand{\teqa}{\ensuremath{{\text{eq0}}}}
\newcommand{\teqb}{\ensuremath{{\text{eq1}}}}
\newcommand{\teqc}{\ensuremath{{\text{eq2}}}}

\newcommand{\ttw}[1]{\ensuremath{{\text{tw#1}}}}
\newcommand{\ttwa}{\ensuremath{{\text{tw1}}}}  % spanwise
\newcommand{\ttwb}{\ensuremath{{\text{tw2}}}} % Divakar D1, lower streamwise
\newcommand{\ttwc}{\ensuremath{{\text{TW3}}}}  % upper streamwise

\newcommand{\bu}{\ensuremath{{\bf u}}}
\newcommand{\bx}{\ensuremath{{\bf x}}}
\newcommand{\be}{{\bf e}}
\newcommand{\ben}[1]{{\be}_{#1}}
\newcommand{\beUBg}[1]{\ensuremath{\be_{#1}}}
\newcommand{\butot}{\ensuremath{{\bf u_{tot}}}}
\newcommand{\bnabla}{\ensuremath{{\bf \nabla}}}
\newcommand{\lapl}{\ensuremath{{\nabla^{2}}}}
\newcommand{\Norm}[1]{\|{#1}\|}

\newcommand{\xeq}[1]{\ensuremath{\bx_{\text{\tiny eq#1}}}}
\newcommand{\xeqa}{\ensuremath{\bx_{\text{\tiny eq0}}}}
\newcommand{\xeqb}{\ensuremath{\bx_{\text{\tiny eq1}}}}
\newcommand{\xeqc}{\ensuremath{\bx_{\text{\tiny eq2}}}}

\newcommand{\xtw}[1]{\ensuremath{\bx_{\text{\tiny tw#1}}}(t)}
\newcommand{\xtwa}{\ensuremath{\bx_{\text{\tiny tw1}}}(t)}
\newcommand{\xtwb}{\ensuremath{\bx_{\text{\tiny tw2}}}(t)}
\newcommand{\xtwc}{\ensuremath{\bx_{\text{\tiny tw3}}}(t)}

%%%%%%%%%%%%%%%%%%%%%%%%%%%%%%%%%%%%%%%%%%%%%%%%%%%%%%%%%%%
% PC: experimental, for stagnation points
\newcommand{\tSP}[1]{{\ensuremath{{\text{SP#1}}}}}
\newcommand{\tSPone}{\ensuremath{{\text{SP1}}}}
\newcommand{\tSPtwo}{\ensuremath{{\text{SP2}}}}
\newcommand{\tSPthr}{\ensuremath{{\text{SP3}}}}
\newcommand{\xSP}[1]{{\ensuremath{\bx_{\text{\tiny SP#1}}}}}
\newcommand{\xSPone}{\ensuremath{\bx_{\text{\tiny SP1}}}}
\newcommand{\xSPtwo}{\ensuremath{\bx_{\text{\tiny SP2}}}}
\newcommand{\xSPthr}{\ensuremath{\bx_{\text{\tiny SP3}}}}

%%%%%%%%%%%%%%%%%%%%%%%%%%%%%%%%%%%%%%%%%%%%%%%%%%%%%%%%%%%
% JFG: Fluids literature uses bold to indicate vector
% quantities, so that one can use {\bf u} for the vector
% and u for the first component of the vector.
% PC \eqva/\reqva directory: \tAA is the name, \sAA is the  symbol
\newcommand{\tLM}{\ensuremath{{\text{EQ0}}}}
\newcommand{\tLB}{\ensuremath{{\text{EQ1}}}}
\newcommand{\tUB}{\ensuremath{{\text{EQ2}}}}
\newcommand{\tNNB}{\ensuremath{{\text{EQ3}}}}
\newcommand{\tNB}{\ensuremath{{\text{EQ4}}}}
\newcommand{\tEQfive}{\ensuremath{{\text{EQ5}}}}
\newcommand{\tEQsix}{\ensuremath{{\text{EQ6}}}}
\newcommand{\tEQsev}{\ensuremath{{\text{EQ7}}}}
\newcommand{\tEQeight}{\ensuremath{{\text{EQ8}}}}
\newcommand{\tEQnine}{\ensuremath{{\text{EQ9}}}}
\newcommand{\tEQten}{\ensuremath{{\text{EQ10}}}}

\newcommand{\tTW}[1]{\ensuremath{{\text{TW#1}}}}
\newcommand{\tTWone}{\ensuremath{{\text{TW1}}}}  % spanwise
\newcommand{\tTWDone}{\ensuremath{{\text{TW2}}}} % Divakar D1, lower streamwise
\newcommand{\tTWthree}{\ensuremath{{\text{TW3}}}}  % upper streamwise


\newcommand{\bCell}{\ensuremath{\Omega}}
\newcommand{\bNarrow}{\ensuremath{\Omega_{\text{\tiny W03}}}}
    % JG: W02 for Waleffe Tokyo proceedings 2002, where this cell first appears,
    % ok'd by wally
\newcommand{\bHKW}{\ensuremath{\Omega_{\text{\tiny{HKW}}}}}
\newcommand{\bSch}{\ensuremath{\Omega_{\text{\tiny{Sch}}}}} % Schmiegel
\newcommand{\bbR}{\mathbb{R}}
%\newcommand{\bbU}{\mathbb{U}}
%\newcommand{\bbUsymm}{\ensuremath{\bbU_{S}}}
\newcommand{\bbUS}{\ensuremath{\bbU_{S}}}
\newcommand{\bbUthree}{\ensuremath{\bbU_{s3}}}

% PC \eqva velocity field naming conventions
\newcommand{\uEQ}{\ensuremath{\bu_{\text{\tiny EQ}}}}
\newcommand{\uLM}{\ensuremath{\bu_{\text{\tiny EQ0}}}}
\newcommand{\uLB}{\ensuremath{\bu_{\text{\tiny EQ1}}}}
\newcommand{\uUB}{\ensuremath{\bu_{\text{\tiny EQ2}}}}
\newcommand{\uNNB}{\ensuremath{\bu_{\text{\tiny EQ3}}}}
\newcommand{\uNB}{\ensuremath{\bu_{\text{\tiny EQ4}}}}
\newcommand{\uEQfive}{\ensuremath{\bu_{\text{\tiny EQ5}}}}
\newcommand{\uEQsix}{\ensuremath{\bu_{\text{\tiny EQ6}}}}
\newcommand{\uEQsev}{\ensuremath{\bu_{\text{\tiny EQ7}}}}
\newcommand{\uEQeight}{\ensuremath{\bu_{\text{\tiny EQ8}}}}
\newcommand{\uEQnine}{\ensuremath{\bu_{\text{\tiny EQ9}}}}
\newcommand{\uEQten}{\ensuremath{\bu_{\text{\tiny EQ10}}}}

\newcommand{\GPKF}{\ensuremath{\Gamma}} % Hoyle notation, equivariant symmetry group
\newcommand{\trDiscr}[2]{\tau_{#1}^{#2}}    % discrete cell translation 1/4, ...
% isotropy subgroup $H \incl G$:
\newcommand{\isotropyG}[1]{\ensuremath{H_{\text{\tiny #1}}}}


%%%%%%%%%%%%%%%%%%% eventually remove these %%%%%%%%%%%%%%%%%%%%%%%%%%%%%
\newcommand{\huUB}{\ensuremath{\hbu_{\text{\tiny EQ2}}}}
\newcommand{\hbu}{\tilde{{\bf u}}}
\newcommand{\hbv}{\tilde{{\bf v}}}
\newcommand{\hu}{\tidle{u}}
\newcommand{\hv}{\tidle{v}}
\newcommand{\vc}{\mathbf}
