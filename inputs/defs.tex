% defs.tex  for Elton
% $Author: predrag $ $Date: 2014-11-07 16:41:13 -0500 (Fri, 07 Nov 2014) $

% Predrag               Oct 13 2007
% extracted from ChaosBook.org, vi Halcrow blog

% Predrag               Oct  3 2013
\newcommand{\PCpost}[2]{\item[#1 Predrag] {#2}}
\newcommand{\APWpost}[2]{\item[#1 Ashley] {#2}}
\newcommand{\KYSpost}[2]{\item[#1 Kimberly] {#2}}
\newcommand{\BEpost}[2]{\item[#1 Bruno] {#2}}
\newcommand{\FFpost}[2]{\item[#1 Franco] {#2}}
\newcommand{\MMFpost}[2]{\item[#1 Mohammad] {#2}}
\newcommand{\AFpost}[2]{\item[#1 Adam] {#2}}
\newcommand{\YLpost}[2]{\item[#1 Lan] {#2}}

%%%%%%%%%%%%%%% SECTIONS, SLICES %%%%%%%%%%%%%%%%%%%%%%%%%%%%%%%%%

\newcommand{\Poincare}{Poincar\'e }
\newcommand{\PoincSec}{Poincar\'e section}
% \newcommand{\reducedsp}{orbit space}
% \newcommand{\Reducedsp}{Orbit space}
\newcommand{\reducedsp}{reduced state space}
\newcommand{\Reducedsp}{Reduced state space}
\newcommand{\fixedsp}{fixed-point subspace}
\newcommand{\Fixedsp}{Fixed-point subspace}
\newcommand{\csection}{cross-section} % eventually eliminate
\newcommand{\Csection}{Cross-section} % eventually eliminate
\newcommand{\slice}{slice}
\newcommand{\Slice}{Slice}
\newcommand{\mslices}{method of slices}
\newcommand{\Mslices}{Method of slices}
\newcommand{\mframes}{method of moving frames}
\newcommand{\Mframes}{Method of moving frames}
\newcommand{\templates}{templates} % {slice-fixing point} % {reference state}
\newcommand{\movframe}{moving frame}
\newcommand{\movFrame}{Moving frame}
\newcommand{\comovframe}{comoving frame}
\newcommand{\comovFrame}{Comoving frame}
% \newcommand{\mconn}{method of connections}
% \newcommand{\Mconn}{Method of connections}
\newcommand{\mconn}{method of \comovframe s}
\newcommand{\Mconn}{Method of \comovframe s}
\newcommand{\chartBord}{chart border}
\newcommand{\ChartBord}{Chart border}
\newcommand{\poincBord}{section border}
\newcommand{\PoincBord}{Section border}
% \newcommand{\poincBord}{\PoincSec\ border}
% \newcommand{\PoincBord}{\PoincSec\ border}
% \newcommand{\poincBord}{border of transversality}
\newcommand{\template}{template} % {slice-fixing point} % {reference state}
\newcommand{\sliceBord}{slice border}
\newcommand{\SliceBord}{Slice border}
\newcommand{\slicePlane}{slice hyperplane}
\newcommand{\SlicePlane}{Slice hyperplane}
\newcommand{\Sset}{Inflection hyperplane}
\newcommand{\sset}{inflection hyperplane} 	% {singularity hyperplane}
											% {singular set}
\newcommand{\pSRed}{\ensuremath{\hat{\cal M}}} % reduced state space Jan 2012
%\newcommand{\pSRed}{\ensuremath{\bar{\cal M}}} % reduced state space
\newcommand{\sspRed}{\ensuremath{\hat{\ssp}}}    % reduced state space point Jan 2012
% \newcommand{\sspRed}{\ensuremath{y}}    % reduced state space point, experiment
% \newcommand{\sspRed}{\ensuremath{\bar{x}}}    % reduced state space point
\newcommand{\velRed}{\ensuremath{\hat{\vel}}}    % ES reduced state space velocity Jan 2012
% \newcommand{\velRed}{\ensuremath{\bar{v}}}    % PC reduced state space velocity
% \newcommand{\velRed}{\ensuremath{u}}    % ES reduced state space velocity

\newcommand{\slicep}{{\ensuremath{\sspRed'}}}   % slice-fixing point Jan 2012
% \newcommand{\slicep}{{\ensuremath{y'}}}   % slice-fixing point, experimental
% \newcommand{\slicep}{\ensuremath{\ssp'}}   % slice-fixing point
%\newcommand{\sliceTan}[1]{\ensuremath{t_{#1}(y')}}    % tangent at slice-fixing, experimental
\newcommand{\sliceTan}[1]{\ensuremath{t'_{#1}}}    % group orbit tangent at slice-fixing
\newcommand{\groupTan}{\ensuremath{t}}    % group orbit tangent
%\newcommand{\Group}{\ensuremath{\Gamma}}    % Siminos Lie group
\newcommand{\Group}{\ensuremath{G}}         % Predrag Lie or discrete group
%\newcommand{\Lg}{\mathfrak{a}}             % Siminos Lie algebra generator
\newcommand{\Lg}{\ensuremath{\mathbf{T}}}   % Predrag Lie algebra generator
%\newcommand{\LieEl}{\ensuremath{\mathbb{G}}}  % Wiczek project Lie group element
\newcommand{\LieEl}{\ensuremath{g}}  % Predrag Lie group element

\newcommand{\zeit}{\ensuremath{t}}  %time variable Ashley
\newcommand{\sspSing}{\ensuremath{\ssp^\ast}} 	% inflection point
\newcommand{\sspRSing}{\ensuremath{\sspRed^\ast}} 	% inflection point, reduced space

%%%%%%%%%%%%%%% LIE GROUP PARAMETRIZATIONS %%%%%%%%%%%%%%%%%%%%%%
\newcommand{\gSpace}{\ensuremath{{\bf \phi}}}   % MA group rotation parameters
% \newcommand{\gSpace}{\ensuremath{{\bf \theta}}}   % PC group rotation parameters
\newcommand{\velRel}{\ensuremath{c}}    % relative state or phase velocity
\newcommand{\angVel}{angular velocity}      % Froehlich
\newcommand{\angVels}{angular velocities}   % Froehlich
\newcommand{\phaseVel}{phase velocity}      % pipe slicing
\newcommand{\phaseVels}{phase velocities}   % pipe slicing
\newcommand{\PhaseVel}{Phase velocity}      % pipe slicing
\newcommand{\PhaseVels}{Phase velocities}   % pipe slicing

%%%%%%%% Siminos macros %%%%%%%%%%%%%%%%%%%%%%%%%%%%%%
\newcommand{\Rls}[1]{\ensuremath{\mathbb{R}^{#1}}}
%\newcommand{\Idg}{\ensuremath{\mathbf{1}}}
%\newcommand{\Clx}[1]{\ensuremath{\mathbb{C}^{#1}}}
\newcommand{\ii}{\ensuremath{\mathrm{i}}} % sqrt{-1}
%\newcommand{\conj}[1]{\ensuremath{\bar{#1}}}
%\newcommand{\trace}{\mbox{\rm trace}\,}
\newcommand{\Un}[1]{\ensuremath{\textrm{U}(#1)}}         % in DasBuch
\newcommand{\SUn}[1]{\ensuremath{\textrm{SU}(#1)}}         % in DasBuch
%\newcommand{\On}[1]{\ensuremath{\mathbf{O}(#1)}}
\newcommand{\On}[1]{\ensuremath{\textrm{O}(#1)}}
%\newcommand{\SOn}[1]{\ensuremath{\mathbf{SO}(#1)}} % in Siminos thesis
\newcommand{\SOn}[1]{\ensuremath{\textrm{SO}(#1)}}         % in DasBuch
\newcommand{\Spn}[1]{\ensuremath{\textrm{Sp}(#1)}}         % in DasBuch
%\newcommand{\Dn}[1]{\ensuremath{\mathbf{D}_{#1}}    % in Siminos thesis
\newcommand{\Dn}[1]{\ensuremath{\textrm{D}_{#1}}}              % in DasBuch
%\newcommand{\Zn}[1]{\ensuremath{\mathbf{Z}_{#1}}}    % in Siminos thesis
\newcommand{\Zn}[1]{\ensuremath{\textrm{C}_{#1}}}              % in DasBuch
%\newcommand{\Ztwo}{\ensuremath{\mathbf{Z}_2}}      % in Siminos thesis
\newcommand{\Ztwo}{\ensuremath{\textrm{C}_2}}                % in DasBuch
%\newcommand{\Refl}{\ensuremath{\kappa}}            % Siminos uses R for rotations.
\newcommand{\Refl}{\ensuremath{\sigma}}             % in DasBuch
%\newcommand{\Shift}{\ensuremath{\tau}}
\newcommand{\Rot}[1]{\ensuremath{C^{#1}}}           % in DasBuch, e.g. C^{1/3}
%\newcommand{\Rot}[1]{\ensuremath{R(#1)}}           % Siminos uses R for rotations.
%\newcommand{\Drot}{\ensuremath{\zeta}}
%\newcommand{\Lg}{\mathcal{G}}
%\newcommand{\stab}[1]{\ensuremath{\Sigma_{#1}}}
\newcommand{\stab}[1]{\ensuremath{G_{#1}}}
\newcommand{\shift}{\ensuremath{d}}
\newcommand{\Fix}[1]{\ensuremath{\mathrm{Fix}\left(#1\right)}}

%%%%%%%%%%%%%%% symmetric, asymmetric orbits: %%%%%%%%%%%%%%%%%%%%%%%%%%%%
\newcommand{\sym}{{s}}
\newcommand{\nsym}{{n_s}}
\newcommand{\asym}{{a}}
\newcommand{\nasym}{{n_a}}
% fundamental domain:
\newcommand{\pf}{{\tilde p}}
\newcommand{\nf}{n_{\tilde p}}
\newcommand{\symf}{{\tilde s}}
\newcommand{\nsymf}{n_{\tilde s}}
%\newcommand\stagn{*}        %equilibrium/stagnation point suffix
\newcommand\stagn{q}      %equilibrium/stagnation point suffix
\newcommand{\rpprime}{{\tilde{p}}}  % relative periodic prime orbit



\newcommand{\pSpace}{x}       % Hamiltonian phase space x=(q,p) coordinate
\newcommand{\Lint}[1]{\frac{1}{L}\!\oint d#1\,}
\newcommand{\expctE}{\ensuremath{E}}    % E space averaged

\newcommand{\twoMode}{Porter-Knobloch}
\newcommand{\cycle}[1]{{\ensuremath{\overline{#1}}}}
\newcommand{\tildeL}{\ensuremath{\tilde{L}}}

%%%%%%%%%%%%    CROSS REFERENCING, STANDARD    %%%%%%%%%%%%%%%%%

\newcommand{\rf}      [1] {~\cite{#1}}
\newcommand{\refref}  [1] {ref.~\cite{#1}}
\newcommand{\refRef}  [1] {Ref.~\cite{#1}}
\newcommand{\refrefs} [1] {refs.~\cite{#1}}
\newcommand{\refRefs} [1] {Refs.~\cite{#1}}
\newcommand{\refeq}   [1] {(\ref{#1})}
\newcommand{\refeqs}  [2] {(\ref{#1}--\ref{#2})}
\newcommand{\refpage} [1] {page~\pageref{#1}}
    % Phys Rev style: Figure to start a sentence, else Fig.
\newcommand{\reffig}  [1] {Fig.~\ref{#1}}
\newcommand{\reffigs} [2] {Figs.~\ref{#1} and~\ref{#2}}
\newcommand{\refFig}  [1] {Figure~\ref{#1}}
\newcommand{\refFigs}  [2] {Figures~\ref{#1} and~\ref{#2}}
\newcommand{\reftab}  [1] {Table~\ref{#1}}
\newcommand{\refTab}  [1] {Table~\ref{#1}}
\newcommand{\reftabs} [2] {Tables~\ref{#1} and~\ref{#2}}
\newcommand{\refsect} [1] {sect.~\ref{#1}}
\newcommand{\refsects}[2] {sects.~\ref{#1}--\ref{#2}}
\newcommand{\refSect} [1] {Sect.~\ref{#1}}
\newcommand{\refSects}[2] {Sects.~\ref{#1}--\ref{#2}}
\newcommand{\refchap}[1] {chapter~\ref{#1}}
\newcommand{\refChap}[1] {Chapter~\ref{#1}}
\newcommand{\refchaps}[2] {chapters~\ref{#1} and \ref{#2}}
\newcommand{\refchaptochap}[2] {chapters~\ref{#1} to \ref{#2}}
\newcommand{\refappe}[1] {appendix~\ref{#1}}
\newcommand{\refappes}[2] {appendices~\ref{#1} and \ref{#2}}
\newcommand{\refAppe}[1] {Appendix~\ref{#1}}
\newcommand{\refrem} [1] {remark~\ref{#1}}
\newcommand{\refexam}[1] {example~\ref{#1}}
\newcommand{\refExam}[1] {Example~\ref{#1}}

%%%%%%%%%%%%%%% EQUATIONS, STANDARD %%%%%%%%%%%%%%%%%%%%%%%%%%%%%%%

\newcommand{\beq}{\begin{equation}}
\newcommand{\eeq}{\end{equation}}
\newcommand{\ee}[1] {\label{#1} \end{equation}}
\newcommand{\bea}{\begin{eqnarray}}
\newcommand{\ceq}{\nonumber \\ & & }
\newcommand{\continue}{\nonumber \\ }
\newcommand{\nnu}{\nonumber}
\newcommand{\eea}{\end{eqnarray}}
\newcommand{\barr}{\begin{array}}
\newcommand{\earr}{\end{array}}


%%%%%%%%%%%%%%%%%%%%%% QUOTATIONS %%%%%%%%%%%%%%%%%%%%%%%%%%%%%%%%%%%%%%
%
%  the learned/witty quotes at the chapter and section headings
%  (liberated from Das Buch defs.tex)
\newsavebox{\bartName}
\newcommand{\bauthor}[1]{\sbox{\bartName}{\parbox{\textwidth}{\vspace*{0.8ex}
       %\hspace*{\fill}
       \small\noindent #1}}}
\newenvironment{bartlett}{\hfill\begin{minipage}[t]{0.65\textwidth}\small}%
{\hspace*{\fill}\nolinebreak[1]\usebox{\bartName}\vspace*{1ex}\end{minipage}}

%
%  a quotation inserted into the text
%
\newenvironment{txtquote}{\begin{quotation} \small}{\end{quotation}}


%%%%%%%%%%%%%%%%%%%%% JH Math Shortcuts %%%%%%%%%%%%%%%%%%%%%%%%%%%%%%%%%
\newcommand{\ident}{\mathbf{1}}
\newcommand{\vect}{\mathbf}
\newcommand{\matr}{\mathit}

%%%%%%%%%%%% SHORTCUTS, project specific %%%%%%%%%%


\newcommand{\bu}{\ensuremath{{\bf u}}}
\newcommand{\bx}{\ensuremath{{\bf x}}}
\newcommand{\butot}{\ensuremath{{\bf u_{tot}}}}
\newcommand{\bnabla}{\ensuremath{{\bf \nabla}}}
\newcommand{\lapl}{\ensuremath{{\nabla^{2}}}}

\newcommand{\newfp}{\ensuremath{\bu_{\text{\tiny NB}}}}
\newcommand{\Newfp}{\ensuremath{\bu_{\text{\tiny NB}}}}

\newcommand{\NS}{Navier-Stokes}
\newcommand{\NSe}{Navier-Stokes equation}
\newcommand{\KS}{Kuramoto-Sivashinsky}
\newcommand{\KSe}{Kuramoto-Sivashinsky equation}
\newcommand{\Reynolds}{\ensuremath{\textit{Re}}}  % Reynolds number
\newcommand{\pCf}{plane Couette flow}
\newcommand{\PCf}{Plane Couette flow}
\newcommand{\ubranch}{upper branch}
\newcommand{\Ubranch}{Upper branch}
\newcommand{\lbranch}{lower branch}
\newcommand{\Lbranch}{Lower branch}

%%% 3D physical flow
\newcommand{\stagp}{stagnation point}
\newcommand{\Stagp}{Stagnation point}
\newcommand{\relstagp}{traveling stagnation point}
\newcommand{\Relstagp}{Traveling stagnation point}
\newcommand{\velgradmat}{matrix of velocity gradients}
\newcommand{\Velgradmat}{Matrix of velocity gradients}
\newcommand{\vel}{\ensuremath{v}}   % state space velocity


%%%% dynamical systems nomenclature:
\newcommand{\stabmat}{stability matrix}
\newcommand{\eqv}{equilibrium}
\newcommand{\Eqv}{equilibrium}
\newcommand{\eqva}{equilibria}
\newcommand{\Eqva}{Equilibria}
\newcommand{\eqpoint}{equilibrium point}
\newcommand{\Eqpoint}{Equilibrium point}
\newcommand{\reqv}{relative equilibrium}
\newcommand{\Reqv}{Relative equilibrium}
\newcommand{\reqva}{relative equilibria}
\newcommand{\Reqva}{Relative equilibria}
%\newcommand{\reqv}{equivariant equilibrium}
%\newcommand{\Reqv}{Equivariant equilibrium}
%\newcommand{\reqva}{equivariant equilibria}
%\newcommand{\Reqva}{Equivariant equilibria}
%\newcommand{\equilibrium}{equilibrium}
%\newcommand{\equilibria}{equilibria}
%\newcommand{\Equilibria}{Equilibria}

%%%% fluid dynamics nomenclature:

% \newcommand{\eqv}{steady state}
% \newcommand{\Eqv}{Steady state}
% \newcommand{\eqva}{steady states}
% \newcommand{\Eqva}{Steady states}
% \newcommand{\reqv}{travelling wave}
% \newcommand{\Reqv}{Travelling wave}
% \newcommand{\reqva}{travelling waves}
% \newcommand{\Reqva}{Travelling waves}
% \newcommand{\equilibrium}{steady state}
% \newcommand{\equilibria}{steady states}
% \newcommand{\Equilibria}{Steady states}

\newcommand{\po}{periodic orbit}
\newcommand{\Po}{Periodic orbit}
\newcommand{\rpo}{relative periodic orbit}
%   \newcommand{\rpo}{equivariant periodic orbit}
\newcommand{\Rpo}{Relative periodic orbit}
%   \newcommand{\Rpo}{Equivariant periodic orbit}
\newcommand{\UPO}{unstable periodic orbit}
\newcommand{\Hec}{Heteroclinic connection}
\newcommand{\hec}{heteroclinic connection}

\newcommand{\cohStr}{coherent state}
\newcommand{\recurrStr}{recurrent coherent state}
\newcommand{\RecurrStr}{Recurrent coherent state}

\newcommand{\statesp}{state space}
\newcommand{\Statesp}{State space}
\newcommand{\nameit}{E}         % equilibrium label
\newcommand{\SIS}{non-wondering set}
\newcommand{\descent}{Newton descent}
\newcommand{\Descent}{Newton Descent}
\newcommand{\jacobian}{Jacobian}        % determinant
%\newcommand{\jacobianM}{fundamental matrix}     % standard name
%\newcommand{\jacobianMs}{fundamental matrices}  %
%\newcommand{\JacobianM}{Fundamental matrix}     %
%\newcommand{\JacobianMs}{Fundamental matrices}  %
\newcommand{\jacobianM}{Jacobian matrix}        % matrix
\newcommand{\jacobianMs}{Jacobian matrices}     % matrices
\newcommand{\JacobianM}{Jacobian matrix}        % matrix
\newcommand{\JacobianMs}{Jacobian matrices}     % matrices

\newcommand{\ew}{eigen\-value}
\newcommand{\Ew}{eigen\-value}
\newcommand{\ev}{eigen\-vector}
\newcommand{\Ev}{eigen\-value}
\newcommand{\ef}{eigen\-function}
\newcommand{\Ef}{eigen\-value}
\newcommand{\steady}{\marginpar{{\color{green}\textdollar}}}


\newcommand{\evOper}{evolution oper\-ator}
\newcommand{\EvOper}{Evolution oper\-ator}
\newcommand{\Fd}{spec\-tral det\-er\-min\-ant}
\newcommand{\fd}{spec\-tral det\-er\-min\-ant}
\newcommand{\FD}{Spec\-tral det\-er\-min\-ant}

%%% 3D physical flow naming conventions
% PC \teq is the name, \seq is the  symbol
\newcommand{\teq}[1]{\ensuremath{{\text{eq#1}}}}
\newcommand{\teqa}{\ensuremath{{\text{eq0}}}}
\newcommand{\teqb}{\ensuremath{{\text{eq1}}}}
\newcommand{\teqc}{\ensuremath{{\text{eq2}}}}

\newcommand{\ttw}[1]{\ensuremath{{\text{tw#1}}}}
\newcommand{\ttwa}{\ensuremath{{\text{tw1}}}}  % spanwise
\newcommand{\ttwb}{\ensuremath{{\text{tw2}}}} % Divakar D1, lower streamwise
\newcommand{\ttwc}{\ensuremath{{\text{TW3}}}}  % upper streamwise

\newcommand{\xeq}[1]{\ensuremath{\bx_{\text{\tiny eq#1}}}}
\newcommand{\xeqa}{\ensuremath{\bx_{\text{\tiny eq0}}}}
\newcommand{\xeqb}{\ensuremath{\bx_{\text{\tiny eq1}}}}
\newcommand{\xeqc}{\ensuremath{\bx_{\text{\tiny eq2}}}}

\newcommand{\xtw}[1]{\ensuremath{\bx_{\text{\tiny tw#1}}}(t)}
\newcommand{\xtwa}{\ensuremath{\bx_{\text{\tiny tw1}}}(t)}
\newcommand{\xtwb}{\ensuremath{\bx_{\text{\tiny tw2}}}(t)}
\newcommand{\xtwc}{\ensuremath{\bx_{\text{\tiny tw3}}}(t)}

%%%%%%%%%%%%%%%%%%%%%%%%%%%%%%%%%%%%%%%%%%%%%%%%%%%%%%%%%%%
% PC: experimental, for stagnation points
\newcommand{\tSP}[1]{{\ensuremath{{\text{SP#1}}}}}
\newcommand{\tSPone}{\ensuremath{{\text{SP1}}}}
\newcommand{\tSPtwo}{\ensuremath{{\text{SP2}}}}
\newcommand{\tSPthr}{\ensuremath{{\text{SP3}}}}
\newcommand{\xSP}[1]{{\ensuremath{\bx_{\text{\tiny SP#1}}}}}
\newcommand{\xSPone}{\ensuremath{\bx_{\text{\tiny SP1}}}}
\newcommand{\xSPtwo}{\ensuremath{\bx_{\text{\tiny SP2}}}}
\newcommand{\xSPthr}{\ensuremath{\bx_{\text{\tiny SP3}}}}


%%%%%%%%%%%%%%%%%%%%%%%%%%%%%%%%%%%%%%%%%%%%%%%%%%%%%%%%%%%
% JFG: Fluids literature uses bold to indicate vector
% quantities, so that one can use {\bf u} for the vector
% and u for the first component of the vector.

% PC \eqva/\reqva directory: \tAA is the name, \sAA is the  symbol
\newcommand{\EQV}[1]{\ensuremath{\text{EQ#1}}} % ELIMINATE
\newcommand{\tEQ}[1]{\ensuremath{{\text{EQ#1}}}}
\newcommand{\tLM}{\ensuremath{{\text{EQ0}}}}
\newcommand{\tLB}{\ensuremath{{\text{EQ1}}}}
\newcommand{\tUB}{\ensuremath{{\text{EQ2}}}}
\newcommand{\tNNB}{\ensuremath{{\text{EQ3}}}}
\newcommand{\tNB}{\ensuremath{{\text{EQ4}}}}
\newcommand{\tEQfive}{\ensuremath{{\text{EQ5}}}}
\newcommand{\tEQsix}{\ensuremath{{\text{EQ6}}}}
\newcommand{\tEQsev}{\ensuremath{{\text{EQ7}}}}
\newcommand{\tEQeight}{\ensuremath{{\text{EQ8}}}}
\newcommand{\tEQnine}{\ensuremath{{\text{EQ9}}}}
\newcommand{\tEQten}{\ensuremath{{\text{EQ10}}}}

\newcommand{\REQV}[2]{{\ensuremath{\text{TW$#1#2$}}}}
\newcommand{\tTW}[1]{\ensuremath{{\text{TW#1}}}}
\newcommand{\tTWone}{\ensuremath{{\text{TW1}}}}  % spanwise
\newcommand{\tTWDone}{\ensuremath{{\text{TW2}}}} % Divakar D1, lower streamwise
\newcommand{\tTWthree}{\ensuremath{{\text{TW3}}}}  % upper streamwise

% PC \eqva labeling symbols for all figures: halcrow/figsSrc/drawsyms.tex
\newcommand{\sLM}{\ensuremath{\odot}}
\newcommand{\sLB}{\colorcomm{{\Large \ensuremath{\color{blue}\circ}}}
                            {{\Large \ensuremath{\circ}}}}
\newcommand{\sUB}{\colorcomm{{\Large \ensuremath{\color{blue}\bullet}}}
                            {{\Large \ensuremath{\bullet}}}}
\newcommand{\sNNB}{\colorcomm{{\scriptsize \ensuremath{\color{red}\square}}}
                            {{\scriptsize \ensuremath{\square}}}}
\newcommand{\sNB}{\colorcomm{{\scriptsize \ensuremath{\color{red}\blacksquare}}}
                            {{\scriptsize \ensuremath{\blacksquare}}}}
\newcommand{\sEQfive}{\colorcomm{\ensuremath{\color{green}\lozenge}}
                            {\ensuremath{\lozenge}}}
\newcommand{\sEQsix}{\colorcomm{\ensuremath{\color{green}\blacklozenge}}
                            {\ensuremath{\blacklozenge}}}
\newcommand{\sEQsev}{\colorcomm{\ensuremath{\blacktriangleleft}}
                            {\ensuremath{\blacktriangleleft}}}
\newcommand{\sEQeight}{\colorcomm{\ensuremath{\triangleleft}}
                            {\ensuremath{\triangleleft}}}
\newcommand{\sEQnine}{\colorcomm{\ensuremath{\blacktriangleright}}
                            {\ensuremath{\blacktriangleright}}}
\newcommand{\sEQten}{\colorcomm{\ensuremath{\triangleright}}
                            {\ensuremath{\triangleright}}}


\newcommand{\sTWone}{\colorcomm{{\large \ensuremath{\color{blue}\triangleleft}}}
                            {{\large \ensuremath{\triangleleft}}}}
\newcommand{\sTWDone}{\colorcomm{\ensuremath{\color{red}\vartriangle}}
                            {\ensuremath{\vartriangle}}}
\newcommand{\sTWthree}{\colorcomm{\ensuremath{\color{green}\blacktriangle}}
                            {\ensuremath{\blacktriangle}}}

% PC \eqva velocity field naming conventions
\newcommand{\uEQ}{\ensuremath{\bu_{\text{\tiny EQ}}}}
\newcommand{\uLM}{\ensuremath{\bu_{\text{\tiny EQ0}}}}
\newcommand{\uLB}{\ensuremath{\bu_{\text{\tiny EQ1}}}}
\newcommand{\uUB}{\ensuremath{\bu_{\text{\tiny EQ2}}}}
\newcommand{\uNNB}{\ensuremath{\bu_{\text{\tiny EQ3}}}}
\newcommand{\uNB}{\ensuremath{\bu_{\text{\tiny EQ4}}}}
\newcommand{\uEQfive}{\ensuremath{\bu_{\text{\tiny EQ5}}}}
\newcommand{\uEQsix}{\ensuremath{\bu_{\text{\tiny EQ6}}}}
\newcommand{\uEQsev}{\ensuremath{\bu_{\text{\tiny EQ7}}}}
\newcommand{\uEQeight}{\ensuremath{\bu_{\text{\tiny EQ8}}}}
\newcommand{\uEQnine}{\ensuremath{\bu_{\text{\tiny EQ9}}}}
\newcommand{\uEQten}{\ensuremath{\bu_{\text{\tiny EQ10}}}}

\newcommand{\uREQV}{\ensuremath{\bu_{\text{\tiny TW}}}}
\newcommand{\uTW}{\ensuremath{\bu_{\text{\tiny TW}}}}
\newcommand{\uTWone}{\ensuremath{\bu_{\text{\tiny TW1}}}}
\newcommand{\uTWDone}{\ensuremath{\bu_{\text{\tiny TW2}}}}
\newcommand{\uTWthree}{\ensuremath{\bu_{\text{\tiny TW3}}}}

\newcommand{\bCell}{\ensuremath{\Omega}}
\newcommand{\bNarrow}{\ensuremath{\Omega_{\text{\tiny W03}}}}
    % JG: W02 for Waleffe Tokyo proceedings 2002, where this cell first appears,
    % ok'd by wally
\newcommand{\bHKW}{\ensuremath{\Omega_{\text{\tiny{HKW}}}}}
\newcommand{\bSch}{\ensuremath{\Omega_{\text{\tiny{Sch}}}}} % Schmiegel
\newcommand{\bbR}{\mathbb{R}}
\newcommand{\bbU}{\mathbb{U}}
\newcommand{\bbUsymm}{\ensuremath{\bbU_{S}}}
\newcommand{\bbUS}{\ensuremath{\bbU_{S}}}
\newcommand{\bbUthree}{\ensuremath{\bbU_{s3}}}

%%%%%multiletter symbols
\newcommand\Real{\mbox{Re}} % cf plain TeX's \Re, not Reynolds number
\newcommand\Imag{\mbox{Im}} % cf plain TeX's \Im

%%%%%%%%%%%%%%% Sundry symbols within math eviron.: %%%%%%%%%%%%
\newcommand{\pd}[2]{\frac{\partial #1}{\partial #2}}
\newcommand{\grad}{{\bf \nabla}}
\newcommand{\half}{\frac{1}{2}}
\newcommand{\Norm}[1]{\|{#1}\|}
\newcommand\flow[2]{{f^{#1}(#2)}}
\newcommand\timeflow{{f^t}}
\newcommand{\reals}{\mathbb{R}}
\newcommand{\PoincS}{{\cal P}}     % symbol for Poincare section
\newcommand{\PoincM}{{P}}      % symbol for Poincare map
\newcommand{\PoincC}{{U}}      % symbol for Poincare constraint function
\newcommand{\pde}{\partial}
\newcommand{\jMps}{{\bf J}}    % jacobiam matrix, full phase space
\newcommand{\Mvar}{{A}}         % matrix of variations
\newcommand{\jMP}{{\bf \hat{J}}}   % jacobiam matrix, Poincare return
\newcommand{\jEigvec}[1][]{\ensuremath{{\bf e}^{(#1)}}} % jacobiam eigenvector
\newcommand{\jEigvecT}[1]{\ensuremath{{\bf e}_{(#1)}}}   % jacobiam eigenvector transposed

\newcommand{\monodromy}{{\bf J}}   % monodromy matrix, full Poincare cut
                   % Fredholm det jacobian weight:
\newcommand{\oneMinJ}[1]
           {\left|\det\!\left(\matId-\monodromy_p^{#1}\right)\right|}

\newcommand{\dmn}{\ensuremath{\!-\!d}}             %  n-dimensional

\newcommand{\obser}{a}      % an observable from state space to R^n
\newcommand{\Obser}{A}      % time integral of an observable
\newcommand{\expct}    [1]{\left\langle {#1} \right\rangle}
\newcommand{\spaceAver}[1]{\left\langle {#1} \right\rangle}
\newcommand{\timeAver} [1]{\overline{#1}}
\newcommand{\Lop}{{\cal L}}    % evolution operator
\renewcommand\Im{{\rm Im\,}}
\renewcommand\Re{{\rm Re\,}}
\renewcommand{\det}{\mbox{\rm det}\,}
\newcommand{\Det}{\mbox{\rm Det}\,}
\newcommand{\tr}{\mbox{\rm tr}\,}
\newcommand{\Tr}{\mbox{\rm tr}\,}
\newcommand{\Shift}{\ensuremath{\mathbf{S}}}
\newcommand{\trHalf}[1]{\tau_{#1}}    % 1/2 cell translation
\newcommand{\trDiscr}[2]{\tau_{#1}^{#2}}    % discrete cell translation 1/4, ...
\newcommand\period[1]{{T_{#1}}}         %continuous cycle period
\newcommand{\cl}[1]{{n_{#1}}}   % discrete length of a cycle, Predrag
\newcommand{\pS}{{\cal M}}          % symbol for state space
% \newcommand\pSpace{x}     % phase space x=(q,p) coordinate
\newcommand{\ssp}{a}            % state space point
\newcommand\velField[1]{{F(#1)}}    % Gibson statespace velocity field
\newcommand\xInit{{a_0}}        %initial x
\newcommand{\deltaX}{{\delta a}}                %trajectory displacement
\newcommand{\msr}{\ensuremath{\rho}}                % measure
\newcommand{\DiffC}{\ensuremath{D}}     % diffusion constant

\newcommand{\costFct}{cost function}    % functional to minimize
\newcommand{\costF}{F^2}        % cost function,
\newcommand{\Loop}{L}
\newcommand{\pVeloc}{v}         % phase-space velocity
\newcommand{\lSpace}{\tilde{x}}     % a point on a loop
\newcommand{\lVeloc}{\tilde{v}}     % loop tangent
\newcommand{\damp}{\Delta\tau}      % descrete fititous time step
\newcommand{\prpgtr}[1]{\delta\negthinspace\left( {#1} \right)}
\newcommand{\matId}{{\bf 1}}       % matrix identity
\newcommand{\inertM}{{\mathcal M}}          % inertial manifold
\newcommand{\ExpaEig}{\Lambda}
\newcommand\Lyap{\lambda}                       %Lyapunov exponent
\newcommand{\eigenvL}{{s}}
\newcommand{\eigenvG}{{m}}         % compact group eigenvalues

%%       optional parameter comes in [\ldots], for example
%%       \newcommand\eigRe[1][ ]{\ensuremath{\mu_{#1}}}
%%       no subscript: \eigRe\
%%       with subscript j: \eigRe[j]
%%
%%      Guckenheimer-Holmes:  lambda = alpha + i beta
%%      Hirsch-Smale:         lambda = a     + i b
%%      Boyce-di Prima:       lambda = mu    + i nu
%%      Gibson:        lambda = mu    + i omega (best of the bunch!)
%
% Re eigen-exponent superscripting
% Getting into the ChaosBookie groove... awesome!
% The groove is groovy when the macros reduce typing...

\newcommand{\eigExp}[1][]{
\ifthenelse{\equal{#1}{}}{\ensuremath{\lambda}}{\ensuremath{\lambda^{(#1)}}}
                        }
\newcommand{\eigRe}[1][]{
\ifthenelse{\equal{#1}{}}{\ensuremath{\mu}}{\ensuremath{\mu^{(#1)}}}
                        }
\newcommand{\eigIm}[1][]{
  \ifthenelse{\equal{#1}{}}{\ensuremath{\omega}}{\ensuremath{\omega^{(#1)}}}
            }

\newcommand{\tny}[1]{{\text{\tiny {#1}}}}

% Guck & Holmes use $W^s$, $W^u$ for stable, unstable manifolds.
% usage: \Wmnfld{u,(n)}{NB} unstable manifold of NB's nth eigenvalue.

\newcommand{\Wmnfld}[2]{%
\ifthenelse{\equal{#2}{}}{\ensuremath{W_{#1}}\!}
                         {\ensuremath{W^{#1}_{\text{\tiny #2}}}\!} %Negative space is screwing up spacing in text
                        }

\newcommand{\derF}[1]{{DF |_{#1}}}        % Gibson stability matrix

%%%%%%%%%%%%%%  Abbreviations %%%%%%%%%%%%%%%%%%%%%%%%%%%%%%%%%%%%%%%%
%%% APS (American Physiology Society, it seems) style:
%%%     Latin or foreign words or phrases should be roman, not italic.

\newcommand{\etc}{{etc.}}       % APS
\newcommand{\etal}{{\em et al.}}    % etal in italics, APS too
\newcommand{\cf}{{\em cf.}}     % APS
\newcommand{\eg}{{e.g.}}        % APS
% \newcommand{\etc}{{\em etc.}}     % etcetera in italics
% \newcommand{\ie}{{that is}}       % use Latin or English?  Decide later.
% \newcommand{\cf}{{cf.}}
% \newcommand{\eg}{{\it e.g.,\ }}   % Wirzba 2sep2001

%%%%%%%%%%%%%%% WALLY's FAVORITE MACROS %%%%%%%%%%%%%%%%%%%%%%
\newcommand{\bvec}[1]{\boldsymbol{#1}}
\def\xh{\mathbf{\hat{x}}}
\def\yh{\mathbf{\hat{y}}}
\def\zh{\mathbf{\hat{z}}}
\def\Pv{{\mathcal{P}_v}}
\def\Pe{{\mathcal{P}_\eta}}
\def\vphi{\bvec{\phi}}
\def\vpsi{\bvec{\psi}}
\def\vv{\bvec{v}}      % \def\vv{{\mbox {\boldmath $v$}}}
\def\vx{\bvec{x}}      % \def\vx{{\mbox {\boldmath $x$}}}
\def\vk{\bvec{k}}      % \def\vk{{\mbox {\boldmath $k$}}}
\def\uh{\hat{u}} \def\vh{\hat{v}} \def\wh{\hat{w}} \def\eh{\hat{\eta}}
\def\ub{\overline{u}} \def\wb{\overline{w}}
\def\Real{{\mathbb R}}

\def\eg{{\it e.g.\ }} \def\ie{{\it i.e.}}

%
%%%%%%%%% JH
\def\ba{\mathbf{a}}

%%%%%%%%%%%% ChaosBook.org macros


\newcommand{\PP}{{\mathbf P}}                   % projection operator
\newcommand{\RR}{{\mathbf R}}    % real part, projection operator
\newcommand{\QQ}{{\mathbf Q}}    % imaginary part, projection operator
\newcommand{\derf}[2]{\ensuremath{{J}^{#1}(#2)}}    % Predrag fundamental matrix
% \newcommand{\derf}[2]{\ensuremath{{\bf J}^{#1}(#2)}}  % Predrag bold fundamental matrix
% \newcommand{\derf}[2]{{Df^{#1}|_{#2}}}   % Gibson fundamental matrix

                    % keep homepage flexible:
       \newcommand{\wwwcb}[1]{
                  {\tt \href{http://ChaosBook.org#1}
              {ChaosBook.org#1}}}
       \newcommand{\wwwQFT}[1]{
                  {\tt \href{http://ChaosBook.org/FieldTheory#1}
              {ChaosBook.org/\-Field\-Theory#1}}}
       \newcommand{\wwwnsQFT}[1]{
                  {\tt \href{http://ChaosBook.org/FieldTheory#1}
              {ChaosBook.org/\-Field\-Theory#1}}}
       \newcommand{\weblink}[1]{{\tt \href{http://#1}{#1}}}
       \newcommand{\HREF}[2]{
              {\href{#1}{#2}}}
       \newcommand{\mpArc}[1]{
              {\tt \href{http://www.ma.utexas.edu/mp_arc-bin/mpa?yn=#1}
                   {mp\_arc~#1}}}
       \newcommand{\arXiv}[1]{
              {\tt \href{http://arXiv.org/abs/#1}{arXiv:#1}}}

\newenvironment{offset}
               {\list{}{\listparindent 3em%
                        \advance\rightmargin -3em}%
                \item\relax}
               {\endlist}
\newtheorem{exmple}{\noindent\small\textsf{\textbf{Example}}}[chapter]
\newcommand{\example}[2]{
    \vskip -13mm
        \begin{offset}
        \begin{exmple}
           \noindent\small
           \textsf{\textbf{#1}} ~
       \slshape\sffamily{#2}
       % \textsl would not work...
    \end{exmple}
    \end{offset}
        \vskip -1mm
             }

\newcommand{\Remarks}{
        \noindent{\textsf{\large\textbf{Commentary}}}

        %\section*{\textsf{\textbf{Commentary}}}
        %\section*{Historical remarks}
        \addcontentsline{toc}{subsection}{{~~~~Historical remarks}}
                        }
\newtheorem{rmark}{{\small\textsf{\textbf{Remark}}}}[chapter]
\newcommand{\remark}[2]{
        % \begin{quotation}
        \begin{rmark}
        {\small\em\noindent {\small\sf \underline{ #1} ~} #2 }
    \end{rmark}
    % \end{quotation}
              }


%   \FIG{#1}    % \includegraphics[width=0.40\textwidth]{../figs/f_name.ps}
%   {#2}    % short caption text
%   {#3}    % full caption text
%   {#4}    % f-figure-label
%       defined here:
\newcommand{\FIG}[4]{\begin{figure}
              \hspace*{0.10\textwidth}%
              \begin{minipage}[b]{1.00\textwidth}
              \noindent{#1}
              %\centering{#1}
                      \caption[#2]{#3}
                      \label{#4}
              \end{minipage}
              \end{figure} }

%  \SFIG{#1}    % f_name.eps
%       {#2}    % short caption text
%       {#3}    % full caption text
%       {#4}    % f-figure-label
\newcommand{\SFIG}[4]{\begin{figure}
              %\hspace*{-0.10\textwidth}
              \hspace*{0.10\textwidth}
              \begin{minipage}[b]{0.55\textwidth}
                      \caption[#2]{#3}
                      \label{#4}
              \end{minipage}~~~~~%
              \begin{minipage}[b]{0.40\textwidth}
                      \includegraphics[width=1.00\textwidth]{../figs/#1}
              \end{minipage}
              %\hfill
              \end{figure} }

%%%%%%%%%%%%%%% VECTORS, MATRICES %%%%%%%%%%%%%%%%%%%%%%%%%%%%%%%%%%%%%%%%%
% Commented out AMS-style pmatrix, which is incompatible with TeX/LaTeX pmatrix
% used throughout dasbuch. Fri Oct 12 15:51:03 EDT 2007

\newcommand{\MatrixII}[4]{\left(
\begin{array}{cc}
{#1}  &  {#2} \\
{#3}  &  {#4} \end{array} \right)}
% a problem with \pmatrix 12oct 2007

\newcommand{\MatrixIII}[9]{
  \pmatrix{ {#1}  &  {#2} &  {#3} \cr
            {#4}  &  {#5} &  {#6} \cr
            {#7}  &  {#8} &  {#9}
          }               }

\newcommand{\VectorII}[2]{\left(
\begin{array}{cc}
{#1}  &  {#2} \end{array} \right)}

\newcommand{\transpVectorII}[2]{
  \pmatrix{ {#1}  &  {#2}}
}

\newcommand{\VectorIII}[3]{
  \pmatrix{ {#1} \cr
            {#2} \cr
            {#3}
          }
}


%%%%%%%%%%% COMMENTS
   \ifdraft    % display comments in text
\newcommand{\Preliminary}[1]
             {\color{magenta}
              \marginpar{$\Downarrow${\color{magenta}\footnotesize PRELIMINARY}}
               #1
              \marginpar{$\Uparrow${\color{magenta}\footnotesize PRELIMINARY}}
              \color{black}
             }
\newcommand{\PublicPrivate}[2]
    {\marginpar{\color{blue}$\Downarrow$\footnotesize PRIVATE}%
    {\color{blue}#2}%
    \marginpar{\color{blue}$\Uparrow$\footnotesize PRIVATE}}
\newcommand{\PC}[1]{$\footnotemark\footnotetext{PC: #1}$}
\newcommand{\PCedit}[1]{{\color{blue}#1}}
\newcommand{\AF}[2]{$\footnotemark\footnotetext{Adam #1: #2}$}
\newcommand{\AFedit}[1]{{\color{green}#1}}
\newcommand{\FW}[1]{$\footnotemark\footnotetext{FW: #1}$}
\newcommand{\FWedit}[1]{{\color{green}#1}}
\newcommand{\JRE}[1]{$\footnotemark\footnotetext{JRE: #1}$}
\newcommand{\JREedit}[1]{{\color{red}#1}}
\newcommand{\JFG}[1]{$\footnotemark\footnotetext{JG: #1}$}
\newcommand{\JFGedit}[1]{{\color{red}#1}}
\newcommand{\JH}[1]{$\footnotemark\footnotetext{JH: #1}$}
\newcommand{\JHedit}[1]{{\color{green}#1}}
\newcommand{\RG}[1]{$\footnotemark\footnotetext{FW: #1}$}
\newcommand{\file}[1]{$\footnotemark\footnotetext{{\bf file} #1}$}
\def\mycomment #1#2 {\noindent \textbf{\underline{#1}}: \emph{#2}}
    \else   % drop comments
\newcommand{\Preliminary}[1]{}
\newcommand{\PublicPrivate}[2]{#1}
\newcommand{\PC}[1]{}
\newcommand{\JFG}[1]{}
\newcommand{\JRE}[1]{}
\newcommand{\AF}[2]{}
\newcommand{\AFedit}[1]{#1}
\newcommand{\FW}[1]{}
\newcommand{\FWedit}[1]{#1}
\newcommand{\JH}[1]{}
\newcommand{\PCedit}[1]{#1}
\newcommand{\JFGedit}[1]{#1}
\newcommand{\JREedit}[1]{#1}
\newcommand{\JHedit}[1]{#1}
% \newcommand{\PCedit}[1]{{\color{blue}#1}}
% \newcommand{\FWedit}[1]{{\color{green}#1}}
% \newcommand{\JFGedit}[1]{{\color{red}#1}}
\newcommand{\RG}{}
\newcommand{\file}[1]{}
\def\mycomment #1#2 {}
    \fi
%%%%%%%%%%%%%%%%%%%%%% COMMENTS END %%%%%%%%%%%%%%%
