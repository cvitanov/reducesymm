% def.tex
% $Author$ $Date$

%               Predrag         29oct2005
%               Predrag         13jul2005
%               Predrag         24apr2005
%               Predrag         14feb2005
%               Predrag         22jan2005
%               Predrag         16nov2004
%               Predrag         13jun2004
%               Predrag          3may2004
%               Predrag         10apr2004
%               Predrag         21feb2004
%               Predrag          4oct2003
%               Predrag         30aug2003
%               Predrag         20jun2003
%               Predrag         17jan2003
%               Predrag          6dec2002
%               Predrag          7jul2002
%               Predrag         19nov2000
%               Ronnie          23sep2000
% Predrag disabled \basedirectory machine identifier    25aug2000
% created               Predrag         30/10-94
%
%%%%%%%%%%%%%%%%%%%%%%%%%%%%%%%%%%%%%%%%%%%%%%%%%%%%%%%%%%%%%%%%%%%%%%%%%
%% define abreviations used throughout DasBuch
%% goal: make it easier to change conventions throughout
%%%%%%%%%%%%%%%%%%%%%%%%%%%%%%%%%%%%%%%%%%%%%%%%%%%%%%%%%%%%%%%%%%%%%%%%%


%%%%%%%%%%%%%%%%%%%%%% WHERE ARE WE? %%%%%%%%%%%%%%%%%%%%%%%%%%%%%%%%%%%%
% OPTIMAL:
% input a single line file that tells you where you are texing at
%
%       \providecommand{\basedirectory}{ronnie\@felix/asc/book/}
%
% or something like it.  Put this file outside the book subdirectory, in
% your own TEXINPUTS path, so when you ftp things back and forth
% the basedirectory.tex file will remain unchanged on each machine.
% then uncomment
%\input{basedirectory}
%
% MACHINE BLIND VERSION:
%
\providecommand{\basedirectory}{}
% \providecommand{\basedirectory}{{$\sim$}DasBuch/book/}
%

%%%%%%%%%%%%%%%%%%%%%% QUOTATIONS %%%%%%%%%%%%%%%%%%%%%%%%%%%%%%%%%%%%%%
%
%  the learned/witty quotes at the chapter and section headings
% 
\newsavebox{\bartName}
\newcommand{\bauthor}[1]{\sbox{\bartName}{\parbox{\textwidth}{\vspace*{0.8ex}
       %\hspace*{\fill}
       \small\noindent #1}}}
\newenvironment{bartlett}{\hfill\begin{minipage}[t]{0.65\textwidth}\small}%
{\hspace*{\fill}\nolinebreak[1]\usebox{\bartName}\vspace*{1ex}\end{minipage}}

%
%  a quotation inserted into the text
%
\newenvironment{txtquote}{\begin{quotation} \small}{\end{quotation}}


%%%%%%%%%%%%%%%%%%%%%% INDEXING %%%%%%%%%%%%%%%%%%%%%%%%%%%%%%%%%%%%%%%%%
\newcommand{\indx}[1] {#1\index{#1}}    % do not need to repeat the word


%%%%%%%%%%%%%%% PAGE STYLE (to write footers) %%%%%%%%%%%%%%%%%%%%%%%%%%%
\makeatletter
\newcommand{\inputfile}{ }  % the rest is in the prlabels.sty
\newcommand{\version}{}


%%%%%%%%%%%%%%% PREPARE PAGINATION FOR bin/mkbook %%%%%%%%%%%%%%%%%%%%%%%
\newcounter{startSect}  %% page numbering used to generate ps pages
\newcounter{endSect}    %% page numbering used to generate ps pages
\newcommand{\chapName}{Unnamed} %% chapter label
\newcommand{\Chapter}[3]{\clearpage\thispagestyle{empty}\cleardoublepage
                \renewcommand{\inputfile}{#1 -  #2}
                         \renewcommand{\chapName}{#1}
                         \chapter[#3]{\textsf{\textbf{#3}}}
                         \label{c-#1}
                         \setcounter{startSect}{\thepage}
                        }
\newcommand{\References}[2]{\renewcommand{\inputfile}{#1 -  #2}
            %\renewcommand{\chapName}{#1}   % do not rename the chapter
            %\setcounter{startSect}{\thepage}   % do not restart counting
        \markright{\MakeUppercase{References}}
            \addtocontents{toc}{{\small\em references \thepage\ -}}
                        }
\newcommand{\Problems}[2]{\clearpage
                \renewcommand{\inputfile}{#1 -  #2}
                         \renewcommand{\chapName}{#1}
                         \setcounter{startSect}{\thepage}
             \markright{\MakeUppercase{Exercises}}
             \section*{\textsf{\textbf{Exercises}}}
             %\addcontentsline{toc}{section}{{~~~~Exercises}}
                         \addtocontents{toc}{
                            {\small\em exercises \thepage \par}
                        }
                        }
\newcommand{\Solution}[3]{
        \section*{Chapter~\ref{c-#1}} 
                \renewcommand{\inputfile}{#2 -  #3}
        % PC:  modified 28 may 2000
                %         \renewcommand{\chapName}{#1}
                %        \setcounter{startSect}{\thepage}
                        }
\newcommand{\Solutions}{
        \section{\textsf{\textbf{Solutions}}}
        %\addcontentsline{toc}{section}{{~~~~Solutions}}
                        }
\newcommand{\Resume}{
        \section*{\textsf{\textbf{R\'esum\'e}}}
    %   \addcontentsline{toc}{section}{{~~~~Resum\'e}}
                \addtocontents{toc}{{\small\em  resum\'e \thepage\ - }}
                    }
\newcommand{\Remarks}{
        \section*{\textsf{\textbf{Commentary}}}
        %\section*{Historical remarks}
        \addcontentsline{toc}{subsection}{{~~~~Historical remarks}}
                        }
\newcommand{\forMkbook}{\setcounter{endSect}{\thepage}
                        \addtocounter{endSect}{-1}
\typeout{ set Chp=\chapName\space set pp=\arabic{startSect}-\arabic{endSect}}
\typeout{ dvips -q -pp $pp book.dvi -o $sdir/$Chp.ps ; echo finished $Chp}
\typeout{ $PS2PAGES $PS2PFLAGS $sdir/$Chp.ps > $sdir/$Chp-2p.ps ; echo finished $Chp-2p}
                         }

%%%%%%%%%%%%%%%%%%%%%% SPLIT ChaosBook.ps INTO VOLUMES %%%%%%%%%%%%%%%%%%%
\newcounter{startVol}   %% page number of the start of a volume 
\newcounter{endVol}     %% page number of the end of a volume
% \setcounter{endVol}{1} %% mess with this to renumber the stqrt v vol II



\ifthenelse{\boolean{prepare_pdf}}
       {
           }{
       }  %end of ifthenelse prepare_pdf}


\ifthenelse{\boolean{prepare_pdf}}
       {
                    % keep homepage flexible:
       \newcommand{\wwwcb}[1]{
                  {\tt \href{http://ChaosBook.org#1}
              {ChaosBook.org#1}}}
       \newcommand{\wwwQFT}[1]{
                  {\tt \href{http://ChaosBook.org/FieldTheory#1}
              {ChaosBook.org/\-Field\-Theory#1}}}
       \newcommand{\wwwnsQFT}[1]{
                  {\tt \href{http://ChaosBook.org/FieldTheory#1}
              {ChaosBook.org/\-Field\-Theory#1}}}
       \newcommand{\weblink}[1]{{\tt \href{http://#1}{#1}}}
       \newcommand{\HREF}[2]{
              {\href{#1}{#2}}}
       \newcommand{\mpArc}[1]{
              {\tt \href{http://www.ma.utexas.edu/mp_arc-bin/mpa?yn=#1}
                   {\goodbreak mp\_arc~#1}}}
       \newcommand{\arXiv}[1]{
              {\tt \href{http://arXiv.org/abs/#1}{\goodbreak #1}}}
       \newcommand{\Volume}[2]{
%            \setcounter{page}{\endVol} %experimenting
                         \addtocontents{toc}{
                                \vspace{18pt}
                                \hrule
                                \vspace{4pt}
                                \noindent
                                {\Large\textsf{\textbf{Part~#1: #2}}}
                                \vspace{4pt}
                                \hrule
                                \vspace{12pt}
                        }
                               }
%   \setcounter{page}{\value{endVol}}   % experimenting with hopping page nos
%   \addtocounter{page}{-22}    % experimenting with hooping page nos
       \newcommand{\VolumeEnd}{\newpage }
           }{
                   %% prepare for postscript printing:
       \newcommand{\wwwcb}[1]{{\tt ChaosBook.org#1}}
       \newcommand{\wwwQFT}[1]{{\tt ChaosBook.org/\-Field\-Theory#1}}
       \newcommand{\wwwcnsQFT}[1]{{\tt ChaosBook.org/\-Field\-Theory#1}}
       \newcommand{\weblink}[1]{{\tt #1}}
       \newcommand{\HREF}[2]{{#2}}
           \newcommand{\arXiv}[1]{ {\tt #1}}
%RM    \newcounter{startVol}   %% page number of the start of a volume 
%RM    \newcounter{endVol}     %% page number of the end of a volume
       \newcommand{\volName}{Unnamed} %% volume label
       \newcommand{\Volume}[2]{
                         \renewcommand{\volName}{ChaosBook-#1}
%            \setcounter{page}{\endVol} % experimenting 
                         \setcounter{startVol}{\thepage}
                         \addtocontents{toc}{
                                \vspace{18pt}
                                \hrule
                                \vspace{4pt}
                                \noindent
                                {\Large\textsf{\textbf{Part~#1: #2}}}
                                \vspace{4pt}
                                \hrule
                                \vspace{12pt}
                        }
                                  }
\newcommand{\VolumeEnd}{
\setcounter{endVol}{\thepage}
\addtocounter{endVol}{-1}
\newpage
\typeout{ set Vlm=\volName\space set pp=\arabic{startVol}-\arabic{endVol}}
\typeout{ dvips -q -pp $pp book.dvi -o $sdir/$Vlm.ps ; echo finished $Vlm}
\typeout{ $PS2PAGES $PS2PFLAGS $sdir/$Vlm.ps > $sdir/$Vlm-2p.ps ; echo finished $Vlm-2p}
                         }
       }  %end of ifthenelse prepare_pdf}


%%%%%%%%%%%%%%%%%%%%%% COMMENTS IN THE TEXT %%%%%%%%%%%%%%%%%%%%%%%%%%%%%
% also search the text for lines starting with %%  to
% locate various internal comments, recent edits etc.
\ifthenelse{\boolean{display_comments}}
{ % if display_comments {true}, display comments in text
  \printlabels            %% turns on labeling of equations on margins
  \usepackage{showidx}    %% display index entries on margins

    \newcommand{\Preliminary}[1]
{\marginpar{\color{magenta}$\Downarrow$\footnotesize PRELIMINARY}%
{\color{magenta}#1}%
\marginpar{\color{magenta}$\Uparrow$\footnotesize PRELIMINARY}}

    \newcommand{\PublicPrivate}[2]
{\marginpar{\color{blue}$\Downarrow$\footnotesize PRIVATE}%
{\color{blue}#2}%
\marginpar{\color{blue}$\Uparrow$\footnotesize PRIVATE}}

    \newcommand{\PC}[1]
{\marginpar{\renewcommand{\baselinestretch}{0.7}\footnotesize #1}%
\renewcommand{\baselinestretch}{1.0}}
    \newcommand{\RM}[1]{\marginpar{
        \renewcommand{\baselinestretch}{0.7}
        \hrule \vspace{0.2ex}
        \footnotesize RM:~ #1
        \\ \hrule}%
        \renewcommand{\baselinestretch}{1.0}%
                    }
  \newcommand{\PCedit}[1]{{\color{red}#1}}
  \newcommand{\AW}[1]{\marginpar{
        \renewcommand{\baselinestretch}{0.7}
        \hrule \vspace{0.2ex}
        \footnotesize AW:~ #1
        \\ \hrule}%
        \renewcommand{\baselinestretch}{1.0}%
                }
  \newcommand{\LR}[1]{\marginpar{
        \renewcommand{\baselinestretch}{0.7}
        \hrule \vspace{0.2ex}
        \footnotesize LR:~ #1
        \\ \hrule}%
        \renewcommand{\baselinestretch}{1.0}%
                }
  %\newcommand{\LR}[1]{$\footnotemark\footnotetext{Lamberto: #1}$}
  \newcommand{\RA}[1]{$\footnotemark\footnotetext{Artuso: #1}$}
  \newcommand{\VB}[1]{$\footnotemark\footnotetext{Baladi: #1}$}
  \newcommand{\OB}[1]{$\footnotemark\footnotetext{Biham: #1}$}
  \newcommand{\PD}[1]{$\footnotemark\footnotetext{Dahlqvist: #1}$}
  \newcommand{\CPD}[1]{$\footnotemark\footnotetext{Carl: #1}$}
  \newcommand{\DJD}[1]{$\footnotemark\footnotetext{Driebe: #1}$}
  \newcommand{\AG}[1]{$\footnotemark\footnotetext{Garg: #1}$}
  \newcommand{\DG}[1]{$\footnotemark\footnotetext{Dorte: #1}$}
  \newcommand{\KTH}[1]{$\footnotemark\footnotetext{Kai H: #1}$}
  \newcommand{\BL}[1]{$\footnotemark\footnotetext{Lautrup: #1}$}
  \newcommand{\DL}[1]{$\footnotemark\footnotetext{Domenico: #1}$}
  \newcommand{\DLedit}[1]{{\color{green}#1}}
  \newcommand{\SFN}[1]{$\footnotemark\footnotetext{Sune: #1}$}
  \newcommand{\PER}[1]{$\footnotemark\footnotetext{Rosenqvist: #1}$}
  \newcommand{\HHR}[1]{$\footnotemark\footnotetext{HHR: #1}$}
  \newcommand{\EAS}[1]{$\footnotemark\footnotetext{Spiegel: #1}$}
  \newcommand{\NS}[1]{$\footnotemark\footnotetext{Niels S: #1}$}
  \newcommand{\GT}[1]{$\footnotemark\footnotetext{Tanner: #1}$}
  \newcommand{\GV}[1]{$\footnotemark\footnotetext{Vattay: #1}$}
  \newcommand{\DV}[1]{$\footnotemark\footnotetext{Divikar: #1}$}
  \newcommand{\NW}[1]{$\footnotemark\footnotetext{Whelan: #1}$}
  \newcommand{\DW}[1]{$\footnotemark\footnotetext{Wojcik: #1}$}
  \newcommand{\RS}[1]{$\footnotemark\footnotetext{Raenell: #1}$}
  \newcommand{\MAP}[1]{$\footnotemark\footnotetext{Mason: #1}$}
  \newcommand{\TB}[1]{$\footnotemark\footnotetext{Bartsch: #1}$}
  \newcommand{\TZ}[1]{$\footnotemark\footnotetext{Zazy: #1}$}
}{  % if display_comments {false}, drop comments 
    % do not turn on labeling of equations on margins
  \newcommand{\Preliminary}[1]{}
  \newcommand{\PublicPrivate}[2]{#1}
  \newcommand{\OB}[1]{}
  \newcommand{\PC}[1]{}
  \newcommand{\PCedit}[1]{}
  \newcommand{\PD}[1]{}
  \newcommand{\DG}[1]{}
  \newcommand{\BL}[1]{}
  \newcommand{\DL}[1]{}
  \newcommand{\DLedit}[1]{}
  \newcommand{\SFN}[1]{}
  \newcommand{\RM}[1]{}
  \newcommand{\GV}[1]{}
  \newcommand{\AG}[1]{}
  \newcommand{\VB}[1]{}
  \newcommand{\LR}[1]{}
  \newcommand{\PER}[1]{}
  \newcommand{\GT}[1]{}
  \newcommand{\AW}[1]{}
  \newcommand{\DW}[1]{}
  \newcommand{\RS}[1]{}
  \newcommand{\NS}[1]{}
  \newcommand{\RA}[1]{}
  \newcommand{\HHR}[1]{}
  \newcommand{\EAS}[1]{}
  \newcommand{\NW}[1]{}
  \newcommand{\KTH}[1]{}
  \newcommand{\CPD}[1]{}
  \newcommand{\DJD}[1]{}
  \newcommand{\DV}[1]{}
  \newcommand{\MAP}[1]{}
  \newcommand{\TB}[1]{}
  \newcommand{\TZ}[1]{}
}
%%%%%%%%%%%%%%%%%%%%%% END OF ON/OFF COMMENTS SWITCH %%%%%%%%%%%%%%%%%%%%

\newcommand{\authorRA}%{\marginpar{RM+PC}}
         {\hfill (R. Artuso)}
\newcommand{\authorRAPC} %{\marginpar{EAS+PC}}
     {\hfill (R. Artuso and P. Cvitanovi\'c)}
\newcommand{\authorOBPC} %{\marginpar{EAS+PC}}
     {\hfill (O. Biham and P. Cvitanovi\'c)}
\newcommand{\authorOBCCPC} %{\marginpar{EAS+PC}}
     {\hfill (O. Biham, C. Chandre and P. Cvitanovi\'c)}
\newcommand{\authorCCPSFKD} %{\marginpar{EAS+PC}}
     {\hfill (C. Chandre, F.K. Diakonos and P. Schmelcher)}
\newcommand{\authorFC} %{\marginpar{FC}}
     {\hfill (F. Christiansen)}
\newcommand{\authorHHR} %{\marginpar{EAS+PC}}
     {\hfill (H.H. Rugh)}
\newcommand{\authorRAHHRPC} %{\marginpar{EAS+PC}}
     {\hfill (R. Artuso, H.H. Rugh and P. Cvitanovi\'c)}
\newcommand{\authorJMPC} %{\marginpar{EAS+PC}}
     {\hfill (J. Mathiesen and P. Cvitanovi\'c)}
\newcommand{\authorGSPC} %{\marginpar{EAS+PC}}
     {\hfill (G. Simon and P. Cvitanovi\'c)}
\newcommand{\authorRAPD} %{\marginpar{EAS+PD}}
     {\hfill (R. Artuso and P. Dahlqvist)}
\newcommand{\authorRAPDGTPC} %{\marginpar{EAS+PD}}
     {\hfill (R. Artuso, P. Dahlqvist, G. Tanner and P. Cvitanovi\'c)}
\newcommand{\authorPCRALREAS}
     {\hfill (P. Cvitanovi\'c, R. Artuso, L. Rondoni, and  E.A. Spiegel)}
\newcommand{\authorPC} %{\marginpar{PC}}
     {\hfill (P. Cvitanovi\'c)}
\newcommand{\authorPD} %{\marginpar{PD+PC}}
     {\hfill (P. Dahlqvist)}
\newcommand{\authorEASPC} %{\marginpar{EAS+PC}}
     {\hfill (E.A. Spiegel and P. Cvitanovi\'c)}
\newcommand{\authorLC} %{\marginpar{DL}
     {\hfill (Y. Lan and P. Cvitanovi\'c)}
\newcommand{\authorDL} %{\marginpar{DL}
     {\hfill (D. Lippolis)}
\newcommand{\authorRMPCEAS}
     {\hfill (R. Mainieri, P. Cvitanovi\'c and  E.A. Spiegel)}
\newcommand{\authorRM} %{\marginpar{RM+PC}}
     {\hfill (R. Mainieri)}
\newcommand{\authorRMPC} %{\marginpar{RM+PC}}
     {\hfill (R. Mainieri and P. Cvitanovi\'c)}
\newcommand{\authorRPPC} %{\marginpar{RM+PC}}
     {\hfill (R. Pa\v skauskas and P. Cvitanovi\'c)}
\newcommand{\authorLRPC} %{\marginpar{LR+PC}}
     {\hfill (L. Rondoni and P. Cvitanovi\'c)}
\newcommand{\authorLR} %{\marginpar{LR}}
     {\hfill (L. Rondoni)}
\newcommand{\authorCPDPC} %{\marginpar{CPD+PC}}
     {\hfill  (C.P. Dettmann and P. Cvitanovi\'c)}
\newcommand{\authorKTHPC} %{\marginpar{KTH+PC}}
     {\hfill  (K.T. Hansen and P. Cvitanovi\'c)}
\newcommand{\authorGT} %{\marginpar{GT}}
     {\hfill  (G. Tanner)}
\newcommand{\authorMJFPC} %{\marginpar{MJF+PC}}
     {\hfill  (M.J. Feigenbaum and P. Cvitanovi\'c)}
\newcommand{\authorGV} %{\marginpar{GV+PC}}
     {\hfill (G. Vattay)}
\newcommand{\authorGVGTPC} %{\marginpar{GV+PC}}
     {\hfill (G. Vattay, G. Tanner and P. Cvitanovi\'c)}
\newcommand{\authorGVPC} %{\marginpar{GV+PC}}
     {\hfill (G. Vattay and P. Cvitanovi\'c)}
\newcommand{\authorNW} %{\marginpar{FC}}
     {\hfill (N. Whelan)}
\newcommand{\authorAW} %{\marginpar{FC}}
     {\hfill (A. Wirzba)}
\newcommand{\authorCV} %{\marginpar{FC}}
     {\hfill (P. Cvitanovi\'c and L.V. Vela-Arevalo)}
\newcommand{\authorCW} %{\marginpar{FC}}
     {\hfill (A. Wirzba and P. Cvitanovi\'c)}
\newcommand{\authorWCW} %{\marginpar{FC}}
     {\hfill (A. Wirzba, P. Cvitanovi\'c and N. Whelan)}


\newcommand{\file}[1]{$\footnotemark\footnotetext{{\bf file} #1}$}
%\newcommand{\lecture}{\addtocontents{toc}{\sf\small lecture: }}
\newcommand{\lecture}[2]{ \addtocontents{toc}
           {{\scriptsize #1}{\sf\small lecture: \scriptsize #2}} }


%%%%%%%%%%%%%%% REFERENCING EQUATIONS ETC. %%%%%%%%%%%%%%%%%%%%%%%%%%%%%%%
\newcommand{\rf}     [1] {~\cite{#1}}
\newcommand{\refref} [1] {ref.~\cite{#1}}
\newcommand{\refRef} [1] {Ref.~\cite{#1}}
\newcommand{\refrefs}[1] {refs.~\cite{#1}}
\newcommand{\refRefs}[1] {Refs.~\cite{#1}}
\newcommand{\refeq}  [1] {(\ref{#1})}
\newcommand{\refeqs} [2]{(\ref{#1}--\ref{#2})}
\newcommand{\refpage}[1] {page~\pageref{#1}}
\newcommand{\reffig} [1] {figure~\ref{#1}}
\newcommand{\reffigs} [2] {figures~\ref{#1} and~\ref{#2}}
\newcommand{\refFig} [1] {Figure~\ref{#1}}
\newcommand{\refFigs} [2] {Figures~\ref{#1} and~\ref{#2}}
\newcommand{\reftab} [1] {table~\ref{#1}}
\newcommand{\refTab} [1] {Table~\ref{#1}}
\newcommand{\reftabs}[2] {tables~\ref{#1} and~\ref{#2}}
\newcommand{\refsect}[1] {sect.~\ref{#1}}
\newcommand{\refsects}[2] {sects.~\ref{#1} and \ref{#2}}
\newcommand{\refSect}[1] {Sect.~\ref{#1}}
\newcommand{\refchap}[1] {chapter~\ref{#1}}
\newcommand{\refChap}[1] {Chapter~\ref{#1}}
\newcommand{\refchaps}[2] {chapters~\ref{#1} and \ref{#2}}
\newcommand{\refappe}[1] {appendix~\ref{#1}}
\newcommand{\refappes}[2] {appendices~\ref{#1} and \ref{#2}}
\newcommand{\refAppe}[1] {Appendix~\ref{#1}}
\newcommand{\refrem} [1] {remark~\ref{#1}}
\newcommand{\refexam}[1] {example~\ref{#1}}
\newcommand{\refExam}[1] {Example~\ref{#1}}
\newcommand{\refexer}[1] {exercise~\ref{#1}}
\newcommand{\refExer}[1] {Exercise~\ref{#1}}
\newcommand{\refsolu}[1] {solution~\ref{#1}}
\newcommand{\exerbox}[1]{
         \marginpar[{%
            \color{green} 
        \raisebox{-0.5ex}[1.0ex][1.0ex]{\huge \ding{46}}
        \sf ~\ref{#1}\\ \footnotesize page~\pageref{#1} 
                   }]% 
          { \raggedright
            \color{green} 
        \raisebox{-0.5ex}[1.0ex][1.0ex]{\huge \ding{46}}
        \sf ~\ref{#1}\\ \footnotesize page~\pageref{#1} 
          } 
                        }
           %\includegraphics[width=0.04\textwidth]{Fig/padPencil.eps}
           %\includegraphics[width=0.04\textwidth]{Fig/exercise.ps}
           %\includegraphics[width=0.04\textwidth]{Fig/archive.ps}

\newcommand{\advSect}{\marginpar{\fbox{\sf $\bullet$ advanced section $\bullet$ } } }
\newcommand{\toSect}[1]{\marginpar[{
        \color{green} 
        \raisebox{-1.0ex}[1.0ex][1.0ex]{\huge \ding{43}}
        {\sf \footnotesize \refsect{#1} }
                  }]{ 
            \raggedright
        \color{green} 
        \raisebox{-1.0ex}[1.0ex][1.0ex]{\huge \ding{43}}
        {\sf \footnotesize \refsect{#1} }
                                  } }
\newcommand{\toChap}[1]{\marginpar[{
        \color{green} 
        \raisebox{-1.0ex}[1.0ex][1.0ex]{\huge \ding{43}}
        {\sf \footnotesize ~\refchap{#1} }
                  }]{ 
            \raggedright
        \color{green} 
        \raisebox{-1.0ex}[1.0ex][1.0ex]{\huge \ding{43}}
        {\sf \footnotesize ~\refchap{#1} }
                                  } }
        %\includegraphics[width=0.6cm]{Fig/thisWay.ps}

\newcommand{\toAppe}[1]{\marginpar[{
        \color{green} 
        \raisebox{-1.0ex}[1.0ex][1.0ex]{\huge \ding{43}}
        {\sf \footnotesize \refappe{#1} }
                  }]{ 
            \raggedright
        \color{green} 
        \raisebox{-1.0ex}[1.0ex][1.0ex]{\huge \ding{43}}
        {\sf \footnotesize \refappe{#1} } 
                                  } }
\newcommand{\toRem}[1]{\marginpar[{
        \color{green} 
        \raisebox{-1.0ex}[1.0ex][1.0ex]{\huge \ding{43}}
        {\sf \footnotesize \refrem{#1} }
                 }]{ 
            \raggedright
        \color{green} 
        \raisebox{-1.0ex}[1.0ex][1.0ex]{\huge \ding{43}}
        {\sf \footnotesize \refrem{#1} }
                                  } }
\newcommand{\toExam}[1]{\marginpar[{
        \color{green} 
        \raisebox{-1.0ex}[1.0ex][1.0ex]{\huge \ding{43}}
        {\sf \footnotesize \refexam{#1} }
                 }]{ 
            \raggedright
        \color{green} 
        \raisebox{-1.0ex}[1.0ex][1.0ex]{\huge \ding{43}}
        {\sf \footnotesize \refexam{#1} }
                                  } }
\newcommand{\tangent}{\includegraphics[width=0.08\textwidth]{Fig/chili.eps}~~}
%\newcommand{\tangent}{\marginpar{\fbox{ \includegraphics{Fig/chili.ps} } } }
\newcommand{\fastTrack}[1]
           {\hfill {\raisebox{-3.0ex}[4.5ex][3.5ex]{
                              \includegraphics[width=1.25cm]{Fig/thisWay.ps}
                                                   }
            \begin{minipage}[t]{0.25\textwidth}
            {\sf \footnotesize
                   fast track: \\
                               \refsect{#1}, p.~\pageref{#1}
                        } 
            \end{minipage}
                    }
           }
\newcommand{\fastTrackChap}[1]
           {\hfill {\raisebox{-3.0ex}[4.5ex][3.5ex]{
                              \includegraphics[width=1.25cm]{Fig/thisWay.ps}
                                                   }
            \begin{minipage}[t]{0.25\textwidth}
            {\sf \footnotesize
                   fast track: \\
                               \refchap{#1}, p.~\pageref{#1}
                        } 
            \end{minipage}
                    }
           }
\newcommand{\inDepth}[1]
           {\hfill {\raisebox{-3.0ex}[4.5ex][3.5ex]{
                         \includegraphics[width=1.24cm]{Fig/eyep.ps}
                                                   }
            \begin{minipage}[t]{0.25\textwidth}
            {\sf \footnotesize
                   in depth: \\
                               \refsect{#1}, p.~\pageref{#1}
                        } 
            \end{minipage}
                    }
           }
\newcommand{\inDepthChap}[1]
           {\hfill {\raisebox{-3.0ex}[4.5ex][3.5ex]{
                         \includegraphics[width=1.24cm]{Fig/eyep.ps}
                                                   }
            \begin{minipage}[t]{0.25\textwidth}
            {\sf \footnotesize
                   in depth: \\
                               \refchap{#1}, p.~\pageref{#1}
                        } 
            \end{minipage}
                    }
           }
\newcommand{\inDepthAppe}[1]
           {\hfill {\raisebox{-3.0ex}[4.5ex][3.5ex]{
                         \includegraphics[width=1.24cm]{Fig/eyep.ps}
                                                   }
            \begin{minipage}[t]{0.25\textwidth}
            {\sf \footnotesize
                   in depth: \\
                               \refappe{#1}, p.~\pageref{#1}
                        } 
            \end{minipage}
                    }
           }

%%%%%%%%%%%%%%% EQUATIONS %%%%%%%%%%%%%%%%%%%%%%%%%%%%%%%
\newcommand{\beq}{\begin{equation}}
\newcommand{\continue}{\nonumber \\ }
\newcommand{\nnu}{\nonumber}
\newcommand{\eeq}{\end{equation}}
\newcommand{\ee}[1] {\label{#1} \end{equation}}
\newcommand{\bea}{\begin{eqnarray}}
\newcommand{\ceq}{\nonumber \\ & & }
\newcommand{\eea}{\end{eqnarray}}
\newcommand{\barr}{\begin{array}}
\newcommand{\earr}{\end{array}}

%%%%%%%%%%%%%%% VECTORS, MATRICES %%%%%%%%%%%%%%%%%%%%%%%%%%%%%%%
\newcommand{\MatrixII}[4]{
   \pmatrix{ {#1}  &  {#2} \cr
             {#3}  &  {#4} \cr} }

\newcommand{\MatrixIII}[9]{
   \pmatrix{ {#1}  &  {#2} &  {#3} \cr
             {#4}  &  {#5} &  {#6} \cr
             {#7}  &  {#8} &  {#9} \cr} }

\newcommand{\transpVectorII}[2]{
   \pmatrix{ {#1}  &  {#2}  \cr} }

\newcommand{\VectorII}[2]{
   \pmatrix{ {#1} \cr
             {#2} \cr} }

\newcommand{\VectorIII}[3]{
   \pmatrix{ {#1} \cr
             {#2} \cr
             {#3} \cr} }

\newcommand{\combinatorial}[2]{ {#1 \choose #2}}

%%%%%%%%%%%%%%% NUMBERED ENVIRONMENTS %%%%%%%%%%%%%%%%%%%%%%%%%%%%%%%

% REPLACE {13.5cm} by \textwidth counter in these environments

\newtheorem{exerc}{\textsf{\textbf{Exercise}}}[chapter]
% \newtheorem{exerc}{}[chapter]
% \newtheorem{exerc}{{$\bullet$}}[chapter]
 \newcommand{\exercise}[2]{
        \vskip -13mm
         \noindent
         \begin{exerc}{
\renewcommand{\theenumi}{\alph{enumi}}
\renewcommand{\labelenumi}{\textbf{(\alph{enumi})\ }}
    {\noindent\small
         ~~\textsf{\textbf{#1}} ~
           \slshape\sffamily{#2}  } % \textsl would not work...
    }
         \vskip -1mm 
% removed the line: % \noindent\rule[.1mm]{\linewidth}{.5mm}
         \end{exerc}
                          }

\newcommand{\Exercise}[2]{      %environment for obligatory problems
        \vskip -13mm
        \noindent
        \begin{exerc}{
\renewcommand{\theenumi}{\alph{enumi}}
\renewcommand{\labelenumi}
    {\textsf{\textbf{ (\alph{enumi})\ }}}
        {\noindent
         ~~\textsf{\textbf{\underline{#1}}} ~
           \slshape\sffamily{#2}  } % \textsl would not work...
        }
         \vskip -1mm
        \end{exerc}
                          }

% from LaTeX style file for the CUP standard designs 5a to 9b
% \newtheorem{exerc}{Exercise}[chapter]
% \newcommand{\exercise}[2]{
% %        \vskip -12mm
%          \begin{quotation}
%         \noindent\rule[.1mm]{10.7cm}{.2mm} 
%    % \vskip -22mm 
%          \noindent
%          \begin{exerc}{
%           \noindent ~~{\small\em\bf #1 ~}\small\em #2 }
%           \vskip -1mm \noindent\rule[.1mm]{10.7cm}{.5mm}
%   \end{exerc}
%          \end{quotation}
%                 }
 
\newcommand{\solution}[3]{
         \vskip -4mm
        {\noindent\small
         ~~\textsf{\textbf{  Solution \ref{#1}:   %NUMBER
                ~#2}}         %TITLE 
           \slshape\sffamily{#3}          %TEXT
         }
         \vskip  4mm 
% removed the line: % \noindent\rule[.1mm]{\linewidth}{.5mm}
                        }

\newtheorem{rmark}{{\small\textsf{\textbf{Remark}}}}[chapter]
\newcommand{\remark}[2]{
        % \begin{quotation}
        \begin{rmark}
        {\small\em\noindent {\small\sf \underline{ #1} ~} #2 }
    \end{rmark}
    % \end{quotation}
              }

\newtheorem{exmple}{\noindent\small\textsf{\textbf{Example}}}[chapter]
\newcommand{\example}[2]{
    \vskip -13mm
        \begin{offset}
        \begin{exmple}
           \noindent\small
           \textsf{\textbf{#1}} ~ 
       \slshape\sffamily{#2}  
       % \textsl would not work...
    \end{exmple}
    \end{offset}
        \vskip -1mm
             }

%   \FIG{#1}    % \includegraphics[width=0.40\textwidth]{Fig/f_name.ps}
%   {#2}    % short caption text
%   {#3}    % full caption text
%   {#4}    % f-figure-label
%       defined here:
\newcommand{\FIG}[4]{\begin{figure}
              \hspace*{0.10\textwidth}%
              \begin{minipage}[b]{1.00\textwidth}
              \noindent{#1}
              %\centering{#1}
                      \caption[#2]{#3}
                      \label{#4}
              \end{minipage}
              \end{figure} }

%% Eventually get rid of {floatingfigure}
%  \FFIG{#1}    % width=?cm, ... here
%       {#2}    % Fig/f_name.ps
%       {#3}    % short caption text
%       {#4}    % full caption text
%       {#5}    % f-figure-label
\newcommand{\FFIG}[5]{\begin{floatingfigure}[v]{0.40\textwidth}
                      \noindent
                      \includegraphics[#1]{#2}
                      \caption[#3]{#4}
                      \label{#5}
              \end{floatingfigure} }

%   \BFIG{#1}   % width=#1\textwidth
%        {#2}   % f_name.ps
%    {#3}   % short caption text
%    {#4}   % full caption text
%    {#5}   % f-figure-label
%       defined here:
\newcommand{\BFIG}[5]{\begin{figure}
              \hspace*{0.10\textwidth}%
              \begin{minipage}[b]{1.00\textwidth}
              \centering{
                      \includegraphics[width=#1\textwidth]{Fig/#2}
                     }
                      \caption[#3]{#4}
                      \label{#5}
              \end{minipage}
              \end{figure} }

%  \SFIG{#1}    % f_name.eps
%       {#2}    % short caption text
%       {#3}    % full caption text
%       {#4}    % f-figure-label
\newcommand{\SFIG}[4]{\begin{figure}
              %\hspace*{-0.10\textwidth}
              \hspace*{0.10\textwidth}
              \begin{minipage}[b]{0.55\textwidth}
                      \caption[#2]{#3}
                      \label{#4} 
              \end{minipage}~~~~~%
              \begin{minipage}[b]{0.40\textwidth}
                      \includegraphics[width=1.00\textwidth]{Fig/#1}
              \end{minipage}
              %\hfill
              \end{figure} }

%%%%%%%%%%%%%%  Abbreviations %%%%%%%%%%%%%%%%%%%%%%%%%%%%%%%%%%%%%%%%
%%% APS (American Physiology Society, it seems) style:
%%%     Latin or foreign words or phrases should be roman, not italic.

\newcommand{\etc}{{etc.}}       % APS
\newcommand{\etal}{{\em et al.}}    % etal in italics, APS too
\newcommand{\ie}{{i.e.}}        % APS
\newcommand{\cf}{{\em cf.}}     % APS
\newcommand{\eg}{{e.g.}}        % APS
% \newcommand{\etc}{{\em etc.}}     % etcetera in italics
% \newcommand{\ie}{{that is}}       % use Latin or English?  Decide later.
% \newcommand{\cf}{{cf.}}
% \newcommand{\eg}{{\it e.g.,\ }}   % Wirzba 2sep2001

%%%%%%%%%%%%%%% TEXT ABBREVS %%%%%%%%%%%%%%%%%%%%%%%%%%%%%

         %% [Evolution operators, zetas, ...] %%%%%%%%
\newcommand{\evOper}{evolution oper\-ator}
\newcommand{\EvOper}{Evolution oper\-ator}
 %% \newcommand{\evOp}{Ruelle operator} %could be ``evolution'' instead?
%\newcommand{\FPoper}{Frobenius-Perron oper\-ator}
\newcommand{\FPoper}{Perron-Frobenius oper\-ator} % Pesin's ordering
\newcommand{\FP}{Perron-Frobenius}

\newcommand{\statesp}{state space}
\newcommand{\Statesp}{State space}
\newcommand{\maslov}{topological}
\newcommand{\Maslov}{Topological}
%\newcommand{\Maslov}{Keller-Maslov}
\newcommand{\jacobian}{Jacobian}        % determinant
\newcommand{\jacobianM}{fundamental matrix} % standard name
\newcommand{\jacobianMs}{fundamental matrices}  % 
\newcommand{\JacobianM}{Fundamental matrix} % 
\newcommand{\JacobianMs}{Fundamental matrices}  % 
% \newcommand{\jacobianM}{Jacobian matrix}  % Predrag's name
% \newcommand{\jacobianMs}{Jacobian matrices}   % matrices
\newcommand{\stabmat}{stability matrix}     % stability matrix
\newcommand{\Stabmat}{Stability matrix}     % Stability matrix
% \newcommand{\stabmat}{matrix of variations}   % Arnold, says Vattay
\newcommand{\monodromyM}{monodromy matrix} % monodromy matrix, Poincare cut
\newcommand{\MonodromyM}{Monodromy matrix} % monodromy matrix, Poincare cut
\newcommand{\dzeta}{dyn\-am\-ic\-al zeta func\-tion}
\newcommand{\Dzeta}{Dyn\-am\-ic\-al zeta func\-tion}
\newcommand{\tzeta}{top\-o\-lo\-gi\-cal zeta func\-tion}
\newcommand{\Tzeta}{Top\-o\-lo\-gi\-cal zeta func\-tion}
\newcommand{\BERzeta}{BER zeta func\-tion}
%\newcommand{\tzeta}{Artin-Mazur zeta func\-tion} %alternative to topological
\newcommand{\qS}{semi\-classical zeta func\-tion}
%\newcommand{\qS}{Gutz\-willer-Voros zeta func\-tion}
\newcommand{\Gt}{Gutz\-willer trace formula}
\newcommand{\Fd}{spec\-tral det\-er\-min\-ant}
\newcommand{\fd}{spec\-tral det\-er\-min\-ant}
\newcommand{\FD}{Spec\-tral det\-er\-min\-ant}
\newcommand{\cFd}{semiclass\-ic\-al spec\-tral det\-er\-mi\-nant}
\newcommand{\cFD}{Semiclass\-ic\-al spec\-tral det\-er\-mi\-nant}
% \newcommand{\cFd}{semiclass\-ic\-al Fred\-holm det\-er\-mi\-nant}
\newcommand{\Vd}{Vattay det\-er\-mi\-nant}
\newcommand{\cycForm}{cycle averaging formula}
\newcommand{\CycForm}{Cycle averaging formula}
\newcommand{\freeFlight}{mean free flight time}
\newcommand{\FreeFlight}{Mean free flight time}
\newcommand{\pdes}{partial differential equations}
\newcommand{\Pdes}{Partial differential equations}
\newcommand{\dof}{dof}         % Hamiltonian deegree of freedom
% \newcommand{\dof}{deegree of freedom}
\newcommand{\equilibrium}{equilibrium}
\newcommand{\equilibria}{equilibria}
\newcommand{\Equilibria}{Equilibria}
% \newcommand{\equilibrium}{steady state}
% \newcommand{\equilibria}{steady states}
% \newcommand{\Equilibria}{Steady states}
\newcommand\Poincare{Poincar\'e }

%%%%%%%%%%%%%%% Sundry symbols within math eviron.: %%%%%%%%%%%%

\newcommand{\obser}{\ensuremath{a}}     % an observable from phase space to R^n
\newcommand{\Obser}{\ensuremath{A}}     % time integral of an observable
\newcommand{\onefun}{\iota} % the function that returns one no matter what
\newcommand{\defeq}{=}      % the different equal for a definition
\newcommand {\deff}{\stackrel{\rm def}{=}}
\newcommand{\reals}{\mathbb{R}}
\newcommand{\complex}{\mathbb{C}}
\newcommand{\integers}{\mathbb{Z}}
\newcommand{\rationals}{\mathbb{Q}}
\newcommand{\naturals}{\mathbb{N}}
\newcommand{\LieD}{{{\cal L}\!\!\llap{-}\,\,}}  % {{\pound}} % Lie Derivative 
\newcommand{\half}{{\scriptstyle{1\over2}}}
\newcommand{\pde}{\partial}
\newcommand{\pdfrac}[2]{{\partial #1 \over \partial #2}}
\renewcommand\Im{{\rm Im\,}}
\renewcommand\Re{{\rm Re\,}}
\renewcommand{\det}{\mbox{\rm det}\,}
\newcommand{\Det}{\mbox{\rm Det}\,}
\newcommand{\tr}{\mbox{\rm tr}\,}
\newcommand{\Tr}{\mbox{\rm tr}\,}
%\newcommand{\Tr}{\mbox{Tr}\,}
\newcommand{\sign}[1]{\sigma_{#1}}
%\newcommand{\sign}[1]{{\rm sign}(#1)}
\newcommand{\DiffC}{\ensuremath{D}}     % diffusion constant
\newcommand{\mInv}{{I}}                 % material invariant 
\newcommand{\msr}{\ensuremath{\rho}}                % measure 
\newcommand{\Msr}{{\mu}}                % coarse measure 
\newcommand{\dMsr}{{d\mu}}              % measure infinitesimal
\newcommand{\SRB}{{\rho_0}}             % natural measure 
\newcommand{\vol}{{V}}                  % volume of i-th tile
\newcommand{\prpgtr}[1]{\delta\negthinspace\left( {#1} \right)}
%\newcommand{\Zqm}{\ensuremath{Z_{qm}}}         % Gutz-Voros zeta function
\newcommand{\Zqm}{\ensuremath{\det(\hat{H} - E)_{sc} }} % semicls spectr. det:
\newcommand{\Fqm}{\ensuremath{F_{qm}}}
\newcommand{\zfct}[1]{\zeta ^{-1}_{#1}}     
\newcommand{\zetaInv}{\ensuremath{1/\zeta}}
% \newcommand{\zetaInv}{{\zeta^{-1}}}
\newcommand{\zetatop}{\ensuremath{1/\zeta_{\mbox{\footnotesize top}} }}
\newcommand{\zetaInvBER}[1]{1/\zeta_{\mbox{\footnotesize BER}}(#1)}
\newcommand{\BER}[1]{{\mbox{\footnotesize BER}}} % Baladi-Ruelle-Eckmann
\newcommand{\eigCond}{\ensuremath{F}}           % eigenvalue cond. function
\newcommand{\expct}    [1]{\left\langle {#1} \right\rangle}
\newcommand{\spaceAver}[1]{\left\langle {#1} \right\rangle}
\newcommand{\timeAver} [1]{\overline{#1}}
\newcommand{\pS}{\ensuremath{{\cal M}}}          % symbol for phase space 
\newcommand{\ssp}{\ensuremath{x}}                % state space point
\newcommand{\DOF}{\ensuremath{D}}          % Hamiltonian deegree of freedom
\newcommand{\NWS}{\ensuremath{\Omega}}     % symbol for the non--wandering set
\newcommand{\AdmItnr}{\Sigma}      % set of admissible itineraries
\newcommand{\intM}[1]{{\int_\pS{\!d #1}\:}} %phasespace integral
\newcommand{\Cint}[1]{\oint\frac{d#1}{2 \pi i}\;} %Cauchy contour integral
\newcommand{\PoincS}{{\cal P}}     % symbol for Poincare section
\newcommand{\PoincM}{\ensuremath{P}}       % symbol for Poincare map
\newcommand{\PoincC}{\ensuremath{U}}       % symbol for Poincare constraint function
\newcommand{\arc}{\ensuremath{s}}          % symbol for billiard wall arc
\newcommand{\mompar}{\ensuremath{p}}       % billiard wall parall. momentum
\newcommand{\restCoeff}{\ensuremath{\gamma}}  % billiard wall restitution coeff
\newcommand{\timeIn}[1]{{t^{-}_{#1}}} % billiard wall time of arrival
\newcommand{\timeOut}[1]{{t^{+}_{#1}}}   % billiard wall time of departure
%\newcommand{\PoincS}{\partial{\cal M}}          % billiard Poincare section
\newcommand{\Lop}{\ensuremath{{\cal L}}}       % evolution operator
\newcommand{\Uop}{\ensuremath{{\cal K}}}       % Koopman operator, Driebe notation
\newcommand{\Aop}{\ensuremath{{\cal A}}}       % evolution generator
\newcommand{\Top}{\ensuremath{{\cal T}}}       % transfer operator, like in statmech
\newcommand{\matId}{\ensuremath{{\bf 1}}}      % matrix identity
\newcommand{\eigenvL}{\ensuremath{s}}      % evolution operator eigenvalue
\newcommand{\eigenvG}{\ensuremath{m}}      % compact group eigenvalues
\newcommand{\gSpace}{\ensuremath{{\bf \theta}}}   % group rotation parameters
\newcommand{\inFix}[1]{{\in \mbox{\footnotesize Fix}f^{#1}}}
\newcommand{\inZero}[1]{{\in \mbox{\footnotesize Zero} \, f^{#1} }}
\newcommand{\xzero}[1]{{x_{#1}^{*}}}
\newcommand{\fractal}{{\cal F}}
\newcommand{\contract}{F}
\newcommand{\presentation}{P}
\newcommand{\orderof}[1]{o(#1)} % Rytis 22mar2005

     %%%%%%%%%% flows: %%%%%%%%%%%%%%%%%%%%%%%%%%%%
\newcommand\flow[2]{{f^{#1}(#2)}}
\newcommand{\vel}{\ensuremath{v}}   % state space velocity
\newcommand\velField[1]{{v(#1)}}    % ODE velocity field
\newcommand\invFlow{F}
\newcommand\hflow[2]{{\hat{f}^{#1}(#2)}}
\newcommand\timeflow{{f^t}}
\newcommand\tflow[2]{{\tilde{f}^{#1}(#2)}}
%\newcommand\tflow{\tilde{f}^\tau}        %RECHECK USE OF THIS!
\newcommand\xInit{{x_0}}        %initial x
%\newcommand\xInit{\xi}     %initial x, Spiegel notation
\newcommand\pSpace{x}       % phase space x=(q,p) coordinate
\newcommand\coord{q}        % configuration space p coordinate
\newcommand{\para}{\parallel}
\newcommand\stagn{*}        %equilibrium/stagnation point suffix
% \newcommand\stagn{q}      %equilibrium/stagnation point suffix
\newcommand\multiX{x}       %multi point n-dim vector
\newcommand\multiF{f}       %multi point n-dim vector mapping

     %%%%%%%%%% periods: %%%%%%%%%%%%%%%%%%%%%%%%%%%%
\newcommand\period[1]{{T_{#1}}}         %continuous cycle period
%\newcommand\period[1]{{\tau_{#1}}}
\newcommand{\cl}[1]{{n_{#1}}}   % discrete length of a cycle, Predrag
%\newcommand{\cl}[1]{|#1|}  % the length of a periodic orbit, Ronnie
\newcommand{\nCutoff}{N}    % maximal cycle length
                % maximal stability cutoff:
\newcommand{\stabCutoff}{\ExpaEig_{\mbox{\footnotesize max}}}
\newcommand{\timeSegm}[1]{{\tau_{#1}}}      %billiard segment time period
\newcommand{\timeStep}{\ensuremath{{\delta \tau}}}  %integration step
\newcommand{\deltaX}{\ensuremath{{\delta x}}}       %trajectory displacement
\newcommand{\unitVec}{\ensuremath{\hat{n}}}     %unit vector
\newcommand{\shift}{\ensuremath{d}}

\newcommand{\Mvar}{\ensuremath{A}}  % stability matrix
\newcommand{\derF}[1]{\ensuremath{A(#1)}}   % Predrag stability matrix
 %\newcommand{\derF}[1]{{DF |_{#1}}}        % Gibson stability matrix
\newcommand{\jMps}{\ensuremath{J}}   % fundamental matrix, phase space
% \newcommand{\jMps}{\ensuremath{{\bf J}}}  % bold fundamental matrix phase space
\newcommand{\derf}[2]{\ensuremath{{J}^{#1}(#2)}}    % Predrag fundamental matrix
% \newcommand{\derf}[2]{\ensuremath{{\bf J}^{#1}(#2)}}  % Predrag bold fundamental matrix
 % \newcommand{\derf}[2]{{Df^{#1}|_{#2}}}   % Gibson fundamental matrix
\newcommand{\jMConfig}{\ensuremath{{\bf j}}}    % fundamental matrix, configuration space
\newcommand{\jConfig}{\ensuremath{j}}      % jacobian, configuration space
\newcommand{\jMP}{\ensuremath{\hat{J}}}   % jacobian matrix, Poincare return
% \newcommand{\jMP}{\ensuremath{{\bf \hat{J}}}}   % bold jacobian matrix, Poincare return
\newcommand{\monodromy}{\ensuremath{M}}   % monodromy matrix, full Poincare cut
% \newcommand{\monodromy}{\ensuremath{{\bf M}}}   % bold monodromy matrix, full Poincare cut
                   % Fredholm det jacobian weight:
\newcommand{\jEigvec}[1]{{\bf e}^{(#1)}}   % jacobiam eigenvector
\newcommand{\oneMinJ}[1]
           {\left|\det\!\left(\matId-\monodromy_p^{#1}\right)\right|}
\newcommand{\maslovInd}{\ensuremath{m}}        % Maslov index
\newcommand{\ExpaEig}{\ensuremath{\Lambda}}
\newcommand{\Lyap}{\ensuremath{\lambda}}            %Lyapunov exponent

%%   optional parameter comes in [\ldots], for example
%%   \newcommand\eigRe[1][ ]{\ensuremath{\mu_{#1}}}
%%   no subscript: \eigRe\
%%   with subscript j: \eigRe[j]
%%
\newcommand{\eigExp}[1][ ]{\ensuremath{\lambda_{#1}}}   % complex eigenexponent
%%  Guckenheimer&Holmes:  lambda = alpha + i beta
%%  Hirsch-Smale:         lambda = a     + i b
%%  Boyce-di Prima:       lambda = mu    + i nu
\newcommand{\eigRe}[1][ ]{\ensuremath{\mu_{#1}}}    % Re eigenexponent
\newcommand{\eigIm}[1][ ]{\ensuremath{\nu_{#1}}}    % Im eigenexponent

\newcommand\LyapTime{T_{\mbox{\footnotesize Lyap}}} %Lyapunov time
\newcommand{\hatx}{{\hat{x}}}
% \newcommand{\hatx}{{\hat{x}_t}}               %RECHECK USE OF THIS!
\newcommand{\hatp}{{\hat{x}_p}}
\newcommand{\phat}{{\hat{p}}}
\newcommand{\curvR}{\rho}           %billiard curvature
\newcommand{\dz}{{\delta z}}
\newcommand{\dth}{{\delta \theta}}
\newcommand{\delh}{{\delta h}}
\newcommand{\PP}{{\cal P}}
\newcommand{\NN}{{\cal N}}

%%%%%%%%%%%%%%% symbolic dynamics %%%%%%%%%%%%%%%%%%%%%%%%%%%%%%%%%%
\newcommand{\MarkGraph}{Markov graph}
\newcommand{\admissible}{admissible}
\newcommand{\inadmissible}{inadmissible}
\newcommand{\cycle}[1]{{$\overline{#1}$}}
\newcommand{\cycpt}{_{p,m}}
\newcommand\sumprime{\mathop{{\sum}'}}
\newcommand{\pseudos}{\pi}
% \newcommand{\pseudos}{{p_1+p_2+\dots+p_k}}
% \newcommand{\pseudos}{{\{p_1 p_2 \dots p_k\}}}
\newcommand{\block}[1]{#1}
\newcommand{\prune}[1]{\_{#1}\_}        % fits into math env.
%\newcommand{\prune}[1]{\ldots{#1}\ldots}
\newcommand{\strng}[1]{$\_#1\_$}    % fits into text without $'s
\newcommand{\biinf}[2]{\cdots#1.#2\cdots}
\newcommand{\rctngl}[2]{[#1.#2]}
\newcommand{\BKsym}[1]{S_{#1}}
\newcommand{\Ksym}[1]{\sigma_{#1}}
\newcommand{\Ssym}[1]{{s_{#1}}}
\newcommand{\gmax}{\hat{\gamma}}
\newcommand{\Spast}{S^\textrm{-}}       % past itenerary
\newcommand{\Sfuture}{S^\textrm{\scriptsize +}} % future itenerary
\newcommand{\Sbiinf}{S}             % biinf. itenerary
\newcommand{\str}{\epsilon_{1},\epsilon_{2}, \ldots } % Ronnie's problems

%%%%%%%%%%%%%%% Ronnie's problems %%%%%%%%%%%%%%%%%%%%%%%%%%%%%%%%
\newcommand{\estr}[1] {\epsilon_{1},\epsilon_{2}, \ldots, \epsilon_{#1}}
\newcommand{\eestr}[2] {#1,\epsilon_{1},\epsilon_{2}, \ldots,\epsilon_{#2}}

%%%%%%%%%%%%%% diag.tex specific %%%%%%%%%%%%%%%%%%%%%%%%%%%%
\newcommand{\Lmat}[1]{{{\bf L}_{#1}}}      % evolution matrix

%%%%%%%%%%%% loopDef.tex, defCrete.tex specific %%%%%%%%%%%%%
% Predrag   defCrete.tex             4mar2003
% Predrag   loopDefs.tex            10jul2003
\newcommand{\descent}{Newton descent}
\newcommand{\Descent}{Newton Descent}
\newcommand{\CostFct}{Cost function}    % functional to minimize
\newcommand{\costFct}{cost function}    % functional to minimize
\newcommand{\costF}{F^2}        % cost function, 
\newcommand{\Loop}{L}
\newcommand{\pVeloc}{v}         % phase-space velocity 
\newcommand{\lSpace}{\tilde{x}}     % a point on a loop 
\newcommand{\lVeloc}{\tilde{v}}     % loop tangent
\newcommand{\damp}{\Delta\tau}      % descrete fititous time step
% \newcommand{\pSpaceDer}[1]{x^{(#1)}}
% \newcommand{\lSpaceDer}[1]{\tilde{x}^{(#1)}}

%%%%%%%%%%%%%% ks.tex specific %%%%%%%%%%%%%%%%%%%%%%%%%%%%
\newcommand{\KS}{Kuramoto-Sivashinsky}
\newcommand{\KSe}{Kuramoto-Sivashinsky equation}
\newcommand{\PCf}{Plane Couette flow}
\newcommand{\dmn}{\ensuremath{\!-\!d}}             %  n-dimensional
\newcommand{\expctE}{\ensuremath{E}}    % E space averaged
\newcommand{\tildeL}{\ensuremath{\tilde{L}}}
% Ruslan likes:
\newcommand{\EQV}[1]{\ensuremath{E_{#1}}}
% E_0: u = 0 - trivial equilibrium
% E_1,E_2,E_3, for 1,2,3-wave equilibria
\newcommand{\REQV}[2]{\ensuremath{TW_{#1#2}}} % #1 is + or -
% TW_1^{+,-} for 1-wave travelling waves (positive and negative velocity).
\newcommand{\PO}[1]{\ensuremath{PO_{#1}}}
% PO_{period to 2-4 significant digits} - periodic orbits
\newcommand{\RPO}[1]{\ensuremath{RPO_{#1}}}
% RPO_{period to 2-4 significant digits} - relative PO.  We use ^{+,-}
% to distinguish between members of a reflection-symmetric pair.

%%%%%%%%%%%%%%% Lorentz gas section %%%%%%%%%%%%%%%%%%%%%%%%%%%%%%%%
\def\hn{\hat n}
\newcommand\hM{\hat \pS}
%\def\hM{\widehat M}
\newcommand\hx{\hat x}
\def\tx{\tilde x}
\def\tpk{_{\tilde p,k}}
\def\tpk{}              %why redefined?
\def\ttime{\sigma_{\tilde{p}}}

%%%%%%%%%%%%%%% Henon map section %%%%%%%%%%%%%%%%%%%%%%%%%%%%%%%%%%
\newcommand{\logisticm}{quadratic map}
\newcommand{\Logisticm}{Quadratic map}
\newcommand{\stretchf}{``stretch \&\ fold''}
\newcommand{\Stretchf}{``Stretch \&\ fold''}
\newcommand{\ofm}{once-folding map}
\newcommand{\Ofm}{Once-folding map}
\newcommand{\mHt}{map of the H\'enon type}
\newcommand{\mHts}{maps of the H\'enon type}
\newcommand{\MHts}{Maps of the H\'enon type}
\newcommand{\opres}{orientation preserving}
\newcommand{\Opres}{Orientation preserving}
\newcommand{\orev}{orientation reversing}
\newcommand{\Orev}{Orientation reversing}
\newcommand{\nws}{non--wandering set}
\newcommand{\stranges}{non--wandering set}
%\newcommand{\stranges}{strange set}
\newcommand{\ki}{kneading value}
\newcommand{\Ki}{Kneading value}
\newcommand{\ks}{kneading sequence}
\newcommand{\turn}{turning point}    % {turnback} ??
\newcommand{\pturn}{primary turning point}    % {turnback} ??
\newcommand{\Pturn}{Primary turning point}    % {Turnback} ??
\newcommand{\topc}{topological coordinate}
\newcommand{\Topc}{Topological coordinate}
\newcommand{\critVal}{f(x_c)}
\newcommand{\topcv}{maximal value}
\newcommand{\Topcv}{Maximal value}
\newcommand{\toppar}{topological parameter}
\newcommand{\toppp}{topological parameter plane}
\newcommand{\topp}{symbol square}
\newcommand{\Topp}{Symbol square}
\newcommand{\bimappr}{bimodal approximation}
\newcommand{\Bimappr}{Bimodal approximation}
\newcommand{\henappr}{bimodal approximation}
\newcommand{\fourfa}{four-folds approximation}
\newcommand{\Fourfa}{Four-folds approximation}
\newcommand{\snbif}{saddle-node bifurcation}
\newcommand{\Snbif}{Saddle-node bifurcation}

%%%%%%%%%%%%%%% symmetric, asymmetric orbits: %%%%%%%%%%%%%%%%%%%%%%%%%%%%
\newcommand{\sym}{{s}}
\newcommand{\nsym}{{n_s}}
\newcommand{\asym}{{a}}
\newcommand{\nasym}{{n_a}}
% fundamental domain:
\newcommand{\pf}{{\tilde p}}
\newcommand{\nf}{n_{\tilde p}}
\newcommand{\symf}{{\tilde s}}
\newcommand{\nsymf}{n_{\tilde s}}

%%%%%%%%%%%%%%% relative periodic orbits: %%%%%%%%%%%%%%%%%%%%%%%%%%%%

\newcommand{\po}{periodic orbit}
\newcommand{\Po}{Periodic orbit}
\newcommand{\rpo}{relative periodic orbit}
%   \newcommand{\rpo}{equivariant periodic orbit}
\newcommand{\Rpo}{Relative periodic orbit}
%   \newcommand{\Rpo}{Equivariant periodic orbit}

\newcommand{\eqv}{equilibrium}
\newcommand{\Eqv}{equilibrium}
\newcommand{\eqva}{equilibria}
\newcommand{\Eqva}{Equilibria}
\newcommand{\reqv}{relative equilibrium}
%   \newcommand{\reqv}{equivariant equilibrium}
%   \newcommand{\reqv}{travelling wave}
\newcommand{\Reqv}{Relative equilibrium}
%   \newcommand{\Reqv}{Equivariant equilibrium}
%   \newcommand{\Reqv}{travelling wave}
\newcommand{\reqva}{relative equilibria}
%   \newcommand{\reqva}{equivariant equilibria}
\newcommand{\Reqva}{Relative equilibria}
%   \newcommand{\Reqva}{Equivariant equilibria}

\newcommand{\rpprime}{{\tilde{p}}}  % relative periodic prime orbit

%%%%%%%%%%%%%% Quantum mechanical stuff %%%%%%%%%%%%%%%%%%%%%%%%%%%%
\newcommand{\HamPrincFct}[4]{R_{#4}({#1},{#2},{#3})}
               % \HamPrincFct{q}{q'}{t}{j}

%%%%%%%%%%%%%% SPECIFIC TO lattFT.tex NOTES %%%%%%%%%%%%%%%%%%%%%%%%%%%%
\newcommand{\unit}{{\bf 1}}
\newcommand{\hopMat}{{\bf h}}
\newcommand{\hop}{h}
\newcommand{\fix}{\marginpar{$\diamond$}}
\newcommand{\source}{{J}}
\newcommand{\sourceFT}{{\tilde{J}}}
\newcommand{\derSource}{{d~\over d\source}}
\newcommand{\derSourceFT}{{d~\over d\sourceFT}}
\newcommand{\field}{{\phi}}     % used in lattFT.tex 
%\newcommand{\field}{{x}}       % not a good notation
\newcommand{\fieldFT}{{\tilde{\phi}}}
\newcommand{\derField}{{d~\over d\field}}
\newcommand{\saddleField}{{\field^c}}
\newcommand{\saddleCoord}{{\coord^c}}
\newcommand{\Laplacian}{\Delta}
% \newcommand{\Prpgtr}{{G_0}}       % modified in lattFT.tex 
\newcommand{\Prpgtr}{{M}}
\newcommand{\PrpgtrFT}{{\tilde{G}_0}}
% \newcommand{\InvPrpgtr}{{G_0^{-1}}}   % modified in lattFT.tex
\newcommand{\InvPrpgtr}{{M^{-1}}}
\newcommand{\GreenF}{{G}}
\newcommand{\Df}[1]{f^{'}_{#1}}
\newcommand{\nosum}{\not\!\!{\scriptstyle\sum}}
\newcommand{\doublespace}{\baselineskip = \normalbaselineskip \multiply\baselineskip by 2}
%%%%%%%%%%%%%% end of SPECIFIC TO lattFT.tex NOTES %%%%%%%%%%%%%%%%%%%%%%%%%%%%


%%%%  gli commandi di Commandottore Roberto   %%%%%%%%%%%%

\newcommand {\tidue}{{\mbox{\bf T}}^{2}}
\newcommand {\bom}[1]{\mbox{\boldmath $#1$}}
\newcommand {\polit}{{\cal P}_{T}}
\newcommand {\id}{{\ \hbox{{\rm 1}\kern-.6em\hbox{\rm 1}}}}
\newcommand {\ep}{\epsilon}

%%%%%%% Wirzba scattering.tex  %%%%%%%%%%%%%%%%%%%%%%%%%%%%

\newcommand{\gesim}{\mbox{\raisebox{-.6ex}{$\,{\stackrel{>}{\sim}}\,$}}}
\newcommand{\lesim}{\mbox{\raisebox{-.6ex}{$\,{\stackrel{<}{\sim}}\,$}}}
\newcommand{\Ageom}[1]{{\bf A}^{#1}}
\newcommand{\Aghost}[1]{{\underline{\bf A}}^{#1}}
\newcommand{\Acreep}[1]{{\mathbb{A}}^{#1}}
%\newcommand{\Acreep}[1]{{\hat{\bf A}}^{#1}}

%%%%%%%%%%%%%%%%%%%%%% birdtracks SPECIFIC %%%%%%%%%%%%%%%%%%%%%%%%%%%%%%%
%% from def_group.tex

%% Young diagrams (multiplication-stuff due to C. Holm -- cheers!)
%% command \btrackYt[size of one box (optional)]{filename}{number of boxes}
\newdimen\onebox
\newdimen\boxsize
\newcount\boxnum
\gdef\mult#1#2#3{% #1 = #2 * #3 
    \ifx#1\relax\else%
      \ifx#2\relax\else%
        #1=#2%
        \ifx#3\relax\else%
          \multiply#1#3%
        \fi%
      \fi%
    \fi}

\newcommand{\btrack}[1]{\raisebox{-2.0ex}[3.5ex][2.5ex]
    {\includegraphics[height=5ex]{Fig/f_#1.eps}\negthinspace} }
    %{\epsfig{file=Fig/f_#1.eps,height=5ex}\negthinspace} }
%% A is 7/5-ths taller
\newcommand{\btrackA}[1]{\raisebox{-3.0ex}[4.5ex][3.5ex]
         { \epsfig{file=Fig/f_#1.eps,height=7ex}\negthinspace} }
%% B is 9/5-ths taller
\newcommand{\btrackB}[1]{\raisebox{-4.0ex}[5.5ex][4.5ex]
          { \epsfig{file=Fig/f_#1.eps,height=9ex}\negthinspace} }
%% BB is 11/5 larger
\newcommand{\btrackBB}[1]{\raisebox{-5.0ex}[6.5ex][5.5ex]
          { \epsfig{file=Fig/f_#1.eps,height=11ex}\negthinspace} } 
%% C is 1/2 smaller
\newcommand{\btrackC}[1]{\raisebox{-0.4ex}[1.75ex][1.25ex]
          { \epsfig{file=Fig/f_#1.eps,height=2.5ex}\negthinspace} }
%% birtrack to be drawn:
\newcommand{\zzzz}{{\tt birdTrack}}
%%   Birdtracks with vertical alignment info
%%%% copied from Anders Johansen inputs/anders_def.tex  15 May 2002 
\newlength{\verti}
\newcommand{\btrackAl}[3]{%
    \setlength{\verti}{-#3pt*5+2.5pt}% -(5pt*m)+2.5pt  m=#3
    \setlength{\boxsize}{#2pt*5}%
    \raisebox{\verti}{\includegraphics[width=\boxsize]{Fig/f_#1.eps}}}
%% Birdtracks with sizes in terms of #Young diagram boxes
\newcommand{\btrackYq}[3][5pt]{%
    \boxnum=#3%
    \onebox=#1%
    \mult{\boxsize}{\onebox}{\boxnum}%
    \parbox{\boxsize}{\includegraphics[width=\boxsize]{Fig/f_#2.eps}}}

%%%%%%%%%%%%%%%%%% FEYNMANN DIAGRAMS %%%%%%%%%%%%%%%%%%%%%%%%%%%%%%%%
\thicklines
\newlength{\Fsize}   % allow for easy resizing of diagrams
\newlength{\Fdotsize}
\setlength{\Fsize}{20pt}
\setlength{\Fdotsize}{5pt}
\setlength{\unitlength}{\Fsize}
\newcommand{\Fdot}{  % vertex
        \begin{picture}(0,0)
        \setlength{\unitlength}{\Fdotsize}
    \put(0,0){\circle*{1}}
        \end{picture}}
%Propagator naming conventions: \F(d|D)(h|[c](u|d|l|r)(u|d|l|r))
%d=dotted, D=solid, h=horizontal, c=curved, u=up, d=down, l=left, r=right
%The straight propagators are specified u|d then l|r
%The curved propagators end up in the location specified by the last letter
\newcommand{\Fdh}{   % horizontal dotted propagator
    \begin{picture}(0,0)
    \setlength{\unitlength}{\Fsize}
    \qbezier[10](0,0)(0.5,0)(1,0)
    \end{picture}}
\newcommand{\FDh}{   % horizontal solid propagator
        \begin{picture}(0,0)
        \setlength{\unitlength}{\Fsize}
    \put(0,0){\line(1,0){1}}
        \end{picture}}
\newcommand{\FDur}{  % diagonal solid propagators
        \begin{picture}(0,0)
        \setlength{\unitlength}{\Fsize}
    \put(0,0){\line(1,1){0.7}}
        \end{picture}}
\newcommand{\FDdr}{ 
        \begin{picture}(0,0)
        \setlength{\unitlength}{\Fsize}
        \put(0,0){\line(1,-1){0.7}}
        \end{picture}}
\newcommand{\FDul}{  
        \begin{picture}(0,0)
        \setlength{\unitlength}{\Fsize}
        \put(0,0){\line(-1,1){0.7}}
        \end{picture}}
\newcommand{\FDdl}{
        \begin{picture}(0,0)
        \setlength{\unitlength}{\Fsize}
        \put(0,0){\line(-1,-1){0.7}}
        \end{picture}}
\newcommand{\Fdcul}{  % curved propagators
        \begin{picture}(0,0)
        \setlength{\unitlength}{\Fsize}
    \qbezier[15](0,0)(-1,1)(-1,0)
        \end{picture}}
\newcommand{\FDcdl}{  
        \begin{picture}(0,0)
        \setlength{\unitlength}{\Fsize}
        \qbezier(0,0)(-1,-1)(-1,0)
        \end{picture}}
\newcommand{\Fdcur}{
        \begin{picture}(0,0)
        \setlength{\unitlength}{\Fsize}
        \qbezier[15](0,0)(1,1)(1,0)
        \end{picture}}
\newcommand{\FDcur}{
        \begin{picture}(0,0)
        \setlength{\unitlength}{\Fsize}
        \qbezier(0,0)(1,1)(1,0)
        \end{picture}}
\newcommand{\FDcdr}{
        \begin{picture}(0,0)
        \setlength{\unitlength}{\Fsize}
        \qbezier(0,0)(1,-1)(1,0)
        \end{picture}}
\newcommand{\Fdclu}{
        \begin{picture}(0,0)
        \setlength{\unitlength}{\Fsize}
        \qbezier[20](0,0)(-1,1)(1,1)
        \end{picture}}
\newcommand{\FDcld}{
        \begin{picture}(0,0)
        \setlength{\unitlength}{\Fsize}
        \qbezier(0,0)(-1,-1)(1,-1)
        \end{picture}}
\newcommand{\FDcru}{
        \begin{picture}(0,0)
        \setlength{\unitlength}{\Fsize}
        \qbezier(0,0)(1,1)(-1,1)
        \end{picture}}
\newcommand{\FDcrd}{
        \begin{picture}(0,0)
        \setlength{\unitlength}{\Fsize}
        \qbezier(0,0)(1,-1)(-1,-1)
        \end{picture}}
\setlength{\unitlength}{1pt}


%%%%%%%%%%%%%%  Bibliography abbreviations %%%%%%%%%%%%%%%%%%%%%%%%%%%%%%%%

%\newcommand{\AP}[1]{{\em Ann.\ Phys.}\/ {\bf #1}}
%\newcommand{\CHAOS}[1]{{\em CHAOS}\/ {\bf #1}}
%\newcommand{\CM}[1]{{\em Cont.\ Math.}\/ {\bf #1}}
%\newcommand{\CMP}[1]{{\em Commun.\ Math.\ Phys.}\/ {\bf #1}}
%\newcommand{\INCB}[1]{{\em Il Nuov.\ Cim.\ B}\/ {\bf #1}}
%\newcommand{\JCP}[1]{{\em J.\ Chem.\ Phys.}\/ {\bf #1}}
%\newcommand{\JETP}[1]{{\em Sov.\ Phys.\ JETP}\/ {\bf #1}}
%\newcommand{\JETPL}[1]{{\em JETP Lett.}\/ {\bf #1}}
%\newcommand{\JMP}[1]{{\em J.\ Math.\ Phys.}\/ {\bf #1}}
%\newcommand{\JMPA}[1]{{\em J.\ Math.\ Pure Appl.}\/ {\bf #1}}
%\newcommand{\JPA}[1]{{\em J.\ Phys.}\/ {\bf A  #1}}
%\newcommand{\JPB}[1]{{\em J.\ Phys.}\/ {\bf B  #1}}
%\newcommand{\JPC}[1]{{\em J.\ Phys.\ Chem.}\/ {\bf #1}}
%\newcommand{\JchemP}[1]{{\em J.\ Chem.\ Phys.}\/ {\bf #1}}
%\newcommand{\NPA}[1]{{\em Nucl.\ Phys.}\/ {\bf A #1}}
%\newcommand{\NPB}[1]{{\em Nucl.\ Phys.}\/ {\bf B #1}}
%\newcommand{\NONLIN}[1]{{\em Nonlinearity}\/ {\bf #1}}
%\newcommand{\PLA}[1]{{\em Phys.\ Lett.}\/ {\bf A #1}}
%\newcommand{\PLB}[1]{{\em Phys.\ Lett.}\/ {\bf B #1}}
%\newcommand{\PRA}[1]{{\em Phys.\ Rev.}\/ {\bf A #1}}
%\newcommand{\PRD}[1]{{\em Phys.\ Rev.}\/ {\bf D #1}}
%\newcommand{\PRL}[1]{{\em Phys.\ Rev.\ Lett.}\/ {\bf #1}}
%\newcommand{\PST}[1]{{\em Phys.\ Scripta}\/ {\bf T #1}}
%\newcommand{\RMS}[1]{{\em Russ.\ Math.\ Surv.}\/ {\bf #1}}
%\newcommand{\USSR}[1]{{\em Math.\ USSR.\ Sb.}\/ {\bf #1}}
%\newcommand{\ZNat}[1]{{\em Z. Naturforschung}\/ {\bf #1}}

%%%%%%%%%%%%      stuff below this line will probably be dropped%%%%%%%%%%%

\renewcommand{\b}{\beta}
\newcommand{\w}{\omega}
\newcommand{\p}{2\pi}
\newcommand{\J}{\mbox{  \rule[.03ex]{.03em}{1.5ex} \hspace*{-0.9em} \rm J}}
\newcommand{\f}{\varphi}

% \newcommand{\R}{\mbox{  \rule[.07ex]{.03em}{1.5ex} \hspace*{-0.75em} \rm R}}
% \newcommand{\nextchapter}{\newpage{\pagestyle{empty}\cleardoublepage}}

%\newcommand {\vett}[2]{\bepar{c} #1 \\ #2 \epar}
%\newcommand {\vettc}[2]{\bepar{c,c} #1 & #2 \epar}
%\newcommand {\vettq}[2]{\left[ \begin{array}{c,c} #1 & #2 \end{array} \right]}
%\newcommand {\ron}{\rho_{n}}
%\newcommand {\ronu}{\rho_{n+1}}
%\newcommand {\ggat}{\Gamma_{T}}
%\newcommand {\gga}{\Gamma}
%\newcommand {\ggan}{\Gamma_{n}}
%\newcommand {\figurino}[3]{\begin{figure}[b]  \vspace{#1}
%            \caption{\protect\small #2}  \label{#3} \end{figure}}
% \newcommand {\zitadef}{\prod_{p}(1-t_{p})}
%\def\t{\tilde}
%\def\h{\hat}
%\def\tf{\tilde f}
%\def\hf{\hat f}
% \def\time{\sigma_p}
%       %% try to understand what does this do?:
% \def\pmb#1{\setbox0=\hbox{$#1$}\kern-.025em\copy0\kern-\wd0
%        \kern-0.05em\copy0\kern-\wd0\kern-.025em\raise.0233em\box0}

    % Roberto's versions:
% \newcommand{\reals}{\mbox{\bf R}} % used in Ronnie's problems
% \newcommand {\natur}{{\hbox{{\rm I}\kern-.2em\hbox{\rm N}}}}
%\newcommand {\relativi}{{\ \hbox{{\rm Z}\kern-.4em\hbox{\rm Z}}}}
% \newcommand {\reali}{{\hbox{{\rm I}\kern-.2em\hbox{\rm R}}}}
% \newcommand{\RR}{{I\negthinspace\!R}}     % Reals: we honor Tresser

%%%% all \Fig to be eliminated in favor of \FIG eventually %%%%%%%%%
%%%% eventually eliminate epsfig by \includegraphics everywhere %%%%
%   \Fig{#1}    % \epsfig{file=Fig/f_name.ps,width=?cm} ... here
%   {#2}    % short caption text
%   {#3}    % full caption text
%   {#4}    % f-figure-label
%       defined here:
% \newcommand{\Fig}[4]{\begin{figure}
%           \centering{#1}
%                       \caption[#2]{#3}
%                       \label{#4} \end{figure} }

%%%%%%%%%%%%%%%%%chapter bibliography%%%%%%%%%%%%%%%%%%%%%%%%%%%%%%%%%
% from LaTeX style file for the CUP standard designs 5a to 9b
% Copyright 1993 Cambridge University Press
% \def\thereferences#1{\section*{References}\bibliographylist}
% \let\endthereferences=\endlist
% \def\bibliographylist{%
%  \small\raggedright
%  \list{}{\labelwidth\z@
%   \leftmargin 2em% \advance\leftmargin\labelsep
%   \itemsep \z@ plus .1pt
%   \itemindent-\leftmargin}
%  \parindent\z@
%  \parskip\z@ plus .1pt\relax
%  \def\newblock{\hskip .11em plus .33em minus .07em}
%  \sloppy\clubpenalty4000\widowpenalty4000
%  \sfcode`\.=1000\relax
% }
% \let\endbibliographylist=\endlist
% 
% 
%%%%%%%%%%%%%%%%%%%%%%%%%%%%%%%%%%%%%%%%%%%%%%%%%%%%%%%%%%%%%%%%%%%%%%%%%%%

%%%%%%%%%%%%%%%%%%%%%%%%%%%%%%%%%%%%%%%%%%%%%%%%%%%%
% changes bars package collides with everything, abandoned Apr 2000
% \setlength{\changebarwidth}{0.5cm}    % margin changes bars width
% %?\setcounter{\changebargray}{85}     % margin changes bars blackness
% \newcommand{\Preliminary}[1]{\cbstart #1 \cbend}
%%%%%%%%%%%%%%%%%%%%%%%%%%%%%%%%%%%%%%%%%%%%%%%%%%%%
 
