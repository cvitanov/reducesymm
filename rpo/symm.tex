% symm.tex
%
% Predrag created file				jul 3 2006

\section{Symmetries of \KSe}

The full KSE has a continuous symmetry: if
$u(x,t)$ is a solution, then so is $u(x+d,t)$ for any
$0 < d \leq L$.  As a result,
the KS has \rpo s with nonzero shift $-L/2 < d \leq L/2$
\[ u(x+d,\period{}) = u(x,0)
\,.
\]
where $\period{}$ is the period. 

Further down the road: we need to add this
$\sigma x = -x$ symmetry
 to the continuous $O(1)$ rotation; then many of the existing \rpo s will
 halve their period, and symmetric pairs will be eliminated.

Fourier coefficients which respect the $x \to -x$ symmetry of
\KSe, see discussion in \refref{Christiansen:97},
and references therein.

By symmetry there might be an equilibrium on the reflection plane that
relates the equilibrium A and its symmetry partner SA; the 3 equilibria would
be analogous to what you see in the Lorentz attractor pictures, crossing
the unstable manifold of the central one throws you into the neighborhood
of the other equilibrium.

On the 
	$[a_?,a_?]$ plane
	the $\sigma x = -x$ symmetry of \KSe\ is explicit.

The two equilibria
capture qualitatively the co-moving frame \rpo\ [NAMEIT] shape,
which follows the
equilibrium for most of the time, except for a quick swing where it
sidesteps by $d/4$, just as it does in \reffig{f:rpoNAMEIT}. 

For discrete rotations the spectral determinants factorize
nicely in terms of rpo's:
read ``Discrete symmetries" chapter of ChaosBook.org - not an easy read, but
it also uses $g\jMps$ rather than the naked $\jMps$,
and a trace formula for irrational
$d/L$ still puzzles me - for $d/L$ rational the determinant factorizes using
discrete Fourier transform.
We are fuzzy on the continuum limit of that.

