% KSe.tex
% Predrag			jul  3 2006
% extracted from newton.tex
% Predrag			jun 20 2006
% Vaggelis			may 20 2006

\section{\KSe\ according to Evangelos}

The \KSe\ % (KSe)
reads:
 \beq
  u_t=(u^2)_x-u_{xx}- u_{xxxx} \, ,
  \label{eq:KS}
 \eeq

 We assume periodic boundary conditions on the $x\in [0,2\pi \tilde{L}]$
 interval:
 \beq
   u(x+2\pi\tilde{L},t)=u(x,t) \, ,
 \eeq
 which allows a Fourier series expansion:
 \beq
  u(x,t)=\sum_{k=-\infty}^{+\infty} a_k (t) e^{ i k x / \tilde{L}} \, .
  \label{eq:Fourier}
 \eeq
 Since $u(x,t)$ is real,
 \beq
  a_{k}=a^*_{-k} \, .
  \label{eq:a*}
 \eeq
 Substituting \refeq{eq:Fourier} into \refeq{eq:KS} we get:
 \beq
  \dot{a}_k=(k/\tildeL)^2\left(1-(1/\tildeL)^2 k^2\right)a_k
        + i (k/\tildeL)  \sum_{m=-\infty}^{+\infty}a_m a_{k-m} \, .
  \label{eq:Fcoef}
 \eeq

 From \refeq{eq:Fcoef} we notice that $\dot{a}_0=0$ and thus $a_0$ is an integral
 of the equations or, from \refeq{eq:Fourier}, the average of the solution $\int dx u(x,t)$
 is a constant. Due to galilean invariance we may set $a_0=0$ without loss of generality 
 and we only have to compute $a_k$'s with $k\geq 1$. % Explain this in detail somewhere.

 Truncating the infinite tower of equations by setting $a_k=0$ for $k>d$, using the identity $a_{-k}=a^*_k$ and splitting the
 resulting equations into real and imaginary part by setting $a_k=b_k+i c_k$, we have
  
 \bea
  \dot{b}_k & = & \left(\frac{k}{\tildeL}\right)^2\left(1- \left(k/\tildeL\right)^2 \right)b_k  \continue
	& & - \frac{k}{\tildeL} \left(\sum_{m=1}^{k-1}c_m b_{k-m}+\sum_{m=k+1}^{N}c_m b_{m-k}
                    -\sum_{m=1}^{N-k}c_m b_{k+m} \right)  \continue
	& & - \frac{k}{\tildeL} \left(\sum_{m=1}^{k-1}b_m c_{k-m}-\sum_{m=k+1}^{N}b_m c_{m-k}
                    +\sum_{m=1}^{N-k}b_m c_{k+m} \right)		  
  \label{eq:tmp:b-Trunc}
 \eea
 \bea
   \dot{c}_k & = & \left(\frac{k}{\tildeL}\right)^2\left(1- \left(k/\tildeL\right)^2 \right)c_k  \continue
	& & - \frac{k}{\tildeL}\left( \sum_{m=1}^{k-1}c_m c_{k-m}-\sum_{m=k+1}^{N}c_m c_{m-k}
                    -\sum_{m=1}^{N-k}c_m c_{k+m} \right)	\continue
	& & + \frac{k}{\tildeL} \left(\sum_{m=1}^{k-1}b_m b_{k-m}+\sum_{m=k+1}^{N}b_m b_{m-k}
                    +\sum_{m=1}^{N-k}b_m b_{k+m} \right)
   \label{eq:tmp:c-Trunc}
 \eea
 where now only terms $c_{k},b_{k}$ with $0<k<d$ appear. Observe
 \beq
	\sum_{m=1}^{N-k}c_m b_{k+m} = \sum_{m=k+1}^{N}b_m c_{m-k}\,,
 \eeq
 \etc and thus \refeq{eq:tmp:b-Trunc} and \refeq{eq:tmp:c-Trunc} simplify to
  \bea
  \dot{b}_k & = & \left(\frac{k}{\tildeL}\right)^2\left(1- \left(k/\tildeL\right)^2 \right)b_k  \continue
	& & - \frac{k}{\tildeL} \left(\sum_{m=1}^{k-1}c_m b_{k-m}-2\sum_{m=1}^{N-k}c_m b_{k+m} \right)  \continue
	& & - \frac{k}{\tildeL} \left(\sum_{m=1}^{k-1}b_m c_{k-m}+2\sum_{m=1}^{N-k}b_m c_{k+m} \right)		  
  \label{eq:b-Trunc}
 \eea
 \bea
   \dot{c}_k & = & \left(\frac{k}{\tildeL}\right)^2\left(1- \left(k/\tildeL\right)^2 \right)c_k  \continue
	& & - \frac{k}{\tildeL}\left( \sum_{m=1}^{k-1}c_m c_{k-m}-2\sum_{m=1}^{N-k}c_m c_{k+m} \right)	\continue
	& &  +\frac{k}{\tildeL}\left( \sum_{m=1}^{k-1}b_m b_{k-m}+2\sum_{m=1}^{N-k}b_m b_{k+m} \right)\,.
   \label{eq:c-Trunc}
 \eea

 We begin by calculating the matrix of variations $A_{ij} \equiv \frac{\partial v_i(x)}{\partial x_j}$ for the antisymmetric
 subspace for which $b_k=0, c_{-k}=-c_{k}$ and thus
 \beq
	   \dot{c}_k =  \left(\frac{k}{\tildeL}\right)^2\left(1- \left(k/\tildeL\right)^2 \right)c_k
	 		- \frac{k}{\tildeL}\left( \sum_{m=1}^{k-1}c_m c_{k-m}
                    		-2\sum_{m=1}^{N-k}c_m c_{k+m} \right)	\,.	
 \eeq
 
 Then
 \bea
	\frac{\partial \dot{c}_k}{\partial c_{j}}  =  
		\left(\frac{k}{\tildeL}\right)^2\left(1- \left(k/\tildeL\right)^2 \right) \delta_{kj}
			- \frac{k}{\tildeL}\frac{\partial}{\partial c_j}\left( \sum_{m=1}^{k-1}c_m c_{k-m}-2\sum_{m=1}^{N-k}c_m c_{k+m} \right)	\,.	
 \eea
 Concider the second term:
 \bea
	- \frac{k}{\tildeL}\frac{\partial}{\partial c_j}\left( \sum_{m=1}^{k-1}c_m c_{k-m}-2\sum_{m=1}^{N-k}c_m c_{k+m} \right)	& = &
		- \frac{k}{\tildeL} \sum_{m=1}^{k-1} \left(\delta_{m,j} c_{k-m}+c_m \delta_{k-m,j} \right) \continue
						& & + 2 \frac{k}{\tildeL}\sum_{m=1}^{N-k} \left(\delta_{m,j} c_{k+m}+c_m \delta_{k+m,j}\right)
 \eea
 We need to consider two cases separately:
 \begin{itemize} 
	\item $k\leq j$
		\bea
			 -\frac{k}{\tildeL}\frac{\partial}{\partial c_j}\left( \sum_{m=1}^{k-1}c_m c_{k-m}-2\sum_{m=1}^{N-k}c_m c_{k+m} \right)	& = &
					-\frac{k}{\tildeL}( 0+0 ) + 2\frac{k}{\tildeL} (c_{k+j} + c_{j-k}) \continue
				& = &   2 \frac{k}{\tildeL} (c_{k+j}-c_{k-j})
		\eea
	\item $k > j$
		\bea
			 -\frac{k}{\tildeL}\frac{\partial}{\partial c_j}\left( \sum_{m=1}^{k-1}c_m c_{k-m}-2\sum_{m=1}^{N-k}c_m c_{k+m} \right)	& = &
					-\frac{k}{\tildeL}(c_{k-j} + c_{k-j} ) + 2\frac{k}{\tildeL} (c_{k+j}  + 0 ) \continue
				& = &  2 \frac{k}{\tildeL} (c_{k+j}-c_{k-j})
		\eea	
 \end{itemize}
 and thus
 \beq
	\frac{\partial \dot{c}_k}{\partial c_{j}} =  \left(\frac{k}{\tildeL}\right)^2\left(1- \left(k/\tildeL\right)^2 \right) + 2 \frac{k}{\tildeL} (c_{k+j}-c_{k-j})
 \eeq

 For the case of the full space we need to consider the four matrices $\frac{\partial \dot{b}_k}{\partial b_j},\frac{\partial \dot{b}_k}{\partial c_j},\frac{\partial \dot{c}_k}{\partial b_j},\frac{\partial \dot{c}_k}{\partial c_j}$. Following the above procedure
 \beq
	\frac{\partial \dot{c}_k}{\partial b_{j}} =  2 \frac{k}{\tildeL} ( b_{k+j}+b_{k-j} )\,,
 \eeq
 \beq
	\frac{\partial \dot{b}_k}{\partial b_{j}} =  \left(\frac{k}{\tildeL}\right)^2\left(1- \left(k/\tildeL\right)^2 \right)\delta_{kj} - 2 \frac{k}{\tildeL} (c_{k+j} + c_{k-j}) \,,
 \eeq
 \beq
	\frac{\partial \dot{b}_k}{\partial c_{j}} = 2 \frac{k}{\tildeL} (b_{k+j}-b_{k-j}) \,.
 \eeq

\section{\KSe\ according to Predrag}
\label{s-KS}
% Predrag 					 4jul2006
% extracted from ~dasbuch/book/chapter/PDEs.tex  5jun2005 version
% Predrag               1 jan 2000
% Predrag              17 sep 99

%  remember to incorporate missing refs from {chapter/refsPDEs}

The \KS\ system\rf{ku,siv},
arising in the description of the flame front flutter of  gas burning in
a cylindrically symmetric burner on your kitchen stove,
and many other problems of greater import,
is one of the simplest partial differential equations that
exhibit turbulence.
The time evolution of the ``height of the flame front'' 
is given by
\beq
u_t=(u^2)_x-u_{xx}-\nu u_{xxxx}
\,,\qquad	x \in [0,L]
\,.
\ee{ks}
In this equation $t \geq 0$ is the time and
$x$ is the spatial coordinate.
The subscripts $x$ and $t$ denote partial derivatives with respect to
$x$ and $t$;
$u_t = du/dt$, $u_{xxxx}$ stands for the 4th spatial
derivative of 
$u=u(x,t)$ at position $x$ and time $t$.
The ``viscosity'' parameter 
$\nu$ controls the 
suppression of solutions with fast spatial variations.
We take note, as in the Navier-Stokes equation \refeq{NavStokes}, of the
$u {\partial_x} u$ ``inertial'' term, the $ {\partial_x^2 } u$
``diffusive'' term (both with a ``wrong'' sign), etc.

The term $(u^2)_x$ makes this a {\em nonlinear system}.
It is one of the
simplest conceivable nonlinear PDE, playing
the role in the theory of spatially extended systems
analogous to the role that
the $x^2$ nonlinearity
% \refeq{LogisMap}
plays in the dynamics of iterated mappings.
The salient feature of such
partial differential equations is a theorem saying that
for any finite value of the phase-space contraction
parameter $\nu$,  the asymptotic dynamics is
describable by a {\em finite} set of ``inertial manifold''
ordinary differential equations. %cite{Foias88}.


\subsection{Scaling and symmetries}

The \KSe\ \refeq{ks} is space translationally invariant,
time translationally invariant, and invariant under
reflection
$x \to -x$, 
$u \to -u$. 

Comparing $u_t$ and $(u^2)_x$ terms we note that $u$ has
dimensions of $[x]/[t]$, so it would be more correct to
refer to it as the ``velocity'' rather than the 
``height'' of the flame front. Indeed, the  \KSe\ is
Galilean invariant: if $u(x,t)$ is a solution, then 
$v+u(x+2vt,t)$, with $v$ an arbitrary constant velocity, is also a solution. 
Without loss of generality, in our calculations we shall set 
% \index{Galilean invariance}
% \index{invariance!Galilean}
\beq
\int dx \, u = 0
\,.
\ee{GalInv}

In terms of the system size $L$, the only length scale available,
the dimensions of terms in \refeq{ks} are
$ %\[
[x]=L
$, $%\,,\quad
[t]=L^2
$, $%\,,\quad
[u]=L^{-1}
$, $%\,,\quad
[\nu]=L^2
\,.
$ %\]
What is the non-dimensional ``Rayleigh'' number for the
\KS\ system? 
 Scaling out the ``viscosity'' $\nu$ 
\[ 
x \to x \nu^{\frac{1}{2}}
\,,\quad
t \to t \nu
\,,\quad
u \to u \nu^{-\frac{1}{2}}
\,,
\]
brings the \KSe\ \refeq{ks}
to a non-dimensional form
\beq
u_t=(u^2)_x-u_{xx}- u_{xxxx}
\,,\qquad	
	x \in  [0,L\nu^{-\frac{1}{2}}] = [0,2\pi\tilde{L}]
\,.
\ee{ks-L}
In this way we trade in the ``viscosity'' $\nu$
and the system size $L$ for a single
dimensionless system size parameter
\beq
	\tilde{L}={L}/{(2 \pi \sqrt{\nu})}
%	\tilde{L}=\frac{L}{2 \pi \sqrt{\nu}}
%	\,,
\ee{tildeL}
which plays the role of a ``Reynolds number''
for the \KS\ system.

In the literature sometimes 
$L$ is used as the system parameter, with $\nu$ fixed to $1$, and
at other times $\nu$ is varied with $L$ fixed to either $1$ or $2\pi$.
To minimize confusion,
in what follows we shall state results of all 
calculations in units of dimensionless system size $\tilde{L}$.
\PC{motivate $2\pi$ factor by the mean wavelength,
    refer to the equation number}
Note that the time units also have to be
rescaled; if $T^*_p$ is a period
of a periodic solution of \refeq{ks} with a given
$\nu$ and $L=2\pi$, then
the corresponding solution of the non-dimensionalized \refeq{ks-L}
has period 
\beq
 T_p= T^*_p/\nu
\ee{Trescl}
\PC{MAKE SURE that
    all periods in tables of computed cycles are stated for that
    case, and not for $L=2\pi$.
   }


\subsection{Fourier space representation} \label{s:FourierModes}

% \index{Fourier!mode!truncation}
% \index{truncations!Fourier}

%from KSe.tex \section{{\KSe}}
%\label{sec:KSequil}
% Predrag                                       15dec2004
% Lan                                           25nov2004
%
\noindent
Spatial periodic boundary condition $u(x,t)=u(x+2\pi\tilde{L},t)$
makes it convenient to work in the Fourier space, 
\beq
  u(x,t)=\sum_{k=-\infty}^{+\infty} b_k (t) e^{ i k x /\tilde{L} }
\, .
% \label{fseries}
\ee{eq:ksexp}
with \refeq{ks-L} replaced by an infinite set of 
ODEs for the Fourier coefficients:
\beq
% \dot{b}_k= \left( \frac{k}{\tilde{L}}\right)^2.
%      \left( 1 - \left( \frac{k}{\tilde{L}}\right)^2  \right) b_k 
%	 + i \frac{k}{\tilde{L}} \sum_{m=-\infty}^{+\infty} b_m b_{k-m}
\dot{b}_k=(k/\tilde{L})^2\left( 1 - (k/\tilde{L})^2  \right)b_k 
 	 + i (k/\tilde{L}) \sum_{m=-\infty}^{+\infty} b_m b_{k-m}
\,.
\ee{expan}
%%% begin NW : added stuff to sentence and commented out old version.
%
This is the infinite set of ordinary differential equations promised
in this chapter's introduction. 
% \index{infinite-dimensional flows}
% \index{flow!infinite-dimensional}
% at the beginning of the section.
%
%%% end NW

Since $u(x,t)$ is real,
$ %\[
b_k=b_{-k}^*
\,,
$ %\] %\label{cplx-b}
so we can replace the sum over $k$ in \refeq{expan} by a
sum over $k \geq 0$.
As  $\dot{b_0}=0$, $b_0$ is a conserved quantity,
in our calculations
fixed to $b_0=0$ by
the Galilean invariance condition \refeq{GalInv}.
% \index{Galilean invariance}
% \index{invariance!Galilean}

The Fourier coefficients $b_k$ are in general complex numbers.
% of time $t$.  
We can
%simplify
isolate the antisymmetric subspace of the system \refeq{ks-L} by
considering the case of $b_k$ pure imaginary, $b_k= i a_k$, where
$a_k$ are real, with the evolution equations
\beq
% \dot{a}_k=(k^2- k^4)a_k - k \sum_{m=-\infty}^{\infty} a_m a_{k-m}
\dot{a}_k = (k/\tilde{L})^2\left( 1 - (k/\tilde{L})^2  \right)a_k 
 	 - (k/\tilde{L}) \sum_{m=-\infty}^{+\infty} a_m a_{k-m}
\,.
\ee{expan-symm}
\PC{create exercise here}
%\exerbox{}
This picks out the subspace of 
antisymmetric solutions $u(x,t)=-u(-x,t)$,
so $a_{-k}= - a_k$. By picking this subspace we eliminate
the continuous translational symmetry from our consideration;
that is probably not an option for an experimentalist,
but will do for our purposes.

In the antisymmetric subspace the translational 
invariance of the full system reduces
to the invariance under discrete
translation by half a spatial period $L$.
In the Fourier representation \refeq{expan}, 
this corresponds to invariance under 
\beq
a_{2m} \to a_{2m}\,, a_{2m+1} \to -a_{2m+1}
% \,, m \in \mathbb{Z}
\,.
\ee{FModInvSymm} 
The antisymmetric condition amounts to imposing
$u(0,t)=0$ boundary condition, with
the size of the system reduced to
the $[0,L/2]$ interval. In
comparing our numerical results with % other authors' 
calculations for
the full, unrestricted dynamics on $[0,L]$, we define
the non-dimensionalized system size as
$\tilde{L} = {L}/{(4 \pi \sqrt{\nu})}$,
corresponding to a system defined on the
$[0,L/2]$ domain. 


In order to find a better representation of the dynamics, we now
turn to its topological invariants.

\section{Equilibria} % of the \KSe}
\label{sec:stks}

% Predrag                                       05dec2004
% Lan                                           25nov2004
% from Lan thesis                                8jun2004

\noindent
Equilibria 
% (or the steady solutions)
are the simplest invariant sets in
the phase space. They,  and 
the connections between them form the
coarsest geometrical framework for organizing
phase space orbits. %\rf{ksgreene88}.
% In certain parameter ranges 
% equilibria they play prominent role in observed dynamics.
% For example, the $n$-cell state
% is stable in discrete parameter windows
% windows for arbitrary large system sizes. %\rf{FSTks86}.

The equilibrium condition $u_t=0$ for the {\KSe} PDE \refeq{ks-L} 
is the ODE
\[
(u^2)_x-u_{xx}- u_{xxxx}=0 
% \,.
\]
which can be analyzed as a dynamical system in its own right.
Integrating once we get
\beq
u^2-u_x- u_{xxx}=c 
\,,
\label{eq:stdks}
\eeq
where $c$ is an integration constant 
whose value strongly influences the nature of
the solutions. %of \refeq{eq:stdks}. 
Written as a 3-dimensional dynamical system
with spatial coordinate $x$ playing the role of ``time'',
%\refeq{eq:stdks} 
this is a volume preserving flow
\beq
u_x = v \,,\qquad
v_x = w \,,\qquad
w_x = u^2-v-c \,,
  \label{eq:3dks}
\eeq
with the ``time'' reversal symmetry, 
\[
x \to -x,\quad u \to -u, \quad v \to v, \quad w \to -w \,.
\]
 From \refeq{eq:3dks} we see that
\[
(u+w)_x=u^2-c \,.
\]
If $c<0$, $u+w$ increases without bound with $x \to \infty$,
and every solution escapes to infinity.
If $c=0$, the origin $(0,0,0)$ is the
only bounded solution. 

For $c>0$ there is much
$c$-dependent interesting dynamics, with
complicated fractal sets of bounded solutions.
The sets of the solutions of the equilibrium condition 
\refeq{eq:3dks} are themselves in turn organized by the  
equilibria of the equilibrium condition, and 
the connections between them.
    For $c>0$ the equilibrium points of \refeq{eq:3dks} are
$c_{+}=(\sqrt{c},0,0)$ and $c_{-}=(-\sqrt{c},0,0)$.
Linearization of the flow around
$c_{+}$ yields 
stability eigenvalues 
$[ 2\lambda \,, -\lambda \pm i \theta ]$
with 
\[
\lambda=\frac{1}{\sqrt{3}}\sinh \phi
\,,\qquad
\theta=\cosh \phi \, ,
\]
and $\phi$ fixed by $\sinh 3\phi=3\sqrt{3c}$. 
Hence $c_{+}$ has a {1-dimensional}
unstable manifold and a 2-dimensional
stable manifold along which solutions spiral 
in. 
By the $x \to -x$ ``time reversal'' symmetry, the 
invariant manifolds of $c_{-}$ 
have reversed stability properties.

However, we do not need to explore the fractal set of the 
\KS\ equilibria for infinite size system here;
for a fixed system size
$L$ with periodic boundary condition, the only surviving equilibria  are
those with periodicity $L$.
They satisfy 
the equilibrium condition for \refeq{expan}
\PC{\rf{ksgreene88} to remarks}
\beq
(k/\tilde{L})^2\left( 1 - (k/\tilde{L})^2  \right)b_k 
	 + i (k/\tilde{L}) \sum_{m=-\infty}^{+\infty} b_m b_{k-m} = 0
\,.
\label{eq:stfks}
\eeq 
% We have proved in \refsect{sec:kspr} that 
Periods of spatially periodic equilibria are multiples of $L$.
Every time $\tilde{L}$ crosses an integer value  $\tilde{L}=n$,
$n$-cell states
are generated through pitchfork bifurcations. 
In the full phase space they
form an invariant circle due to the translational invariance of \refeq{ks-L}. 
In the antisymmetric subspace considered here, they corresponds to two points,
half-period translates of each other of the form
\[
u(x,t)=-2\sum_k b_{kn}\sin (knx) \,,
\]
where $b_{kn} \in \mathbb{R}$.
% By rescaling $u,x$ and $\nu$, the $n$-cell states transform to each other.

%      With the increase of $L$ these periodic solutions 
% may bifurcate into more complicated ones.
For any fixed period $L$
%, however, 
the number 
of spatially periodic solutions is finite up to a spatial translation.
This observation can be heuristically motivated as follows. 
% \PC{this argument keeps worrying me: there are lots of solutions, like
% $u=0$, that are equilibria, but isolated -
% they are noplace near asymptotic dynamics.
% Do they belong to the invariant manifold?
%    }
% Equilibria are solutions valid for all times, and are thus points
% on the finite-dimensional compact inertial manifold\rf{infdymnon}.
Finite dimensionality of the inertial manifold
bounds the size of Fourier components of all solutions.
% This
% compact inertial manifold and the dynamics on it can be 
% described by analytic functions of a finite number of Fourier modes.
\PC{explain the theory; say that in practice it is useless}
On a finite-dimensional compact manifold,
an analytic function can only have a finite number
of zeros. So, the equilibria, {\em i.e.}
the zeros of a smooth velocity field on
the inertial manifold, are finitely many.
% The number of equilibria increases exponentially with $L$,
% \PC{give reference for ``exponential growth''}
% for infinite system size $L \to \infty$,
% there are infinitely many equilibria. 
% \PC{is there a reference where this is explained?}

For a sufficiently small $L$ 
the number of equilibria is small,
mostly
concentrated on the low wave number end of the Fourier spectrum.
These solutions may
be obtained by solving the truncated versions of \refeq{eq:stfks}. 
% Understanding the structure of these solutions, requires a study
% of the full phase space of the 3-dimensional dynamical system \refeq{eq:3dks},
% not attempted here.

% Locally coherent structures are observed for arbitrary system size,
% see \reffig{f:flameFlut}~(b). %\reffig{f:ksev}.


\underline{Important \KS\ equilibria:}{ 
% \label{exam:KurSivEquil}
% \index{Kuramoto-Sivashinsky equilibria}
% \index{equilibria!Kuramoto-Sivashinsky}
%

	} %end \example{Important \KS\ equilibria



\PC{say somewhere: ``
The task of the theory is to describe this spatio-temporal
turbulence and yield quantitative predictions for its measurable
consequences.
   ''}



\section{Why does a flame front flutter?}
%\label{s:StabEqui}
%%%%%%%%%%%%%%%%%%%%%%%%%%%%%%%%%%%%%%%%%%%%%%%%%%%%%%%%%%%%
% Predrag           5jun2005
% extracted from \Chapter{stability}{ 2apr2005}
%       Predrag                  14/3-95
% taken from ks.tex
% Spatiotemporal chaos in terms of recurrent patterns
% PC            30/6/96 

% \index{local!stability}
% \index{linear!stability}
% \index{stability!linear}
% \index{equilibrium!point}

We start by considering the case where
$a_\stagn$ is an equilibrium point \refeq{EquilPoint}. 
Expanding around the equilibrium point $a_\stagn$,
and using the fact that the matrix
${\bf \Mvar}={\bf \Mvar}(a_\stagn)$
% , its matrix of its stability exponents
in \refeq{die}
% local expansion rate
is constant, 
we can apply the simple formula \refeq{eqPointStab} also to the 
{\jacobianM}
of an equilibrium point of a PDE, 
\[
 \jMps^t(a_\stagn) = e^{{\bf \Mvar} t}
    \,\qquad
 {\bf \Mvar}={\bf \Mvar}(a_\stagn)
\,.
\]

The \KS\ flat flame front $u(x,t)=0$ is an 
equilibrium point of \refeq{ks}. The matrix of variations
\refeq{DerMatrix}
follows from \refeq{expan}
% \index{Kuramoto-Sivashinsky system}
\beq
{\Mvar}_{kj}(a) ={\pde v_k(a)\over \pde a_j  }
=((k/\tilde{L})^2- (k/\tilde{L})^4)\delta_{kj} - 2(k/\tilde{L}) a_{k-j}
\,.
\ee{expanMvar}
For the $u(x,t)=0$ equilibrium  solution the matrix of variations
is diagonal, and -- as in \refeq{EqDyn164} -- so is the {\jacobianM}
$
\jMps^t_{kj}(0) = \delta_{kj} e^{((k/\tilde{L})^2- (k/\tilde{L})^4)t}
\,.
$

For $\tilde{L} < 1$,  $u(x,t)=0$ is the globally attractive stable 
equilibrium.
As the system size $\tilde{L}$  is increased,
the ``flame front'' becomes increasingly unstable and turbulent,
the dynamics goes through a rich sequence of
bifurcations on which we shall not dwell here.
% studied e.g. in \refref{KNS90}. 
% , one quickly finds a
% myriad of unstable periodic solutions whose number
% grows exponentially with period.

The $|k|<??$ 
long wavelength perturbations of the flat-front equilibrium
are linearly unstable, while all 
$|k|> ??$ short wavelength perturbations are strongly contractive.  
The high $k$ eigenvalues, corresponding to rapid variations of
the flame front, decay so fast that the corresponding eigendirections
are physically irrelevant.
% \index{Lyapunov exponent!equilibrium}
To illustrate the rapid contraction in the non-leading eigendirections
we plot  in [MAYBE INCLUDE] % \reffig{f:eigenvalues}
the eigenvalues of the equilibrium in the unstable regime,
for relatively small system size, % low viscosity $\nu$,
and compare them with the
stability eigenvalues of the least unstable cycle for the same 
system size.
% value of $\nu$. 
The equilibrium solution is very unstable,
in 5 eigendirections,
the least unstable cycle only in one. 
Note that for $k>7$ the rate of contraction
is so strong that higher eigendirections are numerically meaningless for 
either solution; even though the flow is infinite-dimensional, the attracting
set must be rather thin.

While in general
for $\tilde{L}$ sufficiently large
one expects many 
coexisting attractors in the phase space%
%Hyman and Nicolaenko
\rf{HNZks86} ,
in numerical studies most random initial
conditions settle converge to the same chaotic attractor. 

From \refeq{expan} we see that the origin $u(x,t) = 0$
has Fourier modes as the  linear
stability eigenvectors. 
When $|k| \in (0,\tilde{L})$, the corresponding Fourier modes are
unstable.
The most unstable modes has $|k|=\tilde{L}/\sqrt{2}$ and defines the scale of basic building
blocks of the spatiotemporal dynamics of the {\KSe} in large system size limit,
as shown in \refsect{sec:KSnumer}. 


\noindent
Consider now the case of initial $a_k$ sufficiently small
that the bilinear $ a_m a_{k-m}$ terms in \refeq{expan} can
be neglected. Then we have a set of decoupled linear
equations for $a_k$ whose solutions are exponentials, at most
a finite number for  which
$k^2 > \nu k^4$
is growing with time, and infinitely many with
$
\nu k^4 > k^2
$
decaying in time.
The growth of the unstable long wavelengths (low $|k|$) excites
the short wavelengths
through the nonlinear term in \refeq{expan}.  The excitations thus
transferred are dissipated by the strongly damped short wavelengths,
and a ``chaotic equilibrium'' can emerge. The very short
wavelengths $|k| \gg 1 / \sqrt{\nu}$ remain small for all times,
but the intermediate wavelengths of order $|k| \sim 1 / \sqrt{\nu}$
play an important role in maintaining the dynamical equilibrium.
As the damping parameter decreases, the solutions increasingly take on
% Burgers type
shock front
character poorly represented by the Fourier basis, and many
higher harmonics may need to be kept
% \rf{KNS90,GEP}
in truncations of
\refeq{expan}.


Hence, while one may truncate the high modes in the expansion
\refeq{expan}, care has to be exercised to ensure that no modes
essential to the dynamics are chopped away. 

In other words, even though our starting point
\refeq{ks}
is an infinite-dimensional dynamical system, the asymptotic dynamics
unfolds on a finite-dimensional attracting manifold, and so we are back on
the familiar territory of \refsect{SecDynFlows}:
the theory of a finite number of ODEs applies to this
infinite-dimensional PDE as well.

    {\bf When is an equilibrium important?} There are two kinds of roles
equlibria play:

{\em ``Hole'' in the natural measure}.
The more unstable eigendirections it has (for example, the
$u=0$ solution), the more unlikely it is  that
an orbit will recur in its neighborhood.

{\em unstable manifold of a ``least unstable''equilibrium}.
 Asymptotic dynamics
spends  a large fraction of time in
neighborhoods of a few  equilibria with
only a few unstable eigendirections.


\underline{Stability of \KS\ equilibria:}{ 
% \label{exam:KSEquilStab}
% \index{Kuramoto-Sivashinsky equilibria}
\begin{table}
\caption[]{
Important \KS\ equilibria:
% in the antisymmetric solution 
% space of the Kuramoto-Sivashinsky equation with periodic boundary % % % % condition,
% $ \nu =1$, $L=38.5$;
% their labels,
the first few stability exponents
%, with complex pairs written together.
}

\vskip 1.5cm

{\small
%\lineup
\begin{tabular}{@{}ccccc}
\hline %\br
$~S~~~$ & $~~~~\lambda_1 \pm \,i\,\theta_1$ 
                                & $~~~~\lambda_2 \pm \,i\,\theta_2$ 
                                        & $~~~~\lambda_3 \pm \,i\,\theta_3$ 
\\ 
\hline %\mr
${C_1}$    &{0.04422 $\pm \,i\,$0.26160}   &-0.255 $\pm \,i\,$0.431 
&-0.347 $\pm \,i\,$0.463         \\
% ${C_2}$    &0.33053  & 0.097 $\pm \,i\,$0.243 
% &-0.101 $\pm \,i\,$0.233        \\
\hline %\mr
${R_1}$   &{0.01135 $\pm \,i\,$0.79651} & -0.215 $\pm \,i\,$0.549 
&-0.358 $\pm \,i\,$0.262        \\
%  ${R_2}$   &  0.33223  & -0.001 $\pm \,i\,$0.703  
%  & -0.281 $\pm \,i\,$0.399      \\
\hline %\mr
${T}$     & 0.25480  & -0.07 $\pm \,i\,$0.645 &-0.264  
\\
\hline %\br
\end{tabular}
}
\label{t:stationary}
\end{table}

{\em 
spiraling out in a plane}, all other directions contracting


{\bf
Stability of ``center'' equilibrium
	}

linearized stability exponents: 
\[ % \beq
(\lambda_{1}\pm\,i\,\theta_{1},\lambda_{2} \pm\,i\,\theta_{2}, \cdots)
	= (0.044 \pm \,i\,0.262\,,\,
		-0.255 \pm \,i\,0.431\,,\,\cdots)
\] %\eeq

The plane spanned by $\lambda_{1} \pm\,i\,\theta_{1}$ eigenvectors rotates with angular period
$\period{} ~\approx~2\pi/\theta_{1}=24.02$.
% 2*4*a(1)/0.26160 = 24.0182924586375

a trajectory 
that starts near  the $C_1$~equilibrium point spirals 
away per one rotation
with multiplier
$\ExpaEig_{\mbox{radial}}~\approx~\exp(\lambda_{1}\period{})=2.9$.
% 2*4*a(1)/0.26160*0.04422 = 1.062
% e(1.0620888) = 2.8924063421

each Poincar\' e section return, 
contracted into the stable manifold by 
factor of
{
$\ExpaEig_{2}\approx\exp(\lambda_{2}\period{})=0.002$
}
%2*4*a(1)/0.26160*(-0.255) = -6.12466
%e(-6.12) = .0022


The local Poincar\' e return map is 
{\em
in practice $1-dimensional$
}
	} %end \example{{Stability \KS\ equilibria


% Staring at the solution
% as it evolves in time we should start getting a glimpse of the
% repertoire of the spatiotemporal patterns charcterizing
% the turbulent dynamics.


\underline{Model PDE systems.}{
The theorem on finite dimensionality of inertial manifolds
of phase-space contracting PDE flows is proven in \refref{Foias88}.
% \index{inertial manifold}
% \index{Kuramoto, Y.}
% \index{Sivashinsky, G.I.}
% \index{Kuramoto-Sivashinsky system}
The Kuramoto-Sivashinsky equation was introduced in \refrefs{ku,siv}.
Holmes, Lumley
and Berkooz\rf{Holmes96} offer a delightful discussion of why this system
deserves study as a staging ground for studying turbulence in 
full-fledged Navier-Stokes equation. 
How good 
a description of a flame front this equation
is need not concern us here; suffice it to say that such model
amplitude equations for interfacial instabilities arise in a variety
of contexts - see e.g.~\refref{saddks} - and this one is perhaps the
simplest physically interesting spatially extended nonlinear system.
\PC{Comment om MAWs, BECS and CGLe}
\PC{refer to Trefethen's program for fast integration}
This chapter
% \refsect{s_extend} 
is based on V. Putkaradze's term project
(see \wwwcb{/extras}).
%a gratifying example of a successful course project which led to a
%full-fledged research paper by
and on the Christiansen {\em et al.} article \rf{Christiansen:97}. 
\PC{refer the reader to \refref{Lan:Thesis} for a
 review of the $L \to \infty$ equilibria.
 Shouldn't Michelson\rf{Mks86} be given credit here?}

% \label{r:KuramShiva}
}

\underline{\KS\ system, truncations.}{
We describe here our criterion for reliable
truncations of the infinite ladder of 
ordinary differential equations (\ref{expan}).

Adding an extra dimension to a truncation of the system (\ref{expan})
introduces a small
perturbation, and this can (and often will) 
throw the system into a totally different asymptotic state. 
A  chaotic attractor for $N=15$ can become a period three 
window for $N=16$, and so on. 
If we compute, for example, the Lyapunov exponent
$\Lyap(\nu,N)$ for the strange attractor of the 
system (\ref{expan}), there is no reason to 
expect $\Lyap(\nu,N)$ to smoothly converge to the limit  
value $\Lyap(\nu,\infty)$ as $N \rightarrow \infty$. 
The situation is different in the periodic windows, 
where the system is structurally stable, and it makes sense to compute 
 Lyapunov exponents, escape rates, etc. for the 
{\em repeller}, \ie, the closure of the set of all 
{\em unstable} periodic orbits. 
Here the power of cycle expansions comes in: 
to compute quantities on the repeller by direct averaging methods is 
generally more difficult, because the motion quickly collapses to the 
stable cycle. 
	} %end \remark{Kuramoto-Sivashinsky system, numerical results.}


The problem one faces with high-dimensional flows is 
that their topology is hard to
visualize, and that even with a decent starting guess for a point on
a periodic orbit, methods like the Newton-Raphson method are likely to fail.
Methods that start with initial guesses for a number of points along the
cycle, such as the multipoint shooting method of \refsect{s-MultShoot},
are more robust.
The relaxation  (or variational) methods take this strategy to its
logical extreme,
and start by a guess of not a few points along a periodic orbit,
but a guess of the entire orbit.

At present the theory is in practice applicable only to systems
with a low intrinsic {\em dimension}
-- the minimum number of coordinates necessary to
capture its essential dynamics.
% \index{dimension!intrisic}
% \index{degree of freedom}
If the system is very turbulent
(a description of its long time dynamics requires a space of high
intrinsic dimension) we are out of luck. 
% \index{turbulence}

% \input{chapter/refsPDEs}

%%% end

