% KS-L22.tex
%
% Predrag created file				jul  9 2006


\section{Small $L=22$ system {\rpo s}}

% Davidchack and Crofts
The full space \KS system $L = 22.0$ 
appears to be the smallest $L$ with persistent chaos.  
$L=22$ is a sensible choice because in units of mean wavelength
the size of this small system is about $ \tilde{L}/\sqrt{2}= 2.4758$
2.5 wavelengths, 
so the dynamics is competition between wavenumbers
2 and 3.
Because of the strong contraction in KS we expect at most 10 eigenvalues to be
significant, rest are in the numerical noise. See figure~6 in
\refref{Christiansen:97}.

% Davidchack and Crofts 
We investigate this system in 16 to 64 complex Fourier modes
(32 to 128-dimensional system of real ODEs) truncation, and recheck the results 
by redoing the calculation with the double number of Fourier modes.
% observe how many digits change. 
The \eqv\ points are accurate to at least to $10^{-11}$. Since
Lapack is also double precision accurate, the accuracy of the first
few eigenvalues is similar, and certainly in
excess of 6 significant digits.
%
All digits stated in tables are significant.

The accuracy that can be reached is of order of
$|a(\period{p},d_p) - a_0| \approx \epsilon \exp(\Lyap_p \period{p})$,
where $\epsilon \approx 10^{-17}$ for double precision,
$\lambda_p$ is the largest Lyapunov exponent, 
and $\period{p}$ the period.  With a good starting guess,
Newton's method
typically reaches that accuracy after 2-3 iterates.

Many of the \rpo s can be constructed from segments corresponding to
close approches to some of these equlibria.

\subsection{\Eqva}

Numerically we find \eqva\ of wavenumber $k$ by (when it exists)
by using $sin( k x /\tilde{L})$ as the initial guess for the Newton routine.
The \eqva\ {\nameit}2 and {\nameit}3 found here
essentially lie in the 2nd and 3rd Fourier component complex planes,
respectively, with very
small deformations from higher harmonics.
\RLD{recheck their $a_1$ content}

For $L = 22.0$ there are at least
3 unstable \eqva\ and a pair of unstable \reqva:

\PC{probably make up a table with all these}
Wavenumber 0 \eqv\ {\nameit}0, $u=0$, with multipliers 
(eigenvalues of $\Mvar$):

\PC{these we have analytically, all real.
Enter values here, refer to the equation}

\PC{why is there no wavenumber 1 \eqv\ {\nameit}1?}

Wavenumber 2 \eqv\ {\nameit}2, with multipliers 
\ES{I observe pairs of real eigenvalues,
e.g. -58.3602685 and -58.3602681. As their absolute
value increases they differ even less.
I think it has to do with the linear part being the main
contribution to $\Mvar$ for higher modes, as well as 
with treating real and imaginary components 
as separate variables, which means it will appear twice.
	}
\PC{I think such contracting eigenvalues as -58.3602685 have no meaning.
Even if they are accurate eigenvalues of $\Mvar$,
what use is
$\ExpaEig_{radial} =  e^{\Lyap_p \period{p}} = e^{26\cdot58} = e^{1510}$.
	}

$(\Lyap_i \pm \theta_i)
=(
  0.13903973 \pm i 0.23842023,
  0,
 -0.08402656 \pm i 0.16019413,
 -0.11941393, 
 -0.27112264 \pm i 0.35630716,
 -2.01303043,
 -2.03775342,
 -5.63649418,
\cdots
)$
\PC{need to decide: eigs either appear as complex pairs, or
real pairs, because of complex Fourier coefficients. Should we list
real pairs? Probably best to list them only once.}

These \eqva\ are selfdual under $u(x) \to -u(-x)$ symmetry.

\PC{figures might be in dasbuch/talks/figs/, repository dasbuch}
% Ruslan L Davidchack, 	10 Jul 2006 
Figure steady\_states2.jpg

% Ruslan L Davidchack, 	10 Jul 2006 
2-wave steady state:

$(\Lyap_i \pm \theta_i)=
(
  0.13903972165535 \pm 0.23842023092014i, \\
  0.00000000345431                    , \\
 -0.08402654994820 \pm 0.16019413653788i, \\
 -0.11941394469356                    , \\
 -0.27112264064485 \pm 0.35630715566102i
)$

% Ruslan L Davidchack, 	10 Jul 2006 
Wavenumber 3 \eqv\ {\nameit}3, with multipliers 

$(\Lyap_i \pm \theta_i)=
(
  0.09334635581866                    , \\
  0.09334403319083                    , \\
 -0.00000000000000                    , \\
 -0.41277453725606                    , \\
 -0.61075248808595 \pm 0.37587414026088i
)$


\underline{Need to trace out}
the unstable manifold plane, like Gibson did for plane Couette.
Compute the 2 expanding eigenvectors of the
\eqv\ {\nameit}2, as well as the 3rd, least contracting direction; then
translate and rotate your Fourier modes into this coordinate frame,
and plot the trajectory there, both in the lab and the mean velocity frame.

Ruslan: % 10 Jul 2006
% 
% 119 KB     "long_orbit.jpg"
%  88 KB     "steady_states1.jpg"
%  84 KB     "steady_states2.jpg"
% 197 KB     "rpos1.jpg"
% ----------------------------------------

For all spatial plots color axis $u \in [-3, 3]$ is the same,
same time units and spatial width $L$.
For the steady states the magnitude of the 2-wave is quite 
a bit smaller than that of the 3-wave.

On the 
	$[a_?,a_?]$ plane
	the $\sigma x = -x$ symmetry of \KSe\ is explicit.

The two equilibria
capture qualitatively the mean velocity frame \rpo\ {\nameit}55 shape,
which follows the
equilibrium for most of the time, except for a quick swing where it
sidesteps by $d/4$, just as it does in \reffig{f:rpo55}. 


steady\_states1.jpg shows the numerical evolution and, since the
traveling wave is very unstable, it disappears after awhile. 
The numerically exact solution is plotted in steady\_states2.jpg

rpos1.jpg is attached as a sample. 

\subsection{\Reqva}

Numerical solution of the \reqv\  condition,
transported by travelling wave velocity $c$, 
fixed by the solution for the travelling wave equation:
\[
f_k(u) - i{2\pi\over L} c k u_k = 0
\]

\underline{1-\reqv\  (travelling wave).}
% Ruslan L Davidchack, 	10 Jul 2006 
There is a pair of \reqva\ 
${\nameit}1L$,
${\nameit}1R$
(traveling waves), dual under the
$u(x) \to -u(-x)$ symmetry. They are 
determined numerically by 
adiabatic continuation from a smaller system size
$L~\approx 12$,
where they are stable, to $L=22$
where their velocity is atypically large, $c=7.???$,

Their multipliers are:
$(\Lyap_i \pm \theta_i)=
(
  0.11562229058451 \pm 0.81728916811210i,	\\
  0.03366329210957 \pm 0.41890950137102i,	\\
 -0.00000012005108                    ,	\\
 -0.24572938886392                    ,	\\
 -0.32132109741933 \pm 0.98126192986927i,
\cdots
)$

The pair of \reqva\ 
${\nameit}2L$,
${\nameit}2R$
exists for larger system sizes, but does not continue 
adiabatically\rf{saddks} down to $L=22$.

The \reqva (or travelling waves) appear to have a limiting propagation
velocity $c_{max} = \pm d/\period{}$. 
To visualize them numerically,
start with a localized self-dual $u(x,0)$ such as
\[
u(x,0) = x e^{- x^2/2\sigma^2}
\,,
\]
with typical width $\sigma/2$ of order of typical wavelength 
$\sqrt{2}$ (in $\tilde{L}$ system size units).
Time evolution of this  $u(x,t)$ is bracketed by two constant 
pulses of apparently constant velocity $v=?$.
\RLD{generate figure, state $\sigma/2$, estimate $v$}
The notion of ``velocity''
is fuzzed up by the fact that the large peaks are preceeded
by smaller precursors.

\PC{Determine their velocity ANALYTICALLY?}
\PC{comment on exact soliton solutions, Lan's thesis.}

%%%%%%%%%%%%%%%%%%%%%%%%%%%%%%%%%%%%%%%%%%%%%%%%%%%%%%%%%%%%%%%%
\begin{figure}[t] %[h]
\centering
(a) \includegraphics[width=6.0cm]{figs/ks22.0-2w-neighborhood.eps}
\hspace{0.1in}
(b) \includegraphics[width=4.0cm]{figs/ks22.0-2w-R.eps}
\\
(c) \includegraphics[width=4.0cm]{figs/ks22.0-2w-G.eps}
\hspace{0.1in}
(d) \includegraphics[width=4.0cm]{figs/ks22.0-2w-B.eps}
\caption{
 Trajectories with initial conditions on the unstable subspace of
 the two wavelength equilibria.  
 (a) The coordinates $\tilde{u}_1$ and $\tilde{u}_2$ are along the directions defining the unstable subspace
 and $\tilde{u}_3$  is along the real part of the eigenvector, 
 corresponding to the eigenvaluw $-0.271122+ i\, 0.356307$
% Green curve belongs to \reffig{f:rpo55}(b) % rpo22-55-4-cm.eps
% rather than to  \reffig{f:rpo55}(a), % rpoEq22-55-4.eps?
 (b),(c),(d) u(x,t) representation of the orbit in red,green,blue respectivelly.  We observe
the different "direction" of transition to the same equilibrium point that is obvious in both
representations.}
\label{f:neighborhood2w}
\end{figure}
%%%%%%%%%%%%%%%%%%%%%%%%%%%%%%%%%%%%%%%%%%%%%%%%%%%%%%%%%%%%%%%%%%

\subsection{\Rpo s}

\ES{
The names of the \rpo\ figure files follow the convention
 {\tt rpoL-T-d.eps}s, with suffixes {\tt cm}
and {\tt u} indicating
 mean velocity frame  and $u$ representation respectively.
   }
%
Out of 30 \rpo s they
find,  only three are truly periodic.  The orbit
with $\period{p} = 95.25$ has a very small
$d = -6.5\,e^{-7}$, but it is not periodic 
(they
checked this by decreasing the integration step size and increasing the
number of modes).

The dynamics in this small system is competition between wavenumbers
2 and 3. The 2-\eqv\  and the 3-\eqv\  essentially lie in
the 2nd and 3rd Fourier component complex plane, with very
small deformations from higher harmonics.
Hence plot all \rpo s in these 2 representations:

$[ \Re a_2, \Im a_2, \Re a_3 ]$
(here 2-\eqv\  is a circle, 3-\eqv\ a vertical line)
 and
$[ \Re a_3, \Im a_3, \Re a_2 ]$
(here 3-\eqv\ is a circle, 2-\eqv\ a vertical line)

This stuff is hard to visualize... for ordinary periodic orbits one
plots the unstable plane of the \eqv, shows where the periodic
orbits sit. Other options:

Somewhat better visualization is in the
{\em mean velocity frame}, {\ie} 
a reference frame that rotates with with velocity 
$v_p=d_p/\period{p}$
In the mean velocity frame a \rpo\ becomes
a \po.
Mean velocity frame visualization helps quite a bit.
Put a black (green, respectively) dot
twice thicknes of the line every time unit; it will enable you to see
where the motion is slow and where it is fast.
% (a trick we used to understand plane Couette trajectories).
Mark the inital point on both
mean velocity \rpo\ and on \eqv\  in mean velocity
 frame with a fat triangle
indicating the direction, so we can see how they both move. Probably at the
opposite ends of the two curves - mean velocity frame is the mean motion.

%   rpo/figs/detail1rpo22-55-4.eps
%   rpo/figs/detail2rpo22-55-4.eps
%   rpo/figs/detail3rpo22-55-4.eps
%   break rpo22-55-4 into 3 parts.
%   The script for the fonts somehow crops these images

Each {\rpo} has its own mean velocity frame - and within it, {\eqv}
move on circles (or worse - because in higher Fourier modes they do mmore
complicted things), and it is important to know where the equlibrium is at
a given instant.

As the shift $d$ is defined mod~$L$, better to
state for each {\rpo} its mean velocity $c_p = d_p/\period{p}$,
where $d_p$ is measured on the line (not on the circle). $c_p$ is
preferrable to angle $2\pi d_p/L$ as it does not vary in $L \to$~large 
limit (just like $\sqrt{2}$ wavelength estimate is independent of
system size.

The \rpo\ {\nameit}55 travels between the 2-\eqv\  and a
travelling wave B (?) 
with period and shift
$\period{p}=55.5953\,,\ d=5.24725$
Compared to $L/4 = 5.5$
this is nice, but why not close to periodic after 2nd return? Why 4th return?

\Rpo\ {\nameit}55 looks similar to Davidchack's  orbit
of period 
$\period{p}=47.64$ and $d=5.6759$. The period appears to depend on how
many times the orbit manages to spiral around the \eqv.
For {\nameit}55 that appears to be
1.5 times per period, rather than 2. This would led as
to
think there is a family of \rpo s along with a 3rd unit eigenvalue of
$gJ$,
but such does not exist.
So there has to be a selection mechanism corresponding to
reaching or missing the neighborhood of an \eqv\  point starting from
the neighborhood of the other. 

The $u$ space time evolution \reffig{f:rpo55u} % rpo22-55-4-u.eps 
is plotted with the same starting instant,
so one can also track also the spatial profile $u$ in parallel with
the Fourier space projections.

So it is almost impossible to see \reffig{f:rpo55}(b) %rpo22-55-4-cm.eps
in \reffig{f:rpo55}(a) % rpo22-55-4.eps.
I can see 4 periods in \reffig{f:rpo55}(a), %po22-55-4.eps,
but not in \reffig{f:rpo55}(b) %rpo22-55-4-cm.eps
where it comes back only after full period $\period{p}=55.6$.

It still seems that it could be made relative periodic 
(modulo a reflection symmetry?)
in $\period{p}/4=55.6/4=13.9$? That would be OK 
-
by symmetry the figure 8 connecting
2 symmetric equilibria could consist of 4 identical segments: from
equilibrium A to midplane, then reflected version of the same to SA, and
back again.

\Eqv\ are solutions of 3-$d$ set of ODEs  \refeq{eq:3dks}, so
another convenient way to plot \eqva\ and \reqva\ on a periodic
domain $L$ is to plot 
$\partial u(x)$ vs. $u(x)$ as a curve parametrized by
$x\in [0,L]$. In this representation both \eqva\ and \reqva\ curves are
stationary, but the points on \reqva\ move as functions of time.

\Po s and \Rpo s can be plotted this way as well
$\partial u(x,t)$ vs. $u(x,t)$. Now they are are represented by time-dependent
``tube".



%%%%%%%%%%%%%%%%%%%%%%%%%%%%%%%%%%%%%%%%%%%%%%%%%%%%%%%%%%%%%%%%
\begin{figure}[t] %[h]
\centering
 	\includegraphics[width=2.5cm]{figs/rpo22-55-4-u.eps}
\hspace{0.1in}
\caption{
 The \rpo\ {\nameit}55 in $u(x,t)$ representation. 
        }
\label{f:rpo55u}
\end{figure}
%%%%%%%%%%%%%%%%%%%%%%%%%%%%%%%%%%%%%%%%%%%%%%%%%%%%%%%%%%%%%%%%%%


%%%%%%%%%%%%%%%%%%%%%%%%%%%%%%%%%%%%%%%%%%%%%%%%%%%%%%%%%%%%%%%%
\begin{figure}[t] %[h]
\centering
(a) \includegraphics[width=8.0cm]{figs/rpo22-55-4-clean.eps}
% ./removecache.sh rpo22-55-4.eps
% abandoned rpoEq22-55-4.eps with mean velocity equilibrium embeded.
%
\hspace{0.1in}
(b) \includegraphics[width=6.0cm]{figs/rpoEq22-55-4-cm.eps}
\\
(c) [create rpoEq22-55-4-cm-?.eps]
\caption{
 The \rpo\ {\nameit}55 in: 
 (a) Phase space, traced for four periods $\period{p}$.
% Green curve belongs to \reffig{f:rpo55}(b) % rpo22-55-4-cm.eps
% rather than to  \reffig{f:rpo55}(a), % rpoEq22-55-4.eps?
 (b) mean velocity frame. 
        The continuos family of 
	equilibria A obtained by the action of $g$ is shown in green,
	the SA family shown in red. The \rpo\ {\nameit}55 stays close
	to either A or SA for close to 1/2 of equilibrum rotation
	period, then quickly jumps to the other equilibrium point.
 (c) mean velocity frame A, SA and {\nameit}55 projected on the 
	$[a_?,a_?]$ plane,
	with the $\sigma x = -x$ symmetry of \KSe\ explicit.
        }
\label{f:rpo55}
\end{figure}
%%%%%%%%%%%%%%%%%%%%%%%%%%%%%%%%%%%%%%%%%%%%%%%%%%%%%%%%%%%%%%%%%%


The two equilibria
capture qualitatively the mean velocity frame \rpo\ {\nameit}55 shape,
which follows the
equilibrium for most of the time, except for a quick swing where it
sidesteps by $d/4$, just as it does in \reffig{f:rpo55}. 

Please also plot it in plane, chose small Fourier coefficients
 which respect the $x \to -x$ symmetry of \KSe.
Then the symmetry of 2 mean velocity
equilibria and self-dual symmetry of \rpo\ {\nameit}55 will be explicit.

Eigenvalues of \rpo\ {\nameit}55 $g\jMps$: are
\\
$(-57.17,  1.00009, 1.00001, -0.500, -0.012, \cdots)$ .
%
%  Eigenvalues of the Jacobian without rotation
%  84.15, -33.86 + 28.94 i, c.c. , 0.48, 0.00019
% no good - missing marginal ones

There are two
marginal eigenvalues, one for time translation, one for
rotational invariance. 
The sign of $\ExpaEig_{1}=-57$ says this is a Moebius-kind orbit,
inverse hyperbolic.
Lyapunov $\Lyap=0.07$ says that this neighborhood is much less repelling than
the central equilibrium A, a better candidate for being embedded into the
ergodic attractor.

The \rpo\ initial condition is
so accurate the orbit in \reffig{f:rpo55}(b)
start visibly deviating after retracing the loop 6.5 times.
% the largest unstable multiplier is 
% $-57.17$ per period of the orbit - error would grow to $\approx 60^7
% = 2,800,000,000,000$.

For the \rpo s the accuracy of Jacobian depends
on the time step size, and long runs are needed to refine the results

For a numerical check of the \rpo\ stability eigenvalues,
used two inital
points along an unstable eigenvector $\jEigvec{1}$
at radial distance  $\approx 10^{-4}$ from the \eqv\ {\nameit}2,
and the initial inter-point separation $\Delta(0) \approx 10^{-5}$.
Integrated for time equal to the period $\period{p}=26.3556$ as calculated from
the \jacobianM\ and computed the leading Lyapunov exponent from the ratio of
final to initial distance 
$\Lyap= {1 \over \period{}}\ln( \Delta(\period{})/\Delta(0))$.
Get
$\Delta(\period{})/\Delta(0) =39.01$,
$\Lyap=0.13902$, in agreement with the \eqv\ {\nameit}2 
expanding eigenvalue $\Lyap=0.13904$
\[
\ExpaEig_{radial} =  e^{\Lyap \period{}} =38.99
\,.
\]

% Ruslan L Davidchack, 	10 Jul 2006 
The orbit RLD found has period 60 
rather than 55.  Because it comes so close to the steady states, 
this is probably a numerical precision error.
\\
plots:
  76 rpo60fm23.jpg	\\
 909 rpo60fm23.emf	\\
 180 rpos1.jpg	\\
1594 rpos1.emf	\\


% Ruslan L Davidchack, 	10 Jul 2006 
(the 55 rpo, or whichever seems easiest to explain)
$\period{} = 59.89$,
$c_p = d_p/\period{p}= ?$

$(\ExpaEig_i e^{\pm i\theta_i})=
(
 -27.03397007874626,
   9.34426620337976,
   1.00000000001351,
   0.99999999999986,
  -0.05018967056231,
   0.00015065158255,
)$

The eigenvectors
indicate that an amplitude mode comes paired with the 
group shift-invariant mode $\ExpaEig_4 =1$. It probably says that
the amplitude $|u_k|$ of the associated can be easily perturbed (think of
a large system: $|u(x)|$ can be easily deformed by long wavelength
perturbations. This \underline{must} be understood. Proposal:

Rewrite Fourier modes as $u_k(t) = e^{r_k(t) + k(\theta_k(t))}$, study
dynamics and Jacobinas in the $\dot{r_k},\dot{\theta_k})$ representation.
$2$- and $3$-equilibria are nearly cirlces in this representation - higher
modes will not wind wildly if represented by $\theta_k(t)$? Kind of WKB
representation.

period 77 rpo jumps between the two steady states.
\\
plots:	\\
  84 rpo77fm23.jpg	\\
1133 rpo77fm23.emf	\\
 176 rpos2.jpg	\\
1594 rpos2.emf	\\

