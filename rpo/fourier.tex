\section{Fourier space representation} \label{s:FourierModes}
% 
% \index{Fourier!mode!truncation}
% \index{truncations!Fourier}

%from KSe.tex \section{{\KSe}}
%\label{sec:KSequil}
% Predrag                                       15dec2004
% Lan                                           25nov2004
%
\noindent
Spatial periodic boundary condition $u(x,t)=u(x+2\pi\tilde{L},t)$
makes it convenient to work in the Fourier space, 
\beq
  u(x,t)=\sum_{k=-\infty}^{+\infty} b_k (t) e^{ i k x /\tilde{L} }
\, .
% \label{fseries}
\ee{eq:ksexp}
with \refeq{ks-L} replaced by an infinite set of 
ODEs for the Fourier coefficients:
\beq
% \dot{b}_k= \left( \frac{k}{\tilde{L}}\right)^2.
%      \left( 1 - \left( \frac{k}{\tilde{L}}\right)^2  \right) b_k 
%	 + i \frac{k}{\tilde{L}} \sum_{m=-\infty}^{+\infty} b_m b_{k-m}
\dot{b}_k=(k/\tilde{L})^2\left( 1 - (k/\tilde{L})^2  \right)b_k 
 	 + i (k/\tilde{L}) \sum_{m=-\infty}^{+\infty} b_m b_{k-m}
\,.
\ee{expan}
%%% begin NW : added stuff to sentence and commented out old version.
%
% This is the infinite set of ordinary differential equations promised
% in this chapter's introduction. 
% \index{infinite-dimensional flows}
% \index{flow!infinite-dimensional}
% at the beginning of the section.
%
%%% end NW

Since $u(x,t)$ is real,
$ %\[
b_k=b_{-k}^*
\,,
$ %\] %\label{cplx-b}
so we can replace the sum over $k$ in \refeq{expan} by a
sum over $k \geq 0$.
As  $\dot{b_0}=0$, $b_0$ is a conserved quantity,
in our calculations
fixed to $b_0=0$ by
the Galilean invariance condition \refeq{GalInv}.
% \index{Galilean invariance}
% \index{invariance!Galilean}

% The Fourier coefficients $b_k$ are in general complex numbers.
% % of time $t$.  
% We can
% %simplify
% isolate the antisymmetric subspace of the system \refeq{ks-L} by
% considering the case of $b_k$ pure imaginary, $b_k= i a_k$, where
% $a_k$ are real, with the evolution equations
% \beq
% % \dot{a}_k=(k^2- k^4)a_k - k \sum_{m=-\infty}^{\infty} a_m a_{k-m}
% \dot{a}_k = (k/\tilde{L})^2\left( 1 - (k/\tilde{L})^2  \right)a_k 
%  	 - (k/\tilde{L}) \sum_{m=-\infty}^{+\infty} a_m a_{k-m}
% \,.
% \ee{expan-symm}
% \PC{create exercise here}
% %\exerbox{}

Since \KSe preserves antisymmetric solutions, one can isolate the subspace of 
antisymmetric solutions $u(x,t)=-u(-x,t)$,
or $a_{-k}= - a_k$. In \refrefs{Christiansen:97,Lan:Thesis} 
this option was used to eliminate
the continuous translational symmetry.

% In the antisymmetric subspace the translational 
% invariance of the full system reduces
% to the invariance under discrete
% translation by half a spatial period $L$.
% In the Fourier representation \refeq{expan}, 
% this corresponds to invariance under 
% \beq
% a_{2m} \to a_{2m}\,, a_{2m+1} \to -a_{2m+1}
% % \,, m \in \mathbb{Z}
% \,.
% \ee{FModInvSymm} 
% The antisymmetric condition amounts to imposing
% $u(0,t)=0$ boundary condition, with
% the size of the system reduced to
% the $[0,L/2]$ interval. In
% comparing our numerical results with % other authors' 
% calculations for
% the full, unrestricted dynamics on $[0,L]$, we define
% the non-dimensionalized system size as
% $\tilde{L} = {L}/{(4 \pi \sqrt{\nu})}$,
% corresponding to a system defined on the
% $[0,L/2]$ domain. 
