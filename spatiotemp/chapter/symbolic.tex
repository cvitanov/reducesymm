% siminos/kittens/symbolic.tex  % called by blogCats.tex and CL18.tex
% $Author: predrag $ $Date: 2020-05-07 16:34:06 -0500 (Thu, 07 May 2020) $

\ifblog
\chapter{Symbolic dynamics: a glossary}
\label{s-SymbDynGloss}
\else % called by kittens/CL18.tex
\section{Symbolic dynamics: a glossary}
\label{s-SymbDynGloss}
\fi
%%%%%% ChaosBook convention start %%%%%%%%%%%%%%%%%%%%%
\renewcommand{\statesp}{state space}
\renewcommand{\Statesp}{State space}
\renewcommand{\stateDsp}{state-space}
\renewcommand{\StateDsp}{State-space}

Analysis of a low\dmn\ chaotic dynamical system typically
starts\rf{DasBuch} with establishing that a flow is locally stretching, globally
folding. The flow is then reduced to a discrete time return map by appropriate
Poincar\'e sections. Its state space is partitioned, the partitions labeled by an
alphabet, and the qualitatively distinct solutions classified by their temporal
symbol sequences. Thus our analysis of the cat map and the {\catlatt} requires
recalling and generalising a few standard symbolic dynamics notions.

%\noindent
{\bf Partitions, alphabets.}
A division of {\statesp} $\pS$ into a disjoint union of distinct regions
$\pS_A,\pS_B,\ldots,\pS_Z$ constitutes a {\em
partition}. Label each region by a symbol $\Ssym{}$ from an
$N$-letter  {\em alphabet}
$\A=\{A,B,C,\cdots,Z\}$, where $N=\cl{\A}$ is
the number of such regions. Alternatively, one can distinguish different
regions by coloring them, with colors serving as the ``letters'' of the
alphabet.
% missing in kittens:  , as in \reffigs{fig:SingleCatPartit}{fig:AKScloseActSp}.
For notational convenience, in alphabets we sometimes denote negative integer
$\Ssym{}$ by underlining it, as in
\(
\A = \{ -{2}, -{1}, 0, 1\}
   = \{ \underline{2}, \underline{1}, 0, 1\}
\,.
\)


%\noindent
{\bf Itineraries.}
For a dynamical system evolving in time,
every {\statesp} point $\xInit \in \pS$ has the {\em future itinerary},
an infinite sequence of symbols
$\Sfuture(\xInit)=\Ssym{1}\Ssym{2}\Ssym{3}\cdots$ which indicates the
temporal order in which the regions shall be visited. Given a trajectory
$\ssp_1,\ssp_2,\ssp_3,\cdots$ of the initial point $\xInit$ generated
by a time-evolution law
\( %beq
   \ssp_{n+1}=f(\ssp_n)
    % \,, \quad \ssp_0=\xInit
\,,
\) %ee{CMx-iterated}
the itinerary is given by the symbol sequence
\beq
   \Ssym{n} = \Ssym{} \qquad \mbox{if\ } \qquad  \ssp_n \in \pS_{\Ssym{}}
 \,.
\ee{CMsymbol-def}
The {\em past itinerary} $\Spast(\xInit)=\cdots\Ssym{-2}
\Ssym{-1}\Ssym{0} $ describes the order in which the regions were visited
up to arriving to the point $\xInit$. Each point $\xInit$ thus has
associated with it the bi-infinite itinerary
\beq
\Sbiinf(\xInit) % = (\Ssym{k})_{k\in \integers}
        = \Spast.\Sfuture  =
 \biinf{\Ssym{-2}\Ssym{-1}\Ssym{0}}{\Ssym{1}\Ssym{2} \Ssym{3}}
\,,
\ee{CMbiifs}
or simply `itinerary', if we chose not to use the decimal point
to indicate the present,
\beq
   \{\Ssym{\zeit}\} = \cdots\Ssym{-2}\Ssym{-1}\Ssym{0}\Ssym{1}\Ssym{2} \Ssym{3}\cdots
\ee{Itinerary}


%\noindent
{\bf Shifts.}
A forward iteration of temporal dynamics $x\rightarrow x' = f(x)$ shifts
the entire itinerary to the left through the `decimal point'. This
operation, denoted by the {\shiftOp} \shift{},
\beq
   \shift{}(\biinf{\Ssym{-2}\Ssym{-1}\Ssym{0}}{\Ssym{1}\Ssym{2} \Ssym{3}})
     =  \biinf{ \Ssym{-2}\Ssym{-1}\Ssym{0}\Ssym{1}}{ \Ssym{2} \Ssym{3}}
\,,
\ee{CMshift-s}
demotes the current partition label $\Ssym{1}$ from the future $\Sfuture$
to the past $\Spast$.
The inverse shift $\shift{}^{-1}$ shifts the entire itinerary one step
to the right.

The set of all itineraries that can be formed from the letters of the
alphabet $\A$ is called the {\em full shift}
\beq
% \A^\integers 2017-08-05 Predrag dropped this notation
\hat{\AdmItnr} = \{ (\Ssym{k})
              : \Ssym{k} \in \A \quad \mbox{for all} \quad k \in  \integers \}
\,.
\ee{CMFullSh}

The itinerary is infinite for any trapped (non-escaping or \nws\ orbit) orbit
(such as an orbit that stays on a chaotic
repeller), and infinitely repeating for a periodic orbit $p$ of period \cl{p}.
A map $f$ is said to be a \emph{horseshoe} if its restriction to the \nws\ is
hyperbolic and topologically conjugate to the full $\A$-shift.

%\noindent
{\bf Lattices.}
Consider a $d$\dmn\ hypercubic lattice infinite in extent, with each site
labeled by $d$ integers $z\in \integers^{d}$. Assign to each site $z$ a
letter \Ssym{z}\ from a finite alphabet $\A$. A particular fixed set
of letters  \Ssym{z}\ corresponds to a particular lattice state
\(
\Mm= \{\Ssym{z}\} % \in \A \,,\; z\in \integers^d \}
\,.
\)
%infinite in extent along all directions.
In other words, a $d$\dmn\ lattice requires a {$d$\dmn\ code}
\(
% \{\m_{z}\}
\Mm = \{\m_{n_1 n_2 \cdots n_d}\}
%\,,
\)
for a complete specification of the corresponding state $\Xx$.
In the lattice case, the {\em full shift} is the set of all $d$\dmn\
symbol \brick s that can be formed from the letters of the alphabet $\A$
\beq
% \A^{\integers^d}   2017-08-05 Predrag dropped this notation
\hat{\AdmItnr} = \{ \{\Ssym{z}\} % (\Ssym{z}) %_{z\in\integers^d}
              : \Ssym{z} \in \A \quad \mbox{for all} \quad z \in  \integers^d\}
\,.
\ee{LatticeFullSh}

%\noindent
{\bf Commuting discrete translations.}
%{\bf .}
%%%%%%%%%%%%%%%%%%%%%%%%%%%%%%%%%%%%%%%%%%%%%%%%%%%%%%%%%%%%%%%%%%%%%%%%
%    \PC{2016-01-12} {
% in the spirit of \refRefs{PetCorBol07}:
For an autonomous dynamical system, the evolution law $f$ is of the same form for
all times. If $f$ is also of the same form at every lattice site, the group of
lattice translations (sometimes called multidimensional shifts), acting along
$j$th lattice direction by shift $\shift{j}$, is a spatial symmetry that commutes
with the temporal evolution. A temporal mapping $f$ that satisfies
$f\circ\shift{j}=\shift{j}\circ{f}$ along the $d\!-\!1$ spatial lattice directions
is said to be {\em shift invariant}, with the associated symmetry of dynamics
given by the $d$\dmn\ group of discrete {\spt} translations.

\bigskip

Assign to each site $z$ a
letter \Ssym{z}\ from the alphabet $\A$. A particular fixed set
of letters  \Ssym{z}\ corresponds to a particular lattice symbol array
\(
\Mm= \{\Ssym{z}\} % \in \A \,,\; z\in \integers^d \}
 = \{\Ssym{n_1 n_2 \cdots n_d}\}
\,,
\)
which yields a complete specification of the corresponding state $\Xx$.
In the lattice case, the {\em full shift} is the set of all $d$\dmn\
symbol arrays that can be formed from the letters of the alphabet $\A$

as in \refeq{LatticeFullSh}

A $d$\dmn\ {\spt} field
\(
\Xx=\{\ssp_{z}\}
\)
is determined by the corresponding {\em $d$\dmn} {\spt}
symbol array
\(
\Mm=\{\Ssym{z}\}
\,.
\)
Consider next a finite \brick\ of symbols $\Mm_{\R}\subset\Mm$,
over a finite rectangular $[\speriod{1}\!\times\!\speriod{2}\!\times\cdots\times\!\speriod{d}]$
lattice region $\R\subset \integers^d$.
In particular, let $\Mm_{p}$ over a finite rectangular
$[\speriod{1}\!\times\!\speriod{2}\!\times\cdots\times\!\speriod{d}]$ lattice region be the
$[\speriod{1}\!\times\!\speriod{2}\!\times\cdots\times\!\speriod{d}]$ $d$-periodic \brick\ of
\Mm\ whose repeats tile $\integers^d$.

%\noindent
{\bf {\Brick s}.} In the case of temporal dynamics, a finite itinerary
\\
$\Mm_{\R}={\Ssym{k+1}\Ssym{k+2}\cdots\Ssym{k+\speriod{}}}$ of symbols from
$\A$ is called a {\em \brick} of length $\speriod{}=\cl{\R}$. More generally, let
$\R\subset\integers^d$  be a
$[\speriod{1}\!\times\!\speriod{2}\!\times\!\cdots\speriod{d}]$ rectangular lattice region,
$\speriod{k}\geq1$,
whose lower left corner is the $n=(n_{1}n_{2}\cdots{n_{d}})$ lattice site
\beq
  \R = \R_{n}^{[\speriod{1}\!\times\!\speriod{2}\!\times\!\cdots\speriod{d}]}
  =\{(n_1+j_1,\cdots n_d+j_d) \mid 0\leq j_k\leq \speriod{k}-1\}
\,.
\ee{dDimRect}
The associated finite {\brick} of symbols $\Ssym{z}\in\A$ restricted to  \R,
\(
\Mm_{\R}=\{\Ssym{z}| z\in \R \} \subset \Mm
\)
is called the \brick\ $\Mm_{\R}$ of volume
$\cl{\R} = \speriod{1}\speriod{2}\cdots\speriod{d}$. For example, for a 2\dmn\ lattice
a
$\R = [3\!\times\!2]$ \brick\ is of form
\beq
\Mm_{\R}=\left[\begin{array}{c}
\Ssym{12}\ \Ssym{22}\ \Ssym{32}\\
\Ssym{11}\ \Ssym{21}\ \Ssym{31}
\end{array}\right]
\ee{3times2brick}
and volume (in this case, an area) equals $3\times 2 = 6$.
In our convention, the first index is `space', increasing from left to right,
and the second index is `time', increasing from bottom up.

%\noindent
{\bf Cylinder sets.}
While a particular {\admissible} infinite symbol array
\(
\Mm= \{\Ssym{z}\} % \in \A \,,\; z\in \integers^d \}
\)
defines a point $\Xx$ (a unique lattice state) in the \statesp,
the \emph{cylinder set}
$\pS_{\Mm_{\R}}$,
% $ \pS_{\R}$,
corresponds to the totality  of
\statesp\ points $\Xx$ that share the same given finite {\brick} $\Mm_{\R}$
symbolic representation over the region \R. For example, in $d=1$ case
\beq
\pS_{\Mm_{\R}} =
    \{\cdots a_{-2} a_{-1}\,.\,
   \Ssym{1}\Ssym{2}\cdots \Ssym{\speriod{}}
   a_{\speriod{}+1}a_{\speriod{}+2}\cdots\}
\,,
\ee{finiteBlock}
with the symbols  $a_{j}$ outside of the {\brick}
$\Mm_{\R}=[\Ssym{1}\Ssym{2}\cdots \Ssym{\speriod{}}]$
unspecified.
\index{block!finite sequence}
\index{cylinder!set}

%\noindent
{\bf \Po s, \dtors.}
A {\statesp} point $\ssp_z\in\Xx$ is {\spt}ly
{\em periodic}, $\ssp_z=\ssp_{z+\speriod{}}$, if its spacetime orbit returns to it
after a finite lattice shift
\(
\speriod{}= (\speriod{1},\speriod{2},\cdots,\speriod{d})
\)
over region $\R$ defined in \refeq{dDimRect}.
The infinity of repeats of the corresponding {\brick} $\Mm_{\R}$ then tiles the lattice.
For a {\spt}ly {periodic} state $\Xx$, a {\em prime} {\brick}
$\Mm_{p}$ (or $p$) is a smallest such \brick\
\(
\speriod{p}= (\speriod{1},\speriod{2},\cdots,\speriod{d})
\)
that cannot itself be tiled by repeats of a shorter {\brick}.

The periodic tiling of the lattice by the infinitely many repeats of a prime
{\brick} is denoted by a bar: $\cycle{\Mm}_{p}$. We shall omit the bar whenever
it is clear from the context that the state is periodic.
    \PC{2019-01-19}{eliminate \prune{ \Ssym{-m+1}\cdots \Ssym{0}} and
    \rctngl{ \Ssym{-m+1}\cdots \Ssym{0}},
    notation in favor a single convention}
    \PC{2018-11-07}{
    Generalize to \dtors.
    }


In $d=1$ dimensions, prime {\brick} is called a {\em prime} cycle $p$, or a
single traversal of the orbit; its label is a {\brick} of $\cl{p}$ symbols that
cannot be written as a repeat of a shorter {\brick}.
Each {\em periodic point}
\(
      \ssp_{ \Ssym{1} \Ssym{2} \cdots \Ssym{\cl{p}}}
\)
is then labeled by the starting symbol $\Ssym{1}$, followed by
the next $(\cl{p}-1)$ steps of its future itinerary.
The set of periodic points $\pS_p$ that belong to a given periodic orbit
form a {\em cycle}
\beq
p =  \cycle{ \Ssym{1} \Ssym{2} \cdots \Ssym{\cl{p}}}
  = \{
      \ssp_{ \Ssym{1} \Ssym{2}\cdots \Ssym{\cl{p}}},
      \ssp_{ \Ssym{2} \cdots \Ssym{\cl{p}} \Ssym{1}},
    \cdots,
      \ssp_{ \Ssym{\cl{p}} \Ssym{1}\cdots \Ssym{\cl{p}-1}}
     \}
\,.
\ee{PeriodCyc}

More generally, a {\statesp} point is {\em {\spt}ly periodic} if
it belongs to an \dtor, \ie, its symbolic representation is a \brick\
over region $\R$ defined by \refeq{dDimRect},
\beq
  \Mm_{p} = \Mm_{\R}
  \,,\qquad
  \R = \R_{0}^{[\speriod{1}\!\times\!\speriod{2}\!\times\cdots\times\!\speriod{d}]}
\,,
\ee{dTorus}
that
tiles the lattice state  $\Mm$ periodically, with period $\speriod{j}$ in the
$j$th lattice direction.


%\noindent
{\bf Generating partitions.}
A temporal partition is called {\em generating} if every bi-infinite itinerary
corresponds to a distinct point in {\statesp}.
In practice almost any
generating partition of interest is infinite.
Even when the dynamics assigns a unique infinite itinerary
$\biinf{\Ssym{-2}\Ssym{-1}\Ssym{0}}{\Ssym{1}\Ssym{2} \Ssym{3}}$ to each
distinct orbit, there generically exist full shift itineraries
\refeq{CMFullSh} which are not realized as orbits; such sequences are
called {\em \inadmissible}, and we say that the symbolic dynamics is {\em
pruned}.

%\noindent
{\bf Dynamical partitions.}
If the symbols outside of given temporal {\brick} $b$ remain unspecified, the
set of all {\admissible} {\brick s} of length $\cl{b}$ yield a dynamically
generated partition of the \statesp, $\pS = \cup_b \pS_b$.

%\noindent
{\bf Subshifts.}
A dynamical system $(\pS,f)$ given by a mapping $f : \pS \to \pS$
together with a {partition} $\A$ induces {\em topological dynamics}
$(\AdmItnr,\shift{})$, where the {\em subshift}
\beq
\AdmItnr = \{  (\Ssym{k})_{k\in \integers} \}
\,,
\ee{subshift}
is the set of all {\em \admissible} itineraries, and
$\shift{} \,:\, \AdmItnr \to \AdmItnr$
is the {\shiftOp} \refeq{CMshift-s}. The designation `subshift' comes
from the fact that $\AdmItnr$
% \subset \hat{\AdmItnr}$
% \A^\integers 2017-08-05 Predrag dropped this notation
is a subset of the full shift.

%%%%%%%%%%%%%%%%%%%%%%%%%%%%%%%%%%%%%%%%%%%%%%%%%%%%%%%%%%%%%%%%%%%%%%%%
%    \PC{2016-10-11} { inspired by {Ban} \etal\rf{BHLL11}
Let $\hat{\AdmItnr}$ be the full lattice shift  \refeq{CMFullSh}, \ie,
the set of all possible lattice state $\Mm$ labelings by the alphabet
$\A$, and $\hat{\AdmItnr}(\Mm_{\R})$ is
the set of such {\brick s} over a region {\R}. The principal task
in developing the symbolic dynamics of a dynamical system is to determine
$\AdmItnr$, the set of all \emph{{\admissible}} itineraries/lattice states,
\ie, all states that can be realized by the given system.

%\noindent
{\bf Pruning, grammars, recoding.}
If certain states are {\inadmissible}, the alphabet must be supplemented by a
{\em grammar},
a set of pruning rules.
Suppose that
the grammar can be stated as a finite number of pruning rules, each
forbidding a {\brick} of finite size,
\beq
 {\cal G} = \left\{
        b_1, b_2, \cdots b_k
        \right\}
\,,
\ee{grammar}
where a {\em pruned {\brick}} $b$ is an array of symbols defined over a
finite $\R$ lattice region of size
$[\speriod{1}\!\times\!\speriod{2}\!\times\cdots\times\speriod{d}]$. In
this case we can construct a finite Markov partition by replacing finite
size \brick s of the original partition by letters of a new alphabet. In
the case of a 1\dmn, the temporal lattice, if the longest forbidden {\brick}
is of length $L+1$, we say that the symbolic dynamics is Markov, a shift
of finite type with {$L$-step memory}.

%\noindent
{\bf Subshifts of finite type.}
A {topological dynamical system} $(\AdmItnr,\shift{})$ for which all
{\admissible} states $\Mm$ are generated by recursive application
of the finite set of pruning rules \refeq{grammar}
%of the  finite transition matrix
%\beq
%\AdmItnr = \left\{ (\Ssym{k})_{k\in \integers}
%           \,:\,
%                 T_{\Ssym{k} \Ssym{k+1}} = 1
%        \quad \mbox{for all $k$} \right\}
%\ee{AdmItnr}
is called a subshift of {\em finite type}.

                                            \toCB % to kneading.tex
If a map can be topologically conjugated to a linear map, the symbolic
dynamics of the linear map offers a dramatically simplified description
of all {\admissible} solutions of the original flow, with the temporal
symbolic dynamics and the state space dynamics related by linear recoding
formulas. For example, if a map of an interval, such as a parabola, can
be conjugated to a piecewise linear map, the kneading theory\rf{MilThu88}
classifies \emph{all} of its {\admissible} orbits.

%%%%%% ChaosBook convention,  BORIS conventions start %%%%%%%%%%%%%%%%%%%%%
\renewcommand{\statesp}{phase space}
\renewcommand{\Statesp}{Phase space}
\renewcommand{\stateDsp}{phase-space}
\renewcommand{\StateDsp}{Phase-space}
%%%%%% BORIS convention end   %%%%%%%%%%%%%%%

%%%%%%%%%%%%%%%%%%%%%%%%%%%%%%%%%%%%%%%%%%%%%%%%
\ifblog
% siminos/spatiotemp/chapter/symbolicIns.tex  pdflatex blog; biber blog
% $Author: predrag $ $Date: 2019-12-14 01:25:58 -0600 (Sat, 14 Dec 2019) $

%Predrag            2016-12-20

\section{Symbolic dynamics, inserts}
\label{s-SymbDynDefs}
% from ChaosBoook  \Chapter{knead}{15feb2015}{Charting the state space}

{\bf 2019-01-19 Predrag} Merge everything here to \refchap{s-SymbDynGloss}
{\em Symbolic dynamics: a glossary} then \texttt{svn rm} this file.

{\bf 2017-08-05 Predrag}
Consult / harmonize with  ChaosBook.org Chapter~{\em Charting the state space} (source file knead.tex).



%%%%%%%%%%%%%%%%%%%%%%%%%%%%%%%%%%%%%%%%%%%%%%%%%%%%%%%%%%%%%%%
\bigskip

to Predrag: check that all this is in Chaos\-Book, then erase:

\bigskip


The set of all bi-infinite itineraries that can be formed from the
letters of the alphabet ${\cal A}$ is called the
{\em full shift} (or {\em topological Markov chain})
\index{shift!full}
\index{Markov!chain}\index{topological!Markov chain}
% before \ee{FullSh}

Here we refer to this set of all conceivable itineraries
as the {\em covering} symbolic dynamics.
\index{symbolic dynamics!covering}
\index{covering!symbolic dynamics}

Orbit that starts out as a finite {\brick} followed by infinite number of
repeats of another {\brick} $p = (\Ssym{1} \Ssym{2} \Ssym{3} \cdots
\Ssym{\period{}})$ is said to be {\em heteroclinic} to the cycle $p$. An
orbit that starts out as $p^{\infty}$ followed by a different finite
{\brick} followed by $(p')^{\infty}$ of another {\brick} $p'$ is said to be a
{\em heteroclinic connection} from cycle $p$ to cycle $p'$.
    \index{heteroclinic!connection}



Suppose that
the grammar can be stated as a finite number of pruning rules, each
forbidding a {\brick} of finite length,
\index{symbolic dynamics!coding}
\beq
 {\cal G} = \left\{
        b_1, b_2, \cdots b_k
        \right\}
\,,
\ee{grammar}
where a {\em pruned {\brick}} $b$ is a sequence of symbols
$b=\block{ \Ssym{1} \Ssym{2} \cdots \Ssym{\cl{b}}}$,
 $\Ssym{} \in {\cal A}$,
of finite length $\cl{b}$.
\index{block, pruning}
\index{pruning!block}
\index{shift!finite type}
\index{symbolic dynamics!recoding}
\index{recoding}


\noindent{\bf Subshifts of finite type.}
A {topological dynamical system} $(\Sigma,\sigma)$ for
which all {\admissible} itineraries are generated by a finite
transition matrix
\beq
\Sigma = \left\{ (\Ssym{k})_{k\in \integers} \,:\, T_{s_k s_{k+1}} = 1
        \quad \mbox{for all $k$} \right\}
\ee{AdmItnr}
is called a subshift of {\em finite type}.

\noindent{\bf Reflection symmetries.}
Symmetries of the cat map induce  invariance with respect to
corresponding symbol exchanges. Define $\bar{m}=s\!-\!m\!-\!2$ to be the
conjugate of symbol $m \in \A$. For example, the two exterior
alphabet \Ae\ symbols are conjugate to each other, as illustrated by
\refeq{eq:StateSpCatMap}.
\PC{2019-05-27}{fix this eq. reference; edit it away}
If ${b}=\Ssym{1} \Ssym{2} \dots \Ssym{\ell}$ is a
\brick, and  $\bar{{b}}=\bar{m}_1 \bar{m}_2 \dots
\bar{m}_\ell$ its conjugate, then by  reflection symmetry of the cat
map we have  $|\Pol_{{b}}|= |\Pol_{\bar{{b}}}|$. Similarly, if
$b^*=\Ssym{l}\Ssym{l-1}\dots \Ssym{1}$, the time reversal invariance implies
$|\Pol_{{b}}|=|\Pol_{{b}^*}|$.

There are many ways to skin a cat. For example, due to the space
reflection symmetry about $\ssp=1/2$ of the \PV\ cat map
\refeq{eq:StateSpCatMap}, it is natural (especially in studies of
deterministic diffusion on periodic
lattices\rf{ArtStr97,CBdiffusion,CBappendDiff}) to center  the \statesp\
unit interval\rf{PerViv} as $\ssp\in[-1/2,1/2)$. In this formulation the
\PV\  cat map has a 5-letter alphabet
$\A=\{\underline{2},\underline{1},0,1,2\}$, in which the spatial
reflection symmetry is explicit (the ``conjugate'' of a symbol $m \in \A$
is $\bar{m}= -\!m$).
% , with \statesp\ partitions and pruning rules taking a symmetric form.

%%%%%%%%%%%%%%%%%%%%%%%%%%
% \renewcommand{\cl}[1]{{\ensuremath{|#1|}}}  % the length of a periodic orbit, Ronnie
 %symbolicIns}
\printbibliography[heading=subbibintoc,title={References}]
\fi
%%%%%%%%%%%%%%%%%%%%%%%%%%%%%%%%%%%%%%%%%%%%%%%%
