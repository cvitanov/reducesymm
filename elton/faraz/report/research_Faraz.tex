\documentclass{article}
\usepackage{graphicx}
\graphicspath{{../../figures/}} 
\usepackage{amsmath,amsfonts,amssymb,amsbsy,amscd,amsgen}
\usepackage{MnSymbol}
\usepackage{enumerate}
\usepackage{natbib}
  \newcommand{\arXiv}[1]{{\tt \goodbreak #1}}

\title{Summary of Research}
\author{Mohammad Farazmand}

\newcommand{\id}{\mathrm d}
\newcommand{\vc}{\mathbf}
\newcommand{\bnabla}{\pmb\nabla}

\begin{document}
\maketitle
\begin{enumerate}
\item {\bf Adjoint-based method to find equilibria of the Navier--Stokes equations}

The state-of-the-art method for finding the invariant solutions of Navier--Stokes equations (i.e. Newton-GMRES--hook iterations)
typically has a small domain of convergence. As a result, relatively good initial guesses are required for the iterations to converge.
Such initial guesses are often found through \emph{ad hoc} methods. We introduced an \emph{adjoint} step to the Newton-GMRES-hook iterations
that renders the iterations globally convergent~\cite{Faraz15}. In other words, with the adjoint step, arbitrary initial guesses converge to equilibria of 
the Navier--Stokes equations.

The main idea is to find an equation, adjoint to the Navier--Stokes equations, with the following properties: 1) It has the same set of equilibrium 
solutions as the Navier--Stokes equations. 2) The equilibria of the adjoint equation are asymptotically stable.
As a result, arbitrary initial conditions evolve under the adjoint to an arbitrarily close close neighborhood of a Navier--Stokes equilibrium. 
At this point, a few iterations of the Newton-GMRES--hook iterations yield the equilibrium with the desired accuracy.

\item {\bf State space geometry of the pipe flow}

We use a multi-point shooting algorithm, for the first time, to find long relative periodic orbits of the Navier-Stokes equations in a cylindrical pipe \cite{WFSBC15}.
This led to the discovery of several new relative periodic solutions that belong to a hierarchy 
of period doubling orbits. We show that two of these families have the topology of a M\"obius strip in the state space.
This M\"obius-shaped family has a curious signature in the configuration space. Namely, the longer orbit of the family
resembles the shorter orbit with the same number of vortex tubes. However, the vorticies of the longer orbit reverse sign as the shorter orbit completes one period 

\item{\bf Particle kinematics on the ocean surface}

The John--Sclavounos (JS) equation describes the  motion of a fluid particle on the free-surface of a water wave.
We derived the Lagrangian and Hamiltonian formulation of the JS equations~\cite{FF15,FarazmandJFM2015}. These formulations provides several new insight
into the problem: 1) We show that the JS equations in fact hold for any particle moving under the force of gravity whose motion
is constrained to an unsteady surface. 2) A fundamental open question regarding the JS equation is its well-posedness, i.e., do 
the solutions exist for all times? Our Hamiltonian formulation answers this equation by ruling out the possibility of finite-time blowups.
3) The Hamiltonian formulation further shows the existence of trapping regions on the free surface, i.e., regions that particles cannot scape from.
This in turn partitions the phase space into subsets with coherent and chaotic particle motion.


\end{enumerate}

\bibliographystyle{plain}
\bibliography{../../bibtex/elton}

\end{document}