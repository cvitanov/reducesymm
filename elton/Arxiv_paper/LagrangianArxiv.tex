\documentclass[letter,12pt,openany]{article}

\pdfoutput=1

\usepackage[legalpaper, margin=1in]{geometry}
\usepackage{amsmath,amsfonts,amssymb,amsthm}
\usepackage{authblk}
\usepackage{color}
\usepackage{url}
\usepackage{fancyhdr}
\usepackage{alltt}
\usepackage{ifthen}
\usepackage{svn-multi}
\usepackage[latin1]{inputenc}
\usepackage{times}
\usepackage[T1]{fontenc}
\usepackage[pdftex]{graphicx}
\usepackage{subcaption}
\usepackage[font={small}]{caption}
\usepackage{array}
\usepackage[pdftex,colorlinks]{hyperref}
\graphicspath{{figs/}} %% directories with  graphics files

%% If the svn information should be also placed
%% on the chapter page use:
%\fancypagestyle{plain}{%
%% otherwise use
\pagestyle{fancy}
\fancyhead{}
\fancyhead[er,ol]{\slshape \leftmark }


% elton/inputs/defsElton.tex
% $Author: predrag $ $Date: 2015-09-26 19:42:18 -0400 (Sat, 26 Sep 2015) $

%%%%%%%%%%%% MACROS, noisy project specific %%%%%%%%%%

    \ifboyscout
\newcommand{\toCB}{\marginpar{\footnotesize 2CB}}  % to compare with ChaosBook
\newcommand{\inCB}{\marginpar{\footnotesize now in CB}} % entered in ChaosBook
\newcommand{\JRE}[1]{$\footnotemark\footnotetext{JRE: #1}$}
\newcommand{\JREedit}[1]{{\color{red}#1}}
\renewcommand{\authorPC}[1]{\hfill (P. Cvitanovi\'c, #1)}
\renewcommand{\authorJMH}[1]{\hfill (J.M. Heninger, #1)}
\renewcommand{\authorMMFPC}[1]
     {\hfill (M.M. Farazmand and P. Cvitanovi\'c, #1)}
\renewcommand{\authorMMF}[1]{\hfill (M.M. Farazmand, #1)}
    \else
\newcommand{\toCB}{}
\newcommand{\inCB}{}
\newcommand{\JRE}[1]{}
\newcommand{\JREedit}[1]{#1}
\renewcommand{\authorPC}[1]{\hfill (P. Cvitanovi\'c)}
\renewcommand{\authorJMH}[1]{\hfill (J.M. Heninger)}
\renewcommand{\authorMMFPC}[1]
     {\hfill (M.M. Farazmand and P. Cvitanovi\'c)}
\renewcommand{\authorMMF}[1]{\hfill (M.M. Farazmand)}
   \fi %end of internal draft switch

\newcommand{\PCpost}[2]{\item[#1 Predrag] {#2}}
\newcommand{\APWpost}[2]{\item[#1 Ashley] {#2}}
\newcommand{\KYSpost}[2]{\item[#1 Kimberly] {#2}}
\newcommand{\BEpost}[2]{\item[#1 Bruno] {#2}}
\newcommand{\FFpost}[2]{\item[#1 Franco] {#2}}
\newcommand{\MMFpost}[2]{\item[#1 Mohammad] {#2}}
\newcommand{\AFpost}[2]{\item[#1 Adam] {#2}}
\newcommand{\YLpost}[2]{\item[#1 Lan] {#2}}
\newcommand{\NBBpost}[2]{\item[#1 Burak] {#2}}

%%%%%%%%%%%% MACROS, project specific %%%%%%%%%%


\newcommand{\NS}{Navier-Stokes}
\newcommand{\NSe}{Navier-Stokes equation}
\newcommand{\Reynolds}{\ensuremath{\textit{Re}}}  % Reynolds number
\newcommand{\Velgradmat}{Matrix of velocity gradients}

\newcommand{\steady}{\marginpar{{\color{green}\textdollar}}}

%%%%%%%%%%%%%%% Sundry symbols within math eviron.: %%%%%%%%%%%%
\newcommand{\pd}[2]{\frac{\partial #1}{\partial #2}}
\newcommand{\grad}{{\bf \nabla}}
\newcommand{\trHalf}[1]{\tau_{#1}}    % 1/2 cell translation
\newcommand{\DiffC}{\ensuremath{D}}     % diffusion constant

%%% 3D physical flow naming conventions
% PC \teq is the name, \seq is the  symbol
\newcommand{\teq}[1]{\ensuremath{{\text{eq#1}}}}
\newcommand{\teqa}{\ensuremath{{\text{eq0}}}}
\newcommand{\teqb}{\ensuremath{{\text{eq1}}}}
\newcommand{\teqc}{\ensuremath{{\text{eq2}}}}

\newcommand{\ttw}[1]{\ensuremath{{\text{tw#1}}}}
\newcommand{\ttwa}{\ensuremath{{\text{tw1}}}}  % spanwise
\newcommand{\ttwb}{\ensuremath{{\text{tw2}}}} % Divakar D1, lower streamwise
\newcommand{\ttwc}{\ensuremath{{\text{TW3}}}}  % upper streamwise

\newcommand{\bu}{\ensuremath{{\bf u}}}
\newcommand{\bx}{\ensuremath{{\bf x}}}
\newcommand{\be}{{\bf e}}
\newcommand{\ben}[1]{{\be}_{#1}}
\newcommand{\beUBg}[1]{\ensuremath{\be_{#1}}}
\newcommand{\butot}{\ensuremath{{\bf u_{tot}}}}
\newcommand{\bnabla}{\ensuremath{{\bf \nabla}}}
\newcommand{\lapl}{\ensuremath{{\nabla^{2}}}}
\newcommand{\Norm}[1]{\|{#1}\|}

\newcommand{\xeq}[1]{\ensuremath{\bx_{\text{\tiny eq#1}}}}
\newcommand{\xeqa}{\ensuremath{\bx_{\text{\tiny eq0}}}}
\newcommand{\xeqb}{\ensuremath{\bx_{\text{\tiny eq1}}}}
\newcommand{\xeqc}{\ensuremath{\bx_{\text{\tiny eq2}}}}

\newcommand{\xtw}[1]{\ensuremath{\bx_{\text{\tiny tw#1}}}(t)}
\newcommand{\xtwa}{\ensuremath{\bx_{\text{\tiny tw1}}}(t)}
\newcommand{\xtwb}{\ensuremath{\bx_{\text{\tiny tw2}}}(t)}
\newcommand{\xtwc}{\ensuremath{\bx_{\text{\tiny tw3}}}(t)}

%%%%%%%%%%%%%%%%%%%%%%%%%%%%%%%%%%%%%%%%%%%%%%%%%%%%%%%%%%%
% PC: experimental, for stagnation points
\newcommand{\tSP}[1]{{\ensuremath{{\text{SP#1}}}}}
\newcommand{\tSPone}{\ensuremath{{\text{SP1}}}}
\newcommand{\tSPtwo}{\ensuremath{{\text{SP2}}}}
\newcommand{\tSPthr}{\ensuremath{{\text{SP3}}}}
\newcommand{\xSP}[1]{{\ensuremath{\bx_{\text{\tiny SP#1}}}}}
\newcommand{\xSPone}{\ensuremath{\bx_{\text{\tiny SP1}}}}
\newcommand{\xSPtwo}{\ensuremath{\bx_{\text{\tiny SP2}}}}
\newcommand{\xSPthr}{\ensuremath{\bx_{\text{\tiny SP3}}}}

%%%%%%%%%%%%%%%%%%%%%%%%%%%%%%%%%%%%%%%%%%%%%%%%%%%%%%%%%%%
% JFG: Fluids literature uses bold to indicate vector
% quantities, so that one can use {\bf u} for the vector
% and u for the first component of the vector.
% PC \eqva/\reqva directory: \tAA is the name, \sAA is the  symbol
\newcommand{\tLM}{\ensuremath{{\text{EQ0}}}}
\newcommand{\tLB}{\ensuremath{{\text{EQ1}}}}
\newcommand{\tUB}{\ensuremath{{\text{EQ2}}}}
\newcommand{\tNNB}{\ensuremath{{\text{EQ3}}}}
\newcommand{\tNB}{\ensuremath{{\text{EQ4}}}}
\newcommand{\tEQfive}{\ensuremath{{\text{EQ5}}}}
\newcommand{\tEQsix}{\ensuremath{{\text{EQ6}}}}
\newcommand{\tEQsev}{\ensuremath{{\text{EQ7}}}}
\newcommand{\tEQeight}{\ensuremath{{\text{EQ8}}}}
\newcommand{\tEQnine}{\ensuremath{{\text{EQ9}}}}
\newcommand{\tEQten}{\ensuremath{{\text{EQ10}}}}

\newcommand{\tTW}[1]{\ensuremath{{\text{TW#1}}}}
\newcommand{\tTWone}{\ensuremath{{\text{TW1}}}}  % spanwise
\newcommand{\tTWDone}{\ensuremath{{\text{TW2}}}} % Divakar D1, lower streamwise
\newcommand{\tTWthree}{\ensuremath{{\text{TW3}}}}  % upper streamwise


\newcommand{\bCell}{\ensuremath{\Omega}}
\newcommand{\bNarrow}{\ensuremath{\Omega_{\text{\tiny W03}}}}
    % JG: W02 for Waleffe Tokyo proceedings 2002, where this cell first appears,
    % ok'd by wally
\newcommand{\bHKW}{\ensuremath{\Omega_{\text{\tiny{HKW}}}}}
\newcommand{\bSch}{\ensuremath{\Omega_{\text{\tiny{Sch}}}}} % Schmiegel
\newcommand{\bbR}{\mathbb{R}}
%\newcommand{\bbU}{\mathbb{U}}
%\newcommand{\bbUsymm}{\ensuremath{\bbU_{S}}}
\newcommand{\bbUS}{\ensuremath{\bbU_{S}}}
\newcommand{\bbUthree}{\ensuremath{\bbU_{s3}}}

% PC \eqva velocity field naming conventions
\newcommand{\uEQ}{\ensuremath{\bu_{\text{\tiny EQ}}}}
\newcommand{\uLM}{\ensuremath{\bu_{\text{\tiny EQ0}}}}
\newcommand{\uLB}{\ensuremath{\bu_{\text{\tiny EQ1}}}}
\newcommand{\uUB}{\ensuremath{\bu_{\text{\tiny EQ2}}}}
\newcommand{\uNNB}{\ensuremath{\bu_{\text{\tiny EQ3}}}}
\newcommand{\uNB}{\ensuremath{\bu_{\text{\tiny EQ4}}}}
\newcommand{\uEQfive}{\ensuremath{\bu_{\text{\tiny EQ5}}}}
\newcommand{\uEQsix}{\ensuremath{\bu_{\text{\tiny EQ6}}}}
\newcommand{\uEQsev}{\ensuremath{\bu_{\text{\tiny EQ7}}}}
\newcommand{\uEQeight}{\ensuremath{\bu_{\text{\tiny EQ8}}}}
\newcommand{\uEQnine}{\ensuremath{\bu_{\text{\tiny EQ9}}}}
\newcommand{\uEQten}{\ensuremath{\bu_{\text{\tiny EQ10}}}}

\newcommand{\GPKF}{\ensuremath{\Gamma}} % Hoyle notation, equivariant symmetry group
\newcommand{\trDiscr}[2]{\tau_{#1}^{#2}}    % discrete cell translation 1/4, ...
% isotropy subgroup $H \incl G$:
\newcommand{\isotropyG}[1]{\ensuremath{H_{\text{\tiny #1}}}}


%%%%%%%%%%%%%%%%%%% eventually remove these %%%%%%%%%%%%%%%%%%%%%%%%%%%%%
\newcommand{\huUB}{\ensuremath{\hbu_{\text{\tiny EQ2}}}}
\newcommand{\hbu}{\tilde{{\bf u}}}
\newcommand{\hbv}{\tilde{{\bf v}}}
\newcommand{\hu}{\tidle{u}}
\newcommand{\hv}{\tidle{v}}
\newcommand{\vc}{\mathbf}



% Keywords command
\providecommand{\keywords}[1]
{
  \small	
  \textbf{\textit{Keywords--}} #1
}


\title{Lagrangian Dynamics in Plane Couette Turbulence}


\author[1]{John R. Elton\thanks{Jelton.physics@gmail.com}}
\author[1]{Predrag Cvitanovi\'{c}\thanks{predrag@gatech.edu}}
\author[1]{John F. Gibson}
\author[1]{Jonathan Halcrow}

\affil[1]{School of Physics, Georgia Institute of Technology}


\begin{document}
    
\date{}

\maketitle

\begin{abstract}
\noindent 
The behaviors of tracer particle trajectories for invariant solutions of the Navier-Stokes equations confined to the three-dimensional geometry of plane Couette flow are studied. Treating an equilibrium velocity field as a dynamical system, the transport of such tracers, or passive scalars, along flow trajectories within the field, reveals coherent structures in the Lagrangian frame that form a basic outline for the different types of motion that can occur. Using analytic arguments that make use of the symmetry of the plane Couette geometry, we determine special points in the velocity field for which where there is no fluid movement. We numerically determine the stability of these points of stagnation along with their stable and unstable manifolds, and also find heteroclinic connections between them. These topological features provide a structured phase portrait that outlines the $3D$ Lagrangian flow dynamics for a turbulent fluid, which may be used as a first step to studying a wide range of bulk fluid properties related to particle transport and advective mixing.
\end{abstract}


\keywords{turbulence, mixing, plane Couette flow, Navier-Stokes}


%%% Introduction %%%
\section{\centering Introduction}
\label{sec:intro}

The turbulent transport and mixing of different particles or species within a fluid is a problem with both wide practical application as well as theoretical interest, yet a complete understanding of the phenomena remains elusive; even questions related to how we define or measure various mixing properties are not universally agreed upon. In \cite{Mathew}, some pitfalls of standard approaches such as measuring variation from homogeneity with an $L^2$ or $L^p$ norm, or computing the entropy of the underlying dynamical system, are pointed out. Furthermore, there are experimental and computational challenges involved when studying the problem in the natural Lagrangian frame (\cite{MHPRS07} \cite{ABBBBB08} \cite{Braun1} \cite{Mordant}). Although the idea of taking a dynamical systems approach to the problem is not new, as books by Ottino \cite{Botti89} and Wiggins \cite{wiggbook} attest to the approach of using invariant manifolds to study fluid transport, \cite{MHPRS07} and \cite{HallerLagrangian} point out that Lagrangian coherent structures in real flow data are difficult to identify due to the uncertain stability of individual particles. Thus many of the theoretical and experimental analyses are confined to \textit{two-dimensional} systems, with a large body of the work on Lagrangian dynamics focusing on the statistical properties and fluctuations of particle velocities, and on detecting intermittency or anomalous scaling laws (\cite{Nzerem} \cite{Mordant} \cite{ABBBBB08} \cite{Falkovich}).

In this study, we extend the idea of looking at the Lagrangian transport of passive scalars by means of the invariant structures within the flow in a \textit{truly $3D$ system}, partitioning the physical space of the fluid by distinct types of motion that organize tracer mixing \cite{HallerLagrangian}. By building upon the computational work that has provided exact invariant solutions of the fully resolved Navier-Stokes equations for plane Couette flow, described below, we are able to use equilibrium velocity field solutions to study a tractable, yet still complex problem that lends itself to a dynamical systems analysis. Symmetry considerations allow for a first tangible step that will lead to piecing together a full phase portrait of such an equilibrium flow, by determining the fixed points and their stabilities along with heteroclinic connections.  Our eventual goal is then putting this information
together to assist in understanding how to calculate quantities to best characterize turbulent fluid mixing.

The plane Couette geometry we study is a shear flow in which two infinite plates move in opposite directions at constant speed, with turbulent behavior beginning to set in approximately above Reynolds number $Re=325$ \cite{HGCV09}. 
Eulerian equilibrium velocity fields have been computed for this setup over a number of years, and \pCf\ also admits periodic, relative periodic, and traveling wave solutions (\cite{HGCV09} \cite{DV04}). In 1990 Nagata \cite{N90} discovered what are known as the Upper Branch and Lower Branch equilibria by continuing a known solution from Taylor-Couette flow to plane
Couette. Later, Waleffe \cite{W03}
calculated the same solutions a different way and noted that they
satisfy 'shift-rotate' and 'shift-reflect' symmetry. Gibson et al. \cite{GHCW07} began explorations of plane Couette dynamics around
those equilibria, making use of the symmetries and noting that the subspace of velocity fields under the action of certain symmetry groups was invariant under Navier-Stokes.
The search for new invariant solutions focused on this subspace, from which a Newton search was able to detect additional equilibria. The reader may consult \cite{HGCV09} or \cite{GHCW07} for additional history of the computational discoveries of invariant solutions for plane Couette flow.

Much of the analysis in this work is carried out on a particular equilibrium solution referred to as the Upper Branch or $EQ_2$.
    We also repeat some of our analysis  for another equilibrium velocity field $EQ_8$, for which the flow is more turbulent
    and possesses different invariant symmetries.
For analyzing fluid particle trajectories from the Lagrangian perspective, where we follow the motion of a tracer within a fixed equilibrium,  we need to make a distinction between
$3D$ physical fluid flow for a given invariant solution of Navier-Stokes
and the dynamical $\infty$-dimensional \statesp\ flow. We distinguish between the two by using physically motivated nomenclature
for the $3D$ physical fluid flow: We shall refer to the position
 for which
$\bu(\bx_{_{SP}})=0$
as the {\em \stagp} $\bx_{_{SP}}$ or point $SP$. And when we discuss coherent structures and heteroclinic connections, these again refer to trajectories \textit{within} a known Eulerian equilibrium velocity field, in contrast to the heteroclinic connections described in, for example \cite{HGCV09}, which track the evolution of the velocity fields themselves.

 In \refsect{sec:NS}-\refsect{PCF_symm} we review the underlying equations and geometry for plane Couette flow, describe how the equilibria are computed numerically, and give a deep dive on the symmetries which are crucial for later analysis. Much of the information in these sections is a rehash that can be found in other places including \cite{GHCW07}, but is important for understanding the new contributions of this work.  In \refsect{sec:symm_stag} we show how the known symmetries automatically provide us with critical information for analyzing Lagrangian trajectories within each equilibrium by determining where the velocity field must be exactly 0; in other words we are able to locate the "fixed points" in dynamical systems terminology, or stagnation points in our lingo. In \refsect{sec:Lagrangian} we give our core analysis and results: namely a dynamical systems treatment of Lagrangian trajectories within plane Couette equilibria that includes a treatment of fixed points, stability analysis and invariant manifolds, and heteroclinic connections, providing the basic dynamical skeleton through which transport and mixing properties in a turbulent flow field may be analyzed. We provide an intriguing graphical phase portrait of the turbulent motion within the Upper Branch equilibrium and also provide some results for $EQ_8$ and discuss potential applications.





\section{\centering Plane Couette Flow}
\label{sec:PCF}

\subsection{The Navier-Stokes equations}
\label{sec:NS}
 The underlying equations
that govern the motion of \pCf\ are the \NS\ equations,
along with boundary conditions. The boundary conditions for \pCf\ in the $x$
and $z$ directions are periodic,
 $ \bu(x, y, z) = \bu(x+L_x, y, z) =
\bu(x, y, z + L_z) $.
 In the $y$ direction,
 $\bu = (1,0,0)$ at $\bx = (0,1,0)$ and $\bu = (-1,0,0)$ at $\bx =
 (0,-1,0)$.

 The fluid is taken to be incompressible, so in this case the
 \NS\ equations are
 \beq
 \frac{\partial \bu}{\partial t} + (\bu \cdot \nabla)\bu = -\nabla p + \frac{1}{Re} \nabla^{2} \bu
    \,,\qquad
\nabla \cdot \bu  = 0 \,. \label{eqn:NavierStokes} \eeq 



For an equilibrium velocity field that is not changing in time, the first equation in
\refeq{eqn:NavierStokes} simplifies to \beq
 (\bu \cdot \nabla)\bu = -\nabla p + \frac{1}{Re} \nabla^{2} \bu
    \,,\qquad \label{eqn:NavierStokes2} \eeq
 
  The Reynolds number parameter $\Reynolds$, which gives a measure of fluid viscosity and degree to which fluid motion may become turbulent, is given by \beq Re = \frac{\overline{u}L}{\nu}
\eeq where $\overline{u}$ is the average fluid velocity and $L$ is
the characteristic length. Thus the form of the \NS\ equations and boundary conditions make use of rescaling to use non-dimensionalized variables. 
We use $\Reynolds = 400$, in the regime of moderate turbulence, for the \pCf\ simulations throughout the text unless otherwise indicated.

In most of our work, velocity field  $\bu$
represents the {\em difference} from the laminar flow. 
So we can break up the total velocity field into two components: $\butot =
y \hat{\bf x} + \bu$. Here $y \hat{\bf x}$ is the laminar velocity
field and $\bu$ is then the difference between the total velocity and
laminar. Substitute $y \hat{\bf x} + \bu$ for $\bu$ in the
nondimensionalized Navier-Stokes equations above to get

\beq
    \frac{\partial \bu}{\partial t}
    + y  \frac{\partial \bu}{\partial x}
    + v \, \hat{\bf x}
    + \bu \cdot \bnabla \bu
=
    - \bnabla p
    + \frac{1}{\Reynolds}
        \lapl \bu  \,, \quad \nabla \cdot \bu = 0
\, \label{NavStokesDiff}
\eeq
with boundary conditions $\bu = 0 $ at $y \pm 1$.
The equilibrium velocity fields we study satisfy \refeq{NavStokesDiff}. This equation is a little more complicated than
\refeq{eqn:NavierStokes}, but having Dirichlet boundary conditions on
$\bu$ makes the analysis much easier, since the set of allowable velocity fields (those fields that satisfy incompressibility and boundary
conditions)
forms a vector space. \\


\begin{figure}[!h]
 \begin{center} 
\includegraphics[width=0.5\textwidth]{figs/eq2.png}
  \caption{
   Visualization of the Upper Branch equilibrium velocity field, from {\tt Channelflow.org}.
   }
  \label{eltonFig:UB}
  \end{center}
 \end{figure}



\subsection {\bf Computation of equilibrium velocity fields and trajectories}
\label{channelflow}
 In order to integrate streamlines of {\pCf}
and follow the paths of tracer particles, it is first
necessary to have numerically accurate \eqv\ $3D$-velocity fields.


The starting point for this task is to obtain the necessary data sets for evaluating velocity field values for a given \eqv, e.g. the Upper Branch as shown in \reffig{eltonFig:UB}. These are made available at the website
{\tt Channelflow.org} \cite{channelflow}. Essentially the data \cite{channelflowDat} contains the spectral coefficients $\mathbf{\hat{u}}$ of the
expansion of a velocity field $\mathbf{u(x)}$. The form of the
expansion is
\begin{equation}
 \mathbf{u(x)} = \sum_{m_{y}=0}^{M_{y}-1}\sum_{m_{x}=0}^{M_{x}-1}\sum_{m_{z}=0}^{M_{z}-1}
 {\mathbf{\hat{u}}_{m_{x},m_{y},m_{z}} \bar{T}_{m_{y}}(y)e^{2\pi i(k_{x}x/L_{x} + k_{z}z/L_{z})}
 + \text{\small{(c.c.)}}}
\label{eqn:spectralsum}
 \end{equation}

  The
 $\bar{T}(y)$'s are Chebyshev polynomials defined on the interval [a,b] (in
 most cases [-1,1]). For a given velocity field expansion, the upper bounds on the sums are known
 from the geometry, and the $k$'s are related to the $m$'s through
 the following relations:
 \beq k_{x} = \left \{ \begin{array}{l}
m_{x} \hspace{20 mm} 0 \leq m_{x} \leq M_{x}/2   \\
m_{x} - M_{x} \hspace{10 mm} M_{x} < m_{x} < M_{x}  \\
\end{array}  \right.
\eeq \beq k_{z} = m_{z} \hspace{10 mm} 0 \leq m_{z} < M_{z}
\,.
\eeq
Hence, with a knowledge of the spectral coefficients we can
compute $\mathbf{u(x)}$ by evaluating
this sum at a particular $\bx = (x,y,z)$.

Various internal functions within {\tt Channelflow.org} have been written to
compute $\bu$ on a set of gridpoints. It is possible, by
interpolation of the velocity fields on these gridpoint values, to
integrate a trajectory with great computational speed. However this
will not be nearly as accurate as evaluating the sum
\refeq{eqn:spectralsum} directly. So we evaluate
\refeq{eqn:spectralsum} to give the exact velocity field at every
point along a trajectory. We are able to perform these computations in Matlab with enough speed to compute many tracer particle trajectories within an equilibrium velocity for an adequate length of time to study the flow dynamics.  The code has been
checked to be correct by picking an $(x,y,z)$ coordinate that
\emph{happens} to lie on a gridpoint value and then comparing the
result to the value given by the internal {\tt Channelflow.org} functions. 




\subsection{Symmetries of plane Couette flow}
\label{PCF_symm}

As part of our theoretical analysis of trajectories of fluid particles within an equilibrium velocity field, it will be critical to use and understand the symmetries involved in the special geometry of plane Couette flow. Thus we take a quick detour to discuss these symmetries from a group-theoretic perspective. We focus on the symmetries relevant to the equilibria studied in this work; additional details are provided in \cite{HalcrowThesis}.


\PCf\ is invariant under two reflections $\sigma_1,\sigma_2$ and a
continuous two-parameter group of translations $\tau(\shift_x, \shift_z)$:
\begin{align}
\sigma_1 \, [u,v,w](x,y,z) &= [u, v,-w](x,y,-z) \nnu \\
\sigma_2 \, [u,v,w](x,y,z) &= [-u,-v,w](-x,-y,z)  \label{reflSfit1}\\
\tau(\shift_x, \shift_z)[u,v,w](x,y,z) &= [u,v,w](x+\shift_x,y,z+\shift_z) \nnu\,.
\end{align}
The \NSe s and boundary conditions are invariant for any symmetry $s$
in the group generated by these elements:
$\partial (s \bu) / \partial t = s (\partial \bu / \partial t)$.

The plane Couette symmetries can be interpreted geometrically in the space of
fluid velocity fields. Let $\bbU$ be the space of
square-integrable, real-valued velocity fields that satisfy the kinematic
conditions of \pCf:
\begin{align}
 \bbU  &= \{\bu \in L^2(\Omega) \; | \; \grad \cdot \bu = 0,
               \; \bu(x, \pm 1, z) = 0, \notag  \\
         &\phantom{=} {} \qquad \qquad \qquad \; \; %\text{and }
          \bu(x, y, z) = \bu(x+L_x, y, z) = \bu(x, y, z + L_z)\}  \,.
\end{align} 
The continuous symmetry $\tau(\shift_x, \shift_z)$ maps each state
$\bu \in \bbU$ to a $2D$ torus of states with identical dynamic
behavior. This torus in turn is mapped to four equivalent tori by
the subgroup $\{1,\sigma_1,\sigma_2, \sigma_1 \sigma_2\}$. In
general a given state in $\bbU$ has four $2D$ tori of dynamically
equivalent states.

Most of the Eulerian \eqva\ that are currently known for \pCf\
are invariant under the `shift-reflect' symmetry
$s_1 = \tau(L_x/2,0) \, \sigma_1$ and the `shift-rotate' symmetry
$s_2 = \tau(L_x/2,L_z/2) \, \sigma_2$.  These symmetries form a group
\beq
S = \{1, s_1, s_2, s_3\}, \qquad s_3 = s_1 s_2, 
\eeq

which is isomorphic to
the Abelian dihedral group $D_2$, and is a subgroup of a larger group generated by plane Couette symmetries. The group acts on velocity fields
as:
\begin{align}
s_1 \, [u, v, w](x,y,z) &= [u, v, -w](x+L_x/2,\, y,\, -z) \nnu \\ 
s_2 \, [u, v, w](x,y,z) &= [-u, -v, w](-x+L_x/2,\,-y,\,z+L_z/2) \label{shiftRot} \\
s_3 \, [u, v, w](x,y,z) &= [-u,-v,-w](-x,\, -y,\, -z+L_z/2)  \nnu 
\,
\end{align}


We denote the $S$-invariant subspace of states invariant under
symmetries \refeq{shiftRot} by
\begin{align}
\bbUsymm  &= \{\bu \in \bbU  \: | \;
              s_j \bu = \bu\,, \;\;  s_j \in S \}
              % \bu = \frac{1}{4} (1 + s_1 + s_2 + s_3)\,\bu \}
\,,
\label{symmSubspU}
\end{align}

where $ \bbUsymm \subset \bbU$.
%
$\bbUsymm$ is a flow-invariant subspaces: states initiated
in it remain there under the Navier-Stokes dynamics.


Translations of half the cell length in the spanwise and/or streamwise
directions commute with $S$. These operators generate a discrete
subgroup of the continuous translational symmetry group $SO(2) \times
SO(2)$ :
\beq
T = \{e,\tau_x,\tau_z,\tau_{xz}\}
    \,,\qquad
    \tau_x = \tau(L_x/2,0)
    \,,\;
    \tau_z = \tau(0,L_z/2)
    \,,\;
    \tau_{xz} = \tau_x \tau_z
\,.
\ee{tauD2}
Since the action of $T$ commutes with that of $S$,
% the $T$-group orbit of each velocity field $\bu \in \bbUsymm$
% is also contained in $\bbUsymm$.
% More explicitly:
the three half-cell translations $\tau_x \bu, \, \tau_z \bu,$ and
$\tau_{xz} \bu$ of $\bu \in \bbUsymm$ are also in $\bbUsymm$.


We know that the equilibria  $EQ_1$-$EQ_8$ are symmetric in $S$ because they satisfy those symmetries numerically. There is no
a priori reason that the equilibria should be $S$-symmetric, other than $S$ symmetry
fixes $x,z$ phase and so rules out relative equilibria. But $s_3$ symmetry
alone does the same, and a few equilibria are known that have $s_3$ symmetry
but neither $s_1$ nor $s_2$ symmetry. There are equilibria with other symmetries
that fix $x,z$ phase but have other translations than the half-cell shifts.

It is also possible to form other isotropy subgroups from the plane Couette symmetries $\tau_x$, $\tau_z$, $\sigma_1$, $\sigma_2$. These elements generate a group $G$ of order 16, of which there are various subgroups of possible orders $\{1,2,4,8,16\}$. It is known that other equilibria posses different symmetries, corresponding to different subgroups of $G$. For example, for equilibrium $EQ_8$, we find there is symmetry under an invariance group of order 8, denoted $S_8$, that is isomorphic to the dihedral group $D_4$. 

\beq
S_8 = \{e, s1, s2, s3, s4, s5, s6, s7\}
\eeq
where $s_4 = \tau_z \, \sigma_1$, $s_5 = s_4 s_2$, $s_6 = \tau_x \tau_z$, $s_7 = \sigma_2$. The action of these additional symmetries of $S_8$ on velocity fields is:
\begin{align}
s_4 \, [u, v, w](x,y,z) &= [u, v, -w](x,\, y,\, -z + L_z/2) \nnu \\ 
s_5 \, [u, v, w](x,y,z) &= [-u, -v, -w](-x+L_x/2,\,-y,\,-z) \label{S_8} \\
s_6 \, [u, v, w](x,y,z) &= [u,v,w](x+L_x/2,\, y,\, z+L_z/2)  \nnu  \\
s_7 \, [u, v, w](x,y,z) &= [-u,-v,w](-x,\, -y,\, z)  \nnu 
\,
\end{align}

Which symmetries happen to exist for the different equilibria will have important implications for studying the dynamics of the flow.

\subsection{Symmetry and Stagnation Points}
\label{sec:symm_stag}



From the form of $s_3$ in \refeq{shiftRot}, we can see that any Eulerian equilibrium that
is invariant under $S$ has 4 Lagrangian \stagp s at which the velocity is 0,
which satisfy the condition:
\begin{equation}
 (x,y,z) = (-x, -y, -z+L_z / 2) \label{shiftRot_eqva}
\end{equation}
There are 4 points which satisfy this constraint:
\bea
  \bold{x}_{_{SP_{1}}} &=& (L_x/2,0,L_z/4) \continue
  \bold{x}_{_{SP_{2}}} &=& (L_x/2,0,3L_z/4) \continue
  \bold{x}_{_{SP_{3}}} &=& (0,0,L_z/4) \label{s3lagrange} \\
  \bold{x}_{_{SP_{4}}} &=& (0,0,3L_z/4) \nnu
 \,.
\eea

We refer to these as stagnation points $SP_1$-$SP_4$.
Due to the periodic boundary conditions, we equivalently have
 $(L_x,0,L_z/4)=SP_3$ and $(L_x,0,3L_z/4)=SP_4$.
Also of note is the fact that there can exist no $s_3$-invariant \reqva, since
$s_3$ operation flips both the $x$ and $z$ axes. These stagnation points will exist in all of the equilibria with $S$-symmetry. Additionally, for an equilibrium such as $EQ_8$ which possesses $S_8$ symmetry, from the action of $s_5$ in \refeq{S_8}, we will find stagnation points wherever 
\begin{equation}
 (x,y,z) = (-x+L_x/2, -y, -z) \label{second_condition}
\end{equation}
which gives the additional points:
\bea
  \bold{x}_{_{SP_{5}}} &=& (L_x/4,0,0) \continue
  \bold{x}_{_{SP_{6}}} &=& (3L_x/4,0,0) \continue
  \bold{x}_{_{SP_{7}}} &=& (L_x/4,0,L_z/2) \label{s3lagrange} \\
  \bold{x}_{_{SP_{8}}} &=& (3L_x/4,0,L_z/2) \nnu
 \,.
\eea



In fact, we can generalize the discussion.
Looking at the way the plane Couette symmetries act on velocity
fields in \refeq{reflSfit1},
we see that since $\tau$ does not affect the velocity components,
the condition needed to produce a stagnation point (in which all three velocity components are negated at some shifted position) will work only for the combinations of these
elements which contain both $\sigma_{1}$ and $\sigma_{2}$ an odd
number of times. Within the group $G$ of order 16 of plane Couette
symmetries generated by $\sigma_{1}$, $\sigma_{2}$, $\tau_{x}$,
$\tau_{z}$, the requirement means we just have to identify elements that have a $\sigma_{1}\sigma_{2}$ term. 

There are in fact four such elements of $G$ that contain a
$\sigma_{1}\sigma_{2}$ term. We denote these as $g_1 = \sigma_{1}\sigma_{2}$,
$g_2 = \sigma_{1}\sigma_{2}\tau_{x}$, $g_3 =
\sigma_{1}\sigma_{2}\tau_{z}$, and $g_4 = \sigma_{1}\sigma_{2}\tau_x
\tau_z$. 
\begin{align}
g_1 \, [u,v,w](x,y,z) &= [-u,-v,-w](-x,-y,-z)  \\
g_2 \, [u,v,w](x,y,z) &= [-u,-v,-w](-x+L_{x}/2,-y,-z)  \\
g_3 \, [u,v,w](x,y,z) &= [-u,-v,-w](-x,-y,-z+L_{z}/2)  \\
g_4 \, [u,v,w](x,y,z) &= [-u,-v,-w](-x+L_{x}/2,-y,-z+L_{z}/2)
\end{align}


Different isotropy subgroups of $G$ may or may not contain a symmetry which corresponds to one of these $g_1$-$g_4$ elements, however any $g_i$ that is part of an invariance group for an equilibrium implies the existence of four symmetrically-located \stagp s in the
$y = 0$ plane. Note that $g_3$ and $g_2$ are the elements
already  discussed that produce $SP_1$-$SP_8$.

 Any equilibrium with $g_1$ symmetry implies that there would additionally be \stagp s at $(0,0,0)$, $(L_{x}/2,0,0)$, $(0,0,L_{z}/2)$, and
$(L_{x}/2,0,L_{z}/2)$. And similarly, $g_4$ symmetry implies the existence of
\stagp s at $(L_{x}/4,0,L_{z}/4)$, $(L_{x}/4,0,3L_{z}/4)$,
$(3L_{x}/4,0,L_{z}/4)$, and $(3L_{x}/4,0,3L_{z}/4)$. The set of all possible stagnation points based on various \pCf\ symmetries
 is shown in \reffig{eltonFig:stags7_26}.

So the question of existence of \stagp s for a given equilibrium is, which of the
$g_i$ symmetries does that equilibrium possess? This is a question related to
invariance under the isotropy subgroups. Of importance, this does not
address the question of whether \textit{other} nontrivial \stagp s may exist that are not based on symmetry arguments alone. For the known equilibria of plane Couette flow $EQ_1$-$EQ_{11}$, all of
them have $g_3$ symmetry and $EQ_7$, $EQ_8$ additionally have $g_2$ symmetry. This is likely related to the fact that searches for
equilibria were done in a symmetric subspace which contained the
$g_3$ elements (the $S$-symmetric subspace). \\


\begin{figure}[!h]
\includegraphics[width=1.0\textwidth]{stags7_26.jpg}
  \caption{
   Sets of possible \stagp s. If one of the $g_i$ symmetries is
   possessed, the velocity field will have \stagp s of the color
   corresponding to that symmetry.
   }
  \label{eltonFig:stags7_26}
 \end{figure}



\subsection{Proof that any new \stagp\ must have a partner, symmetric about one of the previously known stagnation points}

Though our symmetry arguments do not determine whether or not there may exist \textit{additional} stagnation points which are not forced by the $g_i$ symmetries in the preceding section, we can in fact that show that for equilibria which exist in one of the flow-invariant subspaces that contains a $g_i$-symmetry (for example, $S$ has $g_3$ symmetry and $S_8$ has both $g_2$ and $g_3$ symmetry), any additional nontrivial stagnation points that exist must occur in symmetric pairs centered around the other known stagnation points.

Consider one of the equilibria in the $S$-invariant subspace, such as $EQ_2$. Again, the
 action of $s_3 \in S$ on velocity fields gives:
 \beq    s_3 \, [u, v, w](x,y,z) = [-u,-v,-w](-x,\, -y,\, -z+L_z/2)\nnu\, .
 \eeq
 If $(x_{_{SP}},y_{_{SP}},z_{_{SP}})$ is a \stagp, $[u, v,
 w](x_{_{SP}},y_{_{SP}},z_{_{SP}}) = [0,0,0]$, then
 \begin{align} s_3 \, [u, v, w](x_{_{SP}},y_{_{SP}},z_{_{SP}}) &= [-u,-v,-w](-x_{_{SP}},\, -y_{_{SP}},\, -z_{_{SP}}+L_z/2) \nnu\, \\
 &= [0,0,0](-x_{_{SP}},\, -y_{_{SP}},\, -z_{_{SP}}+L_z/2) .
 \end{align}
 Thus $(-x_{_{SP}},\, -y_{_{SP}},\, -z_{_{SP}}+L_z/2)$ is also a \stagp.


 \noindent We may parameterize a line passing through two points $(x_{1}, y_{1}, z_{1}),(x_{2}, y_{2}, z_{2})$
 as
 \begin{align}
  x &= x_{1} + (x_{2} - x_{1})t \\
  y &= y_{1} + (y_{2} - y_{1})t \\
  z &= z_{1} + (z_{2} - z_{1})t \\
  t &\in (-\infty,\infty)
 \end{align}
 
 Using the two stagnation points $(x_{_{SP}},y_{_{SP}},z_{_{SP}})$ and $(-x_{_{SP}},-y_{_{SP}},-z_{_{SP}} + L_z/2)$ this becomes
 
 \begin{align}
  x &= x_{_{SP}}(1-2t) \\
  y &= y_{_{SP}}(1-2t) \\
  z &= z_{_{SP}}(1-2t) + \frac{L_{z}}{2} t
 \end{align}
 When $t = 1/2$ this system returns $(x,y,z) = (0,0,L_{z}/4)$, showing
 that $SP_3$ lies on the line between these two \stagp s, halfway
 in between them.

 If we invoke the box periodicities: $x = x + L_{x}$, $z = z +
 L_{z}$, it is easy to show that this pair of stagnation points is also symmetric
 about any of $SP_1$-$SP_4$. For example, \\

 \noindent$\mathbf{x = x + L_{x}}$:

 \noindent $(x_{_{SP}},y_{_{SP}},z_{_{SP}})$ is a \stagp\ $\Rightarrow$
 $(-x_{_{SP}}+L_{x},-y_{_{SP}},z_{_{SP}}+L_{z}/2)$ a \stagp.
 \begin{align}
  x &= x_{_{SP}}(1-2t) + L_{x}t \\
  y &= y_{_{SP}}(1-2t) \\
  z &= z_{_{SP}}(1-2t) + \frac{L_{z}}{2} t
 \end{align}
 When $t = 1/2$ this returns $(x,y,z) = (L_{x}/2,0,L_{z}/4)$, so that the new stagnation
 points lie symmetrically on a line passing through $SP_1$. 

 For an equilibrium invariant under $S_8$, such as $EQ_8$, existence of any additional nontrivial stagnation point will then imply \textit{two} additional stagnation points, based on the action of $g_2$ and $g_3$.
 If $(x_{_{SP}},y_{_{SP}},z_{_{SP}})$ is a \stagp, then  $(-x_{_{SP}},\, -y_{_{SP}},\, -z_{_{SP}}+L_z/2)$ and $(-x_{_{SP}} + L_x/2,\, -y_{_{SP}},\, -z_{_{SP}})$ are also \stagp s. 

 We will investigate numerical methods to determine the possible existence of any such nontrivial stagnation points. In fact for $EQ_2$, as we show in the next section, we do find such a point and it's symmetric partner. These additional stagnation points are critical for understanding the flow dynamics in the equilibrium field, as their stable and unstable manifolds provide us with an outline of the overall dynamics.
 

 
 



\section{\centering Lagrangian Dynamics}
\label{sec:Lagrangian}


We know of the existence of  \stagp s in the flow of an equilibrium velocity field
predicted from the symmetries of \pCf. Thus the starting point for our investigation is clear; treating an equilibrium velocity field as an autonomous dynamical system we have already identified the "fixed points" of the system, which we refer to in this context as the \stagp s.  Using the sum formula for computing velocities at
any point in the \pCf\ domain \refeq{eqn:spectralsum}, by differentiating this formula it is a simple
matter to compute the $[3\!\times\! 3]$ {\velgradmat}
at any point. Eigenvalues and eigenvectors of this matrix will
provide linear stability analysis results for the stagnation points, and allow us to compute and visualize the stable
and unstable manifolds by starting a collection of tracer points along the directions of the eigenvectors. 


 In order to investigate additional locations in the domain for which no movement occurs, we may numerically compute $|\bu|^{2}$ along a fine
grid and try to ascertain regions where the velocity value falls below a given threshold. Then,
using interpolation within these regions, any additional  \stagp s can be
pinned down. 

With the determination of the stagnation points and their invariant manifolds, we find a natural way to view the physical space of the fluid, partitioned into regions wherein the dynamics is dominated by the trajectories following closely to the manifolds themselves. This provides us with a framework for studying how transport may occur within and between the different regions.






\subsection{A colorful portrait of the Upper Branch equilibrium}
\label{sec:eq2}


Our analysis is carried out for the 
Upper Branch \eqv\ velocity field, $EQ_2$, at $\Reynolds = 400$.
The cell size parameters are \beq   [L_x,2,L_z]
         = \; [2\pi/1.14,2,4\pi/5]
         ~ [5.512,2,2.513].
\label{cellW03}
\eeq

To begin, we look at the evolution of Lagrangian tracers starting on a grid of points, shown in \reffig{eltonFig:UBs}. The grid
is chosen to lie in the $[y,z]$ plane, centered at $x = L_x/2$. The initial
points are equally spaced, and offset by one position from the edge
of the box. If the number of points is chosen to be one less than a
multiple of 4, there will be points starting at $\bold{x}_{_{SP_{1}}}=(L_x/2,0,L_z/4)$ and
$\bold{x}_{_{SP_{2}}}=(L_x/2,0,3L_z/4)$. The
trajectories are integrated and run for a relatively short time. Here, the tracers are shown for $\bu$ defined as the difference from laminar flow. Just from evolving the grid of points alone, we begin to get a feel for the dynamics and start to see the formation of interesting patterns and vortical structures.




$EQ_2$ invariance under the symmetry group $S$, explained  in
\refsect{sec:symm_stag}, implies the existence of 4 \stagp s $SP_1$-$SP_4$,
\refeq{s3lagrange}.
In \reffig{eltonFig:UBs_b} the view
from \reffig{eltonFig:UBs_a} has been rotated in order
to reveal these \stagp s. The visualization of the behavior of trajectories near these
fixed points reveals their  qualitative nature.
The point at $3L_z/4$ in \reffig{eltonFig:UBs_b} appears to be an
unstable spiral, whereas the point at $L_z/4$ is hyperbolic. In order to verify these hypotheses, eigenvalues and
stable/unstable manifolds for these \stagp s are computed. \\


\begin{figure}[!h]
\centering
    \begin{subfigure}{0.98\textwidth}
    \includegraphics[width=1.0\textwidth]{fig_UB1.jpg}
      \caption{
        $3D$ perspective view
       }
      \label{eltonFig:UBs_a}
    \end{subfigure}

    \begin{subfigure}{0.98\textwidth}
    \includegraphics[width=1.0\textwidth]{fig_UB1eq.jpg}
      \caption{
        Rotated to show the 2 \stagp s
       }
      \label{eltonFig:UBs_b}
    \end{subfigure}
    \caption{Grid of $19 \times 19$  initial points in the $[y,z]$ plane,
centered at $x = L_x/2$; integrated for 15 time units to produce tracer particle trajectories for $EQ_2$.}
\label{eltonFig:UBs}
 \end{figure}






\noindent \textbf{Linearization and Stability} \\


For a perturbation $\delta$\bx\ away from one of the stagnation points,
the change in the velocity field is given by $\delta\bu = \Mvar
\delta\bx$ where $\Mvar$ is the nine component \velgradmat\ defined
by $\Mvar_{ij}=\frac{\partial u_{i}}{\partial x_{j}}$. Since \bu\ is
given by \refeq{eqn:spectralsum}, it is a relatively simple
extension of this formula to evaluate these partials. To find
$\partial\bu/\partial y$, one needs to use the relation
$\frac{\partial}{\partial y}T_{n}(y) = n U_{n-1}(y)$ where $T_{n}$
is the $n$th Chebyshev polynomial of the first kind and $U_{n}$ is
the $n$th Chebyshev polynomial of the second kind. Everything else
is straightforward.
The eigenvalues of $\Mvar$, evaluated at a stagnation point, tell us local stability
and reveal the qualitative nature of the motion nearby the \stagp.
For the \stagp s $SP_1$ - $SP_4$, the eigenvalues, eigenvectors,
and velocity gradients matrices are as follows. \\

$\bold{x}_{_{SP_{1}}}=(L_x/2,0,L_z/4)$: There are 3 real eigenvalues, two positive and one
negative.
\begin{align}
&\eigExp[1] = -0.4652099 \,,\quad
\jEigvec[1] =
\begin{pmatrix}
             {0.9844417} \cr
             {0.1743315} \cr
             {0.0219779} \cr
   \end{pmatrix} \\
    &\eigExp[2] = 0.4008961 \,,\quad \jEigvec[2] =
\begin{pmatrix}
             {0.5704000} \cr
             {-0.7666749} \cr
             {0.2947091} \cr
   \end{pmatrix} \\
    &\eigExp[3] = 0.0643139 \,,\quad \jEigvec[3] =
\begin{pmatrix}
             {0.4082166} \cr
             {0.7525949} \cr
             {0.5166819} \cr
   \end{pmatrix} \end{align}
   The \velgradmat\ is
\beq
   \Mvar =
   \begin{pmatrix}
   {-0.4305385} &  {-0.3002042} &{0.8282447} \cr
   {-0.1221356} &   {0.2456107} & {-0.1675796} \cr
   {0.0001651}  &   {-0.0828951}  & {0.1849278} \cr
            \end{pmatrix}
\eeq
    The point is a saddle; it has 1 stable dimension and a $2D$ plane
    of instability spanned by $\jEigvec[2]$ and $\jEigvec[3]$, with
    the eigenvalues summing to 0, as required by a volume-preserving flow.
    
     The \stagp\ $SP_4$ at
    $(0,0,3L_z/4)$ has the same eigenvalues as for $SP_1$. It's
    eigenvectors and \velgradmat\ differ by a minus sign
    in the third component (except for $\Mvar_{33}$ where the two minuses
    cancel). \\

$\bold{x}_{_{SP_{2}}}=(L_x/2,0,3L_z/4)$: There is one real, negative eigenvalue and a complex
pair with positive real part.

\begin{align}
&\eigExp[1] = -0.0352362 \,,\quad \jEigvec[1] =
\begin{pmatrix}
             {-0.9452459} \cr
             {-0.1893368} \cr
             {-0.2658228} \cr
   \end{pmatrix}
   \\
&\eigRe[2] \pm i\,\eigIm[2] = 0.0176181 \pm i\,0.0862176
   \\
&\jEigvec[2] =
\begin{pmatrix}
             {0.3737950 + 0.0544113i} \cr
             {0.2098940 - 0.4925773i} \cr
             {0.7554000} \cr
   \end{pmatrix}
\,,\quad
\jEigvec[3] =
\begin{pmatrix}
             {0.3737950 - 0.0544113i} \cr
             {0.2098940 + 0.4925773i} \cr
             {0.7554000} \cr
   \end{pmatrix}
\nnu\,.
\end{align}
The \velgradmat\ is \beq
   \Mvar =
   \begin{pmatrix}
   {-0.0316935} & {-0.0708737} &  {0.0378835} \cr
  {-0.0250579} & {-0.0218884} &  {0.0795969} \cr
   {0.0014742} & {-0.1320575} &  {0.0535818} \cr
   \end{pmatrix}
                    \eeq

    This \stagp\ spirals out in a plane given by the complex pair of
    eigenvectors. It is stable in one dimension that points primarily
    along the $x$ direction. 
    
    $SP_3$
 at $(0,0,L_z/4)$ has the same eigenvalues as $SP_2$ and again, the
    \velgradmat\ is the same except for sign changes in
    the third component. This follows from the plane Couette symmetries. \\


\noindent \textbf{Additional Stagnation Points} \\


Having analyzed stagnation points $SP_1$-$SP_4$, before further investigating the dynamics, it is natural to wonder whether other such stagnation points may exist that do not necessarily follow from a symmetry argument. To answer this question, as mentioned above, we numerically compute $|\bu|^{2}$ along a fine
grid and look for where it's value falls below a given threshold. 

We create a more refined grid of velocities which
  is $144 \times 105 \times 144$. This is three times the 48 $\times$ 35
  $\times$ 48 grid in each dimension used to show the initial tracer trajectories, and contains about 2.2 million
  points. At each point $|\bu|^{2}$ is then calculated and at
  every point that satisfies $|\bu|^{2} < \epsilon$ for some
  arbitrarily chosen $\epsilon$, the point is plotted.

In \reffig{eltonFig:fine_usquare} we show regions in the cell where $|\bu|^{2}$ is very small for $\epsilon = 10^{-4}$, notated by the globs of blue dots. The trajectories shown along with the points of small velocity in this figure, explained below, are also suggestive of the existence of a stagnation point within the spiraling region. The four previously known stagnation points are identified in the figure, but we also see a couple of additional clumps. Honing in one of the suspicious clusters,
starting from the gridpoint value with smallest velocity in the suspicious region,
$\bx_{0} \approx (2.33476, 0.40952, 0.64577)$, and its reflection through
$\bold{x}_{_{SP_{1}}}$, $\bx_{0}' =2 \bold{x}_{_{SP_{1}}} - \bx_{0}$, the Newton iteration
 \beq
 \bx_{k+1} = \bx_{k} -
          {\Mvar}^{-1}(\bx_{k}) \, \bu(\bx_{k})
 \eeq
%  where $\Mvar$ is the \velgradmat.
converges rapidly to verify \textit{another} pair of \stagp s. Because we have already used notation to define points $SP_1$-$SP_8$ in \refsect{sec:symm_stag}, we refer to these new numerically discovered stagnation points as $SP_{N1}$ and $SP_{N2}$: 

\begin{align}
&\bold{x}_{_{SP_{N1}}} =(2.35105561774981,0.42293662349708,0.65200166068573)
\\
&\bold{x}_{_{SP_{N2}}}=(3.16051044117966,-0.42293662349708,0.60463540075018)
\label{eqn:newspNewt}
\,.
\end{align}
%         = [5.51156605892946182182,2,2.51327412287183459075]
We see the
 symmetry in the $y$-component of this pair, and in fact
these points are shown to be
 symmetric about the point $SP_1$, as discussed in \refsect{sec:symm_stag}:
 \beq
    (\bold{x}_{_{SP_{N1}}} +\bold{x}_{_{SP_{N2}}})/2 = \bold{x}_{_{SP_{1}}}
 \,.
 \eeq

 

  \begin{center}
\begin{figure}[!h]
\includegraphics[width=1.0\textwidth]{fine_usquare.jpg}
  \caption{
   Blue clumps of points indicate where the velocity for $EQ_2$ is very close to zero.
   Shown along with the stable manifold
   of $SP_3$ and the unstable manifold of $SP_1$.
          }
  \label{eltonFig:fine_usquare}
 \end{figure}
\end{center}



 Repeating the linear stability analysis for $SP_{N1}$ and $SP_{N2}$: There is one real, positive eigenvalue
 and a complex pair with negative real part.

  \begin{align} &\eigExp[1] = 0.1453207 \,,\quad \jEigvec[1] =
\begin{pmatrix}
             {0.9307982} \cr
             {0.3502306} \cr
             {0.1046576} \cr
   \end{pmatrix}
   \\
&\{ \eigExp[2],\eigExp[3]\}
  = \eigRe[2] \pm i \,\eigIm[2] =  -0.0726603 \pm i\, 0.3733478
   \nnu\\
&\jEigvec[2] =
\begin{pmatrix}
             {~0.5226203} \cr
             {-0.6703938} \cr
             {~0.2065610} \cr
   \end{pmatrix}
    \,,\quad
\jEigvec[3] =
\begin{pmatrix}
             {~0.3779843} \cr
             {~
             0} \cr
             {- 0.3031510} \cr
   \end{pmatrix}
\,.
\end{align}
The \velgradmat\ is
\beq
   {\Mvar} =
   \begin{pmatrix}
   {0.0225166} &  {0.0985763} &{0.7623083} \cr
   {0.1714566} &   {-0.1275193} & {-0.6118476} \cr
   {-0.0615378}  &   {0.1755954}  & {0.1050028} \cr
            \end{pmatrix}
\,.
\eeq

We have this time a $1D$ unstable manifold and a $2D$ spiraling stable
manifold. The trajectories shown in \reffig{eltonFig:fine_usquare}, which originate close to $SP_1$ and $SP3$, wander close to the spiraling stable manifold of the numerically discovered $SP_{N1}$, showing how the dynamics tends to be dominated by these stagnation points.

 \begin{figure}[!h]
\includegraphics[width=1.0\textwidth]{stagps_edited.jpg}
  \caption{
   The 6 unique \stagp s within one periodic box for $EQ_2$. $SP_1$-$SP_4$ are guaranteed by $EQ_2$ symmetries, $SP_{N1}$ and $SP_{N2}$ are determined numerically. 
   }
  \label{eltonFig:stagps_label}
 \end{figure}

 \begin{figure}[!h]
\includegraphics[width=1.0\textwidth]{stagps2_edited.jpg}
  \caption{
   The 4 \stagp s that occur within the domain $\Omega$.
   }
  \label{eltonFig:stagps_label2}
 \end{figure}

   We have been describing all \stagp s which are
   inside a single periodic cell with dimensions $L_x \times 2 \times L_z$, pictured in \reffig{eltonFig:stagps_label}. However even within this cell
   there is a redundancy in labeling all of these points as
   distinct. 
   The interesting dynamics and connections between the different \stagp s occur
 along the $x$ direction. To understand what is happening one needs
 to look only at a subset of these \stagp s that lies in the right or left half of the box, that
 is, in the interval $[0,L_{z}/2]$ or the interval $[L_{z}/2,L_{z}]$. We have chosen
 the interval $[0,L_{z}/2]$. In
 the $x$ direction the most convenient interval is not actually
 $[0,L_{x}]$, rather we look at the \stagp s in the open interval
 $(-L_{x}/2,L_{x})$, open so as to ignore the repeated translations on the boundary. Thus an alternate domain of investigation that will be conventient to sometimes use is
 \beq \Omega = (-L_{x}/2,L_{x}) \times [-1,1] \times [0,L_{z}/2].
 \eeq
 Within this domain $\Omega$ there are then just four \stagp s. They are $SP_1$, $SP_3$, $SP_{N1}$, and
 $SP_{N2}$, shown in
  \reffig{eltonFig:stagps_label2}. Note that $SP_{N2}$ is a
 translated version from the way it was viewed in
 \reffig{eltonFig:stagps_label}. The phase portrait of fundamental dynamics for $EQ_2$ will be viewed in $\Omega$.  \\ 
   
   


\noindent \textbf{Phase portrait and heteroclinic connections} \\


With identification of all of the stagnation points within either the original periodic box or the cell $\Omega$, as well as the corresponding linear stability analysis, we are ready to make a complete phase space portrait for the Upper Branch, $EQ_2$. \\

\begin{figure}[!h]
\includegraphics[width=1.0\textwidth]{manifolds_both.jpg}
  \caption{
   Segments of the stable and unstable manifolds of the \stagp s
   $\bold{x}_{_{SP_{1}}} = (L_x/2,0,L_z/4)$ and
   $\bold{x}_{_{SP_{2}}} = (L_x/2,0,3L_z/4)$. Black and green are unstable.
   }
  \label{eltonFig:manifolds_both}
 \end{figure}


    \begin{figure}[!h]
\includegraphics[width=1.0\textwidth]{man14_june3.jpg}
  \caption{
   Heteroclinic connections of the Upper Branch (red trajectories) from $SP_{N1}$ -> $SP_3$ and $SP_{N2}$ -> $SP_3$, shown in a cell with $x \in$ [-$L_x/2$, $L_x/2$] along with the unstable manifold of $SP_3$.
   }
  \label{eltonFig:hetero1}
 \end{figure}




  \begin{figure}[!h]
\includegraphics[width=1.1\textwidth]{june4_fig7.jpg}
  \caption{
   Portrait of the fundamental dynamics of the \stagp s $SP_1$, $SP_3$, $SP_{N1}$, $SP_{N2}$ within cell $\Omega$ for the Upper Branch.
   }
  \label{eltonFig:hetero2}
 \end{figure}


 


The dynamics between the \stagp s and their translations is
   quite interesting. In \reffig{eltonFig:manifolds_both} we see a partial view of the stable and unstable manifolds of two of the stagnation points, $SP_1$ and $SP_2$, in the original periodic domain, found by integrating trajectories near the fixed points forwards and backwards in time along the stable or unstable eigenvectors. Local stability analysis shows that $SP_1$ has all real eigenvalues
with a $1D$ stable manifold,
and a $2D$ unstable manifold which is locally a plane. $SP_2$
 has a $2D$ unstable manifold with complex eigenvalues which spiral
 out in a plane and a $1D$ stable manifold. 
 
 As alluded to in \reffig{eltonFig:fine_usquare}, $SP_{N1}$ and $SP_{N2}$ sit near the center of the swirl of green coming from the unstable direction of $SP_1$. To better understand what is happening here, referring to \reffig{eltonFig:hetero1}, we compute the  stable and unstable manifolds of $SP_{N1}$ and $SP_{N2}$, where we use the shifted translation of $SP_{N2}$, along with the stable and unstable manifolds of $SP_3$. The blue surface is formed by the overlap of trajectories starting along the unstable manifold of $SP_3$ and the stable manifolds of $SP_{N1}$ and $SP_{N2}$.  We see that the stable manifold of $SP_3$ (shown by the red curves) corresponds with the unstable manifolds of $SP_{N1}$ and $SP_{N2}$, thus we have \textit{heteroclinic connections} from $SP_{N1}$ --> $SP_3$ and $SP_{N2}$ --> $SP_3$!
 The thick appearance of the red curves
 is simply so that they can be seen within the blue surface. They are actually just a single trajectory.
 
 Next we bring $SP_1$ into the picture to see the full dynamical portrait within $\Omega$. $SP_1$ has a $2D$ unstable manifold
 and a $1D$ stable manifold. The result of all of these manifolds
 plotted together is \reffig{eltonFig:hetero2}. Compare to \reffig{eltonFig:stagps_label2} to see the locations of the stagnation points. The relation of the
 stable manifold of $SP_1$ (yellow curve) and the trajectories that are driven away from $SP_1$ in the unstable direction (green)
 to those of the blue surface is quite interesting. These trajectories
 tightly hug the blue surface as they spiral around it, appearing to be shielded from entering the volume it encompasses. This could have significant implications for the consideration of fluid mixing within plane Couette flow, perhaps showing that it is difficult to achieve a uniformly mixed space for this particular equilibrium; a blob of ink that starts outside of the blue surface may have a difficult time ever entering the region!
 
 One merely translates the image in \reffig{eltonFig:hetero2} in the
 $x$ direction by an amount $L_{x}$ to give a complete picture in
 any periodic cell. The same picture will also occur symmetrically
 (translated by $L_{x}/2$ and $L_{z}/2$) in the left half of the
 box.
  
 





   \subsection{Equilibrium $EQ_8$: Additional Symmetries}
   \label{sect:EQ8}

   Having analyzed the Upper Branch equilibrium $EQ_2$, we next look at $EQ_8$, another equilibrium velocity field of plane Couette flow which exhibits turbulent behavior at a lower Reynolds number, 270. \\



\noindent \textbf{Tracer dynamics and analysis} \\


 We start once again with a cleverly chosen grid of initial
 trajectories to get a feel for the significant structures in the
 flow. The grid is
 in a plane at $x = L_{x}/2$. The result, after a short integration
 time, is shown in \reffig{eltonFig:EQ8_grid1}. This perspective
 view already shows us quite a bit of information. Once again we
 have symmetries abound, and we know from the discussion in \refsect{sec:symm_stag} that there will be at least 8 stagnation points $SP_1$-$SP_8$.  Another interesting feature of this
 plot is the four vortical structures on the left half. One final noteworthy point
  from the figure is the appearance of a perfect line segment connecting two of the
 stagnation points, which happen to be $SP_1$ and $SP_2$. 
  This strongly suggests a heteroclinic
 connection between these two \stagp s. To confirm, we
 compute the eigenvalues and eigenvectors of the \velgradmat. For
 $SP_1$, there is
 indeed a real, unstable eigenvector pointing along (0,0,1) and for
 $SP_2$ there is a real, stable eigenvector pointing along (0,0,1).
 This, together with the plot, numerically confirms the existence of the heteroclinic trajectory. The same result  holds for the shifted pair at $x = 0$. The rest of the eigenvalues/eigenvectors are given
 below. We note that for $EQ_8$ there is a heteroclinic connection which is a simple
 horizontal line connecting the pair of trivial \stagp s in the
 \textit{spanwise} direction, whereas for \tUB\ the connection was some
 arbitrary-looking curve in the \textit{streamwise} direction connected to a nontrivial
 \stagp. Factorization of the 
$SP_1$ and $SP_2$ stability eigenspaces for $EQ_8$ occurs because the spanwise $z$ direction is a $1D$ flow-invariant subspace at
the \stagp s \cite{SiCvi10}. That ensures the simplicity of the \hec.

$EQ_8$, $SP_1$: There are two real, positive eigenvalues
 and one real, negative eigenvalue.
\bea
\left(
    \eigExp[1],\eigExp[2],\eigExp[3]
\right) &=&
      (0.363557,0.227831,-0.591389)
\label{E8SP1} \\
\left(
    \jEigvec[1],\jEigvec[2],\jEigvec[3]
\right) &=&
\left(
    \begin{pmatrix}
             {0} \cr
             {0} \cr
             {1}
    \end{pmatrix} \,,
    \begin{pmatrix}
             {-0.733415} \cr
             {-0.679780} \cr
             {0}
    \end{pmatrix} \,,
    \begin{pmatrix}
             {0.991005} \cr
             {0.133824} \cr
             {0}
    \end{pmatrix}
\right) \,.
\nnu
\eea

$EQ_8$, $SP_2$: There are two real, positive eigenvalues
 and one real, negative eigenvalue.
\bea
\left(
    \eigExp[1],\eigExp[2],\eigExp[3]
\right) &=&
      (0.992857,0.255973,-1.248830)
\label{E8SP2} \\
\left(
    \jEigvec[1],\jEigvec[2],\jEigvec[3]
\right) &=&
\left(
    \begin{pmatrix}
             {~0.116961} \cr
             {-0.993136} \cr
             {0}
    \end{pmatrix} \,,
    \begin{pmatrix}
             {0.957795} \cr
             {0.287450} \cr
             {0}
    \end{pmatrix} \,,
    \begin{pmatrix}
             {0} \cr
             {0} \cr
             {1}
    \end{pmatrix}
\right) \,. \\
\nnu
\eea


   \begin{figure}[!h]
\includegraphics[width=0.9\textwidth]{EQ8_grid1.jpg}
  \caption{
    Grid of initial points in the $[y,z]$ plane,
    centered at $x = L_x/2$; integrated to produce tracer particle trajectories for $EQ_8$.
   }
  \label{eltonFig:EQ8_grid1}
 \end{figure}




\noindent \textbf{More symmetries --> More stagnation} \\


Equilibrium $EQ_8$ (as well as $EQ_7$, not discussed here), possesses additional symmetries compared to $EQ_2$. $EQ_2$ is in the $S$-invariant subspace of velocity fields and $EQ_8$ is in $S_8$ (\refsect{PCF_symm} and \refsect{sec:symm_stag}).


From \refeq{second_condition} and \refeq{s3lagrange} we know then that for $EQ_8$ we will have the additional stagnation points:

 \bea
  \bold{x}_{_{SP_{5}}} &=& (L_x/4,0,0) \continue
  \bold{x}_{_{SP_{6}}} &=& (3L_x/4,0,0) \continue
  \bold{x}_{_{SP_{7}}} &=& (L_x/4,0,L_z/2)  \\
  \bold{x}_{_{SP_{8}}} &=& (3L_x/4,0,L_z/2) \nnu
 \,.
\eea
 Interestingly these were actually discovered numerically \textit{before} the symmetry arguments were understood. A Newton search on regions of very low velocity for
$EQ_8$ revealed that $(L_x/4,0,L_z/2)$ and $(3L_x/4,0,L_z/2)$
are \stagp s. From this, one may deduce that symmetry $s_5$ must
hold, and it can then be checked that at any position the velocity
field is indeed invariant under $s_4$ and $s_5$. 

Stability analysis of the additional set of stagnation points for $EQ_8$ gives the
following.

 $SP_5$: There is one real, positive eigenvalue
 and a complex pair with negative real part.
  \begin{align} &\eigExp[1] = 0.03109 \,,\quad \jEigvec[1] =
\begin{pmatrix}
             {0.85275} \cr
             {0.41774} \cr
             {-0.31355} \cr
   \end{pmatrix}
   \\
&\{ \eigExp[2],\eigExp[3]\}
  = \eigRe[2] \pm i \,\eigIm[2] =  -0.01555 \pm i\, 0.59385
   \label{EQSP5eigs}\\
&\jEigvec[2] =
\begin{pmatrix}
             {~0.24762} \cr
             {-0.31442} \cr
             {~0.69906} \cr
   \end{pmatrix}
    \,,\quad
\jEigvec[3] =
\begin{pmatrix}
             {-0.20793} \cr
             {~0.55489} \cr
             {~0} \cr
   \end{pmatrix}
\,.
\end{align}
 We have a $1D$ unstable manifold and a $2D$ inward-spiral
stable manifold. All four of the new points have the same
eigenvalues. $SP_5$ and $SP_8$ have the same eigenvectors, as do $SP_6$
and $SP_7$ whose eigenvectors differ from $SP_5$ only by the sign of
the third component for \jEigvec[1] and by the sign of the first and
second components for \jEigvec[2] and \jEigvec[3].

As a final interesting side consequence of
numerically searching for \stagp s for $EQ_8$, the figures produced by
plotting gridpoints where velocity is small, using a cutoff
value of $|\mathbf{u}|^{2}$ which is too large to actually be useful for
finding \stagp s, we instead find a plot showing more
intricate patterns in the flow. \reffig{eltonFig:usquare_EQ8_1} shows a $3D$
perspective view of these points, and
\reffig{eltonFig:usquare_EQ8_2} 
shows the projection of \reffig{eltonFig:usquare_EQ8_1} onto the $xz$ plane. This volume-preserving flow (area preserving in Poincar\'e sections) may
have invariant tori which, being quasiperiodic,
would not be detected by the stagnation point searching routines. Though the structures in the projection plot in \reffig{eltonFig:usquare_EQ8_2} are not actual tracer trajectories, they are suggestive that a search for such invariant tori in future work may be a fruitful endeavor.  

\begin{figure}[!h]
\centering
    \begin{subfigure}{0.9\textwidth}
    \includegraphics[width=1.0\textwidth]{usquare_EQ8_cute1.jpg}
      \caption{
        Perspective view.
       }
      \label{eltonFig:usquare_EQ8_1}
    \end{subfigure}

    \begin{subfigure}{0.9\textwidth}
    \includegraphics[width=1.0\textwidth]{usquare_EQ8_cute2.jpg}
      \caption{
       Projection onto the $xz$
       plane.
       }
      \label{eltonFig:usquare_EQ8_2}
    \end{subfigure}  
    \caption{
       A plot of points where the velocity field falls below a
       small cutoff for $EQ_8$, showing interesting structures in the flow. \\
       }
    \label{eltonFig:usquare_both}
 \end{figure}







\section{\centering Conclusion}
\label{sec:conclusion}

We have taken a step towards a deeper understanding of the  turbulent fluid flow in a $3D$ system from the Lagrangian perspective by studying tracer trajectory dynamics in plane Couette geometry.
Potential applications that could follow from having an understanding of the Lagrangian dynamics and being able to accurately compute tracer particle trajectories are wide-ranging: computation of velocity profile statistics or correlation functions within different regions, taken over an ensemble of particle trajectories, are within reach. Calculations of mixing time
and diffusion properties for the flow, Lyapunov exponents and material stretching,
striation thickness, among others, are some of the various possible future investigations that could be followed once these turbulent building blocks have been mapped out. By extending the dynamical systems methods that are often confined to simpler $2D$ systems to the $3D$ world of plane Couette flow, we encounter complex coherent structures that partition the physical space of the fluid into regions which exhibit distinct types of motion. Relying on the symmetries of the geometry to shine light upon the situation and guide us, we are able to construct phase portraits for plane Couette equilibria starting by the identification of stagnation or fixed points of the system. Future work could easily extend these analyses to additional equilibria for plane Couette flow, or apply the same methods in other fluid systems which likely posses symmetries.


%%% Acknowledgements %%%
\section{\centering Acknowledgments}
Acknowledge group members not included as authors.









\bibliographystyle{unsrt} %prsty} %apsrev} %plain}
\bibliography{LagrangianArxiv}






\end{document}












