% intro.tex
% $Author: predrag $ $Date: 2009-03-02 12:35:21 -0500 (Mon, 02 Mar 2009) $

% PC                                                    Mar 2 2009
% PC created from y-lan/ks/ks.tex                       Mar 2 2009


% \section{Introduction}
% APS Style: The introductory section between the abstract and
% the "Materials and Methods" section should not be labeled.
% Delete the word "Introduction" if used by the author.
% \label{sec:intro}
%
% PC                                                    Mar 2 2009

Statistical approaches to the study of turbulence\rf{frisch},
rely on assumptions which break
down in presence of large-scale coherent structures typical
of fluid motions\rf{HLBcoh98}.
Description of such coherent structures requires detailed
understanding of the dynamics of underlying equations of
motion.
In E. Hopf's dynamical systems vision\rf{hopf48} turbulence explores a
repertoire of distinguishable patterns; as we watch
a turbulent system evolve,
every so often we catch a glimpse of a familiar whorl. %,
At any instant and a given finite spatial resolution the system
approximately tracks for a finite time a pattern belonging to a finite
alphabet of admissible patterns, and the dynamics can be thought of as a
walk through the space of such patterns, just as chaotic dynamics with a
low dimensional attractor can be thought of as a succession of nearly
periodic (but unstable) motions.

Exploration of Hopf's program close to the onset of
spatiotemporal chaos was initiated in
\refref{ks} which was the first to
extend the periodic orbit theory to a PDE,
the
1-spatial dimension Kuramoto-Sivashinsky\rf{kuturb78,siv} system,
a flow embedded in
an infinite-dimensional \statesp.
Many recurrent patterns were determined
numerically, and the recurrent-patterns theory predictions tested for
several parameter values.
Continuous symmetries of the full periodic domain problem lead to
new important features of dynamics - such as relative periodic orbits -
that merit study on their own\rf{SCD07}. For that reason
both in \refref{ks} and in this paper we found it advantageous to
focus on the dynamics confined to the
antisymmetric subspace, space for which \po s characterize
``turbulent'' dynamics.
In what follows we shall often refer to such periodic orbit solutions of
truncated \KSe\ as ``recurrent patterns'' in order to emphasize their
spatio-temporal periodicity.
In this paper
(and, in a much greater detail, in \refref{lanthe})
we venture into a {\KS} system bigger than the one
studied in \refref{ks}, just large
enough to exhibit ``turbulent'' dynamics arising through
competition of several unstable coherent structures.

Basic properties of the {\KSe} are reviewed in \refsect{sect:ksprop}.
Determining \eqva\ and \po s in high-dimensional \statesp s
opens new challenges, and
in \refsect{sect:ksrecur} we sketch the \descent\ method
that we have developed and deployed
in our searches for recurrent patterns.
Informed by the topology of the flow, the method can
determine even very long \po s, such as the
orbit of \reffig{f:antlong}\,(c).
\Eqva, which play a key role
in organizing the global topology of
\statesp\ dynamics, are investigated
in \refsect{sect:kseqlb}.
We then fix the
size of  {\KS} system  in order to illustrate our
methodology on a concrete example.
Not all {\eqva}
influence the dynamics equally, and in \refsect{sect:sdyneq}
we show how to gauge the relative importance of an \eqv\ by its
proximity to the most recurrent \statesp\ regions.
For this small {\KS} system the dynamics is shaped by
the competition between ``center'' and ``side'' \eqva.
In \refsect{sect:PoincSect} we turn this observation into a
dynamical description of the flow by constructing local,
\eqv-centered Poincar\'{e} sections.
In \refsect{sect:CurvilinRep} we show that with
intrinsic curvilinear coordinates built along unstable manifolds
of \eqva\ and short \po s
(the key observation of \refref{ks}), the dynamics can be reduced
to iteration of low-dimensional
Poincar\'{e} return maps.
The long road from an infinite-dimensional PDE to
essentially 1-dimensional iteration is now completed,
its crowning achievement the bimodal return map of
\reffig{f:antmn1}\,(e).
Such return maps enable us construct
a symbolic dynamics, and initiate a systematic
search for periodic orbits that build up
local Smale horseshoe repellers, as many as
desired.
The periodic points so determined are
overlayed over the return map in \reffig{f:antmn1}\,(f).
Interestingly,
this systematic parsing of {\statesp}
leads to a discovery of a non-trivial
{attracting} \po\ of short period, an orbit highly
unlikely to show up in random initial condition simulations of
\KS\ dynamics.
The hierarchy of \po s so determined can
then be used to predict long-time dynamical averages
via periodic orbit theory.
Our results are summarized in \refsect{sect:sum}.
