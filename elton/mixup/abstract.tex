% reducesymm/elton/Arxiv_paper/abstract.tex
%                      called by LagrangianArxiv.tex

% Predrag   version 0.2                         2024-12-02 
%           edited abstract
% John E.   version 0.1                         2024-11-23                         

% \title{Chaotic mixing in plane Couette turbulence} %% Predrag 2024-12-02
%  {Lagrangian dynamics in plane Couette turbulence} %% John E. 2024-11-23  

        %\begin{abstract}
Lagrangian tracer particle trajectories for invariant solutions 
of the Navier-Stokes equations confined to the three-dimensional geometry of plane 
Couette flow are studied. Treating the Eulerian 
velocity field of an invariant solution as a dynamical system, the transport of these passive 
scalars along Lagrangian flow trajectories reveals a rich repertoire of 
different types of motion that can occur, including stagnation points, 
for which there is no fluid movement, and invariant tori, which obstruct 
chaotic mixing across the full volume of the plane Couette flow minimal 
cell. We determine the stability of these stagnation points, along with 
their stable and unstable manifolds, and find heteroclinic 
connections between them. These topological features produce a skeleton 
that shapes the passive tracer flow for a turbulent fluid, providing a 
first step to elucidating Lagrangian particle transport and mixing in 
three-dimensional Navier-Stokes turbulent flows.

        \ifboyscout{\color{blue}
% PC from mixup09/abstract.tex          Mar 2 2009
We undertake an exploration of Lagrangian mixing in
recurrent patterns in the
...
For a small, but transitionally turbulent system,
the long-time dynamics takes place on a low-dimensional
inertial manifold. A set of equilibria and periodic orbits offers a coarse
partition of this manifold.
...
 The mixing
dynamics appears decomposable into chaotic dynamics within such
local repellers, interspersed by rapid jumps between them.
        } %end\color{blue}
        \fi 
 
        %\end{abstract}

