% summary.tex
% $Author: predrag $ $Date: 2009-03-02 12:35:21 -0500 (Mon, 02 Mar 2009) $

%\section{Summary}
%   \label{sect:sum}
% PC                                                    Mar 2 2009
% PC created from y-lan/ks/ks.tex                       Mar 2 2009

The recurrent patterns program
was first implemented in detail\rf{ks} on the 1-d
Kuramoto-Sivashinsky system at the onset of chaotic dynamics.
For these specific parameter values many recurrent patterns
were determined numerically, and the periodic-orbit theory
predictions tested.
In this paper we venture into a large {\KS} system, just large
enough to exhibit ``turbulent'' dynamics of topologically richer
structure, arising through
competition of several unstable
coherent structures.
Both papers explore dynamics confined to the
antisymmetric subspace, space for which \po s characterize
``turbulent'' dynamics.
\refRef{SCD07} studies {\KS} in the full periodic domain,
where relative periodic orbits due to the continuous translational
symmetry play a key role, and \refref{GHCW07}
applies the lessons learned to a full 3$D$ Navier-Stokes flow.
In this context Kawahara and Kida\rf{KawKida01} have
demonstrated that the recurrent patterns can be determined
in turbulent hydrodynamic flows
by explicitly computing  several important unstable spatio-temporally periodic
solutions in the 3-dimensional plane  Couette turbulence.

We have applied here the ``recurrent pattern program''
to the Kuramoto-Sivashinsky system in a periodic domain,
antisymmetric subspace,
in a larger domain size than explored previously\rf{ks}.
The
\statesp\  non--wandering set for the system of this particular size
appears to consist of three
repelling Smale horseshoes and orbits communicating between them.
Each subregion is characterized by
qualitatively different spatial $u$-profiles in the 1\dmn\ physical space.
The ``recurrent patterns,'' identified in this  investigation by
nearby \eqva\ and \po s, capture well the
\statesp\ geometry and dynamics of the system. Both the
\eqva\ and \po s are efficiently determined by
the {\descent} method. The
\eqva\ so determined, together with their unstable
manifolds, provide the global frame for the non--wandering set.
We utilize these unstable manifolds to build 1\dmn\ curvilinear
coordinates along which the
infinite-dimensional PDE dynamics is well approximated by 1-dimensional
return maps and the associated symbolic dynamics. In principle, these
simple models of dynamics enable us to systematically classify and
search for
recurrent patterns of arbitrary periods.
For the particular examples studied, the approach works well for the
``central'' repeller but not so well for the ``side'' repeller.


Above advances are a proof of principle, first steps in the
direction of implementing the recurrent patterns program.
But there is a large conceptual
gap to bridge between what has been achieved, and what needs to be done:
Even the flame flutter has been probed only in its weakest turbulence
regime, and it is an open question to what extent Hopf's vision remains
viable as such spatio-temporal systems grow larger and more turbulent.
