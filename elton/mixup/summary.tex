% reducesymm/elton/mixup/summary.tex      pdflatex ECHG24; biber ECHG24

\section{Conclusion and perspectives} % as in GHCW07
% or {Summary and future work}
\label{s:summary}

We have taken a step towards a deeper understanding of the  turbulent 
fluid flow in a $3D$ system  by studying 
tracer trajectory dynamics in the Lagrangian frame for {\pC} geometry. Potential 
applications that could follow from having a grasp of the Lagrangian 
dynamics and being able to accurately compute tracer particle 
trajectories are wide-ranging: velocity profile statistics or correlation 
functions taken over an ensemble of particle trajectories within 
different regions, calculations of mixing time and diffusion properties 
for the flow, Lyapunov exponents and material stretching, striation 
thickness, among others, are some of the various possible measures of 
chaotic advection that could be investigated.  

By extending the dynamical systems methods that are often confined to 
simpler $2D$ systems to the $3D$ world of {\pCf}, we encounter complex 
structures that partition the physical space of the fluid into regions 
which exhibit distinct types of motion and allow us to visualize the 
fundamental motions driven by trajectories which lie close to invariant 
manifolds. Relying on the symmetries of the geometry to shine light upon 
the situation and guide us, we are able to construct phase portraits for 
{\pC} Eulerian equilibria starting with the identification and stability 
determination of stagnation or fixed points of the system. As a turbulent 
fluid evolves by visiting equilibria and periodic solutions in a 
recurrent manner, understanding the limitations on mixing between 
different regions of each equilibrium solution provides important 
information for the overall ability of a fluid to become thoroughly 
mixed.  Future work could thus extend these analyses to forming a 
dynamical portrait for all invariant solutions of plane Couette flow, or 
apply the same methods in other fluid systems which likely posses 
symmetries. 

