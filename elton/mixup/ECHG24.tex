% reducesymm/elton/Arxiv_paper/ECHG24.tex
%                        pdflatex ECHG24; biber ECHG24
%                        bibtex ID: ECHG24


% Predrag                                       2024-11-23 
%       for the time being: if you edit
% \bibliography{../../bibtex/pipes} 
%       alert me - the master *.bib is elsewhere

%%%%%%%%%%%%%%%%%%%%%%%%%%%%%%%%%%%%%%%%%%%%%%%%%%%%%%%%%%%%%%%%%%%%%%%
             \newif\ifboyscout\boyscouttrue          %% comments     %%
             \newif\ifsubmission\submissionfalse     %% internal     %%
%%%%% Toggle between draft and public versions:  %%%%%%%%%%%%%%%%%%%%%%
% \boyscoutfalse                         % public, hyperlinked       %%
% \boyscoutfalse\submissiontrue          % for journal ???           %%
%%%%%%%%%%%%%%%%%%%%%%%%%%%%%%%%%%%%%%%%%%%%%%%%%%%%%%%%%%%%%%%%%%%%%%%


% Predrag   version 0.3                         2024-12-20
%           switched to JFM format 
% Predrag   version 0.2                         2024-12-02 
%           linked pipes.bib, cleaned up citations
%           edited abstract
%           edited addresses, emails (still to create EJ's gatech email)                        
% John E.   version 0.1                         2024-11-23                         

   \ifsubmission
\documentclass[lineno]{jfm}
 %% https://www.cambridge.org/core/journals/journal-of-fluid-mechanics/information/author-instructions/preparing-your-materials
 %% https://www.overleaf.com/latex/templates/journal-of-fluid-mechanics/hnmjwjrhcmnf
 %% PC 2024-12-20 was  \documentclass[letter,12pt,openany]{article}
    \else
\documentclass{jfm}
    \fi

\pdfoutput=1

\usepackage{newtxtext}
\usepackage{newtxmath}
    \ifsubmission
\usepackage[numbers]{natbib}  %% PC 2024-12-20 used by \bibliographystyle{jfm}
    \else
\usepackage[numbers]{natbib}  
    \fi
\usepackage{subcaption}
\usepackage{hyperref}
\hypersetup{
    colorlinks = true,
    urlcolor   = blue,
    citecolor  = black,
}
\newtheorem{lemma}{Lemma}
\newtheorem{corollary}{Corollary}
\newcommand{\RomanNumeralCaps}[1]
\linenumbers

%\usepackage[legalpaper, margin=1in]{geometry}
%% \usepackage{amsmath,amsfonts,amssymb,amsthm} %% PC 2024-12-20 removed
%\usepackage{authblk}
%\usepackage{color}
%\usepackage{url}
%\usepackage{alltt}
%\usepackage{ifthen}
%\usepackage[latin1]{inputenc}
%\usepackage{times}
%\usepackage[T1]{fontenc}
%\usepackage{graphicx}  %% PC 2024-12-20 was \usepackage[pdftex]{graphicx}
%\usepackage[font={small}]{caption}
%\usepackage{array}
%\usepackage[pdftex,colorlinks]{hyperref}


\graphicspath{{figs/}} %% directories with  graphics files

    \ifsubmission\else
%\input{biblatex}    % this makes references hyperlinked
%\addbibresource{../../bibtex/pipes.bib}
    \fi

% elton/inputs/defsElton.tex
% $Author: predrag $ $Date: 2015-09-26 19:42:18 -0400 (Sat, 26 Sep 2015) $

%%%%%%%%%%%% MACROS, noisy project specific %%%%%%%%%%

    \ifboyscout
\newcommand{\toCB}{\marginpar{\footnotesize 2CB}}  % to compare with ChaosBook
\newcommand{\inCB}{\marginpar{\footnotesize now in CB}} % entered in ChaosBook
\newcommand{\JRE}[1]{$\footnotemark\footnotetext{JRE: #1}$}
\newcommand{\JREedit}[1]{{\color{red}#1}}
\renewcommand{\authorPC}[1]{\hfill (P. Cvitanovi\'c, #1)}
\renewcommand{\authorJMH}[1]{\hfill (J.M. Heninger, #1)}
\renewcommand{\authorMMFPC}[1]
     {\hfill (M.M. Farazmand and P. Cvitanovi\'c, #1)}
\renewcommand{\authorMMF}[1]{\hfill (M.M. Farazmand, #1)}
    \else
\newcommand{\toCB}{}
\newcommand{\inCB}{}
\newcommand{\JRE}[1]{}
\newcommand{\JREedit}[1]{#1}
\renewcommand{\authorPC}[1]{\hfill (P. Cvitanovi\'c)}
\renewcommand{\authorJMH}[1]{\hfill (J.M. Heninger)}
\renewcommand{\authorMMFPC}[1]
     {\hfill (M.M. Farazmand and P. Cvitanovi\'c)}
\renewcommand{\authorMMF}[1]{\hfill (M.M. Farazmand)}
   \fi %end of internal draft switch

\newcommand{\PCpost}[2]{\item[#1 Predrag] {#2}}
\newcommand{\APWpost}[2]{\item[#1 Ashley] {#2}}
\newcommand{\KYSpost}[2]{\item[#1 Kimberly] {#2}}
\newcommand{\BEpost}[2]{\item[#1 Bruno] {#2}}
\newcommand{\FFpost}[2]{\item[#1 Franco] {#2}}
\newcommand{\MMFpost}[2]{\item[#1 Mohammad] {#2}}
\newcommand{\AFpost}[2]{\item[#1 Adam] {#2}}
\newcommand{\YLpost}[2]{\item[#1 Lan] {#2}}
\newcommand{\NBBpost}[2]{\item[#1 Burak] {#2}}

%%%%%%%%%%%% MACROS, project specific %%%%%%%%%%


\newcommand{\NS}{Navier-Stokes}
\newcommand{\NSe}{Navier-Stokes equation}
\newcommand{\Reynolds}{\ensuremath{\textit{Re}}}  % Reynolds number
\newcommand{\Velgradmat}{Matrix of velocity gradients}

\newcommand{\steady}{\marginpar{{\color{green}\textdollar}}}

%%%%%%%%%%%%%%% Sundry symbols within math eviron.: %%%%%%%%%%%%
\newcommand{\pd}[2]{\frac{\partial #1}{\partial #2}}
\newcommand{\grad}{{\bf \nabla}}
\newcommand{\trHalf}[1]{\tau_{#1}}    % 1/2 cell translation
\newcommand{\DiffC}{\ensuremath{D}}     % diffusion constant

%%% 3D physical flow naming conventions
% PC \teq is the name, \seq is the  symbol
\newcommand{\teq}[1]{\ensuremath{{\text{eq#1}}}}
\newcommand{\teqa}{\ensuremath{{\text{eq0}}}}
\newcommand{\teqb}{\ensuremath{{\text{eq1}}}}
\newcommand{\teqc}{\ensuremath{{\text{eq2}}}}

\newcommand{\ttw}[1]{\ensuremath{{\text{tw#1}}}}
\newcommand{\ttwa}{\ensuremath{{\text{tw1}}}}  % spanwise
\newcommand{\ttwb}{\ensuremath{{\text{tw2}}}} % Divakar D1, lower streamwise
\newcommand{\ttwc}{\ensuremath{{\text{TW3}}}}  % upper streamwise

\newcommand{\bu}{\ensuremath{{\bf u}}}
\newcommand{\bx}{\ensuremath{{\bf x}}}
\newcommand{\be}{{\bf e}}
\newcommand{\ben}[1]{{\be}_{#1}}
\newcommand{\beUBg}[1]{\ensuremath{\be_{#1}}}
\newcommand{\butot}{\ensuremath{{\bf u_{tot}}}}
\newcommand{\bnabla}{\ensuremath{{\bf \nabla}}}
\newcommand{\lapl}{\ensuremath{{\nabla^{2}}}}
\newcommand{\Norm}[1]{\|{#1}\|}

\newcommand{\xeq}[1]{\ensuremath{\bx_{\text{\tiny eq#1}}}}
\newcommand{\xeqa}{\ensuremath{\bx_{\text{\tiny eq0}}}}
\newcommand{\xeqb}{\ensuremath{\bx_{\text{\tiny eq1}}}}
\newcommand{\xeqc}{\ensuremath{\bx_{\text{\tiny eq2}}}}

\newcommand{\xtw}[1]{\ensuremath{\bx_{\text{\tiny tw#1}}}(t)}
\newcommand{\xtwa}{\ensuremath{\bx_{\text{\tiny tw1}}}(t)}
\newcommand{\xtwb}{\ensuremath{\bx_{\text{\tiny tw2}}}(t)}
\newcommand{\xtwc}{\ensuremath{\bx_{\text{\tiny tw3}}}(t)}

%%%%%%%%%%%%%%%%%%%%%%%%%%%%%%%%%%%%%%%%%%%%%%%%%%%%%%%%%%%
% PC: experimental, for stagnation points
\newcommand{\tSP}[1]{{\ensuremath{{\text{SP#1}}}}}
\newcommand{\tSPone}{\ensuremath{{\text{SP1}}}}
\newcommand{\tSPtwo}{\ensuremath{{\text{SP2}}}}
\newcommand{\tSPthr}{\ensuremath{{\text{SP3}}}}
\newcommand{\xSP}[1]{{\ensuremath{\bx_{\text{\tiny SP#1}}}}}
\newcommand{\xSPone}{\ensuremath{\bx_{\text{\tiny SP1}}}}
\newcommand{\xSPtwo}{\ensuremath{\bx_{\text{\tiny SP2}}}}
\newcommand{\xSPthr}{\ensuremath{\bx_{\text{\tiny SP3}}}}

%%%%%%%%%%%%%%%%%%%%%%%%%%%%%%%%%%%%%%%%%%%%%%%%%%%%%%%%%%%
% JFG: Fluids literature uses bold to indicate vector
% quantities, so that one can use {\bf u} for the vector
% and u for the first component of the vector.
% PC \eqva/\reqva directory: \tAA is the name, \sAA is the  symbol
\newcommand{\tLM}{\ensuremath{{\text{EQ0}}}}
\newcommand{\tLB}{\ensuremath{{\text{EQ1}}}}
\newcommand{\tUB}{\ensuremath{{\text{EQ2}}}}
\newcommand{\tNNB}{\ensuremath{{\text{EQ3}}}}
\newcommand{\tNB}{\ensuremath{{\text{EQ4}}}}
\newcommand{\tEQfive}{\ensuremath{{\text{EQ5}}}}
\newcommand{\tEQsix}{\ensuremath{{\text{EQ6}}}}
\newcommand{\tEQsev}{\ensuremath{{\text{EQ7}}}}
\newcommand{\tEQeight}{\ensuremath{{\text{EQ8}}}}
\newcommand{\tEQnine}{\ensuremath{{\text{EQ9}}}}
\newcommand{\tEQten}{\ensuremath{{\text{EQ10}}}}

\newcommand{\tTW}[1]{\ensuremath{{\text{TW#1}}}}
\newcommand{\tTWone}{\ensuremath{{\text{TW1}}}}  % spanwise
\newcommand{\tTWDone}{\ensuremath{{\text{TW2}}}} % Divakar D1, lower streamwise
\newcommand{\tTWthree}{\ensuremath{{\text{TW3}}}}  % upper streamwise


\newcommand{\bCell}{\ensuremath{\Omega}}
\newcommand{\bNarrow}{\ensuremath{\Omega_{\text{\tiny W03}}}}
    % JG: W02 for Waleffe Tokyo proceedings 2002, where this cell first appears,
    % ok'd by wally
\newcommand{\bHKW}{\ensuremath{\Omega_{\text{\tiny{HKW}}}}}
\newcommand{\bSch}{\ensuremath{\Omega_{\text{\tiny{Sch}}}}} % Schmiegel
\newcommand{\bbR}{\mathbb{R}}
%\newcommand{\bbU}{\mathbb{U}}
%\newcommand{\bbUsymm}{\ensuremath{\bbU_{S}}}
\newcommand{\bbUS}{\ensuremath{\bbU_{S}}}
\newcommand{\bbUthree}{\ensuremath{\bbU_{s3}}}

% PC \eqva velocity field naming conventions
\newcommand{\uEQ}{\ensuremath{\bu_{\text{\tiny EQ}}}}
\newcommand{\uLM}{\ensuremath{\bu_{\text{\tiny EQ0}}}}
\newcommand{\uLB}{\ensuremath{\bu_{\text{\tiny EQ1}}}}
\newcommand{\uUB}{\ensuremath{\bu_{\text{\tiny EQ2}}}}
\newcommand{\uNNB}{\ensuremath{\bu_{\text{\tiny EQ3}}}}
\newcommand{\uNB}{\ensuremath{\bu_{\text{\tiny EQ4}}}}
\newcommand{\uEQfive}{\ensuremath{\bu_{\text{\tiny EQ5}}}}
\newcommand{\uEQsix}{\ensuremath{\bu_{\text{\tiny EQ6}}}}
\newcommand{\uEQsev}{\ensuremath{\bu_{\text{\tiny EQ7}}}}
\newcommand{\uEQeight}{\ensuremath{\bu_{\text{\tiny EQ8}}}}
\newcommand{\uEQnine}{\ensuremath{\bu_{\text{\tiny EQ9}}}}
\newcommand{\uEQten}{\ensuremath{\bu_{\text{\tiny EQ10}}}}

\newcommand{\GPKF}{\ensuremath{\Gamma}} % Hoyle notation, equivariant symmetry group
\newcommand{\trDiscr}[2]{\tau_{#1}^{#2}}    % discrete cell translation 1/4, ...
% isotropy subgroup $H \incl G$:
\newcommand{\isotropyG}[1]{\ensuremath{H_{\text{\tiny #1}}}}


%%%%%%%%%%%%%%%%%%% eventually remove these %%%%%%%%%%%%%%%%%%%%%%%%%%%%%
\newcommand{\huUB}{\ensuremath{\hbu_{\text{\tiny EQ2}}}}
\newcommand{\hbu}{\tilde{{\bf u}}}
\newcommand{\hbv}{\tilde{{\bf v}}}
\newcommand{\hu}{\tidle{u}}
\newcommand{\hv}{\tidle{v}}
\newcommand{\vc}{\mathbf}



% Keywords command
\providecommand{\keywords}[1]
{
  \small	
  \textbf{\emph{Keywords--}} #1
}




\begin{document}
        \ifboyscout
\date{{\color{blue}Draft of \today}} %end\color{blue}
        \else
\date{Draft, December 20, 2024}
        \fi 

\title[Mixing in {\pC} turbulence]
      {Chaotic mixing in {\pC} turbulence}
% or {Lagrangian dynamics in {\pC} turbulence} ?


\author[J. R. Elton, P. Cvitanovi\'c, J. Halcrow and J. F. Gibson]
        {John R. Elton\aff{1}
            \corresp{\email{Jelton.physics@gmail.com}}, 
            Predrag Cvitanovi\'c\aff{1}, 
            Jonathan Halcrow\aff{2} 
            and 
            John F. Gibson\aff{3}
        }
%% PC 2024-12-20 obsolete title style:
%        {\center J.\ns R.\ns E\ls L\ls T\ls O\ls N\aff{1}
%            \corresp{\email{JElton.physics@gmail.com}},
%        \ns
%        P.\ns C\ls V\ls I\ls T\ls A\ls N\ls O\ls V\ls I\ls \'C\aff{1}
%            %\corresp{\email{predrag.cvitanovic@physics.gatech.edu}},
%        \ns
%        J.\ns  H\ls A\ls L\ls C\ls R\ls O\ls W\aff{2}
%            %\corresp{\email{halcrow@gmail.com}}
%    \and
%        \ns
%        J.\ns F.\ns G\ls I\ls B\ls S\ls O\ls N\aff{3}
%            %\corresp{\email{John.Gibson@unh.edu}}
%        }

\affiliation{   \aff{1}School of Physics, Georgia Inst. of Technology, Atlanta GA 
                \aff{2}Google Research, Atlanta, GA 
                \aff{3}Mathematics and Statistics, Univ. New Hampshire, Durham NH
            }

\maketitle

\begin{abstract}    %%%%%%%%%%%%%%%%%
    % master file siminos/froehlich/slice/abstract.tex
% $Author$ $Date$

We study symmetry reduction of dynamical systems with
continuous symmetries by the \mslices\ (\mframes) and show that a `slice'
defined by minimizing the distance to a single generic `{\template}'
intersects the group orbit of every point in the full {\statesp}. Global
symmetry reduction by a single slice is, however, not natural for a
chaotic / turbulent flow; it is better to cover the \reducedsp\ by a set
of slices, one for each dynamically prominent unstable pattern.
Judiciously chosen, such tessellation eliminates the dynamical traversals
of the \sset\ that comes along with each slice, an artifact of using the
{\template}'s local group linearization globally. We compute the jump in
the \reducedsp\ induced by crossing a \sset. As an illustration of the
method, we reduce the $\SOn{2}$ symmetry of the \cLe.
  %
  %
\PC{{\bf to Stefan}:
write this  often! this might be the only part of this text that most
people glance at.
%PC 2010-09-30: planted an error into the abstract, just to see how
%   often do you edit it.
%
%
   When you write a project report or a research article, you always
   write abstract, introduction and conclusions first, and then keep
   rewriting them often. They are the most important parts of the text,
   as that is for most people only parts they will look at.
   }

%
% ****** End of file abstract.tex ******

\end{abstract}       %%%%%%%%%%%%%%%%%

\keywords{turbulence, mixing, {\pC} flow, Navier-Stokes}

%%% \section{Introduction} %%% \label{s:intro} %%%%%%%%%%%%%%%%%
     % siminos/kittens/intro.tex      pdflatex CL18
% $Author: predrag $ $Date: 2020-08-02 22:02:31 -0500 (Sun, 02 Aug 2020) $


\section{Introduction}
\label{s:intro}

A temporally chaotic system is exponentially unstable with time: double the
time, and exponentially more \po s are required to cover its strange
attractor to the same accuracy. For large spatial extents, the complexity of
the spatial shapes also needs to be taken into account; double the spatial
extent in a given direction, and exponentially as many distinct
{\spt} patterns will be required to describe the repertoire of
system's shapes to the same accuracy.
The systems whose temporal and spatial correlations decay sufficiently fast,
and whose ``physical'' dimension\rf{ginelli-2007-99,DCTSCD14} grows with
system size, are said to be ``{\spt}ly chaotic.''

    \PC{2019-12-27}{
\catlatt\ = classical field theory on a $d$\dmn\ hyper-cubic lattice, with an
``anti-harmonic" rotor at each site, coupled to its nearest neighbors
    }

    \PC{2019-06-26}{
Physical picture:

Turbulence everywhere in space, with a range of length scales. Discretize into
cells, with each cell turbulent, and cells coupled
to their nearest neighbors\rf{Kaneko83}.

As a function of the strengths of cell-cell couplings, dynamics can exhibit rich
phase-transitions structure\rf{Kaneko84}. In this paper we chose couplings such
that the system is fully turbulent.

Explain word ``turbulence" as used here.

Hamiltonan, so symplectic or area preserving, but that is not essential.
Cite the Hamiltonian zeta function
from ChaosBook.

The main point: we've been doing it all wrong, and we know that since
\Poincare.
In ``explaining'' chaos we talk the talk as though we never moved beyond Newton.
But people who actually compute solutions do something altogether different,
closer to Lagrange (and the late 20th century, `spacetime' physics).
This paper realigns the theory to what we actually {\em do} when
solving ``chaos'' equations, using nothing more than the well known linear
algebra.
    }

coupled map lattice models

the spacetime discretized

dynamics of small-scale spatial structures modeled by discrete time
maps

single cell dynamics  attached to lattice sites,

coupling to neighboring sites

the Gutkin and Osipov\rf{GutOsi15}
$d$\dmn\ coupled cat maps lattice
(``{\catlatt}'' for short, in what follows),
a {\spt} generalization of the Percival and Vivaldi\rf{PerViv} {linear
code} for temporal evolution of a single cat map


the $d$\dmn\ lattice {\sPe}
\[
 (\Box -s+2d)\,\ssp_{z}  =  -\Ssym{z}
 \,.
\]

from the cat maps (modeling the
Hamiltonian dynamics of individual ``particles'') at sites of a
$(d\!-\!1)$\dmn\ spatial lattice, linearly coupled to their nearest
neighbors.

solution $\Xx$ of a global fixed-point condition
$F[\Xx]=0$ is uniquely encoded by a finite alphabet $d$\dmn\ symbol
lattice state  $\Mm$



which symbol \brick s are {\admissible}?

The  linearity of the {\catlatt} enables us to

standard crystallographic  methods\rf{Dresselhaus07} and
integer lattices counting\rf{Barvinok08} enable us to count {\spt}ly finite \brick s,
and give explicit formulas for the number of \dtor\ solutions
for \brick s of any size.

Implementing this program requires several tools not standard in
dynamicist's tool box: lattice Green's functions; lattice determinants.

We start the paper with a reformulation of the 1 degree of freedom
Bernoulli map, because our goal, the \catlatt\ is nothing but its
generalization to a mechanical system in spacetimes of arbitrary
dimension, and thus arguably the simplest possible example of a `chaotic
field theory'.

\bigskip

The paper is organized as follows:
For a reader too busy\rf{focusPOT} to read the book\rf{ChaosBook}, we
start in \refsect{s:coinToss} with a brief course on `chaos' theory,
disguised as a humble coin toss. The deep insight here is the realization
that the volume \refeq{detBern0} of the {\jacobianOrb} (the functional
determinant of the {\fundPip} or the {\HillDet}, see
\reffig{fig:BernCyc2Jacob}, \ref{fig:catCycJacob} and
\ref{fig:BravaisLatt}), counts the numbers of global solutions for a
given `law', true at all times and at all spatial positions.
Before turning to the spatially infinite field theory in
\refsect{s:catlatt}, it is instructive to motivate our formulation of the
{\catlatt} by investigating the temporal lattice Bernoulli and cat
systems (\ie, `\spt\ lattices' with only one site in the spatial
direction).
In \refsect{s:catPV} we review the traditional cat map in its usual,
Hamiltonian formulation,  and \refappe{s:catMapHam} construct an explicit
generating (\AW) partition of the cat map \statesp.
In \refsect{s:catLagrange} we introduce the `\templatt', a global lattice
reformulation of the cat map.
The \po s theory of cat maps, \refsect{s:tempCatCount}, can be developed in either
formulation:
both the Hamiltonian cat map \po s counting (\refappe{s:catHamCount}) and
the Lagrangian {\templatt} \po s counting (\refsect{s:tempCatCount}) lead
to the same {\tzeta}, with the two formulations related by {Hill's
formula} (\refsect{s:HillForm}).
A reader may skip \refSecttosect{s:catPV}{s:tempCatCount} on the
first reading, as the paper proper starts with \refsect{s:catlatt}.

In \refsect{s:catlatt} (after a brief review of the traditional coupled
map lattices, \refsect{s:CCMs}), we extend the \templatt\ to
the $d$\dmn\ \catlatt, \refsect{s:dDcatLatt}.
In \refsect{s:BravaisLatt} we show that the system admits  a natural
$d$\dmn\ symbolic  code with a finite alphabet, and then study finite
{\spt} symbol \brick s.

\refSect{s:catLattCount} %~{\em Invariant tori in $d$\dmn\ \catlatt}
describes our counting solutions of a
$d$\dmn\ \catlatt.

In \refsect{s:catLattShadow}  we use these  results to construct
sets of {\spt} \twots\ that partially shadow each other.

The results are summarized and some open questions discussed in the
\refsect{s:summary}.

In \refappe{s-SymbDynGloss} we collect the symbolic dynamics definitions
needed throughout the paper.



\section{{\PCf}}
\label{s:PCF}

\subsection{The Navier-Stokes equations}
\label{s:NS}
 The underlying equations
that govern the motion of \pCf\ are the {\NSe}s,
along with boundary conditions. The boundary conditions for \pCf\ in the $x$
and $z$ directions are periodic,
 $ \bu(x, y, z) = \bu(x+L_x, y, z) =
\bu(x, y, z + L_z) $.
 In the $y$ direction,
 $\bu = (1,0,0)$ at $\bx = (0,1,0)$ and $\bu = (-1,0,0)$ at $\bx =
 (0,-1,0)$.

 The fluid is taken to be incompressible, so in this case the
 {\NSe}s are
 \beq
 \frac{\partial \bu}{\partial t} + (\bu \cdot \nabla)\bu = -\nabla p + \frac{1}{Re} \nabla^{2} \bu
    \,,\qquad
\nabla \cdot \bu  = 0 \,. \label{eqn:NavierStokes} \eeq 



For an Eulerian equilibrium velocity field that is not changing in time, 
the first equation in \refeq{eqn:NavierStokes} simplifies to 
\beq
 (\bu \cdot \nabla)\bu = -\nabla p + \frac{1}{Re} \nabla^{2} \bu
    \,, 
\ee{eqn:NavierStokes2}
 
The Reynolds number parameter $\Reynolds$, which gives a measure of fluid 
viscosity and degree to which fluid motion may become turbulent, is given 
by 
\beq Re = \frac{\overline{u}L}{\nu} 
\eeq 
where $\overline{u}$ is the average fluid velocity and $L$ is the 
characteristic length. Thus the form of the {\NSe}s and boundary 
conditions make use of rescaling to use non-dimensionalized variables. We 
use $\Reynolds = 400$, in the regime of transitional turbulence, for the 
\pCf\ simulations throughout the text. 

For computational purposes, it is easier to work with a velocity field  that
represents the {difference} from the laminar flow. 
So we can break up the total field into two components: $\butot =
y \hat{\bf x} + \bu$. Here $y \hat{\bf x}$ is the laminar velocity
field and $\bu$ is then the difference between the total velocity and
laminar. Substitute $y \hat{\bf x} + \bu$ for $\bu$ in the
nondimensionalized {\NSe}s above to get
\beq
    \frac{\partial \bu}{\partial t}
    + y  \frac{\partial \bu}{\partial x}
    + v \, \hat{\bf x}
    + \bu \cdot \bnabla \bu
=
    - \bnabla p
    + \frac{1}{\Reynolds}
        \lapl \bu  \,, \quad \nabla \cdot \bu = 0
\,,
\ee{NavStokesDiff}
with boundary conditions $\bu = 0 $ at $y \pm 1$.  Having 
Dirichlet boundary conditions on $\bu$ makes the analysis much easier, 
since the set of allowable velocity fields (those fields that satisfy 
incompressibility and boundary conditions) forms a vector space. The 
equilibrium velocity fields we study start from $\bu$ which satisfies 
\refeq{NavStokesDiff}, and we may then add back the laminar part of the 
flow to produce physical fluid trajectories. 

This study is conducted at $\Reynolds = 400$ in the 
small aspect-ratio cell \citep{GHCW07,HGC08}
\bea
\bNarrow~  &=&       [2\upi/1.14, 2, 2\upi/2.5]
    \continue
           &\approx& [5.51, 2, 2.51]
    \continue
           &\approx& [190, 68, 86] \;\;\; \mbox{wall units}
\label{cellNarrow}
\eea
where the wall units are in relation to a mean shear rate of $\langle
\partial u/ \partial y \rangle = 2.9$ in non-dimensionalized units
computed for a large aspect-ratio simulation at $\Reynolds = 400$.
Empirically, at this Reynolds number the \bNarrow\ cell 
exhibits only short-lived transient
turbulence \citep{GHCW07}. The $z$ length scale $L_z = 4 \upi/5$
of \bNarrow\ was chosen as a compromise between the $L_z = 6 \upi/5$ of
\bNarrow\ and its first harmonic $L_z/2 = 3 \upi/5$ \citep{W02}.
Unless stated otherwise, all calculations
are carried out for $\Re = 400$ and the $\bNarrow$ cell. In the notation
of this paper, the solutions presented in\rf{N90} have wavenumbers
$(\alpha, \gamma) = (0.8, 1.5)$ and fit in the cell $[2 \upi/0.8, 2,
2\upi/1.5] \approx [7.85, 2, 4.18]$.%


%% PC 2024-12-20 copied from n00bs.tex  2009-06-16
\begin{figure}
\centering
{\includegraphics[width=0.31\textwidth]{EQ0}} \hskip -30ex \tEQzero \hskip 25ex
%{\includegraphics[width=0.31\textwidth]{EQ1}} \hskip -30ex \tLB \hskip 25ex
{\includegraphics[width=0.31\textwidth]{EQ2}} \hskip -30ex \tEQtwo \hskip 25ex ~
%{\includegraphics[width=0.31\textwidth]{EQ3}} \hskip -30ex \tNNB    \hskip 25ex
%{\includegraphics[width=0.31\textwidth]{EQ4x}} \hskip -30ex \tNB     \hskip 25ex
%{\includegraphics[width=0.31\textwidth]{EQ5}} \hskip -30ex \tEQfive \hskip 25ex ~
{\includegraphics[width=0.31\textwidth]{EQ8}} \hskip -30ex \tEQeight \hskip 25ex ~
%%                              was    {EQ8xz}
%\\
%{\includegraphics[width=0.31\textwidth]{uEQ6Re330box}} \hskip -30ex \tEQsix   \hskip 25ex
%{\includegraphics[width=0.31\textwidth]{EQ6}} \hskip -30ex \tEQsix   \hskip 25ex
%{\includegraphics[width=0.31\textwidth]{EQ7}} \hskip -30ex \tEQsev   \hskip 25ex
%{\includegraphics[width=0.31\textwidth]{EQ9}}  \hskip -30ex \tEQnine \hskip 25ex
%{\includegraphics[width=0.31\textwidth]{EQ10}} \hskip -30ex \tEQten  \hskip 25ex
%{\includegraphics[width=0.31\textwidth]{EQ11}} \hskip -30ex \tEQelev \hskip 25ex ~
    \caption[$3D$ space plots of \tEQtwo\ and \tEQeight.]{
\tEQzero, \tEQtwo\ and \tEQeight\ Eulerian equilibrium solutions of {\pCf} in 
$\bNarrow = [2\upi/1.14, 2, 2\upi/2.5]$ at $Re = 400$. 
The heat map color indicates
the streamwise ($u$, or $x$ direction) velocity of the fluid:
{red} shows fluid moving at $u=+1$,
{blue}, at $u=-1$.
The heat map color as a function of $u$ is indicated
by the laminar {\eqv}, the front face of the
laminar solution \tEQzero, $u(y) = y$, serving as a reference.
From \citet{HGC08}.
    }
\label{f:eqbaboxes}
\end{figure}

\subsection{Computation of trajectories from Eulerian equilibrium velocity fields}
\label{s:channelflow}

 In order to integrate streamlines of {\pCf}
and follow the paths of tracer particles, it is first
necessary to have numerically accurate \eqv\ $3D$-velocity fields.

The starting point for this task is to obtain the necessary data sets for 
evaluating velocity field values for a given \eqv, e.g. the equilibria 
as shown in \reffig{f:eqbaboxes}. These are made available at the website 
{\tt Channelflow.org} \citep{channelflow}. The data obtained 
\citep{channelflowDat} stores the spectral coefficients $\mathbf{\hat{u}}$ 
of the expansion of a velocity field $\mathbf{u(x)}$ satisfying \refeq{NavStokesDiff}. The form of the 
expansion is 
\begin{equation}
 \mathbf{u(x)} = \sum_{m_{y}=0}^{M_{y}-1}\sum_{m_{x}=0}^{M_{x}-1}\sum_{m_{z}=0}^{M_{z}-1}
 {\mathbf{\hat{u}}_{m_{x},m_{y},m_{z}} \bar{T}_{m_{y}}(y)e^{2\pi i(k_{x}x/L_{x} + k_{z}z/L_{z})}}
\label{eqn:spectralsum}
 \end{equation}

The $\bar{T}(y)$'s are Chebyshev polynomials defined on the interval 
[-1,1]. For a given velocity field expansion, the 
upper bounds on the sums are known from the geometry, and the $k$'s are 
related to the $m$'s through the following relations: 
 \beq 
k_{x} = \left \{ 
\begin{array}{l}
m_{x} \hspace{20 mm} 0 \leq m_{x} \leq M_{x}/2   \\
m_{x} - M_{x} \hspace{10 mm} M_{x} < m_{x} < M_{x}  \\
\end{array}  \right.
\eeq 
\beq k_{z} = m_{z} \hspace{10 mm} 0 \leq m_{z} < M_{z}
\,.
\eeq
Hence, with a knowledge of the spectral coefficients we can compute 
$\mathbf{u(x)}$ by evaluating this sum at a particular $\bx = (x,y,z)$. 

Various internal functions within {\tt Channelflow.org} have been written 
to compute $\bu$ on a set of gridpoints. It is possible, by interpolation 
of the velocity fields on these gridpoint values, to integrate a 
trajectory with great computational speed. However this will not be 
nearly as accurate as evaluating the sum \refeq{eqn:spectralsum} 
directly. So we evaluate \refeq{eqn:spectralsum} to give the exact 
velocity field at every point along a trajectory, adding back the laminar part of the flow. We are able to perform 
these computations in Matlab with enough speed to compute many tracer 
particle trajectories within an Eulerian equilibrium velocity for an adequate 
length of time to study the flow dynamics.  


\subsection{Symmetries of {\pCf}}
\label{s:PCF_symm}

As part of our theoretical analysis of trajectories of fluid particles 
within an Eulerian equilibrium velocity field, it will be critical to use and 
understand the symmetries involved in the special geometry of {\pCf}. 
Thus we take a quick detour to discuss these symmetries from a 
group-theoretic perspective. We focus on the symmetries relevant to the 
Eulerian equilibria studied in this work; additional details are provided in 
\citet{HalcrowThesis}. 

\PCf\ is invariant under two reflections $\sigma_1,\sigma_2$ and a
continuous two-parameter group of translations $\tau(\shift_x, \shift_z)$:
\bea
\sigma_1 \, [u,v,w](x,y,z) &=& [u, v,-w](x,y,-z) \continue
\sigma_2 \, [u,v,w](x,y,z) &=& [-u,-v,w](-x,-y,z)  \label{reflSfit1}\\
\tau(\shift_x, \shift_z)[u,v,w](x,y,z) &=& [u,v,w](x+\shift_x,y,z+\shift_z) \nnu\,.
\eea
The \NSe s and boundary conditions are invariant for any symmetry $s$
in the group generated by these elements:
$\partial (s \bu) / \partial t = s (\partial \bu / \partial t)$.

The {\pC} symmetries can be interpreted geometrically in the space of
fluid velocity fields. Let $\bbU$ be the space of
square-integrable, real-valued velocity fields that satisfy the kinematic
conditions of \pCf:
\bea
 \bbU  &=& \{\bu \in L^2(\Omega) \; | \; \grad \cdot \bu = 0,
               \; \bu(x, \pm 1, z) = 0, 
 %\notag  
 \continue
       &\phantom{=}&  {} \qquad \qquad \qquad \; \; %\mbox{and }
          \bu(x, y, z) = \bu(x+L_x, y, z) = \bu(x, y, z + L_z)\}  
\,.
\nnu
\eea
The continuous symmetry $\tau(\shift_x, \shift_z)$ maps each state
$\bu \in \bbU$ to a $2D$ torus of states with identical dynamic
behavior. This torus in turn is mapped to four equivalent tori by
the subgroup $\{1,\sigma_1,\sigma_2, \sigma_1 \sigma_2\}$. In
general a given state in $\bbU$ has four $2D$ tori of dynamically
equivalent states.

Most of the Eulerian \eqva\ that are currently known for \pCf\
are invariant under the `shift-reflect' symmetry
$s_1 = \tau(L_x/2,0) \, \sigma_1$ and the `shift-rotate' symmetry
$s_2 = \tau(L_x/2,L_z/2) \, \sigma_2$.  These symmetries form a group
\beq
S = \{1, s_1, s_2, s_3\}, \qquad s_3 = s_1 s_2, 
\eeq
which is isomorphic to the Abelian dihedral group $D_2$, and is a 
subgroup of a larger group generated by {\pC} symmetries. The 
group acts on velocity fields as: 
\bea
s_1 \, [u, v, w](x,y,z) &=& [u, v, -w](x+L_x/2,\, y,\, -z) \continue 
s_2 \, [u, v, w](x,y,z) &=& [-u, -v, w](-x+L_x/2,\,-y,\,z+L_z/2) \label{shiftRot} \\
s_3 \, [u, v, w](x,y,z) &=& [-u,-v,-w](-x,\, -y,\, -z+L_z/2)  \nnu 
\,
\eea

We denote the $S$-invariant subspace of states invariant under
symmetries \refeq{shiftRot} by
\bea
\bbUsymm  &=& \{\bu \in \bbU  \: | \;
              s_j \bu = \bu\,, \;\;  s_j \in S \}
              % \bu = \frac{1}{4} (1 + s_1 + s_2 + s_3)\,\bu \}
\,,
\label{symmSubspU}
\eea

where $ \bbUsymm \subset \bbU$.
%
$\bbUsymm$ is a flow-invariant subspaces: states initiated
in it remain there under the \NS\ dynamics.


Translations of half the cell length in the spanwise and/or streamwise
directions commute with $S$. These operators generate a discrete
subgroup of the continuous translational symmetry group $SO(2) \times
SO(2)$ :
\beq
T = \{e,\tau_x,\tau_z,\tau_{xz}\}
    \,,\qquad
    \tau_x = \tau(L_x/2,0)
    \,,\;
    \tau_z = \tau(0,L_z/2)
    \,,\;
    \tau_{xz} = \tau_x \tau_z
\,.
\ee{tauD2}
Since the action of $T$ commutes with that of $S$,
the three half-cell translations $\tau_x \bu, \, \tau_z \bu,$ and
$\tau_{xz} \bu$ of $\bu \in \bbUsymm$ are also in $\bbUsymm$.

We know that the Eulerian equilibria  {\tEQone}-{\tEQeight} are symmetric in $S$ because 
they satisfy those symmetries numerically. There is no a priori reason 
that the Eulerian equilibria should be $S$-symmetric, other than $S$ symmetry 
fixes $x,z$ phase and so rules out relative Eulerian equilibria. But $s_3$ 
symmetry alone does the same, and a few Eulerian equilibria are known that have 
$s_3$ symmetry but neither $s_1$ nor $s_2$ symmetry. There are Eulerian equilibria 
with other symmetries that fix $x,z$ phase but have other translations 
than the half-cell shifts. 

It is also possible to form other isotropy subgroups from the plane 
Couette symmetries $\tau_x$, $\tau_z$, $\sigma_1$, $\sigma_2$. These 
elements generate a group $G$ of order 16, of which there are various 
subgroups of possible orders $\{1,2,4,8,16\}$. It is known that other 
Eulerian equilibria posses different symmetries, corresponding to different 
subgroups of $G$. For example, for Eulerian equilibrium {\tEQeight}, we find there is 
symmetry under an invariance group of order 8, denoted $S_8$, that is 
isomorphic to the dihedral group $D_4$. 
\[
S_8 = \{e, s1, s2, s3, s4, s5, s6, s7\}
\]
where $s_4 = \tau_z \, \sigma_1$, $s_5 = s_4 s_2$, $s_6 = \tau_x \tau_z$, $s_7 = \sigma_2$. The action of these additional symmetries of $S_8$ on velocity fields is:
\bea
s_4 \, [u, v, w](x,y,z) &=& [u, v, -w](x,\, y,\, -z + L_z/2) \continue 
s_5 \, [u, v, w](x,y,z) &=& [-u, -v, -w](-x+L_x/2,\,-y,\,-z) \label{S_8} \\
s_6 \, [u, v, w](x,y,z) &=& [u,v,w](x+L_x/2,\, y,\, z+L_z/2)  \nnu  \\
s_7 \, [u, v, w](x,y,z) &=& [-u,-v,w](-x,\, -y,\, z)  \nnu 
\,
\eea

Which symmetries happen to exist for the different Eulerian equilibria will have 
important implications for studying the dynamics of the flow. 

\subsection{Symmetry and {\stagp}s}
\label{s:symm_stag}



From the form of $s_3$ in \refeq{shiftRot}, we can see that any Eulerian equilibrium that
is invariant under $S$ has 4 Lagrangian \stagp s at which the velocity is 0,
which satisfy the condition:
\begin{equation}
 (x,y,z) = (-x, -y, -z+L_z / 2) \label{shiftRot_eqva}
\end{equation}
There are 4 points which satisfy this constraint:
\bea
  \xSP{1} &=& (L_x/2,0,L_z/4) \continue
  \xSP{2} &=& (L_x/2,0,3L_z/4) \continue
  \xSP{3} &=& (0,0,L_z/4) \label{s3lagrange} \\
  \xSP{4} &=& (0,0,3L_z/4) \nnu
 \,.
\eea

We refer to these as {\stagp}s \tSP{1}--\tSP{4}. Due to the periodic 
boundary conditions, we equivalently have 
 $(L_x,0,L_z/4)=SP_3$ and $(L_x,0,3L_z/4)=SP_4$.
Also of note is the fact that there can exist no $s_3$-invariant \reqva, 
since $s_3$ operation flips both the $x$ and $z$ axes. These {\stagp}s 
will exist in all of the Eulerian equilibria with $S$-symmetry. Additionally, for 
an Eulerian equilibrium such as {\tEQeight} which possesses $S_8$ symmetry, from the 
action of $s_5$ in \refeq{S_8}, we will find {\stagp}s wherever 
\beq
 (x,y,z) = (-x+L_x/2, -y, -z) 
 \,,
\ee{second_condition}
which gives the additional points:
\bea
  \xSP{5}  &=& (L_x/4,0,0) \continue
  \xSP{6}  &=& (3L_x/4,0,0) \continue
  \xSP{7}  &=& (L_x/4,0,L_z/2) \label{s4lagrange} \\ %% PC was {s3lagrange}
  \xSP{8}  &=& (3L_x/4,0,L_z/2) \nnu
 \,.
\eea

In fact, we can generalize the discussion. Looking at the way the plane 
Couette symmetries act on velocity fields in \refeq{reflSfit1}, we see 
that since $\tau$ does not affect the velocity components, the condition 
needed to produce a {\stagp} (in which all three velocity components are 
negated at some shifted position) will work only for the combinations of 
these elements which contain both $\sigma_{1}$ and $\sigma_{2}$ an odd 
number of times. Within the group $G$ of order 16 of {\pC} 
symmetries generated by $\sigma_{1}$, $\sigma_{2}$, $\tau_{x}$, 
$\tau_{z}$, the requirement means we just have to identify elements that 
have a $\sigma_{1}\sigma_{2}$ term. 

There are in fact four such elements of $G$ that contain a
$\sigma_{1}\sigma_{2}$ term. We denote these as $g_1 = \sigma_{1}\sigma_{2}$,
$g_2 = \sigma_{1}\sigma_{2}\tau_{x}$, $g_3 =
\sigma_{1}\sigma_{2}\tau_{z}$, and $g_4 = \sigma_{1}\sigma_{2}\tau_x
\tau_z$. 
\bea
g_1 \, [u,v,w](x,y,z) &=& [-u,-v,-w](-x,-y,-z)  \\
g_2 \, [u,v,w](x,y,z) &=& [-u,-v,-w](-x+L_{x}/2,-y,-z)  \\
g_3 \, [u,v,w](x,y,z) &=& [-u,-v,-w](-x,-y,-z+L_{z}/2)  \\
g_4 \, [u,v,w](x,y,z) &=& [-u,-v,-w](-x+L_{x}/2,-y,-z+L_{z}/2)
\eea

Different isotropy subgroups of $G$ may or may not contain a symmetry 
which corresponds to one of these $g_1$-$g_4$ elements, however any $g_i$ 
that is part of an invariance group for an Eulerian equilibrium implies the 
existence of four symmetrically-located \stagp s in the $y = 0$ plane. 
Note that $g_3$ and $g_2$ are the elements already  discussed that 
produce \tSP{1}--\tSP{8}. 

Any Eulerian equilibrium with $g_1$ symmetry implies that there would additionally 
be \stagp s at $(0,0,0)$, $(L_{x}/2,0,0)$, $(0,0,L_{z}/2)$, and 
$(L_{x}/2,0,L_{z}/2)$. And similarly, $g_4$ symmetry implies the 
existence of \stagp s at $(L_{x}/4,0,L_{z}/4)$, $(L_{x}/4,0,3L_{z}/4)$, 
$(3L_{x}/4,0,L_{z}/4)$, and $(3L_{x}/4,0,3L_{z}/4)$. The set of all 
possible {\stagp}s based on various \pCf\ symmetries is shown in 
\reffig{fig:stags7_26}. 

So the question of existence of \stagp s for a given Eulerian equilibrium is, 
which of the $g_i$ symmetries does that Eulerian equilibrium possess? This is a 
question related to invariance under the isotropy subgroups. Of 
importance, this does not address the question of whether \emph{other} 
nontrivial \stagp s may exist that are not based on symmetry arguments 
alone. All known Eulerian equilibria of {\pCf}.
have $g_3$ symmetry. In addition, {\tEQsev}, {\tEQeight} have $g_2$ symmetry. 
This is likely related to the fact that searches for Eulerian equilibria were done 
in a symmetric subspace which contained the $g_3$ elements (the 
$S$-symmetric subspace). 

\begin{figure}
\includegraphics[width=0.95\textwidth]{stags7_26.jpg}
  \caption{
   Sets of possible \stagp s. If one of the $g_i$ symmetries is
   possessed, the velocity field will have \stagp s of the color
   corresponding to that symmetry.
   }
  \label{fig:stags7_26}
 \end{figure}



\subsection{Any nontrivial \stagp\ has a partner, symmetric about another {\stagp}}
% was {Proof that any new \stagp\ must have a partner, symmetric about one of the previously known {\stagp}s}

Though our symmetry arguments do not determine whether or not there may exist \emph{additional} {\stagp}s which are not forced by the $g_i$ symmetries in the preceding section, we can in fact that show that for Eulerian equilibria which exist in one of the flow-invariant subspaces that contains a $g_i$-symmetry (for example, $S$ has $g_3$ symmetry and $S_8$ has both $g_2$ and $g_3$ symmetry), any additional nontrivial {\stagp}s that exist must occur in symmetric pairs centered around the other known {\stagp}s.

Consider one of the Eulerian equilibria in the $S$-invariant subspace, such as {\tEQtwo}. Again, the
 action of $s_3 \in S$ on velocity fields gives:
 \beq    s_3 \, [u, v, w](x,y,z) = [-u,-v,-w](-x,\, -y,\, -z+L_z/2)\nnu\, .
 \eeq
 If $(x_{_{SP}},y_{_{SP}},z_{_{SP}})$ is a \stagp, $[u, v,
 w](x_{_{SP}},y_{_{SP}},z_{_{SP}}) = [0,0,0]$, then
 \bea s_3 \, [u, v, w](x_{_{SP}},y_{_{SP}},z_{_{SP}}) &=& [-u,-v,-w](-x_{_{SP}},\, -y_{_{SP}},\, -z_{_{SP}}+L_z/2) \nnu\, \\
 &=& [0,0,0](-x_{_{SP}},\, -y_{_{SP}},\, -z_{_{SP}}+L_z/2) .
 \eea
 Thus $(-x_{_{SP}},\, -y_{_{SP}},\, -z_{_{SP}}+L_z/2)$ is also a \stagp.

We may parameterize a line passing through two points 
$(x_{1}, y_{1}, z_{1}),(x_{2}, y_{2}, z_{2})$
 as
 \bea
  x &=& x_{1} + (x_{2} - x_{1})t \continue
  y &=& y_{1} + (y_{2} - y_{1})t \continue
  z &=& z_{1} + (z_{2} - z_{1})t \continue
  t &\in (-\infty,\infty) .
 \eea

 Using the two stagnation points $(x_{_{SP}},y_{_{SP}},z_{_{SP}})$ and $(-x_{_{SP}},-y_{_{SP}},-z_{_{SP}} + L_z/2)$ this becomes
 
 \bea
  x &=& x_{_{SP}}(1-2t) \continue
  y &=& y_{_{SP}}(1-2t) \continue
  z &=& z_{_{SP}}(1-2t) + \frac{L_{z}}{2} t .
 \eea
When $t = 1/2$ this system returns $(x,y,z) = (0,0,L_{z}/4)$, showing 
that \tSP{3} lies on the line between these two \stagp s, halfway in 
between them. 

If we invoke the box periodicities: $x = x + L_{x}$, $z = z + L_{z}$, it 
is easy to show that this pair of {\stagp}s is also symmetric about any 
of \tSP{1}--\tSP{4}. For example, consider the translation $\mathbf{x = x + L_{x}}$: \\

 \noindent $(x_{_{SP}},y_{_{SP}},z_{_{SP}})$ is a \stagp\ $\Rightarrow$
 $(-x_{_{SP}}+L_{x},-y_{_{SP}},z_{_{SP}}+L_{z}/2)$ a \stagp.
 \bea
  x &=& x_{_{SP}}(1-2t) + L_{x}t \continue
  y &=& y_{_{SP}}(1-2t) \continue
  z &=& z_{_{SP}}(1-2t) + \frac{L_{z}}{2} t .
 \eea
When $t = 1/2$ this returns $(x,y,z) = (L_{x}/2,0,L_{z}/4)$, so that the 
new stagnation points lie symmetrically on a line passing through \tSP{1}. 

For an Eulerian equilibrium invariant under $S_8$, such as {\tEQeight}, existence of 
any additional nontrivial {\stagp} will then imply \emph{two} 
additional {\stagp}s, based on the action of $g_2$ and $g_3$. 
 If $(x_{_{SP}},y_{_{SP}},z_{_{SP}})$ is a \stagp, then  
 $(-x_{_{SP}},\, -y_{_{SP}},\, -z_{_{SP}}+L_z/2)$ and 
 $(-x_{_{SP}} + L_x/2,\, -y_{_{SP}},\, -z_{_{SP}})$ are also \stagp s. 

We will investigate numerical methods to determine the possible existence 
of any nontrivial {\stagp}s. In fact for {\tEQtwo}, as we show in the next 
section, we do find such a point and its symmetric partner. These 
additional {\stagp}s are critical for understanding the flow dynamics in 
the Eulerian equilibrium field, as their stable and unstable manifolds provide us 
with an outline of the overall dynamics. 

\section{Lagrangian dynamics}
\label{s:Lagrangian}

We know of the existence of  \stagp s in the flow of an Eulerian equilibrium 
velocity field predicted from the symmetries of \pCf. Thus the starting 
point for our investigation is clear; treating an Eulerian equilibrium velocity 
field as an autonomous dynamical system we have already identified the 
"fixed points" of the system, which we refer to in this context as the 
\stagp s.  Using the sum formula for computing velocities at any point in 
the \pCf\ domain \refeq{eqn:spectralsum}, by differentiating this formula 
it is a simple matter to compute the $[3\!\times\! 3]$ velocity gradients 
or Jacobian matrix at any point. Eigenvalues and eigenvectors of this 
matrix will provide linear stability analysis results for the {\stagp}s, 
and allow us to compute and visualize the local stable and unstable 
manifolds by starting a collection of tracer points along the directions 
of the eigenvectors, integrating them forwards and backwards in time 
(when the local tangent space is $2D$, trajectories are started 
throughout a small radius in the plane spanned by the eigenvectors). 
Though this method may underrepresent a part of the manifold for the $2D$ 
case \citep{SahVla09}, we find that the approximation works for revealing 
the interesting and relevant dynamical behaviors we seek. 

In order to investigate additional locations in the domain for which no 
movement occurs, we may numerically compute $|\bu|^{2}$ along a fine grid 
and try to ascertain regions where the velocity value falls below a given 
threshold. Then, using interpolation within these regions, any additional  
\stagp s can be pinned down. 

With the determination of the {\stagp}s and their invariant manifolds, we 
find a natural way to view the physical space of the fluid, partitioned 
into regions wherein the dynamics is dominated by the trajectories 
following closely the invariant manifolds. This provides us with a 
framework for studying how transport may occur within and between the 
different regions. 

\subsection{The upper branch Eulerian equilibrium}
\label{s:eq2}

Our analysis is carried out for the upper branch \eqv\ velocity field, 
{\tEQtwo}, at $\Reynolds = 400$. 
The cell size parameters are 
\beq   
[L_x,2,L_z]
         = \; [2\pi/1.14,2,4\pi/5]
         ~ [5.512,2,2.513].
\ee{cellW03}

To begin, we look at the evolution of Lagrangian tracers starting on a 
grid of points, shown in \reffig{fig:UBs}. The grid is chosen to lie 
in the $[y,z]$ plane, centered at $x = L_x/2$. The initial points are 
equally spaced, and offset by one position from the edge of the box. If 
the number of points is chosen to be one less than a multiple of 4, there 
will be points starting at $\xSP{1}=(L_x/2,0,L_z/4)$ and 
$\xSP{2}=(L_x/2,0,3L_z/4)$. The trajectories are integrated 
and run for a relatively short time. Just from evolving the 
grid of points alone, we begin to get a feel for the dynamics and start 
to see the formation of interesting patterns and vortical structures. 

{\tEQtwo} invariance under the symmetry group $S$, explained  in 
\refsect{s:symm_stag}, implies the existence of 4 \stagp s 
\tSP{1}--\tSP{4}, \refeq{s3lagrange}. In \reffig{fig:UBs_b} the view 
from \reffig{fig:UBs_a} has been rotated in order to reveal two of 
these \stagp s. The visualization of the behavior of trajectories near 
these fixed points reveals their  qualitative nature. The point at 
$3L_z/4$ in \reffig{fig:UBs_b} appears to be an unstable spiral, 
whereas the point at $L_z/4$ is hyperbolic. In order to verify these 
hypotheses, eigenvalues and stable/unstable manifolds for these \stagp s 
are computed. 

\begin{figure}[!h]
\centering
    \begin{subfigure}{0.98\textwidth}
    \includegraphics[width=1.0\textwidth]{fig_UB1.jpg}
      \caption{
        $3D$ perspective view
       }
      \label{fig:UBs_a}
    \end{subfigure}

    \begin{subfigure}{0.98\textwidth}
    \includegraphics[width=1.0\textwidth]{fig_UB1eq.jpg}
      \caption{
        Rotated to show the 2 \stagp s
       }
      \label{fig:UBs_b}
    \end{subfigure}
    \caption{
Grid of $19 \times 19$  initial points in the $[y,z]$ plane, centered at 
$x = L_x/2$; integrated for 15 time units to produce tracer particle 
trajectories for {\tEQtwo}.} 
\label{fig:UBs}
 \end{figure}


\subsection{Linearization and stability}


For a perturbation $\delta$\bx\ away from one of the {\stagp}s,
the change in the velocity field is given by $\delta\bu = \Mvar
\delta\bx$ where $\Mvar$ is the nine component \velgradmat\ defined
by $\Mvar_{ij}=\frac{\partial u_{i}}{\partial x_{j}}$. Since \bu\ is
given by \refeq{eqn:spectralsum}, it is a relatively simple
extension of this formula to evaluate these partials. To find
$\partial\bu/\partial y$, one needs to use the relation
$\frac{\partial}{\partial y}T_{n}(y) = n U_{n-1}(y)$ where $T_{n}$
is the $n$th Chebyshev polynomial of the first kind and $U_{n}$ is
the $n$th Chebyshev polynomial of the second kind. Everything else
is straightforward.
The eigenvalues of $\Mvar$, evaluated at a {\stagp}, determine local stability
and reveal the qualitative nature of the motion nearby the \stagp.
For the \stagp s \tSP{1}--\tSP{4}, the eigenvalues, eigenvectors,
and velocity gradients matrices are as follows.

$\xSP{1}=(L_x/2,0,L_z/4)$: There are 3 real eigenvalues, two 
positive and one negative. 
\bea
\eigExp[1] &=& -0.4652099 \,,\quad
\jEigvec[1] =
\left[\begin{array}{c}
             {0.9844417} \cr
             {0.1743315} \cr
             {0.0219779}
\end{array}\right]
                                                \label{sp1_evec1} \\ 
\eigExp[2]  &=& 0.4008961 \,,\quad 
\jEigvec[2] =
\left[\begin{array}{c}
             {~~0.5704000} \cr
             {-0.7666749} \cr
             {~~0.2947091} \cr
\end{array}\right] 
                                                \label{sp1_evec2} \\  
\eigExp[3]  &=& 0.0643139 \,,\quad 
\jEigvec[3] =
\left[\begin{array}{c}
             {0.4082166} \cr
             {0.7525949} \cr
             {0.5166819} \cr
\end{array}\right]
                                                \label{sp1_evec3} 
\eea
   The \velgradmat\ is
\beq
   \Mvar =
\left[\begin{array}{ccc}
   {-0.4305385} &  {-0.3002042} &{0.8282447} \cr
   {-0.1221356} &   {0.2456107} & {-0.1675796} \cr
   {0.0001651}  &   {-0.0828951}  & {0.1849278} \cr
\end{array}\right]
\eeq
The point is a saddle; it has 1 stable dimension and a $2D$ plane of 
instability spanned by $\jEigvec[2]$ and $\jEigvec[3]$, with the 
eigenvalues summing to 0, as required by a volume-preserving flow. 
    
The \stagp\ \tSP{4} at $(0,0,3L_z/4)$ has the same eigenvalues as 
\tSP{1}. It's eigenvectors and \velgradmat\ differ by a minus sign in the 
third component (except for $\Mvar_{33}$ where the two minuses cancel). 

$\xSP{2}=(L_x/2,0,3L_z/4)$: 
There is one real, negative eigenvalue and a complex
pair with positive real part.

\bea
&\eigExp[1] = -0.0352362 \,,\quad \jEigvec[1] =
\left[\begin{array}{c}
             {-0.9452459} \cr
             {-0.1893368} \cr
             {-0.2658228} \cr
\end{array}\right]
   \\
&\eigRe[2] \pm i\,\eigIm[2] = 0.0176181 \pm i\,0.0862176
   \\
&\jEigvec[2] =
\left[\begin{array}{c}
             {0.3737950 + 0.0544113i} \cr
             {0.2098940 - 0.4925773i} \cr
             {0.7554000} \cr
\end{array}\right]
\,,\quad
\jEigvec[3] =
\left[\begin{array}{c}
             {0.3737950 - 0.0544113i} \cr
             {0.2098940 + 0.4925773i} \cr
             {0.7554000} \cr
\end{array}\right]
\nnu\,.
\eea
The \velgradmat\ is 
\[ %beq
   \Mvar =
\left[\begin{array}{ccc}
   {-0.0316935} & {-0.0708737} &  {0.0378835} \cr
  {-0.0250579} & {-0.0218884} &  {0.0795969} \cr
   {0.0014742} & {-0.1320575} &  {0.0535818} \cr
\end{array}\right]
\] %\eeq
Trajectories starting near this \stagp\ spiral out in a plane spanned by 
the complex pair of eigenvectors. The stable direction is one-dimensional 
and points primarily along the $x$ direction. 
    
\tSP{3} at $(0,0,L_z/4)$ has the same eigenvalues as \tSP{2} and again, the 
\velgradmat\ is the same except for sign changes in the third component. 
This follows from the {\pC} symmetries. 

\subsection{Further {\stagp}s}

Having analyzed {\stagp}s \tSP{1}--\tSP{4}, before further investigating the 
dynamics, one might wonder whether other such {\stagp}s may exist 
that do not necessarily follow from a symmetry argument. To answer this 
question, as mentioned above, we numerically compute $|\bu|^{2}$ along a 
fine grid and look for where its value falls below a given threshold. 

We create a more refined grid of velocities which is $144 \times 105 
\times 144$. This is three times the 48 $\times$ 35 $\times$ 48 grid in 
each dimension used to show the initial tracer trajectories, and contains 
about 2.2 million points. At each point $|\bu|^{2}$ is then calculated 
and at every point that satisfies $|\bu|^{2} < \epsilon$ for some 
arbitrarily chosen $\epsilon$, the point is plotted. 

In \reffig{fig:fine_usquare} we show regions in the cell where 
$|\bu|^{2}$ is very small for $\epsilon = 10^{-4}$, notated by the globs 
of blue dots. The trajectories shown along with the points of small 
velocity in this figure, explained below, are also suggestive of the 
existence of a {\stagp} within the spiraling region. The four previously 
known {\stagp}s are identified in the figure, but we also see a couple of 
additional clumps. Honing in one of the suspicious clusters, starting 
from the gridpoint value with smallest velocity in the suspicious region, 
$\bx_{0} \approx (2.33476, 0.40952, 0.64577)$, and its reflection through 
$\xSP{1}$, $\bx_{0}' =2 \xSP{1} - \bx_{0}$, the 
Newton iteration 
\[ %beq
 \bx_{k+1} = \bx_{k} -
          {\Mvar}^{-1}(\bx_{k}) \, \bu(\bx_{k})
\] %eeq
%  where $\Mvar$ is the \velgradmat.
converges rapidly to verify \emph{another} pair of \stagp s. Because we 
have already used notation to define points \tSP{1}--\tSP{8} in 
\refsect{s:symm_stag}, we refer to these new numerically discovered 
{\stagp}s as \tSP{N1} and \tSP{N2}: 

\bea
\xSP{N1} &=& (2.35105561774981,0.42293662349708,0.65200166068573)
\continue
\xSP{N2} &=& (3.16051044117966,-0.42293662349708,0.60463540075018)
\label{eqn:newspNewt}
\,.
\eea
%         = [5.51156605892946182182,2,2.51327412287183459075]
We see the
 symmetry in the $y$-component of this pair, and in fact
these points are shown to be
 symmetric about the point \tSP{1}, as discussed in \refsect{s:symm_stag}:
 \beq
    (\xSP{N1} +\xSP{N2})/2 = \xSP{1}
 \,.
 \eeq

 

  \begin{center}
\begin{figure}
\includegraphics[width=0.95\textwidth]{fine_usquare.jpg}
  \caption{
Blue clumps of points indicate where the velocity for {\tEQtwo} is very 
close to zero. Shown along with the stable manifold of \tSP{3} and the 
unstable manifold of \tSP{1}. 
          }
  \label{fig:fine_usquare}
 \end{figure}
\end{center}

Repeating the linear stability analysis for \tSP{N1} and \tSP{N2}: 
There is one real, positive eigenvalue and a complex pair with negative 
real part. 
  \bea \eigExp[1] &=& 0.1453207 \,,\quad \jEigvec[1] =
\left[\begin{array}{c}
             {0.9307982} \cr
             {0.3502306} \cr
             {0.1046576} \cr
\end{array}\right]
   \continue
\{ \eigExp[2],\eigExp[3]\}
  &=& \eigRe[2] \pm i \,\eigIm[2] =  -0.0726603 \pm i\, 0.3733478
   \continue
\jEigvec[2]  &=&
\left[\begin{array}{c}
             {~0.5226203} \cr
             {-0.6703938} \cr
             {~0.2065610} \cr
\end{array}\right]
    \,,\quad
\jEigvec[3] =
\left[\begin{array}{c}
             {~0.3779843} \cr
             {~
             0} \cr
             {- 0.3031510} \cr
\end{array}\right]
\,.
\nnu
\eea
The \velgradmat\ is
\[ %beq
   {\Mvar} =
\left[\begin{array}{ccc}
   {0.0225166} &  {0.0985763} &{0.7623083} \cr
   {0.1714566} &   {-0.1275193} & {-0.6118476} \cr
   {-0.0615378}  &   {0.1755954}  & {0.1050028} \cr
         \end{array}\right]
\,.
\] %eeq

We have this time a $1D$ unstable manifold and a $2D$ spiraling stable 
manifold. The trajectories shown in \reffig{fig:fine_usquare}, which 
originate close to \tSP{1} and \tSP{3}, wander close to the spiraling stable 
manifold of the numerically discovered \tSP{N1}, showing how the 
dynamics tends to be dominated by these {\stagp}s. 

 \begin{figure}
\includegraphics[width=0.95\textwidth]{stagps_edited.jpg}
  \caption{
   The 6 unique \stagp s within one periodic box for {\tEQtwo}. 
   \tSP{1}--\tSP{4} are guaranteed by {\tEQtwo} symmetries, \tSP{N1} and 
   \tSP{N2} are determined numerically. 
   }
  \label{fig:stagps_label}
 \end{figure}

 \begin{figure}
\includegraphics[width=0.95\textwidth]{stagps2_edited.jpg}
  \caption{
   The 4 \stagp s that occur within the domain $\Omega$.
   }
  \label{fig:stagps_label2}
 \end{figure}

We have been describing all \stagp s which are inside a single periodic 
cell with dimensions $L_x \times 2 \times L_z$, pictured in 
\reffig{fig:stagps_label}. However even within this cell there is a 
redundancy in labeling all of these points as distinct. The interesting 
dynamics and connections between the different \stagp s occur along the 
$x$ direction. To understand what is happening one needs to look only at 
a subset of these \stagp s that lies in the right or left half of the 
box, that is, in the interval $[0,L_{z}/2]$ or the interval 
$[L_{z}/2,L_{z}]$. We have chosen the interval $[0,L_{z}/2]$. In the $x$ 
direction the most convenient interval is not actually $[0,L_{x}]$, 
rather we look at the \stagp s in the open interval $(-L_{x}/2,L_{x})$, 
open so as to ignore the repeated translations on the boundary. Thus an 
alternate domain of investigation that will be convenient to sometimes 
use is 
\[ %beq 
\Omega = (-L_{x}/2,L_{x}) \times [-1,1] \times [0,L_{z}/2]
\,. 
\] %eeq 
Within this domain $\Omega$ there are then just four \stagp s. They 
are \tSP{1}, \tSP{3}, \tSP{N1}, and \tSP{N2}, shown in 
\reffig{fig:stagps_label2}. Note that \tSP{N2} is a translated 
version from the way it was viewed in \reffig{fig:stagps_label}. The 
phase portrait of fundamental dynamics for {\tEQtwo} will be viewed in 
$\Omega$.


\subsection{A colorful flow portrait and {\hc}s}

With identification of all of the {\stagp}s within either the original 
periodic box or the cell $\Omega$, as well as the corresponding linear 
stability analysis, we are ready to make a complete phase space portrait 
for the upper branch, {\tEQtwo}.

\begin{figure}
\includegraphics[width=0.95\textwidth]{manifolds_both.jpg}
  \caption{
   Segments of the stable (red/blue) and unstable (green/black) manifolds of the \stagp s
   $\xSP{1} = (L_x/2,0,L_z/4)$ and
   $\xSP{2} = (L_x/2,0,3L_z/4)$ for {\tEQtwo}. 
   }
  \label{fig:manifolds_both}
 \end{figure}


    \begin{figure}
\includegraphics[width=0.95\textwidth]{man14_june3.jpg}
  \caption{
{\Hec}s of the upper branch (red trajectories) from 
$\tSP{N1} \to \tSP{3}$ and $\tSP{N2} \to \tSP{3}$, shown in a cell with $x \in$ 
[-$L_x/2$, $L_x/2$] along with the unstable manifold of \tSP{3}. 
   }
  \label{fig:hetero1}
 \end{figure}

  \begin{figure}
\includegraphics[width=0.95\textwidth]{june4_fig7.jpg}
  \caption{
Portrait of the fundamental dynamics along the manifolds of \stagp s 
\tSP{1}, \tSP{3}, \tSP{N1}, \tSP{N2} within cell $\Omega$ for the upper 
branch. 
   }
  \label{fig:hetero2}
 \end{figure}

The dynamics between the \stagp s and their translations is quite 
interesting. In \reffig{fig:manifolds_both} we see a partial view of 
the stable and unstable manifolds of two of the {\stagp}s, \tSP{1} and 
\tSP{2}, in the original periodic domain, found by integrating 
trajectories near the fixed points forwards and backwards in time along 
the stable or unstable eigenvectors. Local stability analysis shows that 
\tSP{1} has all real eigenvalues with a $1D$ stable manifold, and a $2D$ 
unstable manifold which is locally a plane 
\refeq{sp1_evec1}-\refeq{sp1_evec3}. The fact that one of the eigenvalues 
for the unstable manifold of \tSP{1} is much larger than the other is 
apparent in the figure by the fact that the trajectories in the unstable 
plane become quickly contracted in one of the dimensions, and the 
trajectories appear to leave along a nearly one-dimensional structure in 
the $y$-direction. \tSP{2} has a $2D$ unstable manifold with complex 
eigenvalues which spiral out in a plane and a $1D$ stable manifold. 
 
As alluded to in \reffig{fig:fine_usquare}, \tSP{N1} and \tSP{N2} 
sit near the center of the swirl of green coming from the unstable 
direction of \tSP{1}. To better understand what is happening here, 
referring to \reffig{fig:hetero1}, we compute the  stable and 
unstable manifolds of \tSP{N1} and \tSP{N2}, where we use the shifted 
translation of \tSP{N2}, along with the stable and unstable manifolds of 
\tSP{3}. The blue surface is formed by the overlap of trajectories 
starting along the unstable manifold of \tSP{3} and the stable manifolds 
of \tSP{N1} and \tSP{N2}.  We see that the stable manifold of \tSP{3} 
(shown by the red curves) corresponds with the unstable manifolds of 
\tSP{N1} and \tSP{N2}, thus we have \emph{{\hc}s} from $\tSP{N1}\to\tSP{3}$
 and $\tSP{N2}\to\tSP{3}$! The thick appearance of the red curves 
is simply so that they can be seen within the blue surface. They are 
actually just a single trajectory. 
 
Next we bring trajectories originating near \tSP{1} into the picture to 
see how the manifolds of this {\stagp} connect with those in 
\reffig{fig:hetero1}, producing the full dynamical portrait within 
$\Omega$.  The result is shown in \reffig{fig:hetero2}. Compare to 
\reffig{fig:stagps_label2} to see the locations of the {\stagp}s. 
The relation of the stable manifold of \tSP{1} (yellow curve) and the 
trajectories that are driven away from \tSP{1} in the unstable direction 
(green) to those of the blue surface is significant. These 
trajectories tightly hug the blue surface as they spiral around it, 
appearing to be shielded from entering the volume it encompasses. This 
has important implications for the consideration of fluid mixing 
within {\pCf},  showing that it is difficult to achieve a 
uniformly mixed space for this particular Eulerian equilibrium; a blob of ink that 
starts outside of the blue surface will have a difficult time ever 
entering the region! 
 
One merely translates the image in \reffig{fig:hetero2} in the $x$ 
direction by an amount $L_{x}$ to give a complete picture in any periodic 
cell. The same picture will also occur symmetrically (translated by 
$L_{x}/2$ and $L_{z}/2$) in the left half of the box. 


\subsection{Eulerian equilibrium {\tEQeight}: additional symmetries}
\label{sect:EQ8}

Having analyzed the upper branch Eulerian equilibrium {\tEQtwo}, we next look at 
{\tEQeight}, another Eulerian equilibrium velocity field of {\pCf} which exhibits 
turbulent behavior at a lower Reynolds number, 270.


We start once again with a cleverly chosen grid of initial trajectories 
to get a feel for the significant structures in the flow. The grid is in 
a plane at $x = L_{x}/2$. The result, after a short integration time, is 
shown in \reffig{fig:EQ8_grid1}. This perspective view already shows 
us quite a bit of information. Once again we have symmetries abound, and 
we know from the discussion in \refsect{s:symm_stag} that there will be 
at least 8 {\stagp}s \tSP{1}--\tSP{8}.  Another interesting feature of this 
plot is the four vortical structures on the left half. One final 
noteworthy point from the figure is the appearance of a perfect line 
segment connecting two of the {\stagp}s, which happen to be \tSP{1} and 
\tSP{2}. This strongly suggests a heteroclinic connection between these 
two \stagp s. To confirm, we compute the eigenvalues and eigenvectors of 
the \velgradmat. For \tSP{1}, there is indeed a real, unstable eigenvector 
pointing along (0,0,1) and for \tSP{2} there is a real, stable eigenvector 
pointing along (0,0,1). This, together with the plot, numerically 
confirms the existence of the heteroclinic trajectory. The same result  
holds for the shifted pair at $x = 0$. The rest of the 
eigenvalues/eigenvectors are given below. We note that for {\tEQeight} there 
is a {\hc} which is a simple horizontal line connecting the pair of 
trivial \stagp s in the \emph{spanwise} direction, whereas for \tEQtwo\ 
the connection was some arbitrary-looking curve in the 
\emph{streamwise} direction connected to a nontrivial \stagp. 
Factorization of the \tSP{1} and \tSP{2} stability eigenspaces for {\tEQeight} 
occurs because the spanwise $z$ direction is a $1D$ flow-invariant 
subspace at the \stagp s \citep{SiCvi10}. That ensures the simplicity of 
the \hec. 

{\tEQeight}, \tSP{1}: There are two real, positive eigenvalues
 and one real, negative eigenvalue.
\bea
\left(
    \eigExp[1],\eigExp[2],\eigExp[3]
\right) &=&
      (0.363557,0.227831,-0.591389)
\label{E8SP1} \\
\left(
    \jEigvec[1],\jEigvec[2],\jEigvec[3]
\right) &=&
\left(
    \left[\begin{array}{c}
             {0} \cr
             {0} \cr
             {1}
 \end{array}\right] \,,
    \left[\begin{array}{c}
             {-0.733415} \cr
             {-0.679780} \cr
             {0}
 \end{array}\right] \,,
    \left[\begin{array}{c}
             {0.991005} \cr
             {0.133824} \cr
             {0}
 \end{array}\right]
\right) \,.
\nnu
\eea

{\tEQeight}, \tSP{2}: There are two real, positive eigenvalues
 and one real, negative eigenvalue.
\bea
\left(
    \eigExp[1],\eigExp[2],\eigExp[3]
\right) &=&
      (0.992857,0.255973,-1.248830)
\label{E8SP2} \\
\left(
    \jEigvec[1],\jEigvec[2],\jEigvec[3]
\right) &=&
\left(
    \left[\begin{array}{c}
             {~0.116961} \cr
             {-0.993136} \cr
             {0}
 \end{array}\right] \,,
    \left[\begin{array}{c}
             {0.957795} \cr
             {0.287450} \cr
             {0}
 \end{array}\right] \,,
    \left[\begin{array}{c}
             {0} \cr
             {0} \cr
             {1}
 \end{array}\right]
\right) \,.
\nnu
\eea


   \begin{figure}
\includegraphics[width=0.9\textwidth]{EQ8_grid1.jpg}
  \caption{
    Grid of initial points in the $[y,z]$ plane, centered at $x = L_x/2$; 
    integrated to produce tracer particle trajectories for {\tEQeight}. 
   }
  \label{fig:EQ8_grid1}
 \end{figure}


Equilibrium {\tEQeight} (as well as {\tEQsev}, not discussed here), possesses 
additional symmetries compared to {\tEQtwo}. {\tEQtwo} is in the $S$-invariant 
subspace of velocity fields and {\tEQeight} is in $S_8$ (\refsect{s:PCF_symm}
and \refsect{s:symm_stag}). 

From \refeq{second_condition} and \refeq{s3lagrange} we know then that 
for {\tEQeight} we will have the additional {\stagp}s: 
 \bea
  \xSP{5} &=& (L_x/4,0,0) \continue
  \xSP{6} &=& (3L_x/4,0,0) \continue
  \xSP{7} &=& (L_x/4,0,L_z/2)  \\
  \xSP{8} &=& (3L_x/4,0,L_z/2) \nnu
 \,.
\eea
Interestingly these were actually discovered numerically \emph{before} 
the symmetry arguments were understood. A Newton search on regions of 
very low velocity for {\tEQeight} revealed that $(L_x/4,0,L_z/2)$ and 
$(3L_x/4,0,L_z/2)$ are \stagp s. From this, one may deduce that symmetry 
$s_5$ must hold, and it can then be checked that at any position the 
velocity field is indeed invariant under $s_4$ and $s_5$. 

Stability analysis of the additional set of {\stagp}s for {\tEQeight} gives the
following.

 \tSP{5}: There is one real, positive eigenvalue
 and a complex pair with negative real part.
\bea 
  \eigExp[1] &=& 0.03109 \,,\quad \jEigvec[1] =
\left[\begin{array}{c}
             {0.85275} \cr
             {0.41774} \cr
             {-0.31355} \cr
\end{array}\right]
   \continue
\{ \eigExp[2],\eigExp[3]\}
   &=& \eigRe[2] \pm i \,\eigIm[2] =  -0.01555 \pm i\, 0.59385
   \label{EQSP5eigs}\\
\jEigvec[2]  &=& 
\left[\begin{array}{c}
             {~0.24762} \cr
             {-0.31442} \cr
             {~0.69906} \cr
\end{array}\right]
    \,,\quad
\jEigvec[3] =
\left[\begin{array}{c}
             {-0.20793} \cr
             {~0.55489} \cr
             {~0} \cr
\end{array}\right]
\,.
\nnu
\eea
 We have a $1D$ unstable manifold and a $2D$ inward-spiral
stable manifold. All four of the new points have the same
eigenvalues. \tSP{5} and \tSP{8} have the same eigenvectors, as do \tSP{6}
and \tSP{7} whose eigenvectors differ from \tSP{5} only by the sign of
the third component for \jEigvec[1] and by the sign of the first and
second components for \jEigvec[2] and \jEigvec[3].

As a final interesting consequence of numerically searching for \stagp s 
for {\tEQeight}, the figures produced by plotting gridpoints where velocity is 
small, using a cutoff value of $|\mathbf{u}|^{2}$ which is too large to 
actually be useful for finding \stagp s, we instead find a plot showing 
more intricate patterns in the flow. \reffig{fig:usquare_EQ8_1} 
shows a $3D$ perspective view of these points, and 
\reffig{fig:usquare_EQ8_2} shows the projection of 
\reffig{fig:usquare_EQ8_1} onto the $xz$ plane. This 
volume-preserving flow (area preserving in Poincar\'e sections) may have 
invariant tori which, being quasiperiodic, would not be detected by the 
{\stagp} searching routines. Though the structures in the 
projection plot in \reffig{fig:usquare_EQ8_2} are not actual tracer 
trajectories, they are suggestive that a search for such invariant tori 
in future work may be a fruitful endeavor.  

\begin{figure}
\centering
    \begin{subfigure}{0.9\textwidth}
    \includegraphics[width=1.0\textwidth]{usquare_EQ8_cute1.jpg}
      \caption{
        Perspective view.
       }
      \label{fig:usquare_EQ8_1}
    \end{subfigure}

    \begin{subfigure}{0.9\textwidth}
    \includegraphics[width=1.0\textwidth]{usquare_EQ8_cute2.jpg}
      \caption{
       Projection onto the $xz$
       plane.
       }
      \label{fig:usquare_EQ8_2}
    \end{subfigure}  
    \caption{
A plot of points where the velocity field falls below a small cutoff for 
{\tEQeight}, showing interesting structures in the flow. 
       }
    \label{fig:usquare_both}
 \end{figure}


\section{Conclusion}
\label{s:conclusion}

We have taken a step towards a deeper understanding of the  turbulent 
fluid flow in a $3D$ system  by studying 
tracer trajectory dynamics in the Lagrangian frame for {\pC} geometry. Potential 
applications that could follow from having a grasp of the Lagrangian 
dynamics and being able to accurately compute tracer particle 
trajectories are wide-ranging: velocity profile statistics or correlation 
functions taken over an ensemble of particle trajectories within 
different regions, calculations of mixing time and diffusion properties 
for the flow, Lyapunov exponents and material stretching, striation 
thickness, among others, are some of the various possible measures of 
chaotic advection that could be investigated.  

By extending the dynamical 
systems methods that are often confined to simpler $2D$ systems to the 
$3D$ world of {\pCf}, we encounter complex structures that 
partition the physical space of the fluid into regions which exhibit 
distinct types of motion and allow us to visualize the fundamental 
motions driven by trajectories which lie close to invariant manifolds. 
Relying on the symmetries of the geometry to shine light upon the 
situation and guide us, we are able to construct phase portraits for 
{\pC} Eulerian equilibria starting with the identification and stability 
determination of stagnation or fixed points of the system. As a turbulent fluid evolves by visiting equilibria and periodic solutions in a recurrent manner, understanding the limitations on mixing between different regions of each equilibrium solution provides important information for the overall ability of a fluid to become thoroughly mixed.  Future work could thus extend these analyses to forming a dynamical portrait for all invariant solutions of plane
 Couette flow, or apply the same methods in other fluid systems which likely posses symmetries.

\ifboyscout{\color{blue}
%%% Acknowledgements %%%
\section{Acknowledgments}
Acknowledge group members not included as authors.

\appendix

\section{Computational details}
\label{appe:details}

Blah, blah
        }\fi %end\color{blue}

\bibliographystyle{jfm} %% PC 2024-12-20 was {unsrt} %prsty} %apsrev} %plain}
    \ifsubmission
\section*{References}
    % from ctan.org/tex-archive/biblio/bibtex/contrib/iopart-num/ :
\bibliography{../../bibtex/pipes}{99} %% PC 2024-12-20
% \bibliography{pipes}  % Use when JRE loading in Overleaf
    \else
\bibliography{../../bibtex/pipes}{99} %% PC 2024-12-20
%  \bibliography{pipes}  % Use when JRE loading in Overleaf
%\printbibliography[
%heading=bibintoc,
%title={References}
%				  ] %, type=online]  % if not using default "Bibliography"
    \fi


\end{document}












