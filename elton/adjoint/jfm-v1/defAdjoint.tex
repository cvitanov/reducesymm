% elton/adjoint/defAdjoint.tex for WFSBC15.tex
% $Author: predrag $ $Date: 2015-07-27 08:44:09 -0400 (Mon, 27 Jul 2015) $

% initial version was a copy of pipes/slice/defSlice.tex for slice.tex
%%%%%%%%%%%%%%%%%%%%%% sliced pipe SPECIFIC %%%%%%%%%%%%%%%%%%%%%%%%%%%%%%%

\ifdraft    % display comments in text
   \newcommand{\PublicPrivate}[2]
       {\marginpar{\color{blue}$\Downarrow$\footnotesize PRIVATE}%
       {\color{blue}#2}%
       \marginpar{\color{blue}$\Uparrow$\footnotesize PRIVATE}}
   \newcommand{\toCB}{\marginpar{\footnotesize 2CB}}    % to include in ChaosBook
   \newcommand{\inCB}{\marginpar{\footnotesize in CB}}  % transferred to ChaosBook
   \newcommand{\PC}[1]{$\footnotemark\footnotetext{PC: #1}$}
   \newcommand{\PCedit}[1]{{\color{red}#1}}
   \newcommand{\MA}[1]{$\footnotemark\footnotetext{MA: #1}$}
   \newcommand{\MAedit}[1]{{\color{green}#1}}
   \newcommand{\APW}[1]{$\footnotemark\footnotetext{APW: #1}$}
   \newcommand{\APWedit}[1]{{\color{green}#1}}
   \newcommand{\RG}[1]{$\footnotemark\footnotetext{FW: #1}$}
   \newcommand{\MMF}[1]{$\footnotemark\footnotetext{Mohammad: #1}$}
   \newcommand{\MMFedit}[1]{{\color{green}#1}}
\newcommand{\PCpost}[2]{\item[#1 Predrag ] {#2}}
\newcommand{\APWpost}[2]{\item[#1 Ashley ] {#2}}
\newcommand{\KYSpost}[2]{\item[#1 Kimberly ] {#2}}
\newcommand{\BEpost}[2]{\item[#1 Bruno ] {#2}}
\newcommand{\FFpost}[2]{\item[#1 Franco ] {#2}}
\newcommand{\MMFpost}[2]{\item[#1 Mohammad ] {#2}}
\newcommand{\AFpost}[2]{\item[#1 Adam ] {#2}}
\newcommand{\YLpost}[2]{\item[#1 Lan ] {#2}}
   \newcommand{\file}[1]{$\footnotemark\footnotetext{{\bf file} #1}$}
   \newcommand{\mycomment}[2]{\noindent \textbf{\underline{#1}}: \emph{#2}}
   \newcommand{\edit}[1]{{\color{blue}#1}} % for referees
\else   % drop comments
   \newcommand{\PublicPrivate}[2]{#1}
   \newcommand{\toCB}{}
   \newcommand{\inCB}{}
   \newcommand{\PC}[1]{}
   \newcommand{\JFG}[1]{}
   \newcommand{\MA}[1]{}
   \newcommand{\APW}[1]{}
%   \newcommand{\PCedit}[1]{{\color{magenta}#1}}  % for referees
   \newcommand{\PCedit}[1]{#1}
   \newcommand{\MAedit}[1]{#1}
%   \newcommand{\APWedit}[1]{{\color{green}#1}} % for referees
   \newcommand{\APWedit}[1]{#1}
   \newcommand{\RG}{}
  \newcommand{\MMF}[1]{}
  \newcommand{\MMFedit}[1]{#1}
   \newcommand{\file}[1]{}
   \newcommand{\mycomment}[2]{}
   \newcommand{\edit}[1]{{\color{blue}#1}} % for referees
%   \newcommand{\edit}[1]{#1}               % for the journal

\fi %%%%% COMMENTS END %%%%%%%%%%%%%%%

\ifcolorfigs            %% switch for color/BW figure captions
  %%%%%%%%%%%%%%%%%%%%%% Weblinks in PDF %%%%%%%%%%%%%%%%%%%
  \newcommand{\wwwcb}[1]{{\tt \href{http://ChaosBook.org#1}
         {ChaosBook.org#1}}}
  \newcommand{\wwwQFT}[1]{
         {\tt \href{http://ChaosBook.org/FieldTheory#1}
         {ChaosBook.org/\-Field\-Theory#1}}}
  \newcommand{\wwwnsQFT}[1]{
         {\tt \href{http://ChaosBook.org/FieldTheory#1}
         {ChaosBook.org/\-Field\-Theory#1}}}
  \newcommand{\weblink}[1]{{\tt \href{http://#1}{#1}}}
  \newcommand{\HREF}[2]{{\href{#1}{#2}}}
  \newcommand{\mpArc}[1]{
         {\tt \href{http://www.ma.utexas.edu/mp_arc-bin/mpa?yn=#1}
         {\goodbreak mp\_arc~#1}}}
  \newcommand{\arXiv}[1]{
         {\tt \href{http://arXiv.org/abs/#1}{\goodbreak #1}}}
\else
  %%%%%%%%%%%%%%%%%%%%%% No weblinks in PDF %%%%%%%%%%%%%%%%%%%
  \newcommand{\wwwcb}[1]{{\tt ChaosBook.org#1}}
  \newcommand{\wwwQFT}[1]{{\tt ChaosBook.org/\-Field\-Theory#1}}
  \newcommand{\wwwnsQFT}[1]{{\tt ChaosBook.org/\-Field\-Theory#1}}
  \newcommand{\weblink}[1]{{\tt #1}}
  \newcommand{\HREF}[2]{{#2}}
  \newcommand{\mpArc}[1]{{\tt \goodbreak mp\_arc~#1}}
  \newcommand{\arXiv}[1]{{\tt \goodbreak #1}}
\fi

\newcommand{\rf}     [1] {~\cite{#1}}
%   Phys Rev format:
%\newcommand{\refref} [1] {ref.~\cite{#1}}
%\newcommand{\refRef} [1] {Ref.~\cite{#1}}
%\newcommand{\refrefs}[1] {refs.~\cite{#1}}
%\newcommand{\refRefs}[1] {Refs.~\cite{#1}}
%   JFM format:
\newcommand{\refref} [1] {\cite{#1}}
\newcommand{\refRef} [1] {\cite{#1}}
\newcommand{\refrefs}[1] {\cite{#1}}
\newcommand{\refRefs}[1] {\cite{#1}}

\newcommand{\refeq}  [1] {(\ref{#1})}
\newcommand{\refeqs} [2]{(\ref{#1}--\ref{#2})}
\newcommand{\eqn}[1]{eqn.\ {\ref{#1}}}
\newcommand{\Eqn}[1]{Eqn.\ {\ref{#1}}}
\newcommand{\refpage}[1] {p.~\pageref{#1}}
\newcommand{\reffig} [1] {figure~\ref{#1}}
\newcommand{\reffigs} [2] {figures~\ref{#1} and~\ref{#2}}
\newcommand{\refFig} [1] {Figure~\ref{#1}}
\newcommand{\refFigs} [2] {Figures~\ref{#1} and~\ref{#2}}
\newcommand{\reftab} [1] {table~\ref{#1}}
\newcommand{\refTab} [1] {Table~\ref{#1}}
\newcommand{\reftabs}[2] {tables~\ref{#1} and~\ref{#2}}
\newcommand{\refsect}[1] {\S\,\ref{#1}}
\newcommand{\refsects}[2] {\S\,\ref{#1} and \S\,\ref{#2}}
\newcommand{\refSect}[1] {\S\,\ref{#1}}
\newcommand{\refchap}[1] {chapter~\ref{#1}}
\newcommand{\refChap}[1] {Chapter~\ref{#1}}
\newcommand{\refchaps}[2] {chapters~\ref{#1} and \ref{#2}}
\newcommand{\refchaptochap}[2] {chapters~\ref{#1} to \ref{#2}}
\newcommand{\refappe}[1] {appendix~\ref{#1}}
\newcommand{\refappes}[2] {appendices~\ref{#1} and \ref{#2}}
\newcommand{\refAppe}[1] {Appendix~\ref{#1}}
\newcommand{\refrem} [1] {remark~\ref{#1}}
\newcommand{\refexam}[1] {example~\ref{#1}}
\newcommand{\refExam}[1] {Example~\ref{#1}}
\newcommand{\refexer}[1] {exercise~\ref{#1}}
\newcommand{\refExer}[1] {Exercise~\ref{#1}}
\newcommand{\refsolu}[1] {solution~\ref{#1}}

%%%%%%%%%%%%%%% EQUATIONS %%%%%%%%%%%%%%%%%%%%%%%%%%%%%%%
\newcommand{\beq}{\begin{equation}}
\newcommand{\continue}{\nonumber \\ }
\newcommand{\nnu}{\nonumber}
\newcommand{\eeq}{\end{equation}}
\newcommand{\ee}[1] {\label{#1} \end{equation}}
\newcommand{\ceq}{\nonumber \\ & & }
\newcommand{\bea}{\begin{eqnarray}}
\newcommand{\eea}{\end{eqnarray}}
\newcommand{\barr}{\begin{array}}
\newcommand{\earr}{\end{array}}

%%%%%%%%%%%%%%  Abbreviations %%%%%%%%%%%%%%%%%%%%%%%%%%%%%%%%%%%%%%%%
%%% APS (American Physiology Society, it seems) style:
%%%     Latin or foreign words or phrases should be roman, not italic.

\newcommand{\etc}{{etc.}}       % APS
\newcommand{\etal}{{\em et al.}}    % etal in italics, APS too
\newcommand{\ie}{{i.e.}}        % APS
\newcommand{\cf}{{\em cf.}}     % APS
\newcommand{\eg}{{e.g.}}        % APS

\ifJFM                          % J Fluid Mech macros
\renewcommand\etal{\mbox{\textit{et al.}}}
\renewcommand\etc{etc.\ }
\renewcommand\eg{e.g.\ }
\else
\fi

%%%%%%%%%%%%%%% LIE GROUP PARAMETRIZATIONS %%%%%%%%%%%%%%%%%%%%%%
\newcommand{\gSpace}{\ensuremath{{\bf \phi}}}   % MA group rotation parameters
\newcommand{\velRel}{\ensuremath{c}}    % relative state or phase velocity
\newcommand{\angVel}{angular velocity}      % Froehlich
\newcommand{\angVels}{angular velocities}   % Froehlich
\newcommand{\phaseVel}{phase velocity}      % pipe slicing
\newcommand{\phaseVels}{phase velocities}   % pipe slicing
\newcommand{\PhaseVel}{Phase velocity}      % pipe slicing
\newcommand{\PhaseVels}{Phase velocities}   % pipe slicing

%%%%%%%%%%%% Froehlich's FAVORITE MACROS %%%%%%%%%%

	% without large brackets:
\newcommand{\braket}[2]
		   {\langle{#1}\vphantom{#2}|\vphantom{#1}{#2}\rangle}
\newcommand{\bra}[1]{\langle{#1}\vphantom{ }|}
\newcommand{\ket}[1]{|\vphantom{}{#1}\rangle}

\newcommand{\dual}[1]{{#1}^T}		% SO(n) case
\newcommand{\sspSing}{\ensuremath{\ssp^*}} 	% inflection point
\newcommand{\sspRSing}{\ensuremath{\sspRed^*}} 	% inflection point, reduced space

%%%%%%%%%%%% SIMINOS' FAVORITE MACROS %%%%%%%%%%

\newcommand{\po}{periodic orbit}
\newcommand{\Po}{Periodic orbit}
\newcommand{\rpo}{rela\-ti\-ve periodic orbit}
%   \newcommand{\rpo}{equivariant periodic orbit}
\newcommand{\Rpo}{Rela\-ti\-ve periodic orbit}
%   \newcommand{\Rpo}{Equivariant periodic orbit}
\newcommand{\UPO}{unstable periodic orbit}
\newcommand{\reducedsp}{reduced state space}
\newcommand{\Reducedsp}{Reduced state space}
\newcommand{\fixedsp}{fixed-point subspace}
\newcommand{\Fixedsp}{Fixed-point subspace}
\newcommand{\slice}{slice}
\newcommand{\Slice}{Slice}
\newcommand{\mslices}{method of slices}
\newcommand{\Mslices}{Method of slices}
\newcommand{\mframes}{method of moving frames}
\newcommand{\Mframes}{Method of moving frames}
\newcommand{\fFslice}{first Fourier mode slice}
\newcommand{\FFslice}{First Fourier mode slice}
\newcommand{\chartBord}{chart border}
\newcommand{\ChartBord}{Chart border}
\newcommand{\poincBord}{section border}
\newcommand{\PoincBord}{Section border}
% \newcommand{\poincBord}{\PoincSec\ border}
% \newcommand{\PoincBord}{\PoincSec\ border}
% \newcommand{\poincBord}{border of transversality}
\newcommand{\template}{template} % {slice-fixing point} % {reference state}
\newcommand{\sliceBord}{slice border}
\newcommand{\SliceBord}{Slice border}
\newcommand{\Sset}{Inflection hyperplane}
\newcommand{\sset}{inflection hyperplane} 	% {singularity hyperplane}
\newcommand{\cLe}{complex Lorenz equations}
\newcommand{\cLf}{complex Lorenz flow}
\newcommand{\CLe}{Complex Lorenz equations}
\newcommand{\CLf}{Complex Lorenz flow}

\newcommand{\zeit}{\ensuremath{t}}  %time variable
\newcommand{\normVec}{\ensuremath{\mathbf{n}}}    % group orbit curvature normal
\newcommand{\sliceTan}[1]{\ensuremath{\mathbf{t}{}'_{#1}}}    % group orbit tangent at slice-fixing
\newcommand{\groupTan}{\ensuremath{\mathbf{t}}}    % group orbit tangent
\newcommand{\Lg}{\ensuremath{\mathbf{T}}}   % FrCv11.tex Lie algebra generator
\newcommand{\LieEl}{\ensuremath{g}}  % Predrag Lie group element
\newcommand{\timeStep}{\ensuremath{{\delta \tau}}}  %integration step
\newcommand{\id}{{\ \hbox{{\rm 1}\kern-.6em\hbox{\rm 1}}}}

\newcommand{\On}[1]{\ensuremath{\textrm{O}(#1)}}
\newcommand{\SOn}[1]{\ensuremath{\textrm{SO}(#1)}}         % in DasBuch
%\newcommand{\Dn}[1]{\ensuremath{\mathbf{D}_{#1}}    % in Siminos thesis
\newcommand{\Dn}[1]{\ensuremath{\textrm{D}_{#1}}}              % in DasBuch
%\newcommand{\Zn}[1]{\ensuremath{\mathbf{Z}_{#1}}}    % in Siminos thesis
\newcommand{\Zn}[1]{\ensuremath{\textrm{C}_{#1}}}              % in DasBuch

\newcommand{\pSRed}{\ensuremath{\hat{\cal M}}} % reduced state space
\newcommand{\sspRed}{\ensuremath{\hat{\ssp}}}    % reduced state space point, experiment
\newcommand{\velRed}{\ensuremath{\hat{\vel}}}    % ES reduced state space velocity
%\newcommand{\slicep}{\ensuremath{\hat{\ssp}'}}   % slice-fixing point, experimental
\newcommand{\slicep}{\ensuremath{\ssp'}}   % slice-fixing point, experimental, no hat
\newcommand{\Group}{\ensuremath{G}}         % Predrag Lie or discrete group
\newcommand{\Fix}[1]{\ensuremath{\mathrm{Fix}\left(#1\right)}}

%%%%%%%%%%%%    PREDRAG'S FAVORITE MACROS %%%%%%%%%%%%%

\newcommand{\stabmat}{stability matrix}     % stability matrix, velocity gradients
\newcommand{\Stabmat}{Stability matrix}     % Stability matrix
\newcommand{\stabmats}{stability matrices}
\newcommand{\jacobianM}{Jacobian matrix}  % back to Predrag's name 20oct2009
\newcommand{\jacobianMs}{Jacobian matrices}   % matrices
\newcommand{\JacobianM}{Jacobian matrix} %
\newcommand{\JacobianMs}{Jacobian matrices}  %
\newcommand{\Poincare}{Poincar\'e }
\newcommand{\PoincSec}{Poincar\'e section}
\newcommand{\NS}{Navier--Stokes}
\newcommand{\NSe}{Navier--Stokes equations}
\newcommand{\NSE}{Navier--Stokes Equations}
\newcommand{\KS}{Kuramoto--Sivashinsky}
\newcommand{\Reynolds}{\textit{Re}}  % Reynolds number
\newcommand{\pCf}{plane Couette flow}
\newcommand{\PCf}{Plane Couette flow}
\newcommand{\eqv}{equilib\-rium}
\newcommand{\Eqv}{Equilib\-rium}
\newcommand{\eqva}{equilib\-ria}
\newcommand{\Eqva}{Equilib\-ria}
\newcommand{\reqv}{travelling wave}
\newcommand{\Reqv}{Travelling wave}
\newcommand{\reqva}{travelling waves}
\newcommand{\Reqva}{Travelling waves}
\newcommand{\reqvD}{travelling-wave}
\newcommand{\ReqvD}{Travelling-wave}
\newcommand{\reqvaD}{travelling-waves}
\newcommand{\ReqvaD}{Travelling-waves}
\newcommand{\REqvaD}{Travelling-Waves}

\newcommand{\Mvar}{\ensuremath{A}}  % stability matrix
\newcommand{\jMps}{\ensuremath{J}}   % jacobian matrix, phase space/state space
\newcommand{\matId}{\ensuremath{{\bf 1}}}      % matrix identity
\newcommand\stagn{q}      %equilibrium/stagnation point suffix
\newcommand{\jEigvec}[1][]{\ensuremath{{\bf e}^{(#1)}}} % right jacobiam eigenvector
\newcommand{\jEigvecT}[1][]{\ensuremath{{\bf e}_{(#1)}}} % left jacobiam eigenvector

%%% 3D physical flow
\newcommand{\stagp}{stagnation point}
\newcommand{\Stagp}{Stagnation point}
\newcommand{\relstagp}{traveling stagnation point}
\newcommand{\Relstagp}{Traveling stagnation point}
\newcommand{\cohStr}{coherent structure}
\newcommand{\recurrStr}{recurrent coherent structure}
\newcommand{\RecurrStr}{Recurrent coherent structure}
\newcommand{\stateDsp}{state-space}
\newcommand{\StateDsp}{State-space}
\newcommand{\Statesp}{State space}
\newcommand{\statesp}{state space}
\newcommand{\nameit}{E}         % equilibrium label

\newcommand{\be}{{\bf e}}
\newcommand{\bu}{{\bf u}}
\newcommand{\bx}{{\bf x}}
\newcommand{\Gpipe}{\ensuremath{\Gamma}} % Hoyle notation, equivariant symmetry group
\newcommand{\tEQ}{\ensuremath{{\text{EQ}}}}

% Redefine using mathrm, it is a label not a math symbol
\newcommand{\EQV}[1]{\ensuremath{\mathrm{E}_{#1}}}
% \newcommand{\EQV}[1]{\ensuremath{\text{EQ}_{#1}}} % ELIMINATE
% E_0: u = 0 - trivial equilibrium
% E_1,E_2,E_3, for 1,2,3-wave equilibria
\newcommand{\REQV}[2]{\ensuremath{\mathrm{TW}_{#1#2}}} % #1 is + or -
% \newcommand{\REQV}[2]{{\ensuremath{\text{TW}_{#1#2}}}}
% TW_1^{+,-} for 1-wave traveling waves (positive and negative velocity).
\newcommand{\PO}[1]{\ensuremath{\mathrm{PO}_{#1}}}
% PO_{period to 2-4 significant digits} - periodic orbits
% \newcommand{\PO}[1]{\ensuremath{\text{P$#1$}}}
\newcommand{\RPO}[1]{\ensuremath{\mathrm{RPO}_{#1}}}
% \newcommand{\RPO}[1]{\ensuremath{\text{RP$#1$}}}
% RPO_{period to 2-4 significant digits} - relative PO.  We use ^{+,-}
% to distinguish between members of a reflection-symmetric pair.
% Gibson likes:
\renewcommand{\tEQ}{\ensuremath{\mathrm{EQ}}}


% isotropy subgroup $H \incl G$:
\newcommand{\isotropyG}[1]{\ensuremath{H_{\text{\tiny #1}}}}
\newcommand{\bCell}{\ensuremath{\Omega}}
\newcommand{\Norm}[1]{\|{#1}\|}

%%%%%multiletter symbols
\newcommand\Real{\mbox{Re}\,} % cf plain TeX's \Re, not Reynolds number
\newcommand\Imag{\mbox{Im}\,} % cf plain TeX's \Im

%%%%%%%%%%%%%%% Sundry symbols within math eviron.: %%%%%%%%%%%%
\newcommand\flow[2]{{f^{#1}(#2)}}
\newcommand\timeflow{{f^t}}
\newcommand{\reals}{\mathbb{R}}
\newcommand{\PoincS}{{\cal P}}     % symbol for Poincare section
\newcommand{\PoincM}{{P}}      % symbol for Poincare map
\newcommand{\PoincC}{{U}}      % symbol for Poincare constraint function
\newcommand{\pd}[2]{\frac{\partial #1}{\partial #2}}
\newcommand{\pde}{\partial}
\newcommand{\jMP}{{\bf \hat{J}}}   % jacobiam matrix, Poincare return
\newcommand{\monodromy}{{\bf J}}   % monodromy matrix, full Poincare cut
                   % Fredholm det jacobian weight:
\newcommand{\oneMinJ}[1]
           {\left|\det\!\left(\matId-\monodromy_p^{#1}\right)\right|}

\newcommand{\dmn}{-dimensional} %{\!-\!dimensional}

\newcommand{\obser}{a}      % an observable from state space to R^n
\newcommand{\Obser}{A}      % time integral of an observable
\newcommand{\expct}    [1]{\left\langle {#1} \right\rangle}
\newcommand{\spaceAver}[1]{\left\langle {#1} \right\rangle}
\newcommand{\timeAver} [1]{\overline{#1}}
\newcommand{\Lop}{{\cal L}}    % evolution operator
%\renewcommand\Im{{\rm Im\,}}
%\renewcommand\Re{{\rm Re\,}}
\renewcommand{\det}{\mbox{\rm det}\,}
\newcommand{\Det}{\mbox{\rm Det}\,}
\newcommand{\tr}{\mbox{\rm tr}\,}
\newcommand{\Tr}{\mbox{\rm tr}\,}
\newcommand{\sign}[1]{\sigma_{#1}}

\newcommand{\Refl}{\ensuremath{\mathbf{R}}}
\newcommand{\Shift}{\ensuremath{\mathbf{S}}}
\newcommand{\shift}{\ensuremath{\ell}}
\newcommand{\trHalf}[1]{\ensuremath{\tau_{#1}}}    % 1/2 cell translation
% \newcommand{\trHalf}[1]{\tau_{#1}^{1/2}}    % 1/2 cell translation
\newcommand{\trDiscr}[2]{\tau_{#1}^{#2}}    % discrete cell translation 1/4, ...
\newcommand\period[1]{{T_{#1}}}         %continuous cycle period
\newcommand{\cl}[1]{{n_{#1}}}   % discrete length of a cycle, Predrag
\newcommand{\pS}{{\cal M}}          % symbol for state space
% \newcommand\pSpace{x}     % phase space x=(q,p) coordinate
\newcommand{\ssp}{a}            % state space point
\newcommand{\vel}{\ensuremath{v}}   % state space velocity
\newcommand\velField[1]{{F(#1)}}    % Gibson statespace velocity field
\newcommand\vField{\ensuremath{{\bf F}}} % yet another Gibson statesp vel field
\newcommand\vCM{mean velocity}  % or `center-of-mass velocity?' `bulk momentum?'
\newcommand\xInit{{a_0}}        %initial x
\newcommand{\deltaX}{{\delta a}}                %trajectory displacement

\newcommand{\eigExp}[1][]{
\ifthenelse{\equal{#1}{}}{\ensuremath{\lambda}}{\ensuremath{\lambda^{(#1)}}}
                        }
\newcommand{\eigRe}[1][]{
\ifthenelse{\equal{#1}{}}{\ensuremath{\mu}}{\ensuremath{\mu^{(#1)}}}
                        }
\newcommand{\eigIm}[1][]{
  \ifthenelse{\equal{#1}{}}{\ensuremath{\omega}}{\ensuremath{\omega^{(#1)}}}
            }


%%%%%%%%%%%%%%% ASH'S JFM-FRIENDLY MACROS %%%%%%%%%%%%

\newcommand{\bnabla}{\mbox{\boldmath $\nabla$}}
\renewcommand{\vec}[1]{\mbox{\boldmath $#1$}}
\newcommand{\mat}[1]{\mbox{\sls #1}}


% % % % % % % % Faraz's FAVORITE MACROS % % % % % % % %
\newcommand{\vc}{\mathbf}
\newtheorem{thm}{Theorem}
\newtheorem{rem}{Remark}
