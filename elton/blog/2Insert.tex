% $Author: predrag $ $Date: 2014-06-14 17:52:48 -0400 (Sat, 14 Jun 2014) $
%%%%%%%%%%%%%%%%%%%%%%%%%%%%%%%%%%%%%%%%%%%%%%%%%%%%%%%%%%%%%%%%%%%%%%
% from ChaosBook \Chapter{smale}{ 5jun2005}{Qualitative dynamics, for cyclists}


\section{Going global: Stable/unstable manifolds}
\label{s-StabUnstManif}


\index{stable!manifold}
\index{unstable!manifold}
\index{manifold!stable}

% \inDepth{s-SpatOrd}

In the linear approximation,
the \jacobianM\ $\monodromy^{t}$
describes the shearing of an infinitesimal neighborhood in
after a finite time $t$.
Its eigenvalues and eigendirections
describe deformation of an initial infinitesimal sphere of neighboring
trajectories into an ellipsoid time $t$ later.
Nearby trajectories
separate exponentially along the {unstable directions},
approach each other along the {stable directions},
and maintain their distance along the {marginal directions}.

The fixed or periodic point $\pSpace^{*}$
\jacobianM\ $\monodromy_p(\pSpace^{*})$
eigenvectors \refeq{EigsInvar} form a rectilinear
coordinate frame in which the flow into, out of, or encircling the
fixed point is linear in the sense of \refsect{DynLinFlows}.
These eigendirections are numerically continued into global curvilinear
invariant manifolds as follows.


The global continuations of the local stable, unstable eigendirections
are called the {\em stable}, respectively {\em unstable manifolds}.
\index{stable!manifold!flow}
\index{unstable!manifold, flow}
They consist of all points which march into the fixed point forward,
respectively backward in time
\bea
W^s &=& \left\{ \pSpace \in \pS: \flow{t}{\pSpace}-\pSpace^{*} \to 0 {\rm ~as~} t\to\infty
\right\}
\continue
W^u &=& \left\{ \pSpace \in \pS: \flow{{-t}}{\pSpace}-\pSpace^{*} \to 0 {\rm ~as~} t\to\infty
\right\}
\,.
\label{su_flow}
\eea
% For $t\to\infty$ $W^s$, $W^u$ converge to the linearized
% flow eigenvectors $\epsilon^s$, $\epsilon^u$.
The stable/unstable manifolds of a flow are rather hard
to visualize, so as long as
we are not worried about a global property
such as the number of times they wind around a periodic trajectory before
completing a par-course, we might just as well look at their
Poincar\'e section return maps.
Stable, unstable manifolds
%\refeq{su_flow}
for maps are defined by
\index{stable!manifold!map}
\index{unstable!manifold!map}
\PC{define $y^{*}$; note 2-dimensional only example}
\bea
W^s &=& \left\{ \pSpace \in  \PoincS: \flow{n}{\pSpace}-\pSpace^{*} \to 0 {\rm ~as~}
n\to\infty \right\}
\continue
W^u &=& \left\{ \pSpace \in  \PoincS: \flow{-n}{\pSpace}-\pSpace^{*} \to 0 {\rm ~as~}
n\to\infty \right\}
\,.
\label{su_map}
\eea
Eigenvectors (real or complex pairs) of \jacobianM\
$\monodromy_p(\pSpace^{*})$ play a special role - on them
the action of the dynamics is
the linear multiplication by
$\ExpaEig_{i}$ (for a real eigenvector)
along 1-$d$ invariant curve $W^{u,s}_{(i)}$ or spiral in/out action
in a 2-$D$ surface
(for a complex pair).
For $n\to\infty$ a finite segment on  $W^s_{(e)}$, respectively
$W^u_{(c)}$ converges to the linearized
map eigenvector ${\bf e}^{(e)}$, respectively ${\bf e}^{(c)}$. In
this sense each eigenvector defines a (curvilinear) axis of
the stable, respectively unstable manifold.

Conversely, we can use
an arbitrarily small segment of a fixed point eigenvector to construct
a finite segment of the associated manifold.
Precise construction depends on the type of the eigenvalue(s).

{\bf Expanding real and positive eigendirection}.
% from \section{Cycle stabilities are cycle invariants}
% \label{CyclStabCyclInv}
%   Predrag          2feb2005
%
Consider $i$th expanding eigenvalue, eigenvector pair
$(\ExpaEig_{i},{\bf e}_i)$
computed from $\jMps$ evaluated at a cycle point,
%$\pSpace$ \in p,
% \refeq{e-transp}
\beq
\jMps (\pSpace) {\bf e}_i(\pSpace) =
    \ExpaEig_{i} {\bf e}_i (\pSpace)
    \,,  \quad
\pSpace \in p
    \,,  \quad \ExpaEig_{i} > 1
\,.
\ee{e-transpGlob}
% Assume $\ExpaEig_{i} > 1$.
Take an infinitesimal eigenvector
$\epsilon\, {\bf e}_i (\pSpace)$,
$\epsilon\, \ll 1$, and its image
$
\jMps_p (\pSpace) \epsilon\,{\bf e}_i(\pSpace) =
        \ExpaEig_{i}\epsilon\, {\bf e}_i (\pSpace)
\,.
$
Sprinkle the interval
$
|\ExpaEig_{i} -1| \epsilon
$
with a large number of points $\pSpace_m$, equidistantly spaced
on logarithmic scale
$
\ln|\ExpaEig_{i} -1| + \ln\epsilon
\,.
$
The successive images of these points
$f(\pSpace_j)$,
$f^2(\pSpace_j)$,
$\cdots$,
$f^m(\pSpace_j)$ trace out the curvilinear unstable manifold in direction
${\bf e}_i$. Repeat for $-\epsilon\, {\bf e}_i (\pSpace)$.


{\bf Contracting real, positive eigendirection}. Reverse the
action of the map backwards in time. This turns a contracting
direction into an expanding one, tracing out the curvilinear
stable manifold in continuation of $\epsilon\, {\bf e}_j$.

{\bf Expanding/contracting  real negative eigendirection}. As above, but
every even iterate
$f^2(\pSpace_j)$,
$f^4(\pSpace_j)$,
$f^6(\pSpace_j)$ continues in the direction ${\bf e}_i$, every odd one
in the direction $-{\bf e}_i$.

{\bf Complex eigenvalue pair}. Construct an orthonormal
pair of eigenvectors spanning
the plane $\{ \epsilon\, {\bf e}_j,\epsilon\, {\bf e}_{j+1} \}$.
Iteration of the annulus
between an infinitesimal circle and its image by $\jMps$ spans
the spiralling/circle unstable manifold of the complex eigenvalue
pair $\{\ExpaEig_{i}, \ExpaEig_{i+1} = \ExpaEig_{i}^* \}$.

    \PublicPrivate{
    }{ % switch \PublicPrivate{
Alternatively, one can write
down a parametric equation for the stable (unstable) manifold and
generate the manifold by integrating the equation. %\rf{FR81}.
\PC{explain, draw figure}
    } % end \PublicPrivate{
