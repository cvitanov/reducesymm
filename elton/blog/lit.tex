%           %experimenting with svn-multi
\svnidlong {$HeadURL: svn://zero.physics.gatech.edu/elton/blog/lit.tex $}
{$LastChangedDate: 2018-12-07 13:04:16 -0500 (Fri, 07 Dec 2018) $}
{$LastChangedRevision: 547 $} {$LastChangedBy: predrag $} \svnid{$Id:
lit.tex 176 2013-12-29 19:03:13Z predrag $} \svnkwsave{$RepoFile:
elton/blog/lit.tex $}


\chapter{Literature}
\label{chap:lit}
$\footnotemark\footnotetext{{\tt \svnkw{RepoFile}}, rev. \svnfilerev:
 last edit by \svnFullAuthor{\svnfileauthor},
 \svnfilemonth/\svnfileday/\svnfileyear}$

Throughout:  {\textdollar} on the margin
{\steady}
indicates that the text has
already been incorporated into the
Elton \etal\ article gibson/mixing, or another article, or
the ChaosBook.org.


\noindent
{\color{red} The latest entry at the bottom for this part of the blog}
\bigskip\bigskip


\section{Reading assignments}
\label{sect:Reading}


\subsection{Keywords: Lagrangian mixing in turbulence}

Do literature review for Lagrangian mixing: possible keywords
to google:
\begin{itemize}
\item
    tracer particles in turbulent ...
\item
    Lagrangian dynamics in turbulence
\item
    inertial particles
\item
    Lyapunov exponents of heavy particles in turbulent flows
\end{itemize}

Possible authors (still to check)
\begin{itemize}
\item
Krzysztof Gawedzky (Lyon)
\item
    B. Eckhardt (Marburg):
Geometry of particle paths in chaotic and turbulent flows
\item
    Jean-Francois Pinton (Lyon): Lagrangian experiments
\end{itemize}


\subsection{Articles and books of potential interest}

\medskip\noindent {\bf  PC 2015-08-09}: This indeed is very interesting,
thanks!

\medskip\noindent {\bf Mohammad 2015-08-04}: Of possible interest:
\arXiv{1508.00062} {\em Quantitative Quasiperiodicity}, by
S. Das et al. (2015).

\medskip\noindent {\bf Mohammad 2015-06-30}: Of potential interest to Predrag:
\textit{A dynamical Zeta function for group actions}, Richard Miles.
\arXiv{1506.08555}

\medskip\noindent {\bf  Jean-Luc Thiffeault 2009-3-2}:
Article with Emmanuelle Gouillart, Olivier Dauchot
 and St\'ephane Roux\rf{GDTR08},
``Open-flow mixing: Experimental evidence for strange eigenmodes''
is perhaps worth a read.

\medskip\noindent {\bf  PC 2009-1-20}:
This Master's thesis announcement,
\HREF{http://www.nbi.ku.dk/english/Calendar/Activities_09/msc_abh_23.01.09/}
     {Semi- Lagrangian Methods in Air Pollutions Models},
might have some interesting pointers to
the Lagrangian tracing and mixing literature.

\medskip\noindent {\bf  PC 2008-8-22}:
check out the talk of (and search for articles by)
\HREF{http://www.newton.ac.uk/programmes/HRT/seminars/092215003.html}
     {B. Sawford}:
  	``A Lagrangian view of scalar transport and mixing.''
(Webcast on the Newton Institute homepage.)

\noindent
\HREF{http://www.newton.ac.uk/programmes/HRT/seminars/100210301.html}
    {GL. Eyink} (Johns Hopkins)
  	``Turbulent Lagrangian dynamics of vortex and magnetic-field line''
is a smart guy, this might be worth reading/listening to.
(Webcast on the Newton Institute homepage.)

\noindent
I have not had much luck understanding
\HREF{http://www.newton.ac.uk/programmes/HRT/seminars/100209001.html}
        {L. Biferale} (Tor Vergata)
  	``Lagrangian velocity statistics in turbulence: theory, experiments and numerics,''
in the past, but one could give it a try.
(Webcast on the Newton Institute homepage.)


Ottino\rf{Botti89}
% , J.M. \underline{The kinematics of mixing: stretching, chaos,
% and transport}. Cambridge University Press, 1989.
\\

\noindent
 Chat\'e, H.; Villermaux, E.; and Chomaz, J.M.
{\em Mixing: Chaos
and Turbulence}\rf{chat_mixing99}

\noindent
    Solomon,T.H. \etal "Lagrangian chaos and multiphase processes in
    vortex flows"\rf{solLagr03}.

\noindent
 Fogleman, M.A.; Fawcett, M.J.; and Solomon, "T.H.
Lagrangian chaos and correlated L�vy flights in a non-Beltrami flow:
Transient versus
long-term transport"\rf{fogleman01}.

\noindent
 Du, Yunson and Ott, Edward.
"Growth rates for fast kinematic dynamo instabilities of chaotic
fluid flows"\rf{du_growth06}.

\noindent
 Castelian, Cathy; Mokrani, Asen; Le Guer, Yves;
Peerhossaini, Hassan. "Experimental study of chaotic advection
regime in a twisted duct flow"\rf{castel01}.


d'Ovidio \etal\rf{dovidio08}
     ``Comparison between Eulerian diagnostics and finite-size
    Lyapunov exponents''
    might be worth a quick read.

\medskip\noindent {\bf  Jean-Luc Thiffeault 2001-8-29}:
Read T\'{e}l articles\rf{KarTel97,Tel2000},
they present the basic ideas: the first\rf{KarTel97} is
unphysical and the "openness" of the flow comes from having sinks and
unbounded trajectories.  The second\rf{Tel2000} scatters tracers off a Von Karman
vortex street, where they get stuck behind the cylinder for a while in
the mixing region.

\medskip\noindent {\bf  JRE 2008-12-8}:
G. Haller. "Distinguished material surfaces and coherent structures in three dimensional fluid flows". Physica D, 149. 2001. 248.


\medskip\noindent {\bf  PC 2008-7-12}:
read
\HREF{http://arxiv.org/abs/0807.0678}
     {Dullin and Meiss}\rf{dullin-2008} on ``Quadratic
Volume-Preserving Maps.'' They study the situation when two
\stagp s are saddle-foci with intersecting two-dimensional
stable and unstable manifolds that bound a spherical
``vortex-bubble.'' Probably not seen in your case, but they
write well, and they give you many pointers to $3D$ flows literature.
If you read some of that, please enter it into elton.bib, and
write notes in your blog about what's in these articles.

\medskip\noindent {\bf  PC 2008-7-11}:
Search for, read (and explain to Predrag)
literature on $3D$ flows. There are 2 aspects that surely have much
literature:
    \begin{enumerate}
      \item
$3D$ volume conservation makes these flows
almost symplectic - in particular, Poincar\'e sections of periodic
orbits are area-preserving
      \item
Reversibility. They are presumably the same forward and
backward in time. This discrete symmetry leads to infinite families
of ``symmetry lines'' on which important sets \eqva\ and
\po s lie, and ease the searches for them, as existence of
symmetry line reduces the dimensionality of the search,
They of might be what
leads to symmetry lines discussed by JRE in \refsect{sect:stagpairs}.
However, due to the convective term in \NSe\ I have not been able to
see how reversibility works, other than for
{\velgradmat} $ {\Mvar} $,

    \end{enumerate}

\begin{description}


    \PCpost{2008-5-15}{ Read (and explain to Predrag)
this frequently cited article on $3D$ flows:
Chong, {Perry} and {Cantwell}\rf{ChPeCa90}
    }

    \PCpost{2007-11-2}{ Read (and explain to Predrag)
Schneider recommendations - Homann, Dreher and Grauer\rf{homann-2007}
and  P.K. Yeung and Pope\rf{YeuPop06}
(the P.K. is just next 10 buildings away).
    }

    \PCpost{2007-12-22}{ Read (and explain to Predrag)
Mathur, Haller, Peacock, Ruppert-Felsot, and Swinney, ``Uncovering
the Lagrangian Skeleton of Turbulence''\rf{MHPRS07}. Predrag put a
copy into
\HREF{http://ChaosBook.org/library/index.html\#HallerPRL07}
     {ChaosBook.org/library}.
     }

    \PCpost{2007-12-22}{ Read (and explain to Predrag)
Arneodo \etal\rf{ABBBBB08}, ``Universal intermittent properties of
particle trajectories
              in highly turbulent flows.''
    }


    \PCpost{2006-06-18}{
Not directly relevant to us now, but \refref{OleMog07} has exact analytic
solutions of GOY model - perhaps of interest to make connections with the
Kolmogorov obsessed.
    }

    \MMFpost{2014-10-21}{
Added two references to \texttt{elton/public\_html/papers}
for mix-norms (measures of mixing efficiency):

\emph{A multiscale measure for mixing},
George Mathew, Igor Mezi\'c and Linda Petzold (2005)

\emph{Optimal stirring strategies for passive
scalar mixing},
Zhi Lin, Jean-Luc Thiffeault
and Charles R. Doering
    }

    \PCpost{2014-11-04}{
    Thanks, will study...
Repository gets very huge very quickly if we add *.pdf's of papers to it.
We enter them into \emph{elton/bibtex/elton.bib} and put the pdf's of
papers into \texttt{ChaosBook.org/library/}. Will show you how.
    }

    \PCpost{2014-11-04}{
Noticed two papers of possible interest to Mohammad:

Ajinkya Dhanagare, Stefano Musacchio and Dario Vincenzi,
{\em Weak-strong clustering transition in renewing compressible flows}
(\arXiv{1411.0110}; J. Fluid Mech., in press):
``
  We investigate the statistical properties of Lagrangian tracers transported
by a time-correlated compressible renewing flow. We show that the preferential
sampling of the phase space performed by tracers yields significant differences
between the Lagrangian statistics and its Eulerian counterpart. In particular,
the effective compressibility experienced by tracers has a non-trivial
dependence on the time correlation of the flow. We examine the consequence of
this phenomenon on the clustering of tracers, focusing on the transition from
the weak- to the strong-clustering regime. We find that the critical
compressibility at which the transition occurs is minimum when the time
correlation of the flow is of the order of the typical eddy turnover time.
Further, we demonstrate that the clustering properties in time-correlated
compressible flows are non-universal and are strongly influenced by the
spatio-temporal structure of the velocity field.''

\emph{On the use of the theory of dynamical systems for transient problems}
by Ugo Galvanetto and Luca Magri,
(\arXiv{1411.0111};
Nonlinear Dynamics 74,  pp 373-380,
DOI: 10.1007/s11071-013-0976-7)
\\
 ... address dynamical systems
with parameters varying in time. An idea to predict their behaviour is
proposed. These systems are called \emph{transient systems}, and are
distinguished from \emph{steady systems}, in which parameters are constant. In
particular, in steady systems the excitation is either constant (e.g. nought)
or periodic with amplitude, frequency and phase angle which do not vary in
time. We apply our method to systems which are subjected to a transient
excitation, which is neither constant nor periodic. The effect of switching-off
and full-transient forces is investigated. The former can be representative of
switching-off procedures in machines; the latter can represent earthquake
vibrations, wind gusts, etc. acting on a mechanical system. This class of
transient systems can be seen as the evolution of an ordinary steady system
into another ordinary steady system, for both of which the classical theory of
dynamical systems holds. The evolution from a steady system to the other is
driven by a transient force, which is regarded as a map between the two steady
systems.''

I have not studied either, so I do not know if they are any good.
   }

    \MMFpost{2015-02-12}{
 A cute piece of math: {\it Nonlinear Dynamics of
the Rock-Paper-Scissors Game with Mutations}
Toupo and Strogatz.
Read the paper \arXiv{1502.03370}.
    }

    \PCpost{2015-02-16}{
Very cute. But that's the thing about Strogatz - writes beautifully,
but he'll never leave 2 dimensions :)
    }

    \PCpost{2015-07-20}{
Abud and Caldas\rf{AbuCal15} write in {\em On {Slater's} criterion for
the breakup of invariant curves}: ``Slater's theorem states that an
irrational translation over a circle returns to an arbitrary interval in
at most three different recurrence times expressible by the continued
fraction expansion of the related irrational number. The hypothesis
 is that Slater's theorem can be also verified in
the dynamics of invariant curves. Hence, we use Slater's theorem to
develop a qualitative and quantitative numerical approach to determine
the breakup of invariant curves in the phase space of area-preserving
maps.''
    }

    \PCpost{2015-12-04}{                                        \toCB
Weiss and Knobloch\rf{WeiKno89} write in
{\em Mass transport and mixing by modulated traveling waves}:
``
Particle transport and mixing in modulated traveling waves in a
binary-fluid mixture heated from below is studied numerically. The fluid
divides into three regions separated by Kolmogorov-Arnol'd-Moser curves:
a core region where particles are carried along with the wave (trapped),
an outer region where particles are left behind by the wave (untrapped),
and a separatrix layer between the two where particles chaotically
alternate between being trapped and untrapped. The probability
distributions for the lengths of individual trapped and untrapped events
are sharply peaked at small times, have a power-law decay, and exhibit
similar complex structure. The core and outer regions are responsible for
long-range transport with no diffusion. The chaotic separatrix layer
gives rise to long-range transport with enhanced mixing and anomalous
diffusion.
''

Gottwald and Melbourne\rf{GottMel13} make a distinction
between `strong' and `weak chaos'. They write:
``
we adopt the standard perspective of decomposing the dynamics
into the dynamics on the symmetry group and the dynamics
orthogonal to it. Systems with symmetry, or ``equivariant dynamical
systems,'' thus are cast into a so-called skew product
\[
\ssp=f(\ssp)
    \,,\qquad
\dot{\LieEl} = \LieEl \Lg(\ssp)
\]

Given a certain shape dynamics
(regular, strongly chaotic or weakly chaotic), can we say anything
about the growth rate of solutions on the group? More specifically,
what is the expected diffusive behavior?

For anisotropic systems, strong chaos leads to diffusive behavior
(Brownian motion with drift) and weak chaos leads to superdiffusive
behavior (L�vy processes with drift). For isotropic systems, the drift
term vanishes and strong chaos again leads to Brownian motion. We
establish the existence of a nonlinear Huygens principle for weakly
chaotic systems in isotropic media whereby the dynamics behaves
diffusively in even space dimension and exhibits superdiffusive behavior
in odd space dimensions.
''
    }

\end{description}
