\svnkwsave{$RepoFile: elton/blog/ChaosBook.tex $}
\svnidlong{$HeadURL: svn://zero.physics.gatech.edu/elton/blog/ChaosBook.tex $}
{$LastChangedDate: 2014-11-17 19:43:32 -0500 (Mon, 17 Nov 2014) $}
{$LastChangedRevision: 212 $} {$LastChangedBy: mfarazmand3 $}
\svnid{$Id: ChaosBook.tex 212 2014-11-18 00:43:32Z mfarazmand3 $}

\chapter{Snippets for/from ChaosBook}
\label{chap:ChaosBook}

\begin{description}

\item[2014-11-16 Predrag] I have created this chapter to collect
    all your ChaosBook.org notes in one place.

Get the dasbuch/book/chapter/*.tex source
code, you can just clip and paste formulas to here.

I never refer to a chapter by it's current number,
as chapter numbers change from edition to edition - latter on (years
hence) trying to figure out what ``Chapter 17'' is can be quite
confusing. Internally, each chapter is kept track off by its file name,
for example, in this blog ``stability'' refers to  \refchap{c-stability} {\em
Local stability}.

\end{description}

%%-------------------------------------------------------
%%-----   Cycle stability
\section{Chapter: Cycle stability}
\label{c-invariants}\noindent dasbuch/book/chapter/invariants.tex
\begin{description}

\item[2014-11-16 Predrag]
Always plot ...

\end{description}

%%%%%%%%%%%%%%%%%% EXERCISES %%%%%%%%%%%%%%%%%%%%%%%%%%%%%%%%%%%%%%%%%%
\exercise{A limit cycle with analytic Floquet exponent.}{
	  \label{exer:ExactLyap}
\index{stability!exact}
There are only two examples of nonlinear flows for which the
Floquet multipliers can be evaluated analytically.
Both are cheats.
One example is the $2$\dmn\ flow
\bea
        \dot{q}  &=&  ~p + q(1-q^2-p^2)
		\continue
        \dot{p}  &=&  -q + p(1-q^2-p^2)
        \nnu
\,.
\eea %{eqBE4} %{eqCV4}
Determine all periodic solutions of this flow, and determine
analytically their Floquet exponents.
Hint: go to polar coordinates
$(q,p) = (r \cos \theta,r \sin \theta)$.
%    \label{eqBE5} %{eqCV5}
%
\hfill 				G. Bard Ermentrout % Aug 3 2005
	} %end \exercise{A limit cycle with analytic stability

\solution{exer:ExactLyap}
         {A limit cycle with analytic Floquet exponent.}{
\index{coordinate!change}
\index{stability!exact}
Transforming to the polar coordinates, we have
\begin{equation}
\dot r = r(1-r^2), \quad
\dot \theta = -1.
\label{eq:polar}
\end{equation}
This system has a fixed point at $r=0$. Moreover, for $r_c=1$, $\dot r$ vanishes.
Therefore, $r_c=1$ is an invariant circle in the phase space. Since $\theta$ is constant and independent of $r$, the circle $r_c=1$ is a periodic orbit.

For any initial condition $r_0<1$, the solutions $r(t;r_0)$ increase monotonically since
$r(1-r^2)>0$. Similarly, for initial conditions with $r_0>1$, the solutions $r(t;r_0)$
decrease monotonically. Therefore, $r_c=1$ is the only periodic orbit; it is in fact a
stable limit cycle.

Now let $f^t(r_0,\theta_0)=(r(t;r_0),\theta(t,\theta_0))$ be the solution map of
\refeq{eq:polar}. Then, 
\begin{equation*}
Df^t=
\begin{pmatrix}
\frac{\partial r}{\partial r_0} & 0\\
0 & \frac{\partial \theta}{\partial \theta_0}
\end{pmatrix}.
\end{equation*}
On the other hand, $\theta(t;\theta_0)=\theta_0-t$; therefore $\partial\theta/\partial\theta_0=1$.
It remains to compute $\partial r/\partial r_0$. Taking the derivative with respect to
$r_0$ in \refeq{eq:polar}, we get
$$ \frac{d}{dt}\frac{\partial r}{\partial r_0}=(1-r^2)\frac{\partial r}{\partial r_0}-2r^2\frac{\partial r}{\partial r_0}.$$
Therefore, on the periodic orbit $r_c=1$, we have
$$ \frac{d}{dt}\frac{\partial r}{\partial r_0}=-2\frac{\partial r}{\partial r_0},$$
which implies $\partial r/\partial r_0(t;r_c)=\exp(-2t)$. 

Therefore, for one period $T=2\pi$ of the orbit $r_c=1$,
\begin{equation*}
Df^T(r_0=1,\theta_0)=
\begin{pmatrix}
e^{-2T} & 0\\
0 & 1
\end{pmatrix},
\end{equation*}
i.e., the Floquet exponents are $\mu_1=-2$ and $\mu_2=0$.
\authorMMF{12 Nov. 2014}
    } %end \solution{exer:ExactLyap}{A limit cycle







%%-----   PDEs
\section{Chapter: Turbulence?}
\label{c-PDEs}\noindent dasbuch/book/chapter/PDEs.tex
\begin{description}

\item[2014-11-16 Predrag]
Always plot ...

\end{description}

%%%%%%%%%%%%%%%%%% EXERCISES %%%%%%%%%%%%%%%%%%%%%%%%%%%%%%%%%%%%%%%%%%

\exercise{In high dimensions any two vectors are (nearly) orthogonal.}{
\label{exer:2vecOrthog}
Among humble
plumbers laboring with extremely high\dmn\ ODE discretizations of fluid
and other PDEs, there is an inclination to visualize the $\infty$\dmn\
\statesp\ flow by projecting it onto a basis constructed from a few
random coordinates, let's say the 2nd Fourier mode along the spatial $x$
direction against the 4th Chebyshev mode along the $y$ direction. It's
easy, as these are typically the computational degrees of freedom. As we
will now show, it's easy but not smart, with vectors representing the
dynamical states of interest being almost orthogonal to any such random
basis.

Suppose your \statesp\ $\pS$ is a real 10\,247\dmn\ vector space, and you
pick from it two vectors $\ssp_1, \ssp_2\in \pS$ at random. What is the
angle between them likely to be?

By asking for `angle between two vectors' we have implicitly assumed that
there exist is a dot product
\[
\transp{\ssp_1} \cdot \ssp_2
=
\norm{\ssp_1} \norm{\ssp_2} \cos(\theta_{12})
\,,
\]
so let's make these vectors unit vectors,
\(
\norm{\ssp_j} = 1
\,.
\)
When you think about it, you would be hard put to say what 'uniform
probability' would mean for a vector $\ssp \in \pS= \reals^{10\,247}$,
but for a unit vector it is obvious: probability that $\ssp$ direction
lies within a solid angle $d\Omega$ is $d\Omega/($unit~hyper-sphere~surface).

So what is the surface of the unit sphere (or, the total solid angle)
in $d$~dimensions?
One way to compute it is to evaluate the Gaussian integral
\beq
I_d =
\int_{-\infty}^\infty\!\!\! d x_1 \cdots d x_d \,
e^{-\frac{1}{2}\left(x_1^2+\cdots +x_d^2\right)}
\ee{2vecOrth1}
in cartesian and polar coordinates. Show that
\begin{itemize}
  \item [(a)]
In cartesian coordinates
\(
I_d = (2\pi)^{d/2}
\,.
\)
  \item [(b)]
Recast the integrals in polar coordinate form. You know how to compute
this integral in 2 and 3 dimensions. Show by induction that the surface
$S_{d-1}$ of unit $d$-ball, or the total solid angle in even and odd
dimensions is given by
\beq
S_{2k} = \frac{2 (2\pi)^k}{(2 k-1)!!}
\,,\qquad
S_{2k+1} = \frac{2\pi^{k+1}}{ k!}
% fix this:  = \frac{2 k! (4\pi)^k}{(2k+1)!}
\,.
\ee{2vecOrth3}
  \item [(c)]
Show, by examining the form of the integrand in the polar coordinates,
 that for arbitrary, perhaps even complex dimension $d\in\complex$
\[
S_{d-1} = 2 \pi^{d/2}/\Gamma(d/2)
\,.
\]
In Quantum Field Theory integrals over 4-momenta are brought to polar
  form and evaluated as functions of a complex dimension parameter $d$.
  This procedure is called the `dimensional regularization'.
  \item [(d)]
Check your formula for $d=2$ (1-sphere, or the circle) and $d=3$
(2-sphere, or the sphere).
  \item [(e)]
  %(bonus)
What limit does $S_{d}$ does tend to for large $d$? (Hint: it's not what
you think. Try Sterling's formula).
\end{itemize}
So now that we know the volume of a sphere, what is a the most likely
angle between two vectors $\ssp_1, \ssp_2$ picked at random? We
can rotate coordinates so that $\ssp_1$ is aligned with the `$z$-axis' of
the hypersphere. An angle $\theta$ then defines a meridian around the
`$z$-axis'.
\begin{itemize}
  \item [(f)]
Show that probability $P(\theta)d\theta$ of finding two vectors at angle
$\theta$ is given by the area of the meridional strip of width $d\theta$,
and derive the formula for it:
\[
P(\theta) = \frac{1}{\sqrt{\pi}} \,\frac{\Gamma(d/2)} {\Gamma((d-1)/2)}
\,.
\]
(One can write analytic expression for this in terms of beta functions,
but it is unnecessary for the problem at hand).
  \item [(g)]
Show that for large $d$
the probability $P(\theta)$ tends to a normal distribution
with mean $\theta = \pi / 2$ and variance $1/d$.
\end{itemize}
So, in $d$\dmn\ vector space the two random vectors are nearly
orthogonal, within accuracy of $\theta = \pi/2 \pm 1/d$.

If you are a humble plumber, and the notion of a vector space is some
abstract hocus-pocus to you, try thinking this way. Your 2nd Fourier mode
basis vector is something that wiggles twice along your computation
domain. Your turbulent state is very wiggly. The product of the two
functions integrated over the computational domain will average to zero,
with a small leftover. We have just estimated that with dumb choices of
coordinate bases this leftover will be of order of $1/10\,247$, which is
embarrassingly small for displaying a phenomenon of order $\approx 1$.

Several intelligent choices of coordinates for \statesp\ projections are
described in Gibson \etal\rf{GHCW07} and the web tutorial
\wwwcb{/tutorials}.

\hfill                  Sara A. Solla
                        and P. Cvitanovi\'c
} %end \exercise{Trace-log of a matrix}{


\solution{exer:2vecOrthog}
{In high dimensions any two vectors are (nearly) orthogonal.}{
\begin{itemize}
  \item [(a)] The $d$ Gaussian integrals ...

  \item [(b)] You can Google ...

  \item [(c)] There are two ways of ...

  \item [(d)] ...

  \item [(e)] (bonus) ...

\end{itemize}
\authorMMF{16nov2014}
    } %end \solution{e-Tr-logM}{Trace-log of a matrix.}{





%%-----   Finding cycles variationally
%\section{Chapter: Relaxation for cyclists}
%\label{c-relax}\noindent dasbuch/book/chapter/relax.tex
%\begin{description}\item[2012-01-?? JMH]
%
%\end{description}


%%%-----   Appendices
%%\appendix
%
%
%%%-----   A brief history of chaos
%\section{Chapter: }\label{c-flows}\noindent dasbuch/book/chapter/appendHist.tex
%
%
%%%-----   Maps and billiards
%\section{Chapter: }\label{c-flows}\noindent dasbuch/book/chapter/appendB.tex
%
%
%%%-----   Linear algebra, Hamiltonian Jacobians
%\section{Chapter: }\label{c-flows}\noindent dasbuch/book/chapter/appendStability.tex
%
%
%%%-----   Cycles
%\section{Chapter: }\label{c-flows}\noindent dasbuch/book/chapter/appendCycle.tex
%
%%%-----   Symbolic dynamics techniques
%\section{Chapter: }\label{c-flows}\noindent dasbuch/book/chapter/appendSymb.tex
%
%
%%%-----   Counting
%\section{Chapter: }\label{c-flows}\noindent dasbuch/book/chapter/appendCount.tex
%
%
%%%-----   Implementing evolution
%\section{Chapter: }\label{c-flows}\noindent dasbuch/book/chapter/appendMeasure.tex
%
%%%-----   Applications
%\section{Chapter: }\label{c-flows}\noindent dasbuch/book/chapter/appendApplic.tex
%
%
%%%-----   Discrete symmetries
%\section{Chapter: }\label{c-flows}\noindent dasbuch/book/chapter/appendSymm.tex
%
%
%%%-----   Coveregence of spectral determinants
%\section{Chapter: }\label{c-flows}\noindent dasbuch/book/chapter/appendConverg.tex
%
%%%-----   Stat mech
%\section{Chapter: }\label{c-flows}\noindent dasbuch/book/chapter/appendStatM.tex
%
%
%%%-----   Infinite dimensional operators
%\section{Chapter: }\label{c-flows}\noindent dasbuch/book/chapter/appendWirzba.tex
%
%
%%%-----   Statistical Mechanics
%\section{Chapter: }\label{c-flows}\noindent dasbuch/book/chapter/statmech.tex
