%           %experimenting with svn-multi
\svnidlong
{$HeadURL: svn://zero.physics.gatech.edu/elton/blog/strategy.tex $}
{$LastChangedDate: 2014-11-07 16:41:13 -0500 (Fri, 07 Nov 2014) $}
{$LastChangedRevision: 197 $}
{$LastChangedBy: predrag $}
\svnid{$Id: strategy.tex 197 2014-11-07 21:41:13Z predrag $}
\svnkwsave{$RepoFile: elton/blog/strategy.tex $}


\chapter{Bluesky research}
\label{bluesky}
$\footnotemark\footnotetext{{\tt \svnkw{RepoFile}}, rev. \svnfilerev:
 last edit by \svnFullAuthor{\svnfileauthor},
 \svnfilemonth/\svnfileday/\svnfileyear}$

Throughout:  {\textdollar} on the margin
{\steady}
indicates that the text has
already been incorporated into the
Elton \etal\ article gibson/mixing, or another article, or
the Elton project.

\section{Lagrangian mixing, and what to do about it}
\label{sect:whattodo}

\subsection{Symmetries}

\medskip\noindent{\bf PC July 18 2008}:   JohnE: this is your order 8 group
rationalized (no
       need to write explicit multiplication tables),
       you might want to: (1) implement the binary group labels
      (2) generalize Halcrow argument for \stagp s to any \pCf\ isotropy subgroup
       (3) rewrite your '\stagp s on symmetry lines' argument as a fixed-point subspace of  an isotropy subgroup.


 \medskip\noindent{\bf PC July 21 2008}:  Guys, before I continue the rewrite of symmetries do you have an elegant way
 of generating all order-8 and order-4 isotropy subgroups?
    I think the subset that has the equilibria are the ones which contain both
     $\sigma_1$ and  $\sigma_2$, to fix span-/streamwise translations. If correct, that leads to
     only a few isotropy subroups. Agreed?

 \medskip\noindent{\bf JFG July 21 2008}: Just $s_3$ symmetry by itself fixes both x and z phases, since it
reverses signs in both x and z. We have a couple eqbs that are
symmetric in just $s_3$ (eqbs 9-11). We should be able to generate
all subgroups by permutation, shouldn't we? Each of the sixteen
elements of the group generated by $\sigma_1$, $\sigma_2$, $\tau_x$,
$\tau_z$ is its own inverse. So we have sixteen $D_2$ subgroups
$\{1, g\}$. Then we should have six 4th order groups $\{1, g1, g2,
g1 g2\}$ (six = four chose two among the four generators), two 8th
order groups $\{1, g1, g2, g3, g1 g2, g1 g3, g2 g3, g1 g2 g3\}$
(four choose three), and one 16th order group. Right?


\subsection{Mixing}
\noindent{\bf JRE July 4 2008}:
I was thinking possibly of taking a vertical cross-section of
the flow and making a density plot of local Lyapunov exponents,
in order to indicate regions with the most stretching.
It should be relatively simple to compute with the current setup.

\medskip
\noindent{\bf PC July 4 2008}:
That might be worth a try: Pick 3 planes, a streamwise wall-normal,
spanwise wall-normal, and the midplane, as in Gibson $3D$ plots.
At each point, evolve trajectory $\ssp(t)$
forward and backward for some sensible time, of order of
$t=\pm 200$?, then compute eigenvectors
$\jEigvec[j]$ and eigenvalues $\ExpaEig^{(j)}$ of the {\jacobianM}
$\jMps^{t}(\ssp(-t))$ and $\jMps^{-t}(\ssp(t))$. One of the
eigenvectors should align itself close to $\bu(\ssp(0))$,
with $\ExpaEig^{(3)} \approx 1$; for hyperbolic regions, the other two should
be real, similar for $\jMps^{t}(\ssp(-t))$ and $\jMps^{-t}(\ssp(t))$,
area conserving,
$\ExpaEig^{(1)}\ExpaEig^{(2)}= 1/\ExpaEig^{(3)} \approx 1$,
and align themselves along stable/unstable manifolds with exponential
accuracy as $t \to \pm \infty$.
You can mark the directions $j=1,2$ by a pair of little sticks
through $\ssp(0)$, colored coded by the values of
finite time Lyapunovs
$\eigExp[j](\ssp(0)) = \frac{1}{2t} \ln |\ExpaEig^{(j)}|$, where
$\ExpaEig^{(j)}$ are the hyperbolic eigenvalues of
$\jMps^{2t}(\ssp(-t))$.
The real  eigenvalue eigenvectors should give you a good idea of
the manifolds, the complex ones of the elliptic, non-mixing regions.
Not sure what one learns from finite time Lyapunovs - as you say,
it is interesting to isolate regions of maximal local stretching.

Perhaps before you get started on this, plot the sections of stable/unstable
manifolds of your \stagp s first, by starting along linear eigenvactors
in small neighborhoods of \stagp s.
Once you have some intuition about local stable/unstable manifold structure
it might be much faster to evolve local tangent vectors backward/forward
than to evolve $\jMps^{2t}(\ssp(-t))$. Also, once you have
stable/unstable manifold intersections, getting \po s should be a breeze -
but postpone this, best to write up what we already have first.

The problem is that different wall-normal sections give different
info. Once you have a sense of maximal Lyapunov regions, you might
do the same thing as for \stagp s - obtain $3D$ plots by keeping only
$\eigExp[j](\ssp(0)) > $ cut-off. Except I expect these regions to
be fractal.

Do not spend too much time on this, as anyway it will be the
\po s, not
\stagp s - my hunch - that actually carry the mixing. But it is interesting
to know what regions are quiescent and what strongly stretching
for \stagp s as well.

The whole thing should probably run on the cluster, once you are
set-up.


\subsection{JRE May 27 2008:}

 Updated to-do list: \\

 \textbf{Have done so far:}
\begin{enumerate}
\item Wrote a script for computing exact velocity field at any point.
Uploads Upper Branch $Re$ 400 spectral coefficients and evaluates
the sum in equation \refeq{eqn:spectralsum}.
\item Storing a fine 143 $\times$ 104 $\times$ 143 grid containing
velocity field values.
\item Interpolation routine using this grid gives $\mathbf{u}$ at
any point much more quickly than by method 1.
\item Using these techniques to integrate and plot any number of
tracer trajectories.
\item Using these techniques to make movies showing the evolution of
trajectories of tracer particles.
\item Found all \stagp s in a unit cell.
\item Have a symmetry argument explaining the existence and values
for 4 of the 6 \stagp s. Other two were found by plotting velocity
squared over the fine grid, picking out regions with almost no
motion, and using a Newton iteration which converged.
\item By taking derivatives of \refeq{eqn:spectralsum} can compute
$A_{ij}$ \velgradmat\ at any point.
\item By taking more derivatives of \refeq{eqn:spectralsum} can
compute a matrix of Laplacians.
\item With $A_{ij}$, have found eigenvalues and eigenvectors for all
\stagp s and thus characterized their local stability.
\item Using the eigenvectors, have made plots of stable and unstable
manifolds for all of the \stagp s, some of which show evidence for
likely heteroclinic connections.
\end{enumerate}

\textbf{Need to do, small picture:}
\begin{enumerate}
\item Improve interpolation speed and accuracy. For an ink blob with
a huge number of points it will be necessary to interpolate. Speed
can be improved by using a C compiler. Accuracy can be improved by
either storing an even finer grid or by switching from bilinear to
bicubic interpolation.
\item Improve plots of stable and unstable manifolds. In order to be
convinced that heteroclinic connections exist, we need the
computations of these manifolds to remain accurate all the way from
one \stagp\ to the other. I am computing the exact velocity fields
for this, so I
believe the errors are coming from the integration routine. Matlab's
RK4 routines choose a variable step size based on how smooth they
think the velocity field is. I will probably need to write my own
RK4 so I can force it to take very tiny steps.
\item Convince self numerically that there are no more \stagp s.
\item Plot and label all \stagp s in a clear way.
\item Get movies working, converted to a format than can be viewed
by anyone.
\item Look into local viscosity or local \Reynolds idea.
\item Get set up to run my programs on cluster computers.
\end{enumerate}

\textbf{Need to do, big picture:}
\begin{enumerate}
\item Come up with a theoretical argument for the existence and
values of the \stagp s \xSP{5} and \xSP{6}. Could be a symmetry
argument or possibly something else taken directly from the
Navier-Stokes equations, maybe a topological argument for existence.
Who knows. It is true that the coordinates of these points don't
look to be in any kind of rational relation to $Lx$ and $Lz$, but
there could still be a functional relation, e.g. maybe
$\xSP{5}^{(1)} = e^{\frac{1}{\nu}L_{x}}$.
\item Use Lan-Cvitanovi\'c \descent\ to find periodic orbits. I
don't know what this is yet.
\item Quotient out discrete and continuous symmetries.
\item Applications:
 \begin{enumerate}
 \item Show that in certain regions Lagrangian streamlines are in
 fact chaotic. Lyapunov exponent, etc...
 \item Use cyclist approach to compute deterministic diffusion
 constant.
 \item Mixing. Qualitative and quantitative descriptions. Need a good idea for {\em what is mixing?} in an infinite
    system such as \pCf. Physical
    idea: put a layer of carbon
    across whole ocean, see how quickly \pCf\ mixes it into
    the ocean.
 \end{enumerate}
\end{enumerate}
\textbf{Questions:}
\begin{enumerate}
\item What do we currently see as the hierarchy of importance in the
above "Need to do, big picture" list? Is mixing still the overall
goal? Right now we have a pretty good picture and some novel
behaviors for an infinite, physical 3D flow. Should the next step be
periodic orbits, quotienting, etc... or on to applications?
\item In order to run on the cluster computers I assume someone will
have to set me up to do this. How is this done? Does it mean I will
be able to to control and run their machines through my laptop?
\item Specifics of \NS\ equations? See \refsect{sect:NavierStokes}
\end{enumerate}

 \subsection{PC May 12 2008:}
  Long term things
to check - could be a whole PhD thesis, way beyond a summer project:

\begin{enumerate}
  \item compute ${\Mvar}_{ij}(\ssp)$ \velgradmat\
    from an \uEQ\ dataset
  \item compute stable/unstable manifolds
        for a given {\stagp}
  \item prove that the symmetry between {\stagp}
        pairs amounts to ``time reversal'' (?)
  \item prove that such {\stagp}
        pairs are heteroclinically connected
  \item store a fine $3D$ velocity grid, integrates by
    interpolation (cubic? higher? see \refsect{ssect:IntVel})
  \item map $u^2$ over the whole cell, plotting
    $u^2(x,y,z) < \epsilon$ to identify regions with \stagp s.
  \item within each such region iterate to the \stagp\ by linear
    interpolation between $u$'s bracketing $u=0$, recursively
  \item might be able to prove that we have ALL stagnation points
    for this problem?
  \item a fine 3D velocity grid is also the idea behind
    the Lan-Cvitanovi\'c \descent\rf{CvitLanCrete02,lanVar1},
    so we will be able to find periodic orbits, when
    needed
  \item Eventually, should quotient out discrete symmetries,
        as in Lorenz example in \wwwcb{}, chapter ``World in a mirror.''
  \item Eventually, should quotient out the two continuous symmetries,
        as in the secret work by E.~Siminos.
  \item \wwwcb{}, chapter ``Deterministic diffusion''
    already has the formula for computing diffusion. The physical
    problem:
    put a narrow Gaussian Chernobyl cloud somewhere in the box -
    theory predicts the diffusion constant $D$,
    where the width of the cloud at time $t$ grows as
    $D\sqrt(t)$. The cyclist approach to turbulent diffusion of
    passive tracers:
       \begin{enumerate}
         \item Compute the $3D$ diffusion tensor $(D_\tEQ){}_{ij}$
         for each \tEQ{}\ of \pCf\ at a fixed
         small but turbulent \Reynolds.
         \item Compute $(D_\tTW){}_{ij}$ for \reqva\ \tTW. Expected
         to be strongly anisotropic for all \eqva\ and \reqva\ of
         \pCf, with small diffusion time
         streamwise, and very long one spanwise.
         \item Compute $(D_P){}_{ij}$ for \po s and \rpo s\ $P$.
         \item Derive a formula for Navier-Stokes turbulent diffusion
                $(D_P){}_{ij}$ as average over the above.
       \end{enumerate}
  \item Need a good idea for {\em what is mixing?} in an infinite
    system such as \pCf. Physical
    idea: put a layer of carbon
    across whole ocean, see how quickly \pCf\ mixes it into
    the ocean.
\end{enumerate}

\section{Reading assignments}

\subsection{Articles and books of potential interest}


\subsection{Keywords: Lagrangian mixing in turbulence}

Do literature review for Lagrangian mixing: possible keywords
to google:
\begin{itemize}
\item
    tracer particles in turbulent ...
\item
    Lagrangian dynamics in turbulence
\item
    inertial particles
\item
    Lyapunov exponents of heavy particles in turbulent flows
\end{itemize}

Possible authors (still to check)
\begin{itemize}
\item
Krzysztof Gawedzky (Lyon)
\item
    B. Eckhardt (Marburg):
Geometry of particle paths in chaotic and turbulent flows
\item
    Jean-Francois Pinton (Lyon): Lagrangian experiments
\end{itemize}


\subsection{Diverse literature}


\noindent {\bf  PC 2006-06-18}:
Not directly relevant to us now,
but \refref{OleMog07} has exact analytic
solutions of GOY model - perhaps of interest to make connections
with the Kolmogorov obsessed.


\section{Analyze this: John Elton land}

Check out its symmetry, translate if that brings it back into the
\uUB, \uNB\ fold.


\section{Local Reynolds number \Reynolds(\bx)?}
\label{sec:Rey}

\noindent {\bf  JRE May 16, 2008}: As you say, an \eqv\ of
incompressible Navier Stokes satisfies (neglecting pressure) \beq
\rho(\vec{u} \cdot\nabla)\vec{u} = \mu\nabla^{2} \vec{u} \eeq So it
looks like the position-dependent quantity you have defined is \beq
Q(\bx)_{i} = \frac{(\vec{u} \cdot \nabla) \, \vec{u} \cdot
\jEigvec[i]}
     {\nabla^{2} \vec{u} \cdot \jEigvec[i]} = \frac{\mu}{\rho}
     \equiv \nu, \eeq where $\nu$ is the kinematic viscosity.
  Each
     component of $\mathbf{Q\mathbf(x)}$ defined this way is the same, $\nu$, so
     I'm not sure a vector field plot would be of interest. However
     a local measure of viscosity would be a very interesting and
     useful quantity to know, so I propose defining \beq
     \tilde{\nu}(\bx) = \frac{(\vec{u} \cdot \nabla) \, \vec{u} \cdot
\jEigvec[i]}
     {\nabla^{2} \vec{u} \cdot \jEigvec[i]} \eeq as the local
     viscosity. We should have that $\tilde{\nu}(\bx)$ is the same for each
     $i$. This local viscosity could be computed over a grid of
     points with essentially the same method I am using to compute
     the \velgradmat, just need one more derivative.


\noindent {\bf  PC May 15, 2008}:
We use $3D$ plots of velocity fields and vorticities to visualize
Navier-Stokes flow. Here is another wild proposal that might
give a different insight into boundary layers, by emphasizing effective
local viscosity of the flow.

An \eqv\ solution of the incompressible  \NS\ satisfies (setting $\nu=1$,
$\rho=1$, neglecting pressure)
 \beq
(\vec{u} \cdot \nabla)\vec{u} = \nabla^{2} \vec{u}
    \,,\qquad
\nabla \cdot \vec{u}  = 0
\,,
\ee{NSe}
so one can define position dependent $3D$ vector field
of Reynolds numbers
 \beq
 \Reynolds(x)_i =
     \frac{(\vec{u} \cdot \nabla) \, \vec{u} \cdot \jEigvec[i]}
     {\nabla^{2} \vec{u} \cdot \jEigvec[i]}
     \,,\qquad
     i = 1,2,3
 \,.
 \label{ReField}
 \eeq
For locally large $|\Reynolds(x)|$
fluid flows freely, while for
small $|\Reynolds(x)|$ fluid is  viscous. Particularly interesting
would be ``Eulerian fluid'' points $\bx$ for which
$\nabla^{2} \vec{u} =0$, $|\Reynolds(x)| \to \infty$,
if they exist. For \pCf\ one would
normalize this definition so that average Reynolds number
is
\beq
 \Reynolds ={U d}/{\nu}
 \,,
\eeq
and plot $3D$ sections of this quantity just as we now
plot $3D$ sections of $\bu(\bx)$. Could give us another
glimpse at the boundary layer.

Just a thought.


\section{Find nontrivial stable coherent state}

\noindent {\bf JFG 2008-04-29}: We have one already -
the upper branch right after bifurcation.

\subsection{Divakar's suggestions}

Visualize Lagrangian transport in Divakar's and Jonathan's \rpo\
co-moving frames for \pCf.



\chapter{IAQs}      \label{IAQs}

Infrequently Asked Questions [IAQs].

\section{Papers}

\subsection{JFM mixing paper }

 \noindent {\bf  JRE August 12, 2008}:
 I think that since the first mention of wrap up and start writing a
 paper, a pretty good amount has been done to add to this. So: What
 steps at this point should I be taking towards this? Obviously
 picking out what information to emphasize and starting a draft is one
 thing. Also I feel sure that a lot of these figs will need
 improvement, although I'm not exactly sure in what way. What length
 paper are we shooting for? Etc... \\

 \noindent {\bf  JRE July 4, 2008}:
Will start on the suggestions. Before repeating for other equilibria
though, there was still the remaining question of what quantities to
calculate to characterize the mixing?\\

 \noindent {\bf  PC July 4, 2008}:
I think that's too ambitious at this point - just explaining in the
introduction that our eventual goal is putting this information
together (it will be periodic solutions, not equilibria, is my
hunch, that actually carry the mixing) should cover that. Giving
Lagrangian transport interpretation to equilibria solutions is
already new enough (remember, Nagata found a pair of equilibria in
1990 and we still credit him for that), and various symmetry
arguments, heteroclinic connections etc are enough material for a
nice self-contained paper.\\

 \noindent {\bf  PC July 3, 2008}:
 I think we have enough to wrap up and write a short paper.
    \begin{enumerate}
\item
    ask Jonathan to get you running on the PACE cluster
\item
    repeat the analysis for \tLB; expect less mixing, possibly fewer
    \stagp s.
\item
    repeat the analysis for \tEQeight; more ``turbulent,''
    in the $S_3$-invariant subspace \refeq{symmUthree}.
\item
PC or JFG throw together the first draft, in JFM format.
\item
 Submit JFM paper
\end{enumerate}

\section{There are many paths to wisdom}

\noindent {\bf  JRE July 24, 2008}:
In Physical Space I am in Germany from 7/21 to 8/23 writing up the work with Steve Tomsovic that we started last summer.

You are probably wondering why I am not packing my bags and getting ready to move off for grad school. Long story short, I decided to take a year off, finish up and publish the projects with you and Steve, and re-apply this December. Good news is that means I will have time to continue working towards mixing.

\medskip\noindent {\bf  PC July 25, 2008}:
On very general principles I support taking time off at this juncture - would probably
take off to jungles of Borneo (my pseudo godchile just emailed me from there) or
help Peruvian village school teachers learn how to use Uno Laptop per Ninio.
But doing some physics just for pleasure of it might be just as good - I recommend
\HREF{http://www.flickr.com/photos/birdtracks/sets/72157604913211036/}
     {going into seclusion} for a while and just thinking, with no internet noise.

As to what happens later, I recommend going to graduate school somewhere else, just
because it's healthy and American thing to do. But if you want to stay in Atlanta,
and continue working on chaos and shmurbulence, JohnG and I would love to have you work
with us, so getting into GT physics is kind of guaranteed (once you explain to me why
you vanished for almost a year). In the meantime, you have a desk on the 5th floor,
and please skype Jonathan and learn how to use the PACE cluster well before end of
August, because after that Jonathan will be hard to get.

The track record of senior scientist advising junior ones on life transforming decision
is pitiful; see for example Feynmans letter to Wolfram,
in the recently published collection. Feynman is wrong on every count.
So take the above with a large grain of salt.

\medskip\noindent {\bf  JRE July 26, 2008}:
When it comes to using computers I often feel like a Peruvian
village school teacher.

Taking time off isn't really because I want to take time off. I'm
very excited to start graduate work and would rather be doing that
than traveling - I've gotten the travel bug out of my system. So,
needless to say, it was a complicated decision. In any case I do
want to use the time to my advantage and to spend it thinking about
physics.

As far as graduate school, I have contemplated continuing here at
Georgia Tech. I have enjoyed thinking about chaos/turbulence
particularly in the recent months when this projected started taking
off and I started feeling like I actually kind of understand what's
going on. At the same time I agree with you that it seems healthier
to go off and be immersed in something new, particularly in my case
since I have lived less than 15 minutes from GT for my entire life.
So I guess for now I will just keep my options open and worry about
it in the Fall.

\medskip\noindent {\bf  PC July 25, 2008}: This bit of schmoozing might
be the ultimate turn-off, but one of the deeply secret and almost
never uttered ambitions of the ``Hopf'' program is data compression,
in the following sense.

Spinning around the globe is a zillion satellites, sending back continuously
terabytes of cloud, wind, temperature data. It's stored in the raw format
and very hard to manipulate. If we computed from \NSe\ a finite
vocabulary of patterns explored by this date, we could recode it in a very
compact form. Remember, we have not just images, but videos, so this
encoding would be not only efficient, but also predictive. Lorenz
deconstructed.



\section{Spruce up personal websites}

\begin{enumerate}
\item
\HREF{http://www.cns.gatech.edu/CNS-only/~elton}
     {JRE homepage} with description of the research, publication list,  movies.
Predrag installed a template homepage
from
\\
\centerline{
\HREF{http://www.cns.gatech.edu/CNS-only/WWW_editing.html}
     {www.cns.gatech.edu/CNS-only/WWW\_editing.html}
           }\\
This project is already added to
\HREF{http://ChaosBook.org/projects/index.shtml\#Elton}
     {ChaosBook.org/projects}, I just need to link the
paper once we draft it.

\end{enumerate}


\section{Conferences}

\section{Outreach}

\begin{enumerate}
\item
Create a Lagrangian mixing tutorial for
\HREF{http://ChaosBook.org/tutorials}
     {ChaosBook.org/tutorials}
with the aim of eventual inclusion into
\HREF{http://www.dynamicalsystems.org/tu/tu/}
     {www.dynamicalsystems.org/tu}.

A Lagrangian mixing tutorial example worth reading,
but also an example of how \underline{not}
to write a tutorial (too much fine print, too few illustrations) is
\HREF{http://www.cds.caltech.edu/~shawn/LCS-tutorial/}
     {Shawn C. Shadden} tutorial.

\item
Whistle or other Georgia Tech news item

\item
get announcement approved for GaTech SoP research page

\end{enumerate}

\subsection{Advertise arXiv Lagrangian mixing paper}

\begin{enumerate}
\item
crosslink the paper with nonlin dynamics

\item
email individually arXiv paper link to colleagues who might comment
    on the paper

\item
PC and JFG contact ?? and makes them aware of our work.

\item
snailmail the paper to ??, with a handwritten personal letter.

\end{enumerate}
