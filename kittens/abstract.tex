% siminos/kittens/abstract.tex      pdflatex CL18; bibtex CL18
% $Author: predrag $ $Date: 2020-09-20 15:02:24 -0500 (Sun, 20 Sep 2020) $

% Predrag   CL18 version 1.0            2020-09-19
% Predrag   CL18 version 0.1            2019-08-15
% Predrag   GHJSC16 draft               2017-09-26

While the global dynamics of an extended, spatiotemporally
turbulent system can be extraordinarily complex, the local dynamics,
observed through small spatiotemporal windows, can be thought of
as a visitation sequence through a finite repertoire of finite patterns.
To compute spatiotemporal expectation values of observables from the
defining equations of such systems,  one needs to know how often a given
pattern occurs.
Here we address this fundamental question by constructing a
spatiotemporal cat, a classical $d$-dimensional chaotic lattice field
theory.
Treating the temporal and spatial directions on equal footing, we abandon
initial state evolution, local in time, and enumerate instead global
solutions compatible with system's defining equations.
In such field theory any
spatiotemporal state is labeled by a unique $d$-dimensional lattice block
of symbols from a finite alphabet, a state of the system over a finite
spatiotemporal region is specified uniquely and with exponential
precision by a finite blocks of such symbols, and the likelihood of such
state occurring is given by the Hill determinant of its spatiotemporal
orbit Jacobian matrix.
