% siminos/kittens/abstract.tex      pdflatex CL18; bibtex CL18
% $Author: predrag $ $Date: 2020-01-31 15:09:09 -0600 (Fri, 31 Jan 2020) $

% Predrag   CL18 version            2019-08-15
% Predrag   GHJSC16 draft           2017-09-26

While the global dynamics of an extended, {\spt}ly chaotic (or
`turbulent') system can be extraordinarily complex, the local dynamics,
observed through small {\spt} finite windows, can be thought of as a
visitation sequence through finite repertoires of finite patterns.
To compute \spt\ expectation values of observables from the defining
equations of such systems,  one needs to know how often a given pattern
occurs.
Here we address this fundamental question by constructing a
`\catlatt', a classical $d$\dmn\ {\spt} chaotic lattice field theory.
In this field theory any {\spt} state is labeled by a unique $d$\dmn\
lattice \brick\ of symbols from a finite alphabet, with the lattice
state and its symbolic encoding related linearly.
We show that the state of the system over a finite {\spt} region can be
described uniquely and with exponentially increasing precision by finite
\brick s of such symbols, and that the likelihood of such state
occurring is given by the {\HillDet} of its \spt\ {\jacobianOrb}.
