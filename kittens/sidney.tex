% GitHub cvitanov/reducesymm/kittens/sidney.tex
% Predrag                                       2020-08-18
	
\section{Sidney blog on matters spatiotemporal}
\label{sect:sidney}

\begin{description}

\item[2020-05-20 Predrag]
You can write up your narrative in this file.
Can clip \& paste anything from above sections you
want to discuss, that saves you LaTeXing time.


\item[2020-08-22 Predrag]
First task:

Start reading \refsect{s:Bernoulli}~{\em Bernoulli map}.
Everything up to \refsect{s:1D1dLatt}~{\em Temporal Bernuolli}
you know from the ChaosBook course.

New stuff starts here. See how much you understand. Write your
study notes up here, ask questions - this is your personal blog.

You refer to an equation like this: \refeq{tempBern};

to figure like
this: \reffig{fig:BernCyc2Jacob};

to table like this:
\reftab{tab:LxTs=5/2};

to a reference like this: Gutkin and Osipov\rf{GutOsi15} (\emph{GutOsi15}
refers to an article listed in \emph{../bibtex/siminos.bib}).

and to external link like this:
``For great wallpapers, see overheads in
\HREF{http://www-personal.umich.edu/~engelmm/lectures/ShortCourseSymmetry.html}
{Engel's} course.''

\item[2020-08-22 Sidney]
Why do you write \emph{\jacobianOrb}
\refeq{jacobianOrb} as a partial derivative, when you already
know $\jMorb$, see \refeq{tempFixPoint}?



\end{description}
