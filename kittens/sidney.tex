% GitHub cvitanov/reducesymm/kittens/sidney.tex
% Predrag                                       2020-08-18
	
\section{Sidney blog on matters spatiotemporal}
\label{sect:sidney}

\begin{description}

\item[2020-05-20 Predrag]
You can write up your narrative in this file.
Can clip \& paste anything from above sections you
want to discuss, that saves you LaTeXing time.


\item[2020-08-22 Predrag]
First task:

Start reading \refsect{s:Bernoulli}~{\em Bernoulli map}.
Everything up to \refsect{s:1D1dLatt}~{\em Temporal Bernuolli}
you know from the ChaosBook course.

New stuff starts here. See how much you understand. Write your
study notes up here, ask questions - this is your personal blog.

You refer to an equation like this: \refeq{tempBern};

to figure like
this: \reffig{fig:BernCyc2Jacob};

to table like this:
\reftab{tab:LxTs=5/2};

to a reference like this: Gutkin and Osipov\rf{GutOsi15} (\emph{GutOsi15}
refers to an article listed in \emph{../bibtex/siminos.bib}).

and to external link like this:
``For great wallpapers, see overheads in
\HREF{http://www-personal.umich.edu/~engelmm/lectures/ShortCourseSymmetry.html}
{Engel's} course.''

\item[2020-08-22 Predrag] An example of referring to the main text:
Why do you write \emph{\jacobianOrb}
\refeq{jacobianOrb} as a partial derivative, when you already
know $\jMorb$, see \refeq{tempFixPoint}?

\item[2020-08-24 Sidney]
Started reading from the beginning as that only adds an additional 4 pages, and it would be beneficial to review.

\vspace{3mm}

General Notes:
Showing what modern chaos calculations look like.
The \catlatt\ is the arbitrary dimension generalization of the 1-D Bernoulli map.

(mod 1) subtracts the integer part of $s\phi_t$, this keeps $\phi_{t+1}$
within the unit interval (group theoretic analogue?). Also partitions
the state-space into $s$ sub-intervals.

\item[2020-08-24 Predrag]
The group theory here compatifying translations on (infinite) line
$\phi\in(-\infty,\infty)$ to translations on (compact) circle
$\phi\in[0,2\pi)$.

\item[2020-08-24 Sidney]
Reminder to self: review the symbolic dynamics, and binary operations from chapter 14 of Chaosbook
The unit interval is partitioned into $s^n$ subintervals, each with one unstable period-n point, except the rightmost fixed point is the same as the fixed point at the origin. So there are $s^n-1$ total period-n periodic points.
$\sigma$ in \refeq{tempBern} is a cyclic permutation that translates forward in time the lattice state by one site. Inverse $\sigma$ because the second term is always one step behind the first term and an inverse $\sigma$ moves the state back one.
\vspace{3mm}

Questions
1. I've pretty much never done modular arithmetic before, I understand \refeq{BerStretch} in the idea that the circle map wraps in on itself and contributes the value of its slope after one go around, but I am unsure on how to use the modular arithmetic to do that, should I look into that?

\item[2020-08-24 Predrag]
As I do not know what ``modular arithmetic'' is, don't worry about :)

\item[2020-08-25 Sidney]
General Notes

\refeq{pathBern} appears to be a vector of a periodic (or relative periodic orbit) through the Bernoulli map. Review Multishooting. Total number of of periodic points of period n is $N_n=s^n-1$ but it also equals the magnitude of the determinant of the orbit Jacobian matrix. (got to page 7)

\vspace{3mm}

\begin{itemize}
  \item[Q1]
Is \refeq{tempBernFix} the evolution function $f^t(y)$ that was
referenced throughout ChaosBook?
  \item[Q2]
What exactly is meant by a ``lattice"?
\end{itemize}

\item[2020-08-24 Predrag].

\begin{itemize}
  \item[A1]
The whole point of the paper us that ChaosBook is obsolete - in the new
formulation, there is no `time' evolution, no time trajectory $f^t(y)$,
there are only sets of fields the live on lattice points that satisfy
recurrence relations. Eq.~\refeq{tempBernFix} is \emph{\jacobianOrb}, the
stability of a lattice state, to be related to stability forward in time
in \refsect{s:Hill}. This is a revolution: there is no more time, there
is only spacetime.
  \item[A2]
Temporal lattice $\integers$ is defined in \refeq{pathBern}. Spacetime
integer lattice $\integers^2$, (or more generally $\integers^d$) in
\refeq{KanekoCML}, \refeq{CatMap2d}.
When you get to it, a 2\dmn\ \emph{Bravais lattice} $\Lambda$ is defined
in \refeq{2DBravaisLattice}.

If this is unclear, read up on integer lattices, give your own precise definition.
\end{itemize}

\item[2020-08-26 Sidney]
Point Lattice (integer lattice seems to be a special case of point lattice) notes from \HREF{https://mathworld.wolfram.com/PointLattice.html}{Wolfram}: "A point lattice is a regularly spaced array of points." The integer lattice is where all of these points are integers. I will look at the Barvinok lecture tomorrow, I have to finish moving to a different house today. (Stayed on page 7)

\begin{itemize}
	\item[Q3]
Please correct me if I am wrong, but a lattice seems to be a collection of points where are all regularly spaced, so does "regularly" mean that it is controlled by a deterministic law? If this is the case, the $\phi_n$ states in a periodic orbit can be grouped as a lattice and ordered by location along the periodic orbit, then the associated "winding" number $m_t$ can be grouped in its own lattice, which in this case is an integer lattice. What is the "regular" spacing for the winding numbers? Have missed the point?
\end{itemize}

\begin{itemize}
  \item[A3]
Wolfram is right. When you have a discrete time map, time takes integer
values $\zeit=\cdots,-1,0,1,2,\cdots$. That is called 1\dmn\ integer lattice
$\integers$. Once you are in $d=2$ or higher, the name makes sense, as
you can visualize $\integers^2$ as a `lattice'. It is regular, because all spacings
between neighboring points are 1. There is nothing `deterministic' about
this, it just says that time takes its values on integers, rather than on a continuum.

There is only one lattice, but on each lattice site there is a
real-valued field $\ssp_\zeit$ and the integer valued `source' \Ssym{\zeit}.
\end{itemize}

\item[2020-08-27 Sidney]
Thank you for A1, that makes complete sense now. Calculated the orbit Jacobian using equation \refeq{tempBernFix}, matched with the paper, yay. Orbit Jacobian maps the basis vectors of the unit hyper-cube into a fundamental parallelepiped basis vectors, each of which is given by a column in the orbit Jacobian. $|Det(s/\sigma|=s^n$ because $\sigma$ and its inverse are both unitary matrices, and if you multiply every row of an $[n\times{n}]$ matrix, the determinant is multiplied by the constant raised to the power n. Periodicity $\sigma^n=1$ accounts for $\overline{0}$ and $\overline{s-1}$ fixed points being a single periodic point. (got to page 9)

\begin{itemize}
	\item[Q4]
I was trying to calculate the orbit Jacobian using the $\sigma$ matrix, but the delta function equation \refeq{hopMatrix} for $\sigma$ doesn't seem to work for the Bernoulli map, I know that $\sigma_{2,1}=1$ and $\sigma_{1,2}=1$ which works with the delta function definition. However, $\sigma_{2,1}=\delta_{3,1}$ from \refeq{hopMatrix}, which should equal zero. Other than just the idea of being cyclic, I don't know why it yields one instead of zero, what am I missing?
  \item[A4]
Work it out $\hopMat$ matrices for $n=1,2,3,\cdots$. It will start making sense.
	\item[Q5]
So, does ``lattice state" mean the set of all points (field of all points?) which running through the Bernoulli map requires the specific winding number at that lattice site?
  \item[A5]
Interesting, grad students too seem to confuse coordinates (for example,
$(x,\zeit)=(3.74,-0.02)$ in continuum, $(n,\zeit)=(7,-6)$ on a discrtized space)
and the
fields $\ssp(n,\zeit)$. Physical ``state'' refers to value of field $\ssp$
over every $(n,\zeit)$ - is the grass high or low? rather than the coordinates
of spacetime.

How would you state this precisely if you were trying to explain this
paper to another student?
\end{itemize}

\item[2020-08-30 Sidney].
\begin{itemize}
  \item[A5.1]
Sidney: ``
$\Xx_\Mm$ is the set of all values the field $\ssp_z$ takes over
the set of coordinates $\Mm$.
% Please let me know if I have missed something here.
''
  \item[A5.2]
Predrag: Please reread 2nd paragraph of \refsect{s:1D1dLatt} and explain what
is wrong with your answer A5.1
\end{itemize}

\vspace{3mm}

Notes: For an n-periodic lattice state $\Xx_\Mm$ the Jacobian
matrix is now a function of a [d x d] matrix J, so the formula for the
number of periodic points of period n (number of lattice states of period
n) is now $|\det(1-J_M)|$ where $J_M=\prod^n_{t=1}J_t$ where $J_t$ is the
one-step Jacobian matrix which is assumed to vary in time.

\begin{itemize}
	\item[Note to self:]
look back over the topological zeta function, specifically try to
understand derivation of:
\[
\frac{1}{\zeta_{top}(z)}=\exp\left(-\sum^{\infty}_{n=1}\frac{z^n}{n}N_n\right)
\]
(got to \refpage{s:bernODE})
	\item[Predrag:]
\HREF{http://chaosbook.org/chapters/ChaosBook.pdf\#section.18.4} {(click here)}
	\item[Q6]
Is ``there are ${s}$ fundamental lattice states, and every other lattice state
is built from their concatenations and repeats"
is simply a restatement of the fact that the Bernoulli map is a full shift?
	\item[A6]
For Bernoulli, yes. But search for word `fundamental' in ChaosBook
chapter \HREF{http://chaosbook.org/chapters/ChaosBook.pdf\#chapter.18}
{{\em Counting}}. For example, `We refer to the set of all
non-self-intersecting loops $\{ t_{p_1}, t_{p_2}, \cdots t_{p_f} \}$ as
the {\em fundamental cycles}'. Write up here a more nuanced statement of
`fundamental' cycles might be (I do not have firm grip on this either...).
	\item[Q7]
Is \refeq{bernN_n-s=2} a result of expanding in a Taylor the result of the derivative (and product of $1/\zeta_{top}$ and $z$)? Because the topological zeta function of the Bernoulli map is a closed form function, not an infinite sum.
\end{itemize}

\item[2020-08-31 Sidney]
Via a finite difference method, \refeq{1stepDiffEq} can be viewed as a
first order ODE dynamical system. Back-substituted with \refeq{tempBern}
to show that with $\Delta t=1$ the velocity field does satisfy the diffeq
\refeq{1stepVecEq}.
The Bernoulli system can be recast into a discretized ODE whose global
linear stability is described by the {\jacobianOrb}.
(Stayed on \refpage{s:bernODE}))

\item[2020-09-01 Sidney]
Started reading \refsect{s:kickRot}~{\em A kicked rotor}.

\refeq{PerViv2.1b}
and \refeq{PerViv2.1a} describe the motion of a rotor being subjected to
periodic momentum pulses. The mod is present for the q equation to make
sure that the angle varies from $0$ to $2\pi$. As in the Bernoulli map case,
here
mod is also added to the momentum equation to keep it bounded to a unit
square. Cat maps with the stretching parameter ${s}$ are the same up to a
similarity transformation. An automorphism is an isomorphism of a system
of objects onto itself. An isomorphism is a map that preserves sets and
relations among elements.

\begin{itemize}
	\item[Q8]
Do the kicked rotor equations with Hooke's law force, and bounded momentum (mod 1 added to \refeq{PerViv2.1a}) only take the form of \refeq{catMap} if K is an integer?
	\item[A8]
The text states: ``The
$(\mbox{mod}\;1)$ added to \refeq{PerViv2.1a} makes the map a
discontinuous `sawtooth,' unless $K$ is an integer.'' How would you make that clearer?
	\item[Q9]
How does \refeq{catMap} have a state space which is a 2-torus? I am having a hard time visualizing how this came about.
	\item[A9]
Do you understand how  $(\mbox{mod}\;1)$ operation turns unbounded stretch
\refeq{BerStretch} into a circle map \refeq{n-tuplingMap}? Circle map is
1-torus. If both
\(
(\coord_{\zeit},p_{\zeit}) \in  (0,1]\times(0,1]
\)
are wrapped into unit circles, the phase space $(\coord_{\zeit},p_{\zeit})$
is not an infinite 2\dmn\ plane, but a compact, doubly periodic unit square with
opposite edges glued together, \ie, 2-torus.
\end{itemize}

%	\item[2020-09-02 Predrag]
%Your every commit's description is ``Update sidney.tex'',
%while my commits describe the edits. ``Update sidney.tex'' is not
%informative, especially later, if you try to find a particular commit...

\item[2020-09-03 Sidney]
I was typing my description into "summary" textbox above the commit to
master button. Obviously I was incorrect, I'll try to type in the
"description" for this commit.

\item[2020-09-02 Predrag]
``Tripping Through Fields'' showed up :)

\item[2020-09-03 Sidney].
\begin{itemize}
	\item[A5.3] Sidney:
I'm not actually quite sure what's wrong with my given definition. From
your answer A5 it seems that \Mm\ is a set of
coordinates (the location of the blade of grass) and $\Xx_\Mm$ is
the value at that coordinate (the height of the grass at that point).
Perhaps I forgot that these lattice states are for periodic orbits, so I
forgot the second coordinate (period of length n).
	\item[A5.4] Predrag: %2020-09-03
The textbook inhomogeneous \emph{Helmoltz equation} is an elliptical
equation of form
\beq
   (\Box+k^2)\,\ssp(z)= -\m(z)\,,\qquad z\in \reals^d
\,,
\label{CatMapContinuesPC}
\eeq
where the \emph{field} $\ssp(z)$ is a $C^2$ functions of
\emph{coordinates} $z$, and $\m(z)$ are \emph{sources}. For example,
charge density is a \emph{source} of electrostatic \emph{field}.

Suppose you are so poor, your computer lacks infinite memory, you only
have miserly only 10\,Tb, so you cannot store the infinitely many values
that \emph{coordinates} $z\in \reals^d$ take. So what do you do?

Perhaps a peak at ChaosBook
\HREF{http://chaosbook.org/chapters/ChaosBook.pdf\#section.X.1} {A24.1
Lattice derivatives} can serve as an inspiration. And once you have done
what a person must do, your Helmoltz equation (hopefully) has the form of
\refeq{OneCat}.
What is a \emph{field}, a \emph{source}, a \emph{coordinate} then?
\end{itemize}


\item[2020-09-03 Sidney].
\begin{itemize}
	\item[A8.1]
The sawtooth statement made sense, what made it unclear for me was the second sentence which started with "in this case" it was (again for me, I might not have been paying enough attention) ambiguous, I didn't know if it was talking about the integer case or the sawtooth case.  	
	\item[A8.2] Predrag: thanks, I rephrased that sentence.
	\item[A9.1]
I understand, your explanation makes sense, thank you :).
\end{itemize}

\vspace{3mm}

Notes: The discrete time Hamiltonian system induces forward in time evolution on the 2-torus phase space. The orbit Jacobian can take many different forms depending on the map. Despite this the Hill determinant can still count the number of periodic lattice states.
(got to page \refpage{s:tempCatCountTEMP})

\item[2020-09-05 Sidney].
\begin{itemize}
	\item[A5.5]
If I was so unlucky to only have 10Tb of memory, I would take a finite
interval of points z that I was interested in, and discretize them
(evenly, or unevenly) and then evaluate the field (that was probably the
wrong wording) at a finite set of points, either of particular interest
within the interval, or closely spaced enough so that the values were
representative of the values the field took over a continuum. I think
that a coordinate is a point in state space specified by specific values
of state variables (position, time, momentum etc.). To try to answer
source, and field, I'll be thinking of an electric charge, a source is
what generates the medium by which other sources are effected, and the
field is the medium which acts upon other sources.

	\item[A5.6]
I did look at
Chaosbook \HREF{http://chaosbook.org/chapters/ChaosBook.pdf\#section.X.1}
{A24.1 Lattice derivatives}, but it didn't seem to address quite the
fundamental confusion I seem to be facing. I'm relatively confident in my
coordinate definition, but not at all in my source, and field definition.
\end{itemize}

\item[2020-09-05 Predrag].
\begin{itemize}
	\item[A5.7]
Expression ` state space' $\pS$ refers to `states': cats, dancers,
\ie, `fields' \Xx\ and their
names \Mm. `Coordinates' refer to markings on the floor that they stand on.

Does reading \refsect{s:lattState}~{\em Lattice states} now helps in
distinguish a skater from the skating ring's ice? I rewrote it for your
pleasure :)
	\item[A5.7]
Re. the fundamental confusion of reading
\HREF{http://chaosbook.org/chapters/ChaosBook.pdf\#section.X.1} {A24.1
Lattice derivatives}: If you mark every inch on the floor, this is
`discretization'. But the floor is still a floor, no?
\end{itemize}

%\item[2020-09-05 Predrag] It looks like you do not use `diff' to find out
%what has been edited. Also, GitHub Desktop shows the edited text by coloring
%it. I keep edition your text to help you learn the macros used in writing
%the article.
%
%\item[2020-09-05 Sidney]
%I see why you do the editing, I'll try to incorporate the fancier
%\LaTeX{} footwork into my blog.

\item[2020-09-05 Sidney].
\begin{itemize}
	\item[A5.8]
I read the pink bits of \refsect{s:lattState}~{\em Lattice states} (as I
assume that was the parts that you rewrote specially). From it I (think)
I understand. We're looking at two coordinates for most of the Bernoulli
and cat map stuff: a spatial one, and a temporal one, the maps only
effect the temporal placement, but effect it differently depending on
where the point was in space when the map acted on it, because the field
takes a different value at every point in space (and time). So the
coordinates are the field point placement in time and space. The field is
the value that is assigned to every lattice point. \Mm\ keeps getting
referred to as an alphabet, so that makes me think that it is similar
(perhaps the multidimensional generalization) to the ``alphabet" which was
used to partition state space in the 1D maps of Chaosbook, such as 0 for
the left half of the interval and 1 for the right, and then further
partitioning the more the map is applied. Is that close at least?
\end{itemize}

\item[2020-09-05 Predrag].
\begin{itemize}
	\item[A5.9]
Getting hotter. Look at \refeq{circ-m} and \refeq{catMapNewt};
$\ssp_{\zeit}$ and $\Ssym{\zeit}$ are the same kind of a beast,
$\Ssym{\zeit}$ is just the integer part of the ``stretched'' field in
\refeq{BerStretch}. In this particular, linear map setting, this integer
does double duty, as a letter of an ``alphabet''. It cannot possibly be a
``coordinate'', it like saying that a dancer's head is ``floor.''
	\item[A5.10]
In temporal lattice formulation no ``map is applied.'' That is the
brilliance of the global \spt\ reformulation: there is no stepping
forward in time, so there is no map - the only thing that exists is the
global fixed point condition that has to be satisfied by field values
everywhere on the lattice, simultaneously.

Time is dead.
\end{itemize}

\item[2020-09-08 Sidney].
\begin{itemize}
	\item[Q11]
So the \templatt\ / \catlatt\ equations are moving around points in the lattice instead of
through time?
	\item[Q12]
Is something of the form of \refeq{tempFixPoint} an example
of the ``global fixed point condition"?
\end{itemize}

\item[2020-09-14 Predrag].
\begin{itemize}
	\item[A11]
An equation does not have to be ``moving around'' anything: think of a
quadratic equation $x^2+bx+c=0$. Does it ``move'' anything? No. It's a condition
that a single ``field'' $x$ has to satisfy, and the solution is a root of that
equation.
The \templatt\ / \catlatt\ equations are ``equations'' in the same sense,
[bunch of terms involving $\ssp_z$ ]=0.
	\item[A12]
Yes.
\end{itemize}

\item[2020-09-09 Sidney]
Notes: Equations such as \refeq{catMapNewt} can be solved using similar methods to linear odes: guessing a solution of the form $\Lambda^t$ and finding the characteristic equation. Then assuming all terms are site independent because the difference of any two solutions of \refeq{catMapNewt} solve its homogeneous counterpart \refeq{diffEqs:CatCharEq}. Got to \refpage{s:tempCatZeta}.

Notes: Topological zeta functions count prime orbits, i.e. time invariant sets of equivalent lattice states related by cyclic permutations. The "search for zeros" \refeq{tempCatFixPoint} is the "fixed point condition." Which is a global statement which enforces \refeq{catMapNewt} at every point in the lattice. Got to \refpage{s:catlatt}

\item[2020-09-13 Sidney]
The \templatt\ is a special case of the \catlatt, defined on a one\dmn\
lattice $\integers^1$. In this case the associated topological zeta
function is known in a closed, analytic form.

\emph{Coupled map lattices}: Starts with a review of finite difference
methods for PDEs. The d dimensions in the lattice are d-1 spatial lattice
points and 1 temporal one. The PDE is reduced to dynamics of a coupled
map lattice, with a set of continuous fields on each site.

\begin{itemize}
	\item[A5.11]
I have experience with finite difference methods for solving a
discretized form of a PDE, but I'm having a hard time visualizing the
idea of having a discrete coordinate system in d different directions,
but with a continuous field on each site. This  may be valuable as it is
a specific statement of where I'm getting stuck.
	\item[Q13]
My current understanding is
that at each point in the d-dimensional integer lattice ("point" as in a
lattice node with d specified coordinates), but at each point (site)
there is a continuous field. What is this field continuous over? It's at
one point in a discrete coordinate system. And why is there a continuum
at each point? And finally, I assume that these continuous fields are the
values of the function being solved for at that point, however, shouldn't
that just be a single value? Not a field? I'm sorry if this is a rather
silly question, but I'll keep thinking about it and I'll make a note if
my understanding (or lack thereof) changes.
	\item[A13]
Predrag: In \reffig{fig:BernPart} field $x_\zeit$ or $\ssp_\zeit$ and
$f(\ssp_\zeit)$  on the discrete site $\zeit$ run over continuous values.
For example, at temporal lattice site $\zeit=7$ the field value is
$\ssp_7=0.374569263952942\cdots$. OK now?
	\item[Q13.1]
Slight update, it seems that the field is the state of the system and at each discretized point there is a map acting on the state, although that conflicts with the notion that time is dead, so I'm probably still misunderstanding.
	\item[A13.1]
Predrag: Yes.
\end{itemize}
Thinking of this as a spring mattress. Often starts out with chaotic on-site dynamics weakly coupled to neighboring sites.
In this paper one sets the lattice spacing constant equal to one. Diffusive coupled map lattices introduced by Kaneko:
\[
\ssp_{n,t+1}=
g(\ssp_{n,t})+\epsilon\left[g(\ssp_{n-1,t})-2g(\ssp_{n,t})+g(\ssp_{n+1,t})\right]
\,,
\]
where each individual spatial site's dynamical system $g(x)$ is a 1D map, coupled to the nearest neighbors by the discretized second order \emph{spatial} derivative. The form of time-step map $g(\ssp_{n,t})$ is the same for all time i.e. invariant under the group of discrete time translations. Spatial stability analysis can be combined with temporal stability analysis, with orbit weights depending exponentially both on the space and the time variables: $t_p\propto e^{-L\period{}\lambda_p}$. $\sigma_i$ translates the field by one lattice spacing in the $i^{th}$ direction.
\begin{itemize}
	\item[Q14]
What is a lattice period?
	\item[A14]
Predrag: Does the paragraph above \refeq{catlattFix} answer you question?
I would like to refer to the \emph{set} of numbers
$\{\ell_1,\ell_2,\cdots,\ell_d\}$ as the \emph{period} of lattice
$\Lambda$. Would that be confusing?
	\item[Q15]
Is $z$ in the definition of a lattice state both a temporal and a spatial index? So equivalent to both n and t?
	\item[A15]
Predrag: after \refeq{CatMap2d} I write ``a 1\dmn\ spatial
lattice, with field $\ssp_{n\zeit}$ (the angle of a kicked rotor
\refeq{PerViv2.1b} at instant $\zeit$) at \spt\ site
$z=(n,\zeit)\in\integers^2$.'' Should this
``$z=(n,\zeit)\in\integers^2$'' be repeated elsewhere. If so, where?
	\item[Q16]
Often a member of the alphabet can be a negative number, which I assume
means that the state is taken out of unity in the negative direction.
	\item[A16]
Do you understand \reffig{fig:BernCyc2Jacob} and \reffig{fig:catCycJacob}?
\end{itemize}

The
\catlatt\ has the point-group symmetries of the square lattice.
A lattice state is a set of all field values $\Xx=\{\ssp_z\}$ over the
d-dimensional lattice that satisfies the \catlatt\ equation,
with all field values constrained between zero and one. A lattice state
$\Xx_{\Lambda}$ is a \emph{\twot} if it satisfies
$\Xx_{\Lambda}(z+R)=\Xx_{\Lambda}(z)$ for any
discrete translation $R=n_1\textbf{a}_1+n_2\textbf{a}_2 \in \Lambda$. Got
to \refpage{s:catLatt1x1}.

\item[2020-09-14 Sidney].
\begin{itemize}
	\item[A13.1]
I think I'm OK now. I think what I was trying to visualize was a stack of an infinite number of values at each lattice point, which was confusing, but this makes sense.
	\item[A14.1]
Unfortunately I don't think I quite understand. I understand the idea of the different directions, I understand treating
$\Xx_\Mm(\phi_z)$ as a singular fixed point, but I do not understand $\ell_i$.
	\item[A15.1]
I think that I lost that definition of $\textit{z}$ around \refeq{dDCatsT}, but I think that may have been a factor of how long it takes me personally to digest this material.
	\item[A16.1]
After reading the descriptions and staring at it for awhile, I think that I do.
	\item[Q17]
I tried a couple days back (Thursday or Friday I think, they all blend together) to log in to your bluejeans office. But it must have been one of the times that it had logged you off due to inactivity. There was also another person their I didn't recognize, and I didn't want to step on their toes if they were waiting for you to get back, so I logged off. So, when in general would good times to try hopping into your office?
\end{itemize}

\item[2020-09-16 Sidney]
A Bravais lattice can be denoted $\Lambda=\left[L\times\period{}\right]_S$ where $L$
is the spatial lattice period,  $\period{}$ is the temporal lattice period, $S$
imposes the tilt to the cell. Basis vectors for the Bravais cell can be
written as:
\[
\mathbf{a}_1=\left(\begin{array}{c}
  \speriod{}\\
  0{}
  \end{array}\right)
  \,,\qquad
\mathbf{a}_2=\left(\begin{array}{c}
  \tilt{}\\
  \period{}
  \end{array}\right)
  \,
\]

\begin{itemize}
	\item[Q18]
If something is written as $850[3\times2]_0$ what is the numerical value? More importantly, how is it found? I know it has to do with the cyclic permutations of the prime blocks, but I'm not sure how to get a numerical value.
\end{itemize}
Got to page \refpage{s:catLattCount}

\item[2020-09-17 Sidney]
For the Bernoulli map its stretching uniformity allows the use of combinatorial methods for lattice points. For temporal (not spatiotemporal) the number of periodic lattice states is the same as the volume of the fundamental parallelpiped, so the magnitude of the determinant of the orbit Jacobian. The block $\Mm$ can be used as a 2D symbolic representation of the lattice system state. For a given admissible source block $\Mm$, the periodic field can be computed by:
\[
\ssp_{i_1j_1}=\sum_{i_2=0}^2\sum_{j_2=0}^1\textbf{g}_{i_1j_1,i_2j_2}\Mm_{i_2j_2}
\]


\item[2020-09-19 Predrag]
Sorry, I've been a bit overwhelmed with lecture preparations, so
I will not answer any of the questions quite yet. But I have rewritten the
abstract, and the introduction to the paper, up to
the start of \refsect{s:Bernoulli}~{\em Bernoulli map}.
Can you have a critical look at the new text, report here if
something does not make sense to you?

\item[2020-09-19 Sidney].
%Sure, I'll add to this blog post if anything throws me. Don't worry about being busy, I understand.
\begin{itemize}
	\item[Update]
I read through, and aside from some very minor grammar issues (forgetting
a ``have" after ``we")
% and forgetting a closed parentheses etc. PC fixed that one
it all makes sense.
\end{itemize}

\item[2020-09-20 Predrag].
\begin{itemize}
	\item[A15.2]
I now added the $z$ definition to \refeq{dDCatsT},
is that clearer?
\end{itemize}

\item[2020-09-20 Sidney]
\begin{itemize}
	\item[A15.3]
Yes, that makes it more clear
\end{itemize}
$$-\sum^{\infty}_{r=1}\frac{1}{r}tr\hat{\textbf{J}}_p^r=tr\left(-\sum^{\infty}_{r=1}\frac{1}{r}\hat{\textbf{J}}_p^r\right)=tr\hspace{2mm}ln\left(\hat{1}_1-\hat{\textbf{J}}_p^r\right)=ln\hspace{2mm}det\left(\hat{1}_1-\hat{\textbf{J}}_p^r\right)$$
I liked the text cut from the introduction on page 44, it made the idea of time's death more easily digestible. Finished main paper, will look at the appendices for math. 



\end{description}
