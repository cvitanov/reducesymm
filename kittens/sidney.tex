% GitHub cvitanov/reducesymm/kittens/sidney.tex
% Predrag                                       2020-08-18
	
\section{Sidney blog on matters spatiotemporal}
\label{sect:sidney}

\begin{description}

\item[2020-05-20 Predrag]
You can write up your narrative in this file.
Can clip \& paste anything from above sections you
want to discuss, that saves you LaTeXing time.


\item[2020-08-22 Predrag]
First task:

Start reading \refsect{s:Bernoulli}~{\em Bernoulli map}.
Everything up to \refsect{s:1D1dLatt}~{\em Temporal Bernuolli}
you know from the ChaosBook course.

New stuff starts here. See how much you understand. Write your
study notes up here, ask questions - this is your personal blog.

You refer to an equation like this: \refeq{tempBern};

to figure like
this: \reffig{fig:BernCyc2Jacob};

to table like this:
\reftab{tab:LxTs=5/2};

to a reference like this: Gutkin and Osipov\rf{GutOsi15} (\emph{GutOsi15}
refers to an article listed in \emph{../bibtex/siminos.bib}).

and to external link like this:
``For great wallpapers, see overheads in
\HREF{http://www-personal.umich.edu/~engelmm/lectures/ShortCourseSymmetry.html}
{Engel's} course.''

\item[2020-08-22 Predrag] An example of referring to the main text:
Why do you write \emph{\jacobianOrb}
\refeq{jacobianOrb} as a partial derivative, when you already
know $\jMorb$, see \refeq{tempFixPoint}?

\item[2020-08-24 Sidney]
Started reading from the beginning as that only adds an additional 4 pages, and it would be beneficial to review.

\vspace{3mm}

General Notes:
Showing what modern chaos calculations look like.
The \catlatt\ is the arbitrary dimension generalization of the 1-D Bernoulli map.

(mod 1) subtracts the integer part of $s\phi_t$, this keeps $\phi_{t+1}$
within the unit interval (group theoretic analogue?). Also partitions
the state-space into $s$ sub-intervals.

\item[2020-08-24 Predrag]
The group theory here compatifying translations on (infinite) line
$\phi\in(-\infty,\infty)$ to translations on (compact) circle
$\phi\in[0,2\pi)$.

\item[2020-08-24 Sidney]
Reminder to self: review the symbolic dynamics, and binary operations from chapter 14 of Chaosbook
The unit interval is partitioned into $s^n$ subintervals, each with one unstable period-n point, except the rightmost fixed point is the same as the fixed point at the origin. So there are $s^n-1$ total period-n periodic points.
$\sigma$ in \refeq{tempBern} is a cyclic permutation that translates forward in time the lattice state by one site. Inverse $\sigma$ because the second term is always one step behind the first term and an inverse $\sigma$ moves the state back one.
\vspace{3mm}

Questions
1. I've pretty much never done modular arithmetic before, I understand \refeq{BerStretch} in the idea that the circle map wraps in on itself and contributes the value of its slope after one go around, but I am unsure on how to use the modular arithmetic to do that, should I look into that?

\item[2020-08-24 Predrag]
As I do not know what ``modular arithmetic'' is, don't worry about :)

\item[2020-08-25 Sidney]
General Notes

\refeq{pathBern} appears to be a vector of a periodic (or relative periodic orbit) through the Bernoulli map. Review Multishooting. Total number of of periodic points of period n is $N_n=s^n-1$ but it also equals the magnitude of the determinant of the orbit Jacobian matrix. (got to page 7) 

\vspace{3mm}

Questions

1. Is \refeq{tempBernFix} the evolution function $f^t(y)$ that was referenced throughout Chaosbook?

2. What exactly is meant by a "lattice"? 
 
\end{description}
