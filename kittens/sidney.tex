% GitHub cvitanov/reducesymm/kittens/sidney.tex
% Predrag                                       2020-08-18
	
\section{Sidney blog on matters spatiotemporal}
\label{sect:sidney}

\begin{description}

\item[2020-05-20 Predrag]
You can write up your narrative in this file.
Can clip \& paste anything from above sections you
want to discuss, that saves you LaTeXing time.


\item[2020-08-22 Predrag]
First task:

Start reading \refsect{s:Bernoulli}~{\em Bernoulli map}.
Everything up to \refsect{s:1D1dLatt}~{\em Temporal Bernuolli}
you know from the ChaosBook course.

New stuff starts here. See how much you understand. Write your
study notes up here, ask questions - this is your personal blog.

You refer to an equation like this: \refeq{tempBern};

to figure like
this: \reffig{fig:BernCyc2Jacob};

to table like this:
\reftab{tab:LxTs=5/2};

to a reference like this: Gutkin and Osipov\rf{GutOsi15} (\emph{GutOsi15}
refers to an article listed in \emph{../bibtex/siminos.bib}).

and to external link like this:
``For great wallpapers, see overheads in
\HREF{http://www-personal.umich.edu/~engelmm/lectures/ShortCourseSymmetry.html}
{Engel's} course.''

\item[2020-08-22 Predrag] An example of referring to the main text:
Why do you write \emph{\jacobianOrb}
\refeq{jacobianOrb} as a partial derivative, when you already
know $\jMorb$, see \refeq{tempFixPoint}?

\item[2020-08-24 Sidney]
Started reading from the beginning as that only adds an additional 4 pages, and it would be beneficial to review.

\vspace{3mm}

General Notes:
Showing what modern chaos calculations look like.
The \catlatt\ is the arbitrary dimension generalization of the 1-D Bernoulli map.

(mod 1) subtracts the integer part of $s\phi_t$, this keeps $\phi_{t+1}$
within the unit interval (group theoretic analogue?). Also partitions
the state-space into $s$ sub-intervals.

\item[2020-08-24 Predrag]
The group theory here compatifying translations on (infinite) line
$\phi\in(-\infty,\infty)$ to translations on (compact) circle
$\phi\in[0,2\pi)$.

\item[2020-08-24 Sidney]
Reminder to self: review the symbolic dynamics, and binary operations from chapter 14 of Chaosbook
The unit interval is partitioned into $s^n$ subintervals, each with one unstable period-n point, except the rightmost fixed point is the same as the fixed point at the origin. So there are $s^n-1$ total period-n periodic points.
$\sigma$ in \refeq{tempBern} is a cyclic permutation that translates forward in time the lattice state by one site. Inverse $\sigma$ because the second term is always one step behind the first term and an inverse $\sigma$ moves the state back one.
\vspace{3mm}

Questions
1. I've pretty much never done modular arithmetic before, I understand \refeq{BerStretch} in the idea that the circle map wraps in on itself and contributes the value of its slope after one go around, but I am unsure on how to use the modular arithmetic to do that, should I look into that?

\item[2020-08-24 Predrag]
As I do not know what ``modular arithmetic'' is, don't worry about :)

\item[2020-08-25 Sidney]
General Notes

\refeq{pathBern} appears to be a vector of a periodic (or relative periodic orbit) through the Bernoulli map. Review Multishooting. Total number of of periodic points of period n is $N_n=s^n-1$ but it also equals the magnitude of the determinant of the orbit Jacobian matrix. (got to page 7)

\vspace{3mm}

\begin{itemize}
  \item[Q1]
Is \refeq{tempBernFix} the evolution function $f^t(y)$ that was
referenced throughout ChaosBook?
  \item[Q2]
What exactly is meant by a ``lattice"?
\end{itemize}

\item[2020-08-24 Predrag].

\begin{itemize}
  \item[A1]
The whole point of the paper us that ChaosBook is obsolete - in the new
formulation, there is no `time' evolution, no time trajectory $f^t(y)$,
there are only sets of fields the live on lattice points that satisfy
recurrence relations. Eq.~\refeq{tempBernFix} is \emph{\jacobianOrb}, the
stability of a lattice state, to be related to stability forward in time
in \refsect{s:Hill}. This is a revolution: there is no more time, there
is only spacetime.
  \item[A2]
Temporal lattice $\integers$ is defined in \refeq{pathBern}. Spacetime
integer lattice $\integers^2$, (or more generally $\integers^d$) in
\refeq{KanekoCML}, \refeq{CatMap2d}.
When you get to it, a 2\dmn\ \emph{Bravais lattice} $\Lambda$ is defined
in \refeq{2DBravaisLattice}.

If this is unclear, read up on integer lattices, give your own precise definition.
\end{itemize}

\item[2020-08-26 Sidney]
Point Lattice (integer lattice seems to be a special case of point lattice) notes from \HREF{https://mathworld.wolfram.com/PointLattice.html}{Wolfram}: "A point lattice is a regularly spaced array of points." The integer lattice is where all of these points are integers. I will look at the Barvinok lecture tomorrow, I have to finish moving to a different house today. (Stayed on page 7)

\begin{itemize}
	\item[Q1]
Please correct me if I am wrong, but a lattice seems to be a collection of points where are all regularly spaced, so does "regularly" mean that it is controlled by a deterministic law? If this is the case, the $\phi_n$ states in a periodic orbit can be grouped as a lattice and ordered by location along the periodic orbit, then the associated "winding" number $m_t$ can be grouped in its own lattice, which in this case is an integer lattice. What is the "regular" spacing for the winding numbers? Have missed the point?
\end{itemize}

\begin{itemize}
  \item[A1]
Wolfram is right. When you have a discrete time map, time takes integer
values $\zeit=\cdots,-1,0,1,2,\cdots$. That is called 1\dmn\ integer lattice
$\integers$. Once you are in $d=2$ or higher, the name makes sense, as
you can visualize $\integers^2$ as a `lattice'. It is regular, because all spacings
between neighboring points are 1. There is nothing `deterministic' about
this, it just says that time takes its values on integers, rather than on a continuum.

There is only one lattice, but on each lattice site there is a
real-valued field $\ssp_\zeit$ and the integer valued `source' \Ssym{\zeit}.
\end{itemize}

\item[2020-08-27 Sidney]
Thank you for A1, that makes complete sense now. Calculated the orbit Jacobian using equation \refeq{tempBernFix}, matched with the paper, yay. Orbit Jacobian maps the basis vectors of the unit hyper-cube into a fundamental parallelepiped basis vectors, each of which is given by a column in the orbit Jacobian. $|Det(s/\sigma|=s^n$ because $\sigma$ and its inverse are both unitary matrices, and if you multiply every row of an nXn matrix, the determinant is multiplied by the constant raised to the power n. Periodicity $\sigma^n=1$ accounts for $\overline{0}$ and $\overline{s-1}$ fixed points being a single periodic point. (got to page 9)

\begin{itemize}
	\item[Q1]
I was trying to calculate the orbit Jacobian using the $\sigma$ matrix, but the delta function equation\refeq{hopMatrix} for $\sigma$ doesn't seem to work for the Bernoulli map, I know that $\sigma_{2,1}=1$ and $\sigma_{1,2}=1$ which works with the delta function definition. However, $\sigma_{2,1}=\delta_{3,1}$ from \refeq{hopMatrix}, which should equal zero. Other than just the idea of being cyclic, I don't know why it yields one instead of zero, what am I missing?

	\item[Q2]
So, does "lattice state" mean the set of all points (field of all points?) which running through the Bernoulli map requires the specific winding number at that lattice site? 
\end{itemize}
\end{description}
