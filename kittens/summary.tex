% siminos/kittens/summary.tex      pdflatex CL18
% $Author: predrag $ $Date: 2020-07-02 11:19:27 -0500 (Thu, 02 Jul 2020) $

\section{Summary and discussion}
\label{s:summary}

In this paper we have analyzed the {\catlatt}  \refeq{dDCatsT}
linear symbolic dynamics. We now summarize our main findings.

%%%%%%%%%%%%%%%%%%%%%%%%%%%%%%%%%%%%%%%%%%%%%%%%%%%%%%%%%%%%%%%%%%%%%%%%
    \PC{2016-11-08} {
Say: THE BIG DEAL is

for $d$\dmn\ field theory, symbolic dynamics is not one temporal sequence
with a huge alphabet, but $d$\dmn\ {\spt} tiling by a finite alphabet

``Classical foundations of many-particle
quantum chaos'' I believe could become a game-changer

corresponding dynamical zeta functions
should be sums over {\twots}, rather than $1$-dimensional \po s
    }
%%%%%%%%%%%%%%%%%%%%%%%%%%%%%%%%%%%%%%%%%%%%%%%%%%%%%%%%%%%%%%%%%%%%%%%%

%%%%%%%%%%%%%%%%%%%%%%%%%%%%%%%%%%%%%%%%%%%%%%%%%%%%%%%%%%%%%%%%%%%%%%%%
    \BG{2016-11-01} {
{\bf ``Deeper insight'' into $d=2$ symbolic dynamics}:
Relevance to semiclassics.
    }
%%%%%%%%%%%%%%%%%%%%%%%%%%%%%%%%%%%%%%%%%%%%%%%%%%%%%%%%%%%%%%%%%%%%%%%%

%%%%%%%%%%%%%%%%%%%%%%%%%%%%%%%%%%%%%%%%%%%%%%%%%%%%%%%%%%%%%%%%%%%%%%%%
    \PC{2016-11-10} {Curb you enthusiasm
{\bf How to think about matters {\spt}?}
%\label{s:introSpTemp}
text currently purged from the introduction:
\\

Laws of motion drive a spatially extended system (clouds, say) through a
repertoire of unstable patterns, each defined over a finite  {\spt}
region.

But in dynamics, we have no fear of the infinite extent in time. That is \po\
theory's\rf{DasBuch} raison d'\^{e}tre; the dynamics itself describes the
infinite time strange sets by a hierarchical succession of \po s, of longer
and longer, but always finite periods (with no artificial external
periodicity imposed along the time axis). And, since 1996 we know how to deal
with both spatially and temporally infinite regions by tiling them with
finite {\spt}ly periodic tiles\rf{Christiansen97,GHCW07}. More
precisely: a time periodic orbit is computed in a finite time, with period
\period{}, but its repeats ``tile'' the time axis for all times. Similarly, a
{\spt}ly periodic ``tile'' or ``\twot'' is computed on a finite
spatial region $L$, for a finite period \period{}, but its repeats in both
time and space directions tile the infinite spacetime.

Taken together, these open a path to determining exact solutions on
\emph{spatially infinite} regions.
This is important, as many turbulent flows of physical interest come equipped
with $D$ continuous spatial symmetries. For example, in a pipe flow at
transitional Reynolds number, the azimuthal and radial directions (measured
in viscosity length units) are compact, while the pipe length is infinite.
If the theory is recast as a $d$\dmn\ space-time theory,
\(d= D +1\,,\)
{\spt}ly translational invariant recurrent solutions are \dtors\
(and \emph{not} the $1$\dmn\ \po s of the traditional periodic orbit theory),
and the symbolic dynamics is likewise $d$\dmn\ (rather than what is
today taken for given, a single 1\dmn\ temporal string of symbols).

This changes everything. Instead of studying time evolution of a chaotic
system, one now studies the repertoire of {\spt} patterns allowed by
a given PDE.
To put it more provocatively: junk your old equations and look for guidance
in clouds' repeating patterns.
There is no more \emph{time} in this vision of nonlinear \emph{dynamics}!
Instead, there is the space of all {\spt} patterns, and the
likelihood that a given finite {\spt}ly pattern can appear, like the
mischievous grin of Cheshire cat, anywhere in the turbulent evolution of a flow.
A bold proposal, but how does it work?
\\

and thus a $d$\dmn\ {\spt} pattern is
mapped one-to-one onto a $d$\dmn\ discrete lattice state, symbolic
dynamics labelled configuration - a configuration very much like that of an
Ising model of statistical mechanics.
    } %end censored \PC{2016-11-10 Curb you enthusiasm
%%%%%%%%%%%%%%%%%%%%%%%%%%%%%%%%%%%%%%%%%%%%%%%%%%%%%%%%%%%%%%%%%%%%%%%%



\subsection{Discussion.}

    \PC{2016-12-24}{``
Alternatively, one can consider the dynamics on the
infinite line, and interpret $\Ssym{\zeit}$ as a jump to $\Ssym{\zeit}$th interval. This
leads to the phenomenon of ``deterministic diffusion''\rf{GroFuj82,ScFrKa82},
and its \po\ theory\rf{art91,CGS92}, with unit circle \po s in one-to-one
relation to the relative periodic (``running'') orbits on the line, and
symbolic dynamics given by $\Ssym{\zeit}$'s.

For single-parameter, 1\dmn\
sawtooth maps, it is possible to find infinitely many values of the parameter
such that the grammar is finite (a finite subshift), and the exact diffusion
constant is given by a finite-polynomial \tzeta\rf{CBdiffusion}. For cat maps, deterministic diffusion constants
are not known exactly\rf{ArtStr97}.
''  }

\PC{2020-05-31} {
Politi and Torcini\rf{PolTor92} note that
a problem in reconstructing the statistical properties of an
{\spt\ H{\'e}non} attractor
is ensuring that all \twots\  used are embedded into the inertial manifold.
For instance, in the single H{\'e}non map, one
of the two fixed points is isolated and it does not belong to the strange
attractor.
    }

\PC{2019-06-26}{
Mramor and Rink\rf{MraRin12}
{\bf $d$-dimensional Frenkel-Kontorova lattice:}
Here, the goal is to find a
$d$-dimensional ``lattice configuration'' $x:\integers^d\to \reals$ that satisfies
\beq\label{RR}
V'(\ssp_i) - (\Delta \ssp)_i = 0 \  \ \mbox{for all} \ i\in \mathbb{Z}^d
\,.
\eeq
The smooth function $V: \mathbb{R} \to \mathbb{R}$
satisfies $V(\xi+1)=V(\xi)$ for all $\xi\in\reals$. It has the interpretation
of a periodic onsite potential.

I like their definition of the discrete Laplace operator
$\Delta:\mathbb{R}^{\mathbb{Z}^d}\to\mathbb{R}^{\mathbb{Z}^d}$, defined as
\beq\label{Lap}
(\Delta x)_i := \frac{1}{2d} \sum_{||j-i||=1} \!\! (\ssp_j - \ssp_i)
\ \mbox{for all} \ i \in \integers^d
\,.
\eeq
where $||i||:=\sum_{k=1}^{d}|i_k|$.
Thus, $(\Delta x)_i$ is the average of the quantity $\ssp_j-\ssp_i$ computed
over the lattice points that are nearest to that with index $i$, \ie, the
graph Laplacian\rf{Pollicott01,Cimasoni12} \refeq{gaphLapl} for the case
of hypercubic lattice, or the ``central difference operator''\rf{PerViv}.
  }

\PC{2019-06-26}{
Mramor and Rink\rf{MraRin12}: ``
Eq.~(\ref{RR}) is relevant for statistical mechanics, because it is
related to the Frenkel-Kontorova Hamiltonian lattice differential
equation
\beq \label{FKHam}
\frac{d^2 \ssp_i}{dt^2} + V'(\ssp_i) - (\Delta \ssp)_i = 0 \ \mbox{for all} \ i\in\mathbb{Z}^d.
\eeq
This differential equation describes the motion of particles under the
competing influence of an onsite periodic potential field and nearest
neighbor attraction. Eq.~(\ref{RR}) describes its
stationary solutions.
  }

\bigskip\bigskip

\PC{2018-04-05}{
{\bf Proposition 17.2.} [Godsil and Royle\rf{GodRoy13}]
Given any directed graph G if B is the incidence matrix of G, A is the
adjacency matrix of G, and D is the degree matrix such that
$D_{ii}=d(v_i)$, then
\beq
B\transp{B} = D - A
\,.
\ee{gaphLapl}
The matrix $L=D-A$ is called the (unnormalized) graph Laplacian of
the graph G.
$B\transp{B}$ is independent of the orientation of G and
D-A is symmetric, positive, semidefinite; that is, the eigenvalues of D.

Each row of L sums to zero (because $\transp{B}{\bf1}=0$). Consequently,
the vector $\bf1$ is in the nullspace of L.

There might be a related undirected network
model, with a graph Laplacian \refeq{gaphLapl}. In that case a Lagrangian
formulation (in terms of graph Laplacians) might be a more powerful formulation
than their Hamiltonian one. ``Arrow of time'' is perhaps encoded by the
orientations of the links in a directed complex network.

    ''}

Remarkably, as far as the linear symbolic dynamics  is  concerned, the
above results hold both for the single cat map and its coupled lattice
generalization.
In both cases the  proofs rely only upon ellipticity of the operator
$\Box$ and the linearity of the equations. It is very plausible  that the
same results hold for the lattices $\integers^d$  of an arbitrary
dimension $d$. Furthermore, the restriction to the integer valued
matrices in the definitions of maps appears unnecessary. Cat map is a
smooth version of the sawtooth map, defined by the same equation
\refeq{OneCat}, but for a real (not necessarily integer) value of $s$.
The linear symbolic dynamics  for single saw map has been analyzed in
\rf{PerViv} and its extension  to a coupled $\integers^d$ model along the
lines of the present paper seems to be straightforward. Also,  in the
current paper we sticked to  the Laplacian  form of $\Box$.   Again this
seems to be too restrictive and extension to other elliptic  operators of
higher order should be possible. Such operators are necessarily appear
within the models with higher range of interactions.

A physically necessary extension of  current setting would be addition of an
external periodic potential ${V}$ to \refeq{dDCatsT}, rendering this a
nonlinear problem,
 \begin{equation}
 (\Box - s+ 2d+ {V}'(\ssp_z)) \ssp_z = \Ssym{z}, \qquad z\in \integers^{d}
 \,. \label{LinearConnPerturbed}
\end{equation}
As long as the perturbation ${V}$ is sufficiently weak, this lattice map can
be conjugated to the linear {\catlatt}, with ${V}=0$.
This approach has been used in \refref{GutOsi15} to construct partner
{\twots} for perturbed cat map lattices.
On the other hand, for a sufficiently strong perturbation, such a conjugation
to linear system is no longer possible. Finally, let us note that the lattice
models like  \refeq{LinearConnPerturbed} can be seen  as discretized versions
of PDEs.
In this respect it would be of  interest  to study whether our results  can
be extended to the continuous, PDE setting. In particular,  the following
questions  seem to be  of fundamental importance:

 \begin{itemize}
\item
 Can  an effective $d=2$ symbolic dynamics with finite alphabet  be
 constructed for an example of a PDE with {\spt} chaos, such
 that
 (a)  Connection between periodic field  solutions and their
 symbolic representation is unique;
 (b) The local  symbolic content would
 define  the values of the corresponding fields with the exponentially
 decreasing errors?
\end{itemize}

While the setting is classical,
such classical field-theory advances offer new semi-classical
approaches to quantum field theory and many-body problems.
