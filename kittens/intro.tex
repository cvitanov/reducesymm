% siminos/kittens/intro.tex      pdflatex CL18
% $Author: predrag $ $Date: 2020-08-02 22:02:31 -0500 (Sun, 02 Aug 2020) $


% \section{Introduction}
% \label{s:intro}
\bigskip

\noindent
A temporally chaotic system is exponentially unstable with time: double the
time, and exponentially more orbits are required to cover its strange
attractor to the same accuracy. For large spatial extents, the complexity of
the spatial shapes also needs to be taken into account; double the spatial
extent, and exponentially as many distinct
{\spt} patterns will be required to describe the repertoire of
system's shapes to the same accuracy.
The systems whose temporal and spatial correlations decay sufficiently fast,
and whose ``physical'' dimension\rf{ginelli-2007-99,DCTSCD14} grows with
system size, are said to be ``{\spt}ly chaotic.''

Our goal here is to make this ``{\spt} chaos'' tangible and precise, by
offering the reader what we believe is its simplest explicit
manifestation, the {\em \catlatt} (a $d$\dmn\ lattice {\sPe})
\[
 (\Box -s+2d)\,\ssp_{z}  =  -\Ssym{z}
 \,,
\]
a classical field theory on a $d$\dmn\ hyper-cubic lattice, with an
``anti-harmonic" rotor $\ssp_{z}$ at each site $z$ coupled to its nearest
neighbors.
In contrast to a field theory described by harmonic, Helmholtz equation
oscillatory solutions, {\catlatt} solutions are hyperbolic and
`turbulent', in the same sense that in contrast to stable oscillations of
a harmonic oscillator, Bernoulli coin flip solutions are unstable and
`chaotic'.

A physical motivation for the model comes from studies of {\spt}ly
homogenous turbulent flows: a very rough approximation to such flows is
discretizing them into {\spt} cells, with each cell turbulent, and cells
coupled to their nearest neighbors.  {\catLatt} is arguably the simplest
such model, no closer to physical turbulence than the Lorenz
model\rf{lorenz} is to weather, but still capturing the essential
qualitative features of such phenomena.

The main thrust of this paper is, however, much more radical: we've been
doing `turbulence' all wrong, and we even know that since \Poincare's
time. In ``explaining'' chaos we talk the talk as though we have never
moved beyond Newton; there is the initial states of a system, and there
are differential equations that evolve it forward in time. But when we
-all of us- do the work we do something altogether different, closer to
Lagrange (and the 20th century `spacetime' physics). This paper realigns
the theory to what we actually {\em do} when solving ``chaos'' equations,
using not much more than linear algebra. In our formulation, there is no
time, and there are no ``Lyapunov" horizons; every solution $\Xx$ is a
solution of a \spt\ fixed point condition $F[\Xx]=0$, and there is no
exponential blowup of anything.

As in what we present here we cross much unfamiliar territory to many, we
follow Mephistopheles pedagogical dictum ``You have to say it three
times"\rf{GoetheIstuZim1806}, and we sing our song thrice.

The first time, for a reader too busy\rf{focusPOT} to read the
book\rf{ChaosBook}, we disguise a brief course on chaos theory as
something everyone understands, a Bernoulli coin toss,
\refsect{s:coinToss}.
In \refsect{s:1D1dLatt} we introduce the `\templatt', a lattice
reformulation of the Bernoulli map.
The deep insight here is the realization that the volume \refeq{detBern0}
of the {\em\jacobianOrb} (the {\em\HillDet},
\reffig{fig:BernCyc2Jacob}, \ref{fig:catCycJacob} and
\ref{fig:BravaisLatt}), counts all global solutions of the
defining equations.

The second time, as a `\templatt', a 1\dmn\ lattice of rotors that all
dynamicists understand, \refsect{s:kickRot}.
In \refsect{s:catPV} we review the traditional cat map in its
Hamiltonian formulation (relegating to \refappe{s:catMapHam} the explicit
\AW\ generating partition of the cat map \statesp).
In \refsect{s:catLagrange} we introduce the `\templatt', a lattice
reformulation of the cat map.
The \po s theory of cat maps, \refsect{s:tempCatCount}, can be developed
in either formulation: both the forward-in-time Hamiltonian cat map \po s
counting (\refappe{s:catHamCount}) and the Lagrangian {\templatt} \po s
counting (\refsect{s:tempCatCount}) lead to the same {\tzeta}, with the
two formulations related by {Hill's formula} (to which we return in
\refsect{s:HillForm}).

The third time, herding cats all over spacetime, as a field theory
similar to ones studied by statistical mechanicians and field theorists,
\refsect{s:catlatt}.
After a brief review of the traditional coupled map lattices,
\refsect{s:CCMs}, we extend the \templatt\ to the $d$\dmn\ {\em\spt} cat,
\refsect{s:catLatt}. The reader who already knows everything can start
here.
In \refsect{s:BravaisLatt} we show that the system admits  a natural
$d$\dmn\ symbolic  code with a finite alphabet, and then study finite
{\spt} symbol \brick s.

\refSect{s:catLattCount} %~{\em Invariant tori in $d$\dmn\ \catlatt}
describes our counting solutions of a
$d$\dmn\ \catlatt.

In \refsect{s:catLattShadow}  we use these  results to construct
sets of {\spt} \twots\ that partially shadow each other.

The key to solution counting problem is the enumeration of \emph{prime}
{\twots}, with the notion of `prime' different for
lattices (the coordinate system) and \twots\ (the fields over these
coordinates, \refsect{s:prime}.

The results are summarized and some open questions discussed in the
\refsect{s:summary}.

Finally, in \refappe{s-SymbDynGloss} we collect the symbolic dynamics
definitions used throughout the paper.
