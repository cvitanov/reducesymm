% siminos/kittens/intro.tex      pdflatex CL18
% $Author: predrag $ $Date: 2020-08-02 22:02:31 -0500 (Sun, 02 Aug 2020) $


\section{Introduction}
\label{s:intro}

A temporally chaotic system is exponentially unstable with time: double the
time, and exponentially more \po s are required to cover its strange
attractor to the same accuracy. For large spatial extents, the complexity of
the spatial shapes also needs to be taken into account; double the spatial
extent in a given direction, and exponentially as many distinct
{\spt} patterns will be required to describe the repertoire of
system's shapes to the same accuracy.
The systems whose temporal and spatial correlations decay sufficiently fast,
and whose ``physical'' dimension\rf{ginelli-2007-99,DCTSCD14} grows with
system size, are said to be ``{\spt}ly chaotic.''

    \PC{2019-12-27}{
\catlatt\ = classical field theory on a $d$\dmn\ hyper-cubic lattice, with an
``anti-harmonic" rotor at each site, coupled to its nearest neighbors
    }

    \PC{2019-06-26}{
Physical picture:

Turbulence everywhere in space, with a range of length scales. Discretize into
cells, with each cell turbulent, and cells coupled
to their nearest neighbors\rf{Kaneko83}.

As a function of the strengths of cell-cell couplings, dynamics can exhibit rich
phase-transitions structure\rf{Kaneko84}. In this paper we chose couplings such
that the system is fully turbulent.

Explain word ``turbulence" as used here.

Hamiltonan, so symplectic or area preserving, but that is not essential.
Cite the Hamiltonian zeta function
from ChaosBook.

The main point: we've been doing it all wrong, and we know that since
\Poincare.
In ``explaining'' chaos we talk the talk as though we never moved beyond Newton.
But people who actually compute solutions do something altogether different,
closer to Lagrange (and the late 20th century, `spacetime' physics).
This paper realigns the theory to what we actually {\em do} when
solving ``chaos'' equations, using nothing more than the well known linear
algebra.
    }

coupled map lattice models

the spacetime discretized

dynamics of small-scale spatial structures modeled by discrete time
maps

single cell dynamics  attached to lattice sites,

coupling to neighboring sites

the Gutkin and Osipov\rf{GutOsi15}
$d$\dmn\ coupled cat maps lattice
(``{\catlatt}'' for short, in what follows),
a {\spt} generalization of the Percival and Vivaldi\rf{PerViv} {linear
code} for temporal evolution of a single cat map


the $d$\dmn\ lattice {\sPe}
\[
 (\Box -s+2d)\,\ssp_{z}  =  -\Ssym{z}
 \,.
\]

from the cat maps (modeling the
Hamiltonian dynamics of individual ``particles'') at sites of a
$(d\!-\!1)$\dmn\ spatial lattice, linearly coupled to their nearest
neighbors.

solution $\Xx$ of a global fixed-point condition
$F[\Xx]=0$ is uniquely encoded by a finite alphabet $d$\dmn\ symbol
lattice state  $\Mm$



which symbol \brick s are {\admissible}?

The  linearity of the {\catlatt} enables us to

standard crystallographic  methods\rf{Dresselhaus07} and
integer lattices counting\rf{Barvinok08} enable us to count {\spt}ly finite \brick s,
and give explicit formulas for the number of \dtor\ solutions
for \brick s of any size.

Implementing this program requires several tools not standard in
dynamicist's tool box: lattice Green's functions; lattice determinants.

We start the paper with a reformulation of the 1 degree of freedom
Bernoulli map, because our goal, the \catlatt\ is nothing but its
generalization to a mechanical system in spacetimes of arbitrary
dimension, and thus arguably the simplest possible example of a `chaotic
field theory'.

\bigskip

The paper is organized as follows:
For a reader too busy\rf{focusPOT} to read the book\rf{ChaosBook}, we
start in \refsect{s:coinToss} with a brief course on `chaos' theory,
disguised as a humble coin toss. The deep insight here is the realization
that the volume \refeq{detBern0} of the {\jacobianOrb} (the functional
determinant of the {\fundPip} or the {\HillDet}, see
\reffig{fig:BernCyc2Jacob}, \ref{fig:catCycJacob} and
\ref{fig:BravaisLatt}), counts the numbers of global solutions for a
given `law', true at all times and at all spatial positions.
Before turning to the spatially infinite field theory in
\refsect{s:catlatt}, it is instructive to motivate our formulation of the
{\catlatt} by investigating the temporal lattice Bernoulli and cat
systems (\ie, `\spt\ lattices' with only one site in the spatial
direction).
In \refsect{s:catPV} we review the traditional cat map in its usual,
Hamiltonian formulation,  and \refappe{s:catMapHam} construct an explicit
generating (\AW) partition of the cat map \statesp.
In \refsect{s:catLagrange} we introduce the `\templatt', a global lattice
reformulation of the cat map.
The \po s theory of cat maps, \refsect{s:tempCatCount}, can be developed in either
formulation:
both the Hamiltonian cat map \po s counting (\refappe{s:catHamCount}) and
the Lagrangian {\templatt} \po s counting (\refsect{s:tempCatCount}) lead
to the same {\tzeta}, with the two formulations related by {Hill's
formula} (\refsect{s:HillForm}).
A reader may skip \refSecttosect{s:catPV}{s:tempCatCount} on the
first reading, as the paper proper starts with \refsect{s:catlatt}.

In \refsect{s:catlatt} (after a brief review of the traditional coupled
map lattices, \refsect{s:CCMs}), we extend the \templatt\ to
the $d$\dmn\ \catlatt, \refsect{s:dDcatLatt}.
In \refsect{s:BravaisLatt} we show that the system admits  a natural
$d$\dmn\ symbolic  code with a finite alphabet, and then study finite
{\spt} symbol \brick s.

\refSect{s:catLattCount} %~{\em Invariant tori in $d$\dmn\ \catlatt}
describes our counting solutions of a
$d$\dmn\ \catlatt.

In \refsect{s:catLattShadow}  we use these  results to construct
sets of {\spt} \twots\ that partially shadow each other.

The results are summarized and some open questions discussed in the
\refsect{s:summary}.

In \refappe{s-SymbDynGloss} we collect the symbolic dynamics definitions
needed throughout the paper.
