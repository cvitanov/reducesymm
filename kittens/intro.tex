% siminos/kittens/intro.tex      pdflatex CL18
% $Author: predrag $ $Date: 2020-09-20 17:55:42 -0500 (Sun, 20 Sep 2020) $


% \section{Introduction}
% \label{s:intro}
\bigskip

\noindent
A temporally chaotic system is exponentially unstable with time: double the
time, and exponentially more orbits are required to cover its strange
attractor to the same accuracy. For large spatial extents, the complexity of
the spatial shapes also needs to be taken into account; double the spatial
extent, and exponentially as many distinct
{\spt} patterns will be required to describe the repertoire of
system's shapes to the same accuracy.
The systems whose temporal and spatial correlations decay sufficiently fast,
and whose ``physical'' dimension\rf{ginelli-2007-99,DCTSCD14} grows with
system size, are said to be ``{\spt}ly chaotic.''

Our goal here is to make this ``{\spt} chaos'' tangible and precise, by
offering the reader what we believe is its simplest explicit
example, the {\em \catlatt}\rf{GutOsi15,GHJSC16}
(a screened Poisson or Yukawa equation)
\[
 (\Box -d(s-2))\,\Xx  =  -\Mm
 \,,
\]
a classical field theory on a $d$\dmn\ hyper-cubic lattice, with an
``anti-harmonic" rotor $\ssp_{z}$ at each site $z$ coupled to its nearest
neighbors.
In contrast to a field governed by its close relative, Helmholtz equation, with
oscillatory solutions, {\catlatt} solutions are hyperbolic and
`turbulent', in the same sense that in contrast to stable oscillations of
a harmonic oscillator, Bernoulli coin flip solutions are unstable and
chaotic.

{\Spt}ly homogenous turbulent flows offer one physical motivation for
considering such models: a very rough approximation to such flows is
discretizing them into {\spt} cells, with each cell turbulent, and cells
coupled to their nearest neighbors.  {\catLatt} is arguably the simplest
such model, no closer to physical turbulence than the Lorenz
model\rf{lorenz} is to weather, but still capturing the essential
qualitative features of {\spt} chaos.

The main thrust of this paper is, however, radical: we've been doing
`turbulence' all wrong, and we know that since Poincar{\'e}'s times. In
``explaining'' chaos we talk the talk as though we have never moved
beyond Newton; here is an initial state of a system, local in time, and
here are the differential equations that evolve it forward in time. But
when we -all of us- do the work, we do something altogether different,
closer to Lagrange, and to the 20th century `spacetime' physics. This
paper realigns the theory to what we actually {\em do} when solving
``chaos'' equations, using not much more than linear algebra. In this
formulation, there is no time, and there is no ``Lyapunov" horizon; every
solution $\Xx$ is a global solution of a \spt\ fixed point condition
$F[\Xx]=0$, and there is no exponential blowup of anything.

As we shall here have to traverse territory unfamiliar to many, we
follow Mephistopheles pedagogical dictum ``You have to say it three
times"\rf{GoetheIstuZim1806}, and sing our song thrice.

The first time, for a reader too busy\rf{focusPOT} to read the
book\rf{ChaosBook}, we disguise a brief course on chaos theory as
something everyone understands, a Bernoulli coin toss,
\refsect{s:coinToss}.
In \refsect{s:1D1dLatt} we introduce the `{temporal Bernoulli}', a
lattice reformulation of the Bernoulli map. The deep insight here is the
realization that the {\em\HillDet}, \ie, the volume \refeq{detBern0} of
the {\em\jacobianOrb} (\reffig{fig:BernCyc2Jacob},
\ref{fig:catCycJacob} and \ref{fig:BravaisLatt}) counts all global
solutions of given defining equations.

The second time, as a `\templatt', a 1\dmn\ lattice of rotors that all
dynamicists understand, \refsect{s:kickRot}.
In \refsect{s:catPV} we review the traditional cat map in its Hamiltonian
formulation (but relegate to \refappe{s:catMapHam} the explicit \AW\
generating partition of the cat map \statesp).
In \refsect{s:catLagrange} we introduce the `\templatt', a lattice
reformulation of the cat map.
The \po s theory of cat maps can be developed in either formulation: both
the forward-in-time Hamiltonian cat map, \refappe{s:catHamCount}, and the
Lagrangian {\templatt} %, \refsect{s:tempCatCount},
lead to the same {\tzeta} \refeq{Isola90-13} count of \emph{prime} \po s,
with the two formulations related by {Hill's formula} \refeq{noPerPts}.

The third time, herding cats all over spacetime, as a field theory
similar to ones studied by statistical mechanicians and field theorists,
\refsect{s:catlatt}.
In \refsect{s:CCMs} we review the traditional coupled map lattice model
discretizations of dissipative PDEs, as well as many-body Hamiltonian
models, \refsect{s:HCCMs}.
In \refsect{s:catLatt} we generalize the \templatt\ model to the $d$\dmn\
{\em\spt} cat,
and in \refsect{s:BravaisLatt} show that the system admits  a
$d$\dmn\ symbolic  code with a finite alphabet. We then turn to
study of admissible finite
{\spt} symbol \brick s.
The key to solution counting problem is the enumeration of \emph{prime}
{\twots}, with the notion of `prime' now subtler than what it was for
1\dmn\ lattices, different for the integer lattice coordinate system,
\refsect{s:primeLatt}, from \twots, \ie, the fields over these
coordinates, \refsect{s:lattState}.
\refSect{s:catLattCount} %~{\em Invariant tori in $d$\dmn\ \catlatt}
illustrates our \twot\  \catlatt\ solutions by several explicit examples.
In \refsect{s:catLattShadow}  we use the \catlatt\ symbolic dynamics to show
that {\spt} \twots\ that share finite {\spt} symbol \brick s shadow each
other to exponential precision
(the symbolic dynamics definitions used throughout the paper are
collected in \refappe{s-SymbDynGloss}).

\refSect{s:prime} tabulates our prime \po s counts. {Hill's formula} for
the 2\dmn\ lattice \catlatt\ is derived in \refsect{s:HillForm}.
We evaluate and cross-check  {\HillDet}s by two methods, either the
`fundamental fact' evaluation, \refsect{s:catLattRel3x2}, or by the
discrete Fourier transform diagonalization, \refappe{appe:Fourier}.

The results are summarized and some open questions discussed in the
\refsect{s:summary}.
