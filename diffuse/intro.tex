% diffuse/intro.tex          pdflatex ZhCvGo15

% Predrag                                3dec2015
% Tingnan                                2nov2015
% Predrag initial draft                 21nov2014
%         extracted from ChaosBook.org
% \Chapter{diffusion}{2apr2014}{Deterministic diffusion}

% \section{Introduction}

The advances in the theory of dynamical systems have brought a new life
to Boltzmann's mechanical formulation of statistical mechanics,
especially for systems near or far from equilibrium, and yielded new sets
of microscopic dynamics formulas for macroscopic observables such as the
transport coefficients. Sinai,
Ruelle and Bowen (SRB) have generalized Boltzmann's notion of ergodicity
for a constant energy surface for a Hamiltonian system in equilibrium to
dissipative systems in{nonequilibrium} stationary states. In this more
general setting the attractor plays the role of a constant energy
surface, and the SRB measure is a generalization of the Liouville
measure. Such measures are purely microscopic and indifferent to whether
the system is at equilibrium, close to equilibrium or far from it. ``Far
for equilibrium'' in this context refers to systems with large deviations
from Maxwell's equilibrium velocity distribution. Furthermore, the theory
of dynamical systems has yielded new sets of microscopic dynamics
formulas for macroscopic observables such as diffusion constants, to
which we turn now.

The classical Boltzmann equation for evolution of 1-particle density is
based on stosszahlansatz, neglect of particle correlations prior to, or
after a 2-particle collision. It is a very good approximate description
of dilute gas dynamics, but a difficult starting point for inclusion of
systematic corrections. In the theory of deterministic diffusion
developed in recent years, no correlations are neglected - they are all
included in the exact cycle expansions for transport coefficients such as
the diffusion constant.


%
%%%%%%%%%%%%%%%%%%%%%%%%%%%%%%%%%%%%%%%%%%%%%%%%%%%%%%%%%%%%%%%%%%
%\SFIG{fig_lor_4}
%{}{
%Deterministic diffusion in a
%finite horizon periodic Lorentz gas.
%\hfill (T. Schreiber)
%}{fig-lor-4}
%%%%%%%%%%%%%%%%%%%%%%%%%%%%%%%%%%%%%%%%%%%%%%%%%%%%%%%%%%%%%%%%%%
%


Chaotic motions exist in many field of physics systems, blah. There are
physical problems such as beam defocusing in particle accelerators or
chaotic behavior of passive tracers in $2$\dmn\ rotating flows which can
be described as deterministic diffusion in periodic arrays. In the field
of animal/robotic locomotion, we will show that a macroscopic ``diffusion
view'' also applies.

Chaotic motions exist in many systems. There are physical problems such
as beam defocusing in particle accelerators \TZ{2015-11-02}{reference?}

In the macroscopic world, there are recent
studies shown that robotic locomotion in heterogeneous granular
environment also demonstrates scattering-diffusive pattern.

%In biological field,  many important dynamical processes (often at
%cellular level) are described in  terms of diffusion coefficients. Such
%examples include the transport of ions  across the cell
%membranes\rf{stein2012transport} and the movement of  microorganism(e.g.
%bacterials) through natural  ecosystems\rf{koch1990diffusion}. In this
%paper we will discuss the transport  property of more "macroscopic"
%systems (such as moving  robots\rf{saranli2001rhex}) where a ``diffusive
%description'' also applies.

%Lately, there has been an increased focus on robot locomotion in complex
%environments (check Science and ROPP reference, the systematic study of
%interactions between environment and locomotion, which we now
%call``robophysics''). Many of those studies use substrates that are
%spatially homogeneous and we have a good
%understanding\rf{li2009sensitive,li2013terradynamics}. However, little is
%known for locomotion in heterogeneous environment. There are some limited
%experimental/theoretical studies for relatively simple settings (e.g.
%slopes,ref). In this paper we intend to approach the longterm transport
%properties of locomotion in a more complex environment, i.e. in a field
%of scatterers of which the scales are comparable to the locomotor.

The $2$\dmn\ Lorentz gas is an infinite scatterer array in which
diffusion of a light molecule in a gas of heavy scatterers is modeled by
the motion of a point particle in a plane bouncing off an array of
reflecting disks. The Lorentz gas is called ``gas'' because one can
equivalently think of it as consisting of any number of point-like fast
``light molecules'' interacting only with the stationary ``heavy
molecules'' and not among themselves.  As the scatterer array is built up
from only defocusing concave surfaces, it is a pure hyperbolic system,
and one of the simplest nontrivial dynamical systems that exhibits
deterministic diffusion, \reffig{fig-chaoticBouncing}.

% The original Lorentz gas assumed a random distribution of heavy
% scatterers; in this case a probabilistic description is unavoidable.

The diffusive properties of a `Lorentz gas'\rf{Lorentz1905}
have been studied extensively in the literature,
see \refrefs{Sinai70,BunSin80,Bunimovich85,GasNic90,MacZwa83}.

The approach introduced in \refref{LorentzDiff} and tested in
\refref{CGS92}, exploits the fact that the periodic Lorentz gas can be
constructed by putting together translated copies of an elementary cell.
Therefore quantities characterizing global dynamics, such as the Lyapunov
exponent and the diffusion constant, can be computed from the dynamics
restricted to the elementary cell.

In \refref{LorentzDiff} some effort was made to derive a diffusion
formula for the fully symmetry-reduced dynamics, involving only
quantities computed within the fundamental domain. The fact that lattice
translations do not commute with the symmetry group within the elementary
cell makes this apparently a difficult task.

As a gedanken experiment, suppose a passively controlled robot is moving
in a boulder field at constant speed. The diffusion coefficient, which
describes roughly how much area the robot explored in a unit time, is the
key quantity we would like to investigate. We place the boulder in a
regular, periodic array and assume that we are in the heavy boulder limit
such that after each collision event, only the robot is deflected and
boulders remain immobilized. With the presumptions we effectively created
a periodic Lorentz gas model\rf{Dettm14} for locomotion in a boulder
field.

To investigate the transport property of such systems, we apply cycle
expansions\rf{DasBuch} to the analysis of {\em diffusion coefficient}.
The resulting formulas are exact; no probabilistic assumptions are made,
and all correlations are taken into account by the  inclusion of cycles
of all periods. While existing cycle expansion theory yields the correct
result by tiling the {\statesp} into elementary cells, the convergence
rate is slow because of bad shadowing and poor choice of symbolic
grammar\rf{CGS92}. In this paper we propose a novel approach that
significantly improves the efficiency of cycle expansion formula, by
factorizing the non-commuting rotational and translational symmetry and
using periodic orbits in the fundamental domain.

% The infinite extent systems for which the  periodic orbit theory yields
% formulas for diffusion and other transport  coefficients are spatially
% periodic, the global {\statesp} being tiled with  copies of a elementary
% cell.

In \refsect{s-DiffPerArr} we briefly review the formulas for diffusion
coefficients in the $2$\dmn\ periodic Lorentz gas, using dynamics
restricted in elementary cell. In \refsect{s-fundDom} we
factorize the rotational symmetry and derive the diffusion coefficient
using fundamental domain cycles. Because we will work with different
kinds of \statesp, through out the text we will repeatedly using tildes
($\tilde{\quad}$), nothings and hats ($\hat{\quad}$) atop symbols to
signify the dynamical quantities in the fundamental domain, elementary
cell and full state space, respectively.


\bigskip
=========== TO REUSE ========

    \PC{edits based Cvitanovi\'c,  Eckmann,and Gaspard\rf{LorentzDiff}}

Earlier work\rf{LorentzDiff} has attempted but failed to obtain a formula
for the diffusion matrix expressed in terms of cycles on the fundamental
domain $\tM$.


\bigskip

    \PC{the text from Cvitanovi\'c, Gaspard and Schreiber\rf{CGS92}}
The Lorentz gas\rf{Lorentz1905} is one of the simplest nontrivial models
of deterministic diffusion.
Diffusion of a light molecule in a gas of heavy scatterers
is modelled
by a point particle in a plane bouncing off an array of reflecting disks.

This might seem a hopeless task, as one
has to deal with all periodic and aperiodic
solutions of an infinitely extended system. An
approach based on larger and larger finite portions of the system is described
in \refref{GasNic90}, with the diffusion constant related to
%the scaling behaviour of
the escape rates from such finite portions.
%with its size.
%PG
As far as escape rates are obtained in direct numerical simulations
this approach has been shown to be effective\rf{GaBaMaHo92}.
However, from the cyclists point of view
%PG
where the rates are calculated from the periodic orbits,
this approach is impractical; with each added disk new peculiarities arise
in the enumeration of periodic orbits, and with current
methods there is little hope of getting results for more than a few disks,
and no hope of approaching the desired scaling limit.

When this elementary cell is itself invariant under a discrete symmetry
group $G$ the lattice can be tiled into images under $G$ and the lattice
translations of a fundamental domain.

The exact results are sometimes counterintuitive, and might help us
decide whether a diffusive phenomenon whose microscopic dynamics is hard
to observe directly, such as conductance fluctuations in a mesoscopic
device, is due to impurities or to deterministic transport. For example,
as some parameter (such as mean free flight time) is increased, the
deterministic diffusion coefficient reveals a non-monotone, fractal
dependence.
