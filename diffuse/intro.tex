% diffuse/intro.tex

% Predrag initial draft                 21nov2014
%         extracted from ChaosBook.org
% \Chapter{diffusion}{2apr2014}{Deterministic diffusion}


The advances in the theory of dynamical systems have brought a new life to
Boltzmann's mechanical formulation of statistical mechanics. Sinai, Ruelle and
Bowen (SRB) have generalized Boltzmann's notion of ergodicity for a constant
energy surface for a Hamiltonian system in equilibrium to dissipative systems in
{nonequilibrium} stationary states. In this more general setting the attractor
plays the role of a constant energy surface, and the SRB measure x is a
generalization of the Liouville measure. Such measures are purely microscopic
and indifferent to whether the system is at equilibrium, close to equilibrium or
far from it.  ``Far for equilibrium'' in this context refers to systems with
large deviations from Maxwell's equilibrium velocity distribution. Furthermore,
the theory of dynamical systems has yielded new sets of microscopic dynamics
formulas for macroscopic observables such as diffusion constants, to which we turn now.
%
%%%%%%%%%%%%%%%%%%%%%%%%%%%%%%%%%%%%%%%%%%%%%%%%%%%%%%%%%%%%%%%%%%
%\SFIG{fig_lor_4}
%{}{
%Deterministic diffusion in a
%finite horizon periodic Lorentz gas.
%\hfill (T. Schreiber)
%}{fig-lor-4}
%%%%%%%%%%%%%%%%%%%%%%%%%%%%%%%%%%%%%%%%%%%%%%%%%%%%%%%%%%%%%%%%%%
%

Chaotic motions exist in many field of physics systems, blah. Thre are physical
problems such as beam defocusing in particle accelerators or chaotic behavior of
passive tracers in $2$\dmn\ rotating flows which can be described as
deterministic diffusion in periodic arrays. In the field of animal/robotic
locomotion, we will show that a macroscopic ``diffusion view'' also applies.

Lately, there has been an increased focus on locomotion in complex environments. Researches in the physical sciences and in the field of robotics all find interest/relevance to this topic. (Now we can have a list of reference and recent studies in this topic). Many of those studies uses substrates that are spatially homogeneous and we have a relatively good understanding of it. Little is understood for locomotion in heterogeneous environment and principles are still lacking. Our lab is interested to study the transport properties of locomotors moving on heterogeneous terrain. Such study may potentially be beneficial to practical applications such mars navigation, hazardous rescue.

We consider the following locomotion problem. A passively controlled robot is moving in a boulder field at constant speed. We would want to study the long term dynamics of the system. The diffusion coefficient, which describes roughly how much area the robot explored in a unit time, is the key quantity here. We may place the boulder in a regular, periodic array and assume that we are in the heavy boulder limit such that after each collision event, only the robot is deflected and boudlers remain immobalized. With these preasumptions we effectively created a lorentz gas model for locomotion in a boulder field.

We shall apply cycle expansions to the analysis of {\em transport} properties of
chaotic systems. The resulting formulas are exact; no probabilistic assumptions
are made, and the all correlations are taken into account by the inclusion of
cycles of all periods.  The infinite extent systems for which the periodic orbit
theory yields formulas for diffusion and other transport coefficients are
spatially periodic, the global {\statesp} being tiled with copies of a
elementary cell.

In \refsect{s-DiffPerArr} we derive the formulas for diffusion
coefficients in a simple physical setting, the $2$\dmn\ periodic Lorentz
gas.

the Lorentz systems in \refsect{s-lorentz}.

\section{Diffusion in periodic arrays}
\label{s-DiffPerArr}

The $2$\dmn\ {\em Lorentz gas} is an infinite scatterer array in which
diffusion of a light molecule in a gas of heavy scatterers is modeled
by the motion of a point particle in a plane bouncing off an array of
reflecting disks.  The Lorentz gas is called ``gas'' as one can
equivalently think of it as consisting of any number of pointlike fast
``light molecules'' interacting only with the stationary ``heavy
molecules'' and not among themselves.  As the scatterer array is built
up from only defocusing concave surfaces, it is a pure hyperbolic
system, and one of the simplest nontrivial dynamical systems that
exhibits deterministic diffusion, \reffig{fig-lor-4}.
% The original Lorentz gas assumed a random distribution of heavy
% scatterers; in this case a probabilistic description is unavoidable.
We shall now show that the {\em periodic} Lorentz gas is amenable to a
purely deterministic treatment.  In this class of open dynamical
systems quantities characterizing global dynamics, such as the
Lyapunov exponent, pressure and diffusion constant, can be computed
from the dynamics restricted to the elementary cell.  The method
applies to any hyperbolic dynamical system that is a periodic tiling
$\hM=\bigcup_{ \hn \in T} \pS_{\hn}$ of the dynamical {\statesp} $\hM$
by {\em translates} $\pS_{\hn}$ of an {\em elementary cell} $\pS$,
with $T$ the abelian group of lattice translations.  If the scattering
array has further discrete symmetries, such as reflection symmetry,
each elementary cell may be built from a {\em fundamental domain}
${\widetilde \pS}$ by the action of a discrete (not necessarily
abelian) group $G$.  The symbol $\hM$ refers here to the full
{\statesp}, \ie,, both the spatial coordinates and the momenta.  The
spatial component of $\hM$ is the complement of the disks in the {\em
whole} space.

We shall now relate the dynamics in $\pS$ to diffusive properties of
the Lorentz gas in $\hM$.

These concepts are best illustrated by a specific example, a Lorentz
gas based on the hexagonal lattice Sinai billiard of
\reffig{fig-lor-1}.
%
%%%%%%%%%%%%%%%%%%%%%%%%%%%%%%%%%%%%%%%%%%%%%%%%%%%%%%%%%%%%%%%%%%
%\SFIG{fig_lor_1.ps}
%{}{
%Tiling of $\hM$, a periodic lattice of reflecting disks, by the
%fundamental domain ${\widetilde \pS}$. Indicated is an example of a
%global trajectory $\hx(t)$ together with the corresponding
%elementary cell trajectory ${x}(t)$ and the fundamental domain
%trajectory $\tx(t)$.
%(Courtesy of J.-P. Eckmann)
%}{fig-lor-1}
%%%%%%%%%%%%%%%%%%%%%%%%%%%%%%%%%%%%%%%%%%%%%%%%%%%%%%%%%%%%%%%%%%
%
We distinguish two types of diffusive behavior; the {\em infinite
horizon} case, which allows for infinite length flights, and the {\em
finite horizon} case, where any free particle trajectory must hit a
disk in finite time.  In this chapter we shall restrict our
consideration to the finite horizon case, with disks sufficiently
large so that no infinite length free flight is possible.  In this
case the diffusion is normal, with $\hat{x}(t)^2$ growing like $t$.
We shall discuss the anomalous diffusion case in
\refsect{intersec:andiff}.

PC{ask J.-P. Eckmann for permission to use his figures}

As we will work with three kinds of \statesp s, good manners require
that we repeat what tildes, nothings and hats atop symbols signify:
\bea
\tilde{\ }     &&
    \mbox{fundamental domain, triangle in \reffig{fig-lor-1}}
        \continue
%[nothing] \qquad \qquad &&
[0pt] \qquad \qquad &&
    \mbox{elementary cell, hexagon in \reffig{fig-lor-1}}
        \continue
\hat{\ }   &&
    \mbox{full {\statesp}, lattice in \reffig{fig-lor-1}}
\label{atops}
\eea
It is convenient to define an \evOper\ for each of the 3 cases of
\reffig{fig-lor-1}.
$\hx(t)\,=\,\hflow{t}{\hx}$
denotes the point in the global space
$\hM$
reached by the flow in time $t$.
$x(t)\,=\,\flow{t}{\xInit}$
denotes the corresponding flow in the elementary cell; the two are
related by
\beq
\hn_t(\xInit)= \hflow{t}{\xInit} - \flow{t}{\xInit} \in T
\,,
\ee{l-diff-hatn}
the translation of the endpoint of the global path into the elementary
cell $\pS$.  The quantity $\tx(t)\,=\,\tflow{t}{\tx}$ denotes the flow
in the fundamental domain
${\widetilde \pS}$;
$\tflow{t}{\tx}$ is related to
$\flow{t}{\tx}$ by a discrete symmetry
$g \in G$ which maps $\tx(t)\in {\widetilde \pS}$ to
${x}(t) \in {\pS}$ .

Fix a vector $\beta \in \reals^d$, where $d$ is the dimension of the
{\statesp}. We will compute the diffusive properties of the Lorentz
gas from the leading eigenvalue of the
\evOper\ \refeq{(2)}
\beq
\eigenvL(\beta)\,=\, \lim_{t \rightarrow \infty} \frac{1}{t} \log
\langle e^{\beta \cdot (\hx(t) -x) } \rangle_\pS
~, \quad
\ee{lor-diff-1}
where the average is over all initial points in the elementary cell,
$x \in \pS$. %, $\tx \in {\widetilde \pS}$ respectively.

If all odd derivatives vanish by symmetry, there is no drift and the
second derivatives
\[
2d D_{ij} =
\left . {{\partial} \over {\partial \beta_i}}
{{\partial} \over {\partial \beta_j}}
\eigenvL(\beta)\right |_{\beta=0} \,=\,\lim_{t\rightarrow \infty} {1\over t}
\langle {(\hx(t) -x)_i (\hx(t) -x)_j } \rangle_\pS ~,
\] %ee{lor-diff-4}
yield a diffusion matrix.  This symmetric matrix can, in general, be
anisotropic (\ie, have $d$ distinct eigenvalues and eigen\-vectors).
The spatial diffusion constant is then given by the Einstein relation
\refeq{(5)}
\[
D\,=\,{1\over 2 d} \sum_i
\left .{{\partial}^2 \over {\partial \beta^2_i}}
\eigenvL(\beta)\right |_{\beta=0}
\,=\, \lim_{t\rightarrow \infty} {1\over{2d t}}
\langle {(\hat{q}(t) -q)^2 } \rangle_\pS~
~,
\] %ee{lor_diff_5}
where the $i$ sum is restricted to the spatial components $q_i$ of the
{\statesp} vectors $x=(q,p)$, \ie, if the dynamics is Hamiltonian, the
sum is over the $d$ degrees of freedom.

PC{reinstate mass, velocity, size to get $\beta$, $m$, $\sigma$
    dependencies right}

We now turn to the connection between \refeq{lor-diff-1} and periodic
orbits in the elementary cell.As the full $\hM \rightarrow {\widetilde
\pS}$ reduction is complicated by the non-abelian nature of $G$, we
discuss only the abelian $\hM \rightarrow \pS$ reduction.


\subsection{Reduction from $\hM$ to $\pS$}
\label{s-Red-hMto-pS}

The key idea follows from inspection of the relation
\bea
\expct{ e^{\beta \cdot (\hx(t) -x)} }_\pS
&=& {1 \over {|\pS|}} \int_{x \in \pS \atop \hat{y} \in \hM}
dx d\hat{y}\,
e^{\beta \cdot (\hat{y} -x) }
\delta(\hat{y} - \hflow{t}{x})~.
\nnu
%\ee{lor-diff-6}
\eea
 ${|\pS|=\int_\pS dx}$ is the volume of the elementary cell $\pS$.
Due to translational symmetry, it suffices to start with a density of
trajectories defined over a single elementary cell $\pS$.
As in \refsect{s_aver_ev_op}, we have used the identity $ 1 = \intM{y}
\prpgtr{y - \hx(t)}$ to motivate the introduction of the {\evOper} $
\Lop^t(\hat{y},x)$.  There is a unique lattice translation $\hn$ such
that $\hat{y}=y - \hn$, with the endpoint $y \in \pS$ translated back
to the elementary cell, and $\flow{t}{x}$ given by
\refeq{l-diff-hatn}. The difference is a translation by a constant
lattice vector $\hn$, and the Jacobian for changing integration from
$d\hat{y}$ to $dy$ equals unity.  Therefore, and this is the main
point, translation invariance can be used to reduce this average to
the elementary cell:
\beq
\langle e^{\beta \cdot (\hx(t) -x) } \rangle_\pS
\,=\, {1 \over {|\pS|}} \int_{x,y \in \pS} dx dy\,
e^{\beta \cdot (\hflow{t}{x} -x) }
\delta(y - \flow{t}{x})~.
\ee{lor-diff-7}
As this is a translation, the Jacobian is $|\partial \hat{y}/\partial
y| = 1$.  In this way the global $\hflow{t}{x}$ flow, infinite volume
\statesp\ averages can be computed by following the flow
$\flow{t}{\xInit}$ restricted to the compact, finite volume elementary
cell $\pS$.  The equation \refeq{lor-diff-7} suggests that we study
the
\evOper
\beq
\Lop^t(y,x)\,=\,e^{\beta \cdot (\hx(t) -x) } \delta(y-\flow{t}{x})
\,,
\ee{lor-diff-8}
where $\hx(t)\,=\,\hflow{t}{x} \in \hM$ is the displacement in the
full space, but $x$, $\flow{t}{x}$, $y \in \pS$.  It is
straightforward to check that this operator satisfies the semigroup
property \refeq{SemiGrp},
\[
\int_{\pS} dz\,
\Lop^{t_2}(y,z) \Lop^{t_1}(z,x) \,=\,
\Lop^{t_2+t_1}(y,x)~
\,.
\]
For $\beta=0$, the operator \refeq{lor-diff-8} is the \FPoper\
\refeq{TransOp1}, with the leading eigenvalue $e^{\eigenvL_0}=1$
because there is no escape from this system (see the flow conservation
sum rule \refeq{prob-cons}).

The rest is old hat.
% \toSect{s-tr-flows}
The spectrum of $ \Lop$ is evaluated by taking the trace
\[
\tr\Lop^t\,=\,\int_\pS dx\,
 e^{\beta \cdot \hn_t(x) } \delta(x-x(t))~.
\] %ee{lor_diff_10}
Here $\hn_t(x)$ is the discrete lattice translation defined in
\refeq{l-diff-hatn}.  Two kinds of orbits periodic in the elementary
cell contribute.  A periodic orbit is called {\em standing}
\index{standing orbit!Lorentz gas} if it is also periodic orbit of the
infinite {\statesp} dynamics, $\hflow{\period{p}}{x}=x$, and it is
called {\em running} \index{running orbit!Lorentz gas} if it
corresponds to a lattice translation in the dynamics on the infinite
{\statesp}, $\hflow{\period{p}}{x}=x+\hn_p$.  We recognize the
shortest repeating segment of a running orbit as our old `\rpo' friend
from \refchap{c-discrete}.
In the theory of area--preserving maps such as the standard map of
\refexam{s:StandMap} these orbits are called {\em accelerator modes},
as the diffusion takes place along the momentum rather than the
position coordinate.  The traveled distance
$\hn_p=\hn_{\period{p}}(\xInit)$ is independent of the starting point
$\xInit$, as can be easily seen by continuing the path periodically in
$\hM$.
\index{accelerator mode}

PC{recheck usage ``standing,'' ``running'' with the literature.}

The final result is the \Fd\  \refeq{Z(s)}
\beq
\det(\eigenvL (\beta) - \Aop)  \,=\,\prod_{p}
\exp \left( - {
               \sum_{r=1}^\infty {1 \over r}
               {
                e^{(\beta \cdot \hn_p- s \period{p}) r}
                    % z^{\cl{p} r}
 \over  \oneMinJ{r}
                }
  % { | \det \left( {\bf 1}-\monodromy_p^{r} \right) | } }
              } \right)
\,,
\ee{lor-diff-14}
or the corresponding \dzeta\ \refeq{zet}
\beq
1/\zeta(\beta, s)\,=\,\prod_{p}\left( 1 - \frac{e^{(\beta \cdot
\hn_p- s \period{p})}}{|\ExpaEig_p|} \right)
~.
\label{zeta-diff}
\eeq
The \dzeta\ \cycForm\ \refeq{dzeta} for the diffusion constant
\refeq{(5)}, zero mean drift
$
\expct{ \hat{x}_i } = 0
\,,
$
is given by
\beq
D \,=\,{1 \over 2 d} { \expct{\hat{x}^2}_\zeta \over \expct{\period{}}_\zeta }
  \,=\,{1 \over 2 d } \, {1 \over \expct{\period{}}_\zeta}
  \sumprime \frac{(-1)^{k+1}
  (\hn_{p_1}+ \cdots+ \hn_{p_k})^2}
  {|\ExpaEig_{p_1}\cdots \ExpaEig_{p_k}|}
\, .
\label{(17)}
\eeq
where the sum is over all distinct non-repeating combination of prime
cycles.  The derivation is standard, still the formula is strange.
Diffusion is unbounded motion across an infinite lattice;
nevertheless, the reduction to the elementary cell enables us to
compute relevant quantities in the usual way, in terms of periodic
orbits.

A sleepy reader might protest that $x(\period{p})-x(0)$ is manifestly
equal to zero for a periodic orbit. That is correct; $\hn_p$ in the
above formula refers to a displacement $\hat{x}(\period{p})$ on the
{\em infinite} periodic lattice, while $p$ refers to closed orbit of
the dynamics $\flow{t}{x}$ reduced to the elementary cell, with $x_p$
a periodic point in the closed prime cycle $p$.

Even so, this is not an obvious formula. Globally periodic orbits have
$\hat{x}_p^2 =0$, and contribute only to the time normalization
$\expct{\period{}}_\zeta$. The mean square displacement
$\expct{\hat{x}^2}_\zeta$ gets contributions only from the periodic
runaway trajectories; they are closed in the elementary cell, but on
the periodic lattice each one grows like $\hat{x}(t)^2 =
(\hn_{p}/\period{p})^2 t^2 = v^2_p t^2$.  So the orbits that
contribute to the trace formulas and \Fd s exhibit either ballistic
transport or no transport at all: diffusion arises as a balance
between the two kinds of motion, weighted by the $1/|\ExpaEig_p|$
measure. If the system is not hyperbolic such weights may be
abnormally large, with $1/|\ExpaEig_p| \approx 1/\period{p}^\alpha$
rather than $1/|\ExpaEig_p| \approx e^{-\period{p}\Lyap}$, where
$\Lyap$ is the Lyapunov exponent, and they may lead to anomalous
diffusion - accelerated or slowed down depending on whether the
probabilities of the running or the standing orbits are enhanced.
% \toSect{intersec:andiff}

We illustrate the main idea, tracking of a globally diffusing
orbit by the associated confined orbit restricted to the
elementary cell, with the Lorentz gas.


\PublicPrivate{ }{% switch \PublicPrivate{

\section{Lorentz gas}
\label{s-lorentz}
%old \file{nordita.dk:home/predrag/book/b\_lorentz.tex     \hfill 26/3-95}
% in part based on
% Transport Properties of the Lorentz Gas in Terms of Periodic Orbits
% CVITANOVI\'C ECKMANN GASPARD
% 17 October 1993


We now return to the problem of computing the diffusion coefficient
for a ``finite horizon'' Lorentz gas on a hexagonal lattice introduced
in \refsect{s-DiffPerArr}.
%By finite horizon we mean any arrangement of scatterers which does
%not allow for infinitely long free flights of the moving particle.
%A random arrangement of scatterers always produces finite horizon,
%while for a regular lattice of scatterers different conditions on
%their spacing must be imposed, depending on the particular arrangement.

{Comment by Lamberto}

We are not going to describe in full detail all the calculations that
have performed for such a system, our main goal is to provide a
physically relevant example where the symbolic dynamics is enormously
complicated, and application of formulas like \refeq{(17)} is
problematic. The complexity of symbolic dynamics for this system has a
very clear and profound mathematical justification:

In practice we will point out how serious are the problems encountered
in applying formulas like \refeq{(17)} to this system.  From the point
of view of From a general point of view these difficulties are
expected, since the organization of cycles suffers from the complexity
of the symbolic dynamics for such systems.

Markov partitions for disperse billiards are made of an infinite
countable collection of pieces, very unevenly sized. The source of
troubles are the grazing singularities, corresponding to the orbits
that are tangent to a disk: a cone of trajectories close to the
tangent one is split into two pieces at the tangency point, and the
process repeats at every tangency. This mechanism causes the global
unstable and stable manifolds to consist of a countable number of
smooth components.  Thus the presence of singularities in a sense
weakens the strong chaotic properties induced by complete
hyperbolicity (due to the fact that any single collision leads to
defocusing, and that the time between two successive collisions if
bounded from above by a finite quantity).

Let us first define the parameters of the model: we set the disk
radius $R=1$ and denote by $w$ the disk to disk distance, so that the
centers of two neighboring disks are separated by a distance
$2+w$. For a hexagonal lattice we require $0\leq w \leq 4/\sqrt{3}-2$
in order that the finite horizon condition be satisfied.

\subsection{Lorentz gas: symbolic dynamics}

If we start a trajectory from any scatterer in the lattice, under the
condition of finite horizon, there are two possibilities for the next
hit: either it takes place on one of the six neighboring disks, or on
one of the six next-to-nearest disks. Hence we may introduce an
alphabet, by labeling these disks clockwise from $0$ to $11$: if we
assign the label $0$ to one of the closest obstacles, even labels will
then correspond to short flights, see \reffig{f_simb_lor}.
%as in \reffig{f_simb_lor}
%\PC{get Fig.9.3 ?}
%%%%%%%%%%%%%%%%%%%%%%%%%%%%%%%%%%%%%%%%%%%%%%%%%%%%%%%%%%%%%%%%%%
%\FIG{
%\includegraphics[width=0.60\textwidth]{lor_011}
%}{}{
%Symbolic dynamics for closely packed Lorentz gas.
%}{f_simb_lor}

Then, any trajectory (without grazing collisions) may be uniquely
identified by assigning one of these twelve symbols to each free
flight. The way this is done is to let the symbol for a free flight
depend on the vector separation between initial and final point of
such a segment of trajectory. Then, one should imagine that the center
of \reffig{f_simb_lor} is always translated to the scatterer where the
free flight starts, and that the symbol for the flight is the one of
the scatterer where it ends. So, in practice, for a spacing $w=0.3$,
which is just below the finite horizon condition, one has only four
types of cycles with two collisions, \ie, orbits which indefinitely
repeat the same symbol sequence of two symbols:
%\cite{MR94}:
the are of course short orbits between two neighboring scatterers,
like $(0~6)$ (next neighbors) or $(1~7)$ (next to nearest neighbors),
as well as $(1~5)$ and $(0~5)$ orbits. All these orbits belong to {\em
groups}, characterized by the same layout, and differing just in the
orientation: this is of course due to the symmetry properties of the
system: as the symmetry group is discrete there is a maximum number of
different orbits in each group (which is $24$ in the case under
consideration).  The fact that the different orbits of one kind are
oriented in different directions does not affect their length $\tau$
and their Lyapunov exponent $\lambda$, which are exactly the same for
all orbits within a group.  The number of distinct orbits within one
group, and the corresponding product $\tau \lambda$, which appears in
the definition of the Floquet multipliers are listed in
\reftab{t-diff-1}, for cycles of length 2 and 3.
%
%%%%% Table 1 %%%%%%%%%%%%%%%%%%%%%%%%
%\begin{table}
%{\small
%\begin{tabular}{|c|c|c|}
%\hline
%Orbit & Degeneracy & ~$\tau \lambda$ \\ \hline\hline
%$S$    & 3      & 1.51286399  \\
%$L$    & 3      & 3.51394510  \\
%$V$    & 6      & 4.65944481  \\
%$T$    & 12     & 3.15831208  \\ \hline\hline
%\end{tabular}
%\caption[]{\small
%Dynamical properties of length-2 cycles, at $w=0.3$.
%}
%\label{t-diff-1}
%}  %end of \small
%\end{table}
%%%%%%%%%%%%%%%%%%%%%%%%%%%%%%%%%%%%%%%%%%%%%%%%%%%%%%%%%%%%%%%%%%
%
%In \reftab{t-diff-2-3} we report the same quantities for cycles of length 3,
%along with one representative sysmbol string, for each orbit type.
%
%%%%% Table 2 %%%%%%%%%%%%%%%%%%%%%%%%
%\begin{table}
%{\small
%\begin{tabular}{|c|c|c|}
%\hline
%Orbit type & Degeneracy & ~$\tau \lambda$ \\ \hline\hline
%(0~4~8) & 4      & 3.27128792  \\
%(0~2~7) & 12     & 6.23520184  \\
%(0~2~6) & 24     & 6.31880379  \\
%(0~2~5) & 24     & 8.30080318  \\ \hline\hline
%\end{tabular}
%\caption[]{\small
%Dynamical properties of length-3 cycles, at $w=0.3$.
%}
%\label{t-diff-2}
%}  %end of \small
%\end{table}
%%%%%%%%%%%%%%%%%%%%%%%%%%%%%%%%%%%%%%%%%%%%%%%%%%%%%%%%%%%%%%%%%%
%
%%%%% Table 1 %%%%%%%%%%%%%%%%%%%%%%%%
\begin{table}
{\small
\begin{tabular}{|c|c|c|}
\hline
Orbit & Degeneracy & ~$\tau \lambda$ \\ \hline\hline
(0~6)    & 3      & 1.51286399  \\
(1~7)    & 3      & 3.51394510  \\
(1~5)    & 6      & 4.65944481  \\
(0~5)    & 12     & 3.15831208  \\
(0~4~8) & 4      & 3.27128792  \\
(0~2~7) & 12     & 6.23520184  \\
(0~2~6) & 24     & 6.31880379  \\
(0~2~5) & 24     & 8.30080318  \\ \hline\hline
\end{tabular}
\caption[]{\small
Dynamical properties of length-2 and 3 cycles, at $w=0.3$.
}
\label{t-diff-1}
}  %end of \small
\end{table}
%%%%%%%%%%%%%%%%%%%%%%%%%%%%%%%%%%%%%%%%%%%%%%%%%%%%%%%%%%%%%%%%%%
%
%Note that for the hexagonal Lorentz gas
%$24$ is the highest degeneracy any orbit can have.

It is clear from these considerations that not all possible symbol
strings actually give rise to cycles. This is because the symbolic
dynamics for the Lorentz gas is marred by a large amount of pruning.
For instance, there is an obvious pruning rule that the same symbol
cannot be repeated twice, as this would correspond to a trajectory
crossing a disk.  Many more pruning rules are present: for instance,
in the dense gas we consider here not only a letter cannot repeat, but
the next symbol has to vary at least by two units after a short
flight, and by three units after a long flight. The immense amount of
pruning present in the problem may be appreciated by looking at the
second column of \reftab{t-diff-3},
%{t-diff-4}
in which the number of cycles is reported as a function of cycle
length: order by order we get roughly an increase by $3$, instead of
the factor $11$ which would come from an unrestricted grammar (keeping
into account only the non--crossing condition, which effectively
reduces by one the possible symbols that can follow any given one).


\subsection{Diffusion}

We now apply the formula \refeq{(17)} for a particular value of the
spacing $w=0.3$.

Together with estimates of the diffusion coefficient we may also
provide estimates for the Lyapunov exponent (see
\refsect{s-Lyapunovs}).  Moreover, since no trajectory escapes, the
escape rate is zero, and the \Fd\ and the \dzeta\ must satisfy the
material flow conservation rule \refeq{prob-cons}: the results for all
these quantities are reported in \reftab{t-diff-3}, where cells have
been sampled with the variational method described in
\refrem{r-varPrinc}.
% In \reftab{t-diff-4}, we report the corresponding results as
% obtained with the cycle sorting method of \refrem{r-ergSys} for
% comparison.
%
%%%%% Table 3 %%%%%%%%%%%%%%%%%%%%%%%%
\begin{table}
\caption[]{\small
Elementary cell, $w$=0.3.
The cycles are sorted with the
method of \refrem{r-varPrinc}.
The last column is taken from \refref{MR94}.
}
{\small
\begin{tabular}{rrlll}
\hline
length & \# cycles & ~$\zeta$(0,0) & ~~$\Lyap$ & ~~D \\ \hline
1      & 0      &   -    &   -  &   - \\
2      & 24     & -0.31697 & 1.330 & 0.375 \\
3      & 64     & -0.54152 & 1.435 & 0.338 \\
4      & 156    & -0.09718 & 1.902 & 0.282 \\
5      & 492    &  0.02383 & 2.324 & 0.212 \\
6      & 1484   &  0.02812 & 1.931 & 0.129 \\
7      & 5244   &  0.02044 & 1.836 & 0.185 \\
8      & 19008  & -0.00036 & 1.754 & 0.256 \\ \hline
\multicolumn{4}{l}{\refref{MZ}, numerical experiment} & 0.25(1) \\
\multicolumn{4}{l}{\refref{MR94}, numerical experiment} & 0.2492(3)
\end{tabular}
}  %end of \small
\label{t-diff-3}
\end{table}
%%%%%%%%%%%%%%%%%%%%%%%%%%%%%%%%%%%%%%%%%%%%%%%%%%%%%%%%%%%%%%%%%%
%

{column 5 fixed by Lamberto}
%
%%%%% Table 4 %%%%%%%%%%%%%%%%%%%%%%%%
% \begin{table}
%{\small
%\begin{tabular}{|r|r|l|l|l|l|}
%\hline
%length & \# cycles & ~$\zeta$(0,0) & ~~$\Lyap$ & ~~D & ~~D${}_{trace}$ \\ \hline\hline
%1      & 0      &   -    &   -  &   - & - \\
%2      & 24     & -0.3170 & 1.330 & 0.3754 & 0.3754 \\
%3      & 64     & -0.5415 & 1.435 & 0.3388 & 0.1624 \\
%4      & 168    & -0.0976 & 1.902 & 0.2548 & 0.3800 \\
%5      & 516    &  0.0196 & 2.298 & 0.1772 & 0.1979 \\
%6      & 1262   &  0.0300 & 1.870 & 0.1842 & 0.3109 \\
%7      & 4200   &  0.0264 & 1.789 & 0.2255 & 0.2418 \\
%8      & 14652  & -0.0015 & 1.702 & 0.2684 & 0.2658 \\
%9      & 51252  & -0.0055 & 1.690 & 0.2560 & 0.2400 \\
%10     & 165150 & -0.0071 & 1.740 & 0.2384 & 0.2501 \\ \hline\hline
%\end{tabular}
%\caption[]{\small
%Elementary cell, $w$=0.3. The cycles are sorted with the
%ergodic method of \refrem{r-ergSys}.
%(or \refrem{r-varPrinc}??).
%The last column is taken from \refref{MR94}.
%(drop \reftab{t-diff-3}??)
%}
%\label{t-diff-4}
%}  %end of \small
%\end{table}
%%%%%%%%%%%%%%%%%%%%%%%%%%%%%%%%%%%%%%%%%%%%%%%%%%%%%%%%%%%%%%%%%%
%
%\LR{Table added by Lamberto}

As mentioned above a lot of symbol sequences are pruned when the
required finite horizon is achieved by making the spacing between
disks small enough. So very poor convergence of cycle expansions has
to be expected. With this in mind the numbers given in
\reftab{t-diff-3} are in a reasonable accord with the probability
conservation and offer a rather poor, but not unreasonable, estimate
of the Lyapunov exponent.
% (compare with the more accurate estimates of \reftab{t-diff-5a}).
Nevertheless, the estimates of the diffusion constant up to the number
of cycles employed so far appear to converge very slowly. They seem to
be more sensitive to the bad shadowing than the Lyapunov exponent and
the probability conservation. Of course, one should remember that the
diffusion coefficient is a higher order moment of the generating
function, hence it should be expected to be harder to get.
%
% \subsection{Dilute Lorentz gas}
%
% In order to test the diffusion formula under less trying conditions,
% we eliminate pruning by making the spacing between disks larger, and
% imposing the finite horizon by fiat; in this section we consider the
% measure zero subset of those orbits which after each bounce travel
% only to one of the nearest or next nearest disks. This Cantor set is
% a repeller, and the material flow is not conserved. Estimates of the
% Lyapunov exponent and the diffusion constant for trajectories
% restricted to this set are given in \reftab{t-diff-2a}.
%
%%%%% Table 2 %%%%%%%%%%%%%%%%%%%%%%%%
% \begin{table}
%{\small
%\begin{tabular}{|r|r|l|l|}
%\hline
%length & \# cycles & ~~$\Lyap$ & ~~D \\ \hline\hline
%1      & 0      &   -    &    - \\
%2 &     54      & 0.5528 &  1.6716\\
%3 &    440      & 0.5588 &  1.7006\\
%4 &   3234      & 0.5604 &  1.7054\\
%5 &  27856      & 0.5605 &  1.7049\\ \hline
%\end{tabular}
%\caption[]{\small
%Elementary cell, $w$=2.0, imposed finite horizon.
%}
%\label{t-diff-2a}
%}  %end of \small
%\end{table}
%%%%%%%%%%%%%%%%%%%%%%%%%%%%%%%%%%%%%%%%%%%%%%%%%%%%%%%%%%%%%%%%%%
%
%PG for the closed infinite horizon billiard when w=2, I have that the
% Lyapunov exponent = 0.4893 +/- 0.0003 which is lower than on the
% fractal since orbits which are tangent to the disks are absent in
% the fractal but have lower Lyapunov exponents.
% \caption{Elementary cell, $w$=2.0, finite horizon} \label{TDIF}
% \end{table}
% \eject

% As the set is not the full set of orbits contributing to diffusion,
% there are no results obtained by other means that these numbers
% could be compared with.  Note that as expected the Lyapunov exponent
% is now smaller because the average time of straight motion between
% bounces is larger due to the wider spacing.  Encouragingly, the
% diffusion constant exhibits reasonable convergence, supporting the
% claim that the cycle expansions are in principle convergent, but for
% high density of scatterers the convergence is adversely affected by
% the strong pruning of the allowed orbits.
%
% \subsection{Trace formulas, numerical simulations}
%
% \%authorLR
%
% \noindent
A way to circumvent the poorness of shadowing, due to severe and badly
controllable pruning, is just going back to trace formulas
% $D$ values listed in the last column in \reftab{t-diff-3} are
% calculated from the trace formula
\refeq{averf} \beq D_{tr}\,=\frac{1}{4} \lim_{n \to
  \infty}\frac{\sum_{x\in Fix(n)}\, \hn_x^2 /|\ExpaEig_x|}{\sum_{x\in
    Fix(n)}\, T_x/|\ExpaEig_x|}
\label{DL-trace}
\eeq
% of the same form we used to derive exact results for cat maps
% \refeq{Dcat}.  A very long trajectory of the order of $10^{10}$
% collisions was used to find cycles as described in the previous
% remark.  This way, they were able to carry on their calculation up
% to cycle length $10$, missing a number of cycles, for the reasons
% explained in the previous remark.
For cycle length $9$ \refref{MR94} reports $D_{tr}\simeq 0.240$, and
for cycle length $10$ $D_{tr}\simeq 0.2501$, which is not too far from
the result of direct simulations (see \reffig{f-diff-dpolt}).  The
apparent better performance of trace formulas over cycle expansions in
this context is rather unexpected, and no strong theoretical clue is
offered to justify this observation, but the very simple fact that
trace formulas automatically include probability conservation, as the
average is always computed by using a normalized measure.
%
% and only a partial explanation has been provided so far.  (see
% following Remark) \FIG{} %\input{dpolt}}
{}{ Figure was here Finite order estimates for the diffusion
  coefficient from cycle expansions and trace formulas.  }
% {f-diff-dpolt}
%
% The numerical experiment values reported at the bottom of
% \reftab{t-diff-3} come from the simulations performed in \refref{MZ}
% and in \refref{MR94}. In particular, in \refref{MR94} there is also
% a check that both Einstein relation and Green-Kubo formulas lead to
% the same estimate.  A combination of the two methods produced the
% observed accuracy of the result.
%
%
% \subsection{On the effect of pruning on cycle expansions}
%
% \authorLR
%
%\noindent
%The unexpected better performance of trace formulas over cycle
% expansions, in the study of the Lorentz gas is even more pronounced
% at smaller spacings $w$. This suggests that pruning has something to
% do with that, and that there may be a way of limiting its disastrous
% effects.
%
% First observe the following. In order to have good convergence in
% cycle expansions it is necessary that the curvature correction terms
% like $(t_{ab} - t_a t_b)$, where $ab$ is a generic symbol string,
% ``shadowed'' by the product of strings $a$ and $b$, rapidly tend to
% zero. However, because of pruning in the Lorentz gas, there is a
% large mismatch in the number of prime and shadowing strings.  For
% instance, at $w$=0.3 there are $24$ cycles of length $2$, which
% means that $276$ shadowing approximants of length $4$ can be formed
% by taking their products.  However, there are only $168$ prime
% cycles of length $4$.  At length $5$, there are $1536$ product
% strings to match with only $516$ prime cycles. Similarly, at length
% $6$ there are $6480$ product strings contributing a positive amount
% to the cycle expansion, $2024$ product strings contributing a
% negative amount, and $1262$ prime strings contributing a negative
% amount. Moreover, there are many prime strings which do not even
% have a shadowing approximant. These problems do not ease at higher
% periods, which makes the convergence of the cycle expansions rather
% slow.
%
% On the other hand, trace formulas do not arrange the different terms
% into curvature correction blocks, and they are explicitly normalized
% at all periods. This produces two effects. First, taking longer and
% longer cycles simply makes each single cycle look closer and closer
% to an ergodic trajectory itself (from a numerical point of
% view). Secondly, if the cycles are selected with the method of
% \refrem{r-ergSys}, their contributions are naturally distributed in
% a range wich contains the correct average, and weighing them with a
% probability measure can only produce a number in the same range.
%
% This somehow gives an idea of why trace formulas performed better
% than cycle expansions in Lorentz gas calculations. At the same time,
% these facts
This fact suggests that the performance of cycle expansions for
systems with strong pruning, might be improved by imposing by hand the
condition of probability conservation at all orders of approximation.

PC{\reffig{f-diff-dpolt} tex file is using too much memory, cannot
  latex book}

PC{add to remarks, index: far from equilibrium: NESS = nonequilibrium
  steady state, perhaps with some snide Loch Ness monster quote}

% \subsection{Reduction to the fundamental domain}
%
%
%
% \subsection{Lattice symmetry and diffusion}
% \label{s-Latt-sym}
% \hfill (P. Gaspard)
%
%\noindent
%In order to perform the full reduction for diffusion one has to
% express the \dzeta\ \refeq{zeta-diff} in terms of the prime cycles
% of the fundamental domain $\tilde \pS$ of the lattice (see
% \reffig{fig-lor-1}) rather than \PC{need here fig-lor-2 - where did
% it go?}  those of the elementary (Wigner-Seitz) cell $\pS$.  This
% problem presents the following difficulty. The stumbling block
% appears to be the breaking of the rotational symmetry by the
% auxiliary vector $\beta$, or, in other words, the non-commutativity
% of translations and rotations.  More precisely, the global distance
% $ \hat\phi^{r \period{p}} (\tx{\tpk}) - \tx{\tpk} $, $\tx{\tpk} \in
% {\tilde p}$, $ \hn_{r |\tilde p|}(\tx{\tpk}) $ depends on the
% starting periodic point if ${\tilde p}$ is only a segment of the
% global cycle $p$. An example is the diamond-shaped cycle of
% \reffig{fig-lor-3}; the problem is that the ${\tilde p}$ segment of
% the global trajectory is not a translation in $\hM$.  depending on
% whether one starts at $\tx_1$ or $\tx_2$, the global distance
% covered in time $\period{p}$ is either the short or the long
% diagonal.  This difficulty can be handled in the following way.
%
%
%%%%%%%%%%%%%%%%%%%%%%%%%%%%%%%%%%%%%%%%%%%%%%%%%%%%%%%%%%%%%%%%%%
% \figurewithtex ps/fig3.ps ps/fig3.tex 12 16 \FIG{
% \includegraphics[width=0.80\textwidth]{fig_lor_3.ps}
% }{}{ A fundamental domain 2-cycle ${\tilde p}$ which covers in one
%   return time only 1/2 of the corresponding global cycle $\hat p$.
%   $\hat f(x)$ is the collision-to-collision mapping induced by the
%   flow.  To make the figure more readable, the disks have radii
%   smaller than those needed for the finite horizon condition.  Note
%   that each fundamental domain cycle corresponds to 12 distinct
%   global cycles.  (PC suggestion; four disks in $\hM$ with fund
%   domains indicated lightly, $\tx_{1}$, $\hf^{\ttime}(\tx_{1})$,
%   $\tx_{2}$, $\hf^{\ttime}(\tx_{2})$ marked, so reader can see that
%   the corresponding global distances travelled correspond to two
%   (unequal length) diagonals) (Courtesy of J.-P. Eckmann)
% }{fig-lor-3}
%%%%%%%%%%%%%%%%%%%%%%%%%%%%%%%%%%%%%%%%%%%%%%%%%%%%%%%%%%%%%%%%%%
%
%   We can introduce the flow $\tilde\phi^t$ on the fundamental domain
%   $\tilde \pS$ using the full point group of the lattice.  Moreover,
%   the density $\rho(x)$ on which the Perron-Frobenius operator
%   \evOper\ $\Lop^t$ acts can always be decomposed using the
%   projectors onto the irreducible representations of the full point
%   group.  For an arbitrary wavenumber $k=-i \beta$, the
%   Perron-Frobenius operator \evOper\ will mix the different
%   components of the density $\rho(x)$, which we can express by
%   \PC{harmonize ${\bf R}$ with symmetry chapter} \beq \Lop^t(y,x) \
%   = \ {\bf R}(x;k,t) \ \delta \lbrack y \ - \ \tilde \phi^t (x)
%   \rbrack ~, \ee{dif-lor-3.2} where ${\bf R}(x;k,t)$ is a matrix
%   ruling the dynamics of the different components of the density.
%   With \refeq{dif-lor-3.2}, the Perron-Frobenius operator \evOper\
%   is reduced to the flow in the fundamental domain $\tilde \pS$.
%   The \dzeta\ can then be written as a product over the prime cycles
%   $\tilde p$ of the fundamental domain \beq Z(k,s) \ = \
%   \prod_{\tilde p \in \tilde{\cal P}} \ \exp\Biggl\lbrace \ - \
%   \sum_{r=1}^{\infty} \ {1 \over r} \ {\rm tr} \lbrack \tilde {\bf
%   R}_{\tilde p}(k)^r\rbrack \ {{\exp ( - s \ttime r)} \over {\vert
%   \oneMinJ{r} \vert}}\Biggr\rbrace~ , \ee{dif-lor-3.3} where
%   $\tilde{\bf R}_{\tilde p}(k)$ is the matrix ${\bf R}(x;k,t)$
%   associated with the prime cycle $\tilde p$.  It is only when the
%   little group of the wavenumber $k$ has nontrivial irreducible
%   representations that the matrices $\tilde{\bf R}_{\tilde p}(k)^r$
%   split into block-diagonal submatrices which can be assigned to
%   each irreducible representation so that the \dzeta\ factorizes as
%   explained in \refchap{c-symm}.  \PC{elaborate for
%   \refeq{dif-lor-3.2}}
%
%   The search for a cycle with a given symbol sequence may be carried
%   out in as described in \refsect{s_orbit_min}
%   \refsect{s_long_time}.

}% end \PublicPrivate{


Resume


\noindent
With initial data accuracy $\delta x = |\delta {\bf x}(0)| $ and
system size $L$, a trajectory is predictable only to the finite
Lyapunov time $ \LyapTime \approx {\Lyap}^{-1}\ln {|L / \delta x|} \,.
$ %ee{LyapTime}
Beyond that, chaos rules. We have discussed the implications in
\refsect{s:ChaosGood}: chaos is good news for prediction of long term
observables such as transport in statistical mechanics.
\index{Lyapunov!time}

PC{Boltzmann refs.}


The classical Boltzmann equation for evolution of 1-particle density
is based on {\em stosszahlansatz}, neglect of particle correlations
prior to, or after a 2-particle collision.  \index{stosszahlansatz} It
is a very good approximate description of dilute gas dynamics, but a
difficult starting point for inclusion of systematic corrections.
\index{Boltzmann!equation} In the theory developed here, no
correlations are neglected - they are all included in the \cycForm\
such as the cycle expansion for the diffusion constant
\[
D \,=\,{1 \over 2 d } \, {1 \over \expct{\period{}}_\zeta} \sumprime
(-1)^{k+1} \frac{ (\hn_{p_1} + \cdots + \hn_{p_k})^2 }{
  |\ExpaEig_{p_1} \cdots\ExpaEig_{p_k}| } \, .
\]
Such formulas are {\em exact}; the issue in their applications is what
are the most effective schemes of estimating the infinite cycle sums
required for their evaluation. Unlike most statistical mechanics, here
there are no phenomenological macroscopic parameters; quantities such
as transport coefficients are calculable to any desired accuracy from
the microscopic dynamics.

For systems of a few degrees of freedom these results are on rigorous
footing, but there are indications that they capture the essential
dynamics of systems of many degrees of freedom as well.

Though superficially indistinguishable from the probabilistic random
walk diffusion, deterministic diffusion is quite recognizable, at
least in low dimensional settings, through fractal dependence of the
diffusion constant on the system parameters, and through non-Gaussion
relaxation to equilibrium (non-vanishing Burnett coefficients).
\index{Burnett coefficients}

That Smale's ``structural stability" conjecture turned out to be wrong
is not a bane of chaotic dynamics - it is actually a virtue, perhaps
the most dramatic experimentally measurable prediction of chaotic
dynamics. As long as microscopic periodicity is exact, the prediction
is counterintuitive for a physicist - transport coefficients are {\em
  not} smooth functions of system parameters, rather they are
non-monotonic, {\em nowhere differentiable} functions.
\index{structural stability}

Actual evaluation of transport coefficients is a test of the
techniques developed above in physical settings. In cases of severe
pruning the trace formulas and ergodic sampling of dominant cycles
might be more effective strategy than the cycle expansions of \dzeta s
and systematic enumeration of all cycles.


{Lorentz gas.}{ The original pinball model proposed by
  Lorentz\rf{lore} consisted of randomly, rather than regularly placed
  scatterers.  \index{Lorentz gas} } %end \remark{Original model}{
%


{Who's dunnit?}{ Cycle expansions for the diffusion constant of a
  particle moving in a periodic array have been introduced
  % in May-June 1991
  by R.~Artuso\rf{art91} (exact \dzeta\ for $1$\dmn\ chains of maps
  \refeq{(17)}), by W.N. Vance\rf{vance}%
  \PublicPrivate{,}{% switch \PublicPrivate{
    (the trace formula \refeq{DL-trace} for the Lorentz gas),
  }% end \PublicPrivate{
  and by P. Cvitanovi\'c, J.-P. Eckmann, and
  P. Gaspard\rf{LorentzDiff} (the \dzeta\ cycle expansion \refeq{(17)}
  applied to the Lorentz gas).
  % Vance refers to the Artuso article and Gaspard-Baras\rf{GB92}, so
  % perhaps his only independent contribution was to apply the method
  % to the Lorentz gas.
} %end \remark{Who's dunne it?}{

PC{add Schreiber ref.; write about Gallavotti-Cohen}

{Lack of structural stability for D.}{ Expressions like \refeq{D-KD}
  may lead to an expectation that the diffusion coefficient (and thus
  transport properties) are smooth functions of the chaoticity of the
  system (parameterized, for example, by the Lyapunov exponent $\Lyap
  = \ln \ExpaEig$). This turns out not to be true: $D$ as a function
  of $\ExpaEig$ is a fractal, nowhere differentiable curve shown in
  \reffig{f-Klage}. The dependence of $D$ on the map parameter
  $\ExpaEig$ is rather unexpected - even though for larger $\ExpaEig$
  more points are mapped outside the unit cell in one iteration, the
  diffusion constant does not necessarily grow. We refer the reader to
  \refrefs{detdiff1,detdiff2} for early work on the deterministic
  diffusion induced by 1\dmn\ maps. The sawtooth map \refeq{KD-map}
  was introduced by Grossmann and Fujisaka\rf{gro} who derived the
  integer slope formulas \refeq{D-KD} for the diffusion constant. The
  sawtooth map is also discussed in \refrefs{GB92}. The fractal
  dependence of diffusion constant on the map parameter is discussed
  in \refrefs{RKdiss,KlaDor95,KlaDor97}. \refSect{s:ChaosGood} gives a
  brief summary of the experimental implications; for the the current
  state of the art of fractal transport coefficients consult the first
  part of Klage's monograph\rf{RKbook07}. Would be nice if someone
  would eventually check these predictions in experiments...
  Statistical mechanicians tend to believe that such complicated
  behavior is not to be expected in systems with very many degrees of
  freedom, as the addition to a large integer dimension of a number
  smaller than $1$ should be as unnoticeable as a microscopic
  perturbation of a macroscopic quantity. No fractal-like behavior of
  the conductivity for the Lorentz gas has been detected so
  far\rf{LNRM94}.  \index{Lorentz gas} \hfill (P. Cvitanovi\'c and
  L. Rondoni) } %end \remark{Structural stability for D}

{Symmetry factorization in one dimension.}{ In the $\beta=0$ limit the
  dynamics \refeq{sym2} is symmetric under $x \rightarrow -x$, and the
  zeta functions factorize into products of zeta functions for the
  symmetric and antisymmetric subspaces, as described in
  \refsect{Reflecti}: \bea {1 \over \zeta(0,z)} &=& {1 \over
    \zeta_{s}(0,z)}\,{1 \over \zeta_{a}(0,z)} \continue {\pde \over
    \pde z} { 1 \over \zeta} &=& { 1 \over \zeta_s} {\pde \over \pde
    z} { 1 \over \zeta_a} + { 1 \over \zeta_a} {\pde \over \pde z} { 1
    \over \zeta_s} \,.
  \label{diffSymmetr}
  \eea The leading (material flow conserving) eigenvalue $z=1$ belongs
  to the symmetric subspace \( {1 / \zeta_s(0,1)}=0 \), so the
  derivatives \refeq{MeanTime} also depend only on the symmetric
  subspace: \bea \expct{n}_\zeta &=& \left. z{\pde \over \pde z} {1
      \over \zeta(0,z)}\right|_{z=1} \continue &=& {1 \over
    \zeta_{a}(0,z)}\, z{\pde \over \pde z} \left.{1 \over
      \zeta_{s}(0,z)}\right|_{z=1} \, .
  \label{symm_der}
  \eea Implementing the symmetry factorization is convenient, but not
  essential, at this level of computation.
}  %end \remark{Symmetry factorization in one dimension}

{Lorentz gas in the fundamental domain.}{ The vector valued nature of
  the generating function \refeq{lor-diff-1} in the case under
  consideration makes it difficult to perform a calculation of the
  diffusion constant within the fundamental domain.  Yet we point out
  that, at least as regards scalar quantities, the full reduction to
  $\tilde{\pS}$ leads to better estimates.  A proper symbolic dynamics
  in the fundamental domain has been introduced in \refref{freddy}%
  \PublicPrivate{.}{% switch \PublicPrivate{
    , numerical estimates for scalar quantities are reported in
    \reftab{t-diff-5a}, taken from \refref{CGS}.
  }% end \PublicPrivate{
  % The fundamental domain symbolic dynamics used here due to
  % F.~Christiansen~\cite{freddy}, is given in figs.~3?, 4? and
  % \reftab{t-diff-3a}.  \RA{insert figs. 3 and 4 from \refref{CGS}}
  % Now the symbols indicate relative direction changes instead of the
  % absolute directions.
%
  %%%%% Table 3 %%%%%%%%%%%%%%%%%%%%%%%%
  % \begin{table}
  %   {\small
  %   \begin{tabular}{|c||r|r||c|c|}
  %     \hline symbol & \multicolumn{2}{|c||}{amount of change} &
  %     \multicolumn{2}{|c|}{direction of change} \\
  %     & last long & last short & next the same & next other way \\
  %     \hline
  %     a      &    1      &     2      &     x      &       \\
  %     b      &    3      &     4      &     x      &       \\
  %     c      &    5      &     6      &     x      &       \\
  %     d      &    5      &     4      &            &   x    \\
  %     e      &    3      &     2      &            &    x   \\
  %     f & 1 & - & & x \\ \hline
  %     A      &    2      &     1      &     x      &       \\
  %     B      &    4      &     3      &     x      &       \\
  %     C      &    6      &     5      &     x      &       \\
  %     D      &    4      &     5      &            &    x   \\
  %     E      &    2      &     3      &            &    x   \\
  %     F & - & 1 & & x \\ \hline
  %   \end{tabular}
  %   \caption[]{\small Symbols in the fundamental domain.  }
  %   \label{t-diff-3a}
  % } %end of \small
  % \end{table}
%%%%%%%%%%%%%%%%%%%%%%%%%%%%%%%%%%%%%%%%%%%%%%%%%%%%%%%%%%%%%%%%%%
%
  % \caption{Symbols in the fundamental domain} \label{TSYM}
  % \end{table}
  % \eject The reflection symmetry of the problem also should be taken
  % into account by the symbols.  The right and left turns are not
  % distinguished - instead, one reads off a symbol whether the next
  % turn has to be taken in the same or in the opposite sense.
%
  % Lower case letters denote short flights between closeby disks,
  % upper case denotes the long flights to the next nearest disks.
  % Each symbol corresponds to a given relative change in
  % direction. The exact amount of change depends on whether the last
  % flight was long (odd label in the notation of the last section) or
  % short (even label).  Table~\ref{TSYM} and figures~\ref{FUN1}
  % and~\ref{FUN2} give the definition of the symbols.
%
  % Christiansen~\cite{freddy} had originally proposed to denote
  % symbols `f' and `F' by the same letter, thus reducing the size of
  % the alphabet.  We use two letters `f' and `F' together with the
  % pruning rule that `f' can only follow an uppercase letter (\ie, a
  % long segment) and `F' only a lowercase letter.  The symbols given
  % to an orbit by this scheme are invariant under all spatial
  % symmetries of the system but not under the time reversal.  The
  % fact is that time reversal symmetry is included in the spatial
  % symmetries, at least for running orbits, as a consequence of the
  % absence of external fields, and of the geometry of the model.
%
  % \LR{PLEASE, make sure that this comment makes sense}
%
  % Of the possible twelve symbols, `A', `B', `f' and `F' are pruned
  % as soon as the horizon gets finite.
  % \\
  % (PC: finite or densely packed???)\\
  % TS: finite !  Among the remaining symbols there is still strong
  % pruning, reflected in the fact that for $w = 0.3$ the number of
  % cycles of symbol length $n$ does not grow like $8^n$ but roughly
  % as $3^n$. Figure~5? shows \RA{insert fig. 5 from \refref{CGS}} all
  % the fundamental domain fixed points which are not pruned at
  % $w=0.3$ together with an example of a pruned fixed point.
%
%
  % Table~4 gives some impression of the pruning involved.  In order
  % that longer orbits be shadowed by shorter ones, for every
  % combination of two symbols a two-cycle, and the fixed points
  % corresponding to each of the symbols should exist.  Two-cycles
  % such as `ac', `aC', \ldots are missing while corresponding
  % ``shadowing'' pseudo-cycles `a~c', `a~C' exist, and conversely,
  % two-cycles occur where one of the symbols has no corresponding
  % fixed point (e.g. the symbol `d' in `ad'). So the shadowing is
  % largely disfunctional, at least as long as finite approximate
  % Markov partitions are not developed.  Nevertheless the finite
  % order estimates for the Lyapunov exponent and material flow
  % conservation, reported in \reftab{t-diff-5a}, are rather good, and
  % noticeably better than the corresponding calculation using
  % elementary cell dynamics (\reftab{t-diff-1}).  \PC{recheck if
  % indeed \reftab{t-diff-1}?}.
%
  %%%%% Table 5 %%%%%%%%%%%%%%%%%%%%%%%%
\begin{table}
{\small
\begin{tabular}{|r|r|r|r|}
\hline
length & \# cycles & $\zeta$(0,0) & $\Lyap$ \\ \hline\hline
1      &    5     &  -1.216975 &     -    \\
2      &   10     &  -0.024823 & 1.745407 \\
3      &   32     &  -0.021694 & 1.719617 \\
4      &  104     &   0.000329 & 1.743494 \\
5      &  351     &   0.002527 & 1.760581 \\
6      & 1243     &   0.000034 & 1.756546 \\ \hline
\end{tabular}
\caption[]{\small
Fundamental domain, w=0.3 .
}
\label{t-diff-5a}
}  %end of \small
\end{table}
%%%%%%%%%%%%%%%%%%%%%%%%%%%%%%%%%%%%%%%%%%%%%%%%%%%%%%%%%%%%%%%%%%
%
% \caption{Fundamental domain, w=0.3} \label{CONV}
% \end{table}
% \eject Both in the elementary cell and in the fundamental domain
% convergence is dramatically better with more separated disks because
% there is no more pruning.  Unfortunately, then the dynamics is
% artificially reduced to the cantor set of those orbits which travel
% only over the finite distance imposed by the possible symbols. So
% any diffusion constant cannot be compared to numerical values.
% \PC{remember the table correction by factor 2!}
%
% \begin{table}
%\caption{Elementary cell, $w$=0.3} \label{TCELL}
%\end{table}
%PG I have the following numerical estimate for the Lyapunov exponent
% when w=0.3, Lyap = 1.760 +/- 0.002 which supports the result of this
% table.  \eject



% \begin{table}
%\begin{minipage}{5cm}
%  {\sl Row/column 1 contains a cross when a fixed point of this
%  symbol exists.  Existing two-cycles are marked by crosses in the
%  corresponding places.}
%\end{minipage}\hspace{1cm}
%\begin{minipage}{8cm}
%  \begin{quote} \begin{center}
%      \begin{tabular}{|c|c||c|c|c|c|c|c|c|c|}
% \hline
%   &   & a & b & c & d & e & C & D & E \\ \hline
%   & 1 & x & x & x &   &   & x & x &   \\ \hline\hline
% a & x & - & x &   & x &   &   &   &   \\
% b & x & x & - & x & x & x &   &   & x \\
% c & x &   & x & - & x & x &   & x &   \\
% d &   & x & x & x & - &   &   & x &   \\
% e &   &   & x & x &   & - &   &   &   \\
% C & x &   &   &   &   &   & - &   &   \\
% D & x &   &   & x & x &   &   & - &   \\
% E &   &   & x &   &   &   &   &   & - \\ \hline
% \end{tabular}\\[10pt]
% \end{minipage}
% \caption{ (no) shadowing, w=0.3}\label{NO}
% {Table 4: (no) shadowing, w=0.3} {Table 4: Problematic shadowing,
% w=0.3}
% \end{center}\end{quote}
% % \end{table}
% \eject
%
In order to perform the full reduction for diffusion one should
express the \dzeta\ \refeq{zeta-diff} in terms of the prime cycles of
the fundamental domain $\tilde \pS$ of the lattice (see
\reffig{fig-lor-1}) rather than
%\PC{need here fig-lor-2 - where did it go?}
those of the elementary (Wigner-Seitz) cell $\pS$.  This problem is
complicated by
%presents the following
%difficulty. The stumbling block appears to be
the breaking of the rotational symmetry by the auxiliary vector
$\beta$, or, in other words, the non-commutativity of translations and
rotations: see
%f.ex.
\refref{CEG}.
%for a discussion of the problem.
\label{r:LorentzFundDomain}
}%end \remark{Lorentz gas in the fundamental domain}

{Molecular chaos.}{ Read Gilbert and Lefevere\rf{GilbLef08}, ``Heat
  conductivity from molecular chaos hypothesis in locally confined
  billiard systems,'' } % end \remark{Molecular chaos.}{

Read ``From Deterministic Chaos to Deterministic Diffusion'' by
R. Klages, \arXiv{0804.3068}: `` A set of easy-to-read lecture notes
for a short first-year Ph.D.  student course. The notes cover five
hours of lectures and do not require any prior knowledge on dynamical
systems. The first part introduces to deterministic chaos in
one-dimensional maps in form of Lyapunov exponents and the metric
entropy. The second part first outlines the concept of deterministic
diffusion. Then the escape rate formalism for deterministic diffusion,
which expresses the diffusion coefficient in terms of the above two
chaos quantities, is worked out for a simple map. The notes conclude
with a very brief sketch of anomalous diffusion.

{for `fundamental domain' in hyperbolic geometry, see for example
  \HREF{http://www.math.ou.edu/~kmartin/mfs/ch3.pdf}{these notes} by
  \HREF{http://www.math.ou.edu/~kmartin}{Kimball Martin}.  }
