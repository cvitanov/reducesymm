% diffuse/intro.tex          pdflatex ZhCvGo15

% Tingnan                                2nov2015
% Predrag initial draft                 21nov2014
%         extracted from ChaosBook.org
% \Chapter{diffusion}{2apr2014}{Deterministic diffusion}

% \section{Introduction}

The advances in the theory of dynamical systems have brought a new life
to Boltzmann's mechanical formulation of statistical mechanics,
especially for systems near or far from equilibrium, and yielded new sets
of microscopic dynamics formulas for macroscopic observables such as the
transport coefficients. Sinai,
Ruelle and Bowen (SRB) have generalized Boltzmann's notion of ergodicity
for a constant energy surface for a Hamiltonian system in equilibrium to
dissipative systems in{nonequilibrium} stationary states. In this more
general setting the attractor plays the role of a constant energy
surface, and the SRB measure is a generalization of the Liouville
measure. Such measures are purely microscopic and indifferent to whether
the system is at equilibrium, close to equilibrium or far from it. ``Far
for equilibrium'' in this context refers to systems with large deviations
from Maxwell's equilibrium velocity distribution. Furthermore, the theory
of dynamical systems has yielded new sets of microscopic dynamics
formulas for macroscopic observables such as diffusion constants, to
which we turn now.
%
%%%%%%%%%%%%%%%%%%%%%%%%%%%%%%%%%%%%%%%%%%%%%%%%%%%%%%%%%%%%%%%%%%
%\SFIG{fig_lor_4}
%{}{
%Deterministic diffusion in a
%finite horizon periodic Lorentz gas.
%\hfill (T. Schreiber)
%}{fig-lor-4}
%%%%%%%%%%%%%%%%%%%%%%%%%%%%%%%%%%%%%%%%%%%%%%%%%%%%%%%%%%%%%%%%%%
%


Chaotic motions exist in many field of physics systems, blah. Thre are
physical problems such as beam defocusing in particle accelerators or
chaotic behavior of passive tracers in $2$\dmn\ rotating flows which can
be described as deterministic diffusion in periodic arrays. In the field
of animal/robotic locomotion, we will show that a macroscopic ``diffusion
view'' also applies.

Chaotic motions exist in many systems. There are physical problems such
as beam defocusing in particle accelerators \TZ{2015-11-02}{reference?}
or chaotic behavior of passive tracers in $2$\dmn\ rotating
flows\rf{solomon1994chaotic} which can be described as deterministic
diffusion in periodic arrays. In the macroscopic world, there are recent
studies shown that robotic locomotion in heterogeneous granular
environment also demonstrates scattering-diffusive pattern.

%In biological field,  many important dynamical processes (often at
%cellular level) are described in  terms of diffusion coefficients. Such
%examples include the transport of ions  across the cell
%membranes\rf{stein2012transport} and the movement of  microorganism(e.g.
%bacterials) through natural  ecosystems\rf{koch1990diffusion}. In this
%paper we will discuss the transport  property of more "macroscopic"
%systems (such as moving  robots\rf{saranli2001rhex}) where a ``diffusive
%description'' also applies.

%Lately, there has been an increased focus on robot locomotion in complex
%environments (check Science and ROPP reference, the systematic study of
%interactions between environment and locomotion, which we now
%call``robophysics''). Many of those studies use substrates that are
%spatially homogeneous and we have a good
%understanding\rf{li2009sensitive,li2013terradynamics}. However, little is
%known for locomotion in heterogeneous environment. There are some limited
%experimental/theoretical studies for relatively simple settings (e.g.
%slopes,ref). In this paper we intend to approach the longterm transport
%properties of locomotion in a more complex environment, i.e. in a field
%of scatterers of which the scales are comparable to the locomotor.

In \refref{LorentzDiff} some effort was made to derive a diffusion
formula for the fully symmetry-reduced dynamics, involving only
quantities computed within the fundamental domain. The fact that lattice
translations do not commute with the symmetry group within the elementary
cell makes this apparently a difficult task.

As a gedanken experiment, suppose a passively controlled robot is moving
in a boulder field at constant speed. The diffusion coefficient, which
describes roughly how much area the robot explored in a unit time, is the
key quantity we would like to investigate. We place the boulder in a
regular, periodic array and assume that we are in the heavy boulder limit
such that after each collision event, only the robot is deflected and
boulders remain immobilized. With the presumptions we effectively created
a periodic Lorentz gas model\rf{Dettm14} for locomotion in a boulder
field.

To investigate the transport property of such systems, we apply cycle
expansions\rf{DasBuch} to the analysis of {\em diffusion coefficient}.
The resulting formulas are exact; no probabilistic assumptions are made,
and all correlations are taken into account by the  inclusion of cycles
of all periods. While existing cycle expansion theory yields the correct
result by tiling the {\statesp} into elementary cells, the convergence
rate is slow because of bad shadowing and poor choice of symbolic
grammar\rf{CGS92}. In this paper we propose a novel approach that
significantly improves the efficiency of cycle expansion formula, by
factorizing the non-commuting rotational and translational symmetry and
using periodic orbits in the fundamental domain.

% The infinite extent systems for which the  periodic orbit theory yields
% formulas for diffusion and other transport  coefficients are spatially
% periodic, the global {\statesp} being tiled with  copies of a elementary
% cell.

In \refsect{s-DiffPerArr} we briefly review the formulas for diffusion
coefficients in the $2$\dmn\ periodic Lorentz gas, using dynamics
restricted in elementary cell. In\refsect{s-SymmetryReduction} we
factorize the rotational symmetry and derive the diffusion coefficient
using fundamental domain cycles. Because we will work with different
kinds of \statesp, through out the text we will repeatedly using tildes
($\tilde{\quad}$), nothings and hats ($\hat{\quad}$) atop symbols to
signify the dynamical quantities in the fundamental domain, elementary
cell and full state space, respectively.


=========== TO REUSE ========

    \PC{edits based Cvitanovi\'c,  Eckmann,and Gaspard\rf{LorentzDiff}}

The diffusive properties of a ``Lorentz gas'' [L]
have been studied extensively in the literature,
see \refrefs{Sinai70,BunSin80,Bunimovich85,GasNic90,MacZwa83}.
In \refrefs{art91,LorentzDiff,CGS92,Artuso94,CBdiffusion}  an explicit
connection between the global diffusion and the dynamics restricted to
an elementary cell.
Our method applies to any  hyperbolic dynamical system that is
a periodic tiling $\hM=\bigcup_{ \hn \in T} M_{
\hn}$
of the dynamical phase space $\hM$ by {\sl translates}
$M_{\hn}$
of an {\sl elementary cell} $M$, with $T$ the Abelian group of lattice
translations.
Furthermore, each elementary cell may be built from a
{\sl fundamental domain}
$\tM$
by the action of a discrete (not necessarily Abelian) group $G$.

                                                            \toCB
Generalization to continuous time\rf{bowen,pexp} amounts to the replacement
%$ z\,=\,e^{-s} $,
$ z^{\sigma_p} \rightarrow e^{-s \sigma_p} $,
where $\sigma_p$ is now the (not necessarily integer)
%{\sl time-}
period of the prime cycle $p$:
$$
Z(\beta,s)\,=\,\prod_{p\in\PP} \exp \left( - {
 \sum_{r=1}^\infty {1 \over r}
 { e^{(\beta \cdot \hn_p- s \sigma_p) r } % z^{n_p r}
 \over { | \det \left( {\bf 1}-{\bf J}_p^{r} \right) | } }
 } \right)
\,\, .
%Eq.~(14)
$$

The lattice symmetry of the Lorentz billiard has important consequence on
the properties of the function $Q(\beta)$ are best illustrated by introducing
its analytic continuation at $\beta = i k$.  The function $F(k)=Q(ik)$ is the
rate associated with the incoherent scattering function $\langle \exp i k
\cdot (\hat x_t - x) \rangle_M$ considered in light or
neutron scattering experiments in liquids, in particular, by Van Hove\rf{BoonYip80,VanHove54}.
The vector $k$ is interpreted as the wavenumber of the hydrodynamic modes of
diffusion which we also find in the Lorentz gas.  The function $F(k)$ turns
out to be a dispersion relation since $F(k)=-D k^2 + {\cal O}(k^4)$ in an
isotropic diffusive system.  The isotropy of a liquid implies that the
dispersion relation only depends on the amplitude $\vert k\vert$ of the
wavenumber.


On the other hand, the lattice symmetry of the Lorentz gas
imposes special restrictions on the properties of the dispersion relation
$F(k)$ and on the values taken by the wavenumber.  The present classical
problem of diffusion is similar to the quantum motion of a particle in a
periodic potential.  Hence, the wavenumber takes its values in the so-called
Brillouin zone\rf{BoSmWi36,Harrison70}.  A mode of diffusion is associated with each value of
the wavenumber $k$ so that the direction of $k$ is privileged in the
system.  As a consequence, the symmetry of the lattice is reduced by the choice
of $k$.  This symmetry reduction is formalized by the concept of little group
associated with the wavenumber $k$, which is the subgroup of the lattice point
group leaving invariant the vector $k$.  For most values of $k$ inside the
Brillouin zone, the little group is trivial because it contains only the
identity.  However, the little group is larger when the wavenumber belongs to
special symmetry lines or symmetry points in the Brillouin zone.  In particular,
the little group coincides with the full point group when $k=0$.

remember to cite Cvitanovi\'c and Eckhardt\rf{CvitaEckardt} {\em Symmetry
decomposition of chaotic dynamics}

The preceding considerations concern the consequences of the lattice symmetry
on the factorization of the zeta function.  There is a different problem which
is to express the zeta function in terms of the prime periodic orbits of the
fundamental domain $\tilde M$ of the lattice (see Fig. 2) rather than
those of the elementary (Wigner-Seitz) cell $M$.
%This problem is a priori independent of the factorization
%discussed above and presents the following difficulty.
Here the stumbling block appears to be
the breaking of the rotational symmetry by
the auxilliary vector $\beta$, or, in other words,
the non-commutativity of translations and rotations.
More precisely, the global distance
$ \hat\phi^{r \sigma_{\tilde{p}}} (\tx{\tpk}) - \tx{\tpk} $, $\tx{\tpk} \in \tp$,
% $ \hn_{r |\t p|}(\tx{\tpk}) $
depends on the starting cycle point if
$\tp$ is only a segment of the global cycle $p$. An
example is the diamond-shaped cycle of Fig.~3;
%the problem is that the $\tp$ segment of
%the global trajectory is not a translation in $\hM$.
depending whether one starts at $\tx_1$ or $\tx_2$, the global
distance covered in time $\sigma_{\tilde{p}}$ is either the short or the
long diagonal.

We end this section with the remark that the signature of
the lattice symmetry appears in the behavior of the function $Q(\beta_x,
\beta_y)$ away from $\beta=0$\rf{Gaspard92a}, even in
triangular or square lattices where diffusion is isotropic.  Accordingly, the
full function $Q(\beta)$ contains more information on the lattice symmetry than
the diffusion matrix of the second derivatives of $Q(\beta)$.

Compared to the literature\rf{CvitaEckardt,robb},
the new feature
of the problem at hand is use of vector-valued functions.
The arbitrary vector $\beta$ is only a device
for generating moments~--~the moments themselves are
invariant under discrete symmetries,
but it can be interpreted in terms of the wavenumber of
the hydrodynamic modes of diffusion.

We have attempted to repeat the above
derivation for the diffusion matrix for irreducible representations of $G$,
 factorized subspaces, (others are of less immediate interest),
but have failed to obtain a
formula for the diffusion constant Eq.~(17) expressed in terms
of cycles on the fundamental domain $\tM$. The stumbling block
appears to be
 the breaking of the rotational symmetry by
 the auxilliary vector $\beta$, or, in other words,
the non-commutativity of translations and rotations.
More precisely,
in contrast to Eq.~(11), the global distance
$ \hf^{r \sigma_{\tilde{p}}} (\tx{\tpk}) - \tx{\tpk} $, $\tx{\tpk} \in \tp$,
 $ \hn_{r |\t p|}(\tx{\tpk}) $
depends on the starting cycle point if
$\tp$ is only a segment of the global cycle $p$. An
example is the diamond-shaped cycle of Fig.~3;
the problem is that the $\tp$ segment of
the global trajectory is not a translation in $\hM$.
depending whether one starts at $\tx_1$ or $\tx_2$, the global
distance covered in time $\sigma_{\tilde{p}}$ is either the short or the
long diagonal. We have not found a natural way of associating
a global distance in  formula Eq.~(17) with a fundamental domain
cycle $\tp$.


 As we are concerned with the long time behavior,
 this problem can be circumvented
 by replacing $ \hf^t(\tx{\tpk}) $ by the mean
 drift in the $t \rightarrow \infty$ limit.
 $ \hf^t(\tx{\tpk}) $ is a translation in $\hM$ for each
 complete cycle $p$ in $M$, so we replace
 \bea
 \hf^t(\tx{\tpk}) - \tx{\tpk}
 \,\Longrightarrow \,
 &&
 { { \hf^{m_p t}(\tx{\tpk}) - \tx{\tpk} }
 % \over         m_p
 }
    % \,\equiv \, r {\tilde n}_{\tp} (\tx{\tpk})
 \continue
t &=& r \sigma_{\tilde{p}}, \quad m_p = \sigma_p/\sigma_{\tilde{p}} \quad \tx \in \tp \,\, ,
 \eea
 in Eq.~(240).
 The magnitude of ${\tilde n}_{\tp}(\tx{\tpk})$, the mean
 global drift per one traversal of the fundamental cycle $\tp$, is
 independent of the starting point, but its direction is not; the
 reason is that each fundamental domain cycle corresponds to a set of
 trajectories in $\hM$.
 The ${1 \over {|G|}} \sum$ average in
 Eq.~(240) then generates all distinct global drift
 directions, so we can again replace the
 sum over cycle points by the factor $\sigma_{\tilde{p}}$, and obtain the
 $Z$ function Eq.~(14) for the $\alpha $ irreducible subspace
 $$
 Z(\beta,s)_\alpha\,=\,\prod_{\tp \in \t{\cal{P}} } \exp
 % \left( -
  % \sum_{r=1}^\infty {1 \over r}
 % {{
  % \chi_\alpha(h^r_{\tp})
  % }
 % \over
 % { | \det \left( \bf{1}-\t{\bf J}_{\tp}^{r} \right) | }
 % }
 % e^{ ( \beta \cdot {\tilde n}_{\tp} - s \sigma_{\tilde{p}}) r}
  % \right)
 \,\, .
 Eq.~(24)
 $$
 $\hn_{\sigma_{\tilde{p}}}(\tx_k)$
%{\tt NOnsense...}
 The leading eigenvalue of the
unsymmetrized
 operator Eq.~(8) is
 the leading eigenvalue of the symmetric subspace for which
 $\chi_\alpha(g)=1$ for all $g \in G$.



We have thus obtained a description of global diffusive
properties of an infinite periodic dynamical
system, such as the Lorentz gas, in terms of periodic orbits
restricted to the elementary cell.
These formulas have been tested extensively
in \refrefs{CGS92,BaEvCo93} on the Lorentz gas, and in \refref{art91} on 1-dimensional
mappings.
Related trace formulas have been independently introduced
and tested numerically in \refref{Vance92}.
The formalism has been generalized to evaluation of
power spectra of chaotic time series in \refref{CviPik93}.
However, a derivation of the corresponding formulas for dynamics restricted
to the fundamental domain sketched in Sect. 3 needs further development.

In practice, the periodic orbit evaluations
of the diffusion constant converge poorly compared with averages over scalar
quantities such as the Lyapunov exponents.  These difficulties are due to
several reasons: (1) the diffusion coefficient is not a mean but a variance
which is always more difficult to evaluate than mean quantities like Lyapunov exponents;
(2) there is presently no simple formula for the diffusion
coefficient in terms of the periodic orbits of the fundamental domain; (3)
systems like the Lorentz gas do not have simple symbolic dynamics and
the analyticity of the associated zeta functions may also be affected
by the flow discontinuities associated with the grazing trajectories
(trajectories tangent to the disks).


\bigskip

    \PC{the text from Cvitanovi\'c, Gaspard and Schreiber\rf{CGS92}}
The Lorentz gas\rf{Lorentz1905} is one of the simplest nontrivial models
of deterministic diffusion.
Diffusion of a light molecule in a gas of heavy scatterers
is modelled
by a point particle in a plane bouncing off an array of reflecting disks.
As a billiard built up completely of
concave surfaces and as a pure hyperbolic system, the
Lorentz gas is a good candidate for description in terms of cycle
expansions\rf{AACI}.
This might seem a hopeless task, as one
has to deal with all periodic and aperiodic
solutions of an infinitely extended system. An
approach based on larger and larger finite portions of the system is described
in \refref{GasNic90}, with the diffusion constant related to
%the scaling behaviour of
the escape rates from such finite portions.
%with its size.
%PG
As far as escape rates are obtained in direct numerical simulations
this approach has been shown to be effective\rf{GaBaMaHo92}.
However, from the cyclists point of view
%PG
where the rates are calculated from the periodic orbits,
this approach is impractical; with each added disk new peculiarities arise
in the enumeration of periodic orbits, and with current
methods there is little hope of getting results for more than a few disks,
and no hope of approaching the desired scaling limit.

A recent approach, introduced in \refref{LorentzDiff} and
tested in this paper, exploits the fact that the periodic Lorentz
gas
%(and many other systems)
can be constructed by putting together
translated copies of an elementary cell.
%When this elementary cell is
%itself invariant under a discrete symmetry group $G$ the lattice can be tiled
%into images under $G$ and the lattice translations of a fundamental
%domain.
Therefore quantities characterizing global dynamics, such as
the Lyapunov exponent and the diffusion constant, can be
computed from the dynamics restricted to the elementary cell.

The classical Boltzmann equation for evolution of 1-particle density is
based on stosszahlansatz, neglect of particle correlations prior to, or
after a 2-particle collision. It is a very good approximate description
of dilute gas dynamics, but a difficult starting point for inclusion of
systematic corrections. In the theory of deterministic diffusion
developed in recent years, no correlations are neglected - they are all
included in the exact cycle expansions for transport coefficients such as
the diffusion constant.

The exact results are sometimes counterintuitive, and might help us
decide whether a diffusive phenomenon whose microscopic dynamics is hard
to observe directly, such as conductance fluctuations in a mesoscopic
device, is due to impurities or to deterministic transport. For example,
as some parameter (such as mean free flight time) is increased, the
deterministic diffusion coefficient reveals a non-monotone, fractal
dependence.

We will illustrate the theory with a computation of the
diffusion constant for deterministic diffusion in periodic Lorentz gas.
For systems of a few degrees of freedom these results are on rigorous
footing, but there are indications that they capture the essential
dynamics of systems of many degrees of freedom as well.

Honesty in
advertising requires disclaimer; no such fractal behavior of the
conductance has been detected experimentally so far.
