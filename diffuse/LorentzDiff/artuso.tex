%Date:  Thu, 25 APR 91 17:54 N
%From: ARTUSO%MILANO.INFN.IT@ICINECA.CINECA.IT
%
\input head
%change \nnnn which is a non defined font
\DRAFT
%\TITLE % ins title
%\AUTHOR % ins author
%\FROM
%ins inst
%\AUTHOR
%\FROM
%
%\ENDTITLE
%\ABSTRACT
%ins abstract
%\ENDABSTRACT
                                            25 Aril 1991

\SECTION Deterministic diffusion analysis through periodic orbits

The starting observation is in [DANA]: in the analysis of lifts on the
cylinder of Anosov diffeomorphisms he claims that a positive diffusion
emerges from a balance between closed orbits on the cylinder and
unstable periodic accelerator modes. In a sense this leads to the
picture that diffusion is determined by the relative populations of
closed orbits and accelerating orbits: in the asymptotic limit this
distribution should converge to a Gaussian distribution characterized
by the diffusion coefficient $D$.

Probably the simplest context in which the problem of deterministic
diffusion arises is that of circle maps [DETDIFF]: in this case
diffusion sets in (for suitably chosen parameter values) when one
consider a lift of the circle maps, so that the motion is in principle
unbounded.

Take for instance a map like that in Fig. ~1

\figure fig1 7 7
Overshooting lift of a circle map
\cr

In this case periodic orbits of the map on the circle contain either
closed orbits for the lift or running orbits, to which we can attach a
label $\sigma_p$ which gives the number of box jumped (which can be
either positive or negative) after which the orbit closes on the
torus.

By taking a gaussian distribution
one has that the probability of having jumped
$\sigma$ steps in $n$ time units will be given by;
$$
p_n(\sigma)\, = \, {{1}\over {\sqrt{4 \pi D n}}}\, e^{\left(
-\sigma^2 / 4Dn \right)}
\EQ(1)
$$
The main problem into consideration of periodic orbit expansion in
this context is that computing averages for dynamical systems is
generally feasible if one has to deal with multiplicative weights
associated with the trajectory, and obviously the factors $\sigma_p$
to which we are interested do not have this property.
This can be circumvented by using exponentials, so turning additive
weights into multiplicative ones.

Once we use \equ(1) it is fairly simple to observe that
$$
\langle e^{\alpha \sigma} \rangle _n \, = \, e^{nD \alpha^2}
\EQ(2)
$$
to compute the average we can start from the $(0,1)$ box (so we have
not to subtract any initial box--labelling factor) and we can write
$$
\langle e^{\alpha \sigma}\rangle _n \, = \, \int_0^1\, dx \,
\int_{-\infty}^{+\infty}\, dy \, \delta \left(y-f^n(x)\right) \,
e^{\alpha [y] }
\EQ(3)
$$
This is easily seen to receive contributions (with different weights)
by all periodic orbits of the map on the circle and can be recast in
the form
$$
\langle e^{\alpha \sigma} \rangle _n \,=
\,
\sum_{x\in Fix
\tilde{f^n}}{e^{\sigma_{x}\alpha }\over{\left| 1- \Lambda_x \right|}}
\EQ(4)
$$
Where the subscript individuates the map on the circle.

As usual one can introduce a formal parameter $z$ and observe that
$\alpha ^2 D=-\log z_c$ where $z_c$ is the point where
$$
\sum_1^{\infty} z^n \sum_{x\in Fix
\tilde{f^n}}{e^{\sigma_{x}\alpha}\over{\left| 1- \Lambda_x \right|}}
$$
diverges.

$D$ is then determined
by looking at the first zero of the zeta function
$$
Z(z)\,=\,
\prod_{m=0}^{\infty} \prod_{\{p\} }\, \left( 1- {{z^{n_p}e^{\sigma_p}}
\over {\left| \Lambda_p \right| \Lambda_p^m}} \right)
\EQ(5)
$$
and one has the relation $z_c(\alpha )=e^{-\alpha ^2 D}$.
which in particular means
$$
D\,=\, - {{1}\over {2}}{{\partial^ 2} \over {\partial
\alpha^2}}z_c(\alpha)\big|_{\alpha=0}
\EQ(6)
$$
{\nnnn {
$\bullet \qquad$ It was after looking at your paper that I realized
that this way of finding $D$ does not need involvement of gaussian
distribution, so it should be fairly general. This has been very
stupid of me.
}}

When $\alpha=0$ one recovers the usual escape rate zeta function,
whose first zero is obviously one, as the map on the circle does not
lead to any escape.

Let's now consider a map like the one in Fig. ~1, with $a=2$ (that is
the maximum of the map on the unit box is 2) [DETDIFF], and consider
$\zeta_0^{-1}(z)$ instead of the whole zeta function.
$$
\zeta_0^{-1}(z)\,=\, \prod_{\{ p \} } \left( 1- {{z^{n_p} e^{\alpha
\sigma_p}}\over {\left| \Lambda_p \right|}} \right)
\EQ(7)
$$
We can write explicitly this function as we have an unrestricted
grammar in seven symbols (see Fig. ~2).

\figure fig2 10 10
The map of Fig. ~1 when $a=2$
\cr

We have so, as curvatures cancels exactly due to completeness and
linearity
$$
\zeta_0^{-1}(z)=1-t_1-t_2-t_3-t_+-t_{\oplus}-t_--t_{\ominus}
$$
that is
$$
\zeta_0^{-1}(z)=1-{{3}\over{7}} z -{{4}\over{7}} z \cosh \alpha
$$
so that $z_c(\alpha)={{7}\over {(3+4\cosh \alpha )}}$, and by \equ(6)
we get $D=2/7$ which is known to be the correct result [DETDIFF].

We can easily generalize this result to every integer value of $a$:
as a matter of fact we get a complete grammar for each of these cases
and linearity then allows exact cancellation of all curvatures:
$$
\zeta_0^{-1}(z,a)\,=\, 1-{{3}\over{4a-1}} z
-{{4}\over{4a-1}}z\sum_{j=1}^{a-1} \cosh (j \alpha )
$$
from which we get $z_c(\alpha,a)=(4a-1)/(3+\sum_{j=1}^{a-1} 4\cosh
(j\alpha) )$.
Again by \equ(6) we can get
$$
D(a)\,=\,{{a(a-1)(2a-1)}\over {[3(4a-1)]}}
$$
which is known to be the exact result [DETDIFF], where we have
employed $\sum_{k=1}^n\,k^2=n(n+1)(2n+1)/6$.

Observe that though the complete zeta function factorizes
into $Z_{CO}(z)\cdot Z_{AM}(z)$, where $CO$ refers to closed orbits
and $AM$ to accelerator modes, this factorization would spoil
completely the resummation of the zeta function, $Z_{CO}(z)$ in
particular is the zeta function describing escaping of initial
conditions from the unit box.
\vfill\eject
{\nnnn {
{\centerline{{\bf Notes}}}

1. Also in this simpler context many things again may be considered:
non--linear case, asymmetric maps and inclusion of drift, maybe
anomalous diffusion..

2. What do you think?

}}
\SECTIONNONR References

\ref
\no AACI
\by  R. Artuso, E. Aurell and P. Cvitanovi\'c
\paper  Recycling of strange sets I: Cycle expansions
\jour Nonlinearity
\vol 3
\pages 325
\yr 1990
\endref

\ref
\no corr
\by F. Christiansen, G. Paladin and H.H. Rugh
\paper Determination of correlation spectra in chaotic systems
\jour
\vol
\pages
\yr
\endref

\ref
\no DANA
\by I. Dana
\paper Hamiltonian transport on unstable periodic orbits
\jour Physica
\vol D 39
\pages 205
\yr 1989
\endref

\ref
\no DETDIFF
\by S. Grossmann and H. Fujisaka
\paper Diffusion in discrete nonlinear dynamical systems
\jour Phys.Rev. A
\vol 26
\pages 1779
\yr 1982
\endref

\ref
\no GN
\by T. Geisel and J. Nierwetberg
\paper Onset of diffusion and universal scaling in chaotic systems
\jour Phys.Rev.Lett.
\vol 48
\pages 7
\yr 1982
\endref

\ref
\no Kap
\by M. Schell, S. Fraser and R. Kapral
\paper Diffusive dynamics in systems with translational symmetry: a
one--dimensional--map model
\jour Phys.Rev.
\vol A 26
\pages 504
\yr 1982
\endref


\bye
