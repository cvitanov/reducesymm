%dif14.tex
%14th and final version PG  8/11-93, PC 11/11-93
%13th version JPE, incorporating Gaspard comments  19/10-93
%12th version JPE  16/10-93
%11th version PC  12/10-93
%10th version PC  21/12-92
%       from dif8.tex - added comment in the abstract.
%8th version: PC   1/6-91
%7th version: JPE 22/5-91
%6th version: PC 20/5-91        tex/diffus/dif6.tex
%5th version: PC 12/5-91
%4th version: JPE 6/5-91
%%%%%%%%%%NOTE They really mean 10pt magstep 1.2 undsoweiter
\input head.tex
\input ps/ps_fonts.tex
\null
\parindent=20pt
\nopagenumbers
%%%%%%%%%%%%%%%%%%%%%%%
%%%% paper %%%%%%%%%%%%
%%%%%%%%%%%%%%%%%%%%%%%
\papwidth=18.75truecm
\papheight=26truecm
\voffset=0.0truecm
\hoffset=-1.5truecm
%%%%%%%install variables%%%%%%%%%%%%%%%%%%%
\hsize=\papwidth
\vsize=\papheight
\font\titlefont=ptmb at 15 pt
\def\t{\tilde}
\def\h{\hat}
\def\tr{{\rm tr}\,}
\def\II{{\cal I}}
\def\sumprime{\mathop{{\sum}'}}
\def\hflow{\widehat{\phi}^t}
\def\flow{\phi^t}
\def\tflow{\widetilde{\phi}^t}
\def\tf{\tilde f}
\def\hf{\hat f}
\def\hn{\hat n}
\def\tn{{\tilde n}}
\def\hM{\widehat M}
\def\tM{{\widetilde M}}
\def\tp{{\tilde p}}
\def\hp{h_{\tilde p}}
\def\tphi{\widetilde \phi}
\def\hphi{\widehat \phi}
\def\pk{_{p,k}}
\def\tpk{_{\tilde p,k}}
\def\time{\sigma_p}
\def\ttime{\sigma_{\tilde{p}}}
\def\hx{\hat x}
\def\tx{\tilde x}
\def\txp{\tilde x'}
\def\txgn{\tilde x,g,n}
\def\itt{_{t+t'}}
\def\itp{_{t'}}
\def\WW{{\cal W}}
\def\Xttt{X\itp(x'')}
\def\tO{\widetilde\Omega}
\def\hO{\widehat\Omega}
%%%%% NOTE: redefinitions (de-tildeing away): %%%%%%%%%%%%%%%%
%\def\hn{n}
\def\xi{x}
\def\pk{}
\def\tpk{}
\vskip2.5truecm
{\titlefont
\centerline{Transport Properties of the Lorentz Gas}
\centerline{in Terms of Periodic Orbits}
}
\medskip
\medskip
\smallskip
\centerline{\bf PREDRAG CVITANOVI\'C}
\smallskip
{\eightpoint
\centerline{Niels Bohr Institute, Blegdamsvej 17, DK-2100 Copenhagen \O, Denmark}
}
\medskip
\smallskip
\centerline{\bf JEAN-PIERRE ECKMANN}
\smallskip
{\eightpoint
\centerline{D\'epartement de Physique Th\'eorique, Universit\'e de Gen\`eve,
CH-1211 Geneva 4, Switzerland}}
\smallskip
\centerline{\bf and}
\medskip
\centerline{\bf PIERRE GASPARD}
{\smallskip
\eightpoint
\centerline{Service de Chimie-Physique,
Universit\'e Libre de Bruxelles,
Campus Plaine,
B-1050 Brussels, Belgium
}
}
\medskip
{\eightpoint
\centerline{({\it Received 17 October 1993})}}
\medskip
\medskip
\medskip
\hbox{

{\eightpoint
{\vbox
{\leftskip=20pt\rightskip=20pt{\noindent{\bf Abstract---}We
establish a formula relating
global diffusion in a space
periodic dynamical system to cycles in the elementary cell which
tiles the space under translations.
}}}}}
\SECTION INTRODUCTION

The diffusive properties of a ``Lorentz gas'' [L]
have been studied extensively in the literature,
see refs.~[S, BS, Bu, GN, MZ].
The novelty
of the approach presented here is that it provides an explicit
connection between the global diffusion and the dynamics restricted to
an elementary cell.
Our method applies to any  hyperbolic dynamical system that is
a periodic tiling $\hM=\bigcup_{ \hn \in T} M_{
\hn}$
of the dynamical phase space $\hM$ by {\sl translates}
$M_{\hn}$
of an {\sl elementary cell} $M$, with $T$ the Abelian group of lattice
translations.
Furthermore, each elementary cell may be built from a
{\sl fundamental domain}
$\tM$
by the action of a discrete (not necessarily Abelian) group $G$.

These concepts
are best illustrated by a specific example, a
Lorentz gas based on a Sinai billiard [S] on the hexagonal lattice
as in Fig.~1, with disks sufficiently
large so that no infinite length free flight is possible.
%\figure fig1 19 16
\figurewithtex ps/fig1.ps ps/fig1.tex 15 17
Tiling of $\hM$, a periodic lattice of reflecting disks, by the
fundamental domain $\tM$. Indicated is an example of a
global trajectory $\hx_t$ together with the corresponding
elementary cell trajectory ${x}_t$ and the fundamental domain
trajectory $\tx_t$.
\cr

It should be stressed that $\hM$ refers here to the full phase space, i.e.,
both the spatial coordinates and the momenta.
The spatial component of $\hM$ is the complement
of the disks in the {\sl whole} space.
We shall relate the dynamics in $M$
to diffusive properties of the Lorentz gas in
$\hM$,
using functional determinants and $\zeta$-functions.


It is convenient to define a time evolution operator for each of the 3
cases of Fig.~1.
$\hx_t\,=\,\hflow ( \hx)$
denotes the point in the global space
$\hM$
obtained by the flow in time $t$.
$x_t\,=\,\flow ( x)$
denotes the corresponding flow in the elementary cell;
the two are related by
$$
\hn_t(x)= \hflow ( x) - \flow ( x) \in T \,\, ,
\EQ(hatn)
$$
the translation of the endpoint of the global path into the
elementary cell $M$.
The quantity $\tx_t\,=\,\tflow ( \tx)$
denotes the flow in the fundamental domain
$\tM$;
$\tflow ( \tx)$ is related to
$\flow ( \tx)$ by a discrete symmetry
$g \in G$ which maps $\tx_t\in \tM$ to
${x}_t \in {M}$.

Fix a vector $\beta \in {\bf R}^d$, where $d$ is the dimension
of the phase space. We will compute the diffusive properties
of the Lorentz gas from the generating function %s
$$
\langle e^{\beta \cdot (\hx_t -x) } \rangle_M
\,\, , \quad
% \langle e^{\beta \cdot (\hx_t -{\tx}) } \rangle_{\tM}
% \,\, ,
\EQ(1)
$$
where the average %s are
is over all $x \in M$. %, $\tx \in \tM$ respectively.
% The discrete group $G$ that tiles the cell $M$ by
% copies of the fundamental domain $\tM$ acts on
% spatial coordinates in $\hM$ by symmetry operations
% $g \in G$ and moves the $\beta$ vector, resp. the orbits,
% around. Hence the two generating functions \equ(1) are
% related by
% $$
% \langle e^{\beta \cdot (\hx_t -x) } \rangle_M \,=\,
% {1 \over {|G|}} \sum_{g \in G}
% \langle
% e^{(g\beta) \cdot (\hx_t -\tx) } \rangle_{\tM}
% \,\, .
% \EQ(21)
% $$
%The arbitrary vector $\beta$ is only a device
%for generating moments--the moments themselves are
%invariant under discrete symmetries.
%For example,
%by rotational invariance $g^T g = 1$ and
%the diffusion constant
%formula \equ(5) is also valid restricted to  $\tM$.
% In what follows we shall therefore omit the
% ${1 \over {|G|}} \sum$ average in \equ(21).
%Indeed,

The diffusive properties can be obtained by studying
$$
Q(\beta)\,=\,\lim_{t\rightarrow \infty} {1\over t} \log
\langle e^{\beta \cdot (\hx_t -x) } \rangle_M ~,
\EQ(2)
$$
and its derivatives at $\beta=0$. Clearly $Q(0)=0$, and,
if by symmetry all odd derivatives vanish,
there is no drift
$$
\left .{{\partial} \over {\partial \beta_i}}
Q(\beta)\right |_{\beta=0}
 \,=\, \lim_{t\rightarrow \infty} {1\over t}
\langle {(\hx_t -x)_i } \rangle_M \,=\, 0~.
\EQ(3)
$$
In that case, the second derivatives
$$
\left . {{\partial} \over {\partial \beta_i}}
{{\partial} \over {\partial \beta_j}}
Q(\beta)\right |_{\beta=0} \,=\,\lim_{t\rightarrow \infty} {1\over t}
\langle {(\hx_t -x)_i (\hx_t -x)_j } \rangle_M ~,
\EQ(4)
$$
yield a (generally anisotropic) diffusion matrix.
The spatial diffusion constant is then given by
$$
D\,=\,{1\over 2} \sum_i
\left .{{\partial}^2 \over {\partial \beta^2_i}}
Q(\beta)\right |_{\beta=0}
\,=\, \lim_{t\rightarrow \infty} {1\over{2 t}}
\langle {(\hat{q}_t -q)^2 } \rangle_M~
\,\, ,
\EQ(5)
$$
where the $i$ sum is restricted to the spatial components $q_i$ of
the phase space vectors $x$.
% The squared distances are rotationally
% invariant, so for them the group average \equ(21) drops out
% $$
% \langle {(\hat{q}_t -q)^2 } \rangle_M
 % = \langle {(\hat{q}_t -q)^2 } \rangle_\tM
% \,\, ,
% \EQ(5a)
% $$
% and the  constant can be obtained by averaging over
% a single copy of the fundamental domain $\tM$.

We next describe the connection between Eq.\equ(2)
and periodic orbits in the fundamental cell.
As the full $\hM \rightarrow \tM$ reduction is complicated by
the nonabelian nature of $G$, we first introduce the main ideas in
the abelian $\hM \rightarrow M$ context, and discuss the problems
associated with the full reduction to Sect.~3.

\SECTION REDUCTION FROM $\hM$ TO $M$

The general idea of our approach consists in writing
$$
\eqalign{
\langle e^{\beta \cdot (\hx_t -x) } \rangle_M
\,&=\, \int_{x \in M ,~ \hat{y} \in \hM } dx d\hat{y}\,
e^{\beta \cdot (\hat{y} -x) }
\, {\rm Prob}_t (x \rightarrow \hat{y})
\cr
\,&=\, {1 \over {|M|}} \int_{x \in M ,~\hat{y} \in \hM}
dx d\hat{y}\,
e^{\beta \cdot (\hat{y} -x) }
\delta(\hat{y} - \hflow (x))~.
\cr
}
%\EQ(6)
$$
Here, ${|M|=\int_M dx}$ is the volume of the elementary cell $M$.
Note that there is a unique lattice translation $\hn$ such that
$\hat{y}=y - \hn$, with $y \in M$.
%%  and also note that
%% $y=x_t$ {\sl iff} $\hat{y}=\hx_t$, since the flow is reversible.
Therefore, and this is the main point, translational invariance
together with \equ(hatn) can be used to reduce this average to
the elementary cell:
$$
\langle e^{\beta \cdot (\hx_t -x) } \rangle_M
\,=\, {1 \over {|M|}} \int_{x,y \in M} dx dy\,
e^{\beta \cdot (\hflow (x) -x) }
\delta(y - \flow (x))~.
\EQ(7)
$$
In this way the global $\hflow$ flow averages can be computed
by following the flow $\flow$ restricted to the elementary cell $M$.
As is well known [R], the $t\rightarrow \infty$ limit
of such averages can be recovered by means of transfer operators.
The Equation \equ(7) suggests that we study the operator
$ {\cal L}^t$ whose kernel is given by
$$
{\cal L}^t(y,x)\,=\,e^{\beta \cdot (\hx_t -x) } \delta(y-x_t)~,
\EQ(8)
$$
where $\hx_t\,=\,\hflow (x) \in \hM$, but ${x,x_t,y \in M}$.
It is straightforward to check that
this operator has the semigroup property,
$
\int_{M} dz\,
{\cal L}^{t_2}(y,z) {\cal L}^{t_1}(z,x) \,=\,
{\cal L}^{t_2+t_1}(y,x)~
$.
The quantity of interest \equ(2) is
given by the leading eigenvalue of
$ {\cal L}^t$,  $\lambda_0=e^{t Q(\beta)}$.
In particular, for $\beta=0$, the operator \equ(8) is the
Perron-Frobenius operator, with the
leading eigenvalue $\lambda_0=1$
%and this implies $Q(0)=0$
(the probability conservation).

To evaluate the spectrum of
$ {\cal L}$, consider
$$
\tr{\cal L}^t\,=\,\int_M dx\,
 e^{\beta \cdot \hn_t(x) } \delta(x-x_t)~.
\EQ(10)
$$
Here $\hn_t(x)$
is the discrete lattice translation defined in \equ(hatn).
For discrete time and hyperbolic dynamics
we have
$$
\tr{\cal L}^t\,=\, \sum_{p: \time r=t,~ r \in {\bf N}}
%\sum_{k=1}^{\time}
\sum_{x \in p}
{ e^{\beta \cdot \hn_t(x\pk) }
 \over {|\det({\bf 1}-{\bf J}^{r}(x\pk))|}}~,
\EQ(11)
$$
where the sum is over periodic points of all prime cycles $p$ whose period
$\time$ divides $t$, and
% $x\pk$ is the $k^{\rm th}$ point on the cycle, and
$
{\bf J}_p(x)\,=\, D f^{\time}(x\pk)%_{\vert_{x_k \in p} }
\,\, .
% \EQ(12)
$
Note that the sum over cycle points of $p$
can be replaced by a factor ${\time}$,
as $\det({\bf 1}-{\bf J}_p) = \det({\bf 1}-{\bf J}_p(x))$
and $\hn_p=\hn_{\time}(x\pk)$
are independent of $x$. For the Jacobian ${\bf J}_p$
this follows by the
chain rule, and for the travelled distance $\hn_p$
this follows by continuing the path periodically in $\hM$.
For the discrete time case we finally obtain
$$
\det(1-z{\cal L})\,=\,\prod_{p} \exp \left( - {
 \sum_{r=1}^\infty {{z^{\time r}} \over r}
 { e^{r \beta \cdot \hn_p }
 \over { | \det \left( {\bf 1}-{\bf J}_p^{r} \right) | } }
 } \right)
~,
\EQ(13)
$$
where the product runs over the set $\PP$ of prime cycles.

%As we shall show in more detail in Sect.~4,
Generalization to continuous time
[Bo, CE1] amounts to the replacement
%$ z\,=\,e^{-s} $,
$ z^{\time} \rightarrow e^{-s \time} $,
where $\time$ is now the (not necessarily integer)
%{\sl time-}
period of the prime cycle $p$:
$$
Z(\beta,s)\,=\,\prod_{p\in\PP} \exp \left( - {
 \sum_{r=1}^\infty {1 \over r}
 { e^{(\beta \cdot \hn_p- s \time) r } % z^{n_p r}
 \over { | \det \left( {\bf 1}-{\bf J}_p^{r} \right) | } }
 } \right)
\,\, .
\EQ(14)
$$
The associated Ruelle $\zeta$ function is then
(see e.g., ref. [AAC] for details)
$$
1/\zeta(\beta,s)\,=\,\prod_{p\in \PP}
 \left( 1 -{ e^{\beta \cdot \hn_p- s \time}
 \over {|\Lambda_p|} } \right)
 \,\, ,
\EQ(15)
$$
with $ \Lambda_p\,=\,\prod_e \lambda_{p,e}$
the product of the expanding eigenvalues of ${\bf J}_p$.
(This formula can also be obtained by using the suspension formula in Appendix
C.3 of [R], with the discrete dynamical system leading from collision to
collision with the central disk, and the ceiling function being
the time between these hits.)

Our first result is therefore:
{\sl
The function $Q(\beta)$ of Eq.~\equ(2) is the largest solution
of the equation $Z(\beta, Q(\beta ))=0$ (or equivalently,
of\/ $1/\zeta(\beta, Q(\beta ))=0$).}

The above infinite products can be rearranged as expansions with
improved convergence properties [AAC]. To present the result, we
define $t_p=e^{\beta \cdot \hn_p -s \time} / |\Lambda_p| $,
and expand the $\zeta$ function \equ(15) as a
formal power series,
$$
\prod_{p\in\PP} (1- t_p) \,=\,  1-{\sumprime_{p_1,\dots,p_k}}
t_{\{p_1,\dots,p_k\}}~,
\EQ(aa1)
$$
where
$$
t_{\{p_1,\dots,p_k\}} \,=\,   (-1)^k t_{p_1} t_{p_2}\cdots t_{p_k} ~,
$$
and the sum is over all distinct non-repeating combinations of prime cycles.
Following the derivation of Eq.~(35) and Eq.~(80) in [AAC]
we get, for example,
$$
\left .{{\partial} \over {\partial \beta_i}}
Q(\beta)\right |_{\beta=0}  =
{
 {\sumprime (-1)^k
 {
  {(\hn_{p_1}+ \cdots+ \hn_{p_k})_i}
 /
  {|\Lambda_{p_1}\cdots \Lambda_{p_k}|}
 }
 }
\over
{\sumprime (-1)^k
 {
  (\sigma_{p_1}+ \cdots+ \sigma_{p_k})
 /
  {|\Lambda_{p_1}\cdots \Lambda_{p_k}|}
 }
 }
}~,
\EQ(16)
$$
with sums as in \equ(aa1). Two derivatives yield our second result:

{\sl ~The diffusion constant \equ(5) is given by }
$$
D\,=\,{1 \over 2}
{
 {\sumprime (-1)^k
 {
  {(\hn_{p_1}+ \cdots+ \hn_{p_k})^2}
 /
  {|\Lambda_{p_1}\cdots \Lambda_{p_k}|}
 }
 }
\over
{\sumprime (-1)^k
 {
  (\sigma_{p_1}+ \cdots+ \sigma_{p_k})
 /
  {|\Lambda_{p_1}\cdots \Lambda_{p_k}|}
 }
 }
}~.
\EQ(17)
$$
Note that the global trajectory is in general not periodic,
$\hn_p \neq 0 $; nevertheless, the reduction to the elementary cell
enables us to compute relevant quantities
in the usual way, in terms of periodic orbits.

\SECTION CONSEQUENCES OF THE LATTICE SYMMETRY
% previously: \SECTION Desymmetrization
% orignially: \SECTION Reduction from $M$ to $\tM$

%The lattice symmetry of the Lorentz billiard has important consequence on
The properties of the function $Q(\beta)$ are best illustrated by introducing
its analytic continuation at $\beta = i k$.  The function $F(k)=Q(ik)$ is the
rate associated with the incoherent scattering function $\langle \exp i k
\cdot (\hat x_t - x) \rangle_M$ considered in light or
neutron scattering experiments in liquids, in particular, by Van Hove [BY,VH].
The vector $k$ is interpreted as the wavenumber of the hydrodynamic modes of
diffusion which we also find in the Lorentz gas.  The function $F(k)$ turns
out to be a dispersion relation since $F(k)=-D k^2 + {\cal O}(k^4)$ in an
isotropic diffusive system.  The isotropy of a liquid implies that the
dispersion relation only depends on the amplitude $\vert k\vert$ of the
wavenumber.

On the other hand, the lattice symmetry of the Lorentz gas
imposes special restrictions on the properties of the dispersion relation
$F(k)$ and on the values taken by the wavenumber.  The present classical
problem of diffusion is similar to the quantum motion of a particle in a
periodic potential.  Hence, the wavenumber takes its values in the so-called
Brillouin zone [BSW,H].  A mode of diffusion is associated with each value of
the wavenumber $k$ so that the direction of $k$ is privileged in the
system.  As a consequence, the symmetry of the lattice is reduced by the choice
of $k$.  This symmetry reduction is formalized by the concept of little group
associated with the wavenumber $k$, which is the subgroup of the lattice point
group leaving invariant the vector $k$.  For most values of $k$ inside the
Brillouin zone, the little group is trivial because it contains only the
identity.  However, the little group is larger when the wavenumber belongs to
special symmetry lines or symmetry points in the Brillouin zone.  In particular,
the little group coincides with the full point group when $k=0$.

These results have the following consequences on the factorization of the zeta
function $Z(k,s)$.  The little group associated with $k$ has a single
irreducible representation for most values of $k$, and the zeta function does
not factorize.  Only at those special values of $k$ where the corresponding
little group contains several irreducible representations, there exists a
factorization of the zeta function of the form [CE2]

$$Z(k,s) \ = \ \prod_{\alpha} \ Z_{\alpha} (k,s) \ , \eqno(3.1)$$

\noindent
where $\alpha$ runs over the irreducible representations of the little group
of $k$.  For instance, in the triangular Lorentz gas, the Brillouin zone has
several lines where the little group contains the identity together with a
reflection.  We can then simplify the calculation of the diffusion coefficient
by following the results of Sect. 2 applied to the factor of (3.1) corresponding
to the identity representation.  Nevertheless, the simplification obtained by
taking advantage of this reflection symmetry is modest since, in this case the
zeta function splits in only two factors.
% and using derivation
%along the symmetry line of the Brillouin zone.

%old \figure fig2 1 1
\figurewithtex ps/fig2.ps ps/fig2.tex     3 8
Unfolding of a trajectory of the fundamental domain under the action
of the group $G$.  Note that each fundamental domain orbit corresponds
to 12 distinct global orbits.
%$\hp $ is the discrete transformation that relates the endpoint
%of the cycle $\tp \in \tM$ to the endpoint of the corresponding
%trajectory segment in $M$.
%(PC suggestion; like fig 1, with $M$ divided into 12 copies of
%$\tM$ and $\hp$ indicated)
\cr

The preceding considerations concern the consequences of the lattice symmetry
on the factorization of the zeta function.  There is a different problem which
is to express the zeta function in terms of the prime periodic orbits of the
fundamental domain $\tilde M$ of the lattice (see Fig. 2) rather than
those of the elementary (Wigner-Seitz) cell $M$.
%This problem is a priori independent of the factorization
%discussed above and presents the following difficulty.
Here the stumbling block appears to be
% the breaking of the rotational symmetry by
% the auxilliary vector $\beta$, or, in other words,
the non-commutativity of translations and rotations.
More precisely,
in contrast to \equ(11), the global distance
$ \hat\phi^{r \ttime} (\tx{\tpk}) - \tx{\tpk} $, $\tx{\tpk} \in \tp$,
% $ \hn_{r |\t p|}(\tx{\tpk}) $
depends on the starting cycle point if
$\tp$ is only a segment of the global cycle $p$. An
example is the diamond-shaped cycle of Fig.~3;
%the problem is that the $\tp$ segment of
%the global trajectory is not a translation in $\hM$.
depending whether one starts at $\tx_1$ or $\tx_2$, the global
distance covered in time $\ttime$ is either the short or the
long diagonal.
This difficulty can be handled in the following way.

%old: \figure fig3 1 1
\figurewithtex ps/fig3.ps ps/fig3.tex    12  16
A fundamental domain 2-cycle $\tp$ which covers in one return time only
1/2 of the corresponding global cycle $\h p$. Note that each fundamental
domain cycle corresponds to 12 distinct global cycles.
 $\hat f(x)$ is the
collision-to-collision mapping induced by the flow.
To make the figure more readable, the disks have radii smaller
than those needed for the ``no infinite free flight''
condition.\cr

We can introduce the flow $\tilde\phi^t$ on the fundamental domain
$\tilde M$ using the full point group of the lattice.  Moreover, the density
$\rho(x)$ on which the Perron-Frobenius operator ${\cal L}^t$ acts can always
be decomposed using the projectors onto the irreducible representations of the
full point group.  For an arbitrary wavenumber $k$, the Perron-Frobenius
operator will mix the different components of the density $\rho(x)$, which we
can express by

$${\cal L}^t(y,x) \ = \ {\bf R}(x;k,t) \ \delta \lbrack y \ - \ \tilde
\phi^t (x) \rbrack \ , \eqno(3.2)$$

\noindent
where ${\bf R}(x;k,t)$ is a matrix ruling the dynamics of the different
components of the density.  With Eq. (3.2), the Perron-Frobenius operator is
reduced to the flow in the fundamental domain $\tilde M$.  The zeta function
can then be written as a product over the prime periodic orbits $\tilde p$ of
the fundamental domain

$$Z(k,s) \ = \ \prod_{\tilde p \in \tilde{\cal P}} \ \exp\Biggl\lbrace \ - \
\sum_{r=1}^{\infty} \ {1 \over r} \ {\rm tr} \lbrack \tilde {\bf R}_{\tilde
p}(k)^r\rbrack \ {{\exp ( - s \ttime r)}\over{\vert \det ( 1 \ - \ \tilde{\bf
J}_{\tilde p}^r) \vert}}\Biggr\rbrace \ , \eqno(3.3)$$

\noindent
where $\tilde{\bf R}_{\tilde p}(k)$ is the matrix ${\bf R}(x;k,t)$ associated
with the prime periodic orbit $\tilde p$.  It is only when the little group of
the wavenumber $k$ has nontrivial irreducible representations that the matrices
$\tilde{\bf R}_{\tilde p}(k)^r$ split into block-diagonal submatrices which
can be assigned to each irreducible representation so that the zeta function
factorizes as explained with Eq. (3.2).  The same discussion can be developed
with $\beta = i k$.

We end this section with the remark that the signature of
the lattice symmetry appears in the behavior of the function $Q(\beta_x,
\beta_y)$ away from $\beta=0$ [G], even in
triangular or square lattices where diffusion is isotropic.  Accordingly, the
full function $Q(\beta)$ contains more information on the lattice symmetry than
the diffusion matrix of the second derivatives of $Q(\beta)$.


\SECTION Conclusions

Compared to the literature [CE2,Ro],
the new feature
of the problem at hand is use of vector-valued functions in
Eq.~\equ(1).
The arbitrary vector $\beta$ is only a device
for generating moments~--~the moments themselves are
invariant under discrete symmetries,
but it can be interpreted in terms of the wavenumber of
the hydrodynamic modes of diffusion as discussed in Sect. 3.
% and by \equ(5) we can omit the
% ${1 \over {|G|}} \sum$ average in \equ(240) in what follows.
%The introduction of $\beta$ breaks the rotational symmetry and
%thwarts the factorization outlined above as argued in Sect. 3.
%We have attempted to repeat the above
%derivation for the diffusion matrix for irreducible representations of $G$,
% factorized subspaces, (others are of less immediate interest),
%but have failed to obtain a
%formula for the diffusion constant \equ(17) expressed in terms
%of cycles on the fundamental domain $\tM$. The stumbling block
%appears to be
% the breaking of the rotational symmetry by
% the auxilliary vector $\beta$, or, in other words,
%the non-commutativity of translations and rotations.
%More precisely,
%in contrast to \equ(11), the global distance
%$ \hf^{r \ttime} (\tx{\tpk}) - \tx{\tpk} $, $\tx{\tpk} \in \tp$,
% $ \hn_{r |\t p|}(\tx{\tpk}) $
%depends on the starting cycle point if
%$\tp$ is only a segment of the global cycle $p$. An
%example is the diamond-shaped cycle of Fig.~3;
%the problem is that the $\tp$ segment of
%the global trajectory is not a translation in $\hM$.
%depending whether one starts at $\tx_1$ or $\tx_2$, the global
%distance covered in time $\ttime$ is either the short or the
%long diagonal. We have not found a natural way of associating
%a global distance in  formula \equ(17) with a fundamental domain
%cycle $\tp$.


% As we are concerned with the long time behavior,
% this problem can be circumvented
% by replacing $ \hf^t(\tx{\tpk}) $ by the mean
% drift in the $t \rightarrow \infty$ limit.
% $ \hf^t(\tx{\tpk}) $ is a translation in $\hM$ for each
% complete cycle $p$ in $M$, so we replace
% $$
% \eqalign{
% \hf^t(\tx{\tpk}) - \tx{\tpk}
% \,\Longrightarrow \,&
% { { \hf^{m_p t}(\tx{\tpk}) - \tx{\tpk} }
 % \over         m_p
% }
    % \,\equiv \, r \tn_{\tp} (\tx{\tpk})
% \cr
% & t \,=\, r \ttime, \quad m_p = \time/\ttime \quad \tx \in \tp \,\, ,
% \cr
% }
% $$
% in \equ(240).
% The magnitude of $\tn_{\tp}(\tx{\tpk})$, the mean
% global drift per one traversal of the fundamental cycle $\tp$, is
% independent of the starting point, but its direction is not; the
% reason is that each fundamental domain cycle corresponds to a set of
% trajectories in $\hM$.
% The ${1 \over {|G|}} \sum$ average in
% \equ(240) then generates all distinct global drift
% directions, so we can again replace the
% sum over cycle points by the factor $\ttime$, and obtain the
% $Z$ function \equ(14) for the $\alpha $ irreducible subspace
% $$
% Z(\beta,s)_\alpha\,=\,\prod_{\tp \in \t{\cal{P}} } \exp
 % \left( -
  % \sum_{r=1}^\infty {1 \over r}
 % {{
  % \chi_\alpha(h^r_{\tp})
  % }
 % \over
 % { | \det \left( \bf{1}-\t{\bf J}_{\tp}^{r} \right) | }
 % }
 % e^{ ( \beta \cdot \tn_{\tp} - s \ttime) r}
  % \right)
% \,\, .
% \EQ(24)
% $$
% $\hn_\ttime(\tx_k)$
%%{\tt NOnsense...}
% The leading eigenvalue of the
%unsymmetrized
% operator \equ(8) is
% the leading eigenvalue of the symmetric subspace for which
% $\chi_\alpha(g)=1$ for all $g \in G$.

We have thus obtained a description of global diffusive
properties of an infinite periodic dynamical
system, such as the Lorentz gas, in terms of periodic orbits
restricted to the elementary cell.
These formulas have been tested extensively
in refs.~[CGS,BEC] on the Lorentz gas, and in ref.~[A] on 1-dimensional
mappings.
Related trace formulas have been independently introduced
and tested numerically in ref.~[V].
The formalism has been generalized to evaluation of
power spectra of chaotic time series in ref.~[CFP].
However, a derivation of the corresponding formulas for dynamics restricted
to the fundamental domain sketched in Sect. 3 needs further development.

In practice, the periodic orbit evaluations
of the diffusion constant converge poorely compared with averages over scalar
quantities such as the Lyapunov exponents.  These difficulties are due to
several reasons: (1) the diffusion coefficient is not a mean but a variance
which is always more difficult to evaluate;
%than mean quantities like Lyapunov exponents;
(2) there is presently no simple formula for the diffusion
coefficient in terms of the periodic orbits of the fundamental domain; (3)
systems like the Lorentz gas do not have simple symbolic dynamics and
the analyticity of the associated zeta functions may also be affected
by the flow discontinuities associated with the grazing trajectories
(trajectories tangent to the disks).
% These reasons affect the numerical implementation of the
% formulas proposed here.

\medskip

{\vbox
{\eightpoint
\LIKEREMARK{Acknowledgments}We thank
R.~Artuso for communicating to us his related results on diffusion in
circle maps [A] prior to publication, and to T.~Schreiber for
investigating in detail the applicability of the above formalism.
The work reported here was performed under the auspices of the Nordita
``Quantum Chaos and Measurement'' workshop April--June 1991.
JPE is grateful to the Niels Bohr Institute for hospitality and support,
and to the Fonds National Suisse for support,
PC to the Carlsberg Foundation for support, and PG to the
National Funds for Scientific Research (Belgium) for support.}
}
\vfill\eject
\SECTION REFERENCES

\eightpoint{
\ref
\no A
\by  R. Artuso
\paper  Diffusive dynamics and periodic orbits of
        dynamical systems
\jour   Phys. Lett.
\vol A 160
\pages 528
\yr 1991
\endref
\ref
\no AAC
\by  R. Artuso, E. Aurell and P. Cvitanovi\'c
\paper  Recycling of strange sets I: Cycle expansions
\jour Nonlinearity
\vol 3
\pages 325
\yr 1990
\endref
\ref
\no BEC
\by  A. Baranyai, D.J. Evans and E.G.D. Cohen
%\paper
\jour J. Stat. Phys.
\vol 70
\pages 1085
\yr 1993
\endref
\ref \no BS
\by  L. Bunimovich and Ya.G. Sinai
        % shows that decays are funny exponentials
\paper Markov Partition for Dispersed Billiard
\jour { Commun. Math. Phys. \bf 78}, 247 (1980);
      {\bf 78}, 479, (1980)
      {\sl Erratum,  ibid. \bf 107}, 357 (1986)
\endref
%the flow suspension approach ref, in which a flow is
%replaced by a Poincar\'e map with appropriate measure
%(not due to Ruelle?)
\ref \no  Bo
\by  R. Bowen
\paper Equilibrium states and the ergodic theory of
       Anosov-diffeomorphisms
\jour Springer Lecture Notes in Math.
\vol 470
\yr 1975
\endref
\ref \no BSW
\by  L. P. Bouckaert, R. Smoluchowski, and E. P. Wigner
\jour  Phys. Rev.
\vol 50
\pages 58
\yr 1936
\endref
\ref \no Bu
\by  L. Bunimovich
\paper Decay of correlations in dynamical systems with chaotic
       behavior
\jour  Sov. Phys. JETP
\vol 62
\pages 842
\yr 1985
\endref
\ref
\no BY
\by J.-P. Boon and S. Yip
\book  Molecular Hydrodynamics
\publisher New York, Dover
\yr 1991
\endref
\ref \no C
\by  P. Cvitanovi\'c
\paper Periodic orbits as the skeleton of
       classical and quantum chaos
%Los Alamos proceedings
\jour Physica
\vol D 51
\pages 138
\yr 1991
\endref
\ref \no CE1
\by  P. Cvitanovi\'c and  B. Eckhardt
\paper Periodic orbit expansions for classical smooth flows
\jour J. Phys.
\vol A 24
\pages L237
\yr 1991
\endref
\ref \no CE2
\by  P. Cvitanovi\'c and  B. Eckhardt
\paper Symmetry decomposition of chaotic dynamics
\jour Nonlinearity
\vol 6
\pages 277
\yr 1993
\endref
\ref \no CFP
\by  P. Cvitanovi\'c,  M.J.~Feigenbaum and A.S.~Pikovsky
\paper Periodic orbit expansions for power spectra of chaotic systems
\jour Rockefeller Univ. preprint
%\vol 2
%\pages 85
\yr Oct. 1993
\endref
\ref \no CGS
\by  P. Cvitanovi\'c,  P. Gaspard and T. Schreiber
\paper Investigation of the Lorentz Gas in terms of periodic orbits
\jour CHAOS
\vol 2
\pages 85
\yr 1992
\endref
%diffusion coefficient for Lorentz gas
%related to escape rate, Lyapunovs:
\ref
\no G
\by P. Gaspard
\paper From Dynamical Chaos to Diffusion, in
\inbook {From Phase Transitions to Chaos}
\bybook G. Gy\"orgyi, I. Kondor, L.
Sasv\'ari, and T. T\'el, eds.
\publisher  Singapore, World Scientific
\yr 1992
\endref
\ref \no GN
\by  P. Gaspard and G. Nicolis
\paper Transport properties, Lyapunov exponents, and entropy per
       unit time
\jour Phys. Rev. Lett.
\vol 65
\pages 1693
\yr 1990
\endref
\ref
\no H
\by W. A. Harrison
\book Solid State Theory
\publisher New York, Dover
\yr 1980
\endref
\ref \no L
\by  H.A. Lorentz
         %Lorentz gas introduced here
\jour Proc. Amst. Acad.
\vol 7
\pages 438
\yr 1905
\endref
%\ref \no La
%\by B. Lauritzen
     %It is shown that the periodic orbits that are point-wise
     %invariant under a symmetry operation require a special treatment.
%\paper  Discrete symmetries and the periodic orbit expansions
%\jour  Phys. Rev. A
%\vol 43
%\pages 603
%\yr  1990
%\endref
\ref \no MZ
\by  J. Mechta and R. Zwanzig
        %explicit numbers computed from simulations here
\paper  Diffusion in a periodic Lorentz gas
\jour Phys. Rev. Lett.
\vol 50
\pages 1959
\yr 1983
\endref
\ref \no R
\by  D. Ruelle
\book Statistical Mechanics, Thermodynamic Formalism
\publisher Reading MA, Addison-Wesley, Reading MA
\yr 1978
\endref
\ref \no Ro
\by  J.M. Robbins
\paper Discrete symmetries in periodic-orbit theory
\jour Phys. Rev. A
\vol 40
\pages 2128
\yr 1989
\endref
\ref \no S
\by  Ya.G. Sinai
%\paper Sinai billiards introduced here
\jour Usp. Mat. Nauk
\vol 25
\pages 141
\yr 1970
\endref
\ref
\no V
\by  W.N. Vance
\paper  Unstable periodic orbits and transport properties of
       nonequilibrium steady states
\jour Phys. Rev. Lett.
\vol 96
\pages 1356
\yr 1992
\endref
\ref
\no VH
\by  L. Van Hove
\jour Phys. Rev.
\vol 95
\pages 249
\yr 1954
\endref
}
\bye
