%%%%%%%%%%%%%%%%%%%%%%%%%
%%%%layout %%%%%%%%%%%%%%
%%%%%%%%%%%%%%%%%%%%%%%%%
\magnification \magstep1
%for dvips with figures maybe uncomment next line
%\special{!/magscale false def}
%%%%%%%%%%%%%%%%%%%%%%%
%%%% new parameters %%%
%%%%%%%%%%%%%%%%%%%%%%%
\newdimen\papwidth
\newdimen\papheight
\newskip\beforesectionskipamount  %how much to skip before section title
\newskip\sectionskipamount %how much to skip after section title
\def\sectionskip{\vskip\sectionskipamount}
\def\beforesectionskip{\vskip\beforesectionskipamount}
%%%%%%%%%%%%%%%%%%%%%%%
%%%% paper %%%%%%%%%%%%
%%%%%%%%%%%%%%%%%%%%%%%
\papwidth=16truecm
\papheight=22truecm
\voffset=0.4truecm
\hoffset=-0.4truecm
%%%%%%%install variables%%%%%%%%%%%%%%%%%%%
\hsize=\papwidth
\vsize=\papheight
%%%%%%%%%%%%%%%%%%%%%%%%%%%%%%%%%%%%%%%%%%%%%%%%%%%
%%%% paragraphs ... can be redefined later %%%%%%%%
%%%%%%%%%%%%%%%%%%%%%%%%%%%%%%%%%%%%%%%%%%%%%%%%%%%
\normalbaselineskip=5.25mm
\baselineskip=5.25mm
\parskip=0pt
\parindent=10pt
\beforesectionskipamount=36pt plus8pt minus8pt
\sectionskipamount=3pt plus1pt minus1pt
%\beforesectionskipamount=42pt plus5pt minus2pt
%\sectionskipamount=1truecm
%\overfullrule=0pt
%\hfuzz=2pt
%%%%%%%%%%%%%%%%%%%%%%%%%
%%%%%% headline %%%%%%%%%
%%%%%%%%%%%%%%%%%%%%%%%%%
\nopagenumbers
%%%%%%%%%%%%%%%%%%%%%%%%%
\newdimen\texpscorrection
\texpscorrection=0truecm %must be 0.15truecm in ps_fonts
\headline={\ifnum\pageno>1 {\hss\tenrm-\ \folio\ -\hss} \else {\hfill}\fi}
%%%%%%%%%%%%%%%%%%%%%%%%
%%%%%%% fontsizes %%%%%%
%%%%%%%%%%%%%%%%%%%%%%%%
\def\titlesize{\twelvepoint}
\def\sectionsize{\twelvepoint}
\def\subsectionsize{}
\def\titletype{\bf}
\def\sectiontype{\bf}
\def\subsectiontype{\bf}
\def\em{\sl}  %will be italic in reality
%%%%%%%%%%%%%%%%%%%%%%%%%%%%%%%%%%%%%
\nopagenumbers
\headline={\ifnum\pageno>1 {\hss\tenrm-\ \folio\ -\hss} \else {\hfill}\fi}
%
%%%%%%%%%%%%%%%%%%%%%%%%%
%%%%preloaded fonts%%%%%%
%%%%%%%%%%%%%%%%%%%%%%%%%
\font\twelverm=cmr12
\font\twelvei=cmmi12
\font\twelvesy=cmsy10 scaled\magstep1
\font\twelveex=cmex10 scaled\magstep1
\font\twelveit=cmti12
\font\twelvett=cmtt12
\font\twelvebf=cmbx12 
\font\twelvesl=cmsl12 
\font\tenmsb=msbm10
\font\sevenmsb=msbm7
\font\fivemsb=msbm5
\newfam\msbfam
\font\ninerm=cmr9
%\font\ninei=cmmi9
\font\ninesy=cmsy9
%\font\ninebf=cmbx9
%\font\ninett=cmtt9
%\font\ninesl=cmsl9
%\font\nineit=cmti9
\font\eightrm=cmr8
\font\eighti=cmmi8
\font\eightsy=cmsy8
\font\eightbf=cmbx8
\font\eighttt=cmtt8
\font\eightsl=cmsl8
\font\eightit=cmti8
\font\sixrm=cmr6
\font\sixbf=cmbx6
\font\sixi=cmmi6
\font\sixsy=cmsy6
%%%%%%%%%%%%%%%%%%%%%%%%%%%%%%%%%%%%%%%
%%% the following fonts force true %%%%
%%% computer modern behaviour      %%%%
%%% they are used to override      %%%%
%%% the ps-fonts for math symbols  %%%%
%%% like \dot \ne ....             %%%%
%%%%%%%%%%%%%%%%%%%%%%%%%%%%%%%%%%%%%%%
\newfam\truecmr 
\newfam\truecmsy
\font\twelvetruecmr=cmr10 scaled\magstep1
\font\twelvetruecmsy=cmsy10 scaled\magstep1
\font\tentruecmr=cmr10
\font\tentruecmsy=cmsy10
\font\eighttruecmr=cmr8
\font\eighttruecmsy=cmsy8
\font\seventruecmr=cmr7
\font\seventruecmsy=cmsy7
\font\sixtruecmr=cmr6
\font\sixtruecmsy=cmsy6
\font\fivetruecmr=cmr5
\font\fivetruecmsy=cmsy5
%%%% add the definitions for 10pt %%%%%%%%
\textfont\truecmr=\tentruecmr
\scriptfont\truecmr=\seventruecmr
\scriptscriptfont\truecmr=\fivetruecmr
\textfont\truecmsy=\tentruecmsy
\scriptfont\truecmsy=\seventruecmsy
\scriptscriptfont\truecmr=\fivetruecmr
\scriptscriptfont\truecmsy=\fivetruecmsy
%%%%% size changes%%%%%%
\def \eightpoint{\def\rm{\fam0\eightrm}% switch to 8-point type 
\textfont0=\eightrm \scriptfont0=\sixrm \scriptscriptfont0=\fiverm 
\textfont1=\eighti \scriptfont1=\sixi   \scriptscriptfont1=\fivei 
\textfont2=\eightsy \scriptfont2=\sixsy   \scriptscriptfont2=\fivesy 
\textfont3=\tenex \scriptfont3=\tenex   \scriptscriptfont3=\tenex 
\textfont\itfam=\eightit  \def\it{\fam\itfam\eightit}%
\textfont\slfam=\eightsl  \def\sl{\fam\slfam\eightsl}%
\textfont\ttfam=\eighttt  \def\tt{\fam\ttfam\eighttt}%
\textfont\bffam=\eightbf  \scriptfont\bffam=\sixbf
\scriptscriptfont\bffam=\fivebf  \def\bf{\fam\bffam\eightbf}%
\tt \ttglue=.5em plus.25em minus.15em 
\setbox\strutbox=\hbox{\vrule height7pt depth2pt width0pt}%
\normalbaselineskip=9pt
\let\sc=\sixrm  \let\big=\eightbig  \normalbaselines\rm
\textfont\truecmr=\eighttruecmr
\scriptfont\truecmr=\sixtruecmr
\scriptscriptfont\truecmr=\fivetruecmr
\textfont\truecmsy=\eighttruecmsy
\scriptfont\truecmsy=\sixtruecmsy
}
\catcode`@=11
\def\eightbig#1{{\hbox{$\textfont0=\ninerm\textfont2=\ninesy\left#1\vbox to6.5pt{}\right.\n@space$}}}
\catcode`@=12
%
\def \twelvepoint{\def\rm{\fam0\twelverm}% switch to 8-point type 
\textfont0=\twelverm \scriptfont0=\tenrm \scriptscriptfont0=\eightrm 
\textfont1=\twelvei \scriptfont1=\teni   \scriptscriptfont1=\eighti 
\textfont2=\twelvesy \scriptfont2=\tensy   \scriptscriptfont2=\eightsy 
\textfont3=\twelveex \scriptfont3=\tenex   \scriptscriptfont3=\tenex 
\textfont\itfam=\twelveit  \def\it{\fam\itfam\twelveit}%
\textfont\slfam=\twelvesl  \def\sl{\fam\slfam\twelvesl}%
\textfont\ttfam=\twelvett  \def\tt{\fam\ttfam\twelvett}%
\textfont\bffam=\twelvebf  \scriptfont\bffam=\tenbf
\textfont\truecmr=\twelvetruecmr
\scriptfont\truecmr=\tentruecmr
\scriptscriptfont\truecmr=\eighttruecmr
\textfont\truecmsy=\twelvetruecmsy
\scriptfont\truecmsy=\tentruecmsy
\scriptscriptfont\truecmsy=\eighttruecmsy
 \scriptscriptfont\bffam=\eightbf  \def\bf{\fam\bffam\twelvebf}%
\tt \ttglue=.5em plus.25em minus.15em 
\setbox\strutbox=\hbox{\vrule htwelve7pt depth2pt width0pt}%
\normalbaselineskip=12pt
\let\sc=\tenrm  \let\big=\twelvebig  \normalbaselines\rm
}
%
%%%%%constant subscript positions%%%%%
\fontdimen16\tensy=2.7pt
%\fontdimen13\tensy=2.7pt
\fontdimen13\tensy=4.3pt
\fontdimen17\tensy=2.7pt
\fontdimen14\tensy=4.3pt
\fontdimen18\tensy=4.3pt
\fontdimen16\eightsy=2.7pt
\fontdimen13\eightsy=4.3pt
\fontdimen17\eightsy=2.7pt
\fontdimen14\eightsy=4.3pt
\fontdimen18\eightsy=4.3pt
%
%%%%%%%%%%%%%%%%%%%%%%%%%%%%%%%%%%%%%%%%%%%%%%
%%% macros  for cross reference %%%%%%%%%%%%%%
%%%%%%%%%%%%%%%%%%%%%%%%%%%%%%%%%%%%%%%%%%%%%%
%%
%%  counters %%%
%%  
\newcount\EQNcount \EQNcount=1
\newcount\CLAIMcount \CLAIMcount=1
\newcount\SECTIONcount \SECTIONcount=0
\newcount\SUBSECTIONcount \SUBSECTIONcount=1
%%
%% defining the symbolic value
%%
\def\ifff(#1,#2,#3){\ifundefined{#1#2}
\expandafter\xdef\csname #1#2\endcsname{#3}\else
\write16{!!!!!doubly defined #1,#2}\fi}
\def\NEWDEF #1,#2,#3 {\ifff({#1},{#2},{#3})}
\def\actualnumber{\number\SECTIONcount}
\def\EQ(#1){\lmargin(#1)\eqno\tag(#1)}
\def\NR(#1){&\lmargin(#1)\tag(#1)\cr}  %the same as &\tag(xx)\cr in eqalignno
\def\tag(#1){\lmargin(#1)({\rm \actualnumber}.\number\EQNcount)
 \NEWDEF e,#1,(\actualnumber.\number\EQNcount)
\global\advance\EQNcount by 1
%\write16{ EQ \equ(#1):#1  }
}
\def\SECT(#1)#2\par{\lmargin(#1)
\SECTION #2\par
\NEWDEF s,#1,{\actualnumber}
}
\def\SUBSECT(#1)#2\par{\lmargin(#1)
\SUBSECTION #2\par
\NEWDEF s,#1,{\actualnumber.\number\SUBSECTIONcount}
}
%%%% the actual macro %%%%%%
\def\CLAIM #1(#2) #3\par{
\vskip.1in\medbreak\noindent
{\lmargin(#2)\bf #1~\actualnumber.\number\CLAIMcount.} {\sl #3}\par
\NEWDEF c,#2,{#1~\actualnumber.\number\CLAIMcount}
\global\advance\CLAIMcount by 1
\ifdim\lastskip<\medskipamount
\removelastskip\penalty55\medskip\fi}
\def\CLAIMNONR #1(#2) #3\par{
\vskip.1in\medbreak\noindent
{\lmargin(#2)\bf #1~#2} {\sl #3}\par
\global\advance\CLAIMcount by 1
\ifdim\lastskip<\medskipamount
\removelastskip\penalty55\medskip\fi}
\def\SECTION#1\par{\vskip0pt plus.3\vsize\penalty-75
    \vskip0pt plus -.3\vsize
    \global\advance\SECTIONcount by 1
    \beforesectionskip\noindent
\centerline{\sectionsize\sectiontype #1}
    \EQNcount=1
    \CLAIMcount=1
    \SUBSECTIONcount=1
    \nobreak\sectionskip\noindent}
\def\SECTIONNONR#1\par{\vskip0pt plus.3\vsize\penalty-75
    \vskip0pt plus -.3\vsize
    \global\advance\SECTIONcount by 1
    \beforesectionskip\noindent
{\sectionsize\sectiontype  #1}
     \EQNcount=1
     \CLAIMcount=1
     \SUBSECTIONcount=1
     \nobreak\sectionskip\noindent}
\def\SUBSECTION#1\par{\vskip0pt plus.2\vsize\penalty-75
    \vskip0pt plus -.2\vsize
    \beforesectionskip\noindent
{\subsectionsize\subsectiontype \actualnumber.\number\SUBSECTIONcount.\ #1}
    \global\advance\SUBSECTIONcount by 1
    \nobreak\sectionskip\noindent}
\def\SUBSECTIONNONR#1\par{\vskip0pt plus.2\vsize\penalty-75
    \vskip0pt plus -.2\vsize
\beforesectionskip\noindent
{\subsectionsize\subsectiontype #1}
    \nobreak\sectionskip\noindent\noindent}
%%
%%  referring to something
%%
\def\ifundefined#1{\expandafter\ifx\csname#1\endcsname\relax}
\def\equ(#1){\ifundefined{e#1}$\spadesuit$#1\else\csname e#1\endcsname\fi}
\def\clm(#1){\ifundefined{c#1}$\spadesuit$#1\else\csname c#1\endcsname\fi}
\def\sec(#1){\ifundefined{s#1}$\spadesuit$#1
\else Section \csname s#1\endcsname\fi}
%%%%%%%%%%%%%TITLE PAGE%%%%%%%%%%%%%%%%%%%%
\let\endarg=\par
\def\finish{\def\endarg{\par\endgroup}}
\def\start{\endarg\begingroup}
\def\getNORMAL#1{{#1}}
\def\TITLE{\beginTITLE\getTITLE}
 \def\beginTITLE{\start
   \titlesize\titletype\baselineskip=1.728
   \normalbaselineskip\rightskip=0pt plus1fil
   \noindent
   \def\endarg{\par\vskip.35in\endgroup}}
 \def\getTITLE{\getNORMAL}
\def\AUTHOR{\beginAUTHOR\getAUTHOR}
 \def\beginAUTHOR{\start
   \vskip .25in\rm\noindent\finish}
 \def\getAUTHOR{\getNORMAL}
\def\FROM{\beginFROM\getFROM}
 \def\beginFROM{\start\parskip=0pt\vskip\baselineskip
\def\finish{\def\endarg{\egroup\par\endgroup}}
  \vbox\bgroup\obeylines\eightpoint\em\finish}
 \def\getFROM{\getNORMAL}
\def\ENDTITLE{\endarg}
\def\ABSTRACT#1\par{
\vskip 1in {\noindent\sectionsize\sectiontype Abstract.} #1 \par}
\def\ENDABSTRACT{\vfill\break}
\def\TODAY{\number\day~\ifcase\month\or January \or February \or March \or
April \or May \or June
\or July \or August \or September \or October \or November \or December \fi
\number\year\timecount=\number\time
\divide\timecount by 60
}
\newcount\timecount
\def\DRAFT{\def\lmargin(##1){\strut\vadjust{\kern-\strutdepth
\vtop to \strutdepth{
\baselineskip\strutdepth\vss\rlap{\kern-1.2 truecm\eightpoint{##1}}}}}
\font\footfont=cmti7
\footline={{\footfont \hfil File:\jobname, \TODAY,  \number\timecount h}}
}
%%%subitem an item in a vbox%%%%
\newbox\strutboxJPE
\setbox\strutboxJPE=\hbox{\strut}
\def\subitem#1#2\par{\vskip\baselineskip\vskip-\ht\strutboxJPE{\item{#1}#2}}
\gdef\strutdepth{\dp\strutbox}
\def\lmargin(#1){}
%%%%%%%%%%%%%%%%BIBLIOGRAPHY%%%%%%%%%%%%%%%%%%%%
%%%%%%%%%%%%%%%%%%%%%%%%%%%%%%%%%%%%%%%%%%%%%%%%
\def\period{\unskip.\spacefactor3000 { }}
%
% ...invisible stuff
%
\newbox\noboxJPE
\newbox\byboxJPE
\newbox\paperboxJPE
\newbox\yrboxJPE
\newbox\jourboxJPE
\newbox\pagesboxJPE
\newbox\volboxJPE
\newbox\preprintboxJPE
\newbox\toappearboxJPE
\newbox\bookboxJPE
\newbox\bybookboxJPE
\newbox\publisherboxJPE
\newbox\inprintboxJPE
\def\refclearJPE{
   \setbox\noboxJPE=\null             \gdef\isnoJPE{F}
   \setbox\byboxJPE=\null             \gdef\isbyJPE{F}
   \setbox\paperboxJPE=\null          \gdef\ispaperJPE{F}
   \setbox\yrboxJPE=\null             \gdef\isyrJPE{F}
   \setbox\jourboxJPE=\null           \gdef\isjourJPE{F}
   \setbox\pagesboxJPE=\null          \gdef\ispagesJPE{F}
   \setbox\volboxJPE=\null            \gdef\isvolJPE{F}
   \setbox\preprintboxJPE=\null       \gdef\ispreprintJPE{F}
   \setbox\toappearboxJPE=\null       \gdef\istoappearJPE{F}
   \setbox\inprintboxJPE=\null        \gdef\isinprintJPE{F}
   \setbox\bookboxJPE=\null           \gdef\isbookJPE{F}  \gdef\isinbookJPE{F}
     
   \setbox\bybookboxJPE=\null         \gdef\isbybookJPE{F}
   \setbox\publisherboxJPE=\null      \gdef\ispublisherJPE{F}
     
}
\def\widestlabel#1{\setbox0=\hbox{#1\enspace}\refindent=\wd0\relax}
\def\ref{\refclearJPE\bgroup}
\def\no   {\egroup\gdef\isnoJPE{T}\setbox\noboxJPE=\hbox\bgroup}
\def\by   {\egroup\gdef\isbyJPE{T}\setbox\byboxJPE=\hbox\bgroup}
\def\paper{\egroup\gdef\ispaperJPE{T}\setbox\paperboxJPE=\hbox\bgroup}
\def\yr{\egroup\gdef\isyrJPE{T}\setbox\yrboxJPE=\hbox\bgroup}
\def\jour{\egroup\gdef\isjourJPE{T}\setbox\jourboxJPE=\hbox\bgroup}
\def\pages{\egroup\gdef\ispagesJPE{T}\setbox\pagesboxJPE=\hbox\bgroup}
\def\vol{\egroup\gdef\isvolJPE{T}\setbox\volboxJPE=\hbox\bgroup\bf}
\def\preprint{\egroup\gdef
\ispreprintJPE{T}\setbox\preprintboxJPE=\hbox\bgroup}
\def\toappear{\egroup\gdef
\istoappearJPE{T}\setbox\toappearboxJPE=\hbox\bgroup}
\def\inprint{\egroup\gdef
\isinprintJPE{T}\setbox\inprintboxJPE=\hbox\bgroup}
\def\book{\egroup\gdef\isbookJPE{T}\setbox\bookboxJPE=\hbox\bgroup\em}
\def\publisher{\egroup\gdef
\ispublisherJPE{T}\setbox\publisherboxJPE=\hbox\bgroup}
\def\inbook{\egroup\gdef\isinbookJPE{T}\setbox\bookboxJPE=\hbox\bgroup\em}
\def\bybook{\egroup\gdef\isbybookJPE{T}\setbox\bybookboxJPE=\hbox\bgroup}
\newdimen\refindent
\refindent=5em
\def\endref{\egroup \sfcode`.=1000
 \if T\isnoJPE
 \hangindent\refindent\hangafter=1
      \noindent\hbox to\refindent{[\unhbox\noboxJPE\unskip]\hss}\ignorespaces
     \else  \noindent    \fi
 \if T\isbyJPE    \unhbox\byboxJPE\unskip: \fi
 \if T\ispaperJPE \unhbox\paperboxJPE\unskip\period \fi
 \if T\isbookJPE {\it\unhbox\bookboxJPE\unskip}\if T\ispublisherJPE, \else.
\fi\fi
 \if T\isinbookJPE In {\it\unhbox\bookboxJPE\unskip}\if T\isbybookJPE,
\else\period \fi\fi
 \if T\isbybookJPE  (\unhbox\bybookboxJPE\unskip)\period \fi
 \if T\ispublisherJPE \unhbox\publisherboxJPE\unskip \if T\isjourJPE, \else\if
T\isyrJPE \  \else\period \fi\fi\fi
 \if T\istoappearJPE (To appear)\period \fi
 \if T\ispreprintJPE Pre\-print\period \fi
 \if T\isjourJPE    \unhbox\jourboxJPE\unskip\ \fi
 \if T\isvolJPE     \unhbox\volboxJPE\unskip\if T\ispagesJPE, \else\ \fi\fi
 \if T\ispagesJPE   \unhbox\pagesboxJPE\unskip\  \fi
 \if T\isyrJPE      (\unhbox\yrboxJPE\unskip)\period \fi
 \if T\isinprintJPE (in print)\period \fi
\filbreak}
\def\hexnumber#1{\ifcase#1 0\or1\or2\or3\or4\or5\or6\or7\or8\or9\or
 A\or B\or C\or D\or E\or F\fi}
\textfont\msbfam=\tenmsb
\scriptfont\msbfam=\sevenmsb
\scriptscriptfont\msbfam=\fivemsb
\mathchardef\varkappa="0\hexnumber\msbfam7B
%=============postscript=======
%figure psfile height (in cm) width (in cm) caption  (will be centered)
\newcount\FIGUREcount \FIGUREcount=0
\newskip\ttglue 
\newdimen\figcenter
\def\figure #1 #2 #3 #4\cr{\null\ifundefined{fig#1}\global
\advance\FIGUREcount by 1\NEWDEF fig,#1,{Fig.~\number\FIGUREcount}\fi
\write16{ FIG \number\FIGUREcount: #1}
{\goodbreak\figcenter=\hsize\relax
\advance\figcenter by -#3truecm
\divide\figcenter by 2
\midinsert\vskip #2truecm\noindent\hskip\figcenter
\special{psfile=#1}\vskip 0.8truecm\noindent \vbox{\eightpoint\noindent
{\bf\fig(#1)}: #4}\endinsert}}
%figurewithtex psfile texfile height (in cm) width (in cm) caption  (will be centered)
\def\figurewithtex #1 #2 #3 #4 #5\cr{\null\ifundefined{fig#1}\global
\advance\FIGUREcount by 1\NEWDEF fig,#1,{Fig.~\number\FIGUREcount}\fi
\write16{ FIG \number\FIGUREcount: #1}
{\goodbreak\figcenter=\hsize\relax
\advance\figcenter by -#4truecm
\divide\figcenter by 2
\midinsert\vskip #3truecm\noindent\hskip\figcenter
\special{psfile=#1}{\hskip\texpscorrection\input #2 }\vskip 0.8truecm\noindent \vbox{\eightpoint\noindent
{\bf\fig(#1)}: #5}\endinsert}}
\def\fig(#1){\ifundefined{fig#1}\global
\advance\FIGUREcount by 1\NEWDEF fig,#1,{Fig.~\number\FIGUREcount}
\fi
\csname fig#1\endcsname\relax}
\catcode`@=11
\def\footnote#1{\let\@sf\empty % parameter #2 (the text) is read later
  \ifhmode\edef\@sf{\spacefactor\the\spacefactor}\/\fi
  #1\@sf\vfootnote{#1}}
\def\vfootnote#1{\insert\footins\bgroup\eightpoint
  \interlinepenalty\interfootnotelinepenalty
  \splittopskip\ht\strutbox % top baseline for broken footnotes
  \splitmaxdepth\dp\strutbox \floatingpenalty\@MM
  \leftskip\z@skip \rightskip\z@skip \spaceskip\z@skip \xspaceskip\z@skip
  \textindent{#1}\footstrut\futurelet\next\fo@t}
\def\fo@t{\ifcat\bgroup\noexpand\next \let\next\f@@t
  \else\let\next\f@t\fi \next}
\def\f@@t{\bgroup\aftergroup\@foot\let\next}
\def\f@t#1{#1\@foot}
\def\@foot{\strut\egroup}
\def\footstrut{\vbox to\splittopskip{}}
\skip\footins=\bigskipamount % space added when footnote is present
\count\footins=1000 % footnote magnification factor (1 to 1)
\dimen\footins=8in % maximum footnotes per page
\catcode`@=12 % at signs are no longer letters
%%%%%%%%%%%%%%%%%%%%%%%
%%%  math symbols %%%%%
%%%%%%%%%%%%%%%%%%%%%%%
\def\HB {\hfill\break}
\def\AA{{\cal A}}
\def\BB{{\cal B}}
\def\CC{{\cal C}}
\def\EE{{\cal E}}
\def\HH{{\cal H}}
\def\LL{{\cal L}}
\def\MM{{\cal M}}
\def\NN{{\cal N}}
\def\OO{{\cal O}}
\def\PP{{\cal P}}
\def\RR{{\cal R}}
\def\TT{{\cal T}}
\def\VV{{\cal V}}
\def\HALF{{\textstyle{1\over 2}}}
%%%%%%%%%%%%%%%%%other%%%%%%%%%%%%%%%%%%%%
\def\undertext#1{$\underline{\smash{\hbox{#1}}}$}
\def\QED{\hfill\smallskip
         \line{\hfill\vrule height 1.8ex width 2ex depth +.2ex
               \ \ \ \ \ \ }
         \bigskip}
\def\real{{\bf R}}
\def\natural{{\bf N}}
\def\complex{{\bf C}}
\def\integer{{\bf Z}}
\def\Re{{\rm Re\,}}
\def\Im{{\rm Im\,}}
\def\PROOF{\medskip\noindent{\bf Proof.\ }}
\def\REMARK{\medskip\noindent{\bf Remark.\ }}
\def\LIKEREMARK#1{\medskip\noindent{\it #1---}}
%==================================
%       APOLLO specific
%===============================
\catcode`\^^?=9
\newlinechar=`\^^J

