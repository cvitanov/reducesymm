% intro1.tex  pdflatex ZhCvGo15


\section{Introduction}



The advances in the theory of dynamical systems have brought a new life to Boltzmann's mechanical formulation of statistical mechanics. Sinai, Ruelle and Bowen (SRB) have generalized Boltzmann's notion of ergodicity for a constant energy surface for a Hamiltonian system in equilibrium to dissipative systems in{nonequilibrium} stationary states. In this more general setting the attractor plays the role of a constant energy surface, and the SRB measure is a generalization of the Liouville measure. Such measures are purely microscopic and indifferent to whether the system is at equilibrium, close to equilibrium or far from it. ``Far for equilibrium'' in this context refers to systems with large deviations from Maxwell's equilibrium velocity distribution. Furthermore, the theory of dynamical systems has yielded new sets of microscopic dynamics formulas for macroscopic observables such as diffusion constants, to which we turn now.

Chaotic motions exist in many systems. There are physical problems such as beam defocusing in particle accelerators \TZ{2015-11-02}{reference?} or chaotic behavior of passive tracers in $2$\dmn\ rotating flows\rf{solomon1994chaotic} which can be described as deterministic diffusion in periodic arrays. In the macroscopic world, there are recent studies shown that robotic locomotion in heterogeneous granular environment also demonstrates scattering-diffusive pattern.

%In biological field,  many important dynamical processes (often at cellular level) are described in  terms of diffusion coefficients. Such examples include the transport of ions  across the cell membranes\rf{stein2012transport} and the movement of  microorganism(e.g. bacterials) through natural  ecosystems\rf{koch1990diffusion}. In this paper we will discuss the transport  property of more "macroscopic" systems (such as moving  robots\rf{saranli2001rhex}) where a ``diffusive description'' also applies.
%
%Lately, there has been an increased focus on robot locomotion in complex environments (check Science and ROPP reference, the systematic study of interactions between environment and locomotion, which we now call``robophysics''). Many of those studies use substrates that are spatially homogeneous and we have a good understanding\rf{li2009sensitive,li2013terradynamics}. However, little is known for locomotion in heterogeneous environment. There are some limited experimental/theoretical studies for relatively simple settings (e.g. slopes,ref). In this paper we intend to approach the longterm transport properties of locomotion in a more complex environment, i.e. in a field of scatterers of which the scales are comparable to the locomotor.
%

In \refref{LorentzDiff} some effort was made to derive a diffusion formula
for the fully symmetry-reduced dynamics, involving only
quantities computed within the fundamental domain. The
fact that lattice translations do not commute with the symmetry group within
the elementary cell makes this apparently a difficult task.


As a gedanken experiment, suppose a passively controlled robot is moving in a boulder field at constant speed. The diffusion coefficient, which describes roughly how much area the robot explored in a unit time, is the key quantity we would like to investigate. We place the boulder in a regular, periodic array and assume that we are in the heavy boulder limit such that after each collision event, only the robot is deflected and boulders remain immobilized. With the presumptions we effectively created a periodic Lorentz gas model\rf{Dettm14} for locomotion in a boulder field.

To investigate the transport property of such systems, we apply cycle expansions\rf{DasBuch} to the analysis of {\em diffusion coefficient}. The resulting formulas are exact; no probabilistic assumptions are made, and all correlations are taken into account by the  inclusion of cycles of all periods. While existing cycle expansion theory yields the correct result by tiling the {\statesp} into elementary cells, the convergence rate is slow because of bad shadowing and poor choice of symbolic grammar\rf{CGS92}. In this paper we propose a novel approach that significantly improves the efficiency of cycle expansion formula, by factorizing the non-commuting rotational and translational symmetry and using periodic orbits in the fundamental domain.

% The infinite extent systems for which the  periodic orbit theory yields formulas for diffusion and other transport  coefficients are spatially periodic, the global {\statesp} being tiled with  copies of a elementary cell.

In \refsect{s-DiffPerArr} we briefly review the formulas for diffusion coefficients in the $2$\dmn\ periodic Lorentz gas, using dynamics restricted in elementary cell. In\refsect{s-SymmetryReduction} we factorize the rotational symmetry and derive the diffusion coefficient using fundamental domain cycles. Because we will work with different kinds of \statesp, through out the text we will repeatedly using tildes ($\tilde{\quad}$), nothings and hats ($\hat{\quad}$) atop symbols to signify the dynamical quantities in the fundamental domain, elementary cell and full state space, respectively.
