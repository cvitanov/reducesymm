% POT.tex      pdflatex ZhCvGo15
% Diffuse globally, compute locally: a cyclist tale
% Tingnan Zhang, Daniel I. Goldman and Predrag Cvitanovi\'c

% \subsection{Periodic orbit theory}
% \label{s-POT}

\Po\ theory of deterministic diffusion, introduced in
\refrefs{art91,LorentzDiff}, exploits the fact that the periodic Lorentz
gas can be constructed by putting together translated copies of an
elementary cell. Therefore quantities characterizing global dynamics,
such as the Lyapunov exponents and the diffusion tensor, can be computed
from the dynamics restricted to the elementary cell, as shown numerically
in \refref{CGS92}.

However, when this elementary cell is itself invariant under a discrete
symmetry group $G$ the lattice can be tiled into images under $G$ and the
lattice translations of a fundamental domain.

In \refrefs{art91,LorentzDiff,CGS92,Artuso94,CBdiffusion} it was shown that
deterministic diffusion tensor in the {\em periodic} Lorentz gas can be
expressed in terms of (relative) \po s, and exact \cycForm\ for such
global dynamical averages as Lyapunov exponent and diffusion tensor were
derived, using only the dynamics in the elementary cell. For any
dynamical system that has translational symmetry,the full state space
$\hM$ (i.e., both spatial coordinates and momenta) has aperiodic tiling
\[ %beq
\hM=\bigcup_{ \hn \in T} \pS_{\hn},
\] %eeq
by {\em translating} $\pS_{\hn}$ of an {\em elementary cell} $\pS$, with
$T$ the abelian group of lattice translations.

In the context of Lorentz gas system, the elementary cell is the
hexagonal region centered at the scatterer, see
\reffig{fig-chaoticBouncing}\,(a). The dynamics restricted inside the
elementary cell is understood as the periodic boundary condition: when
the particle leaves the edge of the hexagon cell, it immediately enters
the region again from the opposite edge. We distinguish two types of
diffusive behavior; the {\em infinite horizon} case, which allows for
infinite length flights, and the {\em finite horizon} case, where any
free particle trajectory must hit a disk in finite time. The transition
between horizon and infinite horizon is controlled by the ratio of $w/r$,
where $w$ is the gap between nearest pair of disk and $r$ the radius of
the disk.

    \PC{2015-10-21}
    {edits based Cvitanovi\'c,  Eckmann,and Gaspard\rf{LorentzDiff}}
In \refrefs{art91,LorentzDiff,CGS92,Artuso94,CBdiffusion}  an explicit
connection between the global diffusion and the dynamics restricted to
an elementary cell.
Our method applies to any  hyperbolic dynamical system that is
a periodic tiling $\hM=\bigcup_{ \hn \in T} M_{
\hn}$
of the dynamical phase space $\hM$ by {\sl translates}
$M_{\hn}$
of an {\sl elementary cell} $M$, with $T$ the abelian group of lattice
translations.
Furthermore, each elementary cell may be built from a
{\sl fundamental domain}
$\tM$
by the action of a discrete (not necessarily Abelian) group $G$.

                                                            \toCB
Generalization to continuous time\rf{bowen,pexp} amounts to the replacement
%$ z\,=\,e^{-s} $,
$ z^{\period{p}} \rightarrow e^{-s \period{p}} $,
where $\period{p}$ is now the (not necessarily integer)
%{\sl time-}
period of the prime cycle $p$:
$$
Z(\beta,s)\,=\,\prod_{p\in\PP} \exp \left( - {
 \sum_{r=1}^\infty {1 \over r}
 { e^{(\beta \cdot \hn_p- s \period{p}) r } % z^{n_p r}
 \over { | \det \left( {\bf 1}-{\bf J}_p^{r} \right) | } }
 } \right)
\,\, .
%Eq.~(14)
$$


 As we are concerned with the long time behavior,
 this problem can be circumvented
 by replacing $ \hf^t(\tx{\tpk}) $ by the mean
 drift in the $t \rightarrow \infty$ limit.
 $ \hf^t(\tx{\tpk}) $ is a translation in $\hM$ for each
 complete cycle $p$ in $M$, so we replace
 \bea
 \hf^t(\tx{\tpk}) - \tx{\tpk}
 \,\Longrightarrow \,
 &&
 { { \hf^{m_p t}(\tx{\tpk}) - \tx{\tpk} }
 % \over         m_p
 }
    % \,\equiv \, r {\tilde n}_{\tp} (\tx{\tpk})
 \continue
t &=& r \period{\tilde{p}}, \quad m_p = \period{p}/\period{\tilde{p}} \quad \tx \in \tp \,\, ,
 \eea
 in Eq.~(240).
 The magnitude of ${\tilde n}_{\tp}(\tx{\tpk})$, the mean
 global drift per one traversal of the fundamental cycle $\tp$, is
 independent of the starting point, but its direction is not; the
 reason is that each fundamental domain cycle corresponds to a set of
 trajectories in $\hM$.
 The ${1 \over {|G|}} \sum$ average in
 Eq.~(240) then generates all distinct global drift
 directions, so we can again replace the
 sum over cycle points by the factor $\period{\tilde{p}}$, and obtain the
 $Z$ function Eq.~(14) for the $\alpha $ irreducible subspace
 $$
 Z(\beta,s)_\alpha\,=\,\prod_{\tp \in \t{\cal{P}} } \exp
 % \left( -
  % \sum_{r=1}^\infty {1 \over r}
 % {{
  % \chi_\alpha(h^r_{\tp})
  % }
 % \over
 % { | \det \left( \bf{1}-\t{\bf J}_{\tp}^{r} \right) | }
 % }
 % e^{ ( \beta \cdot {\tilde n}_{\tp} - s \period{\tilde{p}}) r}
  % \right)
 \,\, .
 Eq.~(24)
 $$
 $\hn_{\period{\tilde{p}}}(\tx_k)$
%{\tt NOnsense...}
 The leading eigenvalue of the
unsymmetrized
 operator Eq.~(8) is
 the leading eigenvalue of the symmetric subspace for which
 $\chi_\alpha(g)=1$ for all $g \in G$.


Machta and Zwanzig\rf{MacZwa83} have given numerical results
for the diffusion constant in Lorentz gases,  as well as
estimates based on a random walk approximation. We shall follow
their notation and fix the radius of the disks to 1,
assume unit particle speed, and
denote the spacing between the disks by $w$ (see fig.~1).
The horizon is finite for $w < 4/\sqrt{3}-2 = 0.3094\dots$.

We now relate the dynamics in $\pS$ to diffusive properties of the
Lorentz gas in $\hM$. Let $\hx(t)\,=\,\hflow{t}{\hx_0}$ denotes the point
in the global space $\hM$ reached by the flow in time $t$.
$x(t)\,=\,\flow{t}{\xInit}$ denotes the corresponding flow in the
elementary cell; the two are related by
\beq
\hn_t(\xInit)=\hflow{t}{\xInit} - \flow{t}{\xInit} \in T \,,
\ee{l-diff-hatn1}
the translation of the endpoint of the global path into the elementary cell $\pS$.

Fix a vector $\beta \in \reals^d$, where $d$ is the dimension of
the{\statesp}. We will compute the diffusive properties of the Lorentz
gas from the leading eigenvalue of the Rulle-Frobenius-Perron \evOper\
\beq
\eigenvL(\beta)\,=\, \lim_{t \rightarrow \infty} \frac{1}{t} \log \langle
e^{\beta \cdot (\hx(t) -x) } \rangle_\pS ~, \quad
\label{eq-diff-1}
\eeq
where the average is over all initial points in the elementary cell, $x
\in\pS$. If all odd derivatives vanish by symmetry, there is no drift and
the second derivatives
\begin{widetext}
\beq
2d D_{ij} = \left . {\frac{\partial}{\partial \beta_i}} {\frac{\partial}
{\partial \beta_j}} \eigenvL(\beta)\right\vert_{\beta=0} \,=\,\lim_{t\rightarrow
\infty} {\frac{1}{t}} \langle {(\hx(t) -x)_i (\hx(t) -x)_j } \rangle_\pS ~,
\eeq
\end{widetext}
yield a diffusion matrix.  This symmetric matrix can, in general, be
anisotropic(\ie, have $d$ distinct eigenvalues and eigen\-vectors). The
spatial diffusion constant is then given by the Einstein relation
\beq
D\,=\,{1\over 2 d} \sum_i \left .{{\partial}^2 \over {\partial
      \beta^2_i}} \eigenvL(\beta)\right |_{\beta=0} \,=\,
\lim_{t\rightarrow \infty} {1\over{2d t}} \langle {(\hat{q}(t) -q)^2 }
\rangle_\pS~ ~,
\eeq
where the $i$ sum is restricted to the spatial components $q_i$ of
the{\statesp} vectors $x=(q,p)$, \ie, if the dynamics is Hamiltonian, the
sum is over the $d$ degrees of freedom.
\PC{2014-11-18}{reinstate mass, velocity, size to get $\beta$, $m$, $\sigma$
    dependencies right?}

We now turn to the connection between diffusion and periodic orbits in
the elementary cell. It was shown in \refref{CGS92} that the ensemble
average in\refeq{eq-diff-1} can be written as an integral over the
elementary cell
\beq
\langle e^{\beta\cdot(\hx(t)-x)} \rangle
   = \frac{1}{\vert \pS \vert}\int_{x,y\in \pS} dxdy {\cal L}^t(y,x),
\eeq
given the linear \evOper
\beq
{\cal L}^t(y,x) = e^{\beta\cdot(\hx(t)-x)}\delta(y-x(t))
\label{eq-eOper}
\eeq
The interesting dynamical averages is determined by the spectrum of the operator
\beq \det(\eigenvL - \Lop) \,=\,\prod_{p} \exp \left(
  - { \sum_{r=1}^\infty {1 \over r} { e^{(\beta \cdot \hn_p- s
        \period{p}) r} \over \oneMinJ{r} }
  } \right) \,,
\ee{lor-diff-14}
or the corresponding \dzeta\
\beq
1/\zeta(\beta, s)\,=\,\prod_{p}\left( 1 - \frac{e^{(\beta \cdot \hn_p-
      s \period{p})}}{|\ExpaEig_p|} \right) ~,
\label{zeta-diff}
\eeq
where $\period{p}$ is the period of the cycle and $\ExpaEig_p$ the
product of expanding eigenvalues of the cycle\rf{DasBuch}.

The \dzeta\ \cycForm\ for the diffusion constant, zero mean drift
$ \expct{ \hat{x}_i } = 0 \,, $ is given by
 \beq D \,=\,{1 \over 2 d}
{ \expct{\hat{x}^2}_\zeta \over \expct{\period{}}_\zeta } \,=\,{1
  \over 2 d } \, {1 \over \expct{\period{}}_\zeta} \sumprime
\frac{(-1)^{k+1} (\hn_{p_1}+ \cdots+ \hn_{p_k})^2}
{|\ExpaEig_{p_1}\cdots \ExpaEig_{p_k}|} \, ,
\label{eq-ecDiffCoef}
\eeq
where the sum is over all distinct non-repeating combination of prime
cycles (in the elementary cell). The derivation is standard, still the
formula is strange.Diffusion is unbounded motion across an infinite
lattice; nevertheless, the reduction to the elementary cell enables us to
compute relevant quantities in the usual way, in terms of periodic
orbits.
