% POT.tex      pdflatex ZhCvGo15
% Diffuse globally, compute locally: a cyclist tale
% Tingnan Zhang, Daniel I. Goldman and Predrag Cvitanovi\'c

% \subsection{Periodic orbit theory}
% \label{s-POT}

\Po\ theory of deterministic diffusion, introduced in
\refrefs{art91,LorentzDiff}, exploits the fact that the periodic
Lorentz gas can be constructed by putting together translated copies
of an elementary cell.  Therefore quantities characterizing global
dynamics, such as the Lyapunov exponents and the diffusion tensor, can
be computed from the dynamics restricted to the elementary cell, as
shown numerically in \refref{CGS92}.

In \refrefs{art91,LorentzDiff,CGS92,Artuso94,CBdiffusion} it was shown
that deterministic diffusion tensor in the {\em periodic} Lorentz gas
can be expressed in terms of (relative) \po s, and exact \cycForm\ for
such global dynamical averages as Lyapunov exponent and diffusion
tensor were derived, using only the dynamics in the \emph{elementary
cell}, which we discuss now. For any dynamical system that has
translational symmetry, the full state space $\hM$ (i.e., both spatial
coordinates and momenta) has a periodic tiling
\[ %beq
\hM=\bigcup_{ \hn \in T} \pS_{\hn},
\] %eeq
by {\em translating} $\pS_{\hn}$ of an {\em elementary cell} $\pS$,
with $T$ the abelian group of lattice translations.

\begin{figure}[htbp]
	\begin{center}
    \includegraphics[width=0.25\textwidth]{diffuseLorentzGasParams}
	\end{center}
	\caption[]{\label{fig-LorentzGasParams}
        An elementary cell and its six unit translations. The ratio of
        distance $w$ between the nearest pair of disks to the    disk
        radius $r$ determines the dynamical properties in the system.
	}
	\TZ{2016-01-15}{This figure is also generated by me.}
\end{figure}

In the context of triangular Lorentz gas system, the elementary cell
is the hexagon centered at the scatterer and the translation applys
only to the spatial degrees of freedom, see
\reffig{fig-LorentzGasParams}. The dynamics restricted inside the
elementary cell is understood as the periodic boundary condition: when
the particle leaves the edge of the hexagon cell, it immediately
enters the region again from the opposite edge. The transition between
finite and and infinite horizon is controlled by the ratio of $w/r$,
where $w$ is the gap between nearest pair of disk and $r$ the radius
of the disk. The horizon is finite for $w/r < 4/\sqrt{3}-2 =
0.3094\dots$.

    \PC{2013-02-03} { Roberto says we must incorporate kneading
    determinants from Cristadoro\rf{ArtCri03,Cristad06,CriKnDeEsp12}.
    }

%    \PC{2015-10-21}
%    {edits based Cvitanovi\'c,  Eckmann,and Gaspard\rf{LorentzDiff}}
%In \refrefs{art91,LorentzDiff,CGS92,Artuso94,CBdiffusion}  an explicit
%connection between the global diffusion and the dynamics restricted to
%an elementary cell.
%Our method applies to any  hyperbolic dynamical system that is
%a periodic tiling $\hM=\bigcup_{ \hn \in T} M_{
%\hn}$
%of the dynamical phase space $\hM$ by {\sl translates}
%$M_{\hn}$
%of an {\sl elementary cell} $M$, with $T$ the abelian group of lattice
%translations.
%Furthermore, each elementary cell may be built from a
%{\sl fundamental domain}
%$\tM$
%by the action of a discrete (not necessarily Abelian) group $G$.

\PC{2014-11-18}{reinstate mass, velocity, size to get $\beta$, $m$,
$\sigma$ dependencies right?}

Previous works by Machta and Zwanzig\rf{MacZwa83} have given numerical
results for the diffusion constant in Lorentz gases,  as well as
estimates based on a random walk approximation. We shall follow their
notation and fix the radius of the disks to 1, and assume unit
particle speed.

Let $\hx(t)\,=\,\hflow{t}{\hx_0}$ denotes the point in the full 
space $\hM$ reached by the flow in time $t$.
$x(t)\,=\,\flow{t}{\xInit}$ denotes the corresponding flow in the
elementary cell; the two are related by
\beq
\hn_t(\xInit)=\hflow{t}{\xInit} - \flow{t}{\xInit} \in T \,,
\ee{l-diff-hatn1}
the translation of the endpoint of the global path into the elementary
cell $\pS$. The diffusion tensor, by definition, is the temporal and
ensemble average of the displacement: 
\beq
D_{ij} =
\lim_{t\to\infty}\frac{1}{2dt}\left\langle\hn_t(\xInit)_i\hn_t(\xInit)_j\right\rangle_{\hM}\,,
\label{eq-diff-def}
\eeq
where the index $i$ and $j$ are restricted to the spatial components
$q_i$ of the {\statesp} vectors $x=(q,p)$, \ie, if the dynamics is
Hamiltonian, the sum is over the $d$ degrees of freedom. 

Following the general definition \refeq{eq-diff-def}, one can
numerically compute the diffusion coeffcient for various systems
(which is not limited to the Lorentz gas). However, little insight is
gained in the dynamics of the system. Instead,
\cite{art91,LorentzDiff,CGS92,Artuso94,CBdiffusion} suggest we study
the quantity 
\beq
Q(\beta)\,=\, \lim_{t \rightarrow \infty} \frac{1}{t} \log
\langle e^{\beta \cdot \hn_t(x)} \rangle_{\hM} ~, \quad
\label{eq-diff-1}
\eeq
where $\beta$ is an auxiliary vector quantity. The interesting
dynamical averages such like mean drift and diffusion can be easily
obtained by taking derivatives of $Q(\beta)$ and set $\beta =
0$, e.g.: 
\bea
2d D_{ij} &=& \left . {\frac{\partial}{\partial \beta_i}} {\frac{\partial}
{\partial \beta_j}} Q(\beta)\right\vert_{\beta=0}\\\nonumber
&=&\lim_{t\rightarrow
\infty} {\frac{1}{t}} \langle {\hn_t(x)_i \hn_t(x)_j } \rangle_{\hM} \,,
\eea
yields a diffusion matrix.  The diffusion tensor matrix can, in
general, be anisotropic (\ie, have $d$ distinct eigenvalues and
eigen\-vectors). The spatial diffusion constant is then given by the
Einstein relation
\beq
D\,=\,{1\over 2 d} \sum_i^d \left .{{\partial}^2 \over {\partial
\beta^2_i}} Q(\beta)\right |_{\beta=0} \,=\, \lim_{t\rightarrow
\infty} {1\over{2d t}} \langle {(\hat{q}(t) -q)^2 } \rangle_{\hM}~ ~,
\eeq

Because of the translational invariance, it was shown in \refref{CGS92}
that the ensemble average in~\refeq{eq-diff-1} can be written as an
integral over the elementary cell
\beq
\langle e^{\beta\cdot(\hx(t)-x)} \rangle
   = \frac{1}{\vert \pS \vert}\int_{x,y\in \pS} dxdy {\cal L}^t(y,x),
\eeq
given the Ruelle-Perron-Frobenius \evOper
\beq
{\cal L}^t(y,x) = e^{\beta\cdot(\hx(t)-x)}\delta(y-x(t))\,,
\label{eq-evo-flow}
\eeq
for the flow. It is a linear operator such that it satisfies the
semi-group property: 
    
We may also introduce the same operator for discrete maps, after we
have chosen the appropriate Poincar\'e section and study the dynamics
restricted to the intersections of the flow on the hyper-surface:
\beq
{\cal L}^n(y,x) = e^{\beta\cdot(\hat{f}^n(x)-x)}\delta(y-f^n(x))\,.
\label{eq-evo-map}
\eeq 

For the system under consideration, a convenient choice of Poincar\'e
section is the edge of the disk--due to finite horizon, the particle
will collide with a disk in finite time and the poincare map is truely
a ``return map''. Once we have specified the arch length $\phi$ at
which the collision takes place on the disk and the incident angle
$\psi$ (which gives the tangent velocity component at collision), the
dynamics is fully determined. There are two other  marginal directions
along which the dynamics is trivial, because the system is
Hamiltonian.

For sake of simplicity, we will proceed with the derivation for the
return map and in the end generalize to the continuous flow.

The dynamical averages we are interested does not belong to a
particular finite time (or a finite number of returns). As
$t\to\infty$ (or $n\to\infty$ for the map), the average quantities are
dominated by the leading eigenvalue of ${\cal L}^n$, $\lambda_0 =
e^{\eigenvL(\beta)n}$. While the evolution operator
\refeq{eq-evo-flow}\refeq{eq-evo-map} act on a functional space of
infinite dimension, their spectrum may still be extracted by the trace
formula:
\beq
\det(1-z{\cal L}) =
\exp\left(-\sum_{n=0}^{\infty}\frac{z^n}{n}\tr{\cal L}^n\right)\,,
\label{eq-det-disci}
\eeq
where $z$ is an auxiliary variable, and the trace:
\beq
\Tr{{\cal L}^n} = \int_{\pS}dx ~
e^{\beta\cdot(\hat{f}^n(x)-x)}\delta(x-f^n(x))\,.
\label{eq-trace-disc}
\eeq

With the $\delta$ functon inside, Eq.~\refeq{eq-diff-trace} picks up a
contribution whenever $x = f^n(x)$, e.g. when $x$ is the fixed point
of the map $f^n$. If $n$ is not a prime number, it is possible that
$x$ belongs to a periodic orbit $p$ of $f$ of shorter topological
length $n_p$, and satisfies the muliplicity $n_p r = n$, where $r$ is
the repeat number.

We may now restrict the integral only in the vicinites of the prime
periodic orbts (i.e. those who are not the repeat of other periodic
orbits) of the return map, and write:
\beq
\Tr{{\cal L}^n} = \sum_p\delta_{n,
n_pr}\sum_{x\in p}\frac{e^{r\beta\cdot\hat{n}_p(x)}}{\vert\det\left({\bf 1 -
J}_p^{r}(x)\right)\vert}\,,
\label{eq-trace-expan}
\eeq 
where ${\bf J}_p(x) = Df^{n_p}(x)$ is the Jacobian.  Both ${\bf
J}_p(x) = {\bf J}_p $ and $\hat{n}_p(x) = \hat{n}_p$ are independent
of the point $x$ on the cycle. For the jacobian this follows by the
chain rule of differentiation. For the displacement it is because the
translations are commutative: no matter where the particle starts on
the cycle it will go through the same set of free flights and
translate the same amount in total after finishing the cycle. 

Summing the terms in the determinant \refeq{eq-det-disc}, we finally
obtain:
\beq
\det(1-z{\cal L}) = \prod_p exp\left(-\sum_{r =
1}^{\infty}\frac{z^{n_p r}}{r}
\frac{e^{r\beta\cdot\hat{n}_p}}{\vert\det\left({\bf 1 -
J}_p^{r}\right)\vert}\right)\,.
\eeq
where the product runs over all prime periodic cycles.

Generalization to continuous time\rf{bowen,pexp} amounts to the
replacement
%$ z\,=\,e^{-s} $,
$ z^{n_p r} \rightarrow e^{-s \period{p}} $, where $\period{p}$
is now the (not necessarily integer)
%{\sl time-}
period of the prime cycle $p$:
$$
Z(\beta,s)\,=\,\prod_{p\in\PP} \exp \left( - {
 \sum_{r=1}^\infty {1 \over r}
 { e^{(\beta \cdot \hn_p- s \period{p}) r } % z^{n_p r}
 \over { | \det \left( {\bf 1}-{\bf J}_p^{r} \right) | } }
 } \right)
\,\, .
$$

\beq
1/\zeta(\beta, s)\,=\,\prod_{p}\left( 1 - \frac{e^{(\beta \cdot \hn_p-
      s \period{p})}}{|\ExpaEig_p|} \right) ~,
\label{zeta-diff}
\eeq
where $\ExpaEig_p$ the product of expanding eigenvalues of the
cycle\rf{DasBuch}.

The \dzeta\ \cycForm\ for the diffusion constant, zero mean drift
$ \expct{ \hat{x}_i } = 0 \,, $ is given by
 \beq 
 D \,=\,{1 \over 2 d} { \expct{\hat{x}^2}_\zeta \over
 \expct{\period{}}_\zeta } \,=\,{1 \over 2 d } \, {1 \over
 \expct{\period{}}_\zeta} \sumprime \frac{(-1)^{k+1} (\hn_{p_1}+
 \cdots+ \hn_{p_k})^2} {|\ExpaEig_{p_1}\cdots \ExpaEig_{p_k}|} \, ,
\label{eq-ecDiffCoef}
\eeq
where the sum is over all distinct non-repeating combination of prime
cycles (in the elementary cell). The derivation is standard, still the
formula is strange. Diffusion is unbounded motion across an infinite
lattice; nevertheless, the reduction to the elementary cell enables us
to compute relevant quantities in the usual way, in terms of periodic
orbits.
