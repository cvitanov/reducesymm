% concl.tex     pdflatex ZhCvGo15
% Diffuse globally, compute locally: a cyclist tale
% Tingnan Zhang, Daniel I. Goldman and Predrag Cvitanovi\'c

%\section{Conclusion}
%\label{s-concl}

% \TZ{2015-11-02}{??}

We have thus obtained a description of global diffusive properties of an
infinite periodic dynamical system, such as the Lorentz gas, in terms of
periodic orbits restricted to the elementary cell. These formulas have
been tested extensively in \refrefs{CGS92,BaEvCo93} on the Lorentz gas,
and in \refref{art91} on 1-dimensional mappings. Related trace formulas
have been independently introduced and tested numerically in
\refref{Vance92}. The formalism has been generalized to evaluation of
power spectra of chaotic time series in \refref{CviPik93}.

Honesty in advertising requires disclaimer; no such fractal behavior of
the conductance has been detected experimentally so far.

In practice, the periodic orbit evaluations
of the diffusion constant converge poorly compared with averages over scalar
quantities such as the Lyapunov exponents.  These difficulties are due to
several reasons:

(1) the diffusion coefficient is not a mean but a variance which is
always more difficult to evaluate than mean quantities like Lyapunov
exponents;

(2) until this work, there was no formula for the diffusion coefficient in
terms of the periodic orbits of the fundamental domain;

(3) systems like the Lorentz gas do not have simple symbolic dynamics and
the analyticity of the associated zeta functions may also be affected by
the flow discontinuities associated with the grazing trajectories
(trajectories tangent to the disks).
