% fundDiff.tex      pdflatex ZhCvGo15
% Diffuse globally, compute locally: a cyclist tale
% Tingnan Zhang, Daniel I. Goldman and Predrag Cvitanovi\'c

%\subsection{Fundamental domain diffusion tensor}
%\label{s-fundDiff}

While the fundamental domain blah,

Earlier work\rf{LorentzDiff} has failed to obtain a formula for the
diffusion matrix expressed in terms of the fundamental domain cycles.
For scalar quantities $A(\tx)$ such like the Lyapunovs we can easily
formulate a evolution operator similar to \refeq{eq-evo-flow}, using
the dynamics expressed in the fundamental domain:
\beq
\tilde{{\cal L}}^t(\tilde{y},\tilde{x}) = 
e^{\beta (A(\tilde{y})-A(\tx)}\delta(\tilde{y}-\tx(t)))\,.
\eeq
However, the operator will no longer remain linear for vector 
quantities like the displacement when $\beta \neq 0$, for reasons 
discussed in \refsect{s-FundTranslation}. 
    

Instead, we follows the approach proposed in \rf{} and factorize the 
spectral determinant \refeq{} according to the little group's 
symmetry. blah

We start from the trace trace of elementary cell \evOper\ 
\refeq{eq-trace-disc} and project it into the irreducible 
representations $\II_G$ of the discrete group:
\bea
\tr{\cal L}^n &=& \sum_{\alpha \in\II_G} \tr{\cal
L}_{\alpha}^n\nonumber\\ \tr{\cal L}_{\alpha}^{n} &=&
\frac{d_\alpha}{|G|}\sum_{h \in
G}\chi_\alpha(h)\times\nonumber\\&&\int_{\t {\cal M}} d\tx
\sum_{\sigma \in G}\delta (h\tx -
f^n(\sigma\tx))e^{\beta\cdot\left(\hat{f}^n(\sigma\tx)-\sigma\tx\right)}\,.\nonumber\\
\label{eq-trace-ir}
\eea
Again, we first derive the map version and will discuss the continuous
flow later. Using the equivariance relations
\refeq{eq-equivariance-flow} and \refeq{eq-equivariance-disp}, and
noticing that the little group is closed, we swap the order of the
double sum and replacing $h\to \sigma^{-1} h$:

\bea
\tr{\cal L}_{\alpha}^{n} &=& \frac{d_\alpha}{|G|}\sum_{\sigma \in
  G}\sum_{h \in G}\chi_\alpha(h)\times
  \nonumber\\
  &&\int_{\t {\cal M}} d\tx \delta (h\tx -
f^n(\tx))e^{\beta\cdot\sigma\cdot(\hat{f}^n(\tx)-\tx)}\,.
\label{eq-trace-ir-disc}
\eea

Similar to the analysis in \refsect{s-POT}, the $\delta$-function part
$\delta (h\tx - f^n(\tx))$ in the integral selects only those points
$\tx$ in ${\cal M}$ which belong to the relative periodic orbit $\tp$
in the fundamental domain which satisfies the group condition
$h^{r}_{\tp}(\tx) = h$, where $r$ is the repeat number $r n_\tp = n$.
It follows that the integral in \refeq{eq-trace-ir-disc} takes the
form

\beq
\int_{\t {\cal M}} = \sum_{\tp}\delta_{n, n_pr}\sum_{\tx\in
\tp}\delta_{h^r_{\tp}(\tx), h}
\frac{e^{\beta\cdot\sigma\cdot\hat{L}_\tp(r,
\tx)}}{\vert\det\left({\bf 1 - J}_\tp^r(\tx)\right)\vert}\,.
\label{eq-trace-ir-expan}
\eeq 
The displacement $\hat{L}_\tp(r, \tx)$ along the orbit $r$ 
times takes the form already computed  in~\refeq{eq-fdDisplacement}. 
The Jacobian blah
The spectral determinant for the $\alpha-$irreducible representation 
is:

 \beq
 Z_{\alpha}(\beta,z)
=\exp\left(-\frac{d_\alpha}{|G|}\sum_{\sigma\in G}\sum_{\tp}
    \frac{1}{\cl{\tp}}\sum_{\tx\in\tp}\sum_{r=1}^{\infty}
    \frac{t_\tp(r, \tx)}{r}
    \right)\,,
\label{eq-fd-zeta}
\eeq
where
\beq
    t_{\tp} =\frac {z^{n_\tp
    r}{\chi_{\alpha}(\hp^{r}(\tx))e^{\beta\cdot\sigma\cdot\hat{L}_{\tp}(r,\tx)}}}{\vert\det\left({\bf
     1 - J}_\tp^r(\tx)\right)\vert }
\,,
\eeq
is the weight associated with the orbit. Equation \refeq{eq-fd-zeta}
takes a much more complicated form than its counterpart in the
elementary cell. As a sanity check, if the little group contains only
the identity element $e$, i.e. there is no additional symmetry. it is
reduced to \refeq{eq-det-disc}. Generalization to continuous flow is
also different. Besides replacing $z^{n_p}\to e^{-s\period{\tp}}$, the
summation is changed to an integral
\beq
\frac{1}{n_{\tp}}\sum_{\tx\in\tp}\to 
\frac{1}{\period{\tp}}\oint_{\cal P}d\tau\tx(\tau)\,,
\eeq
which runs along the tangent (marginal) direction of the orbit. Here 
is the interesting discovery, changing the sum to 
the integral will produce \emph{different} results for any 
non-trivial symmetry, which we will discuss later.

The important long term dynamics is contained in the one dimensional, 
symmetric representation: $ d_\alpha = 1 $ and all $ \chi(h) = 1 
$; there by we drop the subscript $\alpha $ in the following 
calculation. The cycle average quantities:

%\bea
%\frac{\partial^{2}}{\partial\beta^{2}}\frac{1}{\zeta(\beta,s,z)}
%&=\frac{1}{\zeta(\beta,s,z)}\left(\left(\frac{1}{|G|}
%\sum_{\sigma\in G}\sum_{\tp}\sum_{\tx_i\in \tp}
%\sum_{r=1}^{\infty}\frac{\sigma\cdot 
%\hat{L}_{\tp}^{r}(\tx_i)t_{\tp}^r
%e^{\beta\cdot\sigma\cdot 
%\hat{L}_{\tp}^{r}(\tx_i)}}{\cl{\tp}r}\right)^{2}
%    \right.
%    \nonumber\\
%&\left.-\frac{1}{|G|}\sum_{\sigma\in G}\left(\sum_{\tp}\sum_{\tx_i\in
%      \tp}\sum_{r=1}^{\infty}\frac{\vert \sigma\cdot
%      
%\hat{L}_{\tp}^{r}(\tx_i)\vert^{2}t_{\tp}^{r}e^{\beta\cdot\sigma\cdot
%        \hat{L}_{\tp}^{r}(\tx_i)}}{\cl{\tp}r}\right)\right).
%\eea
The first term in the formula corresponds to $ \langle\hx\rangle^2 $ and
second to $ \langle\hx^2\rangle $. It is trivial to see that
\beq\sum_{\sigma\in G}\frac{\sigma\cdot
  \hat{L}_{\tp}^r(\tx_i)t_{\tp}^r e^{\beta\cdot\sigma\cdot
  \hat{L}_{\tp}(r,\tx_i)}}{\cl{\tp}r} \equiv 0,
\eeq
because of the summation over the discrete group $\Group$. Thus the
calculated mean drift is zero, consistent with the symmetry of the
system. Observing that the length$\vert \sigma\cdot
\hat{L}_{\tp}^{r}(\tx_i) \vert$ does not change under rotation, we
write
\bea
\langle\hx^2\rangle &=& \left.\frac{1}{\zeta(\beta,s,z)}\sum_{\tp}
\sum_{r=1}^{\infty}\frac{t_{\tp}^{r}}{r} \sum_{\tx_i\in
\tp}\frac{\vert\hat{L}_{\tp}^{r}(\tx_i)\vert^{2}}{\cl{\tp}}
\right\vert_{\beta=0,s=0, z=1} \nonumber\\ &=&
\left.\prod_{\tp}\left(1-\frac{z^{\cl{\tp}}}{\vert\ExpaEig_\tp\vert
}\right)\sum_{\tp}\sum_{r=1}^{\infty}\left(\frac{z^{\cl{\tp}}}{\vert\ExpaEig_\tp\vert
 }\right)^r\frac{\vert\hat{L}_{\tp}^{r}\vert^2}{r}\right\vert_{z=1}
\label{eq-meanSquareDisp}
\eea

with
\[
\vert\hat{L}_{\tp}^{r}\vert^2\equiv\sum_{\tx_i\in
  \tp}\frac{\vert\hat{L}_{\tp}^{r}(\tx_i)\vert^{2}}{\cl{\tp}}
\]
the average square displacement in full space when traveling along a fundamental
domain $r$ times. Formula \refeq{eq-meanSquareDisp} is an infinite polynomial in
the auxiliary variable $z$, and should be truncated to the topological length of
the longest periodic orbits find in calculation.
