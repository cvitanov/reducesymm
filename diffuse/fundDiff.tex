% fundDiff.tex      pdflatex ZhCvGo15
% Diffuse globally, compute locally: a cyclist tale
% Tingnan Zhang, Daniel I. Goldman and Predrag Cvitanovi\'c

%\subsection{Fundamental domain diffusion tensor}
%\label{s-fundDiff}

While discrete symmetries \Fd\ factorization works well for
scalar observables, 
    earlier work\rf{LorentzDiff} has failed to obtain a formula
    for the diffusion matrix expressed in terms of the fundamental
    domain cycles. We shall now explain here
why the usual cycle weights do not work for vector and tensorial quantities 
such as displacement, and then propose a generalization that takes care of that.    
    


We start from the trace trace of \evOper\ \refeq{eq-eOper} and project it
into the individual representations of the discrete group:
\bea
\tr{\cal L}^t &=& \sum_{\alpha \in\II_G} \tr{\cal L}_{\alpha}^t\nonumber\\
\tr{\cal L}_{\alpha}^{t} &=& \frac{d_\alpha}{|G|}\sum_{h \in
  G}\chi_\alpha(h)\int_{\t {\cal M}} d\tx \sum_{\sigma \in G}\delta (h\tx -
\flow{t}{\sigma\tx})e^{\beta\cdot\hn^t(\sigma\tx)}\,,\nonumber\\
\label{eq-traceSum}
\eea
where the double summation are all over the point group. Using the G-equivariance of the flow~\refeq{eq-equivariance-flow} and displacement~\refeq{eq-equivariance-disp}, and noticing that the little group is closed, we can swap the order of the double sum and replacing $h\to \sigma^{-1} h$:

\[
\tr{\cal L}_{\alpha}^{t} = \frac{d_\alpha}{|G|}\sum_{\sigma \in
  G}\sum_{h \in G}\chi_\alpha(h)\int_{\t {\cal M}} d\tx \delta (h\tx -
\flow{t}{\tx})e^{\beta\cdot\sigma\cdot\hn^t(\tx)}\,.
\]


The $\delta$-function part $\delta (h\tx - \flow{t}{\tx})$ in the integral selects only those points $\tx$ in ${\cal M}$ that belong to the relative periodic orbit and satisfy the group condition $h^t(\tx) = h$. 

The displacement traveled starting from each of those points and along the orbit
$r$ times takes the form already computed in\refeq{eq-fdDisplacement}. The rest
is straight forward gymnastics of algebra,which yields the dynamical zeta
function for the $\alpha$ irreducible representation:
\begin{widetext}
 \beq
\frac{1}{\zeta_{\alpha}(\beta,s,z)}
=\exp\left(-\frac{d_\alpha}{|G|}\sum_{\sigma\in G}\sum_{\tp}
    \frac{1}{\cl{\tp}}\sum_{\tx_{i}\in\tp}\sum_{r=1}^{\infty}
    \frac{t_{\tp}^{r}}{r}
    \chi_{\alpha}(\hp^{r}(\tx_i))e^{\beta\cdot\sigma\cdot\hat{L}_{\tp}(r,\tx_i)}
    \right),
\label{eq-fdZeta}
\eeq
\end{widetext}

where
\[
  t_{\tp}\equiv
\frac{z^{\cl{\tp}}e^{-sT_{\tp}}}{|\ExpaEig_\tp|}
\,,
\]
is the weight associated to the orbit. Equation \refeq{eq-fdZeta}
differs from its counterpart in elementary cell, but can be reduced to if
the symmetry group contains only $e$.

We are interested in the one dimensional, symmetric trivial
representation with$ d_\alpha = 1 $ and all $ \chi(h) = 1 $; there by we
drop the subscript $\alpha $ in the following calculation. Partial
derivative with respect to$\beta$ gives:
\begin{widetext}
\bea
\frac{\partial^{2}}{\partial\beta^{2}}\frac{1}{\zeta(\beta,s,z)}
&=\frac{1}{\zeta(\beta,s,z)}\left(\left(\frac{1}{|G|}
\sum_{\sigma\in G}\sum_{\tp}\sum_{\tx_i\in \tp}
\sum_{r=1}^{\infty}\frac{\sigma\cdot \hat{L}_{\tp}^{r}(\tx_i)t_{\tp}^r
e^{\beta\cdot\sigma\cdot \hat{L}_{\tp}^{r}(\tx_i)}}{\cl{\tp}r}\right)^{2}
    \right.
    \nonumber\\
&\left.-\frac{1}{|G|}\sum_{\sigma\in G}\left(\sum_{\tp}\sum_{\tx_i\in
      \tp}\sum_{r=1}^{\infty}\frac{\vert \sigma\cdot
      \hat{L}_{\tp}^{r}(\tx_i)\vert^{2}t_{\tp}^{r}e^{\beta\cdot\sigma\cdot
        \hat{L}_{\tp}^{r}(\tx_i)}}{\cl{\tp}r}\right)\right).
        \eea
\end{widetext}
The first term in the formula corresponds to $ \langle\hx\rangle^2 $ and
second to $ \langle\hx^2\rangle $. It is trivial to see that
\beq\sum_{\sigma\in G}\frac{\sigma\cdot
  \hat{L}_{\tp}^r(\tx_i)t_{\tp}^r e^{\beta\cdot\sigma\cdot
  \hat{L}_{\tp}(r,\tx_i)}}{\cl{\tp}r} \equiv 0,
\eeq
because of the summation over the discrete group $\Group$. Thus the calculated
mean drift is zero, consistent with the symmetry of the system. Observing
that the length$\vert \sigma\cdot \hat{L}_{\tp}^{r}(\tx_i) \vert$ does
not change under rotation, we write
\bea
\langle\hx^2\rangle &=& \left.\frac{1}{\zeta(\beta,s,z)}\sum_{\tp}
\sum_{r=1}^{\infty}\frac{t_{\tp}^{r}}{r}
\sum_{\tx_i\in \tp}\frac{\vert\hat{L}_{\tp}^{r}(\tx_i)\vert^{2}}{\cl{\tp}}
    \right\vert_{\beta=0,s=0, z=1}
\nonumber\\
&=& \left.\prod_{\tp}\left(1-\frac{z^{\cl{\tp}}}{\vert\ExpaEig_\tp\vert
    }\right)\sum_{\tp}\sum_{r=1}^{\infty}\left(\frac{z^{\cl{\tp}}}{\vert\ExpaEig_\tp\vert
    }\right)^r\frac{\vert\hat{L}_{\tp}^{r}\vert^2}{r}\right\vert_{z=1}
\label{eq-meanSquareDisp}
\eea

with
\[
\vert\hat{L}_{\tp}^{r}\vert^2\equiv\sum_{\tx_i\in
  \tp}\frac{\vert\hat{L}_{\tp}^{r}(\tx_i)\vert^{2}}{\cl{\tp}}
\]
the average square displacement in full space when traveling along a fundamental
domain $r$ times. Formula \refeq{eq-meanSquareDisp} is an infinite polynomial in
the auxiliary variable $z$, and should be truncated to the topological length of
the longest periodic orbits find in calculation.
