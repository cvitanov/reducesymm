
  \section{Remarks}

remark === {Lorentz gas.}{
The original pinball model proposed by Lorentz\rf{Lorentz1905} consisted
of randomly, rather than regularly placed scatterers.
\index{Lorentz gas}
} %end remark === {Original model}{
%

remark === {Who's dunnit?}{
Cycle expansions for the diffusion constant of a particle
moving in a periodic array have been introduced
%in May-June 1991
by R.~Artuso\rf{art91} (exact
\dzeta\ for $1$\dmn\ chains of maps  \refeq{(17)}),
by W.N. Vance\rf{Vance92}%
\PublicPrivate{,}{% switch \PublicPrivate{
(the trace formula \refeq{DL-trace} for the Lorentz gas),
      }% end \PublicPrivate{
and by
P. Cvitanovi\'c, J.-P. Eckmann, and P. Gaspard\rf{LorentzDiff}
(the \dzeta\ cycle expansion \refeq{(17)} applied to the  Lorentz gas).
%Vance refers to the Artuso article and Gaspard-Baras\rf{GB92},
%so perhaps his only independent contribution was to apply the
%method to the Lorentz gas.
} %end remark === {Who's dunne it?}{
\PC{add Schreiber ref.; write about Gallavotti-Cohen}

remark === {Lack of structural stability for D.}{
Expressions like \refeq{D-KD} may lead to an expectation that
the diffusion coefficient (and thus transport properties) are
smooth functions of the chaoticity of the system
(parameterized, for example, by the Lyapunov exponent $\Lyap =
\ln \ExpaEig$). This turns out not to be true: $D$ as a
function of $\ExpaEig$ is a fractal, nowhere differentiable
curve shown in \reffig{f-Klage}. The dependence of $D$ on the
map parameter $\ExpaEig$ is rather unexpected - even though for
larger $\ExpaEig$ more points are mapped outside the unit cell
in one iteration, the diffusion constant does not necessarily
grow. We refer the reader to \refrefs{detdiff1,detdiff2} for
early work on the deterministic diffusion induced by 1\dmn\
maps. The sawtooth map \refeq{KD-map} was introduced by
Grossmann and Fujisaka\rf{GroFuj82} who derived the integer slope
formulas \refeq{D-KD} for the diffusion constant. The sawtooth
map is also discussed in \refrefs{GB92}. The fractal dependence
of diffusion constant on the map parameter is discussed in
\refrefs{RKdiss,KlaDor95,KlaDor97}. \refSect{s:ChaosGood} gives
a brief summary of the experimental implications; for the the
current state of the art of fractal transport coefficients consult
the first part of Klage's monograph\rf{RKbook07}. Would be nice if
someone would eventually check these predictions in
experiments...
Statistical
mechanicians tend to believe that such complicated behavior is
not to be expected in systems with very many degrees of
freedom, as the addition to a large integer dimension of a
number smaller than $1$ should be as unnoticeable as a
microscopic perturbation of a macroscopic quantity. No
fractal-like behavior of the conductivity for the Lorentz gas
has been detected so far\rf{LNRM94}.
\index{Lorentz gas}
\hfill (P. Cvitanovi\'c and L. Rondoni)
} %end remark === {Structural stability for D}

remark === {Symmetry factorization in one dimension.}{
In the $\beta=0$ limit the dynamics
\refeq{sym2} is symmetric under $x \rightarrow -x$, and the
zeta functions factorize into products of zeta functions for
the symmetric and antisymmetric subspaces, as
described in \refexam{exam:Reflecti}:
\bea
{1 \over \zeta(0,z)} &=& {1 \over \zeta_{s}(0,z)}\,{1 \over \zeta_{a}(0,z)}
    \continue
 {\pde \over \pde z} { 1 \over \zeta} &=&
 { 1 \over \zeta_s} {\pde \over \pde z} { 1 \over \zeta_a}
 +
 { 1 \over \zeta_a} {\pde \over \pde z} { 1 \over \zeta_s}
\,.
\label{diffSymmetr}
\eea
The leading (material flow conserving) eigenvalue $z=1$ belongs
to the symmetric subspace
\(
{1 / \zeta_s(0,1)}=0
\),
so the derivatives \refeq{MeanTime} also depend
only on the symmetric subspace:
\bea
\expct{n}_\zeta &=&
   \left. z{\pde \over \pde z} {1 \over \zeta(0,z)}\right|_{z=1}
   \continue
&=&  {1 \over \zeta_{a}(0,z)}\, z{\pde \over \pde z}
    \left.{1 \over \zeta_{s}(0,z)}\right|_{z=1} \, .
\label{symm_der}
\eea
Implementing the symmetry factorization is convenient,
but not essential, at this level of computation.
}   %end remark === {Symmetry factorization in one dimension}

remark === {Lorentz gas in the fundamental domain.}{
The vector valued nature of the moment-generating function \refeq{lor-diff-1} in
the case under consideration makes it difficult to perform a calculation of
the diffusion constant within the fundamental domain.
Yet we point out that, at least as regards scalar quantities, the
full reduction to $\tilde{\pS}$ leads to better estimates.
A proper symbolic dynamics in the fundamental domain has been
introduced in \refref{freddy}%
\PublicPrivate{.}{% switch \PublicPrivate{
, numerical estimates for scalar
quantities are reported in \reftab{t-diff-5a}, taken from \refref{CGS}.
      }% end \PublicPrivate{
%The fundamental domain symbolic dynamics used here
% due to F.~Christiansen~\cite{freddy},
%is given in figs.~3?, 4? and \reftab{t-diff-3a}.
%\RA{insert figs. 3 and 4 from \refref{CGS}}
%Now the symbols indicate relative direction changes
%instead of the absolute directions.
%
%%%%% Table 3 %%%%%%%%%%%%%%%%%%%%%%%%
%\begin{table}
%{\small
%\begin{tabular}{|c||r|r||c|c|}
%\hline
%symbol & \multicolumn{2}{|c||}{amount of change} &
% \multicolumn{2}{|c|}{direction of change} \\
%       & last long & last short & next the same & next other way \\ \hline
%a      &    1      &     2      &     x      &       \\
%b      &    3      &     4      &     x      &       \\
%c      &    5      &     6      &     x      &       \\
%d      &    5      &     4      &            &   x    \\
%e      &    3      &     2      &            &    x   \\
%f      &    1      &     -      &            &    x   \\ \hline
%A      &    2      &     1      &     x      &       \\
%B      &    4      &     3      &     x      &       \\
%C      &    6      &     5      &     x      &       \\
%D      &    4      &     5      &            &    x   \\
%E      &    2      &     3      &            &    x   \\
%F      &    -      &     1      &            &    x   \\ \hline
%\end{tabular}
%\caption[]{\small
%Symbols in the fundamental domain.
%}
%\label{t-diff-3a}
%}  %end of \small
%\end{table}
%%%%%%%%%%%%%%%%%%%%%%%%%%%%%%%%%%%%%%%%%%%%%%%%%%%%%%%%%%%%%%%%%%
%
%\caption{Symbols in the fundamental domain} \label{TSYM}
%\end{table}
%\eject
%The reflection symmetry of the problem
%also should be taken into account by the symbols.
%The right and left turns are not distinguished - instead,  one reads
%off a symbol whether the next
%turn has to be taken in the same or in the opposite sense.
%
%Lower case letters denote short flights between closeby disks, upper case
%denotes the long flights to the next nearest disks.
%Each symbol corresponds to a given relative change
%in direction. The exact amount of
%change depends on whether the last flight was
%long (odd label in the notation of the last section) or short (even label).
%Table~\ref{TSYM} and figures~\ref{FUN1} and~\ref{FUN2}  give the definition of
%the symbols.
%
% Christiansen~\cite{freddy} had
% originally proposed to denote symbols `f' and `F' by the same
% letter, thus reducing the size of the alphabet.
%We use two letters `f' and `F' together with the
%pruning rule that `f' can only follow an uppercase
%letter (\ie, a long segment) and `F' only a lowercase letter.
%The symbols given to an orbit by this scheme are invariant under all spatial
%symmetries of the system but not under the time reversal.
%The fact is that time reversal symmetry is included in the spatial
%symmetries, at least for running orbits, as a consequence of the
%absence of external fields, and of the geometry of the model.
%
%\LR{PLEASE, make sure that this comment makes sense}
%
%Of the possible twelve symbols,  `A', `B', `f' and `F'
%are pruned as soon as the horizon gets finite.
%\\
%(PC: finite or densely packed???)\\
%TS: finite !
%Among the remaining symbols there is
%still strong pruning,  reflected in the fact that
%for $w = 0.3$ the
%number of cycles of symbol length $n$
%does not grow like $8^n$ but roughly as
%$3^n$. Figure~5? shows
%\RA{insert fig. 5 from \refref{CGS}}
%all the fundamental domain fixed
%points which are not pruned at $w=0.3$ together with an example
%of a pruned fixed point.
%
%
%Table~4 gives some impression of the pruning involved.
%In order that longer orbits be shadowed by shorter ones,
%for every combination of two symbols a two-cycle,
%and the fixed points corresponding to each of the symbols should exist.
%Two-cycles such as `ac', `aC', \ldots are missing while
%corresponding ``shadowing'' pseudo-cycles `a~c', `a~C' exist, and
%conversely, two-cycles occur where one of the symbols
%has no corresponding fixed point
%(e.g. the symbol `d' in `ad'). So
%the shadowing is largely disfunctional,
%at least as long as finite approximate Markov partitions are
%not developed.
%Nevertheless the finite order estimates for the Lyapunov exponent and
%material flow conservation, reported in \reftab{t-diff-5a}, are rather good, and noticeably
%better than the corresponding calculation using elementary cell dynamics
%(\reftab{t-diff-1}).
%\PC{recheck if indeed \reftab{t-diff-1}?}.
%
%%%%% Table 5 %%%%%%%%%%%%%%%%%%%%%%%%
\begin{table}
{\small
\begin{tabular}{|r|r|r|r|}
\hline
length & \# cycles & $\zeta$(0,0) & $\Lyap$ \\ \hline\hline
1      &    5     &  -1.216975 &     -    \\
2      &   10     &  -0.024823 & 1.745407 \\
3      &   32     &  -0.021694 & 1.719617 \\
4      &  104     &   0.000329 & 1.743494 \\
5      &  351     &   0.002527 & 1.760581 \\
6      & 1243     &   0.000034 & 1.756546 \\ \hline
\end{tabular}
\caption[]{\small
Fundamental domain, w=0.3 .
}
\label{t-diff-5a}
}  %end of \small
\end{table}
%%%%%%%%%%%%%%%%%%%%%%%%%%%%%%%%%%%%%%%%%%%%%%%%%%%%%%%%%%%%%%%%%%
%
%\caption{Fundamental domain, w=0.3} \label{CONV}
%\end{table}
%\eject
%Both in the elementary cell and in the fundamental domain convergence is
%dramatically better with more separated disks because there is no more pruning.
%Unfortunately, then the dynamics is artificially reduced to the cantor set
%of those orbits which travel only over the finite distance imposed by the
%possible symbols. So any diffusion constant cannot be compared to
%numerical values.
%\PC{remember the table correction by factor 2!}
%
%\begin{table}
%\caption{Elementary cell, $w$=0.3} \label{TCELL}
%\end{table}
%PG  I have the following numerical estimate for the Lyapunov exponent
% when w=0.3,   Lyap = 1.760 +/- 0.002 which supports the result of this
%table.
%\eject



%\begin{table}
%\begin{minipage}{5cm}
%{\sl Row/column 1 contains a cross when a fixed point of this symbol exists.
%Existing two-cycles are marked by crosses in the corresponding places.}
%\end{minipage}\hspace{1cm}
%\begin{minipage}{8cm}
% \begin{quote} \begin{center}
% \begin{tabular}{|c|c||c|c|c|c|c|c|c|c|}
% \hline
%   &   & a & b & c & d & e & C & D & E \\ \hline
%   & 1 & x & x & x &   &   & x & x &   \\ \hline\hline
% a & x & - & x &   & x &   &   &   &   \\
% b & x & x & - & x & x & x &   &   & x \\
% c & x &   & x & - & x & x &   & x &   \\
% d &   & x & x & x & - &   &   & x &   \\
% e &   &   & x & x &   & - &   &   &   \\
% C & x &   &   &   &   &   & - &   &   \\
% D & x &   &   & x & x &   &   & - &   \\
% E &   &   & x &   &   &   &   &   & - \\ \hline
% \end{tabular}\\[10pt]
%\end{minipage}
%\caption{ (no) shadowing, w=0.3}\label{NO}
%{Table 4:  (no) shadowing, w=0.3}
% {Table 4:  Problematic shadowing, w=0.3}
% \end{center}\end{quote}
% %\end{table}
% \eject
%
In order to perform the full reduction for diffusion one should
express the \dzeta\ \refeq{zeta-diff} in terms of the prime cycles of the
fundamental domain $\tilde \pS$ of the lattice
(see \reffig{fig-lor-1}) rather than
%\PC{need here fig-lor-2 - where did it go?}
those of the elementary (Wigner-Seitz) cell $\pS$.  This problem
is complicated by
%presents the following
%difficulty. The stumbling block appears to be
the breaking of the rotational symmetry by
the auxiliary vector $\beta$, or, in other words,
the non-commutativity of translations and rotations: see
%f.ex.
\refref{CEG}.
%for a discussion of the problem.
\label{r:LorentzFundDomain}
}   %end remark === {Lorentz gas in the fundamental domain}

remark === {Anomalous diffusion.}{
% PC moved this here from Chapter inter.tex 20sep2000
Anomalous diffusion for $1$\dmn\ intermittent maps was studied in
the  continuous time random walk approach in \refrefs{intGeiTho,intGeiNiZa}.
%%%maybe refer to Klafter Zumofen
The first approach within the framework of cycle expansions
(based on truncated \dzeta s)
was proposed in \refref{intACL}. Our treatment follows methods
introduced in \refref{PDsin}, applied there to
investigate the behavior of the Lorentz gas with unbounded horizon.
\index{intermittency}
%Early treatments of anomalous diffusion (see \refref{intACL}) by means
%of
%cycle expansions involved truncated \dzeta s: as a matter of fact, by
%denoting with $\zeta_{0[t]}^{-1}(z,\beta)$ the truncation of the
%\dzeta\ at order $t$, the following behavior was argued
%\beq
%\sigma^2(t)\,\sim \, -t \left.\frac{\partial^2 /\partial \beta^2\,
%\zeta_{0[t]}^{-1} (z,\beta)}{\partial / \partial z \,
%\zeta_{0[t]}^{-1} (z,\beta)}\right|_{z=1,\beta=0}
%\label{ast}
%\eeq
%If we refer to the example treated in the text we can see how
%the two formulations are
%linked together again by Tauberian arguments (as anomalous diffusion in
%\refeq{ast} is related to divergence of $\sum_{k=1}^t c_k$, where
%\[
%\left. \frac{\partial}{\partial z}
%\zeta_{0[t]}^{-1}(z,\beta)\right|_{\beta=0}\, = \, \sum_{k=0}^t c_k z^k
%\]
%
%{\bf correlation decays?}
%\exerbox{e-accel-diff}
%
%{\bf remark on stochastic modeling}
%
%remark === {A brief history of anomalous deterministic diffusion.}{
}

\index{Jonqui\`ere function}
remark === {Jonqui\`ere functions.}{
In statistical mechanics Jonqui\`ere function \refeq{JonqFunct}
appears in the theory of free Bose-Einstein gas, see \refrefs{FK,Htf1}.
%\PC{give refs to the Jonqui\`ere function}
\label{Jonq-funct}
}

      \PublicPrivate{
      }{% switch \PublicPrivate{
remark === {Molecular chaos.}{
Read  Gilbert  and Lefevere\rf{GilbLef08},
    ``Heat conductivity from molecular chaos hypothesis
             in locally confined billiard systems,''
    } % end remark === {Molecular chaos.}{

\PC{ Read
``From Deterministic Chaos to Deterministic Diffusion''
by R. Klages, \arXiv{0804.3068}: ``
A set of easy-to-read lecture notes for a short first-year Ph.D.
student course. The notes cover five hours of lectures and
do not require any prior knowledge on dynamical systems. The first part introduces
to deterministic chaos in one-dimensional maps in form of Lyapunov exponents
and the metric entropy. The second part first outlines the concept of
deterministic diffusion. Then the escape rate formalism for deterministic
diffusion, which expresses the diffusion coefficient in terms of the above two
chaos quantities, is worked out for a simple map. The notes conclude with a
very brief sketch of anomalous diffusion.
    }
\PC{for `fundamental domain' in hyperbolic geometry, see for example
\HREF{http://www.math.ou.edu/~kmartin/mfs/ch3.pdf}{these notes}
by \HREF{http://www.math.ou.edu/~kmartin}{Kimball Martin}.
    }
\PC{2013-02-03 Roberto: incorporate kneading determinants from
G.~Cristadoro\rf{Cristad06}
{\em Fractal diffusion coefficient from dynamical zeta functions}.
}
      }% end \PublicPrivate{








\subsection{References refsDiff}
{ 7aug2002}

% Predrag                   7aug2002

\refref{MR96}  L. Rondoni and G.P. Morriss,
	``Stationary nonequilibrium ensembles for thermostated systems,''
	{\em Phys. Rev. \bf E 53}, 2143 (1996).


\refref{Vance92} W.N. Vance,
%	``Unstable periodic orbits and transport properties of
%	nonequilibrium steady states,''
	{\em Phys. Rev. Lett. \bf 96}, 1356 (1992).

\refref{RKdiss} R. Klages,
	{\em Deterministic diffusion in one-dimensional chaotic
	dynamical systems} (Wissenschaft \& Technik-Verlag, Berlin, 1996);
    \\
	%link to Ph.D. thesis RK:
	{\tt \href{http://www.mpipks-dresden.mpg.de/~rklages/publ/phd.html}
	          {www.mpipks-dresden.mpg.de/~rklages/publ/phd.html}}.

\refref{KlaDor97} R. Klages and J.R. Dorfman,
	``Dynamical crossover in deterministic diffusion,''
	{\em Phys. Rev. \bf E 55}, R1247 (1997). % R1247-R1250


\refref{RKbook07} R. Klages,
    {\em Microscopic Chaos, Fractals and Transport
     in Nonequilibrium Statistical Mechanics},
     (World Scientific, Singapore 2007).

\refref{LNRM94}  J. Lloyd, M. Niemeyer, L. Rondoni and G.P. Morriss,
%        ``The Nonequilibrium Lorentz Gas,''
	{\em CHAOS \bf  5}, 536 (1995).
%        Univ. of New South Wales preprint (Sept. 1994).

\refref{detdiff1} T. Geisel and J. Nierwetberg,
%	``Onset of diffusion and universal scaling in chaotic systems''
	{\em Phys. Rev. Lett. \bf 48}, 7 (1982).

\refref{ScFrKa82} M. Schell, S. Fraser and R. Kapral,
%	`` Diffusive dynamics in systems with translational symmetry: a
%	one--dimensional--map model,''
	{\em Phys. Rev. \bf A 26}, 504 (1982).
	% they ``predict'' Diffusion constant, OK close to threshold

\refref{GroFuj82} S. Grossmann, H. Fujisaka,
%	``Diffusion in discrete nonlinear dynamical systems,''
	{Phys. Rev. \bf A 26}, 1179 (1982);
	H. Fujisaka and S. Grossmann, {Z. Phys. \bf B 48}, 261 (1982).

\refref{GB92} P. Gaspard and F. Baras, in
	M. Mareschal and B.L. Holian, eds.,
	{\em Microscopic simulations of Complex Hydrodynamic
	Phenomena} (Plenum, NY 1992).

\refref{freddy} F. Christiansen, Master's Thesis, Univ.
        of Copenhagen (June 1989).

\refref{ACL94} R. Artuso, G. Casati and R. Lombardi,
	% just conference proceedings
	  {\em Physica \bf A 205}, 412  (1994).

	% a bit boring...
\refref{DanChe04} I. Dana and V.E. Chernov,
	``Periodic orbits and chaotic-diffusion
	 probability distributions,''
	{\em Physica A \bf 332}, 219 (2004). % 219-229


\refref{GilbLef08} T. Gilbert  and R. Lefevere,
    ``Heat conductivity from molecular chaos hypothesis
             in locally confined billiard systems,''

\refref{breakt} G. Casati, B.V. Chirikov, F.M. Izrailev and J. Ford, in
%{\em Stochastic Behaviour in Classical and Quantum Hamiltonian Systems},
%(G. Casati and J. Ford, eds.), Springer, Berlin (1979)

\refref{WaHu} X.-J. Wang and C.-K. Hu, {\em Phys.Rev.} {\bf E48}, 728
%(1993).

\refref{ZumKla} G. Zumofen and J. Klafter, {\sf Scale Invariant Motion in
%Intermittent Chaotic Systems}, {\em Phys.Rev.} {\bf A}, to appear \RA{Find
%ref.}

\refref{ACL93} R. Artuso, G. Casati and R. Lombardi,
%	``Periodic orbit theory of anomalous diffusion,''
%	{\em Phys. Rev. Lett. \bf 71}, 62 (1993).

\refref{Artuso94} R. Artuso,
	% just conference proceedings
%	{\em  Physica \bf D 76}, 101 (1994).


\refref{BSM} B. Eckhardt,
        % {\em Periodic Orbits and Diffusion in Standard Maps},
        % {\em Marburg preprint} (July 1992)
%        {\em Phys. Lett. \bf 172A} 411 (1993).

\refref{BunM} L.A. Bunimovich,
%	{\em Chaos} {\bf 5}, 349-355 (1995)



\refref{LRM94}  J. Lloyd, L. Rondoni and G.P. Morriss,
%        ``The Breakdown of Ergodic Behaviour in the Lorentz Gas,''
%	{\em Phys. Rev. \bf E 50}, 3416 (1994).

\refref{MRC94}  G.P. Morriss, L. Rondoni and E.G.D. Cohen,
%        ``A Dynamical Partition Function for the Lorentz Gas,''
%        {\em J. Stat. Phys. \bf 80}, 35 (1995).


\refref{GG93} G. Pedro and G. Giovanni, % Garrido Pedro, Gallavotti Giovanni
%	``Billiards correlation functions,''
%	{\em J. Stat. Phys.  \bf ??}, 549 (1984).
	%; \arXiv{chao-dyn/9310005} 27 Oct 93
	%From: giovanni@boltzmann.rutgers.edu (Giovanni Gallavotti)
	% experiments on the time decay of velocity
	% autocorrelation functions in billiards
	% results which are compatible with an exponential mixing hypothesis,
	% first put forward by [FM]: they do not seem compatible with the
	% stretched exponentials believed, in spite of [FM], to describe the
	% mixing.



\refref{UruKoc90} {New metric elements of the universal fine structure due to multifurcations},
 {V. Urumov and L. Kocarev},
{what is Urumov about?}

	%2-d standard map numerical diffusion:
\refref{rw}A.B. Rechester, R.B. White,
%	Phys.Rev.Lett. {\bf 44}, 1586 (1980);
%	A.B. Rechester, M.N. Rosenbluth, R.B. White,
%	Phys.Rev. A {\bf 23}, 2664 (1981)

\refref{cm}J.R. Cary, J.D. Meiss, A. Bhattacharjee,
%	Phys.Rev. A {\bf 23}, 2744 (1981);
%	J.R. Cary, J.D. Meiss, Phys.Rev. A {\bf 24}, 2664 (1981);
%	T.M. Antonsen and E. Ott, Phys.Fluids {\bf 24}, 1635 (1981)

\refref{DaMuPe88} I. Dana, N.W. Murray and I. Percival
%	{\em Phys. Rev. Lett. \bf 62}, 233 (1989).

\refref{PerViv} I. Percival and F. Vivaldi, {\em Physica} {\bf D27}, 373
%(1987)

\refref{BirViv} N. Bird and F. Vivaldi, {\em Physica} {\bf D30}, 164
%(1988)

	%transport by turnstiles:
\refref{MKMP84} R.S.~MacKay, J.D. Meiss and I.C. Percival,
%	{\em Physica \bf D 13}, 55 (1984);
%	Q. Chen and J.D. Meiss,
%	{\em Nonlinearity} {\bf 39}, 347 (1989);
%	Q. Chen {\em et al.}, Physica D {\bf 46}, 217 (1990);
%	J.D. Meiss,
%	``Symplectic Maps, Variational-Principles, and Transport,''
%	{\em Rev. Mod. Phys. \bf 64}, 795 (1992).

\refref{dana89}
% 	 ``Hamiltonian transport on unstable periodic orbits,''
%	{\em Physica \bf D 39}, 205 (1989)

\refref{Dana90} I. Dana,
%        ``Organization of chaos in area-preserving maps,''
%	{\em Phys. Rev. Lett. \bf 64}, 2339 (1990).

\refref{Gaspard92} P. Gaspard,
%	 ``Diffusion, effusion and chaotic scattering:
%	 an exactly solvable Liouvillian dynamics,''
	% - his baker map calculation
%	{\em J. Stat. Phys.} {\bf 68}, 673 (1992).
	% Entropy+Lyapunovs --> transport coefficients:

\refref{ga2} P. Gaspard, {\em Phys. Lett. \bf A 168}, 13 (1992);
	%Chaos {\bf 3}, 427 (1993).
%diffusion coefficient for Lorentz gas:

\refref{jrd} P. Gaspard and J.R. Dorfman,
%        ``Chaotic scattering theory, ...,''
%         {\em Phys. Rev. \bf E 52}, 3525 (1995).

\refref{ech}D.J. Evans, E.G.D. Cohen, G.P. Morris,
%	Phys.Rev. A {\bf 42}, 5990 (1990);
	%construction of finite Markov partitions:

	%discussion of finite Markov partitions:
\refref{bst}C.S. Hsu, M.C. Kim, {\em Phys. Rev. \bf A 31}, 3253 (1985);
%	N. Balmforth, E.A. Spiegel, C. Tresser,
%	{\em Phys. Rev. Lett. \bf 72}, 80 (1994)

	%bloc-circulant matrices:
\refref{bk} T.H. Berlin and M. Kac,
%	Phys.Rev. {\bf 86}, 8211 (1952); see also \cite{PJD}

\refref{PJD} P.J. Davis,
%        {\em Circulant Matrices} (Wiley, New York, 1979)


\refref{HBA94}  A. Hakmi, F. Bosco and I. Antoniou,
%	``The First Return Map of the Periodic Lorentz Gas,''
%	ULB, Bruxelles prperint (aug. 1994).

%%%scattering off crystals: %%%%%%%%%%%%%%%
\refref{BSW36} L. P. Bouckaert, R. Smoluchowski, and E. P. Wigner,
%        {\em Phys. Rev. \bf 50}, 58 (1936).

\refref{RM92} R. Mainieri,
%	``Thermodynamic-Zeta Functions for Ising-Models with Long-
%           Range Interactions
%	{\em Phys.Rev.} {\bf A45} (1992) 3580

\refref{UH} B. Friedman and R.F. Martin, Jr.,
%	{\em Phys. Lett.} {\bf 105A} (1984) 23


\refref{ross} O. R\"ossler, Phys. Lett. {\bf 57A}, 397 (1976).


\refref{pomeau80} Y. Pomeau and P. Manneville, {\em Commun. Math. Phys.}
% 74} (1980) 189; P. Manneville, {\em J. Phys.} (Paris) {\bf 41} (1980)
% 1235

% only short conf proceedings:
\refref{RMLNC94}  L. Rondoni, G.P. Morriss, J.P. Lloyd,
%         M. Niemeyer, and E.G.D. Cohen,
%         ``Lorentz Gas, Periodic Orbit Expansions, Partitions, and Ergodicity,''
%         ICDC, Tokyo 1994, proceedings
