% Diffuse globally, compute locally: a cyclist tale
% Tingnan Zhang, Daniel I. Goldman and Predrag Cvitanovi\'c

The diffusion constant and the Lyapunov exponent for the spatially
periodic Lorentz gas are evaluated numerically in terms of periodic
orbits. A symbolic description of the dynamics reduced to a fundamental
domain is used to generate the shortest periodic orbits. Applied to a
dilute Lorentz gas with finite horizon, the theory works well, but for
the dense Lorentz gas the convergence is hampered by the strong pruning
of the admissible orbits.

\bigskip\bigskip

The 2-dimensional Lorentz gas models the diffusive motion of a light
molecule within a large number of heavy scatters by a point particle
bouncing off a collection of reflecting disks in a plane. Proposed by H.
A. Lorentz at 1905, the model has been not only useful in studying dilute
electron gas thermal and diffusive properties (in approximation where
electron-electron interactions are ignored), but also in dynamical
systems/statistical physics to answer fundamental questions on how
ergodicity arises from determinism. Motivated by a recent potential
application to macroscopic transport (statistics of robotic locomotion
paths over a heterogeneous terrain strewn with scatterers), we present a
very precise computation (not a numerical simulation, but an evaluation
of the exact periodic orbit theory formula for the diffusion constant)
for a periodic triangular Lorentz gas with finite horizon. We formulate a
new approach to tiling the plane in terms of three elementary tiling
generators which, for the first time, enables us to use periodic orbits
computed in the fundamental domain (that is, $1/12$ of the hexagonal
elementary cell whose translations tile the entire plane). Compared with
previous literature (which, amusingly, explicitly states that our
calculation cannot be done), our fundamental domain value of the
diffusion constant converges quickly for inter-disk separation/disk
radius $>0.2$, with the cycle expansion truncated to only a few hundred
periodic orbits of up to $5$ billiard wall bounces. For small inter-disk
separations, with periodic orbits up to $6$ bounces, our diffusion
constants are close ($<10\%$) to direct numerical simulation estimates,
as well as the recent literature probabilistic estimates.
