\documentclass[a4paper,10pt]{article}

\usepackage{hyperref}
\usepackage{url}
\usepackage{color}


%opening
\title{Research Statement}
\author{Evangelos Siminos}
\date{November 2010}

\definecolor{darkblue}{rgb}{0.13,0.17,0.63}

\hypersetup{colorlinks=true,urlcolor=darkblue}


\begin{document}

\maketitle

My research interests lie in the area of nonlinear dynamics of 
spatially extended systems. My focus is on the application of methods
of dynamical system theory in diverse physical settings including
the kinetic description of collective effects in plasmas, 
relativistic intensity laser-plasma interaction and non-equilibrium 
statistical mechanics. 


My recent work focuses on the kinetic theory description of stimulated Raman
scattering (SRS) in inertial confinement fusion plasmas.
SRS is a major source of energy loss for current inertial confinement efforts
and a theoretical understanding of its saturation is still missing.
It has been recently suggested that SRS saturation is triggered by a
secondary instability of spatio-temporally modulated, 
electrostatic plasma waves. However, a general and robust method 
to study linear stability of electrostatic waves is not at our disposal 
even for the simpler case of stationary waves. 
% My main goal since
% I've started working as a postdoc at CEA in July 2009 is developing such a
% method. 
The approach I have developed in collaboration with D. B\'enisti and 
L. Gremillet~\cite{siminos11}
as a postdoc at CEA, involves reduction to a finite-dimensional eigenproblem
through expansion of the relevant field equations, the Vlasov-Poisson system, 
in a basis of orthogonal polynomials. Despite the huge success of similar
methods in fluid dynamics, the conservative nature and multiple-scale dynamics
of Vlasov equation inhibit convergence of the eigenvalue calculation. As
a solution to this problems we have introduced spectral deformation techniques,
most commonly employed in quantum mechanics, to transform the linear 
stability problem into a dissipative one, while preserving eigenvalues 
of interest. This led to a very practical, fast converging scheme 
to compute unstable, collective modes for the Vlasov-Poisson system, 
which we have succesfully applied to compute stability of stationary 
plasma waves. From a theoretical perspective
this contributed to a better understanding of the `vortex fusion'
phenomenon in electrostatic waves. From a practical point of view it opens 
new possibilities for the control of systems with long range interactions 
through the selective excitation of collective modes.

My earlier PhD thesis work at the Georgia Institute of Technology,
involved the study, with P. Cvitanovi\'c and R. L. Davidchack~\cite{SCD07}, 
of the Kuramoto-Sivashinsky system. 
The Kuramoto-Sivashinsky system is a partial differential equation which
has been extensively studied as a simple, 
one-dimensional prototype for fluid turbulence. In dynamical systems 
studies of turbulence the organization of the
infinite-dimensional state space of a spatially extended system
can be understood in terms of equilibrium solutions
and their unstable manifolds, which serve to connect local neighborhoods
of equilibria. However, when traveling wave solutions are
allowed, this approach is obscured by the presence of equivalent, up to 
a continuous symmetry transformation, dynamical trajectories. 
The main achievement of my thesis~\cite{SiminosThesis,SiCvi10} 
was to show that continuous symmetry
reduction, i.e. the identification of symmetry related trajectories, 
can be practically implemented in the high-dimensional state-space 
of spatially extended systems and that the geometric approach of dynamical
systems can then be successfully applied.

These theoretical studies fueled my contribution to a problem of great
practical interest in the growing field of laser-matter interaction. 
The current trend is to develop compact accelerators that
exploit short, high-intensity laser pulses to accelerate particles
in much higher energies than possible with conventional accelerators 
of comparable size. Short laser pulses are hard to produce and control and
therefore an open question is whether longer pulses can be utilized for
the same purpose of particle acceleration. A first step to tackle the
problem is to connect it to the existence of solitary waves 
in a fluid-plasma electromagnetic model which could act as a mean to transfer   
energy to the particles through wavebreaking or parametric instabilities.
With G. S\'anchez-Arriaga and E. Lefebvre~\cite{SSL10}, we reduced
the problem to that of finding homoclinic and heteroclinic connections
of a Hamiltonian system of ordinary differential equations. Then,
dynamical systems theory allowed a systematic determination and classification
of solitary wave solutions. We are currently working on the problem
of stability and practical excitation of such solutions in laser-matter
interaction.

My perspective for future research is still towards cross-fertilization
between different areas of physics, through the unifying framework of
dynamical systems. My short term goal after leaving CEA is to exploit
the framework developed for the stability of Vlasov-Poisson equilibria 
in the study of collective modes of other long-range interacting systems, 
such as coupled oscillators, whose dynamics bears great resemblance
to Vlasov dynamics, with possible applications to biological networks.

\bibliography{../bibtex/plasmas}
\bibliographystyle{unsrt}

Please visit \href{http://cns.gatech.edu/~siminos/publications.php}{\url{http://cns.gatech.edu/~siminos/publications.php}}
for preprints.

\end{document}