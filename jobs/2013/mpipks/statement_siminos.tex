%% start of file `template_en.tex'.
%% Copyright 2007 Xavier Danaux (xdanaux@gmail.com).
%
% This work may be distributed and/or modified under the
% conditions of the LaTeX Project Public License version 1.3c,
% available at http://www.latex-project.org/lppl/.


\documentclass[11pt,a4paper,final]{moderncv}

% moderncv themes
% \moderncvtheme[blue]{casual}                 % optional argument are 'blue' (default), 'orange', 'red', 'green', 'grey' and 'roman' (for roman fonts, instead of sans serif fonts)
\moderncvtheme[grey]{classic}                % idem

% character encoding
\usepackage[utf8]{inputenc}                   % replace by the encoding you are using

% 
% \usepackage{hyperref} %ES
\usepackage{url} %ES
\usepackage{ifthen}
\usepackage{multicol} %ES
\usepackage{textcomp} %ES

\newif\ifpaper 

\papertrue % For committees, usually printed out and handed to members.
% \paperfalse % For individuals, use [doi] and [pdf] links to publications


% adjust the page margins
\usepackage[scale=0.8]{geometry}
\setlength{\hintscolumnwidth}{0.4cm}			% if you want to change the width of the column with the dates
\AtBeginDocument{\setlength{\maketitlenamewidth}{7cm}}  % only for the classic theme, if you want to change the width of your name placeholder (to leave more space for your address details
\AtBeginDocument{\recomputelengths}                     % required when changes are made to page layout lengths

\makeatletter
\renewcommand*{\bibliographyitemlabel}{\@biblabel{\arabic{enumiv}}}
\makeatother

\input ../../../CV/defs

\newcommand{\arXiv}[1]{\href{http://arxiv.org/abs/#1}{arXiv:#1}}

% personal data
\firstname{\large{Evangelos}}
\familyname{\large{Siminos}}
 \title{\large{Activity Report and Research Proposal}}               % optional, remove the line if not wanted
\address{Max Planck Institute for the Physics of Complex Systems\\
	  N\"{o}thnitzer Str. 38}
	{01187 Dresden, Germany}% optional, remove the line if not wanted
% \mobile{mobile (optional)}                    % optional, remove the line if not wanted
\phone{+49 351 871 2412}                      % optional, remove the line if not wanted
% \fax{fax (optional)}                          % optional, remove the line if not wanted
\email{evangelos.siminos@gmail.com}% optional, remove the line if not wanted
% \email{evangelos.siminos@cea.fr}% optional, remove the line if not wanted
% \extrainfo{\href{http://www.cns.gatech.edu/~siminos}{\url{www.cns.gatech.edu/~siminos}}} % Does not work
 \extrainfo{\httplink[http://www.cns.gatech.edu/$\sim$siminos]{www.cns.gatech.edu/~siminos}} % optional, remove the line if not wanted
% \photo[80pt]{siminos_small.jpg}                         % '64pt' is the height the picture must be resized to and 'picture' is the name of the picture file; optional, remove the line if not wanted
% \quote{Some quote (optional)}                 % optional, remove the line if not wanted
% \nopagenumbers{}                             % uncomment to suppress automatic page numbering for CVs longer than one page



\date{\today}

\lfoot{E. Siminos - Activity Report and Research Proposal}

%----------------------------------------------------------------------------------
%            content
%----------------------------------------------------------------------------------
\begin{document}
\maketitle

% \setlength{\parindent}{0.25in} % ES: For research statement. After maketitle!

The common theme in my research activities at MPIPKS is complexity in light-matter interactions. After joining MPIPKS
I had the opportunity to apply nonlinear dynamics techniques, which were part of my training as a PhD student, to 
problems in different areas of laser-matter interaction ranging from relativistic intensity laser-plasma interaction~\cite{siminos2012},
to dynamics of solitons in nonlocal nonlinear media~\cite{maucher2013} and plasmas~\cite{saxena2013}. 
At the same time I kept collaborations in the topic of dynamical systems active~\cite{atlas12} and initiated new ones.
Our work on these topics, as well as proposed new research directions, are explained in detail in the following.
 


\section{Research activities at MPIPKS}
\sep
\inlinesubsect{Relativistic intensity laser-plasma interaction}
Contemporary laser systems reach intensities well above $10^{20}\mathrm{W/cm^2}$, 
leading to new regimes of interaction of light with matter. Motion of
electrons in such intense fields is highly relativistic and the optical
properties of matter are profoundly affected by the nonlinear response
of the medium. After my arrival at MPIPKS I joined the research activities of our
group in this research area, contributing to the developement of a particle-in-cell code (PIC) 
which solves the coupled equations for the evolution of the fields and particles in
a plasma. We used our code to study the effect of relativistic self-induced
transparency in laser-plasma interaction~\cite{kaw1970,palaniyappan2012}.
% , in collaboration with M. Grech, S. Skupin (MPIPKS),
% T. Schlegel (Jena) and V. Tikhonchuk (Bordeux). 
We have shown that an intense pulse 
can propagate in an otherwise opaque medium by heating and expelling electrons towards the vacuum, 
'drilling' its way through it. We interpreted our results through a study of nonlinear 
dynamics of electrons in the plasma-vacuum interface~\cite{siminos2012}. 
Our results are relevant to the stability of radiation pressure acceleration schemes, 
an issue which we are currently studying.
\sep 

\inlinesubsect{Nonlinear dynamics of infinite dimensional systems: from turbulence to nonlinear optics}
One of the main themes in my research is the effort to understand the topological organization of 
infinite dimensional systems (described by partial differential equations PDEs) in terms of compact solutions.
This effort, in which I was first involved during my PhD at Georgia Tech, is continued at MPIPKS along two complementary
directions. In collaboration with P. Cvitanovi\'c (Georgia Tech), H. Chat\'e (CEA/Saclay, organizer of
an MPIPKS advanced study group in 2011/2012) and K. Takeuchi (Tokyo) we study the relevance of
Lyapunov vectors computed on periodic orbits in order to describe invariant low-dimensional manifolds
of \emph{dissipative} PDEs, such as the Kuramoto-Sivashinsky equation. At the same time, we took the initiative
at the MPIPKS to carry this program over to \emph{conservative} PDEs, taking as a model system the nonlocal nonlinear
Schr\"odinger equation. In collaboration with F. Maucher (PhD student in our group), S. Skupin (MPIPKS) and W. W. Krolikowski (Canberra)
we studied the so-called shape-transformations of higher-order bright solitons in nonlinear nonlocal media.
Taking a dynamical systems perspective~\cite{maucher2013}, we projected dynamics onto a symmetry-invariant~\cite{SiCvi10, atlas12},
dynamically informed basis. This led to the identification of quasiperiodic and homoclinic orbits 
as the organizing element behind shape transformations and sets the basis for similar studies in other systems
described by conservative PDEs.
\sep

\inlinesubsect{Relativistic solitary wave interaction}
Relativistic solitary waves arising from the interaction of an intense laser pulse with a plasma have received
a lot of attention, since there is experimental evidence that they are ubiquitous in laser-plasma interactions~\cite{borghesi02,sarri10}
and there is hope that they can be used to devise novel ion acceleration schemes.
% Pirozhkov A S, Ma J, Kando M et al. 2007 Phys. Plasmas 14 123106
% Borghesi M, Bulanov S, Campbell D H et al. 2002 Phys. Rev. Lett. 88 135002
% Borghesi M, Campbell D H, Schiavi A et al. 2002 Phys. Plasmas 9 2214–2220
% Sarri G, Singh D K, Davies J R et al. 2010 Phys. Rev. Lett. 105 175007
In a laser-plasma experiment, it is very likely that a large number of non-propagating 
localized solitary structures are excited which remain in the neighborhood of each other. 
These electromagnetic structures are expected to impart a force on each other
to give rise to interesting phenomena like repulsion, merging of
solitons or oscillatory bound states. 
In collaboration with participants of our ENLITE12 workshop at MPIPKS, V. Saxena and I. Kourakis (Belfast) 
and G. Sanzes-Arriaga (Madrid), we carried out numerical investigations on mutual 
interactions between two spatially overlapping standing 
electromagnetic solitons~\cite{saxena2013}.
We showed, for the first time, the existence of bound states of relativistic solitons in plasmas and analyzed their 
response to perturbations. We are currently working towards the detection of
exact periodic solutions in order to gain better insight in the nonlinear dynamics of pairs of solitons.

\section{Proposed Research}
\sep
\inlinesubsect{Can we see the shape of a plasma wake?}
As an ultra-high-intensity laser pulse propagates through a plasma, it excites behind it a wake (an electron plasma wave) 
which may act as an accelerating structure for plasma particles. Taming the dynamics of electrons inside such wakes 
has been an area of active experimental and theoretical research over 
the last decade, since there is hope that laser-plasma accelerators can surpass the performance of 
conventional accelerators by supporting 
much higher accelerating gradients~\cite{mangles04,geddes04,faure04,esarey09}. Although experimentally obtained 
energy spectra are consistent 
with the so-called bubble acceleration mechanism,
there has yet been no direct observation of the accelerating wakefields within the plasma. 
Ongoing experiments with the POLARIS laser at the
Helmholtz Institute Jena, provide access for the first time to coherent electron structures 
through a pump-probe setup that utilizes shadowgraphy 
through a $6~\mathrm{fs}$ duration probe pulse. Our group participates in this study by providing 
theoretical support through PIC simulations in order to
gain insight into wakefield formation and the imaging process in different regimes of interaction. 
Recently recruited PhD student A. Hussain is expected to join our activities in this topic in September 2013.\sep\\
This new experimental technique is expected to be useful in a variety of 
laser-plasma interaction settings beyond laser wakefield acceleration. 
As an example, preliminary experimental results at the same facility may be interpreted as
arising from excitation and interaction of solitons in the cavity left behind the wake. 
We plan to investigate this issue by applying the expertise we gained through
our previous studies of soliton excitation~\cite{SSL10,SSL10-1} 
and interaction~\cite{saxena2013} (in collaboration with V. Saxena 
who is applying to join our group as a postdoctoral fellow).
This would be the first time that soliton excitation and 
interaction is directly observed in a plasma. 
\sep


\inlinesubsect{Attosecond pulse generation}
High-harmonic generation (HHG) by irradiation of solid targets with intense femtosecond pulses 
provides novel mechanisms for attosecond pulse
generation with high conversion efficiency~\cite{teubner09}. Attosecond pulses can be used as probes of 
processes at the atomic level, but also encode important information
about the nonlinear processes which lead to their generation. 
In order to enter this interesting area of research we recently recruited
H. Vincenti (CEA/Saclay), who is an expert in
attosecond pulse generation, as a short term postdoctoral fellow.
% S. Skupin (MPIPKS), M. Grech (LULI) and G. Bonnaud (CEA/Saclay) 
We plan to study the effect of ionization, which has so far been unexplored,
in the so-called coherent wave emmision (CWE) mechanism~\cite{borot12} of attosecond pulse generation. 
Our goal is to exploit the recently discovered attosecond lighthouse effect~\cite{vincenti12,wheeler12} 
in order to propose ways to experimentally recover time-resolved information 
on the ionization process of an overdense plasma and its influence on high-harmonic generation.
In the longer term we plan to explore the role of complex electron dynamics~\cite{sanz12} on the
newly discovered coherent synchrotron emission scheme~\cite{dromey12}. 
\sep


\inlinesubsect{Signatures of relativistic chaos in laser-plasma interactions}
Chaos in relativistic systems is of interest in many areas ranging from astrophysics and cosmology (general relativity)
to accelerators, beams and plasmas (special relativity). At the mathematical level there has been a lot
of interest in which indicators of chaos are invariant under transformations consistent with the symmetries of
the problem (e.g. Lorentz transformations for special relativity), see e.g.~\cite{motter03,motter09-1}. Beside recent progress 
in establishing the mathematical 
relevance of classical measures of chaos, e.g. of Lyapunov exponents~\cite{motter03,motter09-1}, 
the physical implications of relativistic invariance (or lack thereof) of chaos is still missing.
% there is still need to has been no ellaboration on the physical
% consequences of relativistic invariance. 
I propose to use ultra-high-intensity optical lattices as convenient systems to study
signatures of relativistic chaos in electron dynamics.
I plan to analyze various signatures of chaos, such as
Lyapunov exponents, phase-space topology, spectrum of periodic orbits, and connect them to detectable quantities such
as electron momentum- and energy-spectra, as well as radiation spectra. Of particular interest is the formulation of
the problem as a chaotic scattering problem, which would allow to predict singularities in scattering functions in terms of
the structure of the chaotic saddle. For short pulses (i.e. of duration of a few optical cycles)
the problem takes the form of time-dependent chaotic scattering and its mathematical characterization is still an open
problem. My proposal is to investigate the issue through the use of Lagrangian coherent structures which have been recently used
to describe transport in time-dependent turbulent systems~\cite{MHPRS07}. These studies would also be relevant in laser-plasma
interactions, where chaotic electron motion typically occurs due to the nonlinear response of the plasma. 
For instance in the case of relativistic self-induced transparency (RSIT) we have studied earlier~\cite{siminos2012}, 
propagation is accompanied by electron bunches moving chaotically in the laser and electrostatic fields. 
Our study of chaos would provide experimentally detectable signatures of RSIT in electron spectra. For a different
project, Ecole Polytechnique PhD student Emma Llor, will visit us multiple times during the next three years
in order to investigate the problem of surface waves excited during laser-plasma interaction at ultra-high intensity.
In particular, we will study the chaotic dynamics of electrons in these waves, 
and their radiation. \\ \\ \\ 

\bibliographystyle{../../../tex/statement2012} %../../tex/poster
\bibliography{../../../bibtex/siminos}

\end{document}


%% end of file `template_en.tex'.
