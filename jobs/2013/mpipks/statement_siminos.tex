%% start of file `template_en.tex'.
%% Copyright 2007 Xavier Danaux (xdanaux@gmail.com).
%
% This work may be distributed and/or modified under the
% conditions of the LaTeX Project Public License version 1.3c,
% available at http://www.latex-project.org/lppl/.


\documentclass[11pt,a4paper,final]{moderncv}

% moderncv themes
% \moderncvtheme[blue]{casual}                 % optional argument are 'blue' (default), 'orange', 'red', 'green', 'grey' and 'roman' (for roman fonts, instead of sans serif fonts)
\moderncvtheme[grey]{classic}                % idem

% character encoding
\usepackage[utf8]{inputenc}                   % replace by the encoding you are using

% 
% \usepackage{hyperref} %ES
\usepackage{url} %ES
\usepackage{ifthen}
\usepackage{multicol} %ES
\usepackage{textcomp} %ES

\newif\ifpaper 

\papertrue % For committees, usually printed out and handed to members.
% \paperfalse % For individuals, use [doi] and [pdf] links to publications


% adjust the page margins
\usepackage[scale=0.8]{geometry}
\setlength{\hintscolumnwidth}{0.4cm}			% if you want to change the width of the column with the dates
\AtBeginDocument{\setlength{\maketitlenamewidth}{7cm}}  % only for the classic theme, if you want to change the width of your name placeholder (to leave more space for your address details
\AtBeginDocument{\recomputelengths}                     % required when changes are made to page layout lengths

\makeatletter
\renewcommand*{\bibliographyitemlabel}{\@biblabel{\arabic{enumiv}}}
\makeatother

\input ../../../CV/defs

\newcommand{\arXiv}[1]{\href{http://arxiv.org/abs/#1}{arXiv:#1}}

% personal data
\firstname{\large{Evangelos}}
\familyname{\large{Siminos}}
 \title{\large{Activity Report and Research Proposal}}               % optional, remove the line if not wanted
\address{Max Planck Institute for the Physics of Complex Systems\\
	  N\"{o}thnitzer Str. 38}
	{01187 Dresden, Germany}% optional, remove the line if not wanted
% \mobile{mobile (optional)}                    % optional, remove the line if not wanted
\phone{+49 351 871 2412}                      % optional, remove the line if not wanted
% \fax{fax (optional)}                          % optional, remove the line if not wanted
\email{evangelos.siminos@gmail.com}% optional, remove the line if not wanted
% \email{evangelos.siminos@cea.fr}% optional, remove the line if not wanted
% \extrainfo{\href{http://www.cns.gatech.edu/~siminos}{\url{www.cns.gatech.edu/~siminos}}} % Does not work
 \extrainfo{\httplink[http://www.cns.gatech.edu/$\sim$siminos]{www.cns.gatech.edu/~siminos}} % optional, remove the line if not wanted
% \photo[80pt]{siminos_small.jpg}                         % '64pt' is the height the picture must be resized to and 'picture' is the name of the picture file; optional, remove the line if not wanted
% \quote{Some quote (optional)}                 % optional, remove the line if not wanted
% \nopagenumbers{}                             % uncomment to suppress automatic page numbering for CVs longer than one page



\date{\today}

\lfoot{E. Siminos - Activity Report and Research Proposal}

%----------------------------------------------------------------------------------
%            content
%----------------------------------------------------------------------------------
\begin{document}
\maketitle

% \setlength{\parindent}{0.25in} % ES: For research statement. After maketitle!


\section{Research activities at MPIPKS}
\sep
\inlinesubsect{Relativistic intensity laser-matter interaction}
Laser pulses produced by state of the art systems ionize any material they go through, 
producing a plasma, while accelerating electrons to speeds close to the speed of light within a femtosecond. 
Under such conditions, the optical properties of matter are fundamentally altered, to such an extend that a new field, that of relativistic optics, was born. 
In order to study propagation of such pulses in a medium, one has to take into account the complex interactions of a many-body system with the electromagnetic field.
The resulting nonlinear system of coupled partial differential equations (relativistic Vlasov equation coupled to Maxwell equations) is hard to solve, 
even in powerful parallel supercomputers and even simple questions are hard to answer. 
My research at the MPIPKS is centered around such a simple question: given a pulse of certain intensity, how can we determine the the critical density, 
i.e. the maximum density of a target which would allow propagation of an intense laser pulse? Numerical experiments suggest that 
an intense pulse can propagate in an, otherwise opaque, medium by heating and expelling electrons out of the medium, 'drilling' its way through it. 
To understand this effect we turned to the study of phase-space topology of a simple dynamical system describing electron motion in the laser field. 
Determining separatrices in phase space which act as transport barriers, preventing electrons to escape the potential well created by radiation pressure, 
allows a clear understanding of how pulse propagation is triggered~\cite{siminos2012}.
\sep 

\inlinesubsect{Nonlinear dynamics of infinite dimensional systems}
One of the main themes in my research is the effort to understand the topological organization of 
infinite dimensional systems (described by partial differential equations PDEs) in terms of compact solutions.
This effort, in which I was first involved during my PhD at Georgia Tech, is continued at MPIPKS along two complementary
directions. In collaboration with P. Cvitanovi\'c (Georgia Tech), H. Chat\'e (CEA/Saclay, organizer of
an MPIPKS advanced study group in 2011/2012) and K. Takeuchi (Tokyo) we study the relevance of
Lyapunov vectors computed on periodic orbits in order to describe invariant low-dimensional manifolds
of \emph{dissipative} PDEs, such as the Kuramoto-Sivashinsky equation. At the same time, we took the initiative
at the MPIPKS to carry this program over to \emph{conservative} PDEs, taking as a model system the nonlocal nonlinear
Schr\"odinger equation. In collaboration with F. Maucher, S. Skupin (MPIPKS) and W. W. Krolikowski (Canberra)
we studied the so-called shape-transformations of higher-order bright solitons in nonlinear nonlocal media
from a dynamical systems perspective~\cite{maucher2013}, by projecting dynamics onto a symmetry-invariant~\cite{SiCvi10, atlas12},
dynamically informed basis. This led to the identification of quasiperiodic and homoclinic orbits 
as the organizing element behind shape transformations and sets the basis for similar studies in other systems
described by conservative PDEs.
\sep

\inlinesubsect{Relativistic solitary wave interaction}
Relativistic solitary waves arising from the interaction of an intense laser pulse with a plasma have received
a lot of attention due to their ubiquitous appearance in various settings.
In a laser plasma experiment, it is very likely that a large number of non-propagating 
localized solitary structures are excited which remain in the neighborhood of each other. 
These electromagnetic structures are expected to impart a force on each other
to give rise to interesting phenomena like repulsion, merging of
solitons or oscillatory bound states. 
In collaboration with ENLITE workshop participants V. Saxena and I. Kourakis (Belfast) and G. Sanzes-Arriaga (Madrid)
we carried out numerical investigations on mutual interactions between two spatially overlapping standing 
electromagnetic solitons~\cite{saxena2013}.
We showed, for the first time, the existence of bound states of relativistic solitons in plasmas and analyzed the conditions
under which the interaction can lead to repulsion. 



\section{Proposed Research}
\sep
\inlinesubsect{Can we see the shape of a plasma wake?}
As an ultra-high-intensity laser pulse propagates through a plasma, it leaves behind it a wake (an electron plasma wave) which can act as an accelerating
structure for plasma particles. Taming the dynamics of electrons inside such wakes has been an area of active experimental and theoretical research over 
the last decade, since there is hope that laser-plasma accelerators can surpass the performance of conventional accelerators by supporting 
much higher accelerating gradients. Although experimentally obtained spectra are consistent with the so-called bubble acceleration mechanism,
there has yet been no direct observation of the accelerating wakefields within the plasma. Ongoing experiments with the POLARIS laser at the
Helmholtz Institute Jena, provide access for the first time to coherent electron structures through a pump-probe setup that utilizes shadowgraphy 
through a $6fs$ duration probe pulse. Our group participates in this study by providing theoretical support through PIC simulations in order to
gain insight into the imaging process in different regimes of interaction. This technique is expected to be useful in a variety of 
laser-plasma interaction settings beyond laser wakefield acceleration. As an example, preliminary experimental results at the same facility indicate that solitary waves 
are excited and interact with each other at the cavity left behind the wake. We plan to investigate this issue by applying the expertise we gained through
our previous studies of soliton excitation~\cite{SSL10,SSL10-1} and interaction~\cite{saxena2013} (in collaboration with V. Saxena who is applying to join our group as a postdoctoral fellow).
This would be the first time that soliton excitation and interaction is directly observed in a plasma.
\sep


\inlinesubsect{Attosecond pulse generation}
High-harmonic generation (HHG) by irradiation of solid targets with intense femtosecond pulses provides novel mechanisms for attosecond pulse
generation with high conversion efficiency. Attosecond pulses can be used as probes of processes at the atomic level, but also provide information
about processes in plasma which generate them. In collaboration with H. Vincenti, S. Skupin (MPIPKS), M. Grech (LULI) and G. Bonnaud (CEA/Saclay) 
we began to study the effect of ionization in the so-called coherent wave emmision (CWE) mechanism of attosecond pulse generation. Our goal is
to propose an experimentally realizable mechanism based on the concept of attosecond lighthouse that would allow to recover time- and spatially-resolved
information on the ionization process of an overdense plasma.
\sep


\inlinesubsect{Signatures of relativistic chaos in laser-plasma interactions}
Chaos in relativistic systems is of interest in many contexts ranging from astrophysics and cosmology (general relativity)
to accelerators, beams and plasmas (special relativity). At the mathematical level the pertinent question is what
are the relevant indicators of chaos that are invariant under symmetry transformations consistent with the physics of
the problem (e.g. Lorenz transformations for special relativity). Beside the recent progress in establishing the mathematical 
relevance of classical measures of chaos, e.g. of Lyapunov exponents, there are has been no ellaboration on the physical
consequences of relativistic invariance. I propose to study the implications of relativistic chaos in the interaction of
a beam of electrons with an ultra-high-intensity optical lattice. I plan to analyze various signatures of chaos, such as
Lyapunov exponents, phase-space topology, spectrum of periodic orbits, and connect them with measurable quantities such
as momentum and energy spectra of outgoing electrons and radiation spectra. In particular interest is the formulation of
the problem as a chaotic scattering problem, which would allow to predict singularities in scattering functions in terms of
the structure of the chaotic saddle. For short pulses (i.e. of duration of a few optical cycles)
the problem takes the form of time-dependent chaotic scattering and its mathematical characterization is still an open
problem. My proposal is to investigate the issue through the use of Lagrangian coherent structures which have been recently used
to describe transport in time-dependent systems.
\\ \\ \\ 

\bibliographystyle{../../../tex/statement2011} %../../tex/poster
\bibliography{../../../bibtex/siminos}

\end{document}


%% end of file `template_en.tex'.
