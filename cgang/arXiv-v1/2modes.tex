%%%% for public version, toggle \draftfalse in setup2modes.tex
%    (that removes all comments, the blog)

% reducesymm/cgang/2modes.tex    this is master file:    pdflatex 2modes
%     then:    pdflatex def2modes; bibtex def2modes; pdflatex def2modes; pdflatex def2modes

% until 2012-08-20 this was in svn repo siminos/cgang/2modes.tex

\documentclass[aip,cha,
reprint,
secnumarabic,
nofootinbib, tightenlines,
nobibnotes, showkeys, showpacs,
superscriptaddress,
%preprint,%
%author-year,%
%author-numerical,%
]{revtex4-1}

\newcommand{\version}{2modes ver. 2.3, Nov 11 2014}
% Predrag                   ver. 2.3, Nov 11 2014
% Predrag                   ver. 2.2, Jul 24 2014
% Predrag                   ver. 2.1, Jul 14 2014
% Burak                     ver. 2.0, Jul  8 2014
% Predrag                   ver. 1.3, May 11 2014
% Burak                     ver. 1.2, May  6 2014
% Predrag                   ver. 1.1, Nov 16 2013
% Burak                     ver. 1.0, Oct  6 2013
% Predrag                   ver. 0.3, Aug  1 2012
% Predrag                   ver. 0.2, Apr 30 2012}
% Predrag from atlas12      ver. 0.1, Apr 25 2012}

        \input setup2modes
        \input def2modes

\begin{document}

\title[Periodic orbit analysis of a system with continuous symmetry]
{Periodic orbit analysis of a system with continuous symmetry - a tutorial}

\author{Nazmi Burak Budanur}
\email{budanur3@gatech.edu}
\affiliation{
 School of Physics and Center for Nonlinear Science,
 Georgia Institute of Technology,
 Atlanta GA 30332
}
\author{Daniel Borrero-Echeverry}
\affiliation{
 School of Physics and Center for Nonlinear Science,
 Georgia Institute of Technology,
 Atlanta, GA 30332
}
\affiliation{
 Department of Physics,
 Reed College,
 Portland OR 97202
}
\author{Predrag Cvitanovi\'{c}}
\affiliation{
 School of Physics and Center for Nonlinear Science,
 Georgia Institute of Technology,
 Atlanta GA 30332
}
    \ifdraft
\date{\today}
    \else
\date{14 November 2014}
   \fi

\begin{abstract}
Dynamical systems with translational or rotational symmetry arise
frequently in studies of spatially extended physical systems, such as
Navier-Stokes flows on periodic domains. In these cases, it is natural to
express a state of the fluid in terms of a Fourier series truncated to a
finite number of modes. Here we study a 4-dimensional two-mode
SO(2)-equivariant model of this type, the smallest possible truncation
that retains the symmetry while remaining high-dimensional enough to
allow for chaotic dynamics. A crucial step in analysis of such a system
is symmetry reduction. We use the model to illustrate different
symmetry-reduction schemes. Its relative equilibria are conveniently
determined by rewriting the dynamics in terms of a symmetry-invariant
polynomial basis. However, for study of chaotic dynamics, the `method of
slices', applicable also to very high-dimensional problems, is
preferable. We show that a Poincar\'e section within the `slice' can be
used to further reduce this flow to what is for all practical purposes a
unimodal map. This enables us to systematically determine all relative
periodic orbits and their symbolic dynamics up to any desired period. We
then compute several dynamical averages using relative periodic orbits
and discuss the convergence of such computations.
\end{abstract}

\pacs{02.20.-a, 05.45.-a, 05.45.Jn, 47.27.ed, 47.52.+j, 83.60.Wc}
\keywords{
symmetry reduction,
equivariant dynamics,
relative equilibria,
relative periodic orbits,
periodic orbit theory,
slices,
moving frames, chaos
}
\maketitle

\begin{quotation}
Periodic orbit theory provides a way to compute dynamical averages for
chaotic flows by means of {\cycForm s} that relate the time averages of
observables to the spectra of unstable periodic orbits. Standard
{\cycForm s} are valid under the assumption that the stability
multipliers of all periodic orbits have a single marginal direction
corresponding to time evolution, and are hyperbolic in all other
directions. However, if a dynamical system has $N$ continuous symmetries,
periodic orbits are replaced by relative periodic orbits, invariant
$(N+1)$-dimensional tori with marginal stability in $(N+1)$ directions.
These exact invariant solutions arise in studies of turbulent flows, such
as the pipe flow or the plane Couette flow, where the translational
symmetries of the flow are approximated by carrying out simulations in
periodic domains. A state of the fluid can then be expressed as a Fourier
series, truncated to a large but finite (from tens to thousands) set of
Fourier modes. This paper is a tutorial on how such problems are to be
analyzed using periodic orbit theory. We illustrate all steps of
\rpo s analysis on what is arguably the simplest such dynamical system, a
`\twomode' model.
\end{quotation}

\section{Introduction}
\label{s:intro}

Recent experimental observations of travelling waves in pipe flows have
confirmed dynamical systems theory intuition that the invariant solutions
of \NSe\ play an important role in shaping the \statesp\ of turbulent
flows\rf{science04}. When one recasts fluid flow equations in a
particular basis, the outcome is an infinite dimensional dynamical system
that is often equivariant under transformations such as
translations, reflections and rotations. For example, when a periodic
boundary condition is imposed along the streamwise direction, equations
for pipe flow retain their form under streamwise translations, azimuthal
rotations and reflections about the central axis, \ie, they are invariant
under the actions of $\SOn{2} \times \On{2}$. In that case it is natural
that the states of the fluid be expressed in a Fourier basis. However,
as the system evolves the nonlinear terms in the equations mix the
Fourier modes, and thus the state of the system evolves both along
symmetry directions and directions transverse to them.
This phenomenon in dynamical systems with continuous symmetries
complicates their \statesp, and gives rise to the high dimensional coherent
solutions such as \reqva\ and \rpo s, which take on the roles played by
\eqva\ and \po s in flows without symmetry.

This paper is a tutorial that works through an example in order to
illustrate step-by-step how periodic orbit theory is applied to flows
with continuous symmetries, an analysis that should ultimately be applied
to turbulent flows, once sufficiently many exact invariant solutions
become numerically accessible. For this purpose, we study a \twomode\
\SOn{2} equivariant flow, of minimal dimensionality required for chaotic
dynamics. The paper is organized as follows: In \refsect{s:symm} we
define basic concepts and briefly review the relevant symmetry reduction
literature. In \refsect{s:twoMode}, we introduce the \twomode\ model
system, describe several of its representations, and utilize a
symmetry-reduced polynomial representation to find the only \reqv\ of the
system. In \refsect{s:numerics}, we show how the \mslices\ can be used to
quotient the symmetry and reduce the dynamics onto a symmetry-reduced
\statesp, or `\slice'. A Poincar\'e section within the \slice\ then
reduces the 4\dmn\ chaotic dynamics in the full \statesp\ to a
1-dimensional unimodal Poincar\'e return map. This return map is then
used to construct a finite grammar symbolic dynamics for the flow, and
determine {\em all} \rpo s up to a given period. In \refsect{s:DynAvers},
these orbits are used as input to various {\cycForm s} in order to calculate
dynamically interesting observables. In \refsect{s:concl},
we discuss possible applications of the \mslices\ to various spatially
extended systems. \refAppe{s:newton} describes the multi-shooting method used 
to calculate the \rpo s. \refAppe{s:schur} discusses how periodic Schur 
decomposition can be used to determine their Floquet multipliers, which can 
easily differ by 100s of orders of magnitude even in a model as simple as the 
\twomode\ system.

\section{Continuous symmetries}
\label{s:symm}

A dynamical system $\dot{\ssp}=\vel(\ssp)$ is said to be \emph{equivariant} 
or \emph{\Group-equivariant} under the symmetry group \Group\ transformations 
if
\beq
	\vel( \ssp )
    =  \matrixRep(\LieEl)^{-1}\vel(\matrixRep(\LieEl)\ssp)
	\,
\ee{equiv}
for every \statesp\ point $\ssp \in \pS$ and every element $\LieEl \in
\Group$, where \LieEl\ is an abstract group element, and 
$\matrixRep(\LieEl)$ is its $[d\!\times\!d]$ matrix representation.
Infinitesimally, the equivariance condition \refeq{equiv} is expressed as
a vanishing Lie derivative\rf{DasBuch}
\beq
  \Lg \, \vel(\ssp)  - \Mvar(\ssp) \, \groupTan(\ssp) =0
  \,,
\ee{inftmInv}
where
$\Mvar(\ssp)$ is the $[d\!\times\!d]$ \stabmat\, with elements
$\Mvar_{ij}(\ssp)={\pde \vel_i}/{\pde\ssp_j}|_{\ssp}$, $ \groupTan(\ssp)
= \Lg \ssp $ is the group tangent at $\ssp$, and $\Lg$ is the
$[d\!\times\!d]$ generator of infinitesimal transformations, such that
$\matrixRep(\theta) = \exp(\theta\Lg)$, with phase $\theta \in [0,2\pi)$
parametrizing the group action. (We shall interchangeably use notations
$\matrixRep(\LieEl)$ and $\matrixRep(\theta)$.) In general, there is a
generator associated with each continuous symmetry direction, but as in
the $\SOn{2}$ example studied here there is only one parameter $\theta$,
we have only one generator \Lg.

If the trajectory of a point $\ssp_\stagn$ coincides with its group
orbit, \ie, for every $\zeit$ there is a group transformation such that
\beq
\ssp (\zeit)
    = \ssp_\stagn + \int_0^\zeit \!\!d\zeit' \vel(\ssp (\zeit'))
    = \matrixRep(\theta (\zeit))\,\ssp_\stagn
  \,,
\ee{releq}
$\ssp_\stagn$ is a point on \emph{\reqv} $\stagn$, here a 1-torus in the
\statesp. Expanding both sides of \refeq{releq} for infinitesimal time
verifies that the group orbit tangent and the time evolution trajectory
tangent vectors are parallel,
$\vel(\ssp_\stagn) = \dot{\theta}(0) \, \groupTan(\ssp_\stagn)$.
By symmetry, this must hold for all $\ssp(\zeit) \in q$, so for a \reqv\
the \emph{\phaseVel} is a constant, $\dot{\theta}(\zeit) = \velRel$.
Multiplying the equivariance condition \refeq{inftmInv} by $\velRel$ we
find that velocity is a marginal stability eigenvector in \reqv\
co-moving frame,
\beq
(\Mvar (\ssp) - \velRel \Lg) \vel (\ssp) = 0
\,,\qquad \ssp \in \pS_\stagn
\,.
\ee{ReqvMargEig}

A \statesp\ point $\ssp_\rpprime$ lies on a \emph{\rpo} of period
$\period{\rpprime}$ if its trajectory first intersects its group orbit after
a finite time $\period{\rpprime}$,
\beq
\ssp(\period{\rpprime})
    = \ssp_\rpprime
     + \int_0^\period{\rpprime} \!\!\!\!d\tau' \vel(\ssp (\tau'))
    = \matrixRep(-\theta_\rpprime) \,  \ssp_\rpprime
  \,,
\ee{relpo}
with a non-zero phase $\theta_{\rpprime}$. In systems with \SOn{2}
symmetry, \rpo s are topologically 2-tori, with the trajectory of
$\ssp_\rpprime$ generically ergodically tracing out the torus by
repeating the same path, shifted by the group action
$\matrixRep(\theta_\rpprime)$ after each prime period
$\period{\rpprime}$. As we will see in \refsect{s:numerics}, these tori
can be very convoluted objects, difficult to visualize. In case that
$\theta_{\rpprime}=0$, the solution is a \po, a 1-dimensional loop, and
the 2-torus is generated by all actions of the symmetry group on this
loop.

The linear stability of \rpo s is captured by their \emph{Floquet
multipliers}, the eigenvalues of the time-forward map
$\ssp(\zeit)=\flow{\zeit}{\ssp(0)}$ Jacobian
\beq
\jMpsRed_{\rpprime}
= \matrixRep(\theta_\rpprime ) \jMps^\period{\rpprime} (\ssp_\rpprime)
\,, \; \mbox{~where~}\;
\jMps^{\zeit}_{ij} (\ssp(0)) = \frac{\partial\ssp_i(\zeit)}{\partial\ssp_j(0)}\, .
\ee{e-rpoJacobian}
The magnitude of $\ExpaEig_{p,j}$ determines whether a small perturbation
along its corresponding eigendirection (or Floquet vector) will expand or
contract after one period. If the magnitude of $\ExpaEig_{p,j}$ is
greater than $1$, the perturbation expands; if it is less than $1$, the
perturbation contracts. In systems with $N$ continuous symmetries, \rpo s
have $(N+1)$ marginal directions ($\left|\ExpaEig_{p,j}\right| = 1$),
which correspond to the temporal evolution of the flow and the $N$
symmetries. By applying symmetry reduction, the marginal Floquet
multipliers corresponding to the symmetries are replaced by $0$ and make
periodic orbit theory, which requires that the flow have only one
marginal direction, applicable.

\emph{Symmetry reduction} is a coordinate transformation that maps all the 
points on a group orbit $\matrixRep(\theta) \ssp$, which are equivalent from 
a dynamical perspective, to a single representative point in a symmetry 
reduced space. Such a transformation converts \reqva\ and \rpo s to \eqva\ and 
\po s in a reduced \statesp, with no loss of dynamical information; the full 
\statesp\ trajectory can always be retrieved via the reconstruction equation. 
One well-studied technique for symmetry reduction, which works well for 
low-dimensional dynamical systems, such as the Lorenz system, is to recast the 
dynamical equations in terms of invariant polynomials\rf{GL-Gil07b}. 
Establishing such invariant polynomial bases, however, quickly becomes 
impractical for systems with more than a dozen dimensions\rf{gatermannHab}. In 
contrast, the \mslices\ 
\rf{rowley_reconstruction_2000,BeTh04,SiCvi10,FrCv11,atlas12,ACHKW11,BudCvi14},
which we study in detail here, is a symmetry reduction scheme applicable to
high-dimensional flows like the \NS\ equations\rf{WiShCv14}.

\subsection{\Mslices}
\label{s-slice}

In a system with $N$ continuous symmetries, a \emph{\slice} \pSRed\ is a codimension $N$ submanifold
of \pS\ that cuts every group orbit once and only once. In the \emph{\mslices}, the solution
of a $d$-\dmn\ dynamical system is represented as a symmetry-reduced trajectory $\sspRed (\zeit)$ within the
$(d-N)$-\dmn\ \slice\ and $N$ time dependent group parameters $\theta(\zeit)$, which
map $\sspRed (\zeit)$ to the full \statesp\ by the group action $\matrixRep(\theta(\zeit))$.

While this idea goes back to Cartan\rf{CartanMF},
Rowley and Marsden\rf{rowley_reconstruction_2000}
were the first to apply it to a spatially extended nonlinear flow. They used it to study the dynamics of
the $1D$ \KS\ equation in the neighborhood of
a \reqv, using the \reqv\ itself as the \slice\ `\template'.
Independently, Beyn and Th\"{u}mmler\rf{BeTh04} applied
the \mslices\ to `freeze' spiral waves in reaction-diffusion systems.

The definition given above for the \slice\ puts no restriction on its shape
and offers no guidance on how to construct it. In practice, a
local approximation of the slice called a \emph{\slicePlane} can be constructed
in the neighborhood of a point $\slicep$ by using $\slicep$ as
\emph{\template}. The \slicePlane\ is then defined as the hyperplane that
contains $\slicep$ and is perpendicular to its group tangent $\sliceTan{}
= \Lg \slicep$. The relationship between a \template\, its \slicePlane, and symmetry-reduced trajectories
is illustrated in \reffig{f-ReducTraj1}.

%% ReducTraj*.* - read dasbuch/book/FigSrc/inkscape/00ReadMe.txt
\begin{figure}
\begin{center}
 \setlength{\unitlength}{0.40\textwidth}
 %% \unitlength = units used in the Picture Environment
 \begin{picture}(1,0.8361641)%
   \put(0,0){\includegraphics[width=\unitlength]{ReducTraj5}}%
   \put(0.06854399,0.36282057){\color[rgb]{0,0,0}\rotatebox{-30.34758661}{\makebox(0,0)[lb]{\smash{$\pSRed$}}}}%
   \put(0.57768586,0.29773425){\color[rgb]{0,0,0}\rotatebox{0.0313674}{\makebox(0,0)[lb]{\smash{$\sspRed(0)$}}}}%
   \put(0.59310014,0.69932675){\color[rgb]{0,0,0}\rotatebox{0.03136739}{\makebox(0,0)[lb]{\smash{$\ssp(\zeit)$}}}}%
   \put(0.8268425,0.39772328){\color[rgb]{0,0,0}\rotatebox{0.03136739}{\makebox(0,0)[lb]{\smash{$\sspRed(\zeit)$}}}}%
   \put(0.81220962,0.66529577){\color[rgb]{0,0,0}\rotatebox{0.03136739}{\makebox(0,0)[lb]{\smash{$\matrixRep(\theta(\zeit))\ssp(\zeit)$}}}}%
   %\put(0.21150193,0.63610779){\color[rgb]{0,0,0}\rotatebox{0.0313674}{\makebox(0,0)[lb]{\smash{$\matrixRep(\theta)\\slicep$}}}}%
   \put(0.37740434,0.49597258){\color[rgb]{0,0,0}\rotatebox{0.0313674}{\makebox(0,0)[lb]{\smash{$\slicep$}}}}%
   \put(0.3627714,0.69665188){\color[rgb]{0,0,0}\rotatebox{0.0313674}{\makebox(0,0)[lb]{\smash{$\sliceTan{}$}}}}%
 \end{picture}%
\end{center}
\caption{\label{f-ReducTraj1}
(Color online) The \slicePlane\ \pSRed\ is a hyperplane % \refeq{PCsect0}
passing through the {\template} point $\slicep$ and normal to its group
tangent $\sliceTan{}$. It intersects all group orbits (dotted lines) in
an open neighborhood of $\slicep$.  The full \statesp\ trajectory
$\ssp(\tau)$ (solid black line) and the \reducedsp\ trajectory
$\sspRed(\zeit)$ (solid green line) belong to the same group orbit
$\pS_{\ssp(\zeit)}$ and are equivalent up to a group rotation
$\matrixRep(\theta(\zeit))$.
}%
\end{figure}

Reduced trajectories $\sspRed (t)$ can be obtained in two ways: by
post-processing data or by reformulating the dynamics and integrating
directly in the \slice. The post-processing method (also called the
\emph{method of moving frames}\rf{FelsOlver98,OlverInv}) can be applied
to both numerical and experimental data, one takes the data in the full
\statesp\ and looks for the time dependent group parameter that brings
the trajectory $\ssp(\zeit)$ onto the \slice. That is, one finds $\theta
(\zeit)$ such that $\sspRed(\zeit)=\matrixRep(-\theta (\zeit)) \ssp
(\zeit)$ satisfies the \slice\ condition:
\beq
\braket{\sspRed(\zeit) - \slicep}{\sliceTan{}} = 0
\,.
\ee{SliceCond}

In the second implementation, one reformulates the dynamics (for Abelian
groups) as
\begin{subequations}\label{eq:so2reduced}
  \beq\label{eq:intSlice}
	\velRed(\sspRed) = \vel(\sspRed)
	-\dot{\theta}(\sspRed) \, \groupTan(\sspRed)
  \eeq
  \beq\label{eq:reconstruction}
	\dot{\theta}(\sspRed) = {\braket{\vel(\sspRed)}{\sliceTan{}}}/
				{\braket{\groupTan(\sspRed)}{\sliceTan{}}}
  \, ,
  \eeq
\end{subequations}
and directly calculates the symmetry-reduced trajectory directly by 
integrating $\sspRed (\zeit)$ and $\theta (\zeit)$. In 
\refeq{eq:so2reduced}, $\velRed$ is the projection of the full \statesp\ 
velocity \vel(\ssp) onto the \slicePlane. For a derivation of 
\refeq{eq:so2reduced}, see \refref{DasBuch}.

While early studies\rf{rowley_reconstruction_2000, rowley_reduction_2003,
BeTh04} applied the \mslices\ to a single solution at a time, studying
the nonlinear dynamics of extended systems requires symmetry reduction of
global objects, such as strange attractors or invariant manifolds. In
this spirit, Siminos and Cvitanovi\'{c}\rf{SiCvi10} used the \mslices\ to
quotient the \SOn{2} symmetry from the chaotic dynamics of \cLf. They
showed that the slice-dependent singularity of the reconstruction
equation that occurs when the denominator in \refeq{eq:reconstruction}
vanishes (e.g., when the group tangents of the trajectory and the template 
are orthogonal) causes the reduced flow to make discontinuous jumps. This 
set of points $\sspRed^*$ in the \slicePlane\ that satisfy
\beq
\braket{\groupTan(\sspRed^*)}{\sliceTan{}} = 0
\ee{ChartBordCond}
forms the \emph{\sliceBord}, studied in detail by Froehlich and Cvitanovi\'{c}\rf{FrCv11}.

Two strategies have been proposed in order to handle this problem: The first attempts to
try to identify a template such that slice singularities are not visited
by the dynamics\rf{SiCvi10} or to use multiple `charts' of connected
slices\rf{rowley_reconstruction_2000,FrCv11}.
The latter approach was applied to \cLf\ by Cvitanovi\'{c} \etal~\rf{atlas12} and
to pipe flow by Willis, Cvitanovi\'{c}, and Avila\rf{ACHKW11}.
However, neither approach is straightforward to apply, particularly in
high-dimensional dynamical systems.

\subsection{\FFslice}
\label{sect:fFslice}

A third strategy has recently been proposed by Budanur
\etal\rf{BudCvi14}, who considered Fourier space discretizations of
partial differential equations (PDEs) with \SOn{2} symmetry. They showed
that in these cases a simple choice of \slice\ template, associated with
the first Fourier mode, results in a \slice\ in which it is highly
unlikely that generic dynamics visit the neighborhood of the singularity.
If the dynamics do occasionally come near the singularity, these close
passages can be regularized by means of a time rescaling.

Here, we shall illustrate this approach, which we call the `\fFslice',
and apply it to a 2-mode model system described in \refsect{s:twoMode},
arguably the simplest system with \SOn{2} equivariant dynamics that can
exhibit chaos.

In the discussion so far, we have not specified any constraints on the 
symmetry group to be quotiented, beyond the requirement that it be Abelian, 
which is required for \refeq{eq:so2reduced} to be valid. Since we are 
interested in spatially extended systems with translational symmetry, and in 
order to keep the notation compact, we restrict our discussion to one 
dimensional PDEs describing the evolution of a field $u(x,t)$ in a periodic 
domain. By introducing a Fourier series expansion
\beq
	u(x,\zeit) = \sum\limits_{k=- \infty}^\infty u_k\left(\zeit\right) 
				 e^{i k x}, \,\,\,u_k = x_k + i y_k,
\ee{FourierSeries}
a PDE invariant under translations can be expressed as a system of coupled 
nonlinear ODEs equivariant under the 1-parameter compact group of \SOn{2} 
rotations.

Truncating the expansion to $m$ modes, we write the real and imaginary parts 
of the Fourier coefficients with $k \geq 1$ as the state vector 
$\ssp=\cartpt{x_1, y_1, x_2, y_2,..., x_m, y_m}$. The action of the $\SOn{2}$ 
group on this vector can then be expressed as a block diagonal matrix:
\beq
	\matrixRep(\theta) = \begin{pmatrix}
						R(\theta) & 0 			  & \cdots & 0 \\
						0		   & R(2 \theta) & \cdots & 0 \\
						\vdots	   & \vdots 	  & \ddots & \vdots \\
						0		   & 0	          & \cdots & R (m \theta)
					   \end{pmatrix}
\,,
\ee{mmodeLieEl}
where
\beq
	R(n \theta) =	\begin{pmatrix}
					\cos n \theta & - \sin n \theta \\
					\sin n \theta & ~\cos n \theta
					\end{pmatrix}
\ee{rotationmatrix}
is the rotation matrix for $n$th Fourier mode.
The Lie algebra generator for $\matrixRep(\theta)$ is given by
\beq
	 \Lg =  \begin{pmatrix}
			 0 & -1 & 0 & 0 & \cdots & 0 & 0 \\
			 1 & 0 & 0 & 0 & \cdots & 0 & 0 \\
			 0 & 0 & 0 & -2 & \cdots & 0 & 0 \\
			 0 & 0 & 2 & 0 & \cdots & 0 & 0 \\
			 \vdots & \vdots & \vdots & \vdots & \ddots & \vdots & \vdots \\
			 0 & 0 & 0 & 0 & \cdots & 0 & -m \\
			 0 & 0 & 0 & 0 & \cdots & m & 0
			 \end{pmatrix} .
\ee{mmodeLg}

In order to construct a \slicePlane\ for such a system, let us choose the
following \slice\ \template:
\beq
	\slicep = (1, 0, ..., 0) .
\ee{firstmodetemp}
The \slice\ condition \refeq{SliceCond} then constraints points on the
reduced trajectory to the hyperplane given by
\beq
	\sspRed = (\hat{x}_1, 0, \hat{x}_2, \hat{y}_2, ..., \hat{x}_m, \hat{y}_m) .
\ee{slicetemp}
As discussed earlier, group orbits should cross the \slice\ once and only
once, which we achieve by restricting the \slicePlane\ to the half-space
where $\hat{x}_1 > 0$. In general, a \slicePlane\ can be constructed by
following a similar procedure for any choice of \template, allowing the
symmetry reduction of the dynamics in a neighborhood of the \template\
bounded by the \sliceBord\ \refeq{ChartBordCond}. However, the power of
choosing template \refeq{firstmodetemp} becomes apparent by computing the
border of its \slicePlane. The points on \refeq{slicetemp} lie on the
\sliceBord\ only if $\hat{x}_1 = 0$. This means that as long the dynamics
are such that the magnitude of the first mode never vanishes,
\emph{every} group orbit is guaranteed to have a unique representative
point on the \slicePlane. By symmetry, any template of the form $\slicep
=\cartpt{\hat{x}'_1, \hat{y}'_1, 0,...,0}$  would work just as well. The
\slice\ \template\ \refeq{firstmodetemp} was chosen for notational and
computational convenience.

More insight can be
gained by writing the symmetry-reduced evolution equations 
\refeq{eq:so2reduced} explicitly for the template \refeq{firstmodetemp}:
\begin{subequations}
\beq
\velRed ( \sspRed )  = \vel(\sspRed)
   - \frac{\dot{y}_1\left(\sspRed\right)}{\hat{x}_1} \, \groupTan(\sspRed) \, ,
\label{e-so2red1stmode}
\eeq
\beq\label{eq:reconstruction1stmode}
  \dot{\theta}(\sspRed) = \frac{\dot{y}_1(\sspRed)}{\hat{x}_1}
  \, .
\eeq
\end{subequations}
Since the argument $\phi_1$ of a point $(x_1,y_1)$ in the first Fourier mode 
plane is given by $\phi_1=\tan^{-1}\frac{y_1}{x_1}$, its velocity is
\beq
  \dot{\phi}_1 = \frac{x_1}{r_1^2}\dot{y}_1-\frac{y_1}{r_1^2}\,\dot{x}_1\,,
\eeq
where $r_1^2=x_1^2+y_1^2$. Therefore, on the \slicePlane \refeq{slicetemp}, 
where $\hat{y}_1=0$,
\beq\label{eq:phi1}
  \dot{\theta}(\sspRed) = \dot{\phi}_1(\sspRed)\,.
\eeq
That is, for our choice of \template\ \refeq{firstmodetemp}, the
reconstruction phase coincides with the first Fourier mode phase. In this
sense, from a group-theoretic point of view, this choice of template is
thus more natural than the physically motivated templates used in
\refrefs{rowley_reconstruction_2000,BeTh04,SiCvi10,FrCv11,atlas12,ACHKW11}.

In general, additional care must be taken when the dynamics approach the 
\slice\ border $\hat{x}_1 = 0$. Whenever this happens, the near-divergence of 
$\velRed$ can be regularized by introducing a rescaled time coordinate such 
that $d\hat{\zeit} = d\zeit / \hat{x}_1$\rf{BudCvi14}. However, in our study 
of the \twomode\ system that we will introduce below, we omit this step since 
points with a vanishing first mode are in an invariant subspace of the flow 
and hence are never visited by the dynamics.

\subsection{Post-processing approach}
\label{s-mframes}

\begin{figure}%[H]
\centering
 \includegraphics[width=0.45\textwidth]{BBgorbitsandslice}
\caption{(Color online)
$\SOn{2}$ group orbits of \statesp\ points $\cartpt{0.75, 0, 0.1, 0.1}$
(orange), $\cartpt{0.5, 0, 0.5, 0.5}$ (green)
$\cartpt{0.1, 0, 0.75, 0.75}$ (pink) and the first mode \refeq{slicetemp} 
\slicePlane\ (blue). The group tangents at the intersections with the
\slicePlane\ are shown as red arrows.
As the magnitude of the first Fourier mode decreases relative to the
magnitude of the second one, so does the group tangent angle to the
\slicePlane.}
\label{fig:BBgorbitsandslice}
\end{figure}

In \refsect{s-slice}, we explained the general procedure for reducing the 
\SOn{2} symmetry by \mslices. Here, we focus on its geometrical 
interpretation. The \slice\ defined by \refeq{firstmodetemp} and the 
directional constraint $\hat{x}_1 > 0$ fixes the phase of the first complex 
Fourier mode to $0$. This also follows from the fact that the reconstruction 
phase of the first Fourier mode \slice\ is the phase of the first mode 
In complex representation, we can express the relationship between Fourier 
modes ($\sspC_n = x_n + \ii y_n$) and their representative points 
($\sspRedC = \hat{x}_n +  \ii \hat{y}_n$) on the \slicePlane\ by the $\Un{1}$ 
action:
\beq
	\sspRedC_n = e^{-\ii n \phi_1} \sspC_n \, . \ee{e-1stmodeTransform}
This relation provides another interpretation for the \sliceBord:  For
template \refeq{firstmodetemp}, the \sliceBord\ condition
\refeq{ChartBordCond} yields, $|\sspRedC_1| = |\sspC_1| = 0$, which means
that the phase of the first Fourier mode and hence the
transformation \refeq{e-1stmodeTransform} is not well-defined.
This is illustrated in \reffig{fig:BBgorbitsandslice}, which shows the
first Fourier mode \slicePlane\ along with three-dimensional projections
of the group orbits of points with decreasing $|\sspC_1|$. When the
magnitude of the first mode $\sqrt{\hat{x}_1^2 + \hat{y}_1^2}$, relative
to that of the second mode is small (pink curve in
\reffig{fig:BBgorbitsandslice}), the group tangent has a larger component
parallel to the \slicePlane . If the first mode magnitude was exactly
$0$, the group tangent would lie entirely on the \slicePlane , satisfying
the \sliceBord\ condition.

In \refref{PoKno05}, a polar coordinate representation of two Fourier mode
truncation is obtained by defining the $\Group$-invariant phase: 
$\Phi = \phi_2 - 2 \phi_1$
and three symmetry invariant coordinates 
\polpt{r_1, r_2 \cos \Phi, r_2 \sin \Phi}.
One can see by direct comparison with \refeq{e-1stmodeTransform}, which
yields $\sspRedC_1 = r_1$ and $\sspRedC_2 = r_2 e^{\ii \Phi}$, that this
representation is a special case $(m=2)$, of the \slice\ defined by
\refeq{firstmodetemp}. Corresponding ODEs for the polar representation
were obtained in \refref{PoKno05} by  chain rule and substitution. Note
that \mslices\ provides a general form \refeq{e-so2red1stmode} for symmetry
reduced time evolution.

\section{\twoMode\ $\SOn{2}$-equivariant flow}
\label{s:twoMode}

Dangelmayr,\rf{Dang86} Armbruster, Guckenheimer and Holmes,\rf{AGHO288}
Jones and Proctor,\rf{JoPro87} and Porter and Knobloch\rf{PoKno05} (see
Golubitsky \etal\rf{golubII}, Sect. XX.1) have investigated bifurcations
in 1:2 resonance ODE normal form models to third order in the amplitudes.
Here, we use this model as a starting point from which we derive what may
be one of the simplest chaotic systems with a continuous symmetry. We refer 
to this as the {\twomode} system:
\bea
	\dot{z}_1 &=& (\mu_1-\ii\, e_1)\,z_1+a_1\,z_1|z_1|^2
				 +b_1\,z_1|z_2|^2+c_1\,\overline{z}_1\,z_2
	\continue
	\dot{z}_2 &=& (\mu_2-\ii\, e_2)\,{z_2}+a_2\,z_2|z_1|^2
				 +b_2\,z_2|z_2|^2+c_2\,z_1^2 \,,
	\label{eq:DangSO2}
\eea
where $z_1$ and $z_2$ are complex and all parameters real-valued. The 
parameters $\{e_1,e_2\}$ break the $\On{2}$ symmetry of the normal form 
studied by Dangelmayr\rf{Dang86} leading to an $\SOn{2}$-equivariant system. 
As we will show below in \refeq{PKinvEqs1}, only the combination $(2e_1-e_2)$ 
matters in the symmetry reduced dynamics, so for simplicity we set $e_1=0$. 
This complex two mode system can be expressed as a 4-dimensional first order 
real ODE system by substituting $z_1 = x_1 + i\,y_1$, $z_2 = x_2 + i\,y_2$, 
so that 
\bea
\dot{x}_1 &=& (\mu_1 + a_1 r_1^2 + b_1 r_2^2 + c_1 x_2)x_1 
			  + c_1 y_1 y_2 + e_1 y_1 \, ,
\continue
\dot{y}_1 &=& (\mu_1 + a_1 r_1^2 + b_1 r_2^2 - c_1 x_2)y_1 
			  + c_1 x_1 y_2 - e_1 x_1 \, ,
\continue
\dot{x}_2 &=& (\mu_2 + a_2 r_1^2 + b_2 r_2^2)x_2 
			  + c_2 (x_1^2 - y_1^2) + e_2 y_2 \, ,
\label{2mode4D}
\continue
\dot{y}_2 &=& (\mu_2 + a_2 r_1^2 + b_2 r_2^2)y_2 
			  + 2 c_2 x_1 y_1 - e_2 x_2 \, ,
\continue
		  && \mbox{where } r_1^2 = x_1^2 + y_1^2\, , \quad r_2^2 = x_2^2 + y_2^2
\,.
\eea
This normal form has a large set of parameters 
$\left(\mu_1,\mu_2,a_1,a_2,b_1,b_2,c_1,c_2,e_1,e_2\right)$. Following in the 
tradition of Lorenz\rf{lorenz}, H\'enon\rf{henon} and R\"ossler\rf{ross}, we 
have played with various choices of parameters until settling on the following 
set of values, which we will use in all numerical calculations presented here:
\beq
	\begin{tabular}{c c c c c c c c c c}
	 $\mu_1$ & $\mu_2$ & $e_1$ & $e_2$ & $a_1$ & $a_2$ & $b_1$ & $b_2$ & $c_1$ & $c_2$ \\
	\hline
	 -2.8	& 1		  & 0	  & 1	  & -1	  & -2.66 & 0	  & 0 	  & -7.75 & 1
	\end{tabular}
	\label{eq:pars}
\eeq
This choice of parameters is far from the bifurcation values studied by 
previous authors\rf{Dang86,AGHO288,JoPro87,PoKno05}, so that the model has no 
physical interpretation. However, this set of parameters yield chaotic 
dynamics, making the two-mode system a simple model for the study of chaos in 
the presence of a continuous symmetry: It is a 4\dmn\ $\SOn{2}$-equivariant 
model in which the three dimensional $\SOn{2}$-reduced dynamics are chaotic.

It can be checked by inspection that eqs.~\refeq{eq:DangSO2} are equivariant 
under the \Un{1}\ transformation
\beq
(z_1,z_2) \rightarrow   (e^{i {\gSpace}}z_1,e^{i 2{\gSpace}} z_2)
\,.
\ee{Dang86(1.1)aa}
In the real representation \refeq{2mode4D}, the $\SOn{2}$ group action
\refeq{Dang86(1.1)aa} is given by $\ssp'= \exp\left( \theta \Lg\right)\ssp$,
where $\transp{\ssp} =\cartpt{x_1, y_1,x_2, y_2}$, and $\Lg$ is the Lie algebra
generator
\beq
\Lg  \, =
\left( \begin{array}{cccc}
         0 & -1 & 0 & 0 \\
         1 & 0 & 0 & 0 \\
         0 & 0 & 0 & -2\\
         0 & 0 & 2 & 0
      \end{array} \right)
\,.
\ee{LGTwoMode}
One can easily check that the real \twomode\ system \refeq{2mode4D}
satisfies the equivariance condition \refeq{inftmInv}.

From \refeq{eq:DangSO2}, it is obvious that the \eqv\ point \((z_1,z_2)=(0,0)\)
is an invariant subspace, and that $z_1=0$, $z_2 \neq 0$ is a 2\dmn\
flow-invariant subspace,
\beq
  \dot{z}_1 = 0 
\,,\qquad
  \dot{z}_2 = (\mu_2-\ii\, e_2 +b_2 |z_2|^2)\,{z_2} 
\,,
\ee{eq:DangSO2spsp}
with a single circular \reqv\ of radius 
$r_2 = \norm{z_2} = \sqrt{-\mu_2/b_2}$ with \phaseVel\ $\velRel=-e_2/2$. At 
the origin the stability matrix $\Mvar$ commutes with $\Lg$, and thus can be 
block-diagonalized into two $[2\!\times\!2]$ matrices. The eigenvalues of 
$\Mvar$ at $\cartpt{0,0,0,0}$ are $\Lyap_1 = \mu_1$ with multiplicity 2 and
$\Lyap_2 = \mu_2 \pm i e_2$. The eigenvectors for $\Lyap_1$ are 
$\cartpt{1,0,0,0}$ and $\cartpt{0,1,0,0}$ in the $\cartpt{x_1,y_1,x_2,y_2}$ 
coordinates. The eigenvectors for $\Lyap_2$ are $\cartpt{0,0,1,0}$ and 
$\cartpt{0,0,0,1}$.

In contrast, $z_2 =0$ is not, in general, a flow-invariant subspace, since the dynamics
\[
  \dot{z}_1 = (\mu_1-\ii\, e_1)\,z_1+a_1\,z_1|z_1|^2
\,,\qquad
  \dot{z}_2 = c_2\,z_1^2
\,.
\]
takes the flow out of the $z_2 =0$ plane.

\subsection{Invariant polynomial bases}
\label{s:invPol}

Consider the \statesp\ of a dynamical system constructed from two complex
Fourier modes\rf{Dang86,AGHO288,PoKno05} $m=(1,2)$, with the $\SOn{2}
\simeq \Un{1}$ group action given by \refeq{Dang86(1.1)aa}. In this case,
it is easy to construct a set of four real-valued $\SOn{2}$ invariant
polynomials
\bea
u &=& {z}_1 \overline{z}_1
    \,,\quad
v = {z}_2 \overline{z}_2
    \continue
w &=& z_1^2 \overline{z}_2 + \overline{z}_1^2 {z}_2
    \,,\quad
q = (z_1^2 \overline{z}_2 - \overline{z}_1^2 {z}_2)/\ii
\,.
\label{Dang86(1.2)PK}
\eea
The polynomials $\invpt{u,v,w,q}$ are linearly independent, but related 
through one syzygy,
\beq
w^2+q^2 - 4\,u^2v = 0 
\label{eq:syzPK}
\eeq
that confines the dynamics to a 3-dim\-ens\-ion\-al manifold 
$\pSRed=\pS/\SOn{2}$, a symmetry-invariant repre\-sent\-ati\-on of the 
4-dim\-ens\-ion\-al \SOn{2} equivariant dynamics, which we call the 
\reducedsp. By construction, $u \geq 0$, $v \geq 0$, but $w$ and $q$ can be of 
either sign. That is explicit in in polar coordinates 
$ {z}_1 = |u|^{1/2} e^{\ii\phi_1}$, $ {z}_2 = |v|^{1/2} e^{\ii\phi_2}$, where 
the  $w, q$ invariants take the form
\bea
w &=& 2\,\Re(z_1^2 \overline{z}_2) = 2\,u |v|^{1/2} \cos \psi 
\continue
q &=& 2\,\Im(z_1^2 \overline{z}_2) = 2\,u |v|^{1/2} \sin \psi 
\,,
\label{Dang86(1.2)polar}
\eea
where $\psi = 2 \phi_1 - \phi_2$.

The dynamical equations for $\invpt{u,v,w,q}$ follow from the chain rule, which yields
\bea
  \dot{u} &=& \overline{z}_1 \dot{z}_1 + {z}_1 \dot{\overline{z}}_1 
\,,\qquad
  \dot{v} = \overline{z}_2 \dot{z}_2 + {z}_2 \dot{\overline{z}}_2 
\continue
  \dot{w} &=& 2 \,\overline{z}_2 {z}_1 \dot{z}_1 
           + 2\,{z}_2 \overline{z}_1 \dot{\overline{z}}_1
           + {z}_1^2 \dot{\overline{z}}_2
           + \overline{z}_1^2 \dot{z}_2
\continue
  \dot{q} &=&  (2\,\overline{z}_2 {z}_1 \dot{z}_1 
           - 2\,{z}_2 \overline{z}_1 \dot{\overline{z}}_1
           + {z}_1^2 \dot{\overline{z}}_2
           - \overline{z}_1^2 \dot{z}_2
           )/\ii
\label{PKinvEqs}
\eea
Substituting \refeq{eq:DangSO2} into \refeq{PKinvEqs}, we obtain a set of four 
$\SOn{2}$-invariant equations,
\bea
  \dot{u} &=& 2\,\mu_1\,u+2\,a_1\,u^2+2\,b_1\,u\,v+c_1\,w 
\continue
  \dot{v} &=& 2\,\mu_2\,v+2\,a_2\,u\,v+2\,b_2\,v^2+c_2\,w 
\continue
  \dot{w} &=& (2\,\mu_1+\mu_2)\,w+(2a_1+a_2)\,u\,w+(2b_1+b_2)\,v\,w 
\ceq
             +\, 4c_1\,u\,v + 2c_2\,u^2 +(2e_1 - e_2)\,q
\label{PKinvEqs1}\\
  \dot{q} &=& (2\mu_1+\mu_2)\,q+(2a_1+a_2)\,u\,q
\ceq
             +(2b_1+b_2)\,v\,q
             -(2e_1-e_2)\,w 
\,.
\nnu
\eea
Note that the $\On{2}$-symmetry breaking parameters $\{e_1,e_2\}$ of the
Dangelmayr normal form system\rf{Dang86} appear only in the relative phase 
combination $(2e_1-e_2)$. Using the syzygy \refeq{eq:syzPK}, we can eliminate 
$q$ from \refeq{PKinvEqs1} to get
\bea
  \dot{u} &=& 2\,\mu_1\,u+2\,a_1\,u^2+2\,b_1\,u\,v+c_1\,w \nonumber 
\\
  \dot{v} &=& 2\,\mu_2\,v+2\,a_2\,u\,v+2\,b_2\,v^2+c_2\,w \label{PKinvEqs1syz}  
\\
  \dot{w} &=& (2\,\mu_1+\mu_2)\,w+(2a_1+a_2)\,u\,w+(2b_1+b_2)\,v\,w 
\ceq
             +\, 4c_1\,u\,v + 2c_2\,u^2 +(2e_1 - e_2)(4u^2v-w^2)^{1/2}\,
  \nonumber
\eea
This invariant basis can be used either to investigate the dynamics directly 
or to visualize solutions\rf{GL-Gil07b} computed in the full equivariant basis 
\refeq{eq:DangSO2}. While representations of our model in terms of invariant 
polynomials \refeq{PKinvEqs1} and polar coordinates \refeq{Dang86(1.2)polar} 
are useful for cross-checking calculations in the full \statesp\ 
$\transp{\ssp} =\cartpt{x_1, x_2,y_1, y_2}$ , construction requires a bit of 
algebra even for this simple 4-dimensional flow. For very high\dmn\ flows, 
such as \KS\ and \NS\ flows, we do not know how to carry out such 
constructions.

\subsection{\Eqva\ of the symmetry-reduced dynamics}
\label{s:eqva}

The first step in elucidating the geometry of attracting sets is the
determination of their \eqva. We shall now show that the problem of
determining the \eqva\ of the symmetry-reduced \twomode\
\refeq{PKinvEqs1} system $\invpt{u^*,v^*,w^*,q^*}$ can be reduced to
finding the real roots of a multinomial expression. First, let we define
\beq
A_1= \mu_1+a_1\,u+b_1\,v
    \,,\qquad
A_2 = \mu_2+a_2\,u+b_2\,v
\ee{PKinvEqs2a}
and rewrite \refeq{PKinvEqs1} as
\bea
  0  &=&  2\,A_1\,u +c_1\,w 
    \,,\qquad
  0  =  2\,A_2\,v +c_2\,w 
\continue
  0  &=& (2\,A_1+ A_2)\,w
          +2\,\left(c_2\,u+2\,c_1\,v\right)\,u 
          \ceq
		  + (2e_1-e_2)\,q
\label{PKinvEqs3}\\
  0  &=& (2\,A_1+ A_2)\,q - (2e_1-e_2)\,\,w 
\nnu
\eea
We already know that $\invpt{0,0,0,0}$ and $\cartpt{0,-\mu_2/b_2,0,0}$ are the 
only roots in the $u=0$ and $v=0$ subspaces, so we are looking only for the 
$u>0$, $v>0$, $w,q \in \reals$ solutions; there could be non-generic roots 
with either $w=0$ or $q=0$, but not both simultaneously, since the syzygy 
\refeq{eq:syzPK} precludes that. Either $w$ or $q$ can be eliminated by 
obtaining the following relations from \refeq{PKinvEqs3}:
\bea
	w  &=& - \frac{2\,u}{c_1}\,A_1 = - \frac{2\,v}{c_2}\,A_2 
	\continue
	q &=& \frac{2(-2e_1+\,e_2)\,u\,v}{c_2\,u+2\,c_1\,v} .  root.
	\label{PKinvEqs4}
\eea
Substituting \refeq{PKinvEqs4} into \refeq{PKinvEqs3} we get two bivariate
polynomials whose roots are the \eqva\ of the system \refeq{PKinvEqs1}:
\bea
	f(u,v) &=& c_2\,u\,A_1 - c_1\,v\,A_2 = 0 \,,\qquad  \nonumber
	\\
	g(u,v) &=&
 \left(4\,A_1^2 u^2 - 4\,c_1^2\,u^2 v\right)\left(c_2\,u+2\,c_1\,v\right)^2 \label{PKinvEqs5} 
	\ceq
	+\,4\,c_1^2\,(-2e_1+e_2)^2\,u^2\,v^2 = 0
\,,
	\\
	\mbox{\rm deg}(f) &=& 2, \, \mbox{\rm deg}(g) = 6 \nonumber
\,.
\eea
We divide the common multiplier $u^2$ from the second equation and by doing 
so, eliminate one of the two roots at the origin, as well as the 
$\cartpt{0,-\mu_2/b_2,0,0}$ root within the invariant subspace
\refeq{eq:DangSO2spsp}. Furthermore, we scale the parameters and variables as
$\tilde{u} = c_2\,u$,
$\tilde{v} = c_1\,v$,
$\tilde{a_1} = a_1/c_2$,
$\tilde{b_1} = b_1/c_1$,
$\tilde{a_2} = a_2/c_2$,
$\tilde{b_2} = b_2/c_1$
to get
\bea
\tilde{f}(\tilde{u},\tilde{v}) &=&
  \tilde{u}\,\tilde{A}_1 - \tilde{v}\,\tilde{A}_2 = 0 
\,,\qquad \mbox{\rm deg}(f) = 2 \, , \label{PKinvEqs5a}
\\
\tilde{g}(\tilde{u},\tilde{v}) &=&  
 \left(\tilde{A}_1^2
 - c_1\,\tilde{v}\right)
 \left(\tilde{u}+2\,\tilde{v}\right)^2
 +e_2^2\,\tilde{v}^2 = 0
\,,
\ceq
   \mbox{\rm deg}(g) = 4 \, , \label{PKinvEqs5b}
\\
 && \mbox{where }
\tilde{A}_1 = \mu_1+\tilde{a_1}\,\tilde{u}+\tilde{b_1}\,\tilde{v}
\,,\ceq
\qquad\quad \tilde{A}_2 = \mu_2+\tilde{a_2}\,\tilde{u}+\tilde{b_2}\,\tilde{v}
\,,
\label{PKinvEqs5c}
\eea
Solving coupled bivariate polynomials \refeq{PKinvEqs5a} is not, in general, a 
trivial task. However, for the choice of parameters given by \refeq{eq:pars}, 
eq.~\refeq{PKinvEqs5a} yields 
$\tilde{v} = (\mu_1 + \tilde{a}_1 \tilde{u})/(\mu_2 + \tilde{a}_2 \tilde{u})$. 
Substituting this into \refeq{PKinvEqs5b} makes it a fourth order polynomial 
in $u$, which we can solve. Only the non-negative, real roots of this 
polynomial correspond to \reqva\ in the \twomode\ \statesp\ since $u$ and $v$ 
are the squares of first and second mode amplitudes, respectively. Two roots 
satisfy this condition, the \eqv\ at the origin:
\beq
	\invpol_{\EQV{}} = \invpt{0,0,0,0}\,, 
\ee{eq:origin}
and the \reqv:
\beq
	\invpol_{\REQV{}{}} = \invpt{0.193569,0.154131,-0.149539,-0.027178}\,.
\ee{eq:reqv}
Note that by setting $b_2 = 0$, we send the \reqv\ at
$\cartpt{0,-\mu_2/b_2,0,0}$ to infinity. Thus, \refeq{eq:reqv} is the
only \reqv\ of the \twomode\ system for our choice of parameters. While
this is an \eqv\ in the invariant polynomial basis, in the
\SOn{2}-equivariant, real-valued \statesp\ this is a 1\dmn\ \reqv\ group orbit.
The point on this orbit that lies in first Fourier mode slice is
(see \refFig{fig:2modes-ssp}\,(c)):
\beq
  \left(x_1, y_1, x_2, y_2\right) = \left(0.439966, 0, -0.386267, 0.070204\right)
\,.
  \label{e-req}
\eeq
We computed the linear stability eigenvalues and eigenvectors of this \reqv
, by evaluating \stabmat\ within the first Fourier mode slice
$\MvarRed_{ij} (\sspRed) = \partial \velRed_i / \partial \sspRed_j |_{\sspRed}$
on the \reqv . Linear stability eigenvalues for the \reqv\ \refeq{e-req}
\beq
	\lambda_{1,2} = 0.05073 \pm \ii \, 2.4527, \quad
	\lambda_3 = -5.5055, \quad \lambda_4 = 0 \, .
\eeq
The $0$ eigenvalue corresponds to the direction outside the slice, we expect
this since the reduced trajectory equations \refeq{eq:intSlice} keeps the
solution within the slice. Imaginary part of the expanding complex pair sets
the `winding time' in the neighborhood of the equilibrium to
$T_w = 2 \pi / \Im(\lambda_1) = 2.5617$. The large \eqv\  of the
contracting eigenvalue $\lambda_3$ yields a very thin attractor in the
reduced \statesp, thus, when looked at on a planar Poincare\'{e} section,
the \twomode\ flow is almost one dimensional, see \reffig{fig:psectandretmap}\,(a, b).

\begin{figure*}%[H]
\centering
\includegraphics[height=0.22\textwidth]{2modes-conf-reqv}\quad%
\includegraphics[height=0.22\textwidth]{2modes-confred-reqv}\quad%
\includegraphics[height=0.22\textwidth]{2modes-conf-rpo}\quad%
\includegraphics[height=0.22\textwidth]{2modes-confred-rpo}\quad%
\includegraphics[height=0.22\textwidth]{2modes-conf-ergodic}\quad%
\includegraphics[height=0.22\textwidth]{2modes-confred-ergodic}%
\caption{(Color online)
The \reqv\ \REQV{}{} in
 (a) the system's configuration space becomes an \eqv\ in
 (b) the symmetry-reduced configuration space.
Two cycles of the \rpo\ \cycle{01} in the
 (c) the symmetry-equivariant configuration space become a \po\ in
 (d) the symmetry-reduced configuration space. A typical ergodic trajectory of 
 the \twomode\ system in the system's configuration space (e), in the 
 symmetry-reduced configuration space (f). The color scale used in each figure 
 is different to enhance contrast.
}
\label{fig:2modes-conf}
\end{figure*}

\begin{figure*}%[H]
\centering
(a)\includegraphics[width=0.30\textwidth]{2modes-ssp}
(b)\includegraphics[width=0.30\textwidth]{2modes-invpol}
(c)\includegraphics[width=0.30\textwidth]{2modes-sspRed}
\caption{(Color online)
The same trajectories as in \reffig{fig:2modes-conf}\,(a,c,d),
colored green, red and blue respectively,
	(a) in a 3D projection of the 4\dmn\ \statesp ,
	(b) in a terms of 3 invariant polynomials,
	(c) in the 3\dmn\  first Fourier mode \slicePlane.
Note that in the symmetry reduced representations (b and c), the \reqv\ 
\REQV{}{} is reduced to an \eqv , the green point; and the \cycle{01} (red) 
closes onto itself after one repeat. In contrast to the invariant polynomial 
representation (b), in the first Fourier mode \slicePlane (c), the qualitative 
difference between shifts by $\approx \pi$ and $\approx-\pi$ in near passages 
to the {\sliceBord} is very clear, and it leads to the unimodal Poincar\'e 
return map of \reffig{fig:psectandretmap}.
}
\label{fig:2modes-ssp}
\end{figure*}

\subsection{No chaos when the reflection symmetry is restored}
\label{s:dfsafs}

Before finishing our discussion of invariant polynomials, we make an important 
observation regarding the case when both of the reflection symmetry breaking
parameters, $e_{1}$ and $e_2$ are set to $0$. In this case, 
$\sspC_{1,2} \rightarrow \bar{\sspC}_{1,2}$ symmetry is restored and the 
evolution equations for $u$, $v$, and $w$ in \refeq{PKinvEqs1} become 
independent of $q$. Furthermore, the time evolution equation for $q$ becomes 
linear in $q$ itself, so that it can be expressed as:
\beq
    \dot{q} = \xi (u, v) q \,.
\ee{e-qlinearq}
Hence, the time evolution of $q$ can be written as
\beq
    q(\zeit) =  e^{\int_0^\zeit d \zeit' \xi (u(\zeit'), v(\zeit'))} q(0) \, .
\ee{e-qO2solq}
If we assume that the flow is bounded, then we can also assume that a long 
time average of $\xi$ exists. The sign of this average determines the long 
term behavior of $q(\zeit)$; it will either diverge or vanish depending on the 
sign of $\langle \xi \rangle$ being positive or negative respectively. The 
former case leads to a contradiction: If $q(\zeit)$ diverges, the 
symmetry-invariant flow cannot be bounded since the syzygy \refeq{eq:syzPK} 
must be satisfied at all times. If $q(t)$ vanishes, there are three invariant 
polynomials left, which are still related to each other by the syzygy. Thus, 
the flow is confined to a two dimensional manifold and cannot exhibit chaos.
We must stress that this is a special result which holds for the two-mode 
normal form with terms up to third order.

\subsection{Visualizing \twomode\ dynamics}
\label{s:visual}

We now present visualizations of the dynamics of the \twomode\ system in
four different representations: as 3D projections of the four-dimensional
real-valued \statesp, as 3D projections in the invariant polynomial
basis, as dynamics in the 3D \slicePlane, and as two-dimensional
spacetime diagrams of the color-coded field
$u(\conf,\zeit)$ is defined as follows:
\[
	u(\conf, \tau) = \sum_{k=-2}^{2} \sspC_k(\zeit) \, e^{i k \conf}
\,,
\]
where $\sspC_{-k} = \bar{\sspC}_k \,, \; 	\sspC_0 = 0$ ,  and $\conf
\in [- \pi, \pi]$. We can also define the symmetry reduced configuration
space representation as the inverse Fourier transform of the symmetry
reduced Fourier modes:
\[
	\hat{u}(\conf, \tau) = \sum_{k=-2}^{2} \sspRedC_k(\zeit) e^{i k \conf}
\,,
\]
where $\sspRedC_{-k} = \bar{\sspRedC}_k$ \,, \; 	$\sspRedC_0 = 0$ \;
and$\conf \in [- \pi, \pi]$. \refFig{fig:2modes-conf}\,(a,b) show the
sole \reqv\ \REQV{}{} of the \twomode\ system in the symmetry-equivariant
and symmetry-reduced configuration spaces, respectively. After the
symmetry reduction, the \reqv\ becomes an \eqv.
\refFig{fig:2modes-conf}\,(c,d) show the \rpo\ \cycle{01} again
respectively in the symmetry-equivariant and symmetry-reduced
configuration space representations. Similar to the \reqv, the \rpo\
becomes a \po\ after symmetry reduction. Finally,
\refFig{fig:2modes-conf}\,(e,d) show a typical ergodic trajectory of the
\twomode\ system in symmetry-equivariant and symmetry-reduced
configuration space representations. Note that in each case, symmetry
reduction cancels the `drifts' along the symmetry ($x$) direction.

As can be seen clearly in \reffig{fig:2modes-ssp}\,(a), these drifts show up 
in the Fourier mode representation as $\SOn{2}$ rotations. The \reqv\ 
\REQV{}{} traces its \SOn{2} group orbit (green curve in 
\reffig{fig:2modes-ssp}\,(a)) as it drifts in the configuration space. The
\rpo\ \cycle{01}\,(red) and the ergodic trajectory (blue) rotate in the same 
fashion as they evolve. \refFig{fig:2modes-ssp}\,(b,c) show a three 
dimensional projection onto the invariant polynomial basis and the 3\dmn\ 
trajectory on the \slicePlane\ for the same orbits. In both figures, the 
\reqv\ is reduced to an \eqv\ and the \rpo\ is reduced to a \po.

\section{\Po s}
\label{s:numerics}

The simple structure of the symmetry-reduced dynamics allows us to determine 
the \rpo s of the \twomode\ system by means of a Poincar\'e section and a 
return map. We illustrate this procedure in \reffig{fig:psectandretmap}. 
Starting with an initial point close to the \REQV{}{}, we compute a long, 
symmetry-reduced ergodic trajectory by integrating \refeq{e-so2red1stmode} and 
record where it crosses the Poincar\'e section, which we define as the plane 
that contains \REQV{}{} and is spanned the imaginary part of its unstable 
stability eigenvector and $\hat{y}_2$.

We then project these points onto a basis $(v_1, v_2)$, which spans the 
Poincar\'e section and fit cubic splines to the data as shown in 
\reffig{fig:psectandretmap}\,(b). We then construct a return map along this 
curve and express this in terms of the distance $s$ from \REQV{}{} as 
measured by the arc length along the cubic spline fit. The resulting map, 
which is shown in \reffig{fig:psectandretmap}\,(c), is unimodal with a sharp 
cusp located at its critical point. Note that the region $s \in (0, 0.6)$ 
corresponds to the neighborhood of the \reqv\  and is only visited 
transiently. Once the dynamics fall onto the chaotic attractor, this region is 
never visited again. Removing this region from the return map, we obtain the 
return map shown in \reffig{fig:psectandretmap}\,(d), which we can then use to 
determine the accessible \rpo s  with their respective binary symbol 
sequences.

\begin{figure}
\centering
  (a) \includegraphics[width=0.40\textwidth]{BBpsecthd} \\
  (b) \includegraphics[height=0.19\textwidth]{BBpsectonslice}
  (c) \includegraphics[height=0.19\textwidth]{BBretmaponslice} \\
  (d) \includegraphics[width=0.40\textwidth]{BBretmaponsliceZoom}
\caption{(Color online)
         (a) Symmetry-reduced ergodic trajectory within the slice hyperplane 
             (blue). Green arrows indicate the real and imaginary parts of the 
             complex eigenvectors $v_u$ which span the unstable manifold of 
             \REQV{}{}. The Poincar\'e section, which contains \REQV{}{} and 
             is spanned by $\Im[v_u]$ and $\hat{y}_2$, is visualized as a 
             transparent plane. Points where the flow crosses the are marked 
             in red.
		 (b) Poincar\'e section shown in (a) projected on to two dimensions, 
		 	 where $v_{1,2}$ are the basis which spans the section hyperplane.
		 	 Note that the vertical axis, which corresponds to the direction 
		 	 parallel to $\hat{y}_2$ is magnified by $100$. All (blue) points 
		 	 are located relative to the \REQV{}{}, which it self is on the origin.
		 	 Black curve is a cubic spline interpolation to this data set.		  	
		 (c) Return map of the arclengths (along the interpolation curve, measured
		     from the origin \REQV{}{}) of Poincar\'e section data shown in (b). 
			 Note that once the flow exits the neighborhood of the \REQV{}{} 
			 ($s < 0.6$) it stays on the attractor and never comes back. Thus the data 
			 up to this is transient.
		 (d) The return map without the transient points framed by orbit of the 
		     critical point. Dashed lines show the 3-cycles \cycle{001} (red) and 
		     \cycle{011} (cyan).}
\label{fig:psectandretmap}
\end{figure}

The unimodal return map of \reffig{fig:psectandretmap} diverges around
$s \approx 0.98$ and this neighborhood is visited very rarely by the flow. We
took the furthest point that is visited by the ergodic flow, $s_C=0.98102264$
as the critical point of this map and coded points to the left and right hand sides of this
point as `0' and `1', respectively, to construct binary symbolic dynamics.
Accessible periodic orbits are then those with the topological coordinates
less than that of this critical point. We skip the technical details
regarding symbolic dynamics and kneading theory in this tutorial since
there is a rich literature on these topics and we do not employ any novel
symbolic dynamics technique here. For a pedagogical introduction to the
subject, we refer the reader to \refrefs{devnmap, DasBuch}.

We are now going to summarize the procedure of locating \rpo s in the 
\statesp : Suppose the binary itinerary
$\cycle{I_0 I_1 \dots\ I_{n-1}}, \mbox{where,}\, I_j = 0,1$
corresponds to an admissible `n-cycle', a \rpo\ that intersects our Poincar\'e
section n-times. We first find arc-lengths $\{s_0,\,s_1,\,\dots\,s_n\}$ that
constitutes this cycle on the return map \reffig{fig:psectandretmap}\,(d). We
then find corresponding reduced \statesp\ points
$\{\sspRed_0,\,\sspRed_1,\,\dots\, \sspRed_{n-1}\}$. Finally we integrate the
reduced flow and the phase \refeq{eq:so2reduced} starting from each found
reduced \statesp\ point $\sspRed_j$ until it returns to the Poincare\'e
section, and divide this trajectory into $N$ small pieces. As a result, we obtain
$n \times N$ \statesp\ points, durations and phase shifts
$\{\ssp_i^{(0)}\,,\,\zeit_i^{(0)}\,,\,\theta_i^{(0)}\}$ where
$i=1,\,2,\,\dots\,n \times N$ , which we feed into the multiple shooting Newton
solver (see \refappe{s:newton}) to precisely determine the \rpo , its period
and the associated phase shift. After finding $n \times N$ \statesp\ points
($\ssp_i$), flight times ($\zeit_i$), and phase shifts ($\theta_i$) associated
with the $n$ cycle, we then compute the flow Jacobian associated with each
piece $\jMps^{\zeit_i}(\ssp_i)$, using which we represent the Jacobian
associated with the \rpo\ as
\beq
    \jMpsRed=
    \matrixRep(\theta_{n \times N} ) \jMps^{\zeit_{n \times N}} (\ssp_{n \times N})
    \dots \,
    \matrixRep(\theta_2 ) \jMps^{\zeit_2} (\ssp_2)
    \matrixRep(\theta_1 ) \jMps^{\zeit_1} (\ssp_1) \, .
    \label{e-MultiShootJacobian}
\eeq
This construction \refeq{e-MultiShootJacobian} of Jacobian is equivalent to our
definition in \refeq{e-rpoJacobian}, since both group action $\LieEl$ and flow
Jacobian $\jMps$ are multiplicative and they commute with each other as a
consequence of $\LieEl$-equivariance of the flow. The form 
\refeq{e-MultiShootJacobian} is essential in determining its eigenvalues 
(Floquet multipliers) precisely for which we utilized periodic Schur 
decomposition (\refappe{s:schur}) .

We found the admissible cycles of the \twomode\ system upto the topological 
length 12. We listed binary itineraries of shortest $7$ \rpo s (with 
topological lengths up to 5), along with their periods, phase shifts, Floquet 
multipliers, and Floquet exponents in \reftab{t-rpofirst10}. In 
\reffig{f-2modesrpofirst4} we show shortest $4$ of the \rpo s of the \twomode\ 
system within the first Fourier mode \slicePlane . As seen from 
\reffig{f-2modesrpofirst4}, trajectories of \cycle{001} (red) and \cycle{011} 
(cyan) almost overlap in a large region of the \statesp . This behavior is 
also manifested in the return map of \reffig{fig:psectandretmap}\,{d), where 
we have shown cycles \cycle{001} and \cycle{011} with red and cyan 
respectively. This is a general property of the \twomode\ cycles with odd 
topological lengths: They come in pairs with almost equal leading (largest) 
Floquet exponents , see \reffig{f-2modes-lambdaDist}. Floquet exponents 
($\Lyap_j$) characterize the rate of expansion/contraction of nearby 
perturbations to the \rpo s and are related to Floquet multipliers 
($\ExpaEig_j$) by
\beq
    \Lyap_{\rpprime,j} = \frac{1}{\period{\rpprime}}
                         \ln | \ExpaEig_{\rpprime,j} |
                         \, , \quad j=1,2,\dots,d \, ,
\eeq
where the subscript $\rpprime$ indicates `prime \rpo\ $p$' and 
$\period{\rpprime}$ is its period. Having computed periods, phase shifts, and 
Floquet multipliers of \rpo s, we are now ready to calculate dynamical 
averages and other statistical moments of observables using \cycForm s.

\begin{table}
	\begin{tabular}{c|c|c|c|c}
	Itinerary & $T$ & $\theta$ & $\ExpaEig$ & $\Lyap$ \\ 
	\hline
	1 & 3.64151221 & 0.08096967 & -1.48372354 &0.10834917 \\ 
	01 & 7.34594158 & -2.94647181 & -2.00054831 &0.09439516 \\ 
	001 & 11.07967801 & -5.64504385 & -55.77844510 &0.36295166 \\ 
	011 & 11.07958924 & -2.50675871 & 54.16250810 &0.36030117 \\ 
	0111 & 14.67951823 & -2.74691247 & -4.55966852 &0.10335829 \\ 
	01011 & 18.39155417 & -5.61529803 & -30.00633820 &0.18494406 \\ 
	01111 & 18.38741006 & -2.48213868 & 28.41893870 &0.18202976 \\ 
	\end{tabular}
	\caption{Itinerary, period ($T$), phase shift ($\theta$), 
			 Floquet multiplier ($\ExpaEig$), and Floquet exponent
			 ($\Lyap$) of the found \twomode\ \rpo s with topological
			 lengths up to $n = 5$, more (up to $n=12$) available 
			 upon request.}
	\label{t-rpofirst10}
\end{table}

\begin{figure}%[H]
\centering
 \includegraphics[width=0.45\textwidth]{2modesrpofirst4}
\caption{(Color online)
Shortest four \rpo s of the \twomode\ system: \cycle{1} (dark blue),
\cycle{01} (green), \cycle{001} (red), \cycle{011} (cyan). Note that \rpo
s \cycle{001} and \cycle{011} almost overlap everywhere except $\hat{x}_1
\approx 0$ .}
\label{f-2modesrpofirst4}
\end{figure}

\begin{figure}%[H]
\centering
 \includegraphics[width=0.45\textwidth]{2modes-lambdaDist}
\caption{(Color online)
        Distribution of the expanding Floquet exponents of all \twomode\ cycles with
         topological lengths $n$ from $2$ to $12$.}
\label{f-2modes-lambdaDist}
\end{figure}

\section{Cycle Averages}
\label{s:DynAvers}

So far, we have explained how we find the \rpo s of the \twomode\ system and c
ompute their stability. However, we have not yet said anything about what to 
do with these numbers. We begin this section with an overview of the main 
results of the periodic orbit theory. Our review starts by recapitulating the 
presentation of \refref{DasBuch}, but then, in \refsect{s-ContFac}, explains 
how the theory is modified in the presence of continuous symmetries\rf{Cvi07}.
In \refsect{s-CycExp}, we present cycle expansions and explain how to 
approximate the Poincar\'e section in \reffig{fig:psectandretmap}\,(d), in 
order to obtain a better convergence of the spectral determinants. The 
numerical results are discussed in \refsect{s-NumResults}

\subsection{Classical trace formula}

Consider the {\evOper}, the action of which evolves a density $\rho_0(\ssp)$ 
in the \statesp:
\bea
    \rho(\zeit ,\ssp) &=& [\Lop^\zeit \rho_0 ] (\ssp)
    \continue
    &=& \int d \ssp' \delta (\ssp - \flow{\zeit}{\ssp'})
        e^{\beta \Obser^\zeit (\ssp' )} \rho_0(\ssp') \label{e-EvOper}
\,,
\eea
where $\beta$ is an auxiliary variable and $\Obser^\zeit (\ssp')$ is the 
integrated value of an observable $\obser$ along the orbit 
$\flow{\zeit}{\ssp'}$:
\beq
    \Obser^\zeit (\ssp' ) = \int_0^{\zeit} d \zeit'
                              \obser(\flow{\zeit'}{\ssp'})
\,.
\eeq
Notice that when $\beta = 0$, the \evOper\ \refeq{e-EvOper} simply evolves
the density of \statesp\ points to its new form after time $\zeit$. As we 
shall see, attaching the integrated observables to this operator enables us to
study values of these observables averaged over the invariant measures.

Since we required our observable to be additive along an orbit and 
exponentiated its integrated value in the construction of the \evOper\
\refeq{e-EvOper}; the evolution operator itself is multiplicative:
\beq
    \Lop^{\zeit_1 + \zeit_2} = \Lop^{\zeit_2} \Lop^{\zeit_1} \, .
    \label{eq-SemiGroup}
\eeq
For the kernel of the evolution integral, which we will refer to as 
$\Lop^\zeit (\ssp, \ssp')$ with explicit arguments, we can write this relation
as:
\beq
	\Lop^{\zeit_1 + \zeit_2} (\ssp,\ssp') =
    \int d\ssp'' \Lop^{\zeit_2} (\ssp, \ssp'')
                   \Lop^{\zeit_1} (\ssp'', \ssp) \, .
	\label{eq-SemiGroupKernel}
\eeq
This `semigroup property' \refeq{eq-SemiGroup} of the {\evOper} allows us to
define the {\evOper} as the formal exponential of its infinitesimal generator
\Aop :
\beq
	\Lop^t = e^{\Aop t} \, .
	\label{eq-EvOpExp}
\eeq
By definition \refeq{e-EvOper}, the eigenvalues and eigenfunctions of $\Lop^t$ 
(and thus \Aop ) are functions of $\beta$. Let us define $\rho_{\beta} (x)$ as 
the eigenfunction of \refeq{e-EvOper} corresponding to the leading eigenvalue 
(i.e., the one with the largest real part); we can write the action of 
\refeq{e-EvOper} on this density explicitly as follows:
\beq
    \left[ \Lop^t \rho_{\beta} \right] (x) = e^{t s(\beta )} \rho_{\beta} (x)
    \, .
    \label{eq-EigenvalueRel}
\eeq
Here, $s(\beta)$ is the eigenvalue of $\Aop$. As stated earlier, when 
$\beta = 0$, the {\evOper} simply evolves densities; this form of the 
evolution operator is known as the {\FPoper}. If we assume that the system 
under study is ergodic, then an `invariant measure' $\rho_0(\ssp)$ exists 
with eigenvalue $s(0) = 0$ exists. The long time spectrum of any observable is 
going to be dominated by its average over such a density, hence we define the 
average of an observable as its average over the invariant measure:
\beq
    \langle \obser \rangle = \int d \ssp \, \obser(\ssp) \rho_0 (\ssp) \, .
    \label{e-obserAvg}
\eeq
By evaluating the action of the {\evOper} \refeq{e-EvOper} for infinitesimal 
times and after some algebra, which we skip here, one finds that the averages 
of observables, as well as their higher moments, can be generated from the
derivatives of $s(\beta)$:
\beq
    \langle \obser \rangle =
        \left. \frac{d s}{d \beta} \right|_{\beta = 0} \, , \quad
    \langle (\obser - \langle \obser \rangle )^2 \rangle =
        \left. \frac{d^2 s}{d \beta^2} \right|_{\beta = 0} \,, ...
    \label{eq-moments}
\eeq
In order to obtain $s(\beta)$, we construct the resolvent of \Aop , by taking
the Laplace transform of \refeq{eq-EvOpExp}:
\beq
	\int_0^{\infty} d\zeit e^{-s\zeit} \Lop^\zeit = (s-\Aop)^{-1} \, ,
	\label{eq-ResolventA}
\eeq
the trace of which peaks at the eigenvalues of \Aop. By taking the Laplace 
transform of $\Lop^\zeit$ and computing its trace by 
$\tr \Lop^\zeit = \int d\ssp \Lop^\zeit (\ssp,\ssp)$, one obtains the 
classical trace formula:
\beq
\sum_{\alpha=0}^{\infty} \frac{1}{s-s_{\alpha}} = \sum_p T_p
\sum_{r=1}^{\infty} \frac{e^{r(\beta \Obser_p - s T_p)}}{\oneMinJ{r}}
\ee{e-ClassicalTraceFormula}
that relates the spectrum of the {\evOper} to the spectrum of the periodic 
orbits. Here,  $s$ is the auxiliary variable of the Laplace transform and 
$s_{\alpha}$ are the eigenvalues of \Aop . The outer sum on the right hand 
side runs over the `prime cycles' $p$ of the system, which have periods 
$T_p$. $\Obser_p$ is the value of the observable integrated along the prime 
cycle and $\monodromy_p$ is the transverse monodromy matrix, the eigenvalues 
of which are the Floquet multipliers of $p$ excluding the marginal ones 
($|\Lambda| \neq 1$). In the derivation of \refeq{e-ClassicalTraceFormula}, 
one assumes that the flow has a single marginal direction, namely the 
direction that is parallel to the periodic orbit at all times, and evaluates 
the contribution of each \po\ to the trace integral by transforming to a local 
coordinate system where one of the coordinates is parallel to the flow while 
the rest is transverse. Integration along the parallel direction is what 
contributes the factors of $T_p$. The transverse integral over the delta 
function contributes the factor of $\oneMinJ{r}$.

\subsection{Decomposition of the trace formula over irreducible representations}
\label{s-ContFac}

The classical trace formula \refeq{e-ClassicalTraceFormula} accounts for 
contributions from \po s to long time dynamical averages. However, \rpo s 
of equivariant systems are almost never periodic in the full \statesp. In 
order to compute the contributions of \rpo s to the trace of the \evOper, 
one has to factorize the \evOper\ into the irreducible subspaces of the 
symmetry group. For discrete symmetries, this procedure is studied in 
\refref{CvitaEckardt}. For the quantum systems with continuous symmetries 
(Abelian and 3D rotations), the factorization of the semiclassical Green's 
operator is carried out in \refref{Creagh93}. \refRef{Cvi07} addresses the 
continuous factorization of the \evOper\ and its trace; we provide a sketch of 
this treatment here. We start by stating, without proof, that a 
square-integrable field $\psi (\ssp)$ over a vector space can be factorized 
into its projections over the irreducible subspaces of a group
$\Group$:
\beq
    \psi (\ssp) = \sum_m \mathbb{P}_m \psi (\ssp) \, ,
\eeq
where the sum runs over the irreducible representations of $\Group$ and the 
projection operator onto the $m$th irreducible subspace, for a continuous 
group, is:
\beq
    \mathbb{P}_m = d_m \int_\Group d \mu(\LieEl) \chi_m (\LieEl(\theta))
                            \mathbb{D}(\theta)
\,.
\ee{e-ProjectionOperator}
Here, $d_m$ is the dimension of the representation, $d \mu(g)$ is the
normalized Haar measure, $\chi_m (\LieEl)$ is the character of $m$th
irreducible representation and $\mathbb{D}(\theta)$ is the operator that
transforms a scalar field defined on the \statesp\ as
$\mathbb{D}(\theta) \rho (\ssp) = \rho(\matrixRep(\theta)^{-1} \ssp)$.
For our specific case of a single $\SOn{2}$ symmetry,
\bea
d_m &\rightarrow& 1\, , \\
\int_G d \mu(g) &\rightarrow& \oint \frac{d \theta} {2 \pi} \, , \\
\chi_m (\LieEl(\theta)) &\rightarrow& e^{- \ii m \theta } \, .
\eea
Because the projection operator \refeq{e-ProjectionOperator} decomposes 
scalar fields defined over the \statesp\ into their irreducible subspaces 
under action of $\Group$, it can be used to factorize the \evOper. Thus, the 
kernel of the \evOper\ transforms under the action of $\mathbb{D}(\theta)$ as:
\bea
    \mathbb{D}(\theta) \Lop^t (\ssp', \ssp) &=&
        \Lop^t (\matrixRep(\theta)^{-1} \ssp', \ssp)\,,
    \continue
    &=& \Lop^t (\ssp', \matrixRep(\theta) \ssp) \,, \continue
    &=& \delta (\ssp' - \matrixRep(\theta) f^t (\ssp)) e^{\beta \Obser^t(\ssp)}\, ,
    \label{e-gEvOper}
\eea
where the second step follows from the equivariance of the system under 
consideration. \Rpo s contribute to $\mathbb{P}_m \Lop^t = \Lop_m^t$ since 
when its kernel is modified as in \refeq{e-gEvOper}, the projection involves 
an integral over the group parameters that is non-zero when 
$\theta=\theta_{\rpprime}$, the phase shifts of the \rpo s. By computing the 
trace of $\Lop_m^t$, which in addition to the integral over \statesp , now 
involves another integral over the group parameters, one obtains the $m$th 
irreducible subspace contribution to the classical trace as
\beq
\sum_{\alpha=0}^{\infty} \frac{1}{s-s_{m, \alpha}} = \sum_p T_{\rpprime}
\sum_{r=1}^{\infty} \frac{\chi_m (\LieEl^r(\theta_{\rpprime}))
            e^{r(\beta \Obser_{\rpprime} - s T_{\rpprime})}}{\oneMinJred{r}} .
\ee{e-ReducedTraceFormula}
The reduced trace formula \refeq{e-ReducedTraceFormula} differs from the 
classical trace formula \refeq{e-ClassicalTraceFormula} by the group character
term, which is evaluated at the \rpo\ phase shifts, and the reduced monodromy
matrix $\monodromyRed$, which is the $(d-N-1)\times(d-N-1)$ reduced Jacobian
for the \rpo\ evaluated on a Poincar\'e section in the \reducedsp . The 
eigenvalues of $\monodromyRed$ are those of the \rpo\ Jacobian 
\refeq{e-rpoJacobian} excluding the marginal ones, i.e., the ones 
corresponding to time evolution and evolution along the continuous symmetry 
directions.

Since we are only interested in the leading eigenvalue of the \evOper , we
only consider contributions to the trace \refeq{e-ClassicalTraceFormula} from 
the projections \refeq{e-ReducedTraceFormula} of the $0$th irreducible 
subspace. For the $\SOn{2}$ case at hand, these can be written explicitly as
\beq
\sum_{\alpha=0}^{\infty} \frac{1}{s-s_{0, \alpha}} = \sum_p T_p
\sum_{r=1}^{\infty} \frac{e^{r(\beta \Obser_p - s T_p)}}{\oneMinJred{r}} \, .
\ee{e-tracem0}
This form differs from the classical trace formula
\refeq{e-ClassicalTraceFormula} only by the use of the reduced monodromy 
matrix instead of the full monodromy matrix since the $0$th irreducible 
representation of $\SOn{2}$ has character $1$. For this reason, cycle 
expansions \rf{AACI}, which we cover next, are applicable to 
\refeq{e-tracem0} after the replacement 
$\monodromy \rightarrow \monodromyRed$.

\subsection{Cycle expansions}
\label{s-CycExp}

While the classical trace formula \refeq{e-ClassicalTraceFormula} and its
factorization for systems with continuous symmetry 
\refeq{e-ReducedTraceFormula} manifest the essential duality between the 
spectrum of an observable and that of the \po s and \rpo s, in practice, 
they are hard to work with since the eigenvalues are located at the poles of 
\refeq{e-ClassicalTraceFormula} and \refeq{e-ReducedTraceFormula}. The 
dynamical zeta function \refeq{e-DynamicalZeta}, which we derive below, 
provides a perturbative expansion form that enables us to order terms in 
decreasing importance while computing spectra for the \twomode\ system. 
As stated earlier, \refeq{e-tracem0} is equivalent to 
\refeq{e-ClassicalTraceFormula} via substitution 
$\monodromy \rightarrow \monodromyRed$. We start by defining the
`spectral determinant':
\beq
  \det (s-\Aop) = \exp \left( - \sum_p \sum_{r=1}^{\infty}
      \frac{1}{r} \frac{e^{r(\beta \Obser_p - s T_p)}}{\oneMinJ{r}} \right)\, ,
\ee{e-SpectralDeterminant}
whose logarithmic derivative ($(d/ds) \ln \det(s - \Aop)$) gives
the classical trace formula \refeq{e-ClassicalTraceFormula}.
The spectral determinant \refeq{e-SpectralDeterminant} is easier to work
with since the spectrum of $\mathcal{A}$ is now located at the zeros of
\refeq{e-SpectralDeterminant}. The convergence of 
\refeq{e-SpectralDeterminant} is, however, still not obvious. More insight is 
gained by approximating $\oneMinJ{r}$ by the product of expanding Floquet 
multipliers and then carrying out the sum over $r$ in 
\refeq{e-SpectralDeterminant}. This approximation yields
\bea
\oneMinJ{} &=& | (1 - \ExpaEig_{e,1})(1 - \ExpaEig_{e,2})... \continue
			&&(1 - \ExpaEig_{c,1}) (1 - \ExpaEig_{c,2}) ... | \nonumber \\
			&\approx& \prod_e |\ExpaEig_e| \equiv |\ExpaEig_p|,
    \label{e-LambdapApprox}
\eea
where $|\ExpaEig_{e,i}| > 1$ and $|\ExpaEig_{c,i}| < 1$ are expanding and
contracting Floquet multipliers respectively. By making this approximation, 
the sum over $r$ in \refeq{e-SpectralDeterminant} becomes the Taylor expansion 
of natural logarithm. Carrying out this sum, brings the spectral determinant 
\refeq{e-SpectralDeterminant} to a product (over prime cycles) known as the 
dynamical zeta function:
\beq
1 / \zeta = \prod_p (1 - t_p) \, \mbox{where}, \, t_p = \frac{1}{|\ExpaEig_p|}
            e^{\beta \Obser_p - s T_p} z^{n_p} .
\ee{e-DynamicalZeta}
Each `cycle weight' $t_p$ is multiplied by the `order tracking term' 
$z^{n_p}$, where $n_p$ is the topological length of the $p$th prime cycle. 
This polynomial ordering arises naturally in the study of discrete time 
systems where the Laplace transform is replaced by $z$-transform. Here, 
we insert the powers of $z$ by hand, to keep track of the ordering, and 
then set its value to $1$ at the end of calculation. Doing so allows us to 
write the dynamical zeta function \refeq{e-DynamicalZeta} in the 
`cycle expansion' form by grouping its terms in powers of $z$. For complete 
binary symbolic dynamics, where every binary symbol sequence is accessible, 
the cycle expansion reads
\bea
1 / \zeta &=& 1 - t_0 - t_1 - (t_{01} - t_0 t_1 )  \label{e-CycleExpansion} \\
		  && - [(t_{011} - t_{01}t_1) + (t_{001} - t_{01} t_0)] - ... \continue
		  &=& 1 - \sum_f t_f - \sum_n \hat{c}_n \label{e-CurvatureExpansion},
\eea
where we labeled each prime cycle by its binary symbol sequence. In
\refeq{e-CurvatureExpansion} we grouped the contributions to the zeta function
into two groups: `fundamental' contributions $t_f$ and `curvature' corrections $c_n$.
The curvature correction terms are denoted explicitly by parentheses in \refeq{e-CycleExpansion} and
correspond to `shadowing' combinations where combinations of
shorter cycle weights, also known as `pseudocycle' weights, are subtracted from the weights of longer
prime cycles. Since the cycle weights in \refeq{e-DynamicalZeta} already
decrease exponentially with increasing cycle period, the cycle expansion
\refeq{e-CycleExpansion} converges even faster than exponentially when the
terms corresponding to longer prime cycles are shadowed.

For complete binary symbolic dynamics, the only fundamental contributions to
the dynamical zeta function are from the cycles with topological length $1$, 
and all longer cycles appear in the shadowing pseudocycle combinations. More 
generally, if the symbolic dynamics is a subshift of finite type\rf{DasBuch},
with the grammar of admissible sequences described by a finite set of pruning 
rules, and the flow is uniformly hyperbolic, cycle expansions of {\Fd s} are 
guaranteed to converge super-exponentially\rf{hhrugh92}. A generic unimodal 
map symbolic dynamics is not a subshift of finite type. However, we have 
shown in \refsect{s:numerics} that the Poincar\'e return map for the \twomode\ 
system (\reffig{fig:psectandretmap}\,(d)) diverges at $s \approx 0.98$ and 
approximated it as if its tip was located at the furthest point visited by an 
ergodic trajectory. Can we approximate the map in 
\reffig{fig:psectandretmap}\,(d) in such a way that corresponding symbolic 
dynamics has a finite grammar of pruning rules? The answer is yes.

As shown in \reffig{fig:psectandretmap}\,(d) the cycles \cycle{001}
and \cycle{011} pass quite close to the tip of the cusp. Approximating the
map as if its tip is located exactly at the point where \cycle{001} cuts gives us
what we are looking for: a single grammar rule, which says that the symbol
sequence `00' is inadmissible. This can be made rigorous by the help of
kneading theory, however, the simple result is easy to see from the return map
in \reffig{fig:psectandretmap}\,(d): Cover the parts of the return map, which
are outside the borders set by the red dashed lines, the cycle \cycle{001} and
then start any point to the left of the tip and look at images. You will always
land on a point to the right of the tip, unless you start at the lower left
corner, exactly on the cycle \cycle{001}. As we will show, this `finite grammar
approximation' is reasonable since the orbits that visit outside
the borders set by \cycle{001} are very unstable, and hence, less
important for the description of invariant dynamics.

The binary grammar with only rule that forbids repeats of one of the symbols is
known as the `golden mean' shift\rf{DasBuch}, named after its topological entropy which is
$\ln (1 + \sqrt{5})/2$. Binary itineraries of golden mean cycles can be easily
obtained from the complete binary symbolic dynamics by substitution
$0 \rightarrow 01$ in  the latter. Thus, we can write the dynamical zeta
function for the golden mean pruned symbolic dynamics by replacing $0$s in
\refeq{e-CycleExpansion} by $01$:
\bea
1 / \zeta &=& 1 - t_{01} - t_1 - (t_{011} - t_{01} t_1 )
              \label{e-GoldenMeanCycleExpansion}\\
		  && - [(t_{0111} - t_{011}t_1) + (t_{01011} - t_{01} t_{011} ) ] - ...
          \nonumber
\eea
Note that all the contributions longer than topological length $2$ to the
golden mean dynamical zeta function are in form of shadowing combinations. In \refsect{s-NumResults},
we will compare the convergence of the cycle averages with and without the
finite grammar approximation, but before moving on to numerical results,
we explain the remaining details of computation.

While dynamical zeta functions are useful for investigating the convergence
properties, they are not exact, and their computational cost is same as that of
exact spectral determinants. For this reason, we expand the
spectral determinant \refeq{e-SpectralDeterminant} ordered in the topological
length of cycles and pseudocycles. We start with the following form of the
spectral determinant \refeq{e-SpectralDeterminant}:
\beq
    \det (s-\Aop) =   \prod_p \exp \left( - \sum_{r=1}^{n_p r < N}
                             \frac{1}{r} \frac{e^{r(\beta \Obser_p - s T_p)}
                                          }{\oneMinJ{r}} z^{n_p r} \right) \, ,
\ee{e-SpectralDeterminantExp}
where the sum over the prime cycles in the exponential becomes a
product. We also inserted the order tracking term $z$ and truncated the sum over cycle
repeats at the expansion order $N$. For each prime cycle, we compute the sum in
\refeq{e-SpectralDeterminantExp} and expand the exponential up to order
$N$. We then multiply this expansion with the contributions from previous cycles
and drop terms with order greater than $N$. This way, after setting $z=1$,
we obtain the $N^{th}$ order spectral determinant, which we denote as
\beq
    F_N(\beta , s) = 1 - \sum_{n=1}^{N} Q_n(s, \beta ) \, .
    \label{e-NthOrderSpectDet}
\eeq
Remember that we are searching for the eigenvalues $s ( \beta)$ of the
operator \Aop,
more specifically, we would like to compute the moments \refeq{eq-moments}.
$s ( \beta)$ are located at the zeros of the spectral determinant, hence they
satisfy the implicit equation:
\beq
    F_N(\beta, s(\beta )) = 0 \, .
    \label{e-FNimplicit}
\eeq
By taking derivative of \refeq{e-FNimplicit} with respect to $\beta$ and
applying chain rule we obtain
\beq
    \frac{d s}{d \beta} = - \left. \frac{\partial F}{\partial \beta} \right/
                                     \frac{\partial F}{\partial s}\, .
\eeq
Higher order derivatives can be evaluated similarly.
We define the `mean cycle period' as
\beq
	\langle T \rangle_N = \left. \partial F_N / \partial s
                          \right|_{\beta=0, s=s (0)} \, ,
	\label{eq-Tavg}
\eeq
and write the \cycForm s as
\bea
    \langle \obser \rangle_N &=& - \frac{1}{\langle T \rangle_N} \left.
                              \frac{\partial F_N}{\partial \beta}
                              \right|_{\beta=0, s=s (0)} \, , \label{e-Avga} \\
    \langle (\obser - \langle \obser \rangle )^2 \rangle_N
    &=& - \frac{1}{\langle T \rangle_N} \left. \frac{\partial^2 F_N}{
                        \partial \beta^2} \right|_{\beta=0, s=s (0)} \,
                        \label{e-Avgsigma} .
\eea
By probability conservation, we expect that for an invariant measure
$\rho_0(\ssp)$, the eigenvalue $s(0)$ is $0$. However, we did not make
this substitution in \cycForm s since, in practice, our approximation of
the spectral determinant is always of a finite precision, so that the
solution of $F_N(0, s(0)) = 0$ is small, but not exactly $0$. This
eigenvalue has a special meaning: It indicates how well the \po s cover
the strange attractor. Following this interpretation, we define $\gamma =
- s(0)$ as the `escape rate': the rate at which the dynamics escape the
region that is covered by the \po s. Specifically, for our finite grammar
approximation; the escape rate tells us how frequently the ergodic flow
visits the part of the Poincar\'e map that we cut off by applying our
finite grammar approximation.

We defined $\langle T \rangle$ in \refeq{eq-Tavg} as a shorthand for a partial
derivative, however, we can also develop and interpretation for it by looking
at the definitions of the dynamical zeta function \refeq{e-DynamicalZeta} and the
spectral determinant \refeq{e-SpectralDeterminant}. In both series, the partial
derivative with respect to $s$ turns them into weighted sum of the cycle
periods; with this intuition, we define $\langle T \rangle$ as the `mean cycle
period'.

These remarks conclude our review of the periodic theory and its
extension to the equivariant dynamical systems. We are now ready to present
our numerical results and discuss their quality.

\subsection{Numerical results}
\label{s-NumResults}

We constructed the spectral determinant \refeq{e-NthOrderSpectDet} to different
orders for two observables: phase velocity $\dot{\theta}$ and the leading
Lyapunov exponent. Remember that $\Obser_p$ appearing in
\refeq{e-SpectralDeterminantExp} is the integrated observable, so in order to
obtain the moments of phase velocity and the leading Lyapunov exponent from
\refeq{e-Avga} and \refeq{e-Avgsigma}, we respectively input
$\Obser_p = \theta_p$ phase shift of the prime cycle, and
$\Obser_p = \ln |\Lambda_{p,e}|$ logarithm of the expanding Floquet
multiplier of $\Lambda_{p,e}$ the prime cycle $p$.

In \refsect{s:visual}, we explained that \SOn{2} phase shifts correspond
to the drifts in the analogous configuration space. We now define the
corresponding diffusion constant proportional to the variance of the
phase (drift) velocity,
\beq
    D = \frac{1}{2 d} \sigma_{\dot{\theta}}^2
      = \frac{\langle (\dot{\theta} - \langle \dot{\theta} \rangle)^2
              \rangle}{2}
\,,
\eeq
where $d=1$ since the configuration space is one dimensional.

\refTab{t-DynamicalAverages} and \reftab{t-DynamicalAveragesNoGrammar} shows
the cycle averages of the escape rate $\gamma$, mean period
$\langle T \rangle$, leading Lyapunov exponent $\Lyap$, mean phase velocity
$\langle \dot{\theta} \rangle$ and the diffusion constant $D$ respectively
with and without the finite grammar approximation. In the latter, we input
all the \rpo s we have found into the expansion
\refeq{e-SpectralDeterminantExp}, whereas in the former, we discarded the
cycles with symbol sequence `00'.

\begin{table}
	\begin{tabular}{c|c|c|c|c|c}
	 $N$ & $\gamma$ & $\langle T \rangle$ & $\lambda$ & $\langle \dot{\theta} \rangle$ & $D$ \\
	\hline
	1 & 0.249829963 & 3.6415122 & 0.10834917 & 0.0222352 & 0.000900 \\
 	2 & -0.011597609 & 5.8967605 & 0.10302891 & -0.1391709 & 0.226199 \\
 	3 & 0.027446312 & 4.7271381 & 0.11849761 & -0.1414933 & 0.234572 \\
 	4 & -0.004455525 & 6.2386572 & 0.10631066 & -0.2141194 & 0.472633 \\
 	5 & 0.000681027 & 5.8967424 & 0.11842700 & -0.2120545 & 0.461628 \\
 	6 & 0.000684898 & 5.8968762 & 0.11820050 & -0.1986756 & 0.385774 \\
 	7 & 0.000630426 & 5.9031596 & 0.11835159 & -0.1997353 & 0.392404 \\
 	8 & 0.000714870 & 5.8918832 & 0.11827581 & -0.1982025 & 0.380317 \\
 	9 & 0.000728657 & 5.8897511 & 0.11826873 & -0.1982254 & 0.380338 \\
 	10 & 0.000728070 & 5.8898549 & 0.11826788 & -0.1982568 & 0.380721 \\
 	11 & 0.000727891 & 5.8898903 & 0.11826778 & -0.1982561 & 0.380720 \\
 	12 & 0.000727889 & 5.8898908 & 0.11826780 & -0.1982563 & 0.380724 \\
 	\end{tabular}
	\caption{Cycle expansion estimates based on the golden mean approximation
             \refeq{e-GoldenMeanCycleExpansion} to symbolic dynamics for
             the escape rate $\gamma$, average cycle period $\langle T \rangle$,
             Lyapunov exponent $\lambda$, average phase velocity
             $\langle \dot{\theta} \rangle$ and the diffusion coefficient $D$,
             up to cycle length $N$.}
	\label{t-DynamicalAverages}
\end{table}

\begin{table}
	\begin{tabular}{c|c|c|c|c|c}
	 $N$ & $\gamma$ & $\langle T \rangle$ & $\lambda$ & $\langle \dot{\theta} \rangle$ & $D$ \\ 
	\hline
	1 & 0.249829963 & 3.6415122 & 0.10834917 & 0.0222352 & 0.000900 \\ 
 	2 & -0.011597609 & 5.8967605 & 0.10302891 & -0.1391709 & 0.226199 \\ 
 	3 & 0.022614694 & 4.8899587 & 0.13055574 & -0.1594782 & 0.294880 \\ 
 	4 & -0.006065601 & 6.2482261 & 0.11086469 & -0.2191881 & 0.488973 \\ 
 	5 & 0.000912644 & 5.7771642 & 0.11812034 & -0.2128347 & 0.448434 \\ 
 	6 & 0.000262099 & 5.8364534 & 0.11948918 & -0.2007615 & 0.383389 \\ 
 	7 & 0.000017707 & 5.8638210 & 0.12058951 & -0.2021046 & 0.392894 \\ 
 	8 & 0.000113284 & 5.8511045 & 0.12028459 & -0.2006143 & 0.381039 \\ 
 	9 & 0.000064082 & 5.8587350 & 0.12045664 & -0.2006756 & 0.381315 \\ 
 	10 & 0.000093124 & 5.8536181 & 0.12035185 & -0.2007018 & 0.381465 \\ 
 	11 & 0.000153085 & 5.8417694 & 0.12014700 & -0.2004520 & 0.377156 \\ 
 	12 & 0.000135887 & 5.8455331 & 0.12019940 & -0.2005299 & 0.378651 \\ 
 	\end{tabular}
	\caption{Cyle expansion estimates of the escape rate $\gamma$, average 
    cycle period $\langle T \rangle$, Lyapunov exponent $\lambda$, average 
    phase velocity $\langle \dot{\theta} \rangle$ and the diffusion coefficient 
    $D$ with respect to the expansion order $N$ .}
	\label{t-DynamicalAveragesNoGrammar}
\end{table}

In \refsect{s-CycExp}, we motivated the finite grammar approximation by 
expecting a faster convergence due to the nearly exact shadowing combinations 
of the golden mean zeta function \refeq{e-GoldenMeanCycleExpansion}. This 
claim is clearly supported by the data in \reftab{t-DynamicalAverages} and 
\reftab{t-DynamicalAveragesNoGrammar}. Take, for example, the Lyapunov 
exponent which converges to $7$ digits for the $12^{th}$ order expansion when 
using the finite grammar approximation \reftab{t-DynamicalAverages}, only 
converges to $4$ digits at this order in 
\reftab{t-DynamicalAveragesNoGrammar}. Other observables compare similarly in
terms of their convergence in both cases. Note, however, that the escape rate
in \reftab{t-DynamicalAverages} converges to $\gamma = 0.000727889$, whereas
in \reftab{t-DynamicalAveragesNoGrammar} it gets smaller and smaller with an
oscillatory behavior. This is due to the fact that in the finite grammar
approximation, we threw out the part of attractor that corresponds to the
cusp of the return map in \reffig{fig:psectandretmap}\,(d) above the point that
is cut by \cycle{001}.

In order to compare with the cycle averages, we numerically estimated the
leading Lyapunov exponent of the \twomode\ system using the method of
Wolf \etal\rf{WolfSwift85}. This procedure was repeated 100 times for
different initial conditions, yielding a numerical mean estimate of
$\timeAver{\Lyap} = 0.1198 \pm 0.0008$. While the finite grammar
estimate $\Lyap_{FG} = 0.1183$ is within $0.6\%$ range of this value,
the full cycle expansion agrees with the numerical estimate. This is not
surprising, since in the finite grammar approximation, we discard the
most unstable cycles, thus, obtaining a slightly smaller Lyapunov
exponent while obtaining a significantly better convergence.

\section{Conclusions and discussion}
\label{s:concl}

In this tutorial, we have studied what is probably the simplest dynamical 
system which exhibits chaos and is equivariant under a continuous symmetry 
transformation. We have shown that reducing this symmetry simplifies the 
qualitative dynamics to a great extent and enables one to find all \rpo s of 
the systems via standard techniques such as Poincar\'e sections and return 
maps. In addition, we have shown that one can extract quantitative information
from the \rpo s by computing cycle averages.

We motivated our study of the \twomode\ system by the resemblance of its
symmetry structure to that of the spatially extended systems; and the
steps we outlined here are, in principle, applicable to physical systems
that are described by $N$-Fourier mode truncations of PDEs such as $1D$
\KS\rf{SCD07}, $3D$ pipe flows\rf{WiShCv14}, \etc.

We showed in \refsect{s:numerics} that \twomode\ dynamics can be completely
described by a unimodal return map on the Poincar\'e section we constructed
after continuous symmetry reduction. In a high-dimensional system, finding 
such an easy symbolic dynamics, or any symbolic dynamics at all is a 
challenging problem on its own. For desymmetrized (confined in the odd 
subspace) $1$D spatio-temporally \KS\ system \refref{lanCvit07} finds a 
bimodal return map after reducing the discrete symmetry of the problem, 
however, for turbulent fluids, we do not know any study that simplifies the 
flow to such an extent.

In \refsect{s:DynAvers}, we have shown that the symbolic dynamics and the
associated grammar rules greatly affect the convergence of \cycForm s.
In general, finding a finite symbolic description of a flow is rarely as easy 
is in our model system. There exist other methods of ordering cycle expansion 
terms, for example, ordering pseudo-cycles by their stability and discarding 
terms that are above a threshold\rf{DM97}; one expects the remaining terms to 
form shadowing combinations and converge exponentially.Whichever method of 
term ordering is deployed, the cycle expansions are only as good as the least 
unstable cycle that one fails to find. Symbolic dynamics solves both of 
problems at once since it puts the cycles into a topological order so that one 
cannot miss any accessible cycle and shadowing combinations naturally occur 
when the expansion is ordered in topological length. The question one might 
ask is: When there is no symbolic dynamics, how can we make sure that we find 
all periodic orbits of a flow up to some cycle period?

In the searches of the cycles of high-dimensional flows, one usually looks at
the near recurrences of the ergodic flow, and then runs Newton searches
starting nearby these recurrences to find if they are influenced by an exact
recurrence. Such an approach does not answer the question we just asked to 
full confidence, however, one may argue that the dynamically important cycles
influence recurrences of the ergodic flow, and hence cycles found this way are
those that are relevant for computing averages.

To sum up, we have shown that the periodic orbit theory successfully extends 
to the systems with continuous symmetries. One still needs to think about
remaining challenges, discussed above, associated with high dimensional 
systems. Once these are overcome, we would then be able to extract 
quantitative information from turbulence by using its exact unstable solutions,
which would be big news.

\begin{acknowledgments}
We are grateful to Evangelos Siminos for his contributions to this project
and Mohammad M.~Farazmand for a critical reading of the manuscript.
We acknowledge stimulating discussion with
Xiong Ding,
Ruslan L.~Davidchack,
Ashley P.~Willis,
Al Shapere
and
Francesco Fedele.
We are indebted to the 2012 ChaosBook.org class, in particular to
Keith M.~Carroll,
Sarah Flynn,
Bryce Robbins,
and
Lei Zhang,
for the initial fearless fishing expeditions into the enormous sea of the
parameter values of the \twomode\ model.
P.~C.\ thanks the family of late G.~Robinson,~Jr.
and
NSF~DMS-1211827 for support. D.~B.\ thanks M.~F.\ Schatz for support during
the early stages of this work under NSF~CBET-0853691.
\end{acknowledgments}

\appendix
\section{Multiple shooting method for finding \rpo s}
\label{s:newton}

Let us assume that we have a set of good guesses
for a set of \statesp\ points,
flight times and
$1D$ symmetry group parameter increments
$\{\ssp_i^{(0)}\,,\,\zeit_i^{(0)}\,,\,\gSpace_i^{(0)}\}$
such that the points
$\{\ssp_i^{(0)}\}$
lie close to the \rpo\ $p$,
\beq
	\ssp_{i+1}^{(0)}
\approx
    \matrixRep(\gSpace_i^{(0)}) \flow{{\zeit_i^{(0)}}}{\ssp_i^{(0)}}
\quad
    \mbox{cyclic in $i = 1, ..., n$}
\,.
\eeq
Here the period and the shift of the \rpo\ $p$ are
$\period{p} \approx \sum \zeit_i\,,$
$\gSpace_p \approx \sum \gSpace_i \,.$
and the Lagrangian description of the flow is
$\ssp(\zeit) = \flow{\zeit}{\ssp(0)}$
We want to determine
$(\Delta\ssp_i\,,\,\Delta\zeit_i\,,\,\Delta\gSpace_i)$
corresponding to the exact \rpo,
\bea
	\ssp_{i+1} + \Delta \ssp_{i+1} &=& \matrixRep(\gSpace_i + \Delta \gSpace_i)
		\flow{\zeit_i + \Delta \zeit_i}{\ssp_i + \Delta \ssp_i} \continue
		&&  \mbox{cyclic in } i = 1, ..., n
\,.
\eea
To linear order in
\bea
&& (\Delta\ssp_i^{(m+1)}\,,\, \Delta\zeit_i^{(m+1)}\,,\,\Delta\gSpace_i^{(m+1)}) \\
&&=
(\ssp_i^{(m+1)}-\ssp_i^{(m)}\,,\,
 \zeit_i^{(m+1)}-\zeit_i^{(m)}\,,\,
 \gSpace_i^{(m+1)}-\gSpace_i^{(m)}) \nonumber
\eea
the improved Newton guess
$
(\ssp_i^{(m+1)}\,,\,\zeit_i^{(m+1)}\,,\,\gSpace_i^{(m+1)} )
$
is obtained by minimizing the effect of perturbations along
the spatial, time and phase directions,
\beq
	\ssp_{i+1}^{'} - \matrixRep_{i+1} \flow{{\zeit_i}}{\ssp_i}
= \matrixRep_{i+1}\left(
  \groupTan_{i+1} \Delta \gSpace_i
+ \vel_{i+1} \Delta \zeit_i
+ \jMps_{i+1} \Delta \ssp_i
    \right)
\,,
\ee{PCnetwonStep}
where, for brevity,
$\ssp_{i}^{(m+1)} = \ssp_{i}^{(m)} + \Delta \ssp_{i}^{(m)}
   = \ssp_{i}^{'}$,
$\ssp_{i}^{(m)} = \ssp_{i}$,
$\matrixRep(\gSpace_i) = \matrixRep_{i+1}$,
$\vel (\ssp_{i}(\zeit_{i})) = \vel_{i+1}$,
$\jMps^{\zeit_i}(\ssp_i) = \jMps_{i+1}$,
$\groupTan(\ssp_{i}(\zeit_{i})) = \Lg \ssp_{i}(\zeit_{i}) = \groupTan_{i+1}$,
\etc.
For sufficiently good initial guesses,
the improved values converge under Newton iterations to
the exact values
$(\Delta\ssp_i\,,\,\Delta\zeit_i\,,\,\Delta\gSpace_i)$
=$(\Delta\ssp_i^{(\infty)}\,,\,\Delta\zeit_i^{(\infty)}\,,\,\Delta\gSpace_i^{(\infty)})$
at a super-exponential rate.

In order to deal with the marginal multipliers along the time and group
orbit directions, one needs to apply a pair of constraints, which
eliminate variations along the marginal directions on the \rpo s\ $2D$
torus: a local Poincar\'e section orthogonal to the flow, and a local slice
orthogonal to the group orbit at each point along the orbit,
\beq
   \braket{\vel(\ssp_i )}{\Delta \ssp_i} = 0
\,,\qquad
   \braket{\groupTan(\ssp_i )}{\Delta \ssp_i} = 0
\,.
\ee{RPOConstrsLocal}
We can rewrite everything as one matrix equation:
\beq \label{eq:multishootmatrix}
	A \Delta = E, \quad \mbox{where,}
\eeq
\begin{widetext}
\bea 
	A &=& \left(
	\begin{array}{ccccccccccc}	
	  \matrixRep_{2} \jMps_{2} &
	  \matrixRep_{2} \vel_2 &
	  \Lg \LieEl_{2} \flow{\zeit_1}{\ssp_1} &
	  - \matId & 0 & 0 & 0 & \cdots & 0 & 0 & 0 \\
	  \vel(\ssp_1) & 0 & 0 & 0 & 0 & 0 & 0 & \cdots & 0 & 0 & 0 \\
	  \groupTan(\ssp_1) & 0 & 0 & 0 & 0 & 0 & 0 & \cdots & 0 & 0 & 0 \\
	  0 & 0 & 0 &
	  \matrixRep_{3} \jMps_{3} &
	  \matrixRep_{3} \vel_3 &
	  \Lg \LieEl_{3} \flow{\zeit_2}{\ssp_2}   &
	  - \matId & \cdots & 0 & 0 & 0\\
	  0 & 0 & 0 & \vel(\ssp_2) & 0 & 0 & 0 & \cdots & 0 & 0 & 0 \\
	  0 & 0 & 0 & \groupTan(\ssp_2) & 0 & 0 & 0 & \cdots & 0 & 0 & 0 \\
	  \vdots & \vdots & \vdots & \vdots & \vdots & \vdots & \vdots & \ddots & \vdots & \vdots & \vdots \\
	  - \matId & 0 & 0 & 0 & 0 & 0 & 0 & \cdots &
	  \matrixRep_{1} \jMps_{1} &
	  \matrixRep_{1} \vel_1 &
	  \Lg \matrixRep_{1} \flow{\zeit_1}{\ssp_1} \\
	  0 & 0 & 0 & 0 & 0 & 0 & 0 & \cdots & \vel(\ssp_n) & 0 & 0 \\
	  0 & 0 & 0 & 0 & 0 & 0 & 0 & \cdots & \groupTan(\ssp_n) & 0 & 0
	\end{array} \right) \, , \label{eq:AforNewton} \\
	\Delta &=&
	 (
	  \Delta \ssp_1, \,
	  \Delta \zeit_1, \,
	  \Delta \gSpace_1, \,
	  \Delta \ssp_2, \,
	  \Delta \zeit_2, \,
	  \Delta \gSpace_2, \,
	  \ldots , \,
	  \Delta \ssp_n, \,
	  \Delta \zeit_n, \,
	  \Delta \gSpace_n
	 )^T \, ,
	 \\
	E &=&
	 (
	  \ssp_{2} - \matrixRep_2 \flow{\zeit_1}{\ssp_1} , \,
	   0 	, \,
	   0 	, \,
	  \ssp_{3} - \matrixRep_3 \flow{\zeit_2}{\ssp_2} , \,
	  0 	, \,
	  0 	, \,
	  \ldots , \,
	  \ssp_{1} - \matrixRep_1 \flow{\zeit_n}{\ssp_n} , \,
	  0 	, \,
	  0 	
	  )^T \, . \label{eq:DeltaandE}	 
\eea
\end{widetext}
We then solve \refeq{eq:multishootmatrix} for $\Delta$ and update our initial
guess by adding the vector of the computed $\Delta$ values to it and iterate.

\section{Periodic Schur decomposition}
\label{s:schur}

Here we briefly summarize the periodic eigen decomposition\rf{DingCvit14}
needed for evaluation of Floquet multipliers for \twomode\ \po s. Due to
the non-hyperbolicity of the return map of
\reffig{fig:psectandretmap}\,(d), Floquet multipliers can easily differ
by 100s of orders of magnitude even in a model as simple as the \twomode\
system.

We obtain the Jacobian of the \rpo\ as the following multiplication of short-time
Jacobians from the multiple shooting computation of the previous section:
\bea
    \jMpsRed &=& \matrixRep_n \jMps_n  \matrixRep_{n-1} \jMps_{n-1} \, ... \, \matrixRep_1 \jMps_1  \continue
                 &=& \hat{\jMps}_n \hat{\jMps}_{n-1} \, ... \, \hat{\jMps}_1 \label{e-JacobianProduct} \\
                 && \mbox{where,}\, \hat{\jMps_i} = \matrixRep_i \jMps_i \in
                    \mathbb{R}^{4 \times 4}, i = 1,2,...,n \, . \nonumber
\eea
This Jacobian is the same with our previous definition in \refeq{e-rpoJacobian}
since $J_i$ and $\matrixRep_i$ commute with each other, and are multiplicative 
respectively in time and phase. In order to determine the eigenvalues of 
$\hat{\jMps}$, we bring each term appearing in the product 
\refeq{e-JacobianProduct} into periodic, real Schur form as follows:
\beq
    \jMpsRed_i = Q_i R_i Q_{i-1}^T \, ,
\eeq
where $Q_i$ are orthogonal matrices that satisfy the cyclic property: $Q_0 = Q_n$.
After this similarity transformation, we can define $R = R_k R_{k-1} ... R_1$ and
re-write the Jacobian as:
\beq
    \jMpsRed = Q_n R Q_n^T \, .
\eeq
The matrix $R$ is block-diagonal, in general, with $1 \times 1$ blocks for real
eigenvalues and $2 \times 2$ blocks for the complex pairs. It also has the same
eigenvalues as $\hat{\jMps}$. In our case, it is diagonal since all Floquet 
multipliers are real in the \twomode\ system \rpo s. For each \rpo , we have 
two marginal Floquet multipliers corresponding to the time evolution direction 
and the continuous symmetry direction, in addition to one expanding and one 
contracting eigenvalue.

\bibliography{../../bibtex/siminos}

\end{document}
