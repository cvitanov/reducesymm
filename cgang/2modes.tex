% $Author$ $Date$
% for public version, toggle \draftfalse in setup2modes.tex
%     (that removes all comments, the blog)

% siminos/cgang/2modes.tex    this is master file:    pdflatex 2modes
%     then:    pdflatex def2modes; bibtex def2modes; pdflatex def2modes; pdflatex def2modes

\documentclass[aip,cha,reprint,
secnumarabic,
nofootinbib, tightenlines,
nobibnotes, showkeys, showpacs,
groupedaddress
%preprint,%
%author-year,%
%author-numerical,%
]{revtex4-1}
% aip,cha,preprint,numerical, nofootinbib,
% You should use BibTeX and apsrev.bst for references
% Choosing a journal automatically selects the correct APS
% BibTeX style file (bst file), so only uncomment the line
% below if necessary.
%\bibliographystyle{apsrev4-1}

\newcommand{\version}{atlas ver. 0.1, Apr 26 2012}
% Predrag from atlas12      ver. 1.1, Apr 25 2012}

        \input setup2modes
        \input ../inputs/def
        \input def2modes

\begin{document}

\title[Low-dimensional cartography]
{Cartography of a 4-dimensional flow: A visual guide to sections and slices}

\author{Daniel Borrero-Echeverry}
\email{borrero@gatech.edu.}
\author{Keith M. Carroll}
\author{Bryce Robbins}
\author{Evangelos Siminos}
\author{Lei Zhang}
\affiliation{
 School of Physics and School of Mathematics,
 Georgia Inst. of Technology,
 Atlanta, GA  30332, USA
}

\date{\today}
%\date{18 December 2011}
%\setcounter{page}{1}

    \begin{abstract}
Symmetry reduction by the method of slices [blah blah]
the role they play in organizing chaos.
    \end{abstract}

\pacs{02.20.-a, 05.45.-a, 05.45.Jn, 47.27.ed, 47.52.+j, 83.60.Wc}
%% showpacs class option if PACS display desired
%% copied from siminos/blog/strategy.tex
% \PACS 02.20.-a \sep 05.45.-a \sep 05.45.Jn \sep 47.27.ed \sep 47.52.+j
% 02.20.-a      Group theory, mathematics
% 05.45.-a      Nonlinear dynamics and chaos
% 05.45.Jn      High-dimensional chaos
% 47.27.ed      Dynamical systems approaches (turbulent flows)
% 47.52.+j      Chaos in fluid dynamics
% 83.60.Wc      Flow instabilities
% 95.10.Fh      Chaotic dynamics

\keywords{
symmetry reduction,
equivariant dynamics,
relative equilibria,
relative periodic orbits,
slices,
moving frames
}%Use showkeys class option if keyword display desired
\maketitle

    \begin{quotation}
Today, it is possible to  [blah blah].
    \end{quotation}

\section{Introduction}
\label{s:intro}

Over the last decade, new insights into the dynamics of  [blah blah]

Our goals here are ?-fold:
(i)  [blah blah].
(ii) [blah blah].


\section{Section}
\label{s:cut}


As an example consider the  [blah blah],

%%%%%%%%%%%%%%%%%%%%%%%%%%%%%%%%%%%%%%%%%%%%%%%%%%%%%%%%%%%%%%%%%%%%%
%Predrag 2012-04-20: use Keith & Daniel's
%[x] trajectorynoarrows.png
%[ ] nearequilibriumnoarrows.png
%   make the same size as other 3, to facilitate inkscaping
%[x] bothsectionnoarrows.png
%[x] farequilibriumnoarrows.png
%
\begin{figure}
%\includegraphics[width=0.28\textwidth]{RoessTrjs2}%{Rossler_Equilibria2}{RoessTrjs}%
 \begin{center}
 \setlength{\unitlength}{0.20\textwidth}
(a)
  %tried \colorbox{white}{$\slicep{}^{(-)}$} but too much white space
  \begin{picture}(1,0.8736435)%
    \put(0,0){\includegraphics[width=\unitlength]{RoessTrajLbld}}%
    \put(0.27213233,0.11036704){\color[rgb]{0,0,0}\makebox(0,0)[lb]{\smash{$\slicep{}^{(-)}$}}}%
    \put(0.90759454,0.68935432){\color[rgb]{0,0,0}\makebox(0,0)[lb]{\smash{$\slicep{}^{(+)}$}}}%
  \end{picture}%
(b) %{RoessNearEq3}
  \begin{picture}(1,0.82646416)%
    \put(0,0){\includegraphics[width=\unitlength]{RoessNeareqLbld}}%
    \put(0.2468894,0.13781942){\color[rgb]{0,0,0}\makebox(0,0)[lb]{\smash{$\slicep{}^{(-)}$}}}%
  \end{picture}%
\\
(c)  %{RoessFarEq3}
  \begin{picture}(1,0.82646416)%
    \put(0,0){\includegraphics[width=\unitlength]{RoessFareqLbld}}%
    \put(0.81012394,0.72899125){\color[rgb]{0,0,0}\makebox(0,0)[lb]{\smash{$\slicep{}^{(+)}$}}}%
  \end{picture}%
(d)  %{RoessSctAtlas3}
  \begin{picture}(1,0.82646416)%
    \put(0,0){\includegraphics[width=\unitlength]{RoessBotheqLbld}}%
    \put(0.2468894,0.13781942){\color[rgb]{0,0,0}\makebox(0,0)[lb]{\smash{$\slicep{}^{(-)}$}}}%
    \put(0.81012394,0.66084196){\color[rgb]{0,0,0}\makebox(0,0)[lb]{\smash{$\slicep{}^{(+)}$}}}%
  \end{picture}%
 \end{center}
    \caption{
2-chart atlas for R\"ossler flow.
(a)
  The inner {\eqv} $\slicep{}^{(-)}$  is a (spiral-out) saddle-focus with
  a 2-dimensional unstable manifold and a 1-dimensional stable manifold.
  The outer {\eqv} $\slicep{}^{(+)}$ is a (spiral-in) saddle-focus, with
  a 2-dimensional stable manifold (basin boundary for initial conditions
  that either fall into the  chaotic attractor, or escape to infinity)
  and a 1-dimensional unstable manifold.
(b)
 Chart $\PoincS_{-}$ of the $\slicep{}^{(-)}$ neighborhood carved out of a
 \PoincSec\ plane through the inner {\eqv} $\slicep{}^{(-)}$ and its
 stable eigenvector, with \poincBord\ drawn as the solid red line.
(c)
  Chart $\PoincS_{+}$ (here viewed from below) is bounded by \poincBord\
  (solid red line) of a section through the outer {\eqv}
  $\slicep{}^{(+)}$  and its unstable eigenvector.  Note the ridge
  (dashed blue line): the chart stops at the ridge, and it does not
  intersect the strange attractor.
(d)
  A two-chart atlas of R\"ossler flow, with charts $\PoincS_{-}$ and
  $\PoincS_{+}$ oriented and combined so that the ridge (intersection of
  the two sections, indicated by the dashed blue line in the three
  figures) lies approximately between the \template s. Section
  hyperplanes beyond this ridge do not belong to the atlas.
    }
\label{fig:RoessTrjs}
\end{figure}
%%%%%%%%%%%%%%%%%%%%%%%%%%%%%%%%%%%%%%%%%%%%%%%%%%%%%%%%%%%%%%%%%%%%%

 [blah blah]

\section{Dynamics and symmetry}
\label{s:symm}

\subsection{\twoMode\ $\SOn{2}$-equivariant flow}
\label{s:twoMode}

% \item[2012-04-28 Predrag]
Consider a system constructed from Fourier modes
$m=(1,2)$,\rf{Dang86,AGHO288,PoKno05}, a representation of the symmetry
group $\SOn{2}$ (or `circl group') defined by rotation
\beq
(z_1,z_2) \rightarrow   (e^{i {\gSpace}}z_1,e^{i 2\,{\gSpace}} z_2)
\ee{Dang86(1.1)aa}
$\SOn{2}$ invariant polynomials are
\bea
u &=& {z}_1 \overline{z}_1
    \,,\quad
v = {z}_2 \overline{z}_2
    \continue
w &=& z_1^2 \overline{z}_2 + \overline{z}_1^2 {z}_2
    \,,\quad
q =  \PCedit{
      (z_1^2 \overline{z}_2 - \overline{z}_1^2 {z}_2)/\ii
    }
\,,
\label{Dang86(1.2)PK}
\eea
In polar coordinates $ {z}_1 = |u|^{1/2} e^{\ii\theta_1}$, $ {z}_2 =
|v|^{1/2} e^{\ii\theta_2}$ the  $w, q$ invariants take form
\bea
w &=& \PCedit{
      2\,\Re(z_1^2 \overline{z}_2) = 2\,u |v|^{1/2} \cos \psi
      }
    \continue
q &=&  \PCedit{ 2\,\Im(z_1^2 \overline{z}_2) = 2\,u |v|^{1/2} \sin \psi
    }
\,,
\label{Dang86(1.2)polar}
\eea
where $\psi = 2 \theta_1 - \theta_2$. The polynomials $\{u,v,w,q\}$ are
linearly independent, but related through one syzygy,
\beq
w^2+q^2 - u^2v =0
  \,,
\label{eq:syzPK}
\eeq
yielding a 3-dim\-ens\-ion\-al $\pS/\SOn{2}$ \reducedsp, a
symmetry-invariant repre\-sent\-ati\-on of the 4-dim\-ens\-ion\-al
\SOn{2} equivariant dynamics.
The dynamical equations for $\{u,v,w,q\}$ follow from the chain rule
\( %beq
 \dot{ u}_i= ({\partial u_i}/{\partial x_j}) \, \dot{x}_j
 \,,
\) %ee{HilbChainRl}
upon substitution
$\{{z}_1\,,\overline{z}_1\,, {z}_2\,,\overline{z}_2 \}$ $\to$
$\{u,v,w,q\}$. This yields
\bea
  \dot{u} &=& \overline{z}_1 \dot{z}_1 + {z}_1 \dot{\overline{z}}_1
\,,\qquad
  \dot{v} = \overline{z}_2 \dot{z}_2 + {z}_2 \dot{\overline{z}}_2
\continue
  \dot{w} &=& 2 \,\overline{z}_2 {z}_1 \dot{z}_1
           + 2\,{z}_2 \overline{z}_1 \dot{\overline{z}}_1
           + {z}_1^2 \dot{\overline{z}}_2
           + \overline{z}_1^2 \dot{z}_2
\continue
  \dot{q} &=&  (2\,\overline{z}_2 {z}_1 \dot{z}_1
           - 2\,{z}_2 \overline{z}_1 \dot{\overline{z}}_1
           + {z}_1^2 \dot{\overline{z}}_2
           - \overline{z}_1^2 \dot{z}_2
           )/\ii
\label{PKinvEqs}
\eea

Dangelmayr,\rf{Dang86} Armbruster, Guckenheimer and Holmes,\rf{AGHO288}
Jones and Proctor,\rf{JoPro87} and Porter and Knobloch\rf{PoKno05} (see
Golubitsky \etal\rf{golubII}, Sect. XX.1) have investigated bifurcations
in 1:2 resonance ODE normal form models to third order in the amplitudes.
We shall use here such a system with $\SOn{2}$ symmetry,
\begin{subequations}\label{eq:DangSO2}
\begin{align}
  \dot{z}_1 &= \mu_1\,z_1+a_1\,z_1|z_1|^2+b_1\,z_1|z_2|^2+c_1\,\overline{z}_1\,z_2\,\\
  \dot{z}_2 &= (\mu_2-\ii\, e_2)\,{z_2}+a_2\,z_2|z_1|^2+b_2\,z_2|z_2|^2+c_2\,z_1^2
\,,
\end{align}
\end{subequations}
with $z_1,\,z_2$  complex, and real valued  parameters. For parameters
far from bifurcation values the model has no physical motivation (in
particular, the parameter $(\mu_2-\ii\, e_2)$ is one of the many ways in
which $\On{2}$ symmetry can be broken), we use it purely to compare and
illustrate different symmetry reduction methods when dynamics is chaotic.
We shall refer to \refeq{eq:DangSO2} as the {\twoMode} system.
Substituting \refeq{eq:DangSO2} into \refeq{PKinvEqs} we obtain a set of
4 equations,
    \PC{2012-04-27 by Lei and Predrag, please recheck}
\bea
  \dot{u} &=& 2\,\mu_1\,u+2\,a_1\,u^2+2\,b_1\,u\,v+c_1\,w
\continue
  \dot{v} &=& 2\,\mu_2\,u+2\,a_2\,u\,v+2\,b_2\,v^2+c_2\,w
\continue
  \dot{w} &=& (2\,\mu_1+\mu_2)\,w+(2a_1+a_2)\,u\,w+(2b_1+b_2)\,v\,w
\ceq
             +2\,c_1\,u\,v +2\,c_2\,u^2 -2\,e_2\,q
\label{PKinvEqs1}\\
  \dot{q} &=& (2\mu_1+\mu_2)\,q+(2a_1+a_2)\,u\,q+(2b_1+b_2)\,v\,q
             +\frac{e_2}{2}\,w
\nnu
\eea
One can now either investigate the dynamics in this invariant basis or
plot the `image'\rf{GL-Gil07b} of solutions computed in the original,
equivariant basis in terms of these invariant polynomials.

\subsection{To do}
\label{s:ToDo}

\begin{itemize}
	\item[10.?] The \twoMode system: An $\SOn{2}$-equivariant flow with two Fourier modes
		\begin{itemize}
			\item(a) Show that the complex \twoMode\ system \refeq{eq:DangSO2}
			 may be rewritten as a 				 4-dimensional first order ODE system			
			\bea
				\dot{x}_1 &=& a_1 x_1^3 + b_1 x_1 x_2^2 + c_1 x_1 x_2 + a_1 x_1 y_1^2 + b_1 x_1 y_2^2 + \mu_1 x_1 + c_1 y_1 y_2
				\continue
				\dot{y}_1 &=& a_1 x_1^2 y_1 + c_1 x_1 y_2 + b_1 x_2^2 y_1 - c_1 x_2 y_1 + a_1 y_1^3 + b_1 y_1 y_2^2 + \mu_1 y_1
				\continue
				\dot{x}_2 &=& a_2 x_1^2 x_2 + c_2 x_1^2 + b_2 x_2^3 + a_2 x_2 y_1^2 + b_2 x_2 y_2^2 + \mu_2 x_2 - c_2 y_1^2 + e_2 y_2
				\continue
				\dot{y}_2 &=& a_2 x_1^2 y_2 + 2 c_2 x_1 y_1 + b_2 x_2^2 y_2 - e_2 x_2 + a_2 y_1^2 y_2 + b_2 y_2^3 + \mu_2 y_2
			\eea
			
			by substituting $z_1 = x_1 + i y_1$ and $z_2 = x_2 + i y_2$.
			
			\item(b) Now, show that the stability matrix ({\color{red}insert DasBuch eq. ref here, originally eq. 4.3}) A for this 				system is given by

\beq
A  \, =
\left( \begin{array}{cccc}
         3 a_1 x_1^2 + b_1 x_2^2 + c_1 x_2 + a_1 y_1^2 + b_1 y_2^2 + \mu_1 &  c_1 y_2 + 2 a_1 x_1 y_1 & c_1 x_1 + 2 b_1 x_1 x_2 & c_1 y_1 + 2 b_1 x_1 y_2 \\
        c_1 y_2 + 2 a_1 x_1 y_1  & a_1 x_1^2 + b_1 x_2^2 - c_1 x_2 + 3 a_1 y_1^2 + b_1 y_2^2 + \mu_1 & 2 b_1 x_2 y_1 - c_1 y_1 & c_1 x_1 + 2 b_1 y_1 y_2 \\
          2 c_2 x_1 + 2 a_2 x_1 x_2 & 2 a_2 x_2 y_1 - 2 c_2 y_1  & a_2 x_1^2 + 3 b_2 x_2^2 + a_2 y_1^2 + b_2 y_2^2 + \mu_2 & e_2 + 2 b_2 x_2 y_2\\
          2 c_2 y_1 + 2 a_2 x_1 y_2 & 2 c_2 x_1 + 2 a_2 y_1 y_2 & 2 b_2 x_2 y_2 - e_2 & a_2 x_1^2 + b_2 x_2^2 + a_2 y_1^2 + 3 b_2 y_2^2 + \mu_2
      \end{array} \right)
\,.
\ee{2modeStabMatrix}

\item(c) Now show that the \twoMode\ system is equivariant under infinitesimal SO(2) rotations
({\color{red}insert DasBuch eq ref here, originally 10.18}) by substituting the Lie algebra generator
    \beq
\Lg  \, =
\left( \begin{array}{cccc}
         0 & 1 & 0 & 0 \\
        -1 & 0 & 0 & 0 \\
         0 & 0 & 0 & 2\\
         0 & 0 & -2 & 0
      \end{array} \right)
\ee{LGTwoMode}

and A into into the equivariance condition ({\color{red}insert DasBuch eq. ref. here, originally 10.24}).
\end{itemize}

  \item[10.8]  An $\SOn{2}$-equivariant flow with two Fourier modes
  \item[10.10] $\SOn{2}$ equivariance of the {\twoMode} system
           for infinitesimal angles.
  \item[10.11] Visualizations of the 4-dimensional {\twoMode} system
  \item[10.1?] draw a group orbit for the {\twoMode} model
  \item[10.22] {\twoMode} system in polar coordinates.
  \item[10.23] The relative equilibria of the {\twoMode} system
  \item[10.24] Plotting the relative equilibria of
           the {\twoMode} system in polar coordinates
  \item[10.25] Plotting the relative equilibria of
           the {\twoMode} system in Cartesian coordinates
  \item[10.2?] construct a 2-chart atlas for a {\twoMode} system
\end{itemize}


 [blah blah]

Mercader and Prat\rf{MePrKn01} might
be a candidate if we decide to go with $\On{2}$ symmetry since really the
Rayleigh-Benard problem has $\On{2} \times Z_2$ symmetry and they are really
talking about breaking the $Z_2$ part.


 [blah blah]


 [blah blah]


\begin{enumerate}
  \item
        determine \emph{numerically} the \reqva\ of the
        $\SOn{2}$-equivariant Dangelmayr {\twoMode} system in polar coordinates
\bea
   0 &=&  r_1 (\mu_1 + a_1 r_1^2  + b_1 r_1 r_2
                 + c_1 r_2 \cos(\psi))  \continue
   0 &=& \mu_2 r_2 + a_2 r_1^2 r_2  + b_2 r_2^3
                 + c_2 r_1^2 \cos(\psi)\continue
   0 &=&  e - \left(c_2 \frac{r_1^2}{r_2} + 2\,c_1 r_2\right) \sin(\psi)
\,,
\label{eq:2modesAGpolarREQV}
\eea
        Here I stuck in tentatively an `$e$' term because something like
        that is needed to break $\On{2} \to \SOn{2}$, verify that it
        really does that. The first two equations are cubic, the third one you can use
        to eliminate $\cos(\psi)$, so my guess is that there could  be up to six real
        roots, but I have not thought it through. Once you have found parameters
        for which there are interesting \reqv\ solutions, then
  \item
        compute analytically the \stabmat\ \Mvar\ in polar coordinates
  \item
        Study eigenvalues, keep playing with parameters. We would like
        -preferably- no \reqv\ to be attracting limit cycle, and several of
        the \reqva\ to be complex-pair unstable, leading to chaos, to be
        visualized and sliced in Cartesian coordinates \refeq{eq:2modesAGH}.
  \item
        If you find a nice chaotic attractors, others can join in
        constructing an atlas for it. We just need one and only one
        example with non-trivial \chartBord s and at least 2 charts.
\end{enumerate}


 [blah blah]

the $e$ added really breaks the $O(2)$ symmetry to $SO(2)$
symmetry. As rotations don't change the equations and reflections change
the sign of the last equation, adding a nonzero $e$ will break the
reflection symmetry.

By eliminating $\psi$, I got two polynomial equations of order 10. So
there are 20 complex solutions. I tried parameters $\mu_1=\mu_2=-1$,
$a_1=a_2=b_1=b_2=c_1=c_2=1$, and got 8 real solutions. All the 8
solutions I got for $(r_1,r_2)$ are $(\pm 0.537655,\pm 0.537655)$ and
$(\pm 0.980269,\pm 0.980269)$. In this case, 6 equilibria points have
only real eigenvalues, two equilibria points have complex eigenvalues
with positive real parts. Will this case work? Others please check
whether the calculation is correct or not.

 [blah blah]

By definition, $r_i \geq 0$, so you have
only two roots: $(0.537655,0.537655)$ and $(0.980269,0.980269)$. I find
it surprising that $r_1=r_2$, as equations look asymmetric in $r_i$;
might be consequence of $a_1=a_2=b_1=b_2=c_1=c_2=1$ (what value $e$? also
$e=1$?), you want to break this artificial symmetry if it is the cause. If
$r=r_1=r_2$ you have
\bea
   0 &=&  r (\mu_1 + (a_1 + b_1) r^2
                 + c_1 r \cos(\psi))  \continue
   0 &=& r (\mu_2 + (a_2 + b_2)  r^2
                 + c_2 r \cos(\psi))\continue
   0 &=&  e - r \left(c_2 + 2\,c_1\right) \sin(\psi)
\,,
\label{eq:2modesAGpolarR1R2}
\eea
which looks degenerate for your coefficient values.
There is an \eqv\ for $r_1=r_2=0$. There is a subspace $r_1=0$, $r_2 > 0$
which you can solve analytically - it my cause us some trouble. If what
you end up solving is a polynomial in $r_1$, you want to divide it with
all these known roots, see whether anything of interest is still left.

 [blah blah]

randomly chose all
the parameters and see what kinds of eigenvalues we can get. Under the
condition $r_1$ and $r_2$ being nonnegative real numbers, it seems that I
always got two equilibria points. One possible set of parameters I found
may be of interest are
$\mu_1=-1,\mu_2=-4,a_1=1,a_2=1.5,b_1=3,b_2=2.5,c_1=3,c_2=3.5,e=0.1$. The
equilibria points are $(r_1,r_2)=(0.0516508, 1.26311)$ and
$(0.467095,0.2146)$. The corresponding eigenvalues are
$(19.9398,0.8495,-11.9818)$ and $(1.5352,-4.7992+0.0327i,
-4.7992-0.0327i)$


 [blah blah]

\begin{itemize}
  \item $\REQV{}{1} = (r_1,r_2,\psi)=(0.0516508, 1.26311,?)$ and
        $\REQV{}{2} = (0.467095,0.2146,?)$
  \item their plots in the Cartesian coordinates
  \item $\dot{\theta}$ to see how slow/fast are they. $\dot{\theta}$
        might be related to 4th eigenvalue, when you go back
        to Cartesian coordinates
  \item stability eigenvalues, eigenvectors of the \eqv\ $\EQV{0}$ at
        origin, at your parameter values - if it is stable, everything
        just might fall into it and die.
  \item plots of small perturbations of the above \eqv\ and \reqva\ in
        the Cartesian coordinates to see whether the dynamics looks
        chaotic
  \item $\REQV{}{1}$: 2 large positive eigenvalues looks scary - probably
        nothing re-visits this \reqv. A mildly unstable complex pair
        would have been sweeter. You get complex eigenvalue by Hopf-bifurcating off a
        stable orbit, typically.
  \item $\REQV{}{1}$: Does either unstable eigenvalue become a complex
        eigenvalue pair in Cartesian coordinates?
  \item $\REQV{}{2}$: contracting eigenvalues have very small imaginary
        part, so the presumably just rocket toward the \reqv, not much
        spiraling there. At least the unstable eigenvalue seems slow
        compared to all other eigenvalues.
  \item $\REQV{}{1}$: Does the unstable eigenvalue become a complex
        eigenvalue pair in Cartesian coordinates?
\end{itemize}

 [blah blah]

The Dangelmayr system\rf{Dang86} written in complex
coordinates $z_1,z_2$ reads
\begin{subequations}\label{eq:2modesDangSO2}
\begin{align}
  \dot{z}_1 &= \mu_1\,z_1+a_1\,z_1|z_1|^2+b_1\,z_1|z_2|^2+c_1\,\overline{z}_1\,z_2\,\\
  \dot{z}_2 &= (\mu_2-\ii\, e_2)\,z_1+a_2\,z_2|z_1|^2+b_2\,z_2|z_2|^2+c_2\,z_1^2
\end{align}
\end{subequations}
(I have replaced symmetry breaking term $e$ here and in \refeq{eq:2modesAGpolar}
with $e_2$ since in this constant is naturally paired with $\mu_2$.)
See also Armbruster, Guckenheimer and Holmes\rf{AGHO288} flow with
$\On{2}$ symmetry, Eq. \refeq{eq:2modesAGH} above.
In \texttt{siminos/cgang/Evangelos/dangelmayr\_so2\_int.nb}
I integrate \refeq{eq:2modesDangSO2} rather than the polar form \refeq{eq:2modesAGpolar},
as the former has no dangerous denominators.

the connection
of the constants in \refeq{eq:2modesDangSO2} with $e_2=0$ to the constants in
equation (2.3) of \refref{Dang86} with $n=2$, $m=1$ is $\mu_1=\nu\epsilon\alpha$,
$a_1=-\nu\epsilon$, $b_1=-\nu\epsilon\rho$, $c_1=-\nu\mu$, $\mu_2=\epsilon\beta$,
$a_2=-\epsilon\kappa$, $b_2=-\epsilon\epsilon'$, $c_2=\mu\mu'$.

Our \refeq{eq:2modesDangSO2} seems like a special case of
{\twoMode}\rf{PoKno05} equation (12) but some brave young gangster
has to derive the correspondence of parameters, so that we can properly cite
them. Predrag's introduction of $e_2$
seems to be the minimal modification required to break $\On{2}$ to $\SOn{2}$.
Some exploration of \refeq{eq:2modesDangSO2}
using \texttt{siminos/cgang/Evangelos/dangelmayr\_so2\_int.nb}
shows we can have chaos, so we can stick to it. I leave it to Lei \etal\
to pick most interesting parameter values.

the following set of
parameters may be interesting. $\mu_1=-0.14,\mu_2=1.175,
a_1=-0.245,a_2=\ESedit{-}3.44, b_1=1.326, b_2=-0.47, c_1=1, c_2=-1, e_2=0.855$. All
the eigenvalues have positive real parts and both of the equilibrium
points have conjugate complex eigenvalues. So there are nice spirals
around them. See \reffig{fig:dangelmayr_proj} for projections
3-dim space.

 [blah blah]

Not finished transferring yet: enter here entries from [2012-04-05] or later

 [blah blah]


 [blah blah]



 [blah blah]

we shall illustrate the key ideas by a much
simpler example, the $\SOn{2}$-equivariant  [blah blah],


%%%%%%%%%%%%%%%%%%%%%%%%%%%%%%%%%%%%%%%%%%%%%%%%%
% 2011-09-09, 2012-03-30 Predrag: add BeThMovFr to
%            continuous.tex overheads, and ChaosBook
% replace A27movFrame*.* everywhere
\begin{figure}
  	\begin{center}
  	\setlength{\unitlength}{0.20\textwidth}
  (a)
  	\begin{picture}(1,1.07802818)%
    	\put(0,0){\includegraphics[width=\unitlength]{CLEattractor}}%
    	\put(0.55152995,1.0139628){\color[rgb]{0,0,0}\makebox(0,0)[lb]{\smash{$z$}}}%
    	\put(0.05573445,0.0739776){\color[rgb]{0,0,0}\makebox(0,0)[lb]{\smash{$x_1$}}}%
    	\put(0.90013492,0.16491708){\color[rgb]{0,0,0}\makebox(0,0)[lb]{\smash{$x_2$}}}%
  	\end{picture}%	
  (b)
  	\begin{picture}(1,1.06440474)%
    	\put(0,0){\includegraphics[width=\unitlength]{CLEWurst01}}%
   		\put(0.55961552,1.00214901){\color[rgb]{0,0,0}\makebox(0,0)[lb]{\smash{$z$}}}%
   		\put(0.07008555,0.07304272){\color[rgb]{0,0,0}\makebox(0,0)[lb]{\smash{$x_1$}}}%
    	\put(0.90381504,0.16283301){\color[rgb]{0,0,0}\makebox(0,0)[lb]{\smash{$x_2$}}}%
  	\end{picture}
\\
(c)   \begin{picture}(1,0.94310243)%
    \put(0,0){\includegraphics[width=\unitlength]{CLE1SliceSmall.pdf}}%
    \put(0.48564392,0.89244183){\color[rgb]{0,0,0}\makebox(0,0)[lb]{\smash{$z$}}}%
    \put(0.07181137,0.03185892){\color[rgb]{0,0,0}\makebox(0,0)[lb]{\smash{$y_2$}}}%
    \put(0.77031544,0.100183){\color[rgb]{0,0,0}\makebox(0,0)[lb]{\smash{$x_2$}}}%
  \end{picture}%
(d)   \begin{picture}(1,1.05662086)%
    \put(0,0){\includegraphics[width=\unitlength]{CLE2slicesmall.pdf}}%
    \put(0.47706962,0.83002768){\color[rgb]{0,0,0}\makebox(0,0)[lb]{\smash{$z$}}}%
    \put(0.08719004,0.02997825){\color[rgb]{0,0,0}\makebox(0,0)[lb]{\smash{$y_2$}}}%
    \put(0.73025395,0.09287946){\color[rgb]{0,0,0}\makebox(0,0)[lb]{\smash{$x_2$}}}%
  \end{picture}
    \end{center}
  \caption
  [\CLf: $\cycle{01}$ {\rpo} group orbit]{
  \CLf, $d=5 \to 3$~dimensional $\{x_1,x_2,z\}$ projections:
  (a)
  The strange attractor.
  (b)
  The initial \reqv\ $\REQV{}{1}$ point is shown by the red dot, and its
  group orbit / trajectory by the dashed red line. One period of the
  $\cycle{01}$ {\rpo} is shown by the solid blue line. The group orbit of
  its (arbitrary) starting point is shown by the dashed blue line: after
  one period the trajectory has returned to the group orbit but with a
  different phase. The \wurst, \ie, the group orbit of the $\cycle{01}$
  trajectory (dark blue) is shown by the cyan surface. Following
  $\cycle{01}$ for 15 more periods (faint dotted lines) starts filling
  out this torus; in that time the slowly drifting \reqv\ $\REQV{}{1}$
  has advanced to the next red dot (red line).
Symmetry-reduced \cLf, $d=4 \to 3$~dimensional $\{x_2,y_2,z\}$ projections:
 (c)
 Strange attractor (a) reduced to a single \slice\ hyperplane, using
 $\REQV{}{1}$ as the template. The dynamics exhibits singular jumps
 (shown in red) due to forbidden crossings of the \chartBord. In contrast
 to the 1\dmn\ \poincBord s of \reffig{fig:RoessTrjs}, here \chartBord s
 are 3\dmn\ and hard to visualize.
 (d)
The 2-chart atlas (sketched in \reffig{fig:A29-1ridge}) of the same
strange attractor encounters no \chartBord s and exhibits no
singularities. The ridge (shown in brown) acts as a \PoincSec\ $\PoincS$ with red or
blue ridge points $\sspRed^*$ marking the direction of the crossing.
  }
\label{fig:CLf01group}
\end{figure}
%%%%%%%%%%%%%%%%%%%%%%%%%%%%%%%%%%%%%%%%%%%%%%%%%%

 [blah blah]

 [blah blah]

\section{Chart}
\label{s:slice}

 [blah blah]

One can write the equations for the flow in the \reducedsp\
$\dot{\sspRed} = \velRed(\sspRed)$ (for details see, for example,
\refref{DasBuch}) as
\bea
\velRed(\sspRed) &=& \vel(\sspRed)
     \,-\, \dot{\gSpace}(\sspRed) \, \groupTan(\sspRed)
\label{2modesEqMotMFrame}\\
\dot{\gSpace}(\sspRed) &=& \braket{\vel(\sspRed)}{\sliceTan{}}
                       /\braket{\groupTan(\sspRed)}{\sliceTan{}}
\,
\label{2modesreconstrEq}
\eea
which confines the motion to the \slice\ hyperplane. Thus, the dynamical
system $\{\pS,\map^t\}$ with continuous symmetry \Group\ is replaced by
the {\reducedsp} dynamics $\{\pSRed,\mapRed^t\}$: The velocity in the
full \statesp\ $\vel$ is the sum of $\velRed$, the velocity component in
the \slice\ hyperplane, and $\dot{\gSpace}\,\groupTan$, the velocity
component along the group tangent space. The integral of the {\em
reconstruction equation} for $\dot{\gSpace}$ keeps track of the group
shift in the full \statesp.


 [blah blah]

\section{Charting the \slice}
\label{s:chart}

Let us summarize the voyage so far:

 [blah blah]


How the charts are put together is best told as a graphic tale, in the 5
frames of Figs.  [blah blah]



\section{Conclusions}
\label{s:concl}

 [blah blah]

\begin{acknowledgments}
This report addresses the questions asked in the  2012 Chaos course
[blah blah].
We are indebted to
 [blah blah]
and
 [blah blah]
for inspiring discussions.
\end{acknowledgments}


\bibliography{../bibtex/siminos}


\ifdraft
    \onecolumngrid

    \newpage
\input flotsam
    \newpage
    \section{{\twoMode} daily blog}
    \label{chap:2modes}
\input ../blog/2modes

\begin{quote}
{\color{red} \large
May 1 2012,  11:30am - 2:20pm term projects due, Predrag's office
}
\end{quote}

\fi

\end{document}
