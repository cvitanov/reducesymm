%%%% for public version, toggle \draftfalse in setup2modes.tex
%    (that removes all comments, the blog)

% reducesymm/cgang/2modes.tex    this is master file:    pdflatex 2modes
%     then:    pdflatex def2modes; bibtex def2modes; pdflatex def2modes; pdflatex def2modes

% until 2012-08-20 this was in svn repo siminos/cgang/2modes.tex

\documentclass[aip,cha,
reprint,
secnumarabic,
nofootinbib, tightenlines,
nobibnotes, showkeys, showpacs,
superscriptaddress,
%preprint,%
%author-year,%
%author-numerical,%
]{revtex4-1}

\newcommand{\version}{2modes ver. 2.3, Nov 11 2014}
% Predrag                   ver. 2.3, Nov 11 2014
% Predrag                   ver. 2.2, Jul 24 2014
% Predrag                   ver. 2.1, Jul 14 2014
% Burak                     ver. 2.0, Jul  8 2014
% Predrag                   ver. 1.3, May 11 2014
% Burak                     ver. 1.2, May  6 2014
% Predrag                   ver. 1.1, Nov 16 2013
% Burak                     ver. 1.0, Oct  6 2013
% Predrag                   ver. 0.3, Aug  1 2012
% Predrag                   ver. 0.2, Apr 30 2012}
% Predrag from atlas12      ver. 0.1, Apr 25 2012}

        \input setup2modes
        \input def2modes

\begin{document}

\title[Periodic orbit analysis of a system with continuous symmetry]
{Periodic orbit analysis of a system with continuous symmetry - a tutorial}

\author{Nazmi Burak Budanur}
\email{budanur3@gatech.edu}
\affiliation{
 School of Physics and Center for Nonlinear Science,
 Georgia Institute of Technology,
 Atlanta GA 30332
}
\author{Daniel Borrero-Echeverry}
\affiliation{
 School of Physics and Center for Nonlinear Science,
 Georgia Institute of Technology,
 Atlanta GA 30332
}
\affiliation{
 Department of Physics,
 Reed College,
 Portland OR 97202
}
\author{Predrag Cvitanovi\'{c}}
\affiliation{
 School of Physics and Center for Nonlinear Science,
 Georgia Institute of Technology,
 Atlanta GA 30332
}
    \ifdraft
\date{\today}
    \else
\date{14 November 2014}
   \fi

\begin{abstract}
Dynamical systems with translational or rotational symmetry arise
frequently in studies of spatially extended physical systems, such as
Navier-Stokes flows on periodic domains. In these cases, it is natural to
express a state of the fluid in terms of a Fourier series truncated to a
finite number of modes. Here we study a 4-dimensional two-mode
SO(2)-equivariant model of this type, the smallest possible truncation
that retains the symmetry while remaining high-dimensional enough to
allow for chaotic dynamics. A crucial step in analysis of such a system
is symmetry reduction. We use the model to illustrate different
symmetry-reduction schemes. Its relative equilibria are conveniently
determined by rewriting the dynamics in terms of a symmetry-invariant
polynomial basis. However, for study of chaotic dynamics, the `method of
slices', applicable also to very high-dimensional problems, is
preferable. We show that a Poincar\'e section within the `slice' can be
used to further reduce this flow to what is for all practical purposes a
unimodal map. This enables us to systematically determine all relative
periodic orbits and their symbolic dynamics up to any desired period. We
then compute several dynamical averages using relative periodic orbits
and discuss the convergence of such computations.
\end{abstract}

\pacs{02.20.-a, 05.45.-a, 05.45.Jn, 47.27.ed, 47.52.+j, 83.60.Wc}
\keywords{
symmetry reduction,
equivariant dynamics,
relative equilibria,
relative periodic orbits,
periodic orbit theory,
slices,
moving frames, chaos
}
\maketitle

\begin{quotation}
Periodic orbit theory provides a way to compute dynamical averages for
chaotic flows by means of {\cycForm s} that relate the time averages of
observables to the spectra of unstable periodic orbits. Standard
{\cycForm s} are valid under the assumption that the stability
multipliers of all periodic orbits have a single marginal direction
corresponding to time evolution, and are hyperbolic in all other
directions. However, if a dynamical system has $N$ continuous symmetries,
periodic orbits are replaced by relative periodic orbits, invariant
$(N+1)$-dimensional tori with marginal stability in $(N+1)$ directions.
These exact invariant solutions arise in studies of turbulent flows, such
as the pipe flow or the plane Couette flow, where the translational
symmetries of the flow are approximated by carrying out simulations in
periodic domains. A state of the fluid can then be expressed as a Fourier
series, truncated to a large but finite (from tens to thousands) set of
Fourier modes. This paper is a tutorial on how such problems are to be
analyzed using periodic orbit theory. We illustrate all steps of
\rpo s analysis on what is arguably the simplest such dynamical system, a
`\twomode' model.
\end{quotation}

\section{Introduction}
\label{s:intro}

Recent experimental observations of travelling waves in pipe flows have
confirmed dynamical systems theory intuition that the invariant solutions
of \NSe\ play an important role in shaping the \statesp\ of turbulent
flows\rf{science04}. When one recasts fluid flow equations in a
particular basis, the outcome is an infinite dimensional dynamical system
that is often equivariant under transformations such as
translations, reflections and rotations. For example, when a periodic
boundary condition is imposed along the streamwise direction, equations
for pipe flow retain their form under streamwise translations, azimuthal
rotations and reflections about the central axis, \ie, they are invariant
under the actions of $\SOn{2} \times \On{2}$. In that case it is natural
that the states of the fluid be expressed in a Fourier basis. However,
as the system evolves the nonlinear terms in the equations mix the
Fourier modes, and thus the state of the system evolves both along
symmetry directions and directions transverse to them.
This phenomenon in dynamical systems with continuous symmetries
complicates their \statesp, and gives rise to the high dimensional coherent
solutions such as \reqva\ and \rpo s, which take on the roles played by
\eqva\ and \po s in flows without symmetry.

This paper is a tutorial that works through an example in order to
illustrate step-by-step how periodic orbit theory is applied to flows
with continuous symmetries, an analysis that should ultimately be applied
to turbulent flows, once sufficiently many exact invariant solutions
become numerically accessible. For this purpose, we study a \twomode\
\SOn{2} equivariant flow, of minimal dimensionality required for chaotic
dynamics. The paper is organized as follows: In \refsect{s:symm} we
define basic concepts and briefly review the relevant symmetry reduction
literature. In \refsect{s:twoMode}, we introduce the \twomode\ model
system, describe several of its representations, and utilize a
symmetry-reduced polynomial representation to find the only \reqv\ of the
system. In \refsect{s:numerics}, we show how the \mslices\ can be used to
quotient the symmetry and reduce the dynamics onto a symmetry-reduced
\statesp, or `\slice'. A Poincar\'e section within the \slice\ then
reduces the 4\dmn\ chaotic dynamics in the full \statesp\ to a
1-dimensional unimodal Poincar\'e return map. This return map is then
used to construct a finite grammar symbolic dynamics for the flow, and
determine {\em all} \rpo s up to a given period. In \refsect{s:DynAvers},
these orbits are used as input to various {\cycForm s} in order to calculate
dynamically interesting observables. In \refsect{s:concl},
we discuss possible applications of the \mslices\ to various spatially
extended systems.
\refAppe{s:newton} describes the multi-shooting method used to calculate
the \rpo s.
\refAppe{s:schur} discusses how periodic Schur decomposition can be used
to determine their Floquet multipliers, which can easily differ by 100s
of orders of magnitude even in a model as simple as the \twomode\ system.




\section{Continuous symmetries}
\label{s:symm}

A dynamical system $\dot{\ssp}=\vel(\ssp)$ is said to be
\emph{equivariant} or \emph{\Group-equivariant} under the
symmetry group \Group\ transformations if
                                                \toCB
\beq
	\vel( \ssp )
    =  \matrixRep(\LieEl)^{-1}\vel(\matrixRep(\LieEl)\ssp)
	\,
\ee{equiv}
for every \statesp\ point $\ssp \in \pS$ and every element $\LieEl \in
\Group$, where \LieEl\ is an abstract group element, and
$\matrixRep(\LieEl)$ is its $[d\!\times\!d]$ matrix representation.
Infinitesimally, the equivariance condition \refeq{equiv} is expressed as
a vanishing Lie derivative\rf{DasBuch}
                                                \toCB
\beq
  \Lg\vel(\ssp)  - \Mvar(\ssp) \, \groupTan(\ssp) =0
  \,,
\ee{inftmInv}
where
$ \groupTan(\ssp) = \Lg \ssp $ is the group tangent of $\ssp$,
and $\Mvar(\ssp)$ is the \stabmat\, with elements
$\Mvar_{ij}(\ssp)={\pde \vel_i}/{\pde\ssp_j}|_{\ssp}$.
$\Lg$ is the generator of infinitesimal transformations, such that
$\matrixRep(\theta) = e^{\theta\Lg}$, where the phase $\theta$
parametrizes the group action. As in the example studied here there is
only one continuous symmetry parameter $\theta$, we shall interchangeably
use notations $\matrixRep(\LieEl)$ and $\matrixRep(\theta)$.


If the trajectory of a point $\ssp_\stagn$ coincides with its group
orbit, \ie, the group parameter $\theta (\zeit)$ satisfies
\beq
\ssp (\zeit)
    = \ssp_\stagn + \int_0^\zeit \!\!d\zeit' \vel(\ssp (\zeit'))
    = \LieEl (\theta (\zeit))\,\ssp_\stagn
  \,
\ee{releq}
for all $\zeit$, $\ssp_\stagn$ is a \emph{\reqv}. Expanding
both sides of \refeq{releq} for infinitesimal time yields the
relation $\vel(\ssp_\stagn) = \dot{\theta}(\zeit) \Lg \ssp_\stagn$, which
must hold for all $\zeit$. Thus, for a \reqv\ the  \emph{\phaseVel}
is a constant, $\dot{\theta} = \velRel$. Multiplying the equivariance condition
\refeq{inftmInv} by $\velRel$ we find that \reqva\ satisfy
\beq
(\velRel \Lg - \Mvar (\ssp_\stagn) ) \vel (\ssp_\stagn) = 0
\,.
\ee{ReqvMargEig}

%Separate rpos into new paragraph, added space - DB 9/16/2014
A \statesp\ point $\ssp_\rpprime$ lies on a \emph{\rpo} of period
$\period{\rpprime}$ if its trajectory first intersects its group orbit after
a finite time $\period{\rpprime}$,
\beq
\ssp(\period{\rpprime})
    = \ssp_\rpprime
     + \int_0^\period{\rpprime} \!\!\!d\tau' \vel(\ssp (\tau'))
    = \LieEl (- \theta_\rpprime ) \ssp_\rpprime
  \,,
\ee{relpo}
with a non-zero phase $\theta_{\rpprime}$.

In systems with \SOn{2} symmetry, \rpo s are
topologically 2-tori where the trajectory of $\ssp_\rpprime$ traces out the
same path shifted by the group action over and over again. As we will
show in \refsect{s:numerics}, these tori can be convoluted and
difficult to visualize.

The linear stability of \rpo s is captured by their \emph{Floquet multipliers},
which are important since they appear in the \cycForm s. Floquet multipliers are
defined in the following way: First, we denote the Lagrangian description of the flow as
$\ssp(\zeit) = \flow{\zeit}{\ssp(0)}$ and define the Jacobian as
\beq
\jMpsRed_{\rpprime} = \LieEl (\theta_\rpprime ) \jMps^\period{\rpprime} (\ssp_\rpprime)
\, , \mbox{where}\quad
\jMps^{\zeit}_{ij} (\ssp(\zeit)) = \frac{\partial\ssp_i(\zeit)}{\partial\ssp_j(0)}\, .
\ee{e-rpoJacobian}

The Floquet multipliers are then given by the eigenvalues $\ExpaEig_{p,j}$ of $\jMpsRed_{\rpprime}$.
The magnitude of $\ExpaEig_{p,j}$ determines whether a small perturbation along its corresponding
eigendirection (or Floquet vector) will expand or contract after one period. If the magnitude of
$\ExpaEig_{p,j}$ is greater than $1$, the perturbation expands; if it is less than $1$, the perturbation
contracts. In systems with $N$ continuous symmetries, \rpo s  have $(N+1)$ marginal directions ($\left|\ExpaEig_{p,j}\right| = 1$),
which correspond to the temporal evolution of the flow and the $N$ symmetries. By applying symmetry reduction,
the marginal Floquet multipliers corresponding to the symmetries are replaced by $0$ and make periodic orbit
theory, which requires that the flow have only one marginal direction, applicable.

\emph{Symmetry reduction} is a coordinate transformation that maps
all the points on a group orbit $\LieEl (\theta) \ssp$, which are
equivalent from a dynamical perspective, to a single representative point in a symmetry reduced space.
Such a transformation converts \reqva\ and \rpo s to \eqva\ and \po s in a
reduced \statesp, with no loss of dynamical information; the full \statesp\
trajectory can always be retrieved via the reconstruction equation. One well-studied
technique for symmetry reduction, which works well for low-dimensional
dynamical systems, such as the Lorenz system,
is to recast the dynamical equations in terms of invariant polynomials\rf{GL-Gil07b}.
Establishing such invariant polynomial bases, however, quickly becomes
impractical for systems with more than a dozen dimensions\rf{gatermannHab}. In contrast,
the \mslices\ \rf{rowley_reconstruction_2000,BeTh04,SiCvi10,FrCv11,atlas12,ACHKW11,BudCvi14},
which we study in detail here, is a symmetry reduction scheme applicable to
high-dimensional flows like the \NS\ equations\rf{WiShCv14}.

\subsection{\Mslices}
\label{s-slice}

In a system with $N$ continuous symmetries, a \emph{\slice} \pSRed\ is a codimension $N$ submanifold
of \pS\ that cuts every group orbit once and only once. In the \emph{\mslices}, the solution
of a $d$-\dmn\ dynamical system is represented as a symmetry-reduced trajectory $\sspRed (\zeit)$ within the
$(d-N)$-\dmn\ \slice\ and $N$ time dependent group parameters $\theta(\zeit)$, which
map $\sspRed (\zeit)$ to the full \statesp\ by the group action $\LieEl(\theta(\zeit))$.

While this idea goes back to Cartan\rf{CartanMF},
Rowley and Marsden\rf{rowley_reconstruction_2000}
were the first to apply it to a spatially extended nonlinear flow. They used it to study the dynamics of
the $1D$ \KS\ equation in the neighborhood of
a \reqv, using the \reqv\ itself as the \slice\ `\template'.
Independently, Beyn and Th\"{u}mmler\rf{BeTh04} applied
the \mslices\ to `freeze' spiral waves in reaction-diffusion systems.

The definition given above for the \slice\ puts no restriction on its shape
and offers no guidance on how to construct it. In practice, a
local approximation of the slice called a \emph{\slicePlane} can be constructed
in the neighborhood of a point $\slicep$ by using $\slicep$ as
\emph{\template}. The \slicePlane\ is then defined as the hyperplane that
contains $\slicep$ and is perpendicular to its group tangent $\sliceTan{}
= \Lg \slicep$. The relationship between a \template\, its \slicePlane, and symmetry-reduced trajectories
is illustrated in \reffig{f-ReducTraj1}.

%% ReducTraj*.* - read dasbuch/book/FigSrc/inkscape/00ReadMe.txt
\begin{figure}
\begin{center}
 \setlength{\unitlength}{0.40\textwidth}
 %% \unitlength = units used in the Picture Environment
 \begin{picture}(1,0.8361641)%
   \put(0,0){\includegraphics[width=\unitlength]{ReducTraj5.pdf}}%
   \put(0.06854399,0.36282057){\color[rgb]{0,0,0}\rotatebox{-30.34758661}{\makebox(0,0)[lb]{\smash{$\pSRed$}}}}%
   \put(0.57768586,0.29773425){\color[rgb]{0,0,0}\rotatebox{0.0313674}{\makebox(0,0)[lb]{\smash{$\sspRed(0)$}}}}%
   \put(0.59310014,0.69932675){\color[rgb]{0,0,0}\rotatebox{0.03136739}{\makebox(0,0)[lb]{\smash{$\ssp(\zeit)$}}}}%
   \put(0.8268425,0.39772328){\color[rgb]{0,0,0}\rotatebox{0.03136739}{\makebox(0,0)[lb]{\smash{$\sspRed(\zeit)$}}}}%
   \put(0.81220962,0.66529577){\color[rgb]{0,0,0}\rotatebox{0.03136739}{\makebox(0,0)[lb]{\smash{$\LieEl(\theta(\zeit))\ssp(\zeit)$}}}}%
   \put(0.21150193,0.63610779){\color[rgb]{0,0,0}\rotatebox{0.0313674}{\makebox(0,0)[lb]{\smash{$\LieEl\,\slicep$}}}}%
   \put(0.37740434,0.49597258){\color[rgb]{0,0,0}\rotatebox{0.0313674}{\makebox(0,0)[lb]{\smash{$\slicep$}}}}%
   \put(0.3627714,0.69665188){\color[rgb]{0,0,0}\rotatebox{0.0313674}{\makebox(0,0)[lb]{\smash{$\sliceTan{}$}}}}%
 \end{picture}%
\end{center}
\caption{\label{f-ReducTraj1}The \slicePlane\ \pSRed\ is a hyperplane % \refeq{PCsect0}
passing through the {\template} point $\slicep$
and normal to its group tangent $\sliceTan{}$.
It intersects all group orbits (dotted lines) in an open
neighborhood of $\slicep$.  The full \statesp\ trajectory $\ssp(\tau)$ (solid black line) and the \reducedsp\
trajectory $\sspRed(\zeit)$ (solid green line) belong to the same group orbit
$\pS_{\ssp(\zeit)}$ and are equivalent up to a group rotation
$\LieEl\left(\theta(\zeit)\right)$.
}%
\end{figure}

Reduced trajectories $\sspRed (t)$ can be obtained in two ways: by post-processing data
or by reformulating the dynamics and integrating directly in the \slice. In the post-processing method, which is also called the \emph{method of moving frames}\rf{FelsOlver98,OlverInv} and can be applied to both numerical and experimental data,
one takes the data in the full \statesp\ and looks for the time dependent group parameter
that brings the trajectory $\ssp(\zeit)$ onto the \slice. That is, one finds $\theta (\zeit)$ such that $\sspRed(\zeit) = \LieEl(- \theta (\zeit)) \ssp (\zeit)$
satisfies the \slice\ condition:
\beq
\braket{\sspRed(\zeit) - \slicep}{\sliceTan{}} = 0
\,.
\ee{SliceCond}

In the second implementation, one reformulates the dynamics (for Abelian groups) as
\begin{subequations}\label{eq:so2reduced}
  \beq\label{eq:intSlice}
	\velRed(\sspRed) = \vel(\sspRed)
	-\dot{\theta}(\sspRed) \, \groupTan(\sspRed)
  \eeq
  \beq\label{eq:reconstruction}
	\dot{\theta}(\sspRed) = {\braket{\vel(\sspRed)}{\sliceTan{}}}/
				{\braket{\groupTan(\sspRed)}{\sliceTan{}}}
  \, ,
  \eeq
\end{subequations}
and directly calculates the symmetry-reduced trajectory directly by integrating $\sspRed (\zeit)$ and $\theta (\zeit)$.
In \refeq{eq:so2reduced}, $\velRed$ is the projection of the full \statesp\ velocity \vel(\ssp) onto the \slicePlane.
For a detailed derivation of \refeq{eq:so2reduced}, see \refref{DasBuch}.

While early studies\rf{rowley_reconstruction_2000, rowley_reduction_2003, BeTh04} applied the \mslices\ to a single solution at a time,
studying the nonlinear dynamics of extended systems requires symmetry reduction
of global objects, such as strange attractors or invariant manifolds.
In this spirit, Siminos and Cvitanovi\'{c}\rf{SiCvi10} used the \mslices\ to
quotient the \SOn{2} symmetry from the chaotic dynamics of \cLf. They showed that the
slice-dependent singularity of the reconstruction equation that occurs when the denominator
in \refeq{eq:reconstruction} vanishes (e.g., when the group tangents of the trajectory and the
template are orthogonal) causes the reduced flow to make discontinuous jumps.
This singularity was studied in detail by Froehlich and Cvitanovi\'{c}\rf{FrCv11}.

Two strategies have been proposed in order to handle this problem: The first attempts to
try to identify a template such that slice singularities are not visited
by the dynamics\rf{SiCvi10} or to use multiple `charts' of connected
slices\rf{rowley_reconstruction_2000,FrCv11}.
The latter approach was applied to \cLf\ by Cvitanovi\'{c} \etal~\rf{atlas12} and
to pipe flow by Willis, Cvitanovi\'{c}, and Avila\rf{ACHKW11}.
However, neither approach is straightforward to apply, particularly in
high-dimensional dynamical systems.

A third strategy has recently been proposed by Budanur
\etal\rf{BudCvi14}, who considered Fourier space discretizations of
partial differential equations (PDEs) with \SOn{2} symmetry. They showed
that in these cases a simple choice of \slice template, associated with
the first Fourier mode, results in a \slice\ in which it is highly
unlikely that generic dynamics visit the neighborhood of the singularity.
If the dynamics do occasionally come near the singularity, these close
passages can be regularized by means of a time rescaling.

Here, we shall illustrate this approach, which we call the
`first Fourier mode slice' method, and apply it to a 2-mode ODE normal form. Such
a system is arguably the simplest system with \SOn{2} equivariant dynamics that
can exhibit chaos.

From here on, we will refer to the set of points $\sspRed^*$ on the \slicePlane\ that satisfy
\beq
\braket{\groupTan(\sspRed^*)}{\sliceTan{}} = 0
\,
\ee{ChartBordCond}
as the \emph{\sliceBord}.

In the discussion so far, we have not specified any constraints on the symmetry group
to be quotiented, beyond the requirement that it be Abelian, which is required for \refeq{eq:so2reduced}
to be valid.
\ES{2015-05-20}{Maybe you should discuss why
you imposed the restriction on Abelian groups.}
Since we are interested in spatially extended systems with
translational symmetry, and in order to keep the notation compact,
we restrict our discussion to one dimensional PDEs describing
the evolution of a field $u(x,t)$ in a periodic domain.
By introducing a Fourier series expansion
\beq
	u(x,\zeit) = \sum\limits_{k=- \infty}^\infty u_k\left(\zeit\right) e^{i k x}, \,\,\,u_k = x_k + i y_k,
\ee{FourierSeries}
a PDE invariant under translations can be expressed as a system of coupled nonlinear
ODEs equivariant under the 1-parameter compact group of \SOn{2} rotations.

Truncating the expansion to $m$ modes, we
write the real and imaginary parts of the Fourier coefficients with
$k \geq 1$ as the state vector $\ssp =$ \cartpt{x_1, y_1, x_2, y_2,..., x_m, y_m}.
The action of the $\SOn{2}$ group on this vector
can then be expressed as a block diagonal matrix:
%More explicit form, does not fit in a column:
%\beq
	 %\LieEl (\theta)= \\
					  %\begin{pmatrix}
					  %\cos \theta & \sin \theta & 0               & 0              & \cdots & 0              & 0               \\
					 %-\sin \theta & \cos \theta & 0               & 0              & \cdots & 0              & 0               \\
					  %0             & 0 		   & \cos 2 \theta & \sin 2 \theta & \cdots & 0              & 0               \\
					  %0             & 0            &-\sin 2 \theta & \cos 2 \theta & \cdots & 0              & 0               \\
					  %\vdots       & \vdots      & \vdots         & \vdots        & \ddots & \vdots         & \vdots         \\
					  %0             & 0 		   & 0               & 0              & \cdots & \cos m \theta & \sin m \theta  \\
					  %0             & 0            & 0	             & 0              & \cdots &-\sin m \theta & \cos m \theta
					  %\end{pmatrix}
%\eeq
\beq
	\LieEl(\theta) = \begin{pmatrix}
						R(\theta) & 0 			  & \cdots & 0 \\
						0		   & R(2 \theta) & \cdots & 0 \\
						\vdots	   & \vdots 	  & \ddots & \vdots \\
						0		   & 0	          & \cdots & R (m \theta)
					   \end{pmatrix} ,
\ee{mmodeLieEl}
where
\beq
	R(n \theta) =	\begin{pmatrix}
					\cos n \theta & - \sin n \theta \\
					\sin n \theta & \cos n \theta
					\end{pmatrix}
\ee{rotationmatrix}
is the rotation matrix for $n$th Fourier mode.
The Lie algebra generator for $\LieEl(\theta)$ is given by
\beq
	 \Lg =  \begin{pmatrix}
			 0 & -1 & 0 & 0 & \cdots & 0 & 0 \\
			 1 & 0 & 0 & 0 & \cdots & 0 & 0 \\
			 0 & 0 & 0 & -2 & \cdots & 0 & 0 \\
			 0 & 0 & 2 & 0 & \cdots & 0 & 0 \\
			 \vdots & \vdots & \vdots & \vdots & \ddots & \vdots & \vdots \\
			 0 & 0 & 0 & 0 & \cdots & 0 & -m \\
			 0 & 0 & 0 & 0 & \cdots & m & 0
			 \end{pmatrix} .
\ee{mmodeLg}

In order to construct a \slicePlane\ for such a system, let us choose the following \slice\ \template:
\beq
	\slicep = (1, 0, ..., 0) .
\ee{firstmodetemp}
The \slice\ condition \refeq{SliceCond} then constraints points on the reduced trajectory to the hyperplane given by
\beq
	\sspRed = (\hat{x}_1, 0, \hat{x}_2, \hat{y}_2, ..., \hat{x}_m, \hat{y}_m) .
\ee{slicetemp}
As discussed earlier, group orbits should cross the \slice\ once and only once, which we achive by restricting the \slicePlane\ to the half-space where $\hat{x}_1 > 0$. In general, a \slicePlane\ can be constructed by following a similar procedure for any choice of \template, allowing the symmetry
reduction of the dynamics in a neighborhood of the \template\ bounded by the \sliceBord\ \refeq{ChartBordCond}.
However, the power of choosing template \refeq{firstmodetemp} becomes apparent by computing the border of its \slicePlane.
The points on \refeq{slicetemp} lie on the \sliceBord\ only if $\hat{x}_1 = 0$.
This means that as long the dynamics are such that the magnitude of the first mode never vanishes,
\emph{every} group orbit is guaranteed to have a unique representative point on the \slicePlane.
\footnote{Note that, in general, any template of the form $\slicep =$ \cartpt{\hat{x}'_1, \hat{y}'_1, 0,...,0} would work just as well since the first mode has the symmetry of a circle. The \slice\ \template\
\refeq{firstmodetemp} was chosen for notational and computational convenience.}
%

More insight can be
gained by writing the symmetry-reduced evolution equations \refeq{eq:so2reduced}
explicitly for the template \refeq{firstmodetemp}:
\begin{subequations}
\beq
\velRed ( \sspRed )  = \vel(\sspRed)
   - \frac{\dot{y}_1\left(\sspRed\right)}{\hat{x}_1} \, \groupTan(\sspRed) \, ,
\label{e-so2red1stmode}
\eeq
\ESedit{
  \beq\label{eq:reconstruction1stmode}
	\dot{\theta}(\sspRed) = \frac{\dot{y}_1(\sspRed)}{\hat{x}_1}
  \, .
  \eeq
}
\end{subequations}
Since the argument $\phi_1$ of a point $(x_1,y_1)$ in the first Fourier mode plane is given by $\phi_1=\tan^{-1}\frac{y_1}{x_1}$,
its velocity is
\beq
  \dot{\phi}_1 = \frac{x_1}{r_1^2}\dot{y}_1-\frac{y_1}{r_1^2}\,\dot{x}_1\,,
\eeq
where $r_1^2=x_1^2+y_1^2$. Therefore, on the \slicePlane \refeq{slicetemp}, where $\hat{y}_1=0$,
\beq\label{eq:phi1}
  \dot{\theta}(\sspRed) = \dot{\phi}_1(\sspRed)\,.
\eeq
That is, for our choice of \template\ \refeq{firstmodetemp}, the reconstruction phase coincides with
the first Fourier mode phase. From a group-theoretic point of view,
this choice of template is therefore more natural than the more physically motivated templates used in
\refrefs{rowley_reconstruction_2000,BeTh04,SiCvi10,FrCv11,atlas12,ACHKW11}.

\ES{2014-05-20}{What would you think about introducing a functional
notation for $y_1$ as in \refeq{eq:reconstruction1stmode}?}
\PC{2014-05-25: We should copy and paste from \refref{BudCvi14} all stuff
about the in-slice time here? }
\ES{2014-05-25}{I agree with that suggestion, since rescaled time makes the
method work for more general flows than 2-modes,
and it should be explained here. However, we do not provide an example for it's
usefulness here. How about fishing for a second set of parameters where
trajectories come consistently close to $(0,0,\ldots)$, and using rescaled
time there? Even though $(0,0,\ldots)$ would not be visited, there should still
be apparent jumps that would go away by time-rescaling.}
In general, additional care must be taken when the dynamics approach the \slice\ border $\hat{x}_1 = 0$.
Whenever this happens, the near-divergence of $\velRed$ can be regularized by introducing a rescaled time coordinate such that
$d\hat{\zeit} = d\zeit / \hat{x}_1$\rf{BudCvi14}. However, in our study of the \twomode\ system that we will introduce below,
we omit this step since points with a vanishing first mode are in an invariant subspace of the flow and hence are never
visited by the dynamics.

\subsection{Postproccessing approach}
\label{s-mframes}

\ES{2014-05-15}{This section refers to the ``method of moving frames'' (postprocessing approach)
and as such belongs here. It has to be generalized a bit (I will do it if you agree with the change).
Second mode \slice\ can be used also for integration on the \slice\ and can be introduced earlier.
}

\begin{figure}%[H]
\centering
 \includegraphics[width=0.45\textwidth]{BBgorbitsandslice}
\caption{$\SOn{2}$ Group orbits of \statesp\ points \cartpt{0.75, 0, 0.1, 0.1}
(orange), \cartpt{0.5, 0, 0.5, 0.5} (green)
\cartpt{0.1, 0, 0.75, 0.75} (pink) and the first mode \slicePlane\
\refeq{slicetemp} (blue). The group tangents at the intersections with the
\slicePlane\ are shown as red arrows.
When the magnitude of the first Fourier mode becomes small relative to the
magnitude of the second one, the group tangent becomes more close to parallel to the
\slicePlane.}
\label{fig:BBgorbitsandslice}
\end{figure}

In \refsect{s-slice}, we explained the general procedure for reducing
the \SOn{2} symmetry by \mslices ; here, we focus on its geometrical
interpretation. The \slice\ defined by \refeq{firstmodetemp} and the directional constraint
$\hat{x}_1 > 0$
%
\DB{2014-05-15}{Think this should be $\hat{x}_1$. Revert to $x_1$ if I'm wrong.}
%
fixes the phase of the first complex Fourier mode to $0$. This also follows from
the fact that the reconstruction phase of the first Fourier mode \slice\ is the
phase of the first mode \refeq{eq:phi1}.\DB{2014-10-28}{I don't really understand what this   is saying} In complex representation, we can express
the relationship between Fourier modes ($\sspC_n = x_n + \ii y_n$) and their
representative points ($\sspRedC = \hat{x}_n +  \ii \hat{y}_n$) on the \slicePlane\
by the $\Un{1}$ action:
\beq
	\sspRedC_n = e^{-\ii n \phi_1} \sspC_n \, .
\ee{e-1stmodeTransform}
This relation provides another interpretation for the \sliceBord :  For template \refeq{firstmodetemp},
the \sliceBord\ condition \refeq{ChartBordCond} yields,
$|\sspRedC_1| = |\sspC_1| = 0$, which means that the phase of the first Fourier
mode \BBedit{and hence \DBedit{the transformation \refeq{e-1stmodeTransform}
are not defined.}}
%
This is illustrated in \reffig{fig:BBgorbitsandslice}, which shows the first Fourier mode \slicePlane\ along with three-dimensional projections of
the group orbits of points with decreasing $|\sspC_1|$. When the magnitude of the first
mode $\sqrt{\hat{x}_1^2 + \hat{y}_1^2}$, relative to that of the second mode is
small (pink curve in \reffig{fig:BBgorbitsandslice}), the group tangent has a larger
component parallel to the \slicePlane . If the first mode magnitude was exactly
$0$, the group tangent would lie entirely on the \slicePlane , satisfying the
\sliceBord\ condition.
%\subsection{Polar coordinates}
%\label{s-polar}

In \refref{PoKno05}, a polar coordinate representation of two Fourier mode
truncation is obtained by defining the $\LieEl$-invariant phase: $\Phi = \phi_2 - 2 \phi_1$
and three symmetry invariant coordinates \polpt{r_1, r_2 \cos \Phi, r_2 \sin \Phi}.
One can see by direct comparison with \refeq{e-1stmodeTransform}, which
yields $\sspRedC_1 = r_1$ and $\sspRedC_2 = r_2 e^{\ii \Phi}$, that this
representation is a special case $(m=2)$, of the \slice\ defined by
\refeq{firstmodetemp}. Corresponding ODEs for the polar representation
were obtained in \refref{PoKno05} by  chain rule and substitution. Note
that \mslices\ provides a general form \refeq{e-so2red1stmode} for symmetry
reduced time evolution.\DB{2014-10-28}{This section is kind of weird... doesn't seem to flow and we don't really use this stuff anywhere. Maybe consider deleting.}
\BB{2014-10-30}{In \refref{PoKno05} they use this representation and their plots
of chaotic dynamics (yes, they have parameters for which the \twomode\ system exhibits
chaos) look very similar to ours. We have to state somewhere that this similarity
is not an accident. I agree that it looks awkward as an individual subsection,
I removed the subsection title for now, we may move it to the `flow' section.}

\section{\twoMode\ $\SOn{2}$-equivariant flow}
\label{s:twoMode}

Our goal is to illustrate and compare continuous symmetry reduction methods
applicable to high-dimensional systems exhibiting chaos.
Dangelmayr,\rf{Dang86} Armbruster, Guckenheimer and Holmes,\rf{AGHO288}
Jones and Proctor,\rf{JoPro87} and Porter and Knobloch\rf{PoKno05} (see
Golubitsky \etal\rf{golubII}, Sect. XX.1) have investigated bifurcations
in 1:2 resonance ODE normal form models to third order in the amplitudes.
We use this model as a starting point from which we derive what may
be the simplest chaotic system with a continuous symmetry. We will
refer to as the {\twomode} system:
\bea
	\dot{z}_1 &=& (\mu_1-\ii\, e_1)\,z_1+a_1\,z_1|z_1|^2
				 +b_1\,z_1|z_2|^2+c_1\,\overline{z}_1\,z_2
	\continue
	\dot{z}_2 &=& (\mu_2-\ii\, e_2)\,{z_2}+a_2\,z_2|z_1|^2
				 +b_2\,z_2|z_2|^2+c_2\,z_1^2 \,,
	\label{eq:DangSO2}
\eea
with $z_1,\,z_2$  complex, and all parameters real valued. This complex
\twomode\ system \refeq{eq:DangSO2} is a 4-dimensional
first order real ODE system,
by substitution $z_1 = x_1 + i\,y_1$, $z_2 = x_2 + i\,y_2$,
\bea
\dot{x}_1 &=& (\mu_1 + a_1 r_1^2 + b_1 r_2^2 + c_1 x_2)x_1 + c_1 y_1 y_2 + e_1 y_1 % double checked DBE 05/22/2014
\continue
\dot{y}_1 &=& (\mu_1 + a_1 r_1^2 + b_1 r_2^2 - c_1 x_2)y_1 + c_1 x_1 y_2 - e_1 x_1 % double checked DBE 05/22/2014
\continue
\dot{x}_2 &=& (\mu_2 + a_2 r_1^2 + b_2 r_2^2)x_2 + c_2 (x_1^2 - y_1^2) + e_2 y_2 % double checked DBE 05/22/2014
\continue
\dot{y}_2 &=& (\mu_2 + a_2 r_1^2 + b_2 r_2^2)y_2 + 2 c_2 x_1 y_1 - e_2 x_2 % double checked DBE 05/22/2014
\continue
		  && \mbox{where } r_1^2 = x_1^2 + y_1^2\, , \quad r_2^2 = x_2^2 + y_2^2
\,.
\label{2mode4D}
\eea
The normal-form analysis leaves us with a large set of parameters
$\left(\mu_1,\mu_2,a_1,a_2,b_1,b_2,c_1,c_2,e_1,e_2\right)$.

Following the time-hallowed tradition of Lorenz\rf{lorenz},
H\'enon\rf{henon} and R\"ossler\rf{ross}, we have played with various
choices of parameters until settling on the set values used in all
numerical \twomode\  calculations presented here:
\beq
	\begin{tabular}{c c c c c c c c c c}
	% after \\: \hline or \cline{col1-col2} \cline{col3-col4} ...
	 $\mu_1$ & $\mu_2$ & $e_1$ & $e_2$ & $a_1$ & $a_2$ & $b_1$ & $b_2$ & $c_1$ & $c_2$ \\
	\hline
	 -2.8	& 1		  & 0	  & 1	  & -1	  & -2.66 & 0	  & 0 	  & -7.75 & 1
	\end{tabular}
	\label{eq:pars}
\eeq
While our choice of parameters is far from the bifurcation values studied
by previous authors \rf{Dang86,AGHO288,JoPro87,PoKno05}, and thus the
model has no physical interpretation, it as the simplest model for the
study of chaotic dynamics in systems with continuous symmetry that we
know of: it is a 4\dmn\ $\SOn{2}$-equivariant model in which the three
dimensional $\SOn{2}$-reduced dynamics are chaotic.

It can be checked by inspection that eqs.~\refeq{eq:DangSO2} are
equivariant under the \Un{1}\ transformation
\beq
(z_1,z_2) \rightarrow   (e^{i {\gSpace}}z_1,e^{i 2{\gSpace}} z_2)
\,.
\ee{Dang86(1.1)aa}
In the real representation \refeq{2mode4D}, the $\SOn{2}$ group action
\refeq{Dang86(1.1)aa} is given by $\ssp'= \exp\left( \theta \Lg\right)\ssp$,
where $\transp{\ssp} =$ \cartpt{x_1, y_1,x_2, y_2}, and $\Lg$ is the Lie algebra
generator
\beq
\Lg  \, =
\left( \begin{array}{cccc}
         0 & -1 & 0 & 0 \\
         1 & 0 & 0 & 0 \\
         0 & 0 & 0 & -2\\
         0 & 0 & 2 & 0
      \end{array} \right)
\,.
\ee{LGTwoMode}
One can easily check that the real \twomode\ system \refeq{2mode4D}
satisfied the equivariance condition \refeq{inftmInv}.

The parameters $\{e_1,e_2\}$ break the $\On{2}$ symmetry of the
Dangelmayr normal form system\rf{Dang86} to an $\SOn{2}$-equivariant
system. As we will show below in \refeq{PKinvEqs1}, only the combination
$(2e_1-e_2)$ matters in the symmetry reduced dynamics, so for simplicity
we set $e_1=0$.

From \refeq{eq:DangSO2} we note that the \eqv\ point \((z_1,z_2)=(0,0)\)
is an invariant subspace, and that $z_1=0$, $z_2 \neq 0$ is a 2\dmn\
flow-invariant subspace,
\beq
  \dot{z}_1 = 0 % Double checked DBE 05/26/2014
\,,\qquad
  \dot{z}_2 = (\mu_2-\ii\, e_2 +b_2 |z_2|^2)\,{z_2} % Double checked DBE 05/26/2014
\,,
\ee{eq:DangSO2spsp}
with a single circular \reqv\ of radius $r_2 = \norm{z_2} = \sqrt{-\mu_2/b_2}$ with
\phaseVel\ $\velRel=-e_2/2$.
    \PC{recheck: is $\velRel=e_2$?} \BB{}{I confirm.} \ES{}{ES: If I write $z_2=e^{i2\phi}z2$, then I get $c=-e_2/2$.}
		\DB{}{I agree with Vagelis... If the phase velocity is defined as written down in Sect. II, s.t. $v(a_q) = c \Lg a_q$, then $v(0,0,x_2,y_2) = (0,0,e_2 y_2, -e_2 x_2)$ and $\Lg (0,0,x_2,y_2) = (0,0,-2 y_2, 2 x_2)$ so $c = -e_2/2$.}
At the origin $\Mvar$ commutes with $\Lg$, and thus can be block-diagonalized
into two $[2\!\times\!2]$ matrices.
% According to {\bf [2012-04-27 Daniel]},
The \cartpt{0,0,0,0} \eqv\ eigenvalues are $\Lyap_1 = \mu_1$ with multiplicity 2 and
$\Lyap_3 = \mu_2 \pm i e_2$. The eigenvectors for $\Lyap_1$ are \cartpt{1,0,0,0} and
\cartpt{0,1,0,0} in the \cartpt{x_1,x_2,y_1,y_2} basis. The eigenvectors for $\Lyap_2$
are \cartpt{0,0,1,0} and \cartpt{0,0,0,1}.

In contrast, $z_2 =0$ is not, in general, a flow-invariant subspace, since the dynamics
\[
  \dot{z}_1 = (\mu_1-\ii\, e_1)\,z_1+a_1\,z_1|z_1|^2
\,,\qquad
  \dot{z}_2 = c_2\,z_1^2
\,.
\]
take the flow out of the $z_2 =0$ plane.
    \PC{should we check if anything of interest happens for $c_2 = 0$? }


\subsection{Invariant polynomial bases}
\label{s:invPol}

Consider the \statesp\ of a dynamical system
constructed from two complex Fourier modes\rf{Dang86,AGHO288,PoKno05}
$m=(1,2)$, with the $\SOn{2} \simeq \Un{1}$ group action given by
rotation \refeq{Dang86(1.1)aa}. In this
case it is easy to construct a set of four real-valued
$\SOn{2}$ invariant polynomials
\bea
u &=& {z}_1 \overline{z}_1
    \,,\quad
v = {z}_2 \overline{z}_2
    \continue
w &=& z_1^2 \overline{z}_2 + \overline{z}_1^2 {z}_2
    \,,\quad
q = (z_1^2 \overline{z}_2 - \overline{z}_1^2 {z}_2)/\ii
\,.
\label{Dang86(1.2)PK}
\eea
The polynomials $\{u,v,w,q\}$ are
linearly independent, but related through one syzygy,
\beq
w^2+q^2 - 4\,u^2v = 0 % Double checked syzygy is satisfied by eq Dang86(1.2)PK DBE 05/22/2014
  \,,
\label{eq:syzPK}
\eeq
which confines the dynamics to a 3-dim\-ens\-ion\-al $\pSRed=\pS/\SOn{2}$
\reducedsp\ manifold, a symmetry-invariant repre\-sent\-ati\-on of the
4-dim\-ens\-ion\-al \SOn{2} equivariant dynamics. By construction $u \geq
0$, $v \geq 0$, but $w$ and $q$ can be of either sign. That is explicit
in in polar coordinates $ {z}_1 = |u|^{1/2} e^{\ii\phi_1}$, $ {z}_2 =
|v|^{1/2} e^{\ii\phi_2}$, where the  $w, q$ invariants take the form
\bea
w &=& 2\,\Re(z_1^2 \overline{z}_2) = 2\,u |v|^{1/2} \cos \psi %Triple checked DBE 05/22/2014
\continue
q &=& 2\,\Im(z_1^2 \overline{z}_2) = 2\,u |v|^{1/2} \sin \psi %Triple checked DBE 05/22/2014
\,,
\label{Dang86(1.2)polar}
\eea
where $\psi = 2 \phi_1 - \phi_2$.

The dynamical equations for \invpt{u,v,w,q} follow from the chain rule
\( %beq
 \dot{ u}_i= \sum_j ({\partial u_i}/{\partial \ssp_j}) \, \dot{\ssp}_j
 \,
\) %ee{HilbChainRl}
. This yields
\bea
  \dot{u} &=& \overline{z}_1 \dot{z}_1 + {z}_1 \dot{\overline{z}}_1 % Triple checked DBE 05/22/2014
\,,\qquad
  \dot{v} = \overline{z}_2 \dot{z}_2 + {z}_2 \dot{\overline{z}}_2 % Triple checked DBE 05/22/2014
\continue
  \dot{w} &=& 2 \,\overline{z}_2 {z}_1 \dot{z}_1 % Triple checked DBE 05/22/2014
           + 2\,{z}_2 \overline{z}_1 \dot{\overline{z}}_1
           + {z}_1^2 \dot{\overline{z}}_2
           + \overline{z}_1^2 \dot{z}_2
\continue
  \dot{q} &=&  (2\,\overline{z}_2 {z}_1 \dot{z}_1 % Triple checked DBE 05/22/2014
           - 2\,{z}_2 \overline{z}_1 \dot{\overline{z}}_1
           + {z}_1^2 \dot{\overline{z}}_2
           - \overline{z}_1^2 \dot{z}_2
           )/\ii
\label{PKinvEqs}
\eea
Substituting  \refeq{eq:DangSO2} into \refeq{PKinvEqs} we obtain the set
of 4 $\SOn{2}$-invariant equations,

\bea
  \dot{u} &=& 2\,\mu_1\,u+2\,a_1\,u^2+2\,b_1\,u\,v+c_1\,w % Triple checked DBE 05/22/2014
\continue
  \dot{v} &=& 2\,\mu_2\,v+2\,a_2\,u\,v+2\,b_2\,v^2+c_2\,w % Triple checked DBE 05/22/2014
\continue
  \dot{w} &=& (2\,\mu_1+\mu_2)\,w+(2a_1+a_2)\,u\,w+(2b_1+b_2)\,v\,w % Triple checked DBE 05/22/2014
\ceq
             +\, 4c_1\,u\,v + 2c_2\,u^2 +(2e_1 - e_2)\,q
\label{PKinvEqs1}\\
  \dot{q} &=& (2\mu_1+\mu_2)\,q+(2a_1+a_2)\,u\,q
\ceq
             +(2b_1+b_2)\,v\,q
             -(2e_1-e_2)\,w % Triple checked DBE 05/22/2014
\,.
\nnu
\eea
Note that the $\On{2}$-symmetry breaking parameters
 $\{e_1,e_2\}$ of the
Dangelmayr normal form system\rf{Dang86} appear only in the
relative phase combination $(2e_1-e_2)$.
%[2012-07-31 Evangelos]
Using the syzygy \refeq{eq:syzPK} we can
eliminate $q$ from \refeq{PKinvEqs1} to get
    \PC{
    Note that $4u^2v-w^2 = 4u^2v(1-\cos^2\psi)$, so
    no serious singularity is introduced this way. Perhaps
    write equations of $(u,v,\cos \psi)$ as in the
    ChaosBook exercises?
    }
\bea
  \dot{u} &=& 2\,\mu_1\,u+2\,a_1\,u^2+2\,b_1\,u\,v+c_1\,w \nonumber % Triple checked DBE 05/22/2014
\\
  \dot{v} &=& 2\,\mu_2\,v+2\,a_2\,u\,v+2\,b_2\,v^2+c_2\,w \label{PKinvEqs1syz}  % Triple checked DBE 05/22/2014
\\
  \dot{w} &=& (2\,\mu_1+\mu_2)\,w+(2a_1+a_2)\,u\,w+(2b_1+b_2)\,v\,w % Triple checked DBE 05/22/2014
\ceq
             +\, 4c_1\,u\,v + 2c_2\,u^2 +(2e_1 - e_2)(4u^2v-w^2)^{1/2}\,
  \nonumber
\eea

One can now either investigate the dynamics in this invariant basis or
plot the `image'\rf{GL-Gil07b} of solutions computed in the equivariant
basis \refeq{eq:DangSO2} in terms of invariant polynomials
\refeq{Dang86(1.2)PK}.

%\item[2012-04-29 Predrag]
For the 4\dmn\ model at hand we find the invariant polynomials \refeq{PKinvEqs1}
and the polar coordinates \refeq{Dang86(1.2)polar} very useful for cross-checking the
full \statesp\ $\transp{\ssp} =$ \cartpt{x_1, x_2,y_1, y_2} calculations.
But even
for the simplest conceivable $\SOn{2}$ 4-dimensional flow their
construction requires a bit of algebra, and we do not know
how to carry out such constructions for very high\dmn\ flows,
such as the \KS\ flow, and the Navier-Stokes flow.


\subsubsection{\Eqva\ of the symmetry-reduced dynamics}
\label{s:eqva}

The first step in elucidating the geometry of attracting
sets is the determination of their \eqva. We shall now show that the problem of determining
the \eqva\ of the symmetry-reduced \twomode\ \refeq{PKinvEqs1} system \invpt{u*,v*,w*,q*} can be reduced to finding the roots of a multinomial expression.\ES{2014-05-15}{Why do you count complex
solutions as equilibria?}
%[2012-04-28 Predrag]
First, let we define
\beq
A_1= \mu_1+a_1\,u+b_1\,v
    \,,\qquad
A_2 = \mu_2+a_2\,u+b_2\,v
\ee{PKinvEqs2a}
then rewrite \refeq{PKinvEqs1} as
%     \newpage
\bea
  0  &=&  2\,A_1\,u +c_1\,w % Double checked DBE 05/24/2014
    \,,\qquad
  0  =  2\,A_2\,v +c_2\,w % Double checked DBE 05/24/2014
\continue
  0  &=& (2\,A_1+ A_2)\,w
          +2\,\left(c_2\,u+2\,c_1\,v\right)\,u % Double checked DBE 05/24/2014
          \ceq
		  + (2e_1-e_2)\,q
\label{PKinvEqs3}\\
  0  &=& (2\,A_1+ A_2)\,q - (2e_1-e_2)\,\,w % Double checked DBE 05/24/2014
\nnu
\eea
We already know \invpt{0,0,0,0} and \invpt{0,-\mu_2/b_2,0,0} roots, so we are looking only
for the $u>0$, $v>0$, $w,q \in \reals$ solutions; there could be problems\ES{2014-05-15}{What problems?}
from the non-generic roots with either $w=0$ or $q=0$, but not both
simultaneously, syzygy \refeq{eq:syzPK} precludes that. $w$ and/or $q$
can be eliminated by obtaining the following relations from \refeq{PKinvEqs3}:
\bea
	w  &=& - \frac{2\,u}{c_1}\,A_1 = - \frac{2\,v}{c_2}\,A_2 % Double checked DBE 05/24/2014
	\continue
	q &=& \frac{2(-2e_1+\,e_2)\,u\,v}{c_2\,u+2\,c_1\,v} . % Having issues with this DBE 05/24/2014... potentially drops a w = 0 root.
	\label{PKinvEqs4}
\eea
Substituting \refeq{PKinvEqs4}\DB{}{I think getting to the equation for $q$ throws out a potential $w = 0$ root. Do the first two equations then imply that u,v = 0? If, so then q = 0 and there's no problem, but I don't think that's the most general case.} into \refeq{PKinvEqs3} we get two bivariate
polynomials roots of which are the \eqva\ of the system \refeq{PKinvEqs1}:
\bea
	f(u,v) &=& c_2\,u\,A_1 - c_1\,v\,A_2 = 0 \,,\qquad  \nonumber
	\\
	g(u,v) &=&
 \left(4\,A_1^2 u^2 - 4\,c_1^2\,u^2 v\right)\left(c_2\,u+2\,c_1\,v\right)^2 \label{PKinvEqs5} %Double checked DB 04-30-2012
	\ceq
	+\,4\,c_1^2\,(-2e_1+e_2)^2\,u^2\,v^2 = 0
\,,
	\\
	deg(f) &=& 2, \, deg(g) = 6 \nonumber
\,.
\eea
%\DBedit{DB: Not sure where this factor of 2 comes from in $w =
%-\frac{2}{e_2} (2\,A_1+ A_2)\,q $. From the last equation in
%\refeq{PKinvEqs3}, I get $w = -\frac{1}{e_2} (2\,A_1+ A_2)\,q$.
%Therefore, I get I get  $q = \frac{2 e_2\,u\,v}{c_2\,u+2\,c_1\,v}$}
%2012-04-29 Predrag: thanks!

% \DBedit{DB: I get $g(u,v) = \left(w^2 - 4\,u^2
% v\right)\left(c_2\,u+2\,c_1\,v\right)^2 +\,4\,e_2^2\,u^2\,v^2 = 0$}
%2012-04-29 Predrag: thanks!
%2012-04-29 Predrag: should have I used the syzygy \refeq{eq:syzPK},
%$w^2 - 4\,u^2v = -q^2$ DB: If you plug the syzygy in you trivially get zero....

We divide the common multiplier $u^2$ from the second equation and by doing
so, eliminate one of the two roots at the origin, as well as
and the \cartpt{0,-\mu_2/b_2,0,0} root. Furthermore,
we scale the parameters and variables as
$\tilde{u} = c_2\,u$,
$\tilde{v} = c_1\,v$,
$\tilde{a_1} = a_1/c_2$,
$\tilde{b_1} = b_1/c_1$,
$\tilde{a_2} = a_2/c_2$,
$\tilde{b_2} = b_2/c_1$,
to finally get
\bea
\tilde{f}(\tilde{u},\tilde{v}) &=&
  \tilde{u}\,\tilde{A}_1 - \tilde{v}\,\tilde{A}_2 = 0 %Double checked DB 04-30-2012
\,,\qquad deg(f) = 2 \label{PKinvEqs5a}
\\
\tilde{g}(\tilde{u},\tilde{v}) &=&  %Double checked DB 04-30-2012
 \left(\tilde{A}_1^2
 - c_1\,\tilde{v}\right)
 \left(\tilde{u}+2\,\tilde{v}\right)^2
 +e_2^2\,\tilde{v}^2 = 0
\,,
\ceq
   deg(g) = 4 \label{PKinvEqs5b}
\\
 && \mbox{where }
\tilde{A}_1 = \mu_1+\tilde{a_1}\,\tilde{u}+\tilde{b_1}\,\tilde{v}
\,,\ceq
\qquad\quad \tilde{A}_2 = \mu_2+\tilde{a_2}\,\tilde{u}+\tilde{b_2}\,\tilde{v}
\,,
\label{PKinvEqs5c}
\eea

In order to find \reqva\ of the \twomode\ system, one has to solve two bivariate
polynomials \refeq{PKinvEqs5a} which, in general, is not a trivial task. However,
for the choice of parameters given by \refeq{eq:pars}, \refeq{PKinvEqs5a} yields
$\tilde{v} = (\mu_1 + \tilde{a}_1 \tilde{u})/(\mu_2 + \tilde{a}_2
\tilde{u})$. By substituting this into \refeq{PKinvEqs5b},
we can solve this equation which becomes a fourth order polynomial in $u$. Only
non-negative, real roots of this polynomial has a correspondence in the \twoMode\
\statesp\ since $u$ and $v$ are the squares of first and second mode amplitudes 
respectively. Two roots satisfy this condition, the \eqv\ at the origin:
\beq
	\invpol_{\EQV{}} = (0,0,0,0)^T , %\qquad \mbox{(double)}
\ee{eq:origin}
and the \reqv :
\beq
	\invpol_{\REQV{}{}} = (0.193569,0.154131,-0.149539,-0.027178)^T\,.
\ee{eq:reqv}
Note that by setting $b_2 = 0$, we send the \reqv\ at $\invpol =
(0,-\mu_2/b_2,0,0)$ to infinity. Thus, \refeq{eq:reqv} is the only \reqv\ 
of the \twomode\ system for our particular choice of parameters in \refeq{eq:pars}. 
This \reqv\ traces out its group orbit in the \SOn{2}-equivariant, real-valued
\statesp . One representative point on this orbit can be chosen as:
\[
  \left(x_1, y_1, x_2, y_2\right) = \left(0.439966, 0, -0.386267, 0.070204\right) .
\]


%\subsection{No chaos when the reflection symmetry is restored}
%\label{s:dfsafs}

%Before finishing our discussion of invariant polynomials, we have to make an
%important observation: Consider the case when the reflection symmetry breaking
%parameters $e_{1,2}$ in \refeq{eq:DangSO2} set to $0$. This restores 
%$\sspC_{1,2} \rightarrow \bar{\sspC}_{1,2}$ symmetry, which is broken in 
%the \SOn{2}-equivariant case. Now, note that setting $ e_{1,2} $ to $0$ 
%decouples $q$ in \refeq{PKinvEqs1} from the rest of the invariant polynomials.
%Furthermore, time derivative of $q$ becomes linear in itself, which can be
%expressed as:
%\beq
    %\dot{q} = \xi (u, v) q \, , 
%\ee{e-qlinearq}
%where $\xi$ is not a function of $q$. Hence we can write
%the time evolution of $q$ as
%\beq
    %q(t) =  e^{\int_0^t d \tau \xi (u(\tau), v(\tau))} q(0) \, . 
%\ee{e-qO2solq} 
%If we assume that the flow is bounded, then we can also assume that a long time 
%average of $\xi$ exists. Sign of this average would determine the long term 
%behavior of $q(t)$; it will either diverge or vanish depending on the sign of 
%$\langle \xi \rangle$ being positive or negative respectively. In either case, we 
%are left with three invariant polynomials that are related to each other by the 
%syzygy \refeq{eq:syzPK}, thus the flow is effectively confined on a two dimensional
%manifold, hence cannot exhibit chaos. We must stress that this is a special result
%when one has only first and second Fourier mode, and the equivariant normal form 
%includes terms upto the third order. 

\subsection{Visualizing \twomode\ dynamics}
\label{s:visual}

\begin{figure}%[H]
\centering
(a)\!\!\includegraphics[width=0.22\textwidth]{2modes-conf-reqv}%
(b)\!\!\includegraphics[width=0.22\textwidth]{2modes-conf-reqv}%
\caption{(Color online)
The \reqv\ \REQV{}{} in
 (a) the full \statesp, becomes an \eqv\ in
 (b) the symmetry-reduced configuration space.
}
\label{fig:2modes-conf-reqv}
\end{figure}

\begin{figure}%[H]
\centering
(a)\!\!\includegraphics[width=0.22\textwidth]{2modes-conf-ergodic}
(b)\!\!\includegraphics[width=0.22\textwidth]{2modes-conf-rpo}%
\caption{(Color online)
 (a) A typical ergodic trajectory of the \twomode\ system in the
 symmetry-reduced configuration space; clearly the unstable \reqv\
 \REQV{}{} in \reffig{fig:2modes-conf-reqv}\,(b) is far from a typical
 state.
 (b) Two repeats of \rpo\ \cycle{01}  (note the different time scale),
 in the symmetry-reduced configuration space. The dynamics is mostly
 dominated by the $m=1$ Fourier mode, interspersed by rapid shifts by
 $\approx \pm L/2$, dominated by the  $m=2$ Fourier mode.
}
\label{fig:2modes-conf}
\end{figure}

\begin{figure}%[H]
\centering
\includegraphics[width=0.45\textwidth]{2modes-ssp}
\caption{(Color online)
The same trajectories as in \reffig{fig:2modes-conf},
colored green, red and blue respectively,
in a 3D projection of the 4\dmn\ \statesp.
}
\label{fig:2modes-ssp}
\end{figure}

\begin{figure}%[H]
\centering
\includegraphics[width=0.45\textwidth]{2modes-invpol}
\caption{(Color online)
The same trajectories as in \reffig{fig:2modes-conf} and
\reffig{fig:2modes-conf-reqv}\,(a), colored green, red
and blue respectively, in a terms of 3 invariant polynomials.
The \reqv\ \REQV{}{} is now reduced to an \eqv, the green point.
}
\label{fig:2modes-invpol}
\end{figure}

\begin{figure}%[H]
\centering
\includegraphics[width=0.45\textwidth]{2modes-sspRed}
\caption{(Color online)
The same trajectories as in \reffig{fig:2modes-conf} and
\reffig{fig:2modes-conf-reqv}\,(a), colored green, red
and blue respectively, in the 3\dmn\  first Fourier mode \slicePlane. The
\reqv\ \REQV{}{} is now reduced to an \eqv, the green point. In contrast
to the invariant polynomial representation \reffig{fig:2modes-invpol},
the qualitative difference between shifts by $\approx L/2$ and $\approx
-L/2$ in near passages to the {\sliceBord} is very clear, and it leads to
the unimodal Poincar\'e return map of \reffig{fig:psectandretmap}.
}
\label{2modes-sspRed}
\end{figure}

\PC{2014-07-14
Why $\pm L/2$ in \reffig{fig:2modes-conf} and
\reffig{fig:2modes-conf-reqv} and not $\pm \pi$?
}
\ES{2014-05-15}{I have replaced the second-mode slice, double-angled
figure in \reffig{2modes-conf-reqv}\,(b) with one resulting by integrating on the
$(0,0,1,0)$ slice, for consistency with panel (c). I hope Burak will
replace it with a publication quality figure of the same representation.
The trick of angle doubling will be introduced in its own section. }

We visualize the dynamics of the \twomode\ system in four different
representations: 3D projections of the four-dimensional real valued
\statesp\ and invariant polynomials, in the 3D \slicePlane\ and on the 2D
configuration space plots on which the color-coded field $u(\conf,\zeit)$
is defined as follows:
\bea
	u(\conf, \tau) &=& \sum_{k=-2}^{2} \sspC_k(\zeit) e^{i k \conf}\, ,
	\continue && \mbox{where} \, \sspC_{-k} = \sspC_k^* \, \mbox{and} \,
	\sspC_0 = 0
\, .
\eea
\refFig{fig:2modes-conf}, \reffig{fig:2modes-conf-reqv}\,(a)
and \reffig{2modes-sspRed} show the sole \reqv\
\REQV{}{}, the \rpo\ \cycle{01}, and an ergodic trajectory of the
\twomode\ system in the four different representations discussed above.
Note that translation of the \reqv\ in the configuration space
\reffig{fig:2modes-conf-reqv}\,(a), corresponds to the \SOn{2} rotations in
the \statesp\ of Fourier modes in \reffig{2modes-ssp} (green curve) and
these orbits correspond to a single point in the symmetry reduced
representations of \reffig{fig:2modes-invpol} and \reffig{2modes-sspRed}.
Note also that the \rpo\ \cycle{01} translates/rotates as it advances in
configuration space (\reffig{fig:2modes-conf}\,(b)) and in the
equivariant \statesp\ \reffig{2modes-sspRed} (\reffig{2modes-ssp}),
whereas in the symmetry reduced plots (\reffig{fig:2modes-invpol} and
\reffig{2modes-sspRed}), it closes onto itself after one period.

\section{Numerical results}
\label{s-numerics}

\begin{figure}%[H]
\centering
 (a) \includegraphics[width=0.20\textwidth]{2modes-ssp} 
 (b) \includegraphics[width=0.20\textwidth]{2modes-invpol}
 \\
 (c) \includegraphics[width=0.20\textwidth]{2modes-sspRed}
 (d) \includegraphics[width=0.20\textwidth]{2modes-sspRed2}
\caption{(Color online) The \twomode\ system:
A typical chaotic trajectory (blue), a trajectory spiraling out
from the \reqv\ (green), 10 repeats of the shortest
$\period{} = 3.6415120$ \rpo\
(orange) plotted in (a) a 3D projection of its
four-dimensional \statesp; (b) invariant polynomials (c) first-mode
 \slicePlane , (d) second-mode \slicePlane\ with phase doubling }
\label{fig:Set1}
\end{figure}

To illustrate the \mslices\ on the \twoMode\ system we choose two relatively
simple sets of parameters for which we observe interesting dynamics. These
parameters are listed in \reftab{tab:pars}. In both sets we choose
$b_2 = 0$ and by doing so, we send the \reqv\ at $[0,-\mu_2/b_2,0,0]$ to infinity
and simplify the bivariate polynomials \refeq{PKinvEqs5a} such that from the
first equation \refeq{PKinvEqs5a} we can get the condition $\tilde{v} = (\mu_1 + \tilde{a}_1 \tilde{u})/
(\mu_2 + \tilde{a}_2 \tilde{u} - \tilde{u} \tilde{b}_1)$ and substitute into
the \refeq{PKinvEqs5b} to solve for single variable.
\begin{table}
	\begin{tabular}{c|c|c|c|c|c|c|c|c|c|c}
	% after \\: \hline or \cline{col1-col2} \cline{col3-col4} ...
	Parameters & $\mu_1$ & $\mu_2$ & $e_1$ & $e_2$ & $a_1$ & $a_2$ & $b_1$ & $b_2$ & $c_1$ & $c_2$ \\
	\hline
	(a) 	  & -2.8	& 1		  & 0	  & 1	  & -1	  & -2.66 & 0	  & 0 	  & -7.75 & 1	  \\
	\hline
	(b) 	  & 1		& -1	  & 0	  & 0	  & 0.47  & 0	  & -1	  & 0 	  & 1	  & -1	  \\	
	\end{tabular}
	\caption{Parameter sets that we used to study the \twoMode\ system.}
	\label{tab:pars}
\end{table}

We start with the first set of parameters, \reftab{tab:pars}\,(a). For this set,
after substituting the parameters with values 1 and 0 into the \refeq{eq:DangSO2},
the simplified \twoMode\ system \refeq{eq:DangSO2} has 3-parameters $\{ \mu_1, c_2, a_2 \}$:
\bea
\label{eq:DangSO2set1}
  \dot{z}_1 &=& \mu_1 \,z_1 - z_1|z_1|^2 +c_1\,\overline{z}_1\,z_2
  \continue
  \dot{z}_2 &=& (1-\ii)\,{z_2}+a_2\,z_2|z_1|^2+\,z_1^2
\,,
\eea
By solving the polynomials \refeq{PKinvEqs5} with the parameter set \reftab{tab:pars}\,(a),
we get the \eqva\ of the system in the invariant polynomial basis \refeq{Dang86(1.2)PK} as
\bea
	\label{eq:eqvaset1}
	(u,v,w,q) &=& (0,0,0,0) \qquad \mbox{(double)}
			  \continue
			  &=& (0.193569,0.154131,-0.149539,-0.027178)
			  \continue
			  &=& (0,- \infty,0,0)
			  \continue
			  &=& (-2.8,0,0,0)
			  \continue
			  &=& (-5.52172,0.12361,-3.87834,0.183536)
			  \continue
			  &=& (-0.991847 \mp 0.14571 \ii,
				   \ceq
				   -0.0640782 \pm 0.00260791 \ii,
				   \ceq
				   0.468295 \pm 0.0306953 \ii,
				   \ceq
				   -0.067488 \pm 0.687486 \ii)
\eea
Among these roots, only the origin and the second root has a correspondance
in the $\SOn{2}$-equivariant \statesp\ as \eqv\ and \reqv\ respectively.

Starting close to the relative equilibrium $x_0 = (0.439966, 0, 0.386267, 0.070204)$
corresponding to the second root in \refeq{eq:eqvaset1} we integrate the $\SOn{2}$-equivariant
equations \refeq{2mode4D} for 500 time units and plot two projections of the 4D
\statesp\ in \reffig{fig:Set1}(a and b). In order to compare the symmetry
reduction techniques, we plotted the corresponding flow in the invariant polynomial
basis on \reffig{fig:Set1}(c) and the symmetry reduced flow using \mslices\
on \reffig{fig:Set1}(d). While \reffig{fig:Set1}(c) is generated by simply
integrating \refeq{PKinvEqs1}, we obtained \reffig{fig:Set1}(d) by integrating
\refeq{eq:so2reduced} within the \slicePlane\ of the \template ,
\beq
	\slicep = (1,0,0,0)
\label{eq:firstmodetemplate}
\eeq
with the same initial condition $x_0$ (note that it satisfies \refeq{SliceCond}
for \refeq{eq:firstmodetemplate} and \refeq{LGTwoMode}) and

The Poincar\'e section plane in \reffig{fig:BBpsecthd} includes the origin (PC??)
and is
perpendicular to
\beq
	\hat{n}_{0,GS} = (0, -0.54030, 0.84147)
	\label{eq:nhat0GS-1}
\eeq

%projected the resulting
%flow onto the basis given by
%\beq
	%(x,y)_{i,GS} = g(\pi / 4) (x,y)_i .
%\label{eq:GSbasis}
%\eeq
%From here on, we are going to refer the basis vectors \refeq{eq:GSbasis}
%as "Gram-Schmidt basis" since the solution within the \slicePlane\ of the \template\
%\refeq{eq:firstmodetemplate} has no component in $y_{1,GS}$ direction, hence,
%the flow in \reffig{fig:Set1}(d) is not a projection from a 4D \statesp\ but
 %is a complete visualisation of the solution on the \slicePlane .

\subsection{Symbolic dynamics}

\begin{figure}%[H]
\centering
 \includegraphics[width=0.45\textwidth]{BBpsecthd}
\caption{
Symmetry reduced flow within the slice hyperplane (blue). Arrows
show the unstable directions of the equilibria. Poincar\'e section, shown as
a transparent plane, passes through the \reqv\ $(0.439966, 0, 0.386267, 0.070204)$
and captures the direction along which small perturbations at the \reqv\ expand.
Intersections of the flow with the Poincar\'e section are marked with black dots.
}
\label{fig:BBpsecthd}
\end{figure}

\begin{figure}
\centering
  \includegraphics[width=0.23\textwidth]{BBpsectonslice}
  \includegraphics[width=0.22\textwidth]{BBretmaponslice}
\caption{(a) The Poincar\'e section involving the \reqv\ and its unstable direction.
		  See \reffig{fig:BBpsecthd} for its 3D visualization.
		  (b) The Poincar\'e return map of arclengths along the Poincar\'e section
		  in (a).}
\label{fig:psectandretmap}
\end{figure}

%\begin{figure}%[H]
%\centering
 %\includegraphics[width=0.45\textwidth]{BBpsectonslice}
%\caption{The Poincar\'e section involving the \reqv\ and its unstable direction.
		  %See \reffig{fig:BBpsecthd} for its 3D visualization.}
%\label{fig:BBpsectonslice}
%\end{figure}

%\begin{figure}%[H]
%\centering
 %\includegraphics[width=0.45\textwidth]{BBretmaponslice}
%\caption{Poincar\'e return map of arclengths along the Poincar\'e section
		%shown in \reffig{fig:BBpsectonslice}.}
%\label{fig:BBretmaponslice}
%\end{figure}

\begin{figure}%[H]
  \begin{center}
  \includegraphics[width=0.45\textwidth]{BBrpo}
  \end{center}
  \caption{
	\Rpo s \cycle{1} and \cycle{01} embedded in the strange attractor
    of \reffig{fig:Set1}\,(d).
    }
  \label{fig:BBrpo1-01}
\end{figure}

\rpo s and their binary itineraries, the two shortest cycles
\cycle{1} and \cycle{01} are plotted in
\reffig{fig:BBrpo1-01}.



\section{Cycle Averages}
\label{s:DynAvers}

So far, we have explained how we find the \rpo s of the \twomode\ system in
its \reducedsp\ and how to compute their stability. However, we have not yet
said anything about what to do with these numbers. We begin this section with
an overview of the main results of the periodic orbit theory, referring the reader
to \refref{DasBuch} for a detailed introduction to the subject. Our discussion
closely follows the presentation of \refref{DasBuch} with the addition of how
the theory is modified in the presence of continuous symmetries in
\refsect{s-ContFac}. In \refsect{s-CycExp}, we present cycle expansions and
explain how to approximate the Poincar\'e section in
\reffig{fig:psectandretmap} (d), in order to obtain a better convergence of
the spectral determinants. We finish this section with the numerical results in
\refsect{s-NumResults}

\subsection{Classical trace formula}
Consider the {\evOper}, the action of which evolves a density
$\rho_0(\ssp)$ in the \statesp :\DB{2014-11-11}{There is some notational nastiness here since there 
are $\ssp'$ s floating around. We have previously used $\ssp'$ for the template and used primes in general
mark template related things}
\bea
    \rho(\zeit ,\ssp) &=& [\Lop^\zeit \rho_0 ] (\ssp) \, , \continue
    &=& \int d \ssp' \delta (\ssp - \flow{\zeit}{\ssp'})
        e^{\beta \Obser^\zeit (\ssp' )} \rho_0(\ssp') \, .
        \label{e-EvOper}
\eea
Here, $\beta$ is an auxiliary variable and $\Obser^\zeit (\ssp')$ is the
integrated value of an `additive' observable $\obser$ along the orbit
$\flow{\zeit}{\ssp'}$:
\beq
    \Obser^\zeit (\ssp' ) = \int_0^{\zeit} d \zeit'
                              \obser(\flow{\zeit'}{\ssp'}) \, .
\eeq
Notice that when $\beta = 0$, the \evOper\ \refeq{e-EvOper} simply evolves
the density of \statesp\ points to its new form after time $\zeit$. As we
shall see, attaching\DB{2014-11-3}{What does ``attaching'' mean in this context... might be lacking technical rigor here.}
the integrated observables to this operator enables us to
study values of these observables averaged over the invariant measures.

Since we required our observable to be additive along an orbit and
exponentiated its integrated value in the construction of the \evOper\
\refeq{e-EvOper}; the evolution operator itself is multiplicative:
\beq
    \Lop^{\zeit_1 + \zeit_2} = \Lop^{\zeit_2} \Lop^{\zeit_1} \, .
    \label{eq-SemiGroup}
\eeq
For the kernel of the evolution integral, which we will refer to as
$\Lop^\zeit (\ssp, \ssp')$ with explicit arguments, we can write this relation
as:
\beq
	\Lop^{\zeit_1 + \zeit_2} (\ssp,\ssp') =
    \int d\ssp'' \Lop^{\zeit_2} (\ssp, \ssp'')
                   \Lop^{\zeit_1} (\ssp'', \ssp) \, .
	\label{eq-SemiGroupKernel}
\eeq
This `semigroup property' \refeq{eq-SemiGroup} of the {\evOper} allows us to
define the {\evOper} as the formal exponential of its infinitesimal generator
\Aop :
\beq
	\Lop^t = e^{\Aop t} \, .
	\label{eq-EvOpExp}
\eeq
By definition \refeq{e-EvOper}, the eigenvalues and eigenfunctions of $\Lop^t$ (and
thus \Aop ) are functions of $\beta$. Let us define $\rho_{\beta} (x)$ as the
eigenfunction of \refeq{e-EvOper} corresponding to the leading eigenvalue (i.e., the one with the
largest real part); we can write the action of \refeq{e-EvOper} on this density
explicitly as follows:
\beq
    \left[ \Lop^t \rho_{\beta} \right] (x) = e^{t s(\beta )} \rho_{\beta} (x)
    \, .
    \label{eq-EigenvalueRel}
\eeq
Here, $s(\beta)$ is the eigenvalue of $\Aop$. As stated earlier, when
$\beta = 0$, the {\evOper} simply evolves densities; this form of the evolution
operator is known as the {\FPoper}. If we assume that the system under study is ergodic,
then an `invariant measure' $\rho_0(\ssp)$ exists with eigenvalue
$s(0) = 0$ exists. The long time spectrum of any observable is going to be dominated by
its average over such a density, hence we define the average of an observable
as its average over the invariant measure:
\beq
    \langle \obser \rangle = \int d \ssp \, \obser(\ssp) \rho_0 (\ssp) \, .
    \label{e-obserAvg}
\eeq
By evaluating the action of the {\evOper} \refeq{e-EvOper} for infinitesimal
times and after some algebra, which we skip here, one finds that the
averages of observables, as well as their higher moments, can be generated from the
derivatives of $s(\beta)$:
\beq
    \langle \obser \rangle =
        \left. \frac{d s}{d \beta} \right|_{\beta = 0} \, , \quad
    \langle (\obser - \langle \obser \rangle )^2 \rangle =
        \left. \frac{d^2 s}{d \beta^2} \right|_{\beta = 0} \,, ...
    \label{eq-moments}
\eeq
In order to obtain $s(\beta)$, we construct the resolvent of \Aop , by taking
the Laplace transform of \refeq{eq-EvOpExp}:
\beq
	\int_0^{\infty} d\zeit e^{-s\zeit} \Lop^\zeit = (s-\Aop)^{-1} \, ,
	\label{eq-ResolventA}
\eeq
the trace of which peaks at the eigenvalues of \Aop. By taking the
Laplace transform of $\Lop^\zeit$ and computing its trace
by $\tr \Lop^\zeit = \int d\ssp \Lop^\zeit (\ssp,\ssp)$, one obtains the
classical trace formula:
\beq
\sum_{\alpha=0}^{\infty} \frac{1}{s-s_{\alpha}} = \sum_p T_p
\sum_{r=1}^{\infty} \frac{e^{r(\beta \Obser_p - s T_p)}}{\oneMinJ{r}}
\ee{e-ClassicalTraceFormula}
that relates the spectrum of the {\evOper} to the spectrum of the periodic
orbits. Here,  $s$ is the auxiliary
variable of the Laplace transform and $s_{\alpha}$ are the eigenvalues of \Aop . The
outer sum on the right hand side runs over the `prime cycles' $p$ of the system,
which have periods $T_p$. $\Obser_p$ is the value of
the observable integrated along the prime cycle and $\monodromy_p$ is the transverse
monodromy matrix, the eigenvalues of which are the Floquet multipliers of $p$
excluding the marginal ones ($|\Lambda| \neq 1$). In the derivation of
\refeq{e-ClassicalTraceFormula}, one assumes that the flow has a single marginal direction,
namely the direction that is parallel to the periodic orbit at all times, and evaluates the
contribution of each \po\ to the trace integral by transforming to a local coordinate
system where one of the coordinates is
parallel to the flow while the rest is transverse. Integration along the
parallel direction is what contributes the factors of $T_p$. The transverse integral
over the delta function contributes the factor of $\oneMinJ{r}$.

\subsection{Continuous factorization}
\label{s-ContFac}

The classical trace formula \refeq{e-ClassicalTraceFormula} accounts for contributions from \po s
to long time dynamical averages. However, \rpo s of equivariant systems are almost never
periodic in the full \statesp. In order to compute the contributions of \rpo s
to the trace of the \evOper, one has to factorize
\DB{}{factor or factorize... not sure which sounds better...
are they different?}
the \evOper\
into the irreducible subspaces of the symmetry group. For discrete symmetries,
this procedure is studied in \refref{CvitaEckardt}. For the quantum systems
with continuous symmetries (Abelian and 3D rotations), the factorization of
the semiclassical Green's operator is carried out in \refref{Creagh93}.
\refRef{Cvi07} addresses the continuous factorization of the \evOper\ and its
trace; we provide a sketch of this treatment here. We start by stating, without
proof, that a square-integrable field $\psi (\ssp)$ over a vector space can be
factorized into its projections over the irreducible subspaces of a group
$\Group$:
\beq
    \psi (\ssp) = \sum_m \mathbb{P}_m \psi (\ssp) \, ,
\eeq
where the sum runs over the irreducible representations of $\Group$ and
the projection operator onto the $m$th irreducible subspace, for a continuous
group, is:
    \PC{2014-11-10: From now on, a problem - if I redefine $D(\cdots)$ as \matrixRep,
    cannot easily revert to the concise $g$ notation...}
\beq
    \mathbb{P}_m = d_m \int_\Group d \mu(\LieEl) \chi_m (\LieEl(\theta))
                            \mathbb{D}(\theta)
\,.
\ee{e-ProjectionOperator}
Here, $d_m$ is the dimension of the representation, $d \mu(g)$ is the
normalized `Haar measure', $\chi_m (\LieEl)$ is the `character' of $m$th
irreducible representation and $\mathbb{D}(\theta)$ is the operator that
transforms a scalar field defined on the \statesp\ according to $\matrixRep(\theta)$,
namely, $\mathbb{D}(\theta) \rho (\ssp) = \rho(\matrixRep(\theta)^{-1} \ssp)$.
    \DB{2014-11-10}{Added missing parenthesis back on 2014-11-3 to
    this expression. Did I put it in the right place? I think so, but please double check and then delete this comment.}
For
our specific case of a single $\SOn{2}$ symmetry,
\bea
d_m &\rightarrow& 1\, , \\
\int_G d \mu(g) &\rightarrow& \oint \frac{d \theta} {2 \pi} \, , \\
\chi_m (\LieEl(\theta)) &\rightarrow& e^{- \ii m \theta } \, .
\eea
Because the projection
operator \refeq{e-ProjectionOperator} factorizes scalar fields in the \statesp\
into their projections onto irreducible subspaces of $\Group$, it can be used to
factorize the \evOper\ since the \evOper\ both acts on and returns scalar
fields (densities) on the \statesp . Thus, the kernel
of the \evOper\ transforms under the action of $\mathbb{D}(\theta)$ as:
\bea
    \mathbb{D}(\theta) \Lop^t (\ssp', \ssp) &=&
        \Lop^t (\matrixRep(\theta)^{-1} \ssp', \ssp)\,,
    \continue
    &=& \Lop^t (\ssp', \matrixRep(\theta) \ssp) \,, \continue
    &=& \delta (\ssp' - \matrixRep(\theta) f^t (\ssp)) e^{\beta \Obser^t(\ssp)}\, ,
    \label{e-gEvOper}
\eea
where the second step follows from the equivariance of the system under
consideration. \Rpo s contribute to $\mathbb{P}_m \Lop^t = \Lop_m^t$ since when its
kernel is modified as in \refeq{e-gEvOper}, the projection involves an integral
over the group parameters that is non-zero when $\theta=\theta_{\rpprime}$, the phase shifts of the
\rpo s. By computing the trace of $\Lop_m^t$, which in addition to the integral
over \statesp , now involves another integral over the group parameters, one
obtains the $m$th irreducible subspace contribution to the classical trace as
\beq
\sum_{\alpha=0}^{\infty} \frac{1}{s-s_{m, \alpha}} = \sum_p T_{\rpprime}
\sum_{r=1}^{\infty} \frac{\chi_m (\LieEl^r(\theta_{\rpprime}))
            e^{r(\beta \Obser_{\rpprime} - s T_{\rpprime})}}{\oneMinJred{r}} .
\ee{e-ReducedTraceFormula}
The reduced trace formula \refeq{e-ReducedTraceFormula} differs from the
classical trace formula \refeq{e-ClassicalTraceFormula} by the group character
term, which is evaluated at the \rpo\ phase shifts, and the reduced monodromy
matrix $\monodromyRed$, which is the $(d-N-1)\times(d-N-1)$ reduced Jacobian
for the \rpo\ evaluated on a Poincar\'e section in the \reducedsp . The eigenvalues
of $\monodromyRed$ are those of the \rpo\ Jacobian \refeq{e-rpoJacobian}
excluding the marginal ones, i.e., the ones corresponding to time evolution and evolution
along the continuous symmetry directions.

Since we are only interested in the leading eigenvalue of the \evOper , we
only consider contributions to the
trace \refeq{e-ClassicalTraceFormula} from the projections
\refeq{e-ReducedTraceFormula} of the $0$th irreducible subspace. For the $\SOn{2}$ case at hand, these can be written
explicitly as 
\beq
\sum_{\alpha=0}^{\infty} \frac{1}{s-s_{0, \alpha}} = \sum_p T_p
\sum_{r=1}^{\infty} \frac{e^{r(\beta \Obser_p - s T_p)}}{\oneMinJred{r}} \, .
\ee{e-tracem0}
This form differs from the classical trace formula
\refeq{e-ClassicalTraceFormula} only by the use of the reduced monodromy matrix 
\DBedit{instead of the full monodromy matrix}\DB{2014-11-10}{What is the non-reduced monodromy matrix called? 
I put down full but don't know if this is the correct nomenclature} since
the $0$th irreducible representation of $\SOn{2}$ has character $1$. For this
reason, cycle expansions \rf{AACI}, which we cover next, are applicable
to \refeq{e-tracem0} after the replacement
$\monodromy \rightarrow \monodromyRed$.

\subsection{Cycle expansions}
\label{s-CycExp}

While the classical trace formula \refeq{e-ClassicalTraceFormula} and its
factorization for systems with continuous symmetry \refeq{e-ReducedTraceFormula} manifest
the essential duality between the spectrum of an observable and that of
the \po s and \rpo s, in practice, they are hard to work with since the
eigenvalues are located at the poles of \refeq{e-ClassicalTraceFormula} and
\refeq{e-ReducedTraceFormula}. The dynamical zeta function
\refeq{e-DynamicalZeta}, which we derive below, provides a perturbative expansion form that 
enables us to order terms in decreasing importance while computing
spectra for the \twomode\ system. As stated earlier, \refeq{e-tracem0}
is equivalent to the \refeq{e-ClassicalTraceFormula} via substitution
$\monodromy \rightarrow \monodromyRed$. We start by defining the
`spectral determinant':
\beq
  \det (s-\Aop) = \exp \left( - \sum_p \sum_{r=1}^{\infty}
      \frac{1}{r} \frac{e^{r(\beta \Obser_p - s T_p)}}{\oneMinJ{r}} \right)\, ,
\ee{e-SpectralDeterminant}
whose logarithmic derivative ($(d/ds) \ln \det(s - \Aop)$) gives
the classical trace formula \refeq{e-ClassicalTraceFormula}.
The spectral determinant \refeq{e-SpectralDeterminant} \DB{2014-11-10}{What's up with G65?} is easier to work
with since the spectrum of $\mathcal{A}$ is now located at the zeros of
\refeq{e-SpectralDeterminant}. The convergence of \refeq{e-SpectralDeterminant}
is, however, still not obvious. More insight is gained by approximating
$\oneMinJ{r}$ by the product of expanding Floquet multipliers and then
carrying out the sum over $r$ in \refeq{e-SpectralDeterminant}. This
approximation yields
\bea
\oneMinJ{} &=& | (1 - \ExpaEig_{e,1})(1 - \ExpaEig_{e,2})... \continue
			&&(1 - \ExpaEig_{c,1}) (1 - \ExpaEig_{c,2}) ... | \nonumber \\
			&\approx& \prod_e |\ExpaEig_e| \equiv |\ExpaEig_p|,
    \label{e-LambdapApprox}
\eea
where $|\ExpaEig_{e,i}| > 1$ and $|\ExpaEig_{c,i}| < 1$ are expanding and
contracting Floquet multipliers respectively. By making this approximation, the sum over $r$ in
\refeq{e-SpectralDeterminant} becomes the Taylor expansion of natural logarithm. Carrying out this sum, brings the
spectral determinant \refeq{e-SpectralDeterminant} to a product (over prime
cycles) known as the dynamical zeta function:
\beq
1 / \zeta = \prod_p (1 - t_p) \, \mbox{where}, \, t_p = \frac{1}{|\ExpaEig_p|}
            e^{\beta \Obser_p - s T_p} z^{n_p} .
\ee{e-DynamicalZeta}
Each `cycle weight' $t_p$ is multiplied by the `order tracking term' $z^{n_p}$,
where $n_p$ is the topological length of the $p$th prime cycle. This polynomial
ordering arises naturally in the study of discrete time systems where the
Laplace transform is replaced by $z$-transform. Here, we insert the powers of
$z$ by hand \DBedit{to keep track of the ordering and then} set its value to $1$ at the 
end of calculation. Doing so allows us to write the dynamical zeta function
\refeq{e-DynamicalZeta} in the `cycle expansion' form by grouping its
terms in powers of $z$. For complete binary symbolic dynamics, where every binary symbol
sequence is accessible, the cycle expansion reads
\bea
1 / \zeta &=& 1 - t_0 - t_1 - (t_{01} - t_0 t_1 )  \label{e-CycleExpansion} \\
		  && - [(t_{011} - t_{01}t_1) + (t_{001} - t_{01} t_0)] - ... \continue
		  &=& 1 - \sum_f t_f - \sum_n \hat{c}_n \label{e-CurvatureExpansion},
\eea
\DBedit{where we labeled each prime cycle by its binary symbol sequence. In
\refeq{e-CurvatureExpansion} we grouped the contributions to the zeta function
into two groups: `fundamental' contributions $t_f$ and `curvature' corrections $c_n$.
The curvature correction terms are denoted explicitly by parentheses in \refeq{e-CycleExpansion} and
correspond to `shadowing' combinations where combinations of
shorter cycle weights, also known as `pseudocycle' weights, are subtracted from the weights of longer
prime cycles. Since the cycle weights in \refeq{e-DynamicalZeta} already
decrease exponentially with increasing cycle period, the cycle expansion
\refeq{e-CycleExpansion} converges even faster than exponentially when the
terms corresponding to longer prime cycles are shadowed.}\DB{2014-11-10}{Tried to make this clearer. Think I didn't change the content, but somebody more erudite than me please double check this.}

For complete binary symbolic dynamics, the only fundamental contributions to
the dynamical zeta function are from the cycles with topological length $1$. All 
the longer cycles appear in the shadowing combinations. This is not the case for 
any unimodal map\DB{2014-11-11}{Are we trying to say that there is no unimodal map 
for which this is true or that this is not true for a generic, but could be true for 
some subset of unimodal maps?} since some symbol sequences might be inaccessible and so that 
terms corresponding to pseudocycles that include them fail to appear in the cycle expansion 
\refeq{e-CycleExpansion}. However, if the symbolic dynamics of a system can be obtained 
from a complete binary set by finite number of replacements, then another cycle expansion 
form can be obtained is guaranteed to converge super-exponentially. Such a
duality is be obtained when there exists a finite set of `grammar rules'
to the symbolic dynamics that makes some symbol sequences inaccessible, or
`pruned'. We argued in \refsect{s:numerics} that the Poincar\'e return map for
the \twomode\ system (\reffig{fig:psectandretmap} (d)) diverges at 
$s \approx 0.98$ and approximated it as if its tip was located at the
furthest point visited by an ergodic trajectory. Here, we ask the question: Can we
reasonably approximate the map in \reffig{fig:psectandretmap} (d) in such a way
that corresponding symbolic dynamics has a finite grammar of prunning rules?
The answer, fortunately, is yes.

As shown in \reffig{fig:psectandretmap} (d) the cycles \cycle{001}
and \cycle{011} pass quite close to the tip of the cusp. Approximating the
map as if its tip located exactly at the point where \cycle{001} cuts gives us
exactly what we are looking for: a single grammar rule, which says that the symbol
sequence `00' is inaccessible. This can be made rigorous by the help of
kneading theory, however, the simple result is easy to see from the return map
in \reffig{fig:psectandretmap} (d): Cover the parts of the return map, which
are outside the borders set by the red dashed lines, the cycle \cycle{001} and
then start any point to the left of the tip and look at images. You will always
land on a point to the right of the tip, unless you start at the lower left
corner, exactly on the cycle \cycle{001}. As we will show, this `finite grammar
approximation' is reasonable since the orbits that visit outside
the borders set by \cycle{001} are very unstable, and hence, less
important for the description of invariant dynamics.

The binary grammar with only rule that forbids repeats of one of the symbols is
known as the `golden mean' shift, named after its topological entropy which is
$\ln (1 + \sqrt{5})/2$. Binary itineraries of golden mean cycles can be easiliy
obtained from the complete binary symbolic dynamics by substitution
$0 \rightarrow 01$ in  the latter. Thus, we can write the dynamical zeta
function for the golden mean pruned symbolic dynamics by replacing $0$s in
\refeq{e-CycleExpansion} by $01$:
\bea
1 / \zeta &=& 1 - t_{01} - t_1 - (t_{011} - t_{01} t_1 )
              \label{e-GoldenMeanCycleExpansion}\\
		  && - [(t_{0111} - t_{011}t_1) + (t_{01011} - t_{01} t_{011} ) ] - ...
          \nonumber
\eea
Note that all the contributions longer than topological length $2$ to the
golden mean dynamical zeta function are in form of shadowing combinations. In \refsect{s-NumResults},
we will compare the convergence of the cycle averages with and without the
finite grammar approximation, but before moving on to numerical results, 
we explain the remaining details of computation.

While dynamical zeta functions are useful for investigating the convergence
properties, they are not exact, and their computational cost is same as that of 
exact spectral determinants. For this reason, we expand the
spectral determinant \refeq{e-SpectralDeterminant} ordered in the topological
length of cycles and pseudocycles. We start with the following form of the
spectral determinant \refeq{e-SpectralDeterminant}:
\beq
    \det (s-\Aop) =   \prod_p \exp \left( - \sum_{r=1}^{n_p r < N}
                             \frac{1}{r} \frac{e^{r(\beta \Obser_p - s T_p)}
                                          }{\oneMinJ{r}} z^{n_p r} \right) \, ,
\ee{e-SpectralDeterminantExp}
where the sum over the prime cycles in the exponential becomes a
product. We also inserted the order tracking term $z$ and truncated the sum over cycle
repeats at the expansion order $N$. For each prime cycle, we compute the sum in
\refeq{e-SpectralDeterminantExp} and expand the exponential up to order
$N$. We then multiply this expansion with the contributions from previous cycles
and drop terms with order greater than $N$. This way, after setting $z=1$,
we obtain the $N^{th}$ order spectral determinant, which we denote as
\beq
    F_N(\beta , s) = 1 - \sum_{n=1}^{N} Q_n(s, \beta ) \, .
    \label{e-NthOrderSpectDet}
\eeq
Remember that we are searching for the eigenvalues $s ( \beta)$ of the \Aop ,
more specifically, we would like to compute the moments \refeq{eq-moments}.
$s ( \beta)$ are located at the zeros of the spectral determinant, hence they
satisfy the implicit equation:
\beq
    F_N(\beta, s(\beta )) = 0 \, .
    \label{e-FNimplicit}
\eeq
By taking derivative of \refeq{e-FNimplicit} with respect to $\beta$ and
applying chain rule we obtain
\beq
    \frac{d s}{d \beta} = - \left. \frac{\partial F}{\partial \beta} \right/
                                     \frac{\partial F}{\partial s}\, .
\eeq
Higher order derivatives yield higher can also be obtained similarly, and
finally, we define \DB{2014-11-11}{What is $T$ here... at this point in the paper $T$ has been used to represent both period of cycles and Lie group generator. Okay... reading further it's clearly a period, but we may want to fix this notational ambiguity}
\beq
	\langle T \rangle_N = \left. \partial F_N / \partial s
                          \right|_{\beta=0, s=s (0)} \, ,
	\label{eq-Tavg}
\eeq
and write the \cycForm s as
\bea
    \langle \obser \rangle_N &=& - \frac{1}{\langle T \rangle_N} \left.
                              \frac{\partial F_N}{\partial \beta}
                              \right|_{\beta=0, s=s (0)} \, , \label{e-Avga} \\
    \langle (\obser - \langle \obser \rangle )^2 \rangle_N
    &=& - \frac{1}{\langle T \rangle_N} \left. \frac{\partial^2 F_N}{
                        \partial \beta^2} \right|_{\beta=0, s=s (0)} \,
                        \label{e-Avgsigma} .
\eea
As mentioned earlier, we expect that for an invariant measure  $\rho_0(\ssp)$,
the eigenvalue $s(0)$ is $0$. However, we did not make this substitution in \cycForm s since, in practice,
our approximation to the spectral determinant is always of a finite precision, so that 
the solution of $F_N(0, s(0)) = 0$ is small, but not exactly $0$. This
eigenvalue has a special meaning: It indicates how well the \po s cover the
strange attractor. Following this interpretation, we define $\gamma = - s(0)$
as the `escape rate': the rate at which the dynamics escape the region that is
covered by the \po s. Specifically for our finite grammar approximation; the
escape rate tells us how frequently the flow visits the part of the
Poincar\'e map that we cut off by applying our finite grammar approximation.

We defined $\langle T \rangle$ in \refeq{eq-Tavg} as a shorthand for a partial
derivative, however, we can also develop and interpretation for it by looking
at the definitions of the dynamical zeta function \refeq{e-DynamicalZeta} and the
spectral determinant \refeq{e-SpectralDeterminant}. In both series, the partial
derivative with respect to $s$ turns them into weighted sum of the cycle
periods; with this intuition, we define $\langle T \rangle$ as the `mean cycle
period'.

These remarks conclude our review of the periodic theory and its
extension to the equivariant dynamical systems. We are now ready to present
our numerical results and discuss their quality.

\subsection{Numerical results}
\label{s-NumResults}

We constructed the spectral determinant \refeq{e-NthOrderSpectDet} to different
orders for two observables: phase velocity $\dot{\theta}$ and the leading
Lyapunov exponent. Remember that $\Obser_p$ appearing in
\refeq{e-SpectralDeterminantExp} is the integrated observable, so in order to
obtain the moments of phase velocity and the leading Lyapunov exponent from
\refeq{e-Avga} and \refeq{e-Avgsigma}, we respectively input
$\Obser_p = \theta_p$ phase shift of the prime cycle, and
$\Obser_p = \ln |\Lambda_{p,e}|$ logarithm the expanding Floquet multiplier of
the prime cycle.

In \refsect{s:visual}, we explained that \SOn{2} phase shifts correspond to
the drifts in the configuration space. With this in mind, we can relate the
variance of phase velocity to the diffusion constant for these drifts. We define the
diffusion constant as:
\beq
    D = \frac{1}{2 d} \sigma_{\dot{\theta}}^2
      = \frac{\langle (\dot{\theta} - \langle \dot{\theta} \rangle)^2
              \rangle}{2} ,
\eeq
where $d=1$ since our configuration space is one dimensional.

\refTab{t-DynamicalAverages} and \reftab{t-DynamicalAveragesNoGrammar} shows
the cycle averages of the escape rate $\gamma$, mean period
$\langle T \rangle$, leading Lyapunov exponent $\Lyap$, mean phase velocity
$\langle \dot{\theta} \rangle$ and the diffusion constant $D$ respectively
with and without the finite grammar approximation. In the latter, we input
all the \rpo s we have found into the expansion
\refeq{e-SpectralDeterminantExp}, whereas in the former, we discarded the
cycles with symbol sequence `00'.

\begin{table}
    \begin{tabular}{c|c|c|c|c|c}
     $N$ & $\gamma$ & $\langle T \rangle$ & $\lambda$ & $\langle \dot{\phi} \rangle$ & $D$ \\ 
    \hline
    1 & 0.249829963 & 3.6415122 & 0.10834917 & 0.0222352 & -0.000000 \\ 
    2 & -0.011597609 & 5.8967605 & 0.10302891 & -0.1391709 & 0.143470 \\ 
    3 & 0.027446312 & 4.7271381 & 0.11849761 & -0.1414933 & 0.168658 \\ 
    4 & -0.004455525 & 6.2386572 & 0.10631066 & -0.2141194 & 0.152201 \\ 
    5 & 0.000681027 & 5.8967424 & 0.11842700 & -0.2120545 & 0.164757 \\ 
    6 & 0.000684898 & 5.8968762 & 0.11820050 & -0.1986756 & 0.157124 \\ 
    7 & 0.000630426 & 5.9031596 & 0.11835159 & -0.1997353 & 0.157345 \\ 
    8 & 0.000714870 & 5.8918832 & 0.11827581 & -0.1982025 & 0.156001 \\ 
    9 & 0.000728657 & 5.8897511 & 0.11826873 & -0.1982254 & 0.156091 \\ 
    10 & 0.000728070 & 5.8898549 & 0.11826788 & -0.1982568 & 0.156217 \\ 
    11 & 0.000727891 & 5.8898903 & 0.11826778 & -0.1982561 & 0.156218 \\ 
    12 & 0.000727889 & 5.8898908 & 0.11826780 & -0.1982563 & 0.156220 \\ 
    \end{tabular}
    \caption{Cycle expansion estimates based on the golden mean approximation
             \refeq{e-GoldenMeanCycleExpansion} to symbolic dynamics for
             the escape rate $\gamma$, average cycle period $\langle T \rangle$,
             Lyapunov exponent $\lambda$, average phase velocity
             $\langle \dot{\theta} \rangle$ and the diffusion coefficient $D$,
             up to cycle length $N$.}
    \label{t-DynamicalAverages}
\end{table}

\begin{table}
	\begin{tabular}{c|c|c|c|c|c}
	 $N$ & $\gamma$ & $\langle T \rangle$ & $\lambda$ & $\langle \dot{\phi} \rangle$ & $D$ \\ 
	\hline
	1 & 0.249829963 & 3.6415122 & 0.10834917 & 0.0222352 & 0.000900 \\ 
 	2 & -0.011597609 & 5.8967605 & 0.10302891 & -0.1391709 & 0.226199 \\ 
 	3 & 0.022614694 & 4.8899587 & 0.13055574 & -0.1594782 & 0.294880 \\ 
 	4 & -0.006065601 & 6.2482261 & 0.11086469 & -0.2191881 & 0.488973 \\ 
 	5 & 0.000912644 & 5.7771642 & 0.11812034 & -0.2128347 & 0.448434 \\ 
 	6 & 0.000262099 & 5.8364534 & 0.11948918 & -0.2007615 & 0.383389 \\ 
 	7 & 0.000017707 & 5.8638210 & 0.12058951 & -0.2021046 & 0.392894 \\ 
 	8 & 0.000113284 & 5.8511045 & 0.12028459 & -0.2006143 & 0.381039 \\ 
 	9 & 0.000064082 & 5.8587350 & 0.12045664 & -0.2006756 & 0.381315 \\ 
 	10 & 0.000093124 & 5.8536181 & 0.12035185 & -0.2007018 & 0.381465 \\ 
 	11 & 0.000153085 & 5.8417694 & 0.12014700 & -0.2004520 & 0.377156 \\ 
 	12 & 0.000135887 & 5.8455331 & 0.12019940 & -0.2005299 & 0.378651 \\ 
 	\end{tabular}
	\caption{Cyle expansion estimates of the escape rate $\gamma$, average cycle period $\langle T \rangle$, Lyapunov exponent $\lambda$, average phase velocity $\langle \dot{\phi} \rangle$ and the diffusion coefficient $D$ with respect to the expansion order $N$ .}
	\label{t-DynamicalAveragesNoGrammar}
\end{table}


In \refsect{s-CycExp}, we motivated the finite grammar approximation by expecting a faster convergence
due to the nearly exact shadowing combinations of the golden mean zeta function
\refeq{e-GoldenMeanCycleExpansion}\DB{2014-11-11}{What's with $G69$ here}. This claim is clearly supported by the
data in \reftab{t-DynamicalAverages}\DB{2014-11-11}{Table numbering is all screwy} and
\reftab{t-DynamicalAveragesNoGrammar}. Take, for example, the Lyapunov exponent
which converges to $7$ digits for the $12^{th}$ order expansion when using the finite
grammar approximation
\reftab{t-DynamicalAverages}, only converges to $4$ digits at this order in
\reftab{t-DynamicalAveragesNoGrammar}. Other observables compare similarly in
terms of their convergence in both cases. Note, however, that the escape rate
in \reftab{t-DynamicalAverages} converges to $\gamma = 0.000727889$, whereas
in \reftab{t-DynamicalAveragesNoGrammar} it gets smaller and smaller with an
oscillatory behavior. This is due to the fact that in the finite grammar
approximation, we threw out a finite part of attractor that corresponds to the
cusp of the return map in \reffig{fig:psectandretmap} (d), above the point that
is cut by \cycle{001}.

In order to compare with the cycle averages, we numerically estimated the
leading Lyapunov exponent of the \twomode\ system using the method of
Wolf \etal\rf{WolfSwift85}. This procedure was repeated 100 times for
different initial conditions, yielding a mean estimate of
$\Lyap_{Numerical} = 0.1198 \pm 0.0008$. While the finite grammar
estimate $\Lyap_{FG} = 0.1183$ is within $0.6\%$ range of this value,
the full cycle expansion agrees with the numerical estimate. This is not
surprising, since in the finite grammar approximation, we discard the
most unstable cycles, thus, we obtain a slightly smaller Lyapunov
exponent while obtaining a significantly better convergence.

%  conclusion.texHalcrow thesis
% $Author$ $Date$


As a turbulent flow evolves, every so often we
catch a glimpse of a familiar pattern. For any finite spatial
resolution, the flow approximately follows for a finite time
a pattern belonging to a finite alphabet of admissible fluid states,
represented here by a set of exact coherent structures.
Turbulent dynamics visualized in \statesp\ appears pieced together from
close visitations of exact coherent structures
connected by transient interludes.  This is plainly illustrated
by \reffig{f:bigbox}.  The larger cell is clearly tesselated by states
not dissimilar from those presented here.

For \KS\ \eqva, \reqva\ and
periodic solutions embody the vision of turbulence\rf{Hopf48}:
a repertoire of recurrent spatio-temporal
patterns explored by turbulent dynamics.
The new \eqva\ and \reqva\ that we present here
expand and refine this repertoire.

These orbits lend credence to our view of
turbulence as a walk through this set of patterns.
The heteroclinic connections that we present here are the initial steps in drawing
an atlas of \KS\ \statesp; close passages to \eqva\ form a coarse symbolic
dynamics (nodes of a Markov graph), and their heteroclinic connections are the directed links
connecting these nodes.

The emergence and disappearance of these heteroclinic connections can also be
diagnostic. For instance, in the Lorenz system a series of such bifurcations occur
as the Rayleigh number is increased\rf{jackson89}.
They mark changes
in the topology of \statesp.  For \KS, such bifurcations could be used
to mark the onset of turbulence.

Future work in this direction could serve to clarify such points.  It is still
not entirely clear what happens at the global bifurcations involved in the creation
and annihilation of these heteroclinic connections.  Furthermore,  the list of \eqva\ and
their heteroclinic connections we have found so far should by no means be considered to be exhaustive.
Further investigation of \KS\ for these as well as other geometries
will most likely turn up more \eqva\ and their heteroclinic connections.

Currently, a taxonomy for all of these myriad states eludes us.
To organize these states in a useful way requires
a deeper understanding of the connections between them.
From this work, we only see how the various states are related for a fixed geometry.

This connects to the outstanding issue of all studies undertaken
so far, which must be addressed in future work:
the small aspect cell periodicities imposed for computational convenience.
So far, all numerical
work has focused on spanwise-streamwise periodic cells barely large
enough to allow for sustained turbulence. Such small cells introduce dynamical
artifacts such as lack of structural stability, cell-size dependence of the
sustained turbulence states, and boundary-condition dependent coherent structures
unlike those observed in large aspect ratio experiments.


\begin{acknowledgments}
We are grateful to Evangelos Siminos for his contributions to this project
and Mohammad M.~Farazmand for a critical reading of the manuscript.
We acknowledge stimulating discussion with
Xiong Ding,
Ruslan L.~Davidchack,
Ashley P.~Willis,
Al Shapere
and
Francesco Fedele.
We are indebted to the 2012 ChaosBook.org class, in particular to
Keith M.~Carroll,
Sarah Flynn,
Bryce Robbins,
and
Lei Zhang,
for the initial fearless fishing expeditions into the enormous sea of the
parameter values of the \twomode\ model.
P.~C.\ thanks the family of late G.~Robinson,~Jr.
and
NSF~DMS-1211827 for support. D.~B.\ thanks M.~F.\ Schatz for support during
the early stages of this work under NSF~CBET-0853691.
\end{acknowledgments}

\appendix
% newton.tex
%
% Predrag			jun 20 2006
% Vaggelis			may 20 2006
% $Author$ $Date$


% \section{Newton's method for determining \reqva}
% 
%  Our task is to find \reqva\ solutions of \refeq{eq:KS}.
% Although one can easilly see that this problem can be reduced to that of
%  finding periodic orbits of a 4-dimensional ODE, here we prefer to consider our system in phase space and search for solutions of
%  \beq
% 	\dot{b}_k=\dot{c}_k=0\,,
%  \eeq
%  for every $k$. The reason to do this is just getting experience before pursuing the more difficult task of locating POs and RPOs. 
%  Expanding $\dot{b}_k(a)$ and $\dot{c}_k(a)$ around our initial guess $a_o$ and demanding that they satisfy the equilibrium 
%  condition, we get
%  \bea
% 	\dot{b}_k(a) & = & \dot{b}_k(a_o)+\left.\frac{\partial \dot{b}_k}{\partial b_j}\right|_{a_o}\delta b_j + \left.\frac{\partial \dot{b}_k}{\partial c_j}\right|_{a_o}\delta c_j = 0 \continue
% 	\dot{c}_k(a) & = & \dot{c}_k(a_o)+\left.\frac{\partial \dot{c}_k}{\partial b_j}\right|_{a_o}\delta b_j + \left.\frac{\partial \dot{c}_k}{\partial c_j}\right|_{a_o}\delta c_j = 0
%  \eea
%  or in matrix form
%  \beq
%     \left( \begin{array}{cc}
%         \frac{\partial \dot{b}}{\partial b} & \frac{\partial \dot{b}}{\partial c} \\
%         \frac{\partial \dot{c}}{\partial b}	& \frac{\partial \dot{c}}{\partial c}
%      \end{array}
%      \right)_{a_o}
%      \left(\begin{array}{c}
%        \delta b  \\
%        \delta c
%      \end{array}\right)
%      =
%      \left(\begin{array}{c}
%        -\dot{b}(a_o) \\
%        -\dot{c}(a_o)
%      \end{array}\right)\,,
%      \label{eq:NewtonEquil}
% \eeq
% where $\partial{\dot{b}} / \partial{b}$ \etc are $d \times d$ submatrices. Solving this
% system of equations for the corrections $\delta b$ and  $\delta c$ and using the refined solution
% as an initial guess yields  an approximation to the solution of the system.
%  


\subsection{Implementing Newton's method  for RPOs}
\label{sec:NewtRPOs}

The relative periodic condition
\beq
	u(x+d,t+T)=u(x,t) \,
\eeq
translates in Fourier space into
\beq	
	\sum_{k=-\infty}^{+\infty} a_k (t+T) e^{ i k (x+d) / \tildeL} 
		= \sum_{k=-\infty}^{+\infty} a_k (t) e^{ i k x / \tildeL} \,
\eeq
or
\beq
	e^{ik\, d /\tildeL}a_k(t+T)=a_k(t) \,,\ \forall k \in \mathds{Z}\ \ \ \mathrm{(no\ summation)}.
	\label{eq:RPOcondition}
\eeq
We see that a relative periodic orbit returns after time $T$ to a point in 
phase space with components $a_k(t+T)$ rotated in the complex plane by an 
angle $-k\, d /\tildeL$ with respect to $a_k(t)$. In matrix notation, we write \refeq{eq:RPOcondition} as
\beq
	\mathbf{g}(d)  a(t+T)=a(t)\,,
	\label{eq:RPO}
\eeq
where we have defined
\beq
	\mathbf{g}(d) \equiv Diag[e^{ik\, d/\tildeL}]\,.
\eeq
%We notice that $R(\kappa)$ is not a rotation operator..

% Consider an initial guess $a'$ for a point on a relative periodic orbit and assume that it lies on
% a \Poincare section $\mathcal{P}$ at $t=0$. Suppose that $\mathcal{P}$ is a hyperplane in
% $\mathds{R}^{2d}$. The flow $f^t$ defined by \refeq{eq:Fcoef} transports 
% this point after time $T'$ into $a'(T')=f^{T'}(a')$. Suppose that this point is such that $R(\kappa')f^{T'}(a')$
% is a point on $\mathcal{P}$. Consider next a point $a$ lying on $\mathcal{P}$ and in the neighborhood of $a'$,
% thus satisfying
% \beq
% 	q \cdot (a'-a) = 0\,,
% 	\label{eq:cond a}
% \eeq
% with $q$ a vector normal to $\mathcal{P}$. Point $a$ will be finally identified with the improved 
% approximation of a point on the periodic orbit.
% The flow transports $a$ to $f^{T'}(a)$, but now $R(\kappa')f^{T'}(a)$ is not in general on $\mathcal{P}$.
% Moreover we would like to have the freedom to adjust the guesses for $T'$ and $\kappa'$ into new values
% $T=T'+\Delta T$ and $\kappa=\kappa'+\Delta \kappa$ to improve their accuracy. 
% Let as consider such slightly different values $T$ and $\kappa$ such that $R(\kappa)f^{T}(a)$ lies on 
% $\mathcal{P}$. Then we have the condition
% \beq
% 	q \cdot(R(\kappa')f^{T'}(a')-R(\kappa)f^{T}(a)) = 0\,.
% 	\label{eq:cond Rf(a)}
% \eeq 

Starting with an initial guess $a$ for a point on a \rpo\ we use Newton's method to find an improved approximation to the true solution $a^*$ of condition  \refeq{eq:RPO}:
\beq
	a^*=\mathbf{g}(d^*)  f^{T^*}(a^*)\,,
	\label{eq:RPOcond}
\eeq
with period $T^*$ and shift $d^*$. Let $T$ and $d$ be our guess period and shift, respectively. 
Taylor expanding $\mathbf{g}(d^*)  f^{T^*}(a^*)$ around $a$ to linear order in the small quantities 
$\delta a=a^*-a$, $\delta T=T^*-T$ and $\delta d=d^*-d$, we get
% \bea
% 	f^{T}(a)& \simeq & f^{T}(a')+\J^T(a') \Delta a \label{eq:fTaylorl1} \\ 
% 		& \simeq & f^{T'}(a') + v \Delta T + \J^{T'}(a') \Delta a \label{eq:fTaylorl2} \,, 
% \eea
% where $v$ is evaluated at $f^{T'}(a')$. Here $\J^t(x)$ is the Jacobian matrix, defined for a general flow through
% \beq
%    	J^t_{ij}(x_o)=\left.\frac{\partial x_i(t)}{\partial x_j}\right|_{x=x_0}\,.
% \eeq
% The Jacobian matrix is obtained by integrating the equation:
% \beq
%    	\dot{\mathbf{J}}^t=\mathbf{A J}^t \, ,
% 	\label{eq:Adef}
% \eeq
% subject to the initial condition:
% \beq
%    	\mathbf{J}^0=\mathbf{1} \, ,
% \eeq
% Here $\mathbf{A}$ is the matrix of variations defined as:
% \beq
% 	A_{kj}=\frac{\partial \dot{x}_k}{\partial x_j}\,.
% \eeq
% 
% In passing from \refeq{eq:fTaylorl1} to \refeq{eq:fTaylorl2} we have used the multiplicative 
% structure of the Jacobian, $\mathbf{J}^{T'+\delta T}(a')=\mathbf{J}^{\delta T}(f^{T'}(a'))\mathbf{J}^{T'}(a')$, 
% noticed that $\mathbf{J}^{\delta T}(f^{T'}(a'))=e^{\mathbf{A}\delta T}=\mathbf{1}+\mathbf{A}\delta T+\ldots$ 
% and dropped second order terms in the small quantities.
% 
% On the other hand, we have
% \bea
% 	R(\kappa'+\Delta\kappa) & = & R(\kappa')R(\Delta\kappa) \continue
% 				& \simeq & R(\kappa')(\mathbf{1}+iDiag[k]\Delta\kappa/\tildeL)\,.
% 	\label{eq:TaylorR}	
% \eea
% 
% Substituting \refeq{eq:fTaylorl2},\refeq{eq:TaylorR} into \refeq{eq:RPOcond} and keeping only first
% order terms in the small quantities, we get
% \beq
% 	a+\delta a \simeq \mathbf{g}(d)  f^{T}(a) + \mathbf{D[g]}(\mathbf{g}(d) f^{T}(a))\delta d
% 				+ \mathbf{g} (d)v(f^{T}(a)) \delta T + \mathbf{g}(d) \J^{T}(a) \delta a\,,
% \eeq
% or
\beq
	\left(\mathbf{1}-\mathbf{g}(d)\J^{T}(a)\right) \delta a - \mathbf{g}(d)v(f^{T}(a)) \delta T 
							- \mathbf{D[g]}(\mathbf{g}(d)f^{T}(a))\delta d  
					\,\simeq\, \mathbf{g}(d)f^{T}(a)-a\,,
	\label{eq:NewtonBasicCond}			
\eeq
where $D[g]_{kj}=\frac{ik}{\tildeL}\delta_{kj}$. The matrix $\mathbf{g}(d)\J^{T}(a)$ has two unit eigenvalues in 
the limit $a\rightarrow a^*$, one associated with the invariance along the direction of the flow and the other with the
translational invariance of the system. Thus \refeq{eq:NewtonBasicCond} needs to be augmented by two conditions to
eliminate the indeterminacy introduced by the (close to) zero eigenvalues of $\mathbf{1}-\mathbf{g}(d)\J^{T}(a)$. Following 
\refref{ViswanathPC06} we choose the conditions 
\bea
	v(a)\cdot\delta a & = & 0 \label{eq:NewtonAux1} \,\\
	(\mathbf{D[g]}a)\cdot \delta a & = & 0 \label{eq:NewtonAux2}\,.
\eea
The requirement imposed by \refeqs{eq:NewtonAux1}{eq:NewtonAux2}\ on the solution vector $\delta a$ of \refeq{eq:NewtonBasicCond} 
is that it vanishes along the directions of the flow and of infinitesimal translation of the initial condition.

Equations \refeq{eq:NewtonBasicCond} and \refeqs{eq:NewtonAux1}{eq:NewtonAux2}
can be compactly represented in a single matrix equation:
\beq
    \left( \begin{array}{ccc}
       \mathbf{1}-\mathbf{g}(d)\mathbf{J}^{T}(a) 	& -\mathbf{g}(d)v(f^{T}(a))	  & -\mathbf{D[g]}(\mathbf{g}(d)f^{T}(a))  \\
        v(a)^{\dagger}			& 0  	& 0 	\\
        (\mathbf{D[g]}a)^\dagger	& 0 	& 0 
     \end{array}
     \right)
     \left(\begin{array}{c}
       \delta a \\
       \delta T \\
       \delta d
     \end{array}\right)
     =
     \left(\begin{array}{c}
       \mathbf{g}(d)f^{T}(a)-a \\
       0     \\
       0
     \end{array}\right)\,.
     \label{eq:NewtonScheme}
\eeq
where $v^\dagger$ denotes the adjoint of $v$. 



\section{Periodic Schur decomposition}
\label{s:schur}

Here we briefly summarize the periodic eigendecomposition\rf{DingCvit14}
needed for evaluation of Floquet multipliers for \twomode\ \po s. Due to
the non-hyperbolicity of the return map of
\reffig{fig:psectandretmap}\,(d), Floquet multipliers can easily differ
by 100s of orders of magnitude even in a model as simple as the \twomode\
system.
    \PC{2014-07-14: cannot find anyplace in the blog numerical value of
    any of the allegedly very large unstable multipliers. $\ExpaEig
    \approx 80,000$ does not seem so large compared to the numerical
    precision? I guess it shows I did not have to compute them myself
    :)}

We obtain the Jacobian of the \rpo\ as the following multiplication of short-time
Jacobians from the multiple shooting computation of the previous section:
\bea
    \hat{\jMps} &=& \LieEl_n \jMps_n  \LieEl_{n-1} \jMps_{n-1} \, ... \, \LieEl_1 \jMps_1  \continue
                 &=& \hat{\jMps}_n \hat{\jMps}_{n-1} \, ... \, \hat{\jMps}_1 \\
                 && \mbox{where,}\, \hat{\jMps_i} = \LieEl_i \jMps_i \in
                    \mathbb{R}^{4 \times 4}, i = 1,2,...,n \nonumber
\eea
In order to determine the eigenvalues of $\hat{\jMps}$, we bring each term appearing
in the multiplication into periodic real Schur form as follows:
\beq
    \hat{\jMps}_i = Q_i R_i Q_{i-1}^T
\eeq
where, $Q_i$ is an orthogonal matrix and they satisfy the cyclic property: $Q_0 = Q_n$ .
After this similarity transformation, we can define $R = R_k R_{k-1} ... R_1$ and
re-write the Jacobian as:
\beq
    \hat{\jMps} = Q_n R Q_n^T \, .
\eeq
The matrix $R$ is block-diagonal, in general, with $1 \times 1$ blocks for real
eigenvalues and $2 \times 2$ blocks for the complex pairs; and it has the same
eigenvalues with $\hat{\jMps}$. In our case, it is diagonal since all Floquet multipliers
are real in the \twoMode\ system \rpo s. For each \rpo , we have two unit Floquet
multiplier corresponding to the time evolution direction and continuous symmetry direction,
in addition to one expanding and one contracting eigenvalue.


\bibliography{../bibtex/siminos}


%%%%%%%%% end of the paper proper %%%%%%%%%%%%%%%%%%%%%%%%

\ifdraft
    \onecolumngrid

    \newpage
\input ../blog/flotsam2modes
% \input flotsam % Predrag 2014-07-14 absorbed into flotsam2modes
    \newpage
    \section{{\twoMode} simulations blog}
    \label{chap:Mathematica}
\input ../blog/Mathematica

    \newpage
    \section{{\twoMode} daily blog}
    \label{chap:2modes}
\input ../blog/2modes
    \newpage
    \section{Burak' s {\twomode}}
    \label{chap:2modesBB}
\input ../blog/2modesBB % Predrag 2013-0810 Burak, git version only

\addcontentsline{toc}{section}{last blog entry}

\fi % end \ifdraft

\end{document}
