%%%% for public version, toggle \draftfalse in setup2modes.tex
%    (that removes all comments, the blog)

% reducesymm/cgang/2modes.tex    this is master file:    pdflatex 2modes
%     then:    pdflatex def2modes; bibtex def2modes; pdflatex def2modes; pdflatex def2modes

% until 2012-08-20 this was in svn repo siminos/cgang/2modes.tex

\documentclass[aip,cha,
reprint,
secnumarabic,
nofootinbib, tightenlines,
nobibnotes, showkeys, showpacs,
groupedaddress,
%preprint,%
%author-year,%
%author-numerical,%
]{revtex4-1}

\newcommand{\version}{atlas ver. 1.1, Nov 16 2013}
% Burak                     ver. 1.0, Oct  6 2013
% Predrag                   ver. 0.3, Aug  1 2012
% Predrag                   ver. 0.2, Apr 30 2012}
% Predrag from atlas12      ver. 0.1, Apr 25 2012}

        \input setup2modes
        \input ../inputs/def
        \input def2modes

\begin{document}

\title[Low-dimensional cartography]
{Cartography of a 4-dimensional flow with a continuous symmetry:
How to slice it}

\author{Nazmi Burak Budanur}
\email{budanur3@gatech.edu}
\author{Daniel Borrero-Echeverry}
% \author{Keith M. Carroll} %no response by 2013-08-28
\author{Predrag Cvitanovi\'{c}}
% \author{Bryce Robbins} %no response by 2012-07-26, readded 2013-08-28
\author{Evangelos Siminos}
% \author{Lei Zhang} %no response by 2012-07-26, removed
\affiliation{
 School of Physics and Center for Nonlinear Dynamics,
 Georgia Inst. of Technology,
 Atlanta, GA  30332, USA
}
    \ifdraft
\date{\today}
    \else
\date{1 September 2013}
%\affiliation{
% School of Physics and School of Mathematics,
% Georgia Inst. of Technology,
% Atlanta, GA  30332, USA
% \\\\
% Georgia Tech PHYS 7224 spring 2012 course project
% \\
% \emph{Advisers:
% Predrag Cvitanovi\'{c},
% Daniel Borrero-Echeverry
% and
% Evangelos Siminos}
%}
   \fi


    \begin{abstract}

Dangelmayr\rf{Dang86} and Porter~\&\ Knobloch\rf{PoKno05} have
introduced a family of 2-Fourier mode \SOn{2}-equivariant ODEs  in order to
study bifurcations of solutions of dynamical systems in presence of
symmetries. A 4\dmn\ system of this kind is perhaps the simplest
example of a system with a continuous symmetry that can exhibit chaos, so
we use it to illustrate the role symmetries play in chaotic dynamics. We
show that a continuous symmetry induces drifts in the full
{\statesp} dynamics, drifts which obscure the chaotic dynamics. Change of
equations of motions to a {\comovframe} frame
does not eliminate these drifts: that is only attained by a
\emph{symmetry reduction} - reformulation of dynamics in a 3-dimensional
symmetry-reduced {\statesp}, where every group orbit (set of all points
reached by actions of the group of all symmetries of the equations of
motion) is replaced by a point. {\twoMode} system is a
particularly nice illustration of how this works, as in 3 dimensions we
are able to visualize everything.

We compare three symmetry reduction methods: polar coordinates, invariant
polynomial bases, and the `{\mslices}'. While an invariant polynomial basis
can be convenient for determination of all relative equilibria of such system,
we show that {\mslices} is equally powerful for this task. Our conclusion, 
however, is that the most insight is offered by the {\mslices}.

\BBedit{I changed these because I recently computed \twoMode\ \reqva\ using the
sliced dynamics equations and I think it is even more useful than invariant
polynomials.}

%An invariant polynomial
%basis is convenient for determination of all relative equilibria of such
%system. Our conclusion, however, is that the most insight is offered by
%the {\mslices}. 
While in general a number of local slices are
needed to cover a strange attractor\rf{atlas12}, for the {\twomode}
system there we define a unique slice hyperplane that captures
\emph{all} symmetry-reduced dynamics. \BBedit{I think this is misleading 
since we have shown that single slice treatment works fine for \KS }

A Poincar\'e return map within the
slice hyperplane enables us to reduce the dynamics further, essentially
to a unimodal map, and determine, in principle, all relative periodic
orbits of the system. We can visualize each step of this process without
having to project solutions onto a submanifold since the slice hyperplane
for this system is three dimensional.

    \end{abstract}

\pacs{02.20.-a, 05.45.-a, 05.45.Jn, 47.27.ed, 47.52.+j, 83.60.Wc}
\keywords{
symmetry reduction,
equivariant dynamics,
relative equilibria,
relative periodic orbits,
slices,
moving frames
}
\maketitle

%\ifdraft\onecolumngrid % 2012-08-06 temporary \onecolumngrid


    %\begin{quotation}
%Today, it is possible to  [blah blah].
    %\end{quotation}

\section{Introduction}
\label{s:intro}

Experimental observation of the travelling waves in pipe flows \rf{science04} 
confirmed the anticipation based on the theories of nonlinear dynamics that 
the coherent structures shapes the \statesp s of turbulent flows \rf{REFERENCE}. 
Finding such solutions can lead to a better understanding of transition to 
turbulence in \NS\ and in the long term computation of dynamical averages
over the periodic orbits. In order to find the coherent solutions such as
\reqva\ and \rpo\ an essential step to take, as we shall demonstrate, is
continuous symmetry reduction.

Symmetry reduction by casting solutions on to an invariant polynomial basis 
is well studied\rf{GL-Gil07b} for low-dimensional systems such as Lorenz 
and R\"{o}ssler, however, invariants of the continuous symmetries becomes 
practically incomputable as the system dimension exceeds 12\rf{REFERENCE}. 
\Mslices  
\rf{rowley_reconstruction_2000,BeTh04,SiCvi10,FrCv11,atlas12,ACHKW11,BudCviDav14}
which we study in detail in this paper, offers a symmetry reduction scheme
for high dimensional flows, such as \NS .

While the history of the method traces back to Cartan\rf{CartanMF}, best  
to our knowledge, Rowley and Marsden\rf{rowley_reconstruction_2000} were 
first to apply \mslices\ in the context of nonlinear flows. Rowley \etal 
\rf{rowley_reduction_2003} generalized the reconstruction equations in\rf{rowley_reconstruction_2000} 
and formulated them using a ``rescaled time'' variable where they demonstrated 
the method on 1 \dmn\ \KS\ system in a periodic domain in the neighborhood 
of a travelling wave, picking the travelling wave solution as the \slice\ 
``\template ''.  Beyn and Th\"{u}mmler \rf{BeTh04} applied the method 
slices to ``freeze'' the spiral waves in reaction-diffusion systems. Siminos 
and Cvitanov\'{c} \rf{SiCvi10} used \mslices\ to reduce the \SOn{2} symmetry 
of complex Lorenz system's chaotic solutions where they showed the slice-dependent 
singularity of reconstruction equation causes the reduced flow to make discontinuous 
jumps. This singularity is studied in detail in \refref{FrCv11}, and the 
solution of using multiple ``charts'' of connected slices was proposed, 
and this approach was applied to Complex Lorenz system in \refref{atlas12} 
and to pipe flows in \refref{ACHKW11}. In a recent paper, Budanur \etal 
\rf{BudCviDav14} followed a geometrical approach and showed, for \SOn{2}, 
that for a specific choice of a \slice\ \template , one can move the \slice\ 
singularity to the edge of the symmetry-reduced sub-manifold and regulate 
the close-passages to the singularity by means of a time rescaling. Here
we follow this approach and apply the method to a 2-mode \SOn{2} ODE normal
form which perhaps is the simplest system that exhibits chaos with evolution
equations equivariant under continuous \SOn{2} transformations.

In the following section, we start by setting the terminology for the rest 
of the paper and formulate the \mslices\ as in \refref{BudCviDav14}. In 
section \ref{s:twoMode}, we introduce the \twoMode\ system and its representations
in its real valued \statesp , invariant polynomials, polar coordinates and
on a \slice . We then show that one can use either symmetry reduced representation
to compute all \reqva of the system. We then find the \rpo s and construct
symbolic dynamics by the help of Poincar\'e sections within the \slice . 
We conclude by discussing the possible applications of this method in various 
spatially extended systems.

%Over the last decade, new insights into the dynamics of  [blah blah]

%Our goals here are two-fold:
%    \PC{{\bf[2013-10-07]} Burak's and Daniels outline of Das Artikel is
%in \reffig{fig:131007outline}.
%}
%(i)  Illustrate \mslices\ in the lowest\dmn\ setting possible.
%(ii) [blah blah].

%As a motivation, consider the chaotic dynamics exhibited by the
%small-cell \KS\ system studied in \refref{lanCvit07}. Examination of
%typical long-time simulations shows that the spatio-temporal chaos arises
%from visits to two kinds of unstable patterns, a `central wobble' region
%$S_C$, and a symmetric pair of right/left `drifts' $\{S_L,S_R\}$. In
%\statesp\ projections orbits stay in one neighborhood for a while, then
%hop to another neighborhood, as illustrated in \reffig{f:antlong}. The
%strange attractor that they explore is curved and folded in such a way
%that a single local linear chart cannot cover the whole attractor,
%several charts are needed, as illustrated by \reffig{fig:2ModeAtlas}\,(d).

%The \statesp s of \KS\ and fluid-dynamical flows are high\dmn\ and
%difficult to visualize, so here we shall illustrate the key ideas by a
%much simpler example, the $\SOn{2}$-equivariant  \twoMode\ system.

%%%%%%%%%%%%%%%%%%%%%%%%%%%%%%%%%%%%%%%%%%%%%%%%%%%%%%%%%%%%%
%\begin{figure} %[tbp] %[h]
    %\centering
 %\includegraphics[width=0.40\textwidth]{kslong12}
%\caption[]{
%A long \KS\ \po\ of period $\period{}=355.34$ that connects
%neighborhoods called `$S_C$' and `$S_R$',
%(c) $[a_1,a_2]$  projection on the first two spatial Fourier modes
%(from \refref{lanCvit07}).
      %}
%\label{f:antlong}
%\end{figure}
%%%%%%%%%%%%%%%%%%%%%%%%%%%%%%%%%%%%%%%%%%%%%%%%%%%%%%%%%%%%%%%

% [blah blah]

\section{Continuous Symmetries}
\label{s:symm}

\subsection{Definitions}

A dynamical system, $\dot{\ssp}=\vel(\ssp)$, is \emph{equivariant} under a continuous
symmetry transformation
\beq
	\ssp'= \LieEl (\theta) \ssp = \exp\left( \theta \Lg\right)\ssp,
\ee{contSymTrans}
if the condition 
$\vel(\LieEl (\theta) \ssp ) = \LieEl (\theta) \vel( \ssp ) $ 
or equivalently \rf{DasBuch}
\beq
  \groupTan(\vel)  - \Mvar(\ssp) \, \groupTan(\ssp) =0
  \,,
\ee{inftmInv}
is satisfied for every point of its \statesp . In equations \refeq{contSymTrans} 
and \refeq{inftmInv}, $\LieEl (\theta)$ is the Lie group element, $\Lg$ 
is the Lie algebra generator, $\Mvar(\ssp)_{ij} = {\pde \vel_i}/{\pde\ssp_j} |_x$  
is the \stabmat , $ \groupTan(\ssp) = \Lg \ssp $ is the group tangent evaluated 
at the point $\ssp$ , and $ \groupTan(\vel) = \Lg \vel(\ssp) $ is the group 
tangent for the velocity vector evaluated at \ssp .

\label{s:relatives}

If the dynamical orbit of a point $\ssp_\stagn$ coincides with its group 
orbit, namely if one can find a group parameter $\theta (t)$, for every point $\ssp (t)$
on the orbit of $\ssp_\stagn$ such that
\beq
  \ssp (t) = \ssp_\stagn + \int_0^t d\tau \vel(\ssp (\tau)) = \LieEl (\theta (t)) \ssp_\stagn
  \, ,
\ee{releq}
 $\ssp_\stagn$ is called a \emph{ \reqv }. By expanding both sides of \refeq{releq}
for infinitesimal time, we can get the relation 
$\vel(\ssp_\stagn) = \dot{\theta}(\ssp_\stagn) \Lg \ssp_\stagn$,
and since this has to hold for every point on the group orbit of $\ssp_\stagn$
we deduce that $\dot{\theta} = c$ is a constant. We now can multiply the equivariance
condition \refeq{inftmInv} evaluated at the \reqv\ by $c$ and write the
\reqv\ condition compactly as
\beq
(\velRel \Lg - \Mvar ) \vel (\ssp_\stagn) =0
\,.
\ee{ReqvMargEig}
The constant group parameter velocity $c$  is called the \phaseVel .

We define the  \emph{ \rpo} as a \statesp\ point $\ssp_\rpprime$ which satisfies
\beq
  \ssp (T) = \ssp_\rpprime  + \int_0^T d\tau \vel(\ssp (\tau)) = \LieEl (\theta_\rpprime ) \ssp_\rpprime
  \,,
\ee{relpo}
meaning that the trajectory of $\ssp_\rpprime$ intersects its group orbit at
time $T$. While the trajectory of a \rpo\ traces out the same path shifted
by the group action over and over again, as we shall see in the examples of
\ref{s:twoModeSymRed}, such a trajectory can be very complicated if the 
continuous symmetry of the system is not reduced.

\subsection{\Mslices}
\label{s:slice}

A \emph{\slice } \pSRed\ is a codimension $N$ submanifold that is cut 
by every group orbit once and only once, where $N$ is the number of continous 
symmetries in presence. In \emph{\mslices }, one represents the solution 
of a $d$-\dmn\ dynamical system as a solution $\sspRed (t)$ within the 
$(d-N)$-\dmn\ \slice\  and $N$ time dependent group parameters $\theta(t)$ 
that maps $\sspRed (t)$ to the full \statesp\ by the group action $\LieEl(\theta(t))$. 
While this definition does not make any restriction on the shape of the \slice 
, in parctice, a linear condition is constructed by picking a particular 
\template\ ($\slicep$) and defining the \slicePlane\ as the hyperplane 
that involvesthe \template\ $\slicep$ and is perpendicular to the group 
tangent evaluated at this template $\sliceTan{} = \groupTan(\slicep) = \Lg \slicep$. 
\SlicePlane\ and reduced trajectories are illustrated in \reffig{f-ReducTraj1}.

%% ReducTraj*.* - read dasbuch/book/FigSrc/inkscape/00ReadMe.txt
\begin{figure}
\begin{center}
 \setlength{\unitlength}{0.40\textwidth}
 %% \unitlength = units used in the Picture Environment
 \begin{picture}(1,0.8361641)%
   \put(0,0){\includegraphics[width=\unitlength]{ReducTraj5.pdf}}%
   \put(0.09054399,0.38282057){\color[rgb]{0,0,0}\rotatebox{-30.34758661}{\makebox(0,0)[lb]{\smash{$\pSRed$}}}}%
   \put(0.57768586,0.29773425){\color[rgb]{0,0,0}\rotatebox{0.0313674}{\makebox(0,0)[lb]{\smash{$\sspRed(0)$}}}}%
   \put(0.59310014,0.69932675){\color[rgb]{0,0,0}\rotatebox{0.03136739}{\makebox(0,0)[lb]{\smash{$\ssp(\tau)$}}}}%
   \put(0.8268425,0.39772328){\color[rgb]{0,0,0}\rotatebox{0.03136739}{\makebox(0,0)[lb]{\smash{$\sspRed(\tau)$}}}}%
   \put(0.81220962,0.66529577){\color[rgb]{0,0,0}\rotatebox{0.03136739}{\makebox(0,0)[lb]{\smash{$\LieEl(\tau)$}}}}%
   \put(0.23150193,0.63610779){\color[rgb]{0,0,0}\rotatebox{0.0313674}{\makebox(0,0)[lb]{\smash{$\LieEl\,\slicep$}}}}%
   \put(0.37740434,0.49597258){\color[rgb]{0,0,0}\rotatebox{0.0313674}{\makebox(0,0)[lb]{\smash{$\slicep$}}}}%
   \put(0.3627714,0.69665188){\color[rgb]{0,0,0}\rotatebox{0.0313674}{\makebox(0,0)[lb]{\smash{$\sliceTan{}$}}}}%
 \end{picture}%
\end{center}
\caption{\label{f-ReducTraj1}
% (b)
\SlicePlane\ \pSRed\ is a hyperplane % \refeq{PCsect0}
passing through the {\template} point $\slicep$,
and normal to the group tangent $\sliceTan{}$ at $\slicep$.
It intersects all
group orbits (indicated by dotted lines here) in an open
neighborhood of $\slicep$.  The full
\statesp\ trajectory $\ssp(\tau)$ and the \reducedsp\
trajectory $\sspRed(\tau)$ belong to the same group orbit
$\pS_{\ssp(\tau)}$ and are equivalent up to a group rotation
$\LieEl(\gSpace)$ %, defined in   refeq{sspOrbit}
(from \wwwcb{}).
}%
\end{figure}

Reduced trajectories $\sspRed (t)$, can be obtained in two ways: post-processing
or integration within the \slice . In the former, one takes the symmetry 
equivariant data and looks for the time dependent group parameter  ($\theta (t)$) 
for which the group operation transforms the $\ssp (t)$ to 
$\sspRed (t) = \LieEl(- \theta (t)) \ssp (t)$ that satisfies
the \slice\ condition:
\beq
(\sspRed(t) - \slicep)\cdot \sliceTan{} = 0
\,.
\ee{SliceCond}
This method and is applicable to both computer and laboratory experiments.
In the second method, one computes the reduced trajectory $\sspRed (t)$ and 
the time dependent group parameter $\theta (t)$ directly by integrating
\bea
\velRed(\sspRed) &=& \vel(\sspRed)
   -\dot{\theta}(\sspRed) \, \groupTan(\sspRed)
\continue
\dot{\theta}(\sspRed) &=& {\braket{\vel(\sspRed)}{\sliceTan{}}}/
               {\braket{\groupTan(\sspRed)}{\sliceTan{}}}
\, ,
\label{eq:so2reduced}
\eea
for Abelian groups. In \refeq{eq:so2reduced}, $\velRed$ is the projection 
of the velocity function onto the \slicePlane . For a detailed derivation 
of \refeq{eq:so2reduced}, see \refref{DasBuch}. Note that the time derivative 
of the group parameter, which appears also in the reduced velocity function, 
becomes singular if the dot product $\braket{\groupTan(\sspRed)}{\sliceTan{}}$ 
vanishes. We will refer the set of points within a \slicePlane\ which satisfies
\beq
\braket{\groupTan(\sspRed^*)}{\sliceTan{}} = 0
\,
\ee{ChartBordCond}
as the \emph{\sliceBord } . 

%In general, \chartBord\ can be avoided by use of
%multiple \template s and arranging them in a way that the trajectories do not
%intersect the \chartBord , as done in \refref{atlas12}. In the particular
%case of the \twoMode\ system that we study in this paper, we can describe
%the reduced dynamics using a single \slice\ by picking a \template\ for which
%the \chartBord\ is a flow invariant subspace, hence never visited.

%If a $d$\dmn\ dynamical flow is $\Group$-equivariant under actions of
%an $N$ continuous parameters symmetry group $\Group$, its $d$\dmn\ \statesp\ is foliated
%by $N$\dmn\ group orbits, and the symmetry-\reducedsp\
%$\pS/\Group$ is $(d\!-\!N)$\dmn.
%The simplest continuous symmetry groups are the 1-parameter compact rotation
%group $\SOn{2}$ and the 1-parameter noncompact translation group
%$T(1)$; here we shall focus on the $\SOn{2}$ case.

So far, we have not made any specification on the symmetry group that we 
are reducing, other than Abelian requirement for \refeq{eq:so2reduced} . 
From here on, we restrict our discussion to 1-parameter compact $\SOn{2}$ 
rotations which arise in spatially extended systems. In order to construct
a representation for $\SOn{2}$ let us take a Fourier series expansion of 
a real valued smooth periodic function:
\beq
	u(\phi) = a_0 + \sum\limits_{k=- \infty}^\infty a_k e^{i k \phi} .
\ee{FourierSeries}
Truncating the expansion to $m$ modes we
write the real and imaginary parts of the Fourier coefficients with
$k \geq 1$ in a state vector $(x_1, y_1, x_2, y_2,..., x_m, y_m)$, where
$a_i = x_i + i y_i$, and represent the $\SOn{2}$ group action on this vector
as a block diagonal matrix:
%More explicit form, does not fit in a column:
%\beq
	 %\LieEl (\theta)= \\
					  %\begin{pmatrix}
					  %\cos \theta & \sin \theta & 0               & 0              & \cdots & 0              & 0               \\
					 %-\sin \theta & \cos \theta & 0               & 0              & \cdots & 0              & 0               \\
					  %0             & 0 		   & \cos 2 \theta & \sin 2 \theta & \cdots & 0              & 0               \\
					  %0             & 0            &-\sin 2 \theta & \cos 2 \theta & \cdots & 0              & 0               \\
					  %\vdots       & \vdots      & \vdots         & \vdots        & \ddots & \vdots         & \vdots         \\
					  %0             & 0 		   & 0               & 0              & \cdots & \cos m \theta & \sin m \theta  \\
					  %0             & 0            & 0	             & 0              & \cdots &-\sin m \theta & \cos m \theta
					  %\end{pmatrix}
%\eeq
\beq
	\LieEl(\theta) = \begin{pmatrix}
						R(\theta) & 0 			  & \cdots & 0 \\
						0		   & R(2 \theta) & \cdots & 0 \\
						\vdots	   & \vdots 	  & \ddots & \vdots \\
						0		   & 0	          & \cdots & R (m \theta)
					   \end{pmatrix} ,
\ee{mmodeLieEl}
and
\beq
	R(n \theta) =	\begin{pmatrix}
					\cos n \theta & - \sin n \theta \\
					\sin n \theta & \cos n \theta
					\end{pmatrix}
\ee{rotationmatrix}
is the rotation matrix for $n$th Fourier mode.
The Lie algebra generator is
\beq
	 \Lg =  \begin{pmatrix}
			 0 & -1 & 0 & 0 & \cdots & 0 & 0 \\
			 1 & 0 & 0 & 0 & \cdots & 0 & 0 \\
			 0 & 0 & 0 & -2 & \cdots & 0 & 0 \\
			 0 & 0 & 2 & 0 & \cdots & 0 & 0 \\
			 \vdots & \vdots & \vdots & \vdots & \ddots & \vdots & \vdots \\
			 0 & 0 & 0 & 0 & \cdots & 0 & -m \\
			 0 & 0 & 0 & 0 & \cdots & m & 0
			 \end{pmatrix} .
\ee{mmodeLg}
Now let us consider the following specific choice of a \slice\ \template\:
\beq
	\slicep = (1, 0, ..., 0) .
\ee{firstmodetemp}
Eq.~\refeq{SliceCond} restricts the points on the hyperplane defined
by the template point \refeq{slicetemp} to the following form:
\beq
	\sspRed = (\hat{x}_1, 0, \hat{x}_2, \hat{y}_2, ..., \hat{x}_m, \hat{y}_m) .
\ee{slicetemp}
These points satisfy the \sliceBord\ condition \refeq{ChartBordCond} only
if $\hat{x}_1 = 0$, in other words, as long as the first mode magnitude is
non zero, there is a corresponding unique point to every group orbit on the
\slicePlane\ defined by \refeq{firstmodetemp}. We pick the \slice\ \template\
\refeq{firstmodetemp} for computational convenience, in general, any first
mode template $\slicep = (x_1, y_1, 0,...,0)$ would do as good since the
first mode has the symmetry of a circle. We also restrict the \slicePlane\
to the half-space where $x_1 > 0$ to have a unique representative point for
each group orbit since otherwise each group orbit pierces the \slicePlane\
twice. These concepts are illustrated in \reffig{fig:BBgorbitsandslice}

\begin{figure}%[H]
\centering
 \includegraphics[width=0.45\textwidth]{BBgorbitsandslice}
\caption{ $\SOn{2}$ Group orbits of $(0.75, 0, 0.1, 0.1)$ (orange), $(0.5, 0, 0.5, 0.5)$ (green)
$(0.1, 0, 0.75, 0.75)$ (pink) of two Fourier modes and the first mode \slicePlane\
projected on three dimensions. 3D projections of the group tangents at the
intersections with the \slicePlane\ are shown as red arrows.}
\label{fig:BBgorbitsandslice}
\end{figure}

\refFig{fig:BBgorbitsandslice} shows a 3D projection of the 4D \statesp\ corresponding
to an $m=2$ truncation of \refeq{FourierSeries}, for which \refeq{mmodeLieEl} and \refeq{mmodeLg}
are $4 \times 4$ matrices. The \slicePlane\ defined by \refeq{firstmodetemp}, three
different group orbits and the group tangents evaluated at their intersections with the \slicePlane\
are visualized in \reffig{fig:BBgorbitsandslice}. One can see as the magnitude of the second mode (the vertical axis)
relative to the first mode increases, the group tangent gets closer to being
parallel to the \slicePlane , however, it still has a non zero perpendicular component. The vertical
axis ($x_2$) in \reffig{fig:BBgorbitsandslice} lies on the \sliceBord\ of the
\slicePlane .

The \template\ \refeq{firstmodetemp} for $\SOn{2}$ was introduced in \refref{BudCviDav14}
and is different than the ones that were used in the previous implementations
\rf{rowley_reconstruction_2000,BeTh04,SiCvi10,FrCv11,atlas12,ACHKW11}
of the \mslices , in the sense that it is solely geometrical and it does
not necessarily carry any information about the dynamics. More insight is 
gained by writing the reduced evolution equations \refeq{eq:so2reduced} 
explicitely for the template \refeq{firstmodetemp}:
\bea
\velRed ( \sspRed )          &=& \vel(\sspRed)
   + \frac{\vel(\sspRed)_2}{\sspRed_1} \, \groupTan(\sspRed)
\label{e-so2red1stmode} 
\,.
\eea
In order to regulate the near-divergent behavior of the velocity function 
in \refeq{e-so2red1stmode} close to $\sspRed_2 = 0$, rescaled time variable
$d \tau = dt / \sspRed_1$ was defined in \refref{BudCviDav14}. While rescaling 
time is essential for resolving the orbits during close passages to 
$\sspRed_1 = 0$, in the study of the \twoMode\ system, we will omit this
step since the vanishing 1st mode corresponds to the invariant subspace of 
the flow and hence is never visited by the dynamics.

\section{\twoMode\ $\SOn{2}$-equivariant flow}
\label{s:twoMode}

Dangelmayr,\rf{Dang86} Armbruster, Guckenheimer and Holmes,\rf{AGHO288}
Jones and Proctor,\rf{JoPro87} and Porter and Knobloch\rf{PoKno05} (see
Golubitsky \etal\rf{golubII}, Sect. XX.1) have investigated bifurcations
in 1:2 resonance ODE normal form models to third order in the amplitudes.
Our starting point is such a model
that we shall refer to as the {\twoMode} system:
\bea
	\dot{z}_1 &=& (\mu_1-\ii\, e_1)\,z_1+a_1\,z_1|z_1|^2
				 +b_1\,z_1|z_2|^2+c_1\,\overline{z}_1\,z_2
	\continue
	\dot{z}_2 &=& (\mu_2-\ii\, e_2)\,{z_2}+a_2\,z_2|z_1|^2
				 +b_2\,z_2|z_2|^2+c_2\,z_1^2 \,,
	\label{eq:DangSO2}
\eea
with $z_1,\,z_2$  complex, and all parameters real valued. The complex
\twoMode\ system \refeq{eq:DangSO2} may be rewritten as a 4-dimensional
first order ODE system,
by substitution $z_1 = x_1 + i\,y_1$, $z_2 = x_2 + i\,y_2$,
\bea
\dot{x}_1 &=& (\mu_1 + a_1 r_1^2 + b_1 r_2^2 + c_1 x_2)x_1 + c_1 y_1 y_2 + e_1 y_1
\continue
\dot{y}_1 &=& (\mu_1 + a_1 r_1^2 + b_1 r_2^2 - c_1 x_2)y_1 + c_1 x_1 y_2 - e_1 x_1
\continue
\dot{x}_2 &=& (\mu_2 + a_2 r_1^2 + b_2 r_2^2)x_2 + c_2 (x_1^2 - y_1^2) + e_2 y_2
\continue
\dot{y}_2 &=& (\mu_2 + a_2 r_1^2 + b_2 r_2^2)y_2 + 2 c_2 x_1 y_1 - e_2 x_2
\continue
		  && \mbox{where } r_1^2 = x_1^2 + y_1^2\, , \quad r_2^2 = x_2^2 + y_2^2
\,.
\label{2mode4D}
\eea
As our goal is only to
illustrate and compare continuous symmetry reduction schemes, we shall
study here several simplified versions of model \refeq{2mode4D}, in
the dimensionally lowest possible setting, with the full \statesp\ of
dimension $d=4$, and the $\SOn{2}$-reduced dynamics taking place in 3
dimensions. For these models the
parameters are far from the bifurcation values, and have no
physical interpretation.

It can be checked by inspection that equations \refeq{eq:DangSO2} are
equivariant under the \Un{1}\ transformation
\beq
(z_1,z_2) \rightarrow   (e^{i {\gSpace}}z_1,e^{i 2{\gSpace}} z_2)
\,.
\ee{Dang86(1.1)aa}
In the real representation \refeq{2mode4D}, the $\SOn{2}$ group action
\refeq{Dang86(1.1)aa} is given by $\ssp'= \exp\left( \theta \Lg\right)\ssp$,
where $\transp{\ssp} = \{ x_1, y_1,x_2, y_2\}$, and $\Lg$ is the Lie algebra
generator
\beq
\Lg  \, =
\left( \begin{array}{cccc}
         0 & -1 & 0 & 0 \\
         1 & 0 & 0 & 0 \\
         0 & 0 & 0 & -2\\
         0 & 0 & 2 & 0
      \end{array} \right)
\,.
\ee{LGTwoMode}
One can easily check that the real \twoMode\ system \refeq{2mode4D}
satisfies the equivariance condition \refeq{inftmInv}.

The parameters $\{e_1,e_2\}$ break the $\On{2}$ symmetry of the
Dangelmayr normal form system\rf{Dang86} to an $\SOn{2}$-equivariant
system. As we show in \refeq{PKinvEqs1} below, only the combination
$(2e_1-e_2)$ matters, so for simplicity we set $e_1=0$.

From \refeq{eq:DangSO2} we note that the \eqv\ point \((z_1,z_2)=(0,0)\)
is an invariant subspace, and that $z_1=0$, $z_2 \neq 0$ is a 2\dmn\
flow-invariant subspace,
\beq
  \dot{z}_1 = 0
\,,\qquad
  \dot{z}_2 = (\mu_2-\ii\, e_2 +b_2 |z_2|^2)\,{z_2}
\,,
\ee{eq:DangSO2spsp}
with a single circular \reqv\ of radius $r_2 = \norm{z_2} = \sqrt{-\mu_2/b_2}$ with
\phaseVel\ $\velRel=e_2$.
    \PC{recheck: is $\velRel=e_2$? \BBedit{I confirm.}}
At the origin $\Mvar$ commutes with $\Lg$, and thus can be block-diagonalized
into two $[2\!\times\!2]$ matrices.
% According to {\bf [2012-04-27 Daniel]},
The $[0,0,0,0]$ \eqv\ eigenvalues are $\lambda_1 = \mu_1$ with multiplicity 2 and
             $\lambda_3 = \mu_2 \pm i e_2$. The eigenvectors for
             $\lambda_1$ are $(1,0,0,0)$ and $(0,1,0,0)$ in the
             $(x_1,x_2,y_1,y_2)$ basis.
             The eigenvectors for
             $\lambda_2$ are $(0,0,1,0)$ and $(0,0,0,1)$



By contrast, for $c_2 \neq 0$, $z_2 =0$ is not a flow-invariant subspace,
as the flow exits the $z_2 =0$ plane,
    \PC{should we check if anything of interest happens for $c_2 = 0$? }
\[
  \dot{z}_1 = (\mu_1-\ii\, e_1)\,z_1+a_1\,z_1|z_1|^2
\,,\qquad
  \dot{z}_2 = c_2\,z_1^2
\,.
\]




\subsection{Invariant polynomial basis}
\label{s:invPol}

% \item[2012-04-28 Predrag]
Consider the \statesp\ of a dynamical system
constructed from two complex Fourier modes
$m=(1,2)$, with the $\SOn{2} \simeq \Un{1}$ group action given by
rotation\rf{Dang86,AGHO288,PoKno05} \refeq{Dang86(1.1)aa}. In this
case it is easy to construct a set of four real
$\SOn{2}$ invariant polynomials
\bea
u &=& {z}_1 \overline{z}_1
    \,,\quad
v = {z}_2 \overline{z}_2
    \continue
w &=& z_1^2 \overline{z}_2 + \overline{z}_1^2 {z}_2
    \,,\quad
q = (z_1^2 \overline{z}_2 - \overline{z}_1^2 {z}_2)/\ii
\,.
\label{Dang86(1.2)PK}
\eea
The polynomials $\{u,v,w,q\}$ are
linearly independent, but related through one syzygy,
%2012-04-29 Double checked, added missing factors of 2 for w and q terms
%2012-04-29 Predrag: thanks!
\beq
w^2+q^2 - 4\,u^2v =0
  \,,
\label{eq:syzPK}
\eeq
which confines the dynamics to a 3-dim\-ens\-ion\-al $\pSRed=\pS/\SOn{2}$
\reducedsp\ manifold, a symmetry-invariant repre\-sent\-ati\-on of the
4-dim\-ens\-ion\-al \SOn{2} equivariant dynamics. By construction $u \geq
0$, $v \geq 0$, but $w$ and $q$ can be of either sign. That is explicit
in in polar coordinates $ {z}_1 = |u|^{1/2} e^{\ii\theta_1}$, $ {z}_2 =
|v|^{1/2} e^{\ii\theta_2}$, where the  $w, q$ invariants take form
\bea
w &=& 2\,\Re(z_1^2 \overline{z}_2) = 2\,u |v|^{1/2} \cos \psi %Double checked DB 04-29-2012
\continue
q &=& 2\,\Im(z_1^2 \overline{z}_2) = 2\,u |v|^{1/2} \sin \psi %Double checked DB 04-29-2012
\,,
\label{Dang86(1.2)polar}
\eea
where $\psi = 2 \theta_1 - \theta_2$.

The dynamical equations for $\{u,v,w,q\}$ follow from the chain rule
\( %beq
 \dot{ u}_i= \sum_j ({\partial u_i}/{\partial \ssp_j}) \, \dot{\ssp}_j
 \,,
\) %ee{HilbChainRl}
upon substitution
$\{{z}_1\,,\overline{z}_1\,, {z}_2\,,\overline{z}_2 \}$ $\to$
$\{u,v,w,q\}$. This yields
\bea
  \dot{u} &=& \overline{z}_1 \dot{z}_1 + {z}_1 \dot{\overline{z}}_1 %Double checked DB 04-29-2012
\,,\qquad
  \dot{v} = \overline{z}_2 \dot{z}_2 + {z}_2 \dot{\overline{z}}_2 %Double checked DB 04-29-2012
\continue
  \dot{w} &=& 2 \,\overline{z}_2 {z}_1 \dot{z}_1 %Double checked DB 04-29-2012
           + 2\,{z}_2 \overline{z}_1 \dot{\overline{z}}_1
           + {z}_1^2 \dot{\overline{z}}_2
           + \overline{z}_1^2 \dot{z}_2
\continue
  \dot{q} &=&  (2\,\overline{z}_2 {z}_1 \dot{z}_1 %Double checked DB 04-29-2012
           - 2\,{z}_2 \overline{z}_1 \dot{\overline{z}}_1
           + {z}_1^2 \dot{\overline{z}}_2
           - \overline{z}_1^2 \dot{z}_2
           )/\ii
\label{PKinvEqs}
\eea
Substituting  \refeq{eq:DangSO2} into \refeq{PKinvEqs} we obtain the set
of 4 $\SOn{2}$-invariant equations,
%    \PC{2012-04-27 to Lei and all, please recheck! $e_2$ terms differ
%    from Lei. DB 04-29: Double checked using computer algebra. Found a
%    couple of discrepancies. Fixed them in red.}
\bea% Triple checked ES 04-30-2012
  \dot{u} &=& 2\,\mu_1\,u+2\,a_1\,u^2+2\,b_1\,u\,v+c_1\,w %Double checked DB 04-29-2012
\continue
  \dot{v} &=& 2\,\mu_2\,v+2\,a_2\,u\,v+2\,b_2\,v^2+c_2\,w %Double checked DB 04-29-2012
\continue
  \dot{w} &=& (2\,\mu_1+\mu_2)\,w+(2a_1+a_2)\,u\,w+(2b_1+b_2)\,v\,w %Double checked DB 04-29-2012 corrected coefficients for uv and u^2 terms
\ceq
             +\, 4c_1\,u\,v + 2c_2\,u^2 +(2e_1 - e_2)\,q
\label{PKinvEqs1}\\
  \dot{q} &=& (2\mu_1+\mu_2)\,q+(2a_1+a_2)\,u\,q
\ceq
             +(2b_1+b_2)\,v\,q
             -(2e_1-e_2)\,w %Double checked DB 04-29-2012
\,.
\nnu
\eea
Note that the $\On{2}$-symmetry breaking parameters
 $\{e_1,e_2\}$ of the
Dangelmayr normal form system\rf{Dang86} appear only in the
relative phase combination $(2e_1-e_2)$.
%[2012-07-31 Evangelos]
Using the syzygy \refeq{eq:syzPK} we can
eliminate $q$ from \refeq{PKinvEqs1} to get
    \PC{
    Note that $4u^2v-w^2 = 4u^2v(1-\cos^2\psi)$, so
    no serious singularity is introduced this way. Perhaps
    write equations of $(u,v,\cos \psi)$ as in the
    ChaosBook exercises?
    }
\bea% Triple checked ES 04-30-2012
  \dot{u} &=& 2\,\mu_1\,u+2\,a_1\,u^2+2\,b_1\,u\,v+c_1\,w \nonumber %Double checked DB 04-29-2012
\\
  \dot{v} &=& 2\,\mu_2\,v+2\,a_2\,u\,v+2\,b_2\,v^2+c_2\,w \label{PKinvEqs1syz}  %Double checked DB 04-29-2012
\\
  \dot{w} &=& (2\,\mu_1+\mu_2)\,w+(2a_1+a_2)\,u\,w+(2b_1+b_2)\,v\,w %Double checked DB 04-29-2012 corrected coefficients for uv and u^2 terms
\ceq
             +\, 4c_1\,u\,v + 2c_2\,u^2 +(2e_1 - e_2)(4u^2v-w^2)^{1/2}\,
  \nonumber
\eea

One can now either investigate the dynamics in this invariant basis or
plot the `image'\rf{GL-Gil07b} of solutions computed in the equivariant
basis \refeq{eq:DangSO2} in terms of invariant polynomials
\refeq{Dang86(1.2)PK}.

%\item[2012-04-29 Predrag]
For the 4\dmn\ model at hand we find the invariant polynomials \refeq{PKinvEqs1}
and the polar coordinates \refeq{Dang86(1.2)polar} very useful for cross-checking the
full \statesp\ $\transp{\ssp} = \{ x_1, x_2,y_1, y_2\}$ calculations.
But even
for the simplest conceivable $\SOn{2}$ 4-dimensional flow their
construction requires a bit of algebra, and we do not know
how to carry out such constructions for very high\dmn\ flows,
such as the \KS\ flow, and the Navier-Stokes flow.


\subsection{\Eqva\ of the symmetry-reduced dynamics}
\label{s:eqva}

The first step in elucidating the geometry of attracting
sets is a determination of their \eqva. For the flows
with velocity fields of multinomial form, the \eqv\
condition $\dot{\sspRed}=0$ reduces to finding roots of
multinomials. We shall now show that the symmetry-reduced
{\twoMode} system
\refeq{PKinvEqs1} has 8 \eqva, real or complex pairs.
%[2012-04-28 Predrag]
Define
\beq
A_1= \mu_1+a_1\,u+b_1\,v
    \,,\qquad
A_2 = \mu_2+a_2\,u+b_2\,v
\ee{PKinvEqs2a}
then rewrite \refeq{PKinvEqs1} as
%     \newpage
\bea
  0  &=&  2\,A_1\,u +c_1\,w
    \,,\qquad
  0  =  2\,A_2\,v +c_2\,w
\continue
  0  &=& (2\,A_1+ A_2)\,w
          +2\,\left(c_2\,u+2\,c_1\,v\right)\,u
          \ceq
		  + (2e_1-e_2)\,q
\label{PKinvEqs3}\\
  0  &=& (2\,A_1+ A_2)\,q - (2e_1-e_2)\,\,w
\nnu
\eea
We already know $[0,0,0,0]$ and $[0,-\mu_2/b_2,0,0]$ roots, so we are looking only
for the $u>0$, $v>0$, $w,q \in \reals$ solutions; there could be problems
from the non-generic roots with either $w=0$ or $q=0$, but not both
simultaneously, syzygy \refeq{eq:syzPK} precludes that. $w$ and/or $q$
can be eliminated by obtaining the following relations from \refeq{PKinvEqs3}:
\bea
	w  &=& - \frac{2\,u}{c_1}\,A_1 = - \frac{2\,v}{c_2}\,A_2
	\continue
	q &=& \frac{2(-2e_1+\,e_2)\,u\,v}{c_2\,u+2\,c_1\,v} .
	\label{PKinvEqs4}
\eea
Substituting \refeq{PKinvEqs4} into \refeq{PKinvEqs3} we get two bivariate
polynomials roots of which are the \eqva\ of the system \refeq{PKinvEqs1}:
\bea
	f(u,v) &=& c_2\,u\,A_1 - c_1\,v\,A_2 = 0 \,,\qquad  \nonumber
	\\
	g(u,v) &=&
 \left(4\,A_1^2 u^2 - 4\,c_1^2\,u^2 v\right)\left(c_2\,u+2\,c_1\,v\right)^2 \label{PKinvEqs5} %Double checked DB 04-30-2012
	\ceq
	+\,4\,c_1^2\,(-2e_1+e_2)^2\,u^2\,v^2 = 0
\,,
	\\
	deg(f) &=& 2, \, deg(g) = 6 \nonumber
\,.
\eea
%\DBedit{DB: Not sure where this factor of 2 comes from in $w =
%-\frac{2}{e_2} (2\,A_1+ A_2)\,q $. From the last equation in
%\refeq{PKinvEqs3}, I get $w = -\frac{1}{e_2} (2\,A_1+ A_2)\,q$.
%Therefore, I get I get  $q = \frac{2 e_2\,u\,v}{c_2\,u+2\,c_1\,v}$}
%2012-04-29 Predrag: thanks!

% \DBedit{DB: I get $g(u,v) = \left(w^2 - 4\,u^2
% v\right)\left(c_2\,u+2\,c_1\,v\right)^2 +\,4\,e_2^2\,u^2\,v^2 = 0$}
%2012-04-29 Predrag: thanks!
%2012-04-29 Predrag: should have I used the syzygy \refeq{eq:syzPK},
%$w^2 - 4\,u^2v = -q^2$ DB: If you plug the syzygy in you trivially get zero....

We divide the common multiplier $u^2$ from the second equation and by doing
so, eliminate one of the roots at the origin (there still is another root at
the origin), and the $[0,-\mu_2/b_2,0,0]$ root from the equations. Furthermore,
we scale the parameters and variables as
$\tilde{u} = c_2\,u$,
$\tilde{v} = c_1\,v$,
$\tilde{a_1} = a_1/c_2$,
$\tilde{b_1} = b_1/c_1$,
$\tilde{a_2} = a_2/c_2$,
$\tilde{b_2} = b_2/c_1$,
to finally get
\bea
\tilde{f}(\tilde{u},\tilde{v}) &=&
  \tilde{u}\,A_1 - \tilde{v}\,A_2 = 0 %Double checked DB 04-30-2012
\,,\qquad deg(f) = 2 \label{PKinvEqs5a}
\\
\tilde{g}(\tilde{u},\tilde{v}) &=&  %Double checked DB 04-30-2012
 \left(A_1^2
 - c_1\,\tilde{v}\right)
 \left(\tilde{u}+2\,\tilde{v}\right)^2
 +e_2^2\,\tilde{v}^2 = 0
\,,
\ceq
   deg(g) = 4 \label{PKinvEqs5b}
\\
 && \mbox{where }
A_1 = \mu_1+\tilde{a_1}\,\tilde{u}+\tilde{b_1}\,\tilde{v}
\,,\ceq
\qquad\quad A_2 = \mu_2+\tilde{a_2}\,\tilde{u}+\tilde{b_2}\,\tilde{v}
\,,
\label{PKinvEqs5c}
\eea

In order to find \reqva\ of the \twoMode\ system, one has to solve two bivariate
polynomials \refeq{PKinvEqs5a} which, in general, is not a trivial task. However,
as we shall see in the examples of the next section, for particular choices
of parameters, equations\refeq{PKinvEqs5a} symplify significantly allowing
us to determine all \reqva\ of the \twoMode\ system.

\subsection{Reduction of $\SOn{2}$ symmetry of the \twoMode\ system using
\mslices}
\label{s:twoModeSymRed}

To illustrate the \mslices\ on the \twoMode\ system we choose two relatively
simple sets of parameters for which we observe interesting dynamics. These
parameters are listed in \reftab{tab:pars}. In both sets we choose
$b_2 = 0$ and by doing so, we send the \reqv\ at $[0,-\mu_2/b_2,0,0]$ to infinity
and simplify the bivariate polynomials \refeq{PKinvEqs5a} such that from the
first equation \refeq{PKinvEqs5a} we can get the condition $\tilde{v} = (\mu_1 + \tilde{a}_1 \tilde{u})/
(\mu_2 + \tilde{a}_2 \tilde{u} - \tilde{u} \tilde{b}_1)$ and substitute into
the \refeq{PKinvEqs5b} to solve for single variable.
\begin{table}
	\begin{tabular}{c|c|c|c|c|c|c|c|c|c|c}
	% after \\: \hline or \cline{col1-col2} \cline{col3-col4} ...
	Parameters & $\mu_1$ & $\mu_2$ & $e_1$ & $e_2$ & $a_1$ & $a_2$ & $b_1$ & $b_2$ & $c_1$ & $c_2$ \\
	\hline
	(a) 	  & -2.8	& 1		  & 0	  & 1	  & -1	  & -2.66 & 0	  & 0 	  & -7.75 & 1	  \\
	\hline
	(b) 	  & 1		& -1	  & 0	  & 0	  & 0.47  & 0	  & -1	  & 0 	  & 1	  & -1	  \\	
	\end{tabular}
	\caption{Parameter sets that we used to study the \twoMode\ system.}
	\label{tab:pars}
\end{table}

We start with the first set of parameters, \reftab{tab:pars}\,(a). For this set,
after substituting the parameters with values 1 and 0 into the \refeq{eq:DangSO2},
the simplified \twoMode\ system \refeq{eq:DangSO2} has 3-parameters $\{ \mu_1, c_2, a_2 \}$:
\bea
\label{eq:DangSO2set1}
  \dot{z}_1 &=& \mu_1 \,z_1 - z_1|z_1|^2 +c_1\,\overline{z}_1\,z_2
  \continue
  \dot{z}_2 &=& (1-\ii)\,{z_2}+a_2\,z_2|z_1|^2+\,z_1^2
\,,
\eea
By solving the polynomials \refeq{PKinvEqs5} with the parameter set \reftab{tab:pars}\,(a),
we get the \eqva\ of the system in the invariant polynomial basis \refeq{Dang86(1.2)PK} as
\bea
	\label{eq:eqvaset1}
	(u,v,w,q) &=& (0,0,0,0) \qquad \mbox{(double)}
			  \continue
			  &=& (0.193569,0.154131,-0.149539,-0.027178)
			  \continue
			  &=& (0,- \infty,0,0)
			  \continue
			  &=& (-2.8,0,0,0)
			  \continue
			  &=& (-5.52172,0.12361,-3.87834,0.183536)
			  \continue
			  &=& (-0.991847 \mp 0.14571 \ii,
				   \ceq
				   -0.0640782 \pm 0.00260791 \ii,
				   \ceq
				   0.468295 \pm 0.0306953 \ii,
				   \ceq
				   -0.067488 \pm 0.687486 \ii)
\eea
Among these roots, only the origin and the second root has a correspondance
in the $\SOn{2}$-equivariant \statesp\ as \eqv\ and \reqv\ respectively.

Starting close to the relative equilibrium $x_0 = (0.439966, 0, 0.386267, 0.070204)$
corresponding to the second root in \refeq{eq:eqvaset1} we integrate the $\SOn{2}$-equivariant
equations \refeq{2mode4D} for 500 time units and plot two projections of the 4D
\statesp\ in \reffig{fig:Set1}(a and b). In order to compare the symmetry
reduction techniques, we plotted the corresponding flow in the invariant polynomial
basis on \reffig{fig:Set1}(c) and the symmetry reduced flow using \mslices\
on \reffig{fig:Set1}(d). While \reffig{fig:Set1}(c) is generated by simply
integrating \refeq{PKinvEqs1}, we obtained \reffig{fig:Set1}(d) by integrating
\refeq{eq:so2reduced} within the \slicePlane\ of the \template ,
\beq
	\slicep = (1,0,0,0)
\label{eq:firstmodetemplate}
\eeq
with the same initial condition $x_0$ (note that it satisfies \refeq{SliceCond}
for \refeq{eq:firstmodetemplate} and \refeq{LGTwoMode}) and

The Poincar\'e section plane in \reffig{fig:BBpsecthd} includes the origin (PC??)
and is
perpendicular to
\beq
	\hat{n}_{0,GS} = (0, -0.54030, 0.84147)
	\label{eq:nhat0GS-1}
\eeq

%projected the resulting
%flow onto the basis given by
%\beq
	%(x,y)_{i,GS} = g(\pi / 4) (x,y)_i .
%\label{eq:GSbasis}
%\eeq
%From here on, we are going to refer the basis vectors \refeq{eq:GSbasis}
%as "Gram-Schmidt basis" since the solution within the \slicePlane\ of the \template\
%\refeq{eq:firstmodetemplate} has no component in $y_{1,GS}$ direction, hence,
%the flow in \reffig{fig:Set1}(d) is not a projection from a 4D \statesp\ but
 %is a complete visualisation of the solution on the \slicePlane .

\begin{figure}%[H]
\centering
 (a) \includegraphics[width=0.35\textwidth]{Set1ssp1}
 \\
 (b) \includegraphics[width=0.35\textwidth]{Set1ssp2}
 \\
 (c) \includegraphics[width=0.35\textwidth]{Set1invpol}
 \\
 (d) \includegraphics[width=0.35\textwidth]{Set1sspred}
\caption{\twoMode\ flow integrated for 500 time units with the initial
condition $x_0 = (0.439966, 0, 0.386267, 0.070204)$ close to the relative
equilibrium corresponding to the second root in \refeq{eq:eqvaset1}.
Projections of 4D $\SOn{2}$ equivariant \statesp\ (a, b), symmetry invariant
flow (c) and symmetry reduced flow obtained using \mslices\ (d). In each figure,
first 100 time units are drawn red. Note that in the equivariant projections (a and b)
flow traces the group orbit and then falls into the strange attractor, whereas
in the symmetry reduced representations (c and d) time orbit spirals out from
a point as one expects.
}
\label{fig:Set1}
\end{figure}

\subsection{Symbolic dynamics}

\begin{figure}%[H]
\centering
 \includegraphics[width=0.45\textwidth]{BBpsecthd}
\caption{
Symmetry reduced flow within the slice hyperplane (blue). Arrows
show the unstable directions of the equilibria. Poincar\'e section, shown as
a transparent plane, passes through the \reqv\ $(0.439966, 0, 0.386267, 0.070204)$
and captures the direction along which small perturbations at the \reqv\ expand.
Intersections of the flow with the Poincar\'e section are marked with black dots.
}
\label{fig:BBpsecthd}
\end{figure}

\begin{figure}
\centering
  \includegraphics[width=0.23\textwidth]{BBpsectonslice}
  \includegraphics[width=0.22\textwidth]{BBretmaponslice}
\caption{(a) The Poincar\'e section involving the \reqv\ and its unstable direction.
		  See \reffig{fig:BBpsecthd} for its 3D visualization.
		  (b) The Poincar\'e return map of arclengths along the Poincar\'e section
		  in (a).}
\label{fig:psectandretmap}
\end{figure}

%\begin{figure}%[H]
%\centering
 %\includegraphics[width=0.45\textwidth]{BBpsectonslice}
%\caption{The Poincar\'e section involving the \reqv\ and its unstable direction.
		  %See \reffig{fig:BBpsecthd} for its 3D visualization.}
%\label{fig:BBpsectonslice}
%\end{figure}

%\begin{figure}%[H]
%\centering
 %\includegraphics[width=0.45\textwidth]{BBretmaponslice}
%\caption{Poincar\'e return map of arclengths along the Poincar\'e section
		%shown in \reffig{fig:BBpsectonslice}.}
%\label{fig:BBretmaponslice}
%\end{figure}

\begin{figure}%[H]
  \begin{center}
  \includegraphics[width=0.45\textwidth]{BBrpo}
  \end{center}
  \caption{
	\Rpo s \cycle{1} and \cycle{01} embedded in the strange attractor
    of \reffig{fig:Set1}\,(d).
    }
  \label{fig:BBrpo1-01}
\end{figure}

\rpo s and their binary itineraries, the two shortest cycles
\cycle{1} and \cycle{01} are plotted in
\reffig{fig:BBrpo1-01}.


%SO2-equivariant equations:
%\bea
%\dot{x}_1 &=& (\mu_1 - r_1^2 + c_1 x_2)x_1 + c_1 y_1 y_2
%\continue
%\dot{y}_1 &=& (\mu_1 - r_1^2 - c_1 x_2)y_1 + c_1 x_1 y_2
%\continue
%\dot{x}_2 &=& (1 + a_2 r_1^2)x_2 + (x_1^2 - y_1^2) + y_2
%\continue
%\dot{y}_2 &=& (1 + a_2 r_1^2)y_2 + 2 x_1 y_1 - x_2
%\continue
		  %&& \mbox{where } r_1^2 = x_1^2 + y_1^2\, , \quad r_2^2 = x_2^2 + y_2^2
%\,.
%\label{2mode4Dset1}
%\eea


%%%%%%%%%%%%%%%%%%%%%%%%%%%%%%%%%%%%%%%%%%%%%%%%%%%%%%%%%%%%%%%%%%%%%%%%%%%%%%%%%%%%%%%%%%%%%%%%%%
%\subsection{To do}
%\label{s:ToDo}

%\begin{itemize}
  %\item[10.11] Visualizations of the 4-dimensional {\twoMode} system
  %\item[10.1?] draw a group orbit for the {\twoMode} model
  %\item[10.22] {\twoMode} system in polar coordinates (maybe skip)
  %\item[10.23] The \reqva\ of the {\twoMode} system
  %\item[10.24] Plotting the \reqva\ of
           %the {\twoMode} system in invariant coordinates
  %\item[10.25] Plotting the \reqva\ of
           %the {\twoMode} system in Cartesian coordinates
           %\refeq{2mode4D}
  %\item[10.2?] construct a 2-chart atlas
           %\reffig{fig:2ModeAtlas} for a {\twoMode} system
  %\item
        %compute analytically the \stabmat\ \Mvar\ in polar coordinates
  %\item
        %Study eigenvalues, keep playing with parameters. We would like
        %-preferably- no \reqv\ to be attracting limit cycle, and several of
        %the \reqva\ to be complex-pair unstable, leading to chaos, to be
        %visualized and sliced in Cartesian coordinates.
  %\item
        %If you find a nice chaotic attractors, others can join in
        %constructing an atlas for it. We just need one and only one
        %example with non-trivial \chartBord s and at least 2 charts.
%\end{itemize}

 %[blah blah]

%\begin{itemize}
  %\item $\REQV{}{1} = (r_1,r_2,\psi)=(0.0516508, 1.26311,?)$ and
        %$\REQV{}{2} = (0.467095,0.2146,?)$
  %\item their plots in the Cartesian coordinates
  %\item $\dot{\theta}$ to see how slow/fast are they. $\dot{\theta}$
        %might be related to 4th eigenvalue, when you go back
        %to Cartesian coordinates
  %\item stability eigenvalues, eigenvectors of the \eqv\ $\EQV{0}$ at
        %origin, at your parameter values - if it is stable, everything
        %just might fall into it and die.
  %\item plots of small perturbations of the above \eqv\ and \reqva\ in
        %the Cartesian coordinates to see whether the dynamics looks
        %chaotic
  %\item $\REQV{}{1}$: 2 large positive eigenvalues looks scary - probably
        %nothing re-visits this \reqv. A mildly unstable complex pair
        %would have been sweeter. You get complex eigenvalue by Hopf-bifurcating off a
        %stable orbit, typically.
  %\item $\REQV{}{1}$: Does either unstable eigenvalue become a complex
        %eigenvalue pair in Cartesian coordinates?
  %\item $\REQV{}{2}$: contracting eigenvalues have very small imaginary
        %part, so the presumably just rocket toward the \reqv, not much
        %spiraling there. At least the unstable eigenvalue seems slow
        %compared to all other eigenvalues.
  %\item $\REQV{}{1}$: Does the unstable eigenvalue become a complex
        %eigenvalue pair in Cartesian coordinates?
%\end{itemize}

 [blah blah]



%%%%%%%%%%%%%%%%%%%%%%%%%%%%%%%%%%%%%%%%%%%%%%%%%
% 2011-09-09, 2012-03-30 Predrag: add BeThMovFr to
%            continuous.tex overheads, and ChaosBook
% replace A27movFrame*.* everywhere
%\begin{figure}
  	%\begin{center}
%(a)
%(b)
%(c)
%(d)
    %\end{center}
  %\caption{
  %\twoMode, $d=4 \to 3$~dimensional $\{x_1,x_2,z\}$ projections:
  %(a)
  %The strange attractor.
  %(b)
 %(c)
 %In contrast
 %to the 1\dmn\ \poincBord s of \reffig{fig:2modeSects}, here ...
 %(d)
  %}
%\label{fig:2ModeAtlas}
%\end{figure}
%%%%%%%%%%%%%%%%%%%%%%%%%%%%%%%%%%%%%%%%%%%%%%%%%%

 [blah blah]

 [blah blah]

%\section{Chart}
%\label{s:slice}

 %[blah blah]

%One can write the equations for the flow in the \reducedsp\
%$\dot{\sspRed} = \velRed(\sspRed)$ (for details see, for example,
%\refref{DasBuch}) as
%\bea
%\velRed(\sspRed) &=& \vel(\sspRed)
     %\,-\, \dot{\gSpace}(\sspRed) \, \groupTan(\sspRed)
%\label{2modesEqMotMFrame}\\
%\dot{\gSpace}(\sspRed) &=& \braket{\vel(\sspRed)}{\sliceTan{}}
                       %/\braket{\groupTan(\sspRed)}{\sliceTan{}}
%\,
%\label{2modesreconstrEq}
%\eea
%which confines the motion to the \slice\ hyperplane. Thus, the dynamical
%system $\{\pS,\map^t\}$ with continuous symmetry \Group\ is replaced by
%the {\reducedsp} dynamics $\{\pSRed,\mapRed^t\}$: The velocity in the
%full \statesp\ $\vel$ is the sum of $\velRed$, the velocity component in
%the \slice\ hyperplane, and $\dot{\gSpace}\,\groupTan$, the velocity
%component along the group tangent space. The integral of the {\em
%reconstruction equation} for $\dot{\gSpace}$ keeps track of the group
%shift in the full \statesp.


 %[blah blah]


\section{Conclusions}
\label{s:concl}

 [blah blah]
Whether our method can reduce
the \SOn{2}-symmetry also for $N$-Fourier modes truncations of PDEs such as
the Kuramoto-Sivashinsky, pipe flows, etc., as long as the amplitude of
the first Fourier mode is non-zero.


\begin{acknowledgments}
We acknowledge stimulating discussion with Al Shapere.
This report addresses the questions asked in
the 2012 ChaosBook.org class.
We are indebted to Keith M. Carroll, Sarah Flynn,
Bryce Robbins,
%    \PC{Is Bryce a co-author or somebody we acknowledge?
%    Nov 16 2013 Burak talked to him: dropped out}
and
Lei Zhang
for many inspiring discussions and cross-checks of the model.
% and Edgar Knobloch for [...].
\end{acknowledgments}


\bibliography{../bibtex/siminos}
%   \fi % 2012-08-06 end of temporary \onecolumngrid


\ifdraft
    \onecolumngrid

    \newpage
\input flotsam
    \newpage
    \section{{\twoMode} simulations blog}
    \label{chap:Mathematica}
\input ../blog/Mathematica

    \newpage
    \section{{\twoMode} daily blog}
    \label{chap:2modes}
\input ../blog/2modes
    \newpage
    \section{Burak' s {\twoMode}}
    \label{chap:2modesBB}
\input ../blog/2modesBB % Predrag 2013-0810 Burak, git version only

\addcontentsline{toc}{section}{last blog entry}

%\vfill
%\begin{quote}
%{\color{red} \large
%May 1 2012,  11:30am - 2:20pm term projects due, Predrag's office
%}
%\end{quote}

\fi

\end{document}
