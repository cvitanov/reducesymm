\section{\Po s}
\label{s:numerics}

The simple structure of the symmetry reduced dynamics allows us to
determine the \rpo s of the \twomode\ system by means of a Poincar\'e
section and a return map. We illustrated this procedure in
\reffig{fig:psectandretmap}. Starting with an initial point close to the
\REQV{}{}, we computed a long ergodic trajectory of the symmetry reduced
\twomode\ system by integrating \refeq{e-so2red1stmode} (blue curve in
\reffig{fig:psectandretmap}\,(a)) and recorded its intersections (marked
with red in \reffig{fig:psectandretmap}\,{a)) with the Poincar\'e section
(transparent plane in \reffig{fig:psectandretmap}\,(a)), which includes
the \REQV{}{} and the imaginary part of its unstable stability
eigenvector (one of the green arrows in \reffig{fig:psectandretmap}(a)).
We then projected these intersections onto a basis $(v_1, v_2)$, which
spans the Poincar\'e section, and fit cubic splines to this set of
points, see \reffig{fig:psectandretmap}\,(b). The return map of arclengths
from the origin which is set to \REQV{}{} in
\reffig{fig:psectandretmap}\,(b), is unimodal with an sharp cusp located at its critical point, shown
in \reffig{fig:psectandretmap}\,(c). Note that the region corresponds to
the neighborhood of the \reqv\ $s = (0, 0.6)$ is never visited once the
flow leaves it and falls onto the chaotic attractor. For this reason, we
re-drew this return map after discarding the data corresponding to the
initial transients in \reffig{fig:psectandretmap}\,(d). We use this return
map to determine the accessible \rpo s  with their respective binary
symbol sequences.

\begin{figure}
\centering
  (a) \includegraphics[width=0.45\textwidth]{BBpsecthd} \\
  (b) \includegraphics[height=0.19\textwidth]{BBpsectonslice}
  (c) \includegraphics[height=0.19\textwidth]{BBretmaponslice} \\
  (d) \includegraphics[width=0.45\textwidth]{BBretmaponsliceZoom}
\caption{(a) Symmetry reduced flow within the slice hyperplane (blue).
			Green arrows show the real and imaginary part of the unstable stability
			eigenvector $v_u$ of \REQV{}{}. A Poincar\'e section which includes
			$Im[v_u]$ is visualized as a transparent plane, and sections
			of the flow by the Poincar\'e section are marked with red.
		 (b) The Poincar\'e section which includes the \REQV{}{} and $v_u$ projected
			on to the basis within the plane shown in (a). Included is a
            transient trajectory initiated close to \REQV{}{}. Note that
		  	the vertical axis is magnified by $100$.
		 (c) The Poincar\'e arclength return map for the
		    Poincar\'e section (b).
		 (d) The return map without the transient points, framed by
            orbit of the critical point.
		 	Dashed lines show the 3-cycles \cycle{001} (red) and \cycle{011} (cyan).}
%\ES{on my screen, cyan line appears to change color in vertical parts of the figure.}
%I commented out this edit because it was preventing the file from compiling.
\label{fig:psectandretmap}
\end{figure}

Unimodal return map of \reffig{fig:psectandretmap} diverges around 
$s \approx 0.98$ and its neighborhood is visited very rarely by the flow. We 
took the furthest point that is visited by the ergodic flow, $s_C=0.98102264$ 
as the critical point of this map and coded left and right hand sides of this
point as `0' and `1' to construct binary symbolic dynamics. Accessible periodic
orbits then are those with the topological coordinates lesser than that of this
critical point. We skip the technical details regarding symbolic dynamics and the
kneading theory in this tutorial since there is a rich literature on these topics 
and we do not employ any novel symbolic dynamics technique here. For a pedagogical 
introduction to the subject, we refer the reader to \refrefs{deva87, DasBuch}. 
After determining binary itineraries 
($I_0 I_1 I_2 I_3 \dots\, \mbox{where,}\, I_j = 0,\,1$) 
of the admissible cycles, we found corresponding arclengths 
(${s_0,\,s_1,\,s_2,\,s_3,\,\dots}$) on the return map, and ten computed 

these cycles on the return map, and then computed corresponding reduced \statesp\ 
coordinates for each point on every cycle. 
%
%
%
After determining the
admissible cycles, we find candidates corresponding to the admissible
symbol sequences from the return map, and feed them into a multiple
shooting Newton solver (see Appendix \ref{s:newton}) to precisely
determine the \rpo s. This way, we found the admissible cycles of the
\twomode\ system upto the topological length 12. In
\reffig{f-2modesrpofirst4} we show shortest $4$ of the \rpo s of the
\twomode\ system within the first Fourier mode \slicePlane . As seen from
\reffig{f-2modesrpofirst4}, trajectories of \cycle{001} and \cycle{011}
almost overlap in a large region of the \statesp . This behavior is also
manifested in the return map of \reffig{fig:psectandretmap}\,{d), where
we have shown cycles \cycle{001} and \cycle{011} with red and cyan
respectively. This is a general property of the \twomode\ cycles with odd
topological lengths: They come in pairs with almost equal Floquet
exponents, see \reffig{f-2modes-lambdaDist}.

\begin{figure}%[H]
\centering
 \includegraphics[width=0.45\textwidth]{2modesrpofirst4}
\caption{Shortest four \rpo s of the \twomode\ system: \cycle{1} (dark blue), \cycle{01} (green), \cycle{001} (red), \cycle{011} (cyan). Note that \rpo s \cycle{001} and \cycle{011} almost overlap everywhere except $\hat{x}_1 \approx 0$ .}
\label{f-2modesrpofirst4}
\end{figure}

\begin{figure}%[H]
\centering
 \includegraphics[width=0.45\textwidth]{2modes-lambdaDist}
\caption{Distribution of the leading Floquet exponents of \twomode\ cycles.}
\label{f-2modes-lambdaDist}
\end{figure}

