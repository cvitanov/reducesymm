\section{Numerical results}
\label{s-numerics}

\begin{figure}%[H]
\centering
 (a) \includegraphics[width=0.20\textwidth]{2modes-ssp} 
 (b) \includegraphics[width=0.20\textwidth]{2modes-invpol}
 \\
 (c) \includegraphics[width=0.20\textwidth]{2modes-sspRed}
 (d) \includegraphics[width=0.20\textwidth]{2modes-sspRed2}
\caption{(Color online) The \twomode\ system:
A typical chaotic trajectory (blue), a trajectory spiraling out
from the \reqv\ (green), 10 repeats of the shortest
$\period{} = 3.6415120$ \rpo\
(orange) plotted in (a) a 3D projection of its
four-dimensional \statesp; (b) invariant polynomials (c) first-mode
 \slicePlane , (d) second-mode \slicePlane\ with phase doubling }
\label{fig:Set1}
\end{figure}

To illustrate the \mslices\ on the \twoMode\ system we choose two relatively
simple sets of parameters for which we observe interesting dynamics. These
parameters are listed in \reftab{tab:pars}. In both sets we choose
$b_2 = 0$ and by doing so, we send the \reqv\ at $[0,-\mu_2/b_2,0,0]$ to infinity
and simplify the bivariate polynomials \refeq{PKinvEqs5a} such that from the
first equation \refeq{PKinvEqs5a} we can get the condition $\tilde{v} = (\mu_1 + \tilde{a}_1 \tilde{u})/
(\mu_2 + \tilde{a}_2 \tilde{u} - \tilde{u} \tilde{b}_1)$ and substitute into
the \refeq{PKinvEqs5b} to solve for single variable.
\begin{table}
	\begin{tabular}{c|c|c|c|c|c|c|c|c|c|c}
	% after \\: \hline or \cline{col1-col2} \cline{col3-col4} ...
	Parameters & $\mu_1$ & $\mu_2$ & $e_1$ & $e_2$ & $a_1$ & $a_2$ & $b_1$ & $b_2$ & $c_1$ & $c_2$ \\
	\hline
	(a) 	  & -2.8	& 1		  & 0	  & 1	  & -1	  & -2.66 & 0	  & 0 	  & -7.75 & 1	  \\
	\hline
	(b) 	  & 1		& -1	  & 0	  & 0	  & 0.47  & 0	  & -1	  & 0 	  & 1	  & -1	  \\	
	\end{tabular}
	\caption{Parameter sets that we used to study the \twoMode\ system.}
	\label{tab:pars}
\end{table}

We start with the first set of parameters, \reftab{tab:pars}\,(a). For this set,
after substituting the parameters with values 1 and 0 into the \refeq{eq:DangSO2},
the simplified \twoMode\ system \refeq{eq:DangSO2} has 3-parameters $\{ \mu_1, c_2, a_2 \}$:
\bea
\label{eq:DangSO2set1}
  \dot{z}_1 &=& \mu_1 \,z_1 - z_1|z_1|^2 +c_1\,\overline{z}_1\,z_2
  \continue
  \dot{z}_2 &=& (1-\ii)\,{z_2}+a_2\,z_2|z_1|^2+\,z_1^2
\,,
\eea
By solving the polynomials \refeq{PKinvEqs5} with the parameter set \reftab{tab:pars}\,(a),
we get the \eqva\ of the system in the invariant polynomial basis \refeq{Dang86(1.2)PK} as
\bea
	\label{eq:eqvaset1}
	(u,v,w,q) &=& (0,0,0,0) \qquad \mbox{(double)}
			  \continue
			  &=& (0.193569,0.154131,-0.149539,-0.027178)
			  \continue
			  &=& (0,- \infty,0,0)
			  \continue
			  &=& (-2.8,0,0,0)
			  \continue
			  &=& (-5.52172,0.12361,-3.87834,0.183536)
			  \continue
			  &=& (-0.991847 \mp 0.14571 \ii,
				   \ceq
				   -0.0640782 \pm 0.00260791 \ii,
				   \ceq
				   0.468295 \pm 0.0306953 \ii,
				   \ceq
				   -0.067488 \pm 0.687486 \ii)
\eea
Among these roots, only the origin and the second root has a correspondance
in the $\SOn{2}$-equivariant \statesp\ as \eqv\ and \reqv\ respectively.

Starting close to the relative equilibrium $x_0 = (0.439966, 0, 0.386267, 0.070204)$
corresponding to the second root in \refeq{eq:eqvaset1} we integrate the $\SOn{2}$-equivariant
equations \refeq{2mode4D} for 500 time units and plot two projections of the 4D
\statesp\ in \reffig{fig:Set1}(a and b). In order to compare the symmetry
reduction techniques, we plotted the corresponding flow in the invariant polynomial
basis on \reffig{fig:Set1}(c) and the symmetry reduced flow using \mslices\
on \reffig{fig:Set1}(d). While \reffig{fig:Set1}(c) is generated by simply
integrating \refeq{PKinvEqs1}, we obtained \reffig{fig:Set1}(d) by integrating
\refeq{eq:so2reduced} within the \slicePlane\ of the \template ,
\beq
	\slicep = (1,0,0,0)
\label{eq:firstmodetemplate}
\eeq
with the same initial condition $x_0$ (note that it satisfies \refeq{SliceCond}
for \refeq{eq:firstmodetemplate} and \refeq{LGTwoMode}) and

The Poincar\'e section plane in \reffig{fig:BBpsecthd} includes the origin (PC??)
and is
perpendicular to
\beq
	\hat{n}_{0,GS} = (0, -0.54030, 0.84147)
	\label{eq:nhat0GS-1}
\eeq

%projected the resulting
%flow onto the basis given by
%\beq
	%(x,y)_{i,GS} = g(\pi / 4) (x,y)_i .
%\label{eq:GSbasis}
%\eeq
%From here on, we are going to refer the basis vectors \refeq{eq:GSbasis}
%as "Gram-Schmidt basis" since the solution within the \slicePlane\ of the \template\
%\refeq{eq:firstmodetemplate} has no component in $y_{1,GS}$ direction, hence,
%the flow in \reffig{fig:Set1}(d) is not a projection from a 4D \statesp\ but
 %is a complete visualisation of the solution on the \slicePlane .

\subsection{Symbolic dynamics}

\begin{figure}%[H]
\centering
 \includegraphics[width=0.45\textwidth]{BBpsecthd}
\caption{
Symmetry reduced flow within the slice hyperplane (blue). Arrows
show the unstable directions of the equilibria. Poincar\'e section, shown as
a transparent plane, passes through the \reqv\ $(0.439966, 0, 0.386267, 0.070204)$
and captures the direction along which small perturbations at the \reqv\ expand.
Intersections of the flow with the Poincar\'e section are marked with black dots.
}
\label{fig:BBpsecthd}
\end{figure}

\begin{figure}
\centering
  \includegraphics[width=0.23\textwidth]{BBpsectonslice}
  \includegraphics[width=0.22\textwidth]{BBretmaponslice}
\caption{(a) The Poincar\'e section involving the \reqv\ and its unstable direction.
		  See \reffig{fig:BBpsecthd} for its 3D visualization.
		  (b) The Poincar\'e return map of arclengths along the Poincar\'e section
		  in (a).}
\label{fig:psectandretmap}
\end{figure}

%\begin{figure}%[H]
%\centering
 %\includegraphics[width=0.45\textwidth]{BBpsectonslice}
%\caption{The Poincar\'e section involving the \reqv\ and its unstable direction.
		  %See \reffig{fig:BBpsecthd} for its 3D visualization.}
%\label{fig:BBpsectonslice}
%\end{figure}

%\begin{figure}%[H]
%\centering
 %\includegraphics[width=0.45\textwidth]{BBretmaponslice}
%\caption{Poincar\'e return map of arclengths along the Poincar\'e section
		%shown in \reffig{fig:BBpsectonslice}.}
%\label{fig:BBretmaponslice}
%\end{figure}

\begin{figure}%[H]
  \begin{center}
  \includegraphics[width=0.45\textwidth]{BBrpo}
  \end{center}
  \caption{
	\Rpo s \cycle{1} and \cycle{01} embedded in the strange attractor
    of \reffig{fig:Set1}\,(d).
    }
  \label{fig:BBrpo1-01}
\end{figure}

\rpo s and their binary itineraries, the two shortest cycles
\cycle{1} and \cycle{01} are plotted in
\reffig{fig:BBrpo1-01}.


