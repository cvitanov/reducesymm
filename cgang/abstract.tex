\begin{abstract}
Periodic orbit theory provides accurate estimates for the dynamical averages, such as dissipation rates or diffusion constants, of nonlinear flows by means of trace formulas which relates the spectra of observables to the spectra of unstable periodic orbits. These formulas are valid under the assumption that the periodic orbits have a single marginal direction along the time evolution, and hyperbolic in all other directions. Dynamical systems with continuous symmetries, however, have relative periodic orbits, which trace $(N+1)$ dimensional tori, where $N$ is the number of continuous symmetries. These systems arise in the study of turbulent flows, such as Navier-Stokes in a pipe or plane Couette flow, where one usually imposes periodic boundary conditions along the stream direction and truncates the corresponding infinite Fourier series at a large number (on the order of thousands), for finite accuracy. In this paper, we avoid the numerical challenges of thousands of dimensions while maintaining the same symmetry structure in order to illustrate the extension of periodic orbit theory to such problems. We study a chaotic 1:2 spatial resonance with broken reflection symmetry, which can be thought as $2$-mode truncation of a Fourier expansion in one dimension. A crucial step in this study is the symmetry reduction, which is a coordinate transformation that separates physical, `shape changing' dynamics from the drifts along the symmetry direction. We start by reviewing continuous symmetries and symmetry reduction methods with a focus on the `\mslices ', which, to the best of our knowledge, is the only symmetry reduction method that can be applied to the infinite dimensional problems. We then introduce the \twomode\ \SOn{2}-equivariant flow, and compare its different representations, and determine its \reqva\ using invariant polynomials. We show that a Poincar\'e section within the `\slice ' can be used to further reduce this flow to a unimodal map; and hence we can find all \rpo s and their binary symbolic dynamics. We finally compute dynamical averages using \rpo s, and discuss convergence of the spectral determinants. 
\end{abstract}
