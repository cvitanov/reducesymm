\begin{abstract}
Dynamical systems with translational or rotational symmetry arise
frequently in studies of spatially extended physical systems, such as
Navier-Stokes flows on periodic domains. In these cases, it is natural to
express a state of the fluid in terms of a Fourier series truncated to a
finite number of modes. Here we study a 4-dimensional two-mode
SO(2)-equivariant model of this type, the smallest possible truncation
that retains the symmetry while remaining high-dimensional enough to
allow for chaotic dynamics. A crucial step in analysis of such a system
is symmetry reduction. We use the model to illustrate different
symmetry-reduction schemes. Its relative equilibria are conveniently
determined by rewriting the dynamics in terms of a symmetry-invariant
polynomial basis. However, for study of chaotic dynamics, the `method of
slices', applicable also to very high-dimensional problems, is
preferable. We show that a Poincar\'e section within the `slice' can be
used to further reduce this flow to what is for all practical purposes a
unimodal map. This enables us to systematically determine all relative
periodic orbits and their symbolic dynamics up to any desired period. We
then compute several dynamical averages using relative periodic orbits
and discuss the convergence of such computations.
\end{abstract}
