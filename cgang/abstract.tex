\begin{abstract}
Periodic orbit theory provides estimates for dynamical
averages, such as dissipation rates or diffusion constants, of nonlinear
flows by means of trace formulas which relate the spectra of observables
to the spectra of unstable periodic orbits. These formulas are valid
under the assumption that the periodic orbits have a single marginal
direction along the time evolution, and hyperbolic in all other
directions. Dynamical systems with continuous symmetries, however, have
relative periodic orbits, which trace $(N+1)$ dimensional tori, where $N$
is the number of continuous symmetries. These systems arise in the study
of turbulent flows, such as Navier-Stokes in a pipe or plane Couette
flow, where one often imposes periodic boundary conditions along the
stream direction and truncates the corresponding infinite Fourier series
to a finite but large number (from tens to thousands) Fourier modes. In
this paper we study a `\twomode' model of this type,
the smallest possible truncation with a 4\dmn\ \statesp,
which has the same continuous symmetry structure as the $1D$ PDE
and is just high-dimensional enough to allow for chaotic dynamics.
The
crucial step in analysis of such systems is symmetry reduction, here a
coordinate transformation that separates physical, `shape changing'
dynamics from the drifts along the symmetry direction. We start by
reviewing continuous symmetries and symmetry reduction methods with a
focus on the `\mslices ', which, to the best of our knowledge, is the
only symmetry reduction method that can be applied to the infinite
dimensional problems. We then define our \twomode\ \SOn{2}-equivariant
model, compare different symmetry-reduction schemes, and determine its
\reqva\ using invariant polynomials. We show that a Poincar\'e section
within the `\slice ' can be used to further reduce this flow to what is
for all practical purposes a
unimodal map; and hence we can find all \rpo s and their binary symbolic
dynamics up to any desired period.
We finally compute dynamical averages using \rpo s, and discuss
convergence of the spectral determinants.
\end{abstract}
