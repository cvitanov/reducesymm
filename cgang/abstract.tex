\begin{abstract}
Nonlinear flows with continuous symmetries drift towards the symmetry direction
and such drifts complicates their dynamics while having no physical importance. 
Symmetry reduction is a coordinate transformation by which the \statesp\ 
points that are `symmetric' mapped to a single point in a `reduced' \statesp ,
hence the physically important dynamics is seperated from the drifts along 
the symmetry direction. In this paper, we study a 2-mode ODE normal form, 
and illustrate how its dynamics is resolved by symmetry reduction. Our motivation 
for studying such a system is two-fold: Since it is equivariant under \SOn{2} 
transformations, this system can be thought as a toy-model for the spatially 
extended systems; in addition to this resemblence, being only four dimensional, 
$1:2$ resonance also serves as a nice illustration tool for explaining how 
the symmetry reduction works. We compare three different symmetry reduction
methods: polar coordinates, invariant polynomial bases, and the '{\mslices}'.
An invariant polynomial basis is convenient for determination  of all relative
equilibria of such system. Our conclusion, however, is that the most insight 
is offered by the {\mslices}. A Poincar\'e return map within the
slice hyperplane enables us to reduce the dynamics further, essentially
to a unimodal map, and determine, in principle, all relative periodic
orbits of the system. We can visualize each step of this process without
having to project solutions onto a submanifold since the slice hyperplane
for this system is three dimensional.
\end{abstract}
