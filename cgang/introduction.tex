\section{Introduction}
\label{s:intro}

Recent experimental observations of travelling waves in pipe flows have
confirmed dynamical systems theory predictions that the invariant
solutions of \NSe\ play an important role in shaping the \statesp\ of turbulent
flows\rf{science04}. When one formulates a turbulent fluid flow as a
dynamical system, the outcome is an infinite dimensional system usually
symmetric under various transformations such as translations,
reflections and rotations. For example, when a periodic boundary
condition along the stream direction is imposed, the pipe flow is symmetric
under streamwise translations, azimuthal rotations and reflections about
the central axis, \ie, it is equivariant under the actions of
$\SOn{2} \times \On{2}$. This
suggests the formulation of these problems in Fourier spaces; however,
due to the nonlinear nature of the problem, Fourier modes mix as the
system evolves. This mode mixing phenomenon in dynamical systems with continuous
symmetries gives rise to the high dimensional coherent solutions
such as \reqva\ and \rpo s, which respectively take the roles of \eqva\ and 
\po s of the flows without such symmetries. 

Our objective is to provide a prototype of the study that should
ultimately applied to the turbulent problems. For this purpose, we study
the \twomode\ \SOn{2} equivariant flow, probably the simplest dynamical
dynamical system with a continuous symmetry that exhibits chaos.

In \refsect{s:symm} we define the basic terms used in the tutorial, and
review briefly the relevant symmetry reduction literature.
In \refsect{s:twoMode} we introduce the \twomode\ model system, describe
several of its representations, and
utilize a symmetry-reduced polynomial representation to find the only \reqv\
of the system.
In \refsect{s:numerics} we show how the \mslices\ combined with a
Poincar\'e section within the \slice\ enables us to reduce this 4\dmn\
chaotic flow to what is for all practical purposes is a unimodal Poincar\'e
return map. This return map enables us to construct a finite grammar
symbolic dynamics for
this flow and determine {\em all} \rpo s up to a given period, which
then are deployed
in \refsect{s:DynAvers} as input to various {\cycForm s} for estimates
of dynamically interesting observables.
A reader who gets this far in the tutorial has essentially covered the
first 500 pages of \HREF{http://ChaosBook.org} {ChaosBook.org}.
Finally, in \refsect{s:concl} we discuss possible applications of the
\mslices\ to various spatially extended systems.
\refAppe{s:newton} describes the multi-shooting method deployed for
determining the cycles,
and
\refappe{s:schur} the periodic Schur decomposition used to determine
their Floquet multipliers (which can easily differ by 100s of orders of
magnitude even in a model as simple as the \twomode\ system).
