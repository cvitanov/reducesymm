\section{Conclusions and discussion}
\label{s:concl}

In this tutorial, we have studied a simple dynamical 
system that exhibits chaos and is equivariant under a continuous 
symmetry transformation. We have shown that reducing this symmetry 
simplifies the qualitative dynamics to a great extent and enables 
one to find all \rpo s of the systems via standard techniques such 
as Poincar\'e sections and return maps. In addition, we have shown 
that one can extract quantitative information from the \rpo s by 
computing cycle averages.

We motivated our study of the \twomode\ system by the resemblance of its
symmetry structure to that of spatially extended systems. The
steps outlined here are, in principle, applicable to physical systems
that are described by $N$-Fourier mode truncations of PDEs such as $1D$
\KS,\rf{SCD07} $3D$ pipe flows,\rf{WiShCv15} \etc

In \refsect{s:numerics}, we showed that the dynamics of our \twomode\ model 
can be completely described by a unimodal return map of the Poincar\'e 
section that we constructed after continuous symmetry reduction. In a 
high-dimensional system, finding such an easy symbolic dynamics, 
or any symbolic dynamics at all is a challenging problem on its 
own. In \refref{lanCvit07}, the authors found that for 
the desymmetrized (confined in the odd subspace) $1$D 
spatio-temporally chaotic \KS\ system a bimodal return map could 
be obtained after reducing the discrete symmetry of the 
problem. However, we do not know any study that has been able to 
simplify turbulent fluid flow to such an extent.

In \refsect{s:DynAvers}, we showed that symbolic dynamics and their
associated grammar rules greatly affect the convergence of \cycForm s.
In general, finding a finite symbolic description of a flow is
rarely as easy as it is in our model system.
There exist other methods of ordering cycle
expansion terms, for example, ordering pseudo-cycles by their stability and discarding terms
that are above a threshold.\rf{DM97} In this case, one expects the remaining terms to form
shadowing combinations and converge exponentially.
Whichever method of term ordering is employed, the cycle expansions are only as good
as the least unstable cycle that one fails to find. Symbolic dynamics solves both 
problems at once since it puts the cycles in order by topological length so that
one cannot miss any accessible cycle and shadowing combinations occur naturally. 
The question one might then ask is: When there is no symbolic dynamics, how can you make 
sure that you have found all the periodic orbits of a flow up to some cycle period?

In searching for cycles in high-dimensional flows, one usually looks at
the near recurrences of the ergodic flow and then runs Newton searches
starting near these recurrences to find if they are influenced by a nearby exactly
recurrent solution. Such an approach does not answer the question we just asked with full
confidence, although one might argue that the dynamically important cycles
influence the ergodic flow, leading to recurrences, and thus, cycles found this way are
those that are relevant for computing averages.

To sum up, we have shown that periodic orbit theory successfully 
extends to systems with continuous symmetries. When dealing with 
high dimensional systems, one still needs to think about some of the 
remaining challenges outlined above. Once these are overcome, it 
should become possible to extract quantitative information about 
turbulence by using exact unstable solutions of the Navier-Stokes 
equations.
