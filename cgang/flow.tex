\section{\twoMode\ $\SOn{2}$-equivariant flow}
\label{s:twoMode}

Dangelmayr,\rf{Dang86} Armbruster, Guckenheimer and Holmes,\rf{AGHO288}
Jones and Proctor,\rf{JoPro87} and Porter and Knobloch\rf{PoKno05} (see
Golubitsky \etal\rf{golubII}, Sect. XX.1) have investigated bifurcations
in 1:2 resonance ODE normal form models to third order in the amplitudes.
Here, we use this model as a starting point from which we derive what may
be one of the simplest chaotic systems with continuous symmetry. We refer to this as the {\twomode} system:
\bea
	\dot{z}_1 &=& (\mu_1-\ii\, e_1)\,z_1+a_1\,z_1|z_1|^2
				 +b_1\,z_1|z_2|^2+c_1\,\overline{z}_1\,z_2
	\continue
	\dot{z}_2 &=& (\mu_2-\ii\, e_2)\,{z_2}+a_2\,z_2|z_1|^2
				 +b_2\,z_2|z_2|^2+c_2\,z_1^2 \,,
	\label{eq:DangSO2}
\eea
where $z_1$ and $z_2$ are complex and all parameters real-valued. The parameters $\{e_1,e_2\}$ break the reflectional symmetry of the $\On{2}$-equivariant normal form studied by Dangelmayr\rf{Dang86} leading to an
$\SOn{2}$-equivariant system.\DB{1-19-2015}{Moved discussion of $e$'s to more appropriate spot later} This complex two mode
system can be expressed as a 4-dimensional system of real-valued first order ODEs by
substituting $z_1 = x_1 + i\,y_1$, $z_2 = x_2 + i\,y_2$, so that \bea
\dot{x}_1 &=& (\mu_1 + a_1 r_1^2 + b_1 r_2^2 + c_1 x_2)x_1 + c_1 y_1 y_2 + e_1 y_1 \, ,% double checked DBE 05/22/2014
\continue
\dot{y}_1 &=& (\mu_1 + a_1 r_1^2 + b_1 r_2^2 - c_1 x_2)y_1 + c_1 x_1 y_2 - e_1 x_1 \, ,% double checked DBE 05/22/2014
\continue
\dot{x}_2 &=& (\mu_2 + a_2 r_1^2 + b_2 r_2^2)x_2 + c_2 (x_1^2 - y_1^2) + e_2 y_2 \, ,% double checked DBE 05/22/2014
\label{2mode4D}
\continue
\dot{y}_2 &=& (\mu_2 + a_2 r_1^2 + b_2 r_2^2)y_2 + 2 c_2 x_1 y_1 - e_2 x_2 \, ,% double checked DBE 05/22/2014
\continue
		  && \mbox{where } r_1^2 = x_1^2 + y_1^2\, , \quad r_2^2 = x_2^2 + y_2^2
\,.
\eea

The large number of parameters $\left(\mu_1,\mu_2,a_1,a_2,b_1,b_2,c_1,c_2,e_1,e_2\right)$ in this system makes full exploration of the parameter space impractical. Following in the tradition of Lorenz\rf{lorenz},
H\'enon\rf{henon} and R\"ossler\rf{ross}, we have tried various
choices of parameters until settling on the following set of values, which we will use in all
numerical calculations presented here:
\beq
	\begin{tabular}{c c c c c c c c c c}
	% after \\: \hline or \cline{col1-col2} \cline{col3-col4} ...
	 $\mu_1$ & $\mu_2$ & $e_1$ & $e_2$ & $a_1$ & $a_2$ & $b_1$ & $b_2$ & $c_1$ & $c_2$ \\
	\hline
	 -2.8	& 1		  & 0	  & 1	  & -1	  & -2.66 & 0	  & 0 	  & -7.75 & 1
	\end{tabular}
	\label{eq:pars}
\eeq
This choice of parameters is far from the bifurcation values studied
by previous authors\rf{Dang86,AGHO288,JoPro87,PoKno05}, so that the
model has no physical interpretation. However, these parameters yield 
chaotic dynamics, making the two-mode system a convenient minimal model for the study of 
chaos in the presence of a continuous symmetry: 
It is a 4\dmn\ $\SOn{2}$-equivariant model, whose symmetry-reduced dynamics are chaotic and take place on a three-dimensional manifold. For another example of parameter values that result in chaotic dynamics, see \refref{PoKno05}.

It can be checked by inspection that eqs.~\refeq{eq:DangSO2} are
equivariant under the \Un{1}\ transformation
\beq
(z_1,z_2) \rightarrow   (e^{i {\gSpace}}z_1,e^{i 2{\gSpace}} z_2)
\,.
\ee{Dang86(1.1)aa}
In the real representation \refeq{2mode4D}, the $\SOn{2}$ group action
\refeq{Dang86(1.1)aa} on a state space point $\ssp$ is given $\exp\left( \theta \Lg\right)\ssp$,\DB{1-19-2015}{primes are reserved for template-related things... wrote out group action in words to avoid using primes in equation}
where $\transp{\ssp} =\cartpt{x_1, y_1,x_2, y_2}$ and $\Lg$ is the Lie algebra
element
\beq
\Lg  \, =
\left( \begin{array}{cccc}
         0 & -1 & 0 & 0 \\
         1 & 0 & 0 & 0 \\
         0 & 0 & 0 & -2\\
         0 & 0 & 2 & 0
      \end{array} \right)
\,.
\ee{LGTwoMode}
One can easily check that the real \twomode\ system \refeq{2mode4D}
satisfies the equivariance condition \refeq{inftmInv}.

From \refeq{eq:DangSO2}, it is obvious that the \eqv\ point \((z_1,z_2)=(0,0)\)
is an invariant subspace and that $z_1=0$, $z_2 \neq 0$ is a 2\dmn\
flow-invariant subspace
\beq
  \dot{z}_1 = 0 % Double checked DBE 05/26/2014
\,,\qquad
  \dot{z}_2 = (\mu_2-\ii\, e_2 +b_2 |z_2|^2)\,{z_2} % Double checked DBE 05/26/2014
\,
\ee{eq:DangSO2spsp}
with a single circular \reqv\ of radius $r_2 = \norm{z_2} = \sqrt{-\mu_2/b_2}$ with
\phaseVel\ $\velRel=-e_2/2$. At the origin the stability matrix $\Mvar$ commutes with $\Lg$,
and so, can be block-diagonalized into two $[2\!\times\!2]$ matrices.
% According to {\bf [2012-04-27 Daniel]},
The eigenvalues of $\Mvar$ at $\cartpt{0,0,0,0}$ are $\Lyap_1 = \mu_1$ with multiplicity 2 and
$\Lyap_2 = \mu_2 \pm i e_2$. In the $\cartpt{x_1,y_1,x_2,y_2}$ coordinates, the eigenvectors for $\Lyap_1$ are $\cartpt{1,0,0,0}$ and
$\cartpt{0,1,0,0}$ and the eigenvectors for $\Lyap_2$
are $\cartpt{0,0,1,0}$ and $\cartpt{0,0,0,1}$.

In contrast, $z_2 =0$ is not, in general, a flow-invariant subspace since the dynamics
\[
  \dot{z}_1 = (\mu_1-\ii\, e_1)\,z_1+a_1\,z_1|z_1|^2
\,,\qquad
  \dot{z}_2 = c_2\,z_1^2
\,.
\]
take the flow out of the $z_2 =0$ plane.


\subsection{Invariant polynomial bases}
\label{s:invPol}

Before continuing our tutorial on the use of the method of slices using the \fFslice, we briefly discuss the symmetry reduction of the \twomode\ system using invariant polynomials. While representations of our model in terms of invariant polynomials and polar coordinates are useful for cross-checking our calculations in the full \statesp\ $\transp{\ssp} =\cartpt{x_1, x_2,y_1, y_2}$, their construction requires a bit of algebra even for this simple 4-dimensional flow. For very high\dmn\ flows, such as \KS\ and \NS\ flows, we do not know how to carry out such constructions. As discussed in \refrefs{Dang86,AGHO288,PoKno05}, for the \twomode\ system, it is easy to construct a set of four real-valued $\SOn{2}$ invariant
polynomials
\bea
u &=& {z}_1 \overline{z}_1
    \,,\quad
v = {z}_2 \overline{z}_2
    \continue
w &=& z_1^2 \overline{z}_2 + \overline{z}_1^2 {z}_2
    \,,\quad
q = (z_1^2 \overline{z}_2 - \overline{z}_1^2 {z}_2)/\ii
\,.
\label{Dang86(1.2)PK}
\eea
The polynomials $\invpt{u,v,w,q}$ are
linearly independent, but related through one syzygy,
\beq
w^2+q^2 - 4\,u^2v = 0 % Double checked syzygy is satisfied by eq Dang86(1.2)PK DBE 05/22/2014
\label{eq:syzPK}
\eeq
that confines the dynamics to a 3-dim\-ens\-ion\-al manifold $\pSRed=\pS/\SOn{2}$, which is a symmetry-invariant repre\-sent\-ati\-on of the
4-dim\-ens\-ion\-al \SOn{2} equivariant dynamics. We call this the \reducedsp. By construction, $u \geq
0$, $v \geq 0$, but $w$ and $q$ can be of either sign. That is explicit if we express $Z_1$ and $z_2$ in polar coordinates ($ {z}_1 = |u|^{1/2} e^{\ii\phi_1}$, $ {z}_2 =
|v|^{1/2} e^{\ii\phi_2}$), so that $w$ and $q$ take the form
\bea
w &=& 2\,\Re(z_1^2 \overline{z}_2) = 2\,u |v|^{1/2} \cos \psi %Triple checked DBE 05/22/2014
\continue
q &=& 2\,\Im(z_1^2 \overline{z}_2) = 2\,u |v|^{1/2} \sin \psi %Triple checked DBE 05/22/2014
\,,
\label{Dang86(1.2)polar}
\eea
where $\psi = 2 \phi_1 - \phi_2$.

The dynamical equations for $\invpt{u,v,w,q}$ follow from the chain rule,
which yields
\bea
  \dot{u} &=& \overline{z}_1 \dot{z}_1 + {z}_1 \dot{\overline{z}}_1 % Triple checked DBE 05/22/2014
\,,\qquad
  \dot{v} = \overline{z}_2 \dot{z}_2 + {z}_2 \dot{\overline{z}}_2 % Triple checked DBE 05/22/2014
\continue
  \dot{w} &=& 2 \,\overline{z}_2 {z}_1 \dot{z}_1 % Triple checked DBE 05/22/2014
           + 2\,{z}_2 \overline{z}_1 \dot{\overline{z}}_1
           + {z}_1^2 \dot{\overline{z}}_2
           + \overline{z}_1^2 \dot{z}_2
\continue
  \dot{q} &=&  (2\,\overline{z}_2 {z}_1 \dot{z}_1 % Triple checked DBE 05/22/2014
           - 2\,{z}_2 \overline{z}_1 \dot{\overline{z}}_1
           + {z}_1^2 \dot{\overline{z}}_2
           - \overline{z}_1^2 \dot{z}_2
           )/\ii
\label{PKinvEqs}
\eea
Substituting \refeq{eq:DangSO2} into \refeq{PKinvEqs}, we obtain a set
of four $\SOn{2}$-invariant equations,

\bea
  \dot{u} &=& 2\,\mu_1\,u+2\,a_1\,u^2+2\,b_1\,u\,v+c_1\,w % Triple checked DBE 05/22/2014
\continue
  \dot{v} &=& 2\,\mu_2\,v+2\,a_2\,u\,v+2\,b_2\,v^2+c_2\,w % Triple checked DBE 05/22/2014
\continue
  \dot{w} &=& (2\,\mu_1+\mu_2)\,w+(2a_1+a_2)\,u\,w+(2b_1+b_2)\,v\,w % Triple checked DBE 05/22/2014
\ceq
             +\, 4c_1\,u\,v + 2c_2\,u^2 +(2e_1 - e_2)\,q
\label{PKinvEqs1}\\
  \dot{q} &=& (2\mu_1+\mu_2)\,q+(2a_1+a_2)\,u\,q
\ceq
             +(2b_1+b_2)\,v\,q
             -(2e_1-e_2)\,w % Triple checked DBE 05/22/2014
\,.
\nnu
\eea
Note that the $\On{2}$-symmetry breaking parameters
 $\{e_1,e_2\}$ of the
Dangelmayr normal form system\rf{Dang86} appear only in the
relative phase combination $(2e_1-e_2)$, so one of the two can be set to zero without loss of generality. This consideration motivated our choice of $e_1 = 0$ in \refeq{eq:pars}.
%[2012-07-31 Evangelos]
Using the syzygy \refeq{eq:syzPK}, we can
eliminate $q$ from \refeq{PKinvEqs1} to get
    \PC{
    Note that $4u^2v-w^2 = 4u^2v(1-\cos^2\psi)$, so
    no serious singularity is introduced this way. Perhaps
    write equations of $(u,v,\cos \psi)$ as in the
    ChaosBook exercises?
    }
\bea
  \dot{u} &=& 2\,\mu_1\,u+2\,a_1\,u^2+2\,b_1\,u\,v+c_1\,w \nonumber % Triple checked DBE 05/22/2014
\\
  \dot{v} &=& 2\,\mu_2\,v+2\,a_2\,u\,v+2\,b_2\,v^2+c_2\,w \label{PKinvEqs1syz}  % Triple checked DBE 05/22/2014
\\
  \dot{w} &=& (2\,\mu_1+\mu_2)\,w+(2a_1+a_2)\,u\,w+(2b_1+b_2)\,v\,w % Triple checked DBE 05/22/2014
\ceq
             +\, 4c_1\,u\,v + 2c_2\,u^2 +(2e_1 - e_2)(4u^2v-w^2)^{1/2}\,
  \nonumber
\eea

This invariant basis can be used either to investigate the dynamics directly or
to visualize solutions\rf{GL-Gil07b} computed in the full equivariant basis \refeq{eq:DangSO2}.

\subsection{\Eqva\ of the symmetry-reduced dynamics}
\label{s:eqva}

The first step in elucidating the geometry of attracting sets is the
determination of their \eqva. We shall now show that the problem of
determining the \eqva\ of the symmetry-reduced \twomode\
\refeq{PKinvEqs1} system $\invpt{u^*,v^*,w^*,q^*}$ can be reduced to
finding the real roots of a multinomial expression. First, let we define
\beq
A_1= \mu_1+a_1\,u+b_1\,v
    \,,\qquad
A_2 = \mu_2+a_2\,u+b_2\,v
\ee{PKinvEqs2a}
and rewrite \refeq{PKinvEqs1} as
%     \newpage
\bea
  0  &=&  2\,A_1\,u +c_1\,w % Double checked DBE 05/24/2014
    \,,\qquad
  0  =  2\,A_2\,v +c_2\,w % Double checked DBE 05/24/2014
\continue
  0  &=& (2\,A_1+ A_2)\,w
          +2\,\left(c_2\,u+2\,c_1\,v\right)\,u % Double checked DBE 05/24/2014
          \ceq
		  + (2e_1-e_2)\,q
\label{PKinvEqs3}\\
  0  &=& (2\,A_1+ A_2)\,q - (2e_1-e_2)\,\,w % Double checked DBE 05/24/2014
\nnu
\eea
We already know that $\invpt{0,0,0,0}$ and $\cartpt{0,-\mu_2/b_2,0,0}$
are the only roots in the $u=0$ and $v=0$ subspaces, so we are
looking only for the $u>0$, $v>0$, $w,q \in \reals$ solutions; there
could be non-generic roots with either $w=0$ or $q=0$, but not both
simultaneously, since the syzygy \refeq{eq:syzPK} precludes that. Either
$w$ or $q$ can be eliminated by obtaining the following relations from
\refeq{PKinvEqs3}:
\bea
	w  &=& - \frac{2\,u}{c_1}\,A_1 = - \frac{2\,v}{c_2}\,A_2 % Double checked DBE 05/24/2014
	\continue
	q &=& \frac{2(-2e_1+\,e_2)\,u\,v}{c_2\,u+2\,c_1\,v} . % Having issues with this DBE 05/24/2014... potentially drops a w = 0 root.
	\label{PKinvEqs4}
\eea
Substituting \refeq{PKinvEqs4}
\DB{}{I think getting to the equation for $q$ throws out a potential $w =
0$ root. Do the first two equations then imply that u,v = 0? If, so then
q = 0 and there's no problem, but I don't think that's the most general
case.}
into \refeq{PKinvEqs3} we get two bivariate
polynomials whose roots are the \eqva\ of the system \refeq{PKinvEqs1}:
\bea
	f(u,v) &=& c_2\,u\,A_1 - c_1\,v\,A_2 = 0 \,,\qquad  \nonumber
	\\
	g(u,v) &=&
 \left(4\,A_1^2 u^2 - 4\,c_1^2\,u^2 v\right)\left(c_2\,u+2\,c_1\,v\right)^2 \label{PKinvEqs5} %Double checked DB 04-30-2012
	\ceq
	+\,4\,c_1^2\,(-2e_1+e_2)^2\,u^2\,v^2 = 0
\,,
	\\
	\mbox{\rm deg}(f) &=& 2, \, \mbox{\rm deg}(g) = 6 \nonumber
\,.
\eea
We divide the common multiplier $u^2$ from the second equation and by
doing so, eliminate one of the two roots at the origin, as well as the
$\cartpt{0,-\mu_2/b_2,0,0}$ root within the invariant subspace
\refeq{eq:DangSO2spsp}. Furthermore, we scale the parameters and
variables as
$\tilde{u} = c_2\,u$,
$\tilde{v} = c_1\,v$,
$\tilde{a_1} = a_1/c_2$,
$\tilde{b_1} = b_1/c_1$,
$\tilde{a_2} = a_2/c_2$,
$\tilde{b_2} = b_2/c_1$
to get
\bea
\tilde{f}(\tilde{u},\tilde{v}) &=&
  \tilde{u}\,\tilde{A}_1 - \tilde{v}\,\tilde{A}_2 = 0 %Double checked DB 04-30-2012
\,,\qquad \mbox{\rm deg}(f) = 2 \, , \label{PKinvEqs5a}
\\
\tilde{g}(\tilde{u},\tilde{v}) &=&  %Double checked DB 04-30-2012
 \left(\tilde{A}_1^2
 - c_1\,\tilde{v}\right)
 \left(\tilde{u}+2\,\tilde{v}\right)^2
 +e_2^2\,\tilde{v}^2 = 0
\,,
\ceq
   \mbox{\rm deg}(g) = 4 \, , \label{PKinvEqs5b}
\\
 && \mbox{where }
\tilde{A}_1 = \mu_1+\tilde{a_1}\,\tilde{u}+\tilde{b_1}\,\tilde{v}
\,,\ceq
\qquad\quad \tilde{A}_2 = \mu_2+\tilde{a_2}\,\tilde{u}+\tilde{b_2}\,\tilde{v}
\,,
\label{PKinvEqs5c}
\eea

Solving coupled bivariate polynomials \refeq{PKinvEqs5a} is not, in general, a trivial task. However,
for the choice of parameters given by \refeq{eq:pars}, eq.~\refeq{PKinvEqs5a} yields
$\tilde{v} = (\mu_1 + \tilde{a}_1 \tilde{u})/(\mu_2 + \tilde{a}_2
\tilde{u})$. Substituting this into \refeq{PKinvEqs5b} makes it a fourth order polynomial in $u$,
which we can solve. Only the non-negative, real roots of this polynomial correspond to \reqva\ in the \twomode\
\statesp\ since $u$ and $v$ are the squares of first and second mode amplitudes,
respectively. Two roots satisfy this condition, the \eqv\ at the origin:
\beq
	\invpol_{\EQV{}} = \invpt{0,0,0,0}\,, %\qquad \mbox{(double)}
\ee{eq:origin}
and the \reqv:
\beq
	\invpol_{\REQV{}{}} = \invpt{0.193569,0.154131,-0.149539,-0.027178}\,.
\ee{eq:reqv}
Note that by setting $b_2 = 0$, we send the \reqv\ at
$\cartpt{0,-\mu_2/b_2,0,0}$ to infinity. Thus, \refeq{eq:reqv} is the
only \reqv\ of the \twomode\ system for our choice of parameters. While
this is an \eqv\ in the invariant polynomial basis, in the
\SOn{2}-equivariant, real-valued \statesp\ this is a 1\dmn\ \reqv\ group orbit.
The point on this orbit that lies in first Fourier mode slice is
(see \refFig{fig:2modes-ssp}\,(c)):
\beq
  \left(x_1, y_1, x_2, y_2\right) = \left(0.439966, 0, -0.386267, 0.070204\right)
\,.
  \label{e-req}
\eeq
We computed the linear stability eigenvalues and eigenvectors of this \reqv
, by evaluating \stabmat\ within the first Fourier mode slice
$\MvarRed_{ij} (\sspRed) = \partial \velRed_i / \partial \sspRed_j |_{\sspRed}$
on the \reqv . Linear stability eigenvalues for the \reqv\ \refeq{e-req}
%\bea
%	\lambda_{1,2} &=& 0.05073 \pm \ii 2.4527, \continue
%	\lambda_3 &=& -5.5055, \quad \lambda_4 = 0 \, .
%\eea
\beq
	\lambda_{1,2} = 0.05073 \pm \ii \, 2.4527, \quad
	\lambda_3 = -5.5055, \quad \lambda_4 = 0 \, .
\eeq
The $0$ eigenvalue corresponds to the direction outside the slice, we expect
this since the reduced trajectory equations \refeq{eq:intSlice} keeps the
solution within the slice. Imaginary part of the expanding complex pair sets
the `winding time' in the neighborhood of the equilibrium to
$T_w = 2 \pi / \Im(\lambda_1) = 2.5617$. The large \eqv\  of the
contracting eigenvalue $\lambda_3$ yields a very thin attractor in the
reduced \statesp, thus, when looked at on a planar Poincare\'{e} section,
the \twomode\ flow is almost one dimensional, see \reffig{fig:psectandretmap}\,(a, b).


\begin{figure*}%[H]
\centering
\includegraphics[height=0.22\textwidth]{2modes-conf-reqv}\quad%
\includegraphics[height=0.22\textwidth]{2modes-confred-reqv}\quad%
\includegraphics[height=0.22\textwidth]{2modes-conf-rpo}\quad%
\includegraphics[height=0.22\textwidth]{2modes-confred-rpo}\quad%
\includegraphics[height=0.22\textwidth]{2modes-conf-ergodic}\quad%
\includegraphics[height=0.22\textwidth]{2modes-confred-ergodic}%
\caption{(Color online)
The \reqv\ \REQV{}{} in
 (a) the system's configuration space becomes an \eqv\ in
 (b) the symmetry-reduced configuration space.
Two cycles of the \rpo\ \cycle{01} in the
 (c) the symmetry-equivariant configuration space become a \po\ in
 (d) the symmetry-reduced configuration space.
A typical ergodic trajectory of the \twomode\ system
in the system's configuration space (e),
 in the symmetry-reduced configuration space (f).
The color scale used in each figure is different to enhance contrast.
}
\label{fig:2modes-conf}
\end{figure*}

\begin{figure*}%[H]
\centering
(a)\includegraphics[width=0.30\textwidth]{2modes-ssp}
(b)\includegraphics[width=0.30\textwidth]{2modes-invpol}
(c)\includegraphics[width=0.30\textwidth]{2modes-sspRed}
\caption{(Color online)
The same trajectories as in \reffig{fig:2modes-conf}\,(a,c,e),
colored green, red and blue respectively,
	(a) in a 3D projection of the 4\dmn\ \statesp ,
	(b) in a terms of 3 invariant polynomials,
	(c) in the 3\dmn\  first Fourier mode \slicePlane.
Note that in the symmetry reduced representations (b and c), the \reqv\ \REQV{}{}
is reduced to an \eqv , the green point; and the \cycle{01} (red) closes onto
itself after one repeat. In contrast to the invariant polynomial representation (b),
in the first Fourier mode \slicePlane (c), the qualitative difference between shifts
by $\approx \pi$ and $\approx-\pi$ in near passages to the {\sliceBord} is very clear,
and it leads to the unimodal Poincar\'e return map of \reffig{fig:psectandretmap}.
}
\label{fig:2modes-ssp}
\end{figure*}


\subsection{No chaos when the reflection symmetry is restored}
\label{s:dfsafs}

Before finishing our discussion of invariant polynomials, we make an
important observation regarding the case when both of the reflection symmetry breaking
parameters, $e_{1}$ and $e_2$ are set to $0$. In this case, $\sspC_{1,2} \rightarrow \bar{\sspC}_{1,2}$
symmetry is restored and the evolution equations for $u$, $v$, and $w$ in \refeq{PKinvEqs1} become
independent of $q$. Furthermore, the time evolution equation for $q$ becomes linear in $q$ itself, so that
it can be expressed as
\beq
    \dot{q} = \xi (u, v) q \,.
\ee{e-qlinearq}
Hence, the time evolution of $q$ can be written as
\beq
    q(\zeit) =  e^{\int_0^\zeit d \zeit' \xi (u(\zeit'), v(\zeit'))} q(0) \, .
\ee{e-qO2solq}
If we assume that the flow is bounded, then we can also assume that a long time
average of $\xi$ exists. The sign of this average determines the long term
behavior of $q(\zeit)$; it will either diverge or vanish depending on the sign of
$\langle \xi \rangle$ being positive or negative respectively. The former case
leads to a contradiction: If $q(\zeit)$ diverges, the symmetry-invariant flow cannot
be bounded since the syzygy \refeq{eq:syzPK} must be satisfied at all times. If
$q(t)$ vanishes, there are three invariant polynomials left, which are still
related to each other by the syzygy. Thus, the flow is confined
to a two dimensional manifold and cannot exhibit chaos. 
We must stress that this is a special result which holds for the two-mode 
normal form with terms up to third order.

\DB{2014-11-03}{Do we know
that it holds ONLY in this case? Or do we just not know if it holds in other cases?}.
\BB{2014-11-06}{We know that it doesn't hold for PDEs and it is likely to
break down if we had mode couplings of higher order.}\DB{2014-11-10}{Do you have a notion of why it might break down
as you add more modes or higher order couplings? It'd be nice to include this here.}

\subsection{Visualizing \twomode\ dynamics}
\label{s:visual}

We now present visualizations of the dynamics of the \twomode\ system in
four different representations: as 3D projections of the four-dimensional
real-valued \statesp, as 3D projections in the invariant polynomial
basis, as dynamics in the 3D \slicePlane, and as two-dimensional
spacetime diagrams of the color-coded field
$u(\conf,\zeit)$\DB{11-3-2014}{Using $u$ here is confusing since we've
just spent the last few pages talking about $u$ in the $(u,v,w,q)$ basis}
is defined as follows:
\[
	u(\conf, \tau) = \sum_{k=-2}^{2} \sspC_k(\zeit) \, e^{i k \conf}
\,,
\]
where $\sspC_{-k} = \bar{\sspC}_k \,, \; 	\sspC_0 = 0$ ,  and $\conf
\in [- \pi, \pi]$. We can also define the symmetry reduced configuration
space representation as the inverse Fourier transform of the symmetry
reduced Fourier modes:
\[
	\hat{u}(\conf, \tau) = \sum_{k=-2}^{2} \sspRedC_k(\zeit) e^{i k \conf}
\,,
\]
where $\sspRedC_{-k} = \bar{\sspRedC}_k$ \,, \; 	$\sspRedC_0 = 0$ \;
and$\conf \in [- \pi, \pi]$. \refFig{fig:2modes-conf}\,(a,b) show the
sole \reqv\ \REQV{}{} of the \twomode\ system in the symmetry-equivariant
and symmetry-reduced configuration spaces, respectively. After the
symmetry reduction, the \reqv\ becomes an \eqv.
\refFig{fig:2modes-conf}\,(c,d) show the \rpo\ \cycle{01} again
respectively in the symmetry-equivariant and symmetry-reduced
configuration space representations. Similar to the \reqv, the \rpo\
becomes a \po\ after symmetry reduction. Finally,
\refFig{fig:2modes-conf}\,(e,d) show a typical ergodic trajectory of the
\twomode\ system in symmetry-equivariant and symmetry-reduced
configuration space representations. Note that in each case, symmetry
reduction cancels the `drifts' along the symmetry ($x$) direction.

As can be seen clearly in \reffig{fig:2modes-ssp}\,(a), these drifts show up in
the Fourier mode representation as $\SOn{2}$ rotations. The \reqv\ \REQV{}{}
traces its \SOn{2} group orbit (green curve in \reffig{fig:2modes-ssp}\,(a))
as it drifts in the configuration space. The
\rpo\ \cycle{01}\,(red) and the ergodic trajectory (blue) rotate
in the same fashion as they evolve. \refFig{fig:2modes-ssp}\,(b,c)
show a three dimensional projection onto the invariant polynomial basis and the 3\dmn\
trajectory on the \slicePlane\ for the same orbits. In both figures, the \reqv\ is reduced
to an \eqv\ and the \rpo\ is reduced to a \po.
