% siminos/cgang/flotsam.tex    master file: main.tex
% $Author$ $Date$

% Predrag 2014-07-14 absorbed into  ../blog/flotsam2modes

%\section{Flotsam}
%\label{s:flotsam}

%In general, \chartBord\ can be avoided by use of
%multiple \template s and arranging them in a way that the trajectories do not
%intersect the \chartBord , as done in \refref{atlas12}. In the particular
%case of the \twomode\ system that we study in this paper, we can describe
%the reduced dynamics using a single \slice\ by picking a \template\ for which
%the \chartBord\ is a flow invariant subspace, hence never visited.

%If a $d$\dmn\ dynamical flow is $\Group$-equivariant under actions of
%an $N$ continuous parameters symmetry group $\Group$, its $d$\dmn\ \statesp\ is foliated
%by $N$\dmn\ group orbits, and the symmetry-\reducedsp\
%$\pS/\Group$ is $(d\!-\!N)$\dmn.
%The simplest continuous symmetry groups are the 1-parameter compact rotation
%group $\SOn{2}$ and the 1-parameter noncompact translation group
%$T(1)$; here we shall focus on the $\SOn{2}$ case.

For instance, in the method of moving frames\rf{FelsOlver98,OlverInv}, the post-processing variant of
the method of slices described in \refsect{s-mframes}, one naturally defines a ``moving frame''
in terms of the group-parameter. In the context of symmetry reduction
in \BBedit{low dimensional systems, \template\ choice} \refeq{firstmodetemp} has been used for
a different representation of \SOn{2}, for the case of \cLf \rf{SiCvi10}.
In the study of spatially extended systems, the ``first-Fourier mode slice''
was first introduced by Budanur~\etal\rf{BudCvi14}, by making the crucial observation that
for a general \SOn{2} equivariant system, the probability that the first Fourier mode
vanishes is extremely small.
