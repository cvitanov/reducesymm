% siminos/cgang/flotsam.tex    master file: main.tex
% $Author$ $Date$

\section{Flotsam}
\label{s:flotsam}

\noindent
{\bf [2012-04-26 Predrag]}
\\
How to use this section? When you remove cogent text from the article
proper - clip  it and paste it into here, for possible reuse later.
\\\\\\

The basic equivariants include
\beq
  \{{z}_1\,,\overline{z}_1 {z}_2 \}
            \,,\qquad
  \{{z}_2 \,, z_1 \overline{z}_2\}
\,.
\label{Dang86(1.3)PK}
\eeq
We have equations \refeq{eq:DangSO2} for $\{{z}_1\,,\overline{z}_1\,,
{z}_2\,,\overline{z}_2 \}$ but perhaps need equations for the
equivariants \refeq{Dang86(1.3)PK} - have not thought this through.
[ChaosBook.org says:]

%[2013-10-07 Burak]
Differential operators acting on functions defined on a periodic domain
usually diagonalized by representing the solution as a Fourier expansion.

%[2013-10-30 Burak] cut from 2modes.tex :
\PC{notation for the translation group?}
As at least 3 dimensions are required for a continuous time flow to
exhibit chaos, in this case the \statesp\ of a $\Group$-equivariant flow
has to be at least 4\dmn. Under linear actions of $\SOn{2}$ the \statesp\
decomposes into 2\dmn\ irreducible subspaces (Fourier components) labeled
by integers $m = 0,1,2,\cdots$, which thus form the natural basis in which
to study $\SOn{2}$-equivariant flows. Nonlinear flows, such as the
1 spatial dimension \KS\ PDE for a `flame front velocity' field
$u=u(x,t)$ on a periodic domain $u(x,t) = u(x+L,t)$, given by
\beq
  u_t = F(u) = -{\textstyle\frac{1}{2}}(u^2)_x-u_{xx}-u_{xxxx}
    \,,\qquad   x \in [-L/2,L/2]
    \,,
\ee{BBks1}
can be
expressed in terms of complex Fourier coefficients $a_k(t)$,
\beq
  u(x,t)=\sum_{k=-\infty}^{+\infty} a_k (t) e^{ i q_k x}
\,,\qquad
  q_k = 2\pi k/L
\,,
\ee{BBeq:ksexp1}
as
\beq
\dot{a}_k= \pVeloc_k(a)
     = ( q_k^2 - q_k^4 )\, a_k
    - i \frac{q_k}{2} \sum_{m=-\infty}^{+\infty} a_m a_{k-m}
\,.
\ee{2m:expan}
($t \geq 0$ is the time, $x$ is the spatial coordinate, subscripts $x$
and $t$ denote partial derivatives with respect to $x$ and $t$).
Nonlinear terms mix an infinity of
Fourier components. In practice, they are represented by truncations to
a finite number of Fourier modes; the most radical truncation that
still might capture some qualitative features of a chaotic flow is to keep
only a pair of Fourier modes.
