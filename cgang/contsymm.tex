\section{Continuous symmetries}
\label{s:symm}

A dynamical system $\dot{\ssp}=\vel(\ssp)$ is said to be
\emph{equivariant} or \emph{\Group-equivariant} under the
symmetry group \Group\ transformations if
                                                \toCB
\beq
	\vel( \ssp )
    =  \matrixRep(\LieEl)^{-1}\vel(\matrixRep(\LieEl)\ssp)
	\,
\ee{equiv}
for every \statesp\ point $\ssp \in \pS$ and every element $\LieEl \in
\Group$, where \LieEl\ is an abstract group element, and
$\matrixRep(\LieEl)$ is its $[d\!\times\!d]$ matrix representation.
Infinitesimally, the equivariance condition \refeq{equiv} is expressed as
a vanishing Lie derivative\rf{DasBuch}
                                                \toCB
\beq
  \Lg\vel(\ssp)  - \Mvar(\ssp) \, \groupTan(\ssp) =0
  \,,
\ee{inftmInv}
where
$ \groupTan(\ssp) = \Lg \ssp $ is the group tangent of $\ssp$,
and $\Mvar(\ssp)$ is the \stabmat\, with elements
$\Mvar_{ij}(\ssp)={\pde \vel_i}/{\pde\ssp_j}|_{\ssp}$.
$\Lg$ is the generator of infinitesimal transformations, such that
$\matrixRep(\theta) = e^{\theta\Lg}$, where the phase $\theta$
parametrizes the group action. As in the example studied here there is
only one continuous symmetry parameter $\theta$, we shall interchangeably
use notations $\matrixRep(\LieEl)$ and $\matrixRep(\theta)$.


If the trajectory of a point $\ssp_\stagn$ coincides with its group
orbit, \ie, the group parameter $\theta (\zeit)$ satisfies
\beq
\ssp (\zeit)
    = \ssp_\stagn + \int_0^\zeit \!\!d\zeit' \vel(\ssp (\zeit'))
    = \LieEl (\theta (\zeit))\,\ssp_\stagn
  \,
\ee{releq}
for all $\zeit$, $\ssp_\stagn$ is a \emph{\reqv}. Expanding
both sides of \refeq{releq} for infinitesimal time yields the
relation $\vel(\ssp_\stagn) = \dot{\theta}(\zeit) \Lg \ssp_\stagn$, which
must hold for all $\zeit$. Thus, for a \reqv\ the  \emph{\phaseVel}
is a constant, $\dot{\theta} = \velRel$. Multiplying the equivariance condition
\refeq{inftmInv} by $\velRel$ we find that \reqva\ satisfy
\beq
(\velRel \Lg - \Mvar (\ssp_\stagn) ) \vel (\ssp_\stagn) = 0
\,.
\ee{ReqvMargEig}

%Separate rpos into new paragraph, added space - DB 9/16/2014
A \statesp\ point $\ssp_\rpprime$ lies on a \emph{\rpo} of period
$\period{\rpprime}$ if its trajectory first intersects its group orbit after
a finite time $\period{\rpprime}$,
\beq
\ssp(\period{\rpprime})
    = \ssp_\rpprime
     + \int_0^\period{\rpprime} \!\!\!d\tau' \vel(\ssp (\tau'))
    = \LieEl (- \theta_\rpprime ) \ssp_\rpprime
  \,,
\ee{relpo}
with a non-zero phase $\theta_{\rpprime}$.

In systems with \SOn{2} symmetry, \rpo s are
topologically 2-tori where the trajectory of $\ssp_\rpprime$ traces out the
same path shifted by the group action over and over again. As we will
show in \refsect{s:numerics}, these tori can be convoluted and
difficult to visualize.

The linear stability of \rpo s is captured by their \emph{Floquet multipliers},
which are important since they appear in the \cycForm s. Floquet multipliers are
defined in the following way: First, we denote the Lagrangian description of the flow as
$\ssp(\zeit) = \flow{\zeit}{\ssp(0)}$ and define the Jacobian as
\beq
\jMpsRed_{\rpprime} = \LieEl (\theta_\rpprime ) \jMps^\period{\rpprime} (\ssp_\rpprime)
\, , \mbox{where}\quad
\jMps^{\zeit}_{ij} (\ssp(\zeit)) = \frac{\partial\ssp_i(\zeit)}{\partial\ssp_j(0)}\, .
\ee{e-rpoJacobian}

The Floquet multipliers are then given by the eigenvalues $\ExpaEig_{p,j}$ of $\jMpsRed_{\rpprime}$.
The magnitude of $\ExpaEig_{p,j}$ determines whether a small perturbation along its corresponding
eigendirection (or Floquet vector) will expand or contract after one period. If the magnitude of
$\ExpaEig_{p,j}$ is greater than $1$, the perturbation expands; if it is less than $1$, the perturbation
contracts. In systems with $N$ continuous symmetries, \rpo s  have $(N+1)$ marginal directions ($\left|\ExpaEig_{p,j}\right| = 1$),
which correspond to the temporal evolution of the flow and the $N$ symmetries. By applying symmetry reduction,
the marginal Floquet multipliers corresponding to the symmetries are replaced by $0$ and make periodic orbit
theory, which requires that the flow have only one marginal direction, applicable.

\emph{Symmetry reduction} is a coordinate transformation that maps
all the points on a group orbit $\LieEl (\theta) \ssp$, which are
equivalent from a dynamical perspective, to a single representative point in a symmetry reduced space.
Such a transformation converts \reqva\ and \rpo s to \eqva\ and \po s in a
reduced \statesp, with no loss of dynamical information; the full \statesp\
trajectory can always be retrieved via the reconstruction equation. One well-studied
technique for symmetry reduction, which works well for low-dimensional
dynamical systems, such as the Lorenz system,
is to recast the dynamical equations in terms of invariant polynomials\rf{GL-Gil07b}.
Establishing such invariant polynomial bases, however, quickly becomes
impractical for systems with more than a dozen dimensions\rf{gatermannHab}. In contrast,
the \mslices\ \rf{rowley_reconstruction_2000,BeTh04,SiCvi10,FrCv11,atlas12,ACHKW11,BudCvi14},
which we study in detail here, is a symmetry reduction scheme applicable to
high-dimensional flows like the \NS\ equations\rf{WiShCv14}.

\subsection{\Mslices}
\label{s-slice}

In a system with $N$ continuous symmetries, a \emph{\slice} \pSRed\ is a codimension $N$ submanifold
of \pS\ that cuts every group orbit once and only once. In the \emph{\mslices}, the solution
of a $d$-\dmn\ dynamical system is represented as a symmetry-reduced trajectory $\sspRed (\zeit)$ within the
$(d-N)$-\dmn\ \slice\ and $N$ time dependent group parameters $\theta(\zeit)$, which
map $\sspRed (\zeit)$ to the full \statesp\ by the group action $\LieEl(\theta(\zeit))$.

While this idea goes back to Cartan\rf{CartanMF},
Rowley and Marsden\rf{rowley_reconstruction_2000}
were the first to apply it to a spatially extended nonlinear flow. They used it to study the dynamics of
the $1D$ \KS\ equation in the neighborhood of
a \reqv, using the \reqv\ itself as the \slice\ `\template'.
Independently, Beyn and Th\"{u}mmler\rf{BeTh04} applied
the \mslices\ to `freeze' spiral waves in reaction-diffusion systems.

The definition given above for the \slice\ puts no restriction on its shape
and offers no guidance on how to construct it. In practice, a
local approximation of the slice called a \emph{\slicePlane} can be constructed
in the neighborhood of a point $\slicep$ by using $\slicep$ as
\emph{\template}. The \slicePlane\ is then defined as the hyperplane that
contains $\slicep$ and is perpendicular to its group tangent $\sliceTan{}
= \Lg \slicep$. The relationship between a \template\, its \slicePlane, and symmetry-reduced trajectories
is illustrated in \reffig{f-ReducTraj1}.

%% ReducTraj*.* - read dasbuch/book/FigSrc/inkscape/00ReadMe.txt
\begin{figure}
\begin{center}
 \setlength{\unitlength}{0.40\textwidth}
 %% \unitlength = units used in the Picture Environment
 \begin{picture}(1,0.8361641)%
   \put(0,0){\includegraphics[width=\unitlength]{ReducTraj5.pdf}}%
   \put(0.06854399,0.36282057){\color[rgb]{0,0,0}\rotatebox{-30.34758661}{\makebox(0,0)[lb]{\smash{$\pSRed$}}}}%
   \put(0.57768586,0.29773425){\color[rgb]{0,0,0}\rotatebox{0.0313674}{\makebox(0,0)[lb]{\smash{$\sspRed(0)$}}}}%
   \put(0.59310014,0.69932675){\color[rgb]{0,0,0}\rotatebox{0.03136739}{\makebox(0,0)[lb]{\smash{$\ssp(\zeit)$}}}}%
   \put(0.8268425,0.39772328){\color[rgb]{0,0,0}\rotatebox{0.03136739}{\makebox(0,0)[lb]{\smash{$\sspRed(\zeit)$}}}}%
   \put(0.81220962,0.66529577){\color[rgb]{0,0,0}\rotatebox{0.03136739}{\makebox(0,0)[lb]{\smash{$\LieEl(\theta(\zeit))\ssp(\zeit)$}}}}%
   \put(0.21150193,0.63610779){\color[rgb]{0,0,0}\rotatebox{0.0313674}{\makebox(0,0)[lb]{\smash{$\LieEl\,\slicep$}}}}%
   \put(0.37740434,0.49597258){\color[rgb]{0,0,0}\rotatebox{0.0313674}{\makebox(0,0)[lb]{\smash{$\slicep$}}}}%
   \put(0.3627714,0.69665188){\color[rgb]{0,0,0}\rotatebox{0.0313674}{\makebox(0,0)[lb]{\smash{$\sliceTan{}$}}}}%
 \end{picture}%
\end{center}
\caption{\label{f-ReducTraj1}The \slicePlane\ \pSRed\ is a hyperplane % \refeq{PCsect0}
passing through the {\template} point $\slicep$
and normal to its group tangent $\sliceTan{}$.
It intersects all group orbits (dotted lines) in an open
neighborhood of $\slicep$.  The full \statesp\ trajectory $\ssp(\tau)$ (solid black line) and the \reducedsp\
trajectory $\sspRed(\zeit)$ (solid green line) belong to the same group orbit
$\pS_{\ssp(\zeit)}$ and are equivalent up to a group rotation
$\LieEl\left(\theta(\zeit)\right)$.
}%
\end{figure}

Reduced trajectories $\sspRed (t)$ can be obtained in two ways: by post-processing data
or by reformulating the dynamics and integrating directly in the \slice. In the post-processing method, which is also called the \emph{method of moving frames}\rf{FelsOlver98,OlverInv} and can be applied to both numerical and experimental data,
one takes the data in the full \statesp\ and looks for the time dependent group parameter
that brings the trajectory $\ssp(\zeit)$ onto the \slice. That is, one finds $\theta (\zeit)$ such that $\sspRed(\zeit) = \LieEl(- \theta (\zeit)) \ssp (\zeit)$
satisfies the \slice\ condition:
\beq
\braket{\sspRed(\zeit) - \slicep}{\sliceTan{}} = 0
\,.
\ee{SliceCond}

In the second implementation, one reformulates the dynamics (for Abelian groups) as
\begin{subequations}\label{eq:so2reduced}
  \beq\label{eq:intSlice}
	\velRed(\sspRed) = \vel(\sspRed)
	-\dot{\theta}(\sspRed) \, \groupTan(\sspRed)
  \eeq
  \beq\label{eq:reconstruction}
	\dot{\theta}(\sspRed) = {\braket{\vel(\sspRed)}{\sliceTan{}}}/
				{\braket{\groupTan(\sspRed)}{\sliceTan{}}}
  \, ,
  \eeq
\end{subequations}
and directly calculates the symmetry-reduced trajectory directly by integrating $\sspRed (\zeit)$ and $\theta (\zeit)$.
In \refeq{eq:so2reduced}, $\velRed$ is the projection of the full \statesp\ velocity \vel(\ssp) onto the \slicePlane.
For a detailed derivation of \refeq{eq:so2reduced}, see \refref{DasBuch}.

While early studies\rf{rowley_reconstruction_2000, rowley_reduction_2003, BeTh04} applied the \mslices\ to a single solution at a time,
studying the nonlinear dynamics of extended systems requires symmetry reduction
of global objects, such as strange attractors or invariant manifolds.
In this spirit, Siminos and Cvitanovi\'{c}\rf{SiCvi10} used the \mslices\ to
quotient the \SOn{2} symmetry from the chaotic dynamics of \cLf. They showed that the
slice-dependent singularity of the reconstruction equation that occurs when the denominator
in \refeq{eq:reconstruction} vanishes (e.g., when the group tangents of the trajectory and the
template are orthogonal) causes the reduced flow to make discontinuous jumps.
This singularity was studied in detail by Froehlich and Cvitanovi\'{c}\rf{FrCv11}.

Two strategies have been proposed in order to handle this problem: The first attempts to
try to identify a template such that slice singularities are not visited
by the dynamics\rf{SiCvi10} or to use multiple `charts' of connected
slices\rf{rowley_reconstruction_2000,FrCv11}.
The latter approach was applied to \cLf\ by Cvitanovi\'{c} \etal~\rf{atlas12} and
to pipe flow by Willis, Cvitanovi\'{c}, and Avila\rf{ACHKW11}.
However, neither approach is straightforward to apply, particularly in
high-dimensional dynamical systems.

A third strategy has recently been proposed by Budanur
\etal\rf{BudCvi14}, who considered Fourier space discretizations of
partial differential equations (PDEs) with \SOn{2} symmetry. They showed
that in these cases a simple choice of \slice template, associated with
the first Fourier mode, results in a \slice\ in which it is highly
unlikely that generic dynamics visit the neighborhood of the singularity.
If the dynamics do occasionally come near the singularity, these close
passages can be regularized by means of a time rescaling.

Here, we shall illustrate this approach, which we call the
`first Fourier mode slice' method, and apply it to a 2-mode ODE normal form. Such
a system is arguably the simplest system with \SOn{2} equivariant dynamics that
can exhibit chaos.

From here on, we will refer to the set of points $\sspRed^*$ on the \slicePlane\ that satisfy
\beq
\braket{\groupTan(\sspRed^*)}{\sliceTan{}} = 0
\,
\ee{ChartBordCond}
as the \emph{\sliceBord}.

In the discussion so far, we have not specified any constraints on the symmetry group
to be quotiented, beyond the requirement that it be Abelian, which is required for \refeq{eq:so2reduced}
to be valid.
\ES{2015-05-20}{Maybe you should discuss why
you imposed the restriction on Abelian groups.}
Since we are interested in spatially extended systems with
translational symmetry, and in order to keep the notation compact,
we restrict our discussion to one dimensional PDEs describing
the evolution of a field $u(x,t)$ in a periodic domain.
By introducing a Fourier series expansion
\beq
	u(x,\zeit) = \sum\limits_{k=- \infty}^\infty u_k\left(\zeit\right) e^{i k x}, \,\,\,u_k = x_k + i y_k,
\ee{FourierSeries}
a PDE invariant under translations can be expressed as a system of coupled nonlinear
ODEs equivariant under the 1-parameter compact group of \SOn{2} rotations.

Truncating the expansion to $m$ modes, we
write the real and imaginary parts of the Fourier coefficients with
$k \geq 1$ as the state vector $\ssp =$ \cartpt{x_1, y_1, x_2, y_2,..., x_m, y_m}.
The action of the $\SOn{2}$ group on this vector
can then be expressed as a block diagonal matrix:
%More explicit form, does not fit in a column:
%\beq
	 %\LieEl (\theta)= \\
					  %\begin{pmatrix}
					  %\cos \theta & \sin \theta & 0               & 0              & \cdots & 0              & 0               \\
					 %-\sin \theta & \cos \theta & 0               & 0              & \cdots & 0              & 0               \\
					  %0             & 0 		   & \cos 2 \theta & \sin 2 \theta & \cdots & 0              & 0               \\
					  %0             & 0            &-\sin 2 \theta & \cos 2 \theta & \cdots & 0              & 0               \\
					  %\vdots       & \vdots      & \vdots         & \vdots        & \ddots & \vdots         & \vdots         \\
					  %0             & 0 		   & 0               & 0              & \cdots & \cos m \theta & \sin m \theta  \\
					  %0             & 0            & 0	             & 0              & \cdots &-\sin m \theta & \cos m \theta
					  %\end{pmatrix}
%\eeq
\beq
	\LieEl(\theta) = \begin{pmatrix}
						R(\theta) & 0 			  & \cdots & 0 \\
						0		   & R(2 \theta) & \cdots & 0 \\
						\vdots	   & \vdots 	  & \ddots & \vdots \\
						0		   & 0	          & \cdots & R (m \theta)
					   \end{pmatrix} ,
\ee{mmodeLieEl}
where
\beq
	R(n \theta) =	\begin{pmatrix}
					\cos n \theta & - \sin n \theta \\
					\sin n \theta & \cos n \theta
					\end{pmatrix}
\ee{rotationmatrix}
is the rotation matrix for $n$th Fourier mode.
The Lie algebra generator for $\LieEl(\theta)$ is given by
\beq
	 \Lg =  \begin{pmatrix}
			 0 & -1 & 0 & 0 & \cdots & 0 & 0 \\
			 1 & 0 & 0 & 0 & \cdots & 0 & 0 \\
			 0 & 0 & 0 & -2 & \cdots & 0 & 0 \\
			 0 & 0 & 2 & 0 & \cdots & 0 & 0 \\
			 \vdots & \vdots & \vdots & \vdots & \ddots & \vdots & \vdots \\
			 0 & 0 & 0 & 0 & \cdots & 0 & -m \\
			 0 & 0 & 0 & 0 & \cdots & m & 0
			 \end{pmatrix} .
\ee{mmodeLg}

In order to construct a \slicePlane\ for such a system, let us choose the following \slice\ \template:
\beq
	\slicep = (1, 0, ..., 0) .
\ee{firstmodetemp}
The \slice\ condition \refeq{SliceCond} then constraints points on the reduced trajectory to the hyperplane given by
\beq
	\sspRed = (\hat{x}_1, 0, \hat{x}_2, \hat{y}_2, ..., \hat{x}_m, \hat{y}_m) .
\ee{slicetemp}
As discussed earlier, group orbits should cross the \slice\ once and only once, which we achive by restricting the \slicePlane\ to the half-space where $\hat{x}_1 > 0$. In general, a \slicePlane\ can be constructed by following a similar procedure for any choice of \template, allowing the symmetry
reduction of the dynamics in a neighborhood of the \template\ bounded by the \sliceBord\ \refeq{ChartBordCond}.
However, the power of choosing template \refeq{firstmodetemp} becomes apparent by computing the border of its \slicePlane.
The points on \refeq{slicetemp} lie on the \sliceBord\ only if $\hat{x}_1 = 0$.
This means that as long the dynamics are such that the magnitude of the first mode never vanishes,
\emph{every} group orbit is guaranteed to have a unique representative point on the \slicePlane.
\footnote{Note that, in general, any template of the form $\slicep =$ \cartpt{\hat{x}'_1, \hat{y}'_1, 0,...,0} would work just as well since the first mode has the symmetry of a circle. The \slice\ \template\
\refeq{firstmodetemp} was chosen for notational and computational convenience.}
%

More insight can be
gained by writing the symmetry-reduced evolution equations \refeq{eq:so2reduced}
explicitly for the template \refeq{firstmodetemp}:
\begin{subequations}
\beq
\velRed ( \sspRed )  = \vel(\sspRed)
   - \frac{\dot{y}_1\left(\sspRed\right)}{\hat{x}_1} \, \groupTan(\sspRed) \, ,
\label{e-so2red1stmode}
\eeq
\ESedit{
  \beq\label{eq:reconstruction1stmode}
	\dot{\theta}(\sspRed) = \frac{\dot{y}_1(\sspRed)}{\hat{x}_1}
  \, .
  \eeq
}
\end{subequations}
Since the argument $\phi_1$ of a point $(x_1,y_1)$ in the first Fourier mode plane is given by $\phi_1=\tan^{-1}\frac{y_1}{x_1}$,
its velocity is
\beq
  \dot{\phi}_1 = \frac{x_1}{r_1^2}\dot{y}_1-\frac{y_1}{r_1^2}\,\dot{x}_1\,,
\eeq
where $r_1^2=x_1^2+y_1^2$. Therefore, on the \slicePlane \refeq{slicetemp}, where $\hat{y}_1=0$,
\beq\label{eq:phi1}
  \dot{\theta}(\sspRed) = \dot{\phi}_1(\sspRed)\,.
\eeq
That is, for our choice of \template\ \refeq{firstmodetemp}, the reconstruction phase coincides with
the first Fourier mode phase. From a group-theoretic point of view,
this choice of template is therefore more natural than the more physically motivated templates used in
\refrefs{rowley_reconstruction_2000,BeTh04,SiCvi10,FrCv11,atlas12,ACHKW11}.

\ES{2014-05-20}{What would you think about introducing a functional
notation for $y_1$ as in \refeq{eq:reconstruction1stmode}?}
\PC{2014-05-25: We should copy and paste from \refref{BudCvi14} all stuff
about the in-slice time here? }
\ES{2014-05-25}{I agree with that suggestion, since rescaled time makes the
method work for more general flows than 2-modes,
and it should be explained here. However, we do not provide an example for it's
usefulness here. How about fishing for a second set of parameters where
trajectories come consistently close to $(0,0,\ldots)$, and using rescaled
time there? Even though $(0,0,\ldots)$ would not be visited, there should still
be apparent jumps that would go away by time-rescaling.}
In general, additional care must be taken when the dynamics approach the \slice\ border $\hat{x}_1 = 0$.
Whenever this happens, the near-divergence of $\velRed$ can be regularized by introducing a rescaled time coordinate such that
$d\hat{\zeit} = d\zeit / \hat{x}_1$\rf{BudCvi14}. However, in our study of the \twomode\ system that we will introduce below,
we omit this step since points with a vanishing first mode are in an invariant subspace of the flow and hence are never
visited by the dynamics.

\subsection{Postproccessing approach}
\label{s-mframes}

\ES{2014-05-15}{This section refers to the ``method of moving frames'' (postprocessing approach)
and as such belongs here. It has to be generalized a bit (I will do it if you agree with the change).
Second mode \slice\ can be used also for integration on the \slice\ and can be introduced earlier.
}

\begin{figure}%[H]
\centering
 \includegraphics[width=0.45\textwidth]{BBgorbitsandslice}
\caption{$\SOn{2}$ Group orbits of \statesp\ points \cartpt{0.75, 0, 0.1, 0.1}
(orange), \cartpt{0.5, 0, 0.5, 0.5} (green)
\cartpt{0.1, 0, 0.75, 0.75} (pink) and the first mode \slicePlane\
\refeq{slicetemp} (blue). The group tangents at the intersections with the
\slicePlane\ are shown as red arrows.
When the magnitude of the first Fourier mode becomes small relative to the
magnitude of the second one, the group tangent becomes more close to parallel to the
\slicePlane.}
\label{fig:BBgorbitsandslice}
\end{figure}

In \refsect{s-slice}, we explained the general procedure for reducing
the \SOn{2} symmetry by \mslices ; here, we focus on its geometrical
interpretation. The \slice\ defined by \refeq{firstmodetemp} and the directional constraint
$\hat{x}_1 > 0$
%
\DB{2014-05-15}{Think this should be $\hat{x}_1$. Revert to $x_1$ if I'm wrong.}
%
fixes the phase of the first complex Fourier mode to $0$. This also follows from
the fact that the reconstruction phase of the first Fourier mode \slice\ is the
phase of the first mode \refeq{eq:phi1}.\DB{2014-10-28}{I don't really understand what this   is saying} In complex representation, we can express
the relationship between Fourier modes ($\sspC_n = x_n + \ii y_n$) and their
representative points ($\sspRedC = \hat{x}_n +  \ii \hat{y}_n$) on the \slicePlane\
by the $\Un{1}$ action:
\beq
	\sspRedC_n = e^{-\ii n \phi_1} \sspC_n \, .
\ee{e-1stmodeTransform}
This relation provides another interpretation for the \sliceBord :  For template \refeq{firstmodetemp},
the \sliceBord\ condition \refeq{ChartBordCond} yields,
$|\sspRedC_1| = |\sspC_1| = 0$, which means that the phase of the first Fourier
mode \BBedit{and hence \DBedit{the transformation \refeq{e-1stmodeTransform}
are not defined.}}
%
This is illustrated in \reffig{fig:BBgorbitsandslice}, which shows the first Fourier mode \slicePlane\ along with three-dimensional projections of
the group orbits of points with decreasing $|\sspC_1|$. When the magnitude of the first
mode $\sqrt{\hat{x}_1^2 + \hat{y}_1^2}$, relative to that of the second mode is
small (pink curve in \reffig{fig:BBgorbitsandslice}), the group tangent has a larger
component parallel to the \slicePlane . If the first mode magnitude was exactly
$0$, the group tangent would lie entirely on the \slicePlane , satisfying the
\sliceBord\ condition.
%\subsection{Polar coordinates}
%\label{s-polar}

In \refref{PoKno05}, a polar coordinate representation of two Fourier mode
truncation is obtained by defining the $\LieEl$-invariant phase: $\Phi = \phi_2 - 2 \phi_1$
and three symmetry invariant coordinates \polpt{r_1, r_2 \cos \Phi, r_2 \sin \Phi}.
One can see by direct comparison with \refeq{e-1stmodeTransform}, which
yields $\sspRedC_1 = r_1$ and $\sspRedC_2 = r_2 e^{\ii \Phi}$, that this
representation is a special case $(m=2)$, of the \slice\ defined by
\refeq{firstmodetemp}. Corresponding ODEs for the polar representation
were obtained in \refref{PoKno05} by  chain rule and substitution. Note
that \mslices\ provides a general form \refeq{e-so2red1stmode} for symmetry
reduced time evolution.\DB{2014-10-28}{This section is kind of weird... doesn't seem to flow and we don't really use this stuff anywhere. Maybe consider deleting.}
\BB{2014-10-30}{In \refref{PoKno05} they use this representation and their plots
of chaotic dynamics (yes, they have parameters for which the \twomode\ system exhibits
chaos) look very similar to ours. We have to state somewhere that this similarity
is not an accident. I agree that it looks awkward as an individual subsection,
I removed the subsection title for now, we may move it to the `flow' section.}
