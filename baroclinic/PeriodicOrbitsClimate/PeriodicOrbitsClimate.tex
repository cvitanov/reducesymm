% siminos/baroclinic/PeriodicOrbitsClimate/PeriodicOrbitsClimate.tex

% $Author: predrag $ $Date: 2012-08-03 10:27:11 -0400 (Fri, 03 Aug 2012) $
%% Based on a TeXnicCenter-Template by Gyorgy SZEIDL.
%%%%%%%%%%%%%%%%%%%%%%%%%%%%%%%%%%%%%%%%%%%%%%%%%%%%%%%%%%%%%

%------------------------------------------------------------
%
\documentclass{article}%
%Options -- Point size:  10pt (default), 11pt, 12pt
%        -- Paper size:  letterpaper (default), a4paper, a5paper, b5paper
%                        legalpaper, executivepaper
%        -- Orientation  (portrait is the default)
%                        landscape
%        -- Print size:  oneside (default), twoside
%        -- Quality      final(default), draft
%        -- Title page   notitlepage, titlepage(default)
%        -- Columns      onecolumn(default), twocolumn
%        -- Equation numbering (equation numbers on the right is the default)
%                        leqno
%        -- Displayed equations (centered is the default)
%                        fleqn (equations start at the same distance from the right side)
%        -- Open bibliography style (closed is the default)
%                        openbib
% For instance the command
%           \documentclass[a4paper,12pt,leqno]{article}
% ensures that the paper size is a4, the fonts are typeset at the size 12p
% and the equation numbers are on the left side
%
\usepackage{amsmath}%
\usepackage{amsfonts}%
\usepackage{amssymb}%
\usepackage[latin1]{inputenc}
\usepackage[T1]{fontenc}
\usepackage{times}
\usepackage[pdftex]{graphicx}
\usepackage{array}
\usepackage{verbatim}
\usepackage[pdftex,colorlinks]{hyperref}

\newcommand{\arXiv}[1]{{\tt \href{http://arXiv.org/abs/#1}{arXiv:#1}}}

%-------------------------------------------
\begin{document}

\title{Exploring the Implication of Periodic Orbit Theory in Climate}
\author{Sebasti\'{a}n Ortega Arango
\\Georgia Institute of Technology
\\School of Earth and Atmospheric Sciences
\\Atlanta, GA}
\date{May 4, 2012}
\maketitle

\section*{Abstract}
The implications of nonlinear dynamical systems in climate and weather have been
acknowledge widely in the scientific community; the lessons this systems teaches us
in predictability are invaluable, and today's computing advances make it possible to explore
even more these dynamics. In this short abstract a possible research horizon is proposed for
my future research work, inspired in the recent finding of periodic orbits embedded in high
dimensional flows (for instance see references \cite{GHCW07}, \cite{CvWiAv12}, \cite{Visw07b}) and what I have learned during the course of this semester in my research
and course work at GeorgiaTech.
%For instance, if we think in a situation where we
%can model the atmosphere as a Lorenz attractor, then depending where our initial
%conditions are, our predictability skills are affected \ref{PalmerPWC}.

When considering the Navier-Stokes equations, either adapted for a particular
purpose as to filter unwanted phenomenologies (i.e. the quasi-geostrophic equations) or
solving directly the small scale, it is often\footnote{
If the boundaries conditions of the flow are regular enough, which for large scale flows of the atmosphere is
commonly the case.
}
possible to decompose the flow field in terms of a global orthogonal basis and the
amplitudes of each of this functions. That is:

\begin{equation}
\textbf{u}=\sum \widehat{u} \phi (\textbf{x})
\end{equation}

where $\phi$ are basis function defined for the entire domain\footnote{
This could be any set of orthogonal functions (i.e.
Fourier, Legendre or Chebyshev series)
}
and $\widehat{u}$ their respective amplitudes. The convenience of this approach, apparat
from the stability and fast convergence of the solutions, is that is allows us to think
of the flow as a multidimensional dynamical system; each harmonic defining a state
space coordinate of it.

Probably the first to show the power of this methods in climate was Edward Lorenz. His approach to study
dynamical systems was to reduce the dimensionality of the governing equations by considering
only a couple of harmonics of a spectral decomposition (as in \cite{Lorenz60}). The equations
so obtained where not adequate for climate weather prediction, but revealed the intrinsical
complexity of this systems and allowed their study. Indeed, from this kind of simplifications
the well know Lorenz attractor emerged; which lead to intense research in the area of predictability
and chaos.

The result of the study of this low dimensional nonlinear systems lead then to great advances, and
the richness of even relatively simple systems was recognized. The approach taken was to search for topological invariant solutions embedded in this manifolds\footnote{
i.e. the state space of the system.
}.
That is, to search for equilibrium points, relative equilibrium, periodic orbits, relative periodic orbits and heteroclinic connections,
and analyze their stability properties and robustness to changing parameters. It was shown that the properties of the system
depend on these invariant solutions, and that averages over time can be expressed as series of the
properties of this invariant manifolds.

It is interesting to think about the reason why such a limited amount of harmonics can reveal
such important characteristics of the flow, and how important are this solutions for the full systems.
An answer was provided by Hopf when thinking about turbulence, in his view of this matter he imagined the fluid as a dynamical
system of a infinite-dimensional state space dominated by it's invariant solutions. The recurrent motions
of turbulence being explained as "walks" of the solution near this topological features. Furthermore, he noticed
that the flow itself belonged to a manifold of lower dimensionality; for instance, a steady laminar solution in a pipe
flow can be explained with a single point in the infinite-dimensional space. However, at the time was no sufficient
computational power to test these ideas.

With the increase in computing power recent studies has shown the validity of these ideas, and techniques have been developed to search for invariant
solutions of these flows. For instant in \cite{CvWiAv12} for pipe flow or in \cite{n00bs} for couette flow; there, dimensions are in the
order of $10^5$ and invariant solutions are still found playing important roles in determining the
averaged properties of the flow. Moreover, it is found that only a handful of unstable dimensions are important
in these high dimensional space, and novel approaches to visualize this structures are proposed.

In geophysical fluid dynamics, the search for high dimensional invariant solutions is just beginning\footnote{
I am only aware of studies of this type being carried in GeorgiaTech. This, for Baroclinic Instability by Professors Predrag Cvitanovi{\'c} and Annalisa Bracco. For my Chaos term project I worked on this; trying to reduce the symmetries of these simulations. If possible, this would allow to find periodic orbit easier.
}. And it would be interesting to
see what these tell us about weather and climate. The motivation being that recurrent patterns are common place
in the atmosphere, so that exploring if there is a connection between periodic orbits and low frequency phenomena should be interesting. However, we will have to wait
until more work is done in this area.

I propose then to explore this field for my research work, specifically focusing in the tropics and its dominant modes of low frequency
variability. Many steps have to be taken first and these would hopefully become clear as the research evolves. For the moment, and if this path of research is to be followed, then it would be useful to search for an adequate set of equations to describe the tropics (maybe those used in \cite{Majda03}) and model them them as done in reference \cite{Webster72}. This would aloud us to keep the model as simple as possible an increase in complexity if it is seen to be necessary.

\bibliographystyle{unsrt}
\bibliography{../../bibtex/siminos}


\end{document}
