% \Solution{henlatt}{soluHenlatt}{23jan2018}{Cat map}
% GitHub/reducesymm/Solutions/soluHenlatt.tex called by kittens/sidney.tex
% $Author: predrag $ $Date: 2019-08-13 12:09:47 -0500 (Tue, 13 Aug 2019) $

% Predrag                                               2020-11-30

%%%%%%%%%%%%%%%%%%%%%%%%%%%%%%%%%%%%%%%%%%%%%%%%%%%%%%%%%%%%%%%%%%%
\solution{exer:catMapGreenInf}{H\'enon temporal lattice.}{

(a) Here's my initial attempt, I'm trying to see if the flow conservation law \refeq{Det(jMorb)eights} still works for the Henon map:
$$\phi_{n+1}+a\phi^2_n-b\phi_{n-1}=1$$
The first step seems to be to construct the orbit Jacobian $\mathcal{J}$:
\begin{equation}\label{1}
F[\mathbf{\Phi}]=\mathcal{J}\mathbf{\Phi}-I
\end{equation}
Where $I$ is the identity matrix, and $F$ is the function where we want to find the zeros for (the orbits). We can rewrite this as:
\begin{equation}\label{2}
\left(\sigma+aI\mathbf{\Phi}-b\sigma^{-1}\right)\mathbf{\Phi}=I
\end{equation}
Therefore, $\mathcal{J}$ is
\begin{equation}\label{3}
\mathcal{J}=\sigma+aI\mathbf{\Phi}-b\sigma^{-1}=
  \left[ {\begin{array}{ccccc}
   a\phi_1 & 1 & 0 & \cdots & -b \\
   -b & a\phi_2 & 1 & \cdots & 0 \\
   0 & \ddots & \ddots & \ddots & \vdots \\
   \vdots & \cdots & -b & a\phi_{n-1} & 1 \\
   1 & 0 & \cdots & -b & a\phi_n\\
  \end{array} } \right]
\end{equation}

Before I take a crack at seeing if this flow conservation still holds, I do have some questions:
\begin{itemize}
\item[Q1 Sidney]
It appears that the derivation from chapter 23 (eqn 23.17, I don't know how to cite that specifically) the denominator of the sum rule is a product of the eigenvalues $\Lambda_{pi}$, which (if I remember correctly) are just the eigenvalues of the orbit Jacobian of the flow or map, which from basic linear algebra I know to be just the determinant of the orbit Jacobian. It cannot be that straightforward, where is the flaw in my logic?

\item[Q2 Sidney]
How do I go from the periodic orbit formulation of the sum rule from Ch 23 to the lattice formulation? My initial thought is that since lattice states are akin to a periodic orbit (right?) that the sum can just be immediately changed from a sum over all periodic orbits, to a sum over all lattice states. Is this reasoning correct? 

\item[Comment Sidney]
I now realize that the flow sum rule involving the orbit Jacobian (NOT the Hill matrix) is a fundamental property that applies to all systems (at least all closed systems), what I now know is that I need to work out if I can convert between the determinant of the orbit Jacobian and the determinant of the Hill matrix. 

\item[Plan Sidney]
I am going to try to see what I can do with the block matrix proof, and I will get back to everyone on Friday
\end{itemize}
\hfill (Sidney Williams) %{2018-03-08}{
    } % end \solution{exer:catMapGreenInf}
%%%%%%%%%%%%%%%%%%%%%%%%%%%%%%%%%%%%%%%%%%%%%%%%%%%%%%%%%%%%%%%%%%%%%%%%
