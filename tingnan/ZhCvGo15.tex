% reducesymm/tingnan/flotsam.tex    master file: diffuse/main.tex

\section{ZhCvGo15 flotsam}
\label{s:flotsam}

Squirrel away here potentially recyclable text from the
paper\rf{ZhCvGo15} proper, \texttt{diffuse/ZhCvGo15.tex}

\bigskip\bigskip
\begin{description}

\item[2016-07-24 Predrag]
    make sure the source code for all figures is in
    \emph{reducesymm/figSrc/}. In ideal world, it would be listed in
    \emph{00ReadMe.txt} and code would also be commented, so one can
    easily find the source for a given figure
\item[2016-07-24 Predrag]
    check ChaosBook.org references for \emph{diffusion.tex},
    make sure that they are all included both here and there.
    \\
    ~[ ] mark here when done

%\item[2016-07-24 Predrag]
%    \refFig{fig-chaoticBouncing} is Tingnan's, move to ChaosBook.org
%    (otherwise
%	would have to ask J.-P. Eckmann for permission to use it. It has not been
%	published.)
%    % Similarly, we need our own \reffig{fig-chaoticBouncing}\,(b),
%	% I do not remember where this one is from.


\end{description}
\bigskip\bigskip

%remember to cite Cvitanovi\'c and Eckhardt\rf{CvitaEckardt} {\em Symmetry
%decomposition of chaotic dynamics}

%A test of hyperlinking: what looks better?
%
%DasBuch\rf{DasBuch}
%or
%\refref{DasBuchMirror} {Chapter ``{World} in a mirror''}
%or trace~\refref{CBtrace}
%or \refref{Froeh10}
%or ChaosBook convergence\rf{CBconverg}
%or Predrag\rf{Cvi07}


In the international crystallographic notation, the hexagonal lattice is
called $p6mm$, with point group $6mm$, where prefix $p$ indicates that
the unit cell is primitive (not centered),


                                                            \toCB
\refRef{solomon1994chaotic} {\em Chaotic advection in a two-dimensional
flow: L\'evy flights and anomalous diffusion} studies chaotic transport
experimentally passive tracers in $2$\dmn\ rotating in laminar, chaotic,
and turbulent flows which can be described as anomolous deterministic
diffusion in periodic arrays. We do not touch that here.


%
%%%%%%%%%%%%%%%%%%%%%%%%%%%%%%%%%%%%%%%%%%%%%%%%%%%%%%%%%%%%%%%%%%
%\SFIG{fig_lor_4}
%{}{
%Deterministic diffusion in a
%finite horizon periodic Lorentz gas.
%\hfill (T. Schreiber)
%}{fig-lor-4}
%%%%%%%%%%%%%%%%%%%%%%%%%%%%%%%%%%%%%%%%%%%%%%%%%%%%%%%%%%%%%%%%%%
%

As a gedanken experiment, suppose a passively controlled robot is moving
in a boulder field at constant speed. The diffusion coefficient, which
describes roughly how much area the robot explored in a unit time, is the
key quantity we would like to investigate. We place the boulder in a
regular, periodic array and assume that we are in the heavy boulder limit
such that after each collision event, only the robot is deflected and
boulders remain immobilized. With the presumptions we effectively created
a periodic Lorentz gas model\rf{Dettm14} for locomotion in a boulder
field.

In biological field,  many important dynamical processes (often at
cellular level) are described in  terms of diffusion coefficients. Such
examples include the transport of ions  across the cell
membranes\rf{Stein12} and the movement of  microorganism(e.g.
bacterials) through natural  ecosystems\rf{koch1990diffusion}. In this
paper we will discuss the transport  property of more "macroscopic"
systems (such as moving  robots\rf{saranli2001rhex}) where a ``diffusive
description'' also applies.

Chaotic motions exist in many field of physics systems, blah. There are
physical problems such as beam defocusing in particle accelerators or
chaotic behavior of passive tracers in $2$\dmn\ rotating flows which can
be described as deterministic diffusion in periodic arrays. In the field
of animal/robotic locomotion, we will show that a macroscopic ``diffusion
view'' also applies.

In the macroscopic world, there are recent
studies shown that robotic locomotion in heterogeneous granular
environment also demonstrates scattering-diffusive pattern.

Lately, there has been an increased focus on robot locomotion in complex
environments (check Science and ROPP reference, the systematic study of
interactions between environment and locomotion, which we now
call``robophysics''). Many of those studies use substrates that are
spatially homogeneous and we have a good
understanding\rf{li2009sensitive,li2013terradynamics}. However, little is
known for locomotion in heterogeneous environment. There are some limited
experimental/theoretical studies for relatively simple settings (e.g.
slopes,ref). In this paper we intend to approach the longterm transport
properties of locomotion in a more complex environment, \ie, in a field
of scatterers of which the scales are comparable to the locomotor.


 As we are concerned with the long time behavior,
 this problem can be circumvented
 by replacing $ \hf^t(\tx{\tpk}) $ by the mean
 drift in the $t \rightarrow \infty$ limit.
 $ \hf^t(\tx{\tpk}) $ is a translation in $\hM$ for each
 complete cycle $p$ in $M$, so we replace
 \bea
 \hf^t(\tx{\tpk}) - \tx{\tpk}
 \,\Longrightarrow \,
 &&
 { { \hf^{m_p t}(\tx{\tpk}) - \tx{\tpk} }
 % \over         m_p
 }
    % \,\equiv \, r {\tilde n}_{\tp} (\tx{\tpk})
 \continue
t &=& r \period{\tilde{p}}, \quad m_p = \period{p}/\period{\tilde{p}} \quad \tx \in \tp \,\, ,
 \eea
 in Eq.~(240).
 The magnitude of ${\tilde n}_{\tp}(\tx{\tpk})$, the mean
 global drift per one traversal of the fundamental cycle $\tp$, is
 independent of the starting point, but its direction is not; the
 reason is that each fundamental domain cycle corresponds to a set of
 trajectories in $\hM$.
 The ${1 \over {|G|}} \sum$ average in
 Eq.~(240) then generates all distinct global drift
 directions, so we can again replace the
 sum over cycle points by the factor $\period{\tilde{p}}$, and obtain the
 $Z$ function Eq.~(14) for the $\alpha $ irreducible subspace
 $$
 Z(\beta,s)_\alpha\,=\,\prod_{\tp \in \t{\cal{P}} } \exp
 % \left( -
  % \sum_{r=1}^\infty {1 \over r}
 % {{
  % \chi_\alpha(h^r_{\tp})
  % }
 % \over
 % { | \det \left( \bf{1}-\t{\bf J}_{\tp}^{r} \right) | }
 % }
 % e^{ ( \beta \cdot {\tilde n}_{\tp} - s \period{\tilde{p}}) r}
  % \right)
 \,\, .
 Eq.~(24)
 $$
 $\hn_{\period{\tilde{p}}}(\tx_k)$
%{\tt NOnsense...}
 The leading eigenvalue of the
unsymmetrized
 operator Eq.~(8) is
 the leading eigenvalue of the symmetric subspace for which
 $\chi_\alpha(g)=1$ for all $g \in G$.

The interesting dynamical averages is determined by the spectrum of the operator
\beq \det(\eigenvL - \Lop) \,=\,\prod_{p} \exp \left(
  - { \sum_{r=1}^\infty {1 \over r} { e^{(\beta \cdot \hn_p- s
        \period{p}) r} \over \oneMinJ{r} }
  } \right) \,,
\ee{lor-diff-14}
or the corresponding \dzeta\
