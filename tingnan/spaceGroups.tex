% GitHub cvitanov/reducesymm/tingnan/spaceGroups.tex

% Predrag                                       2014-07-14
	
\chapter{Notes on space groups}
\label{c-spaceGroups}




\begin{description}

\item[2014-07-14 Predrag]
Separated out here our notes on condensed matter theorists'
standard version of space groups (the one we are too lazy to learn)

\item[2014-04-18 Predrag]
Dresselhaus \etal\ textbook\rf{Dresselhaus07}
(\HREF{http://chaosbook.org/library/Dresselhaus07.pdf}{click here})
is good on discrete
and space (but not continuous) groups.
The MIT~course~6.734
\HREF{http://stuff.mit.edu/afs/athena/course/6/6.734j/www/group-full02.pdf}
{online version} contains much of the same material.

Chapter {\em 9. Space Groups in Real Space} is quite clear on matrix
representation of space groups. The translation group $T$ is a normal
subgroup of \Group\ and defines the Bravais lattice. The cosets by
translation $T$ (set all all group elements obtained by all translations)
form a factor group $\Group/T$, isomorphic to the point group $g$
(rotations). All irreducible representations of \Group\ can be compounded
from irreducible representations of $g$ and $T$.

Section {\em 9.3 Two-Dimensional Space Groups}: In the international
crystallographic notation, our hexagonal lattice \#17 is called $p6mm$,
with point group $6mm$.
\beq
\Group = \{
E, C_6^+, C_6^-, C_3^+, C_3^-, C_2,
\sigma_{d1}, \sigma_{d2}, \sigma_{d3},
\sigma_{v1},\sigma_{v2}, \sigma_{v3}
\}
\ee{D12generators}
Prefix $p$ indicates that the unit cell is primitive (not centered). This
is a simple or {\em symmorphic} group, which makes calculations easier.
The Bravais lattice is two equilateral triangles, not sure how to relate
it to our hexagonal `elementary cell'? A Brilloun zone? Bravais `unit cell'
is illustrated in Fig.~E.2. ChaosBook `Fundamental
domain' makes an appearance in Fig.~10.2.

The main trick in quantum-mechanical calculations is to go to the
\emph{reciprocal} space (see Fig.~E.2), in our case with the full
$\Gamma$ point, $k=0$, wave vector symmetry (see Table~10.1), and `Large
Representations'. This is something we have not tried in deriving the
trace formula for deterministic diffusion.

Sect. {\em 10.5 Characters for the Equivalence Representation} look
like those for the point group, sort of. We should probably work
out problems 10.1 and 10.2.

\item[2014-04-26 Predrag] We should read Chapter 7 of Gaspard\rf{PG97}.
Kimberly and I both have a hard copy. Gaspard discusses irreps of
the translation group in some detail.

\item[2014-04-26 Predrag]
I'm quite convinced that this problem is a solved problem, we just have
to understand how characters are used to project irreps of space groups.
One has to go to the reciprocal lattice, and utilize utilize the concept
of the `star'. All physical chemists and crystallographers know how to do
this - we just need to be good students and read the stuff. Our case is
the most symmetric, $p6mm$ lattice - it is surely worked out in some
paper in a way that we can understand. They call our `fundamental domain'
the `motif' or the `asymmetric unit'.

I found projects in \HREF{http://www-f1.ijs.si/~ziherl/SF.html} {this
course} easy to read, especially Toma\v{z} \v{C}endak, who reviews the
space groups theory in a pretty simple way, and Zavadlav has very pretty
wallpaper groups illustrations. Ziherl recommends Elliott and  Dawber,
{\em Symmetry in Physics}\rf{ELLIOTT}.

Joseph Sidighi likes Cotton\rf{Cotton08} {\em Chemical applications of
group theory}, which has no characters for space groups, but a very
pretty discussion of their geometry in Chapt.~11. Cotton was ``the most
influential inorganic chemist to ever have lived.''

If we succeed in factorization, this would merit a publication.
It is OK if you do not succeed in factorization - I have failed myself, so
who am I to cast the first stone:)

\item[2014-05-05 Tingnan] My numbers of elementary cell prime cycles,
Lyapunov exponents and diffusion coefficients are listed and compared
to the Schreiber calculation in \reftab{TCELL1}. I still have the problem
for the convergence of diffusion coefficients. Did we miss a factor of 2
somewhere?

\item[2014-05-28 Tingnan] I got it!

I see the trouble more clearly as indicated in the previous blogs: The
averaging over the space group (a point group action followed by a
translation). I think in Pavel's approach we still needs to worry
because his three group actions do not commute with each other?

Let us speculate a bit about the group actions in the lattice group. We
will use the same notation as in Dresselhaus \etal\
textbook\rf{Dresselhaus07},
\[
	\mathbf{a} = \{R_g\vert\tau\}
\,
\]
where $g$ is an element of the point group and $\tau$ is a translation.

%A potentially useful commutation relation is posted here (9.15):
%\[	\{R_g\vert\tau\}\{\epsilon\vert\tau\}\{R_g\vert\tau\}^{-1} = \{\epsilon\vert R_gt\}
%\]

Since our group is $p6mm$ and is symmorphic, the irreducible representation
of the group (of the wave vector) is simple (10.35),
\[
D_{k}^{\Gamma_{i}}(\{R_{g}\vert\mathbf{R}_{n}\})=e^{i\mathbf{k\cdot R}_{n}}D^{\Gamma_{i}}(R_{g})
\,
\]
where $D^{\Gamma_{i}}(R_{g})$ is the irreps of the point group $C_{6v}$,
which can be readily found (\HREF{http://www.cryst.ehu.es}{click here}).

\item[2014-06-03] Finally I got Gaspard's book from the GIL service. Here
are updates on reciprocal lattice.

Suppose we have $M$ dimensional phase space flow $f^{t}(\mathbf{X})$
with $L<M$ position coordinates. The equation of motion is invariant
under a space group $G$. Let us first examine the translation subgroup
$T$ of $G$.

With a proper choice of {\Poincare} section $\mathcal{P}$, the flow
$f^{t}(\mathbf{X})$ can be expressed as a suspended flow $F^{t}(\mathbf{x},\tau,\mathbf{l})$
(Gaspard 7.8), where $\mathbf{x}\in\mathcal{P}$ is on the section,
$\tau\in[0,T(\mathbf{x}))$ the time interval before first return,
and $\mathbf{l}$ a spatial vector denotes the center of the element
cell the current point belongs to. The flow is controlled by a set
of mappings:

\[
\begin{cases}
\mathbf{x}_{n+1}=\phi(\mathbf{x}_{n}) & \mathbf{x\in\mathcal{P}}\\
t_{n+1}=t_{n}+T(\mathbf{x}_{n})\\
\mathbf{l}_{n+1}=\mathbf{l}_{n}+a(\mathbf{x}_{n}) & a(\mathbf{x}_{n})\in\mathcal{L}
\end{cases},
\]
where we denote the Bravais lattice $\mathcal{L}$. We can relate
the mappings with the flow:
\begin{align*}
F^{t}(\mathbf{x},\tau,\mathbf{l}) & =\left\{ \phi^{n}\mathbf{x},\tau+t-\sum_{j=0}^{n-1}T(\phi^{j}\mathbf{x}),\mathbf{l}+\sum_{j=0}^{n-1}a(\phi^{j}\mathbf{x})\right\} ,\\
 & \qquad\mathrm{for}\qquad0\leq\tau+t-\sum_{j=0}^{n-1}T(\phi^{j}\mathbf{x})<T(\phi^{n}\mathbf{x}).
\end{align*}


We can define invariant measures based on the special coordinates
$(\mathbf{x},\tau,\mathbf{l})$ (Gaspard 7.20-7.21):
\begin{align*}
\left\langle A(\mathbf{X})\right\rangle  & =\sum_{\mathbf{l}\in\mathcal{L}}\int\mu(d\mathbf{x}d\tau)A(\mathbf{x},\tau,\mathbf{l})\\
 & =\sum_{\mathbf{l}\in\mathcal{L}}\frac{1}{\vert\mathcal{P}\vert}\int_{\mathcal{P}}d\mathbf{x}\int_{0}^{T(\mathbf{x})}\frac{d\tau}{\left\langle T\right\rangle _{\nu}}A(\mathbf{x},\tau,\mathbf{l})\\
 & =\sum_{\mathbf{l}\in\mathcal{L}}\frac{1}{\left\langle T\right\rangle _{\nu}}\left\langle \int_{0}^{T(\mathbf{x})}d\tau A(\mathbf{x},\tau,\mathbf{l})\right\rangle _{\nu}
\end{align*}
where the subscript $\nu$ denotes integration over the volume of
the {\Poincare} section.

Now with some quantities defined let us proceed to the fourier transforms
of the quantities we are interested. Define the projection operators by
\begin{align*}
\hat{E}_{\mathbf{k}} & =\sum_{\mathbf{R}_{n}}e^{-i\mathbf{k}\cdot\mathbf{R}_{n}}\hat{P}_{\{\varepsilon\vert\mathbf{R}_{n}\}},
\end{align*}
in terms of spatial translation operators
\[
\hat{P}_{\{\varepsilon\vert\mathbf{R}_{n}\}}\phi(\mathbf{r})=\phi(\mathbf{r}+\mathbf{R}_{n}),
\]
with $\mathbf{R}_{n}$ is the Bravais lattice vector, and $\mathbf{k}\in\mathcal{B}$,
the first Brillouin zone. For an infinite lattice $\mathbf{k}$ is
continuous. We first needs to check and orthogonality and completeness
of the projection operators (Gaspard 7.30-7.32) :
\begin{align*}
\hat{E}_{\mathbf{k}}\hat{E}_{\mathbf{k^{\prime}}} & =\sum_{\mathbf{R}_{n}}e^{-i\mathbf{k}\cdot\mathbf{R}_{n}}\hat{P}_{\{\varepsilon\vert\mathbf{R}_{n}\}}\sum_{\mathbf{R}_{m}}e^{-i\mathbf{k^{\prime}}\cdot\mathbf{R}_{m}}\hat{P}_{\{\varepsilon\vert\mathbf{R}_{m}\}}\\
 & =\sum_{\mathbf{R}_{n}\mathbf{R}_{m}}e^{-i(\mathbf{k}\cdot\mathbf{R}_{n}+\mathbf{k}^{\prime}\cdot\mathbf{R}_{m})}\hat{P}_{\{\varepsilon\vert\mathbf{R}_{n}+\mathbf{R}_{m}\}}\\
 & =\sum_{\mathbf{R}_{m}}\sum_{\mathbf{R}_{l}}e^{-i(\mathbf{k}\cdot(\mathbf{R}_{l}-\mathbf{R}_{m})+\mathbf{k}^{\prime}\cdot\mathbf{R}_{m})}\hat{P}_{\{\varepsilon\vert\mathbf{R}_{l}\}}\\
 & =\sum_{\mathbf{R}_{m}}e^{-i(\mathbf{k}^{\prime}-\mathbf{k})\cdot\mathbf{R}_{m}}\sum_{\mathbf{R}_{l}}e^{-i\mathbf{k}\cdot\mathbf{R}_{l}}\hat{P}_{\{\varepsilon\vert\mathbf{R}_{l}\}}\\
 & =\hat{E}_{\mathbf{k}}\sum_{\mathbf{R}_{m}}e^{-i(\mathbf{k}^{\prime}-\mathbf{k})\cdot\mathbf{R}_{m}}\\
 & =\hat{E}_{\mathbf{k}}\vert\mathcal{B}\vert\delta(\mathbf{k}^{\prime}-\mathbf{k})
\end{align*}

\begin{align*}
\frac{1}{\vert\mathcal{B}\vert}\int_{\mathcal{B}}\hat{E}_{\mathbf{k}}d\mathbf{k} & =\frac{1}{\vert\mathcal{B}\vert}\sum_{\mathbf{R}_{n}}\hat{P}_{\{\varepsilon\vert\mathbf{R}_{n}\}}\int_{\mathcal{B}}e^{-i\mathbf{k}\cdot\mathbf{R}_{n}}d\mathbf{k}\\
 & =\frac{1}{\vert\mathcal{B}\vert}\sum_{\mathbf{R}_{n}}\hat{P}_{\{\varepsilon\vert\mathbf{R}_{n}\}}\vert\mathcal{B}\vert\delta_{\mathbf{R}_{n},0}\\
 & =\hat{P}_{\{\varepsilon\vert0\}}\\
 & =\hat{I}.
\end{align*}

to be continued.

\item[2016-05-07 Predrag]                                       \toCB
Reading on
`fundamental domain',
`chamber',
`M\"obius triangle',
`Voronoi cell',
`primitive cell',
`Wigner-Seitz cell',
`orbifold',
`signature',
`reflection group',
`Coxeter group',
`triangle group',
`bisected hexagonal tiling',
`',
$\cdots$

% https://en.wikipedia.org/wiki/Fundamental_domain
%
Given a topological space and a group acting on it, the images of a
single point under the group action form an \emph{orbit} of the action. A
\emph{fundamental domain} (also called fundamental region) is a subset of
the space which \emph{contains exactly one point} from each of these
orbits.
Typically, a fundamental domain is required to be a connected subset with
some restrictions on its boundary, for example, smooth or polyhedral. The
images of a chosen fundamental domain under the group action then tile
the space. One general construction of fundamental domains uses Voronoi
cells.

Example: for reflection in a hyperplane: an orbit is either a set of 2
points, one on each side of the hyperplane, or a single point in the
hyperplane; the fundamental domain is a half-space bounded by that
hyperplane.

Example: for discrete translational symmetry in three directions: the
orbits are translates of the lattice; the fundamental domain is a
\emph{primitive cell} which is e.g. a parallelepiped, or a
\emph{Wigner-Seitz cell}, also called \emph{Voronoi cell/diagram}. In the
case of translational symmetry combined with other symmetries, the
fundamental domain is part of the primitive cell. For example, for
wallpaper groups the fundamental domain is a factor 1, 2, 3, 4, 6, 8, or
12 smaller than the primitive cell.

    If one identifies equivalent points of a symmetric pattern, the
    resulting space is called an orbifold. In Thurston's
    \HREF{http://www.geom.uiuc.edu/docs/doyle/mpls/handouts/handouts.html}
    {original definition}, an orbifold is a quotient of a 2\dmn\ space
    with a finite discrete group. Conway has generalized this to 3- and
    4-dimensions\rf{CBG16}, but I have not found anything about higher
    dimensions.

In string theory, the word ``orbifold" has a slightly new meaning.

% https://en.wikipedia.org/wiki/Orbifold
%
For mathematicians, an orbifold is a generalization of the notion of
manifold that allows the presence of the points whose neighborhood is
diffeomorphic to a quotient of $\reals^n$ by a \emph{finite} group, i.e.
$\reals^n/\Group$.

In physics, the notion of an orbifold usually describes an object that
can be globally written as an orbit space $M/\Group$ where $M$ is a manifold
(or a theory), and $G$ is a group of its isometries (or symmetries) -- not
necessarily all of them.

A theory defined on an orbifold becomes singular near the fixed points of
$\Group$.

When the orbifold group $\Group$ is a discrete group, and it has fixed
points, then these usually have conical singularities, because
$\reals^n/Z_k$ has such a singularity at the fixed point of $Z_k$.

\item[2016-05-07 Predrag] From Tim Perutz
\HREF{https://www.ma.utexas.edu/smmg/archive/2012/Perutz/PerutzSlides.pdf}
{slides}, a very brief glimpse of Conway, Burgiel and
Goodman-Strauss\rf{CBG16}:

Symmetry operations that leave a plane pattern unchanged:
\begin{description}
  \item[Rotation] rotate by an angle about a center.
  \item[Reflection] mirror reflection across a line
  \item[Glide] reflect across a line, then translate in the
      direction of that line. Think: footprints!
\end{description}
A sequence of two reflections can be a rotation.

Patterns are classified by their \emph{orbifolds} and \emph{signatures}.
In our case, the pattern is a hexagonal tiling, and orbifold is the
fundamental domain. Thurston's signature in that case is $*632$; see
p.~310 in Thurston's \HREF{http://library.msri.org/books/gt3m/PDF/13.pdf}
{notes}.

The \emph{orbifold} of a pattern is the geometric object we get from the
plane when we regard two points to be equal if they are related by
a symmetry of the pattern.

The \emph{signature} of an orbifold describes the features you would need
to add to make it.

$*$ means you punch a hole, so as to make a boundary, etc - see the slides.
It is far from obvious, and it works only in $2D$.

Centers of rotation give sharp points like cones.

% https://en.wikipedia.org/wiki/Reflection_group
%
A \emph{reflection group} is a discrete group which is generated by a set of
reflections of a finite-dimensional Euclidean space. The symmetry group
of a regular polytope or of a tiling of the Euclidean space by congruent
copies of a regular polytope is necessarily a reflection group.
Reflection groups also include Weyl groups and crystallographic Coxeter
groups.

A finite reflection group is a subgroup of the general linear group of E
which is generated by a set of orthogonal reflections across hyperplanes
passing through the origin.

In two dimensions, the finite reflection groups are the dihedral groups,
which are generated by reflection in two lines that form an angle of
$2\pi/n$ and correspond to the Coxeter diagram $I_2(n)$.

Infinite reflection groups include the frieze groups and the wallpaper
groups (our example is is $*632$).

H.S.M. Coxeter\rf{Coxeter34,Coxeter35}: The reflections in the faces of a
fixed fundamental ``chamber'' (our ``fundamental domain'') are generators
$r_i$ of a reflection group of order 2. All relations between them follow
from
\[
    (r_i r_j)^{c_{ij}}=1
\]
expressing the fact that the product of the reflections $r_i$  and $r_j$
in two hyperplanes $H_i$  and $H_j$ meeting at an angle $\pi/c_{ij}$ is a
rotation by the angle $2\pi/c_{ij}$ fixing the subspace $H_i \cap H_j$ of
codimension 2.

% https://en.wikipedia.org/w/index.php?title=Triangle_group&action=edit
%
A \emph{triangle group} is a group that can be realized geometrically by
sequences of reflections across the sides of a triangle. Each triangle
group is the symmetry group of a tiling of the Euclidean plane by
congruent triangles, a fundamental domain for the action
 (the triangle defined by the lines of reflection), called a
M\"obius triangle.
A triangle group is a reflection group with a  group presentation
\beq
\Delta(l,m,n) = \langle a,b,c
\mid a^{2} =  b^{2} = c^{2} = (ab)^{l} = (bc)^{n} = (ca)^{m} =  1 \rangle
\,.
\ee{TriangleGrPresent}
An abstract group with this presentation is a Coxeter group with three
generators.

The triangle group is the infinite symmetry group of a certain
tessellation (or tiling) of the Euclidean plane by triangles. Up to
permutations, the triple (l, m, n) is one of the triples (2,3,6),
(2,4,4), (3,3,3). The corresponding triangle groups are instances of
wallpaper groups. Our case is the ``bisected hexagonal tiling'' (2,3,6).

% https://people.math.osu.edu/davis.12/papers/lectures%20on%20orbifolds.pdf
%
{\em Reflectofolds}

% https://www.ime.usp.br/~gorodski/ps/orbit-spaces-revision2016.pdf
%
A Riemannian orbifold is called a Coxeter orbifold if all local groups
are Coxeter groups acting as reflection groups on the corresponding
tangent spaces (Davis\rf{Davis08,Davis10} calls them reflectofolds,
\ie, orbifolds coming from groups generated by reflections).

% http://www.wikiwand.com/en/Orbifold
%
orbifold fundamental group



\end{description}
