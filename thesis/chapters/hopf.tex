% hopf.tex, poetry from nsf06am
% $Author$ $Date$

% Predrag                   oct 10 2006
% Predrag                   sep 13 2005

% \section{Dynamicist's vision of turbulence}
% \label{s:hopf}
% \file{rescued from nsf06am/TEX/hopf.tex Oct 10 2006}

%%%%%%%%%%%%% bring these from y-lan/bibtex or?
% "cycprl"
% "Ezr91"
% "crete03"
% "LanDescent"
%
%%%%%%%%%%%%% bring these form gibson/bibtex
% `hopf48' on page 858 undefined on input line 26.
% `FNSTks85' on page 858 undefined on input line 45.
% `Aubry88' on page 859 undefined on input line 68.
% `ku' on page 859 undefined on input line 71.
% `ZG96' on page 859 undefined on input line 74.
% `Lan:Thesis' on page 860 undefined on input line 172.



% We will focus on:
% (1)
% {\em elucidation of the {\statesp} flow topology and the symbolic
% dynamics for intermediate system dimensions.}

\PC{as soon as you approve an edit marked ``PCedit\{\PCedit{edited text}\}'',
    please remove the ``PCedit\{...\}'' brackets}
This thesis is part of a wider effort\rf{chfield} to describe
turbulence from a dynamical systems perspective that goes back
to the seminal paper of Hopf\rf{hopf48}. The relation of
dynamics to turbulence underlies many fundamental developments
in dynamical systems theory, from the very (re)discovery of
chaos by Lorenz\rf{lorenz} to the Ruelle-Takens\rf{ruell71}
view of turbulence, to the work on inertial
manifolds\rf{constantin_integral_1989} of partial differential
equations (PDEs). The emphasis here is not on the transition to
turbulence or on the derivation of reduced models of a partial
differential equation. On the contrary we ask: For a given
system, with given boundary conditions, which we are able to
numerically simulate to sufficient accuracy
to resolve its finest features, how do we develop a dynamical
description? Hopf's answer\rf{hopf48}
    \ES{Is this reference accurate? In
    Christiansen \etal\ an anecdotal reference is given.
    {\bf PC:} Read it. You destroyed Lan's paper copy, but I
    put a PDF in ChaosBook.org/library}
is to consider the dynamics of a PDE not as the evolution of
snapshots of the underlying field but as dynamics on an
$\infty$-dimensional \statesp\ in which every point
corresponds to a state of the system. In this space a generic turbulent
trajectory visits neighborhood of a ``regular'' solution for a while, then
switches to another one, and so on. For any given system,
parameter values and boundary conditions there are two
ingredients to implementing this vision: (a) the geometry of
the \statesp\ and (b) the associated natural measure, \ie, the
likelihood that asymptotic dynamics visits a given \statesp\
region.

The first successful quantitative implementation of Hopf's
vision for a spatially extended system, to the best of the
author's knowledge, can be found in
Christiansen~\etal\rf{Christiansen97}. They study \KS\ system,
a dissipative PDE in one spatial dimension, as one of the
simplest systems that exhibits features reminiscent of fluid
turbulence (see \refchap{chap:KSe} for details). The ``regular''
solutions in Hopf's picture were realized as a set of periodic
orbits, embedded into the attractor and ordered hierarchically.
Shorter orbits provided the basic building blocks of the
attractor, while longer ones contributed corrections. The
quantitative framework within which those periodic orbits were
used to approximate the natural measure and calculate
``observable'' quantities, such as Lyapunov exponents and
escape rates, was that of periodic orbit theory\rf{DasBuch}\PublicPrivate{}{,
briefly summarized here in \refappe{chap:POT}}. This
investigation was continued for a ``more turbulent'' \KS\
system by Y.~Lan and Cvitanovi\'c\rf{LanThesis,lanCvit07}.

Recently, \statesp\ of moderate Reynolds number wall bounded
shear flows became experimentally\rf{science04} and
computationally\rf{KawKida01,FE03,WK04,Visw07b,GHCW07}
accessible. The charting of Navier-Stokes \statesp, for
specific boundary conditions, with \eqva, \reqva\
and heteroclinic connections has provided the basic elements of
the geometry of the turbulent flow and there is hope that it
will eventually lead to approximation of the natural measure
using a set of ``regular'' solutions.
