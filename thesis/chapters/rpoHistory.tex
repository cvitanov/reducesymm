\renewcommand{\inputfile}{\version\ - edited 2007-03-11 rpoHistory}
% $Author$ $Date$

% \section{{\Rpo s} - a brief history}
% Predrag created file              jul 3 2006

The KS equation  \refeq{ks} is time-translationally invariant,
and
space-translationally invariant
under the 1-$d$ Lie group of $O(2)$ rotations: if
$u(x,t)$ is a solution, then $u(x+d,t)$ is an equivalent
solution for any $-L/2 < d \leq L/2$.
As a result,
KS can have \rpo\ solutions with nonzero shift
\beq
u(x+d,\period{}) = u(x,0)
\,.
\ee{KSrpos}
where $\period{}$ is the period.


{\Rpo s} were introduced by Poincar\'e in his study of
the 3-body problem\rf{ChencinerLink,rtb}.
% says Chenciner\rf{ChencinerLink}
Consider motion of a test particle of mass
$\mu \ll 1$ in the
restricted three-body problem,
under the
influence of the gravitational force of two heavy bodies with masses $1$ and
$\mu \ll 1$ fixed at $(-\mu,0)$ and $(1-\mu,0)$. \Reqva\ of this problem
are known as the Lagrange points. They are stationary in
the co-rotating frame, but
in the inertial frame they execute circular motions.
Similarly, in appropriate co-rotating frame
{\rpo s} are periodic orbits,
but in the inertial frame their trajectories
are quasiperiodic.
The \reqva\ and \rpo s
respectively arise from
\eqva\ and periodic solutions of Hamiltonians reduced by symmetries.
They arise in dynamics of many physical systems
with continuous symmetries, such as motions of rigid bodies, gravitational
$N$-body problems, molecules and nonlinear waves.
A striking recent application of \rpo s has been the discovery
of ``choreographies" of $N$-body problems\rf{McC7,McC8,McC}.
%add \PC{some McCord references}

Lan has some \reqva\ (travelling waves) for KS in his
thesis\rf{LanThesis}, %http://chaosbook.org/projects/theses.html
 and for complex LG in a paper on ``MAWs".
Viswanath\rf{ViswanathPC06} % arXiv.org/physics/0604062
found them in the plane Couette problem.

Here we determine
several \eqva\ and many \rpo s for
KS in a periodic cell of size $L=22$.
