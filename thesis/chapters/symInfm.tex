\renewcommand{\inputfile}{\version\ - Predrag edited 2009-08-23 symInfm.tex}
% siminos/thesis/chapters/symInfm.tex
% $Author$ $Date$

% Predrag                           Aug 22 2009
%       extracted from wilczak/blog/flow.tex

\PCedit{

\subsection{Equivariance under infinitesimal transformations}
% \subsection{$\SOn{2}$ equivariance}

    \PC{This subsection belongs to \refchap{chap:Symmetry},
    as the second, infinitesimal rotations part of
    \refsect{sec:symIntro} (where the first part is the
    statement of equivariance under global rotations). Do not
    see where - Lie algebras seem not to be discussed.
    Checking equivariance as a condition on Lie algebra is
    much easier than checking it for global, finite angle
    rotations, especially for non-trivial Lie groups.}
    %
A flow $\dot{x}= \vel(x)$ is equivariant under % an operation
$\LieEl$ if
\beq
\dot{\ssp}=\LieEl^{-1} \, \vel(\LieEl \, \ssp)
\,.
\ee{eq:FiniteRot}
An element of a compact Lie group, continuously connected to identity
% relating different \statesp\ points by a
% linear continuous symmetry operation
can be written as
\beq
\LieEl(\theta)=e^{\gSpace \cdot \Lg }
\,,
\ee{FiniteRot}
where
$\gSpace \cdot \Lg = \sum \theta_a \Lg_a,\; a=1,2, \cdots, N$
is an Lie algebra element, $\Lg_a$ is a basis set of
antihermitean Lie algebra generators, and $\theta_a$ are the
parameters of the transformation. For an infinitesimal
transformation,
% $\theta \ll 1$,
$ %\[
\LieEl(\theta)=1+ \gSpace \cdot \Lg + \cdots
\,,
$ %\]
the statement of equivariance \refeq{eq:FiniteRot} becomes
\[
\dot{x}=(1-\gSpace \cdot \Lg) \, \vel(x+\gSpace \cdot \Lg \, \ssp) + \cdots
       =\vel(\ssp)- \gSpace \cdot \left(
            \Lg \, \vel(\ssp) - \frac{d\vel}{d\ssp} \, \Lg \ssp
                     \right)  + \cdots
\,.
\]
The $\dot{x}$ and $\vel(x)$ cancel, and $\theta_a$ are
arbitrary. We are left with the infinitesimal transformations
version of the equivariance condition \refeq{eq:FiniteRot}:
\beq
  \left(
    \Lg_a  - (\Lg_a \ssp) \cdot \frac{\partial}{\partial \ssp}
  \right) \vel(\ssp) =
  (\Lg_a)_{ij} \vel_j(\ssp) - \Mvar_{ik} (\Lg_a)_{kj} \ssp_j =0
  \,.
\ee{inftmInv}
where $\Mvar = \frac{\pde \vel}{\pde x}$ is the \stabmat\ \refeq{stabMat}.
%    \PC{to Vaggelis: is the equivariance condition \refeq{inftmInv}
%    anyplace in the thesis, THE article, drafts of articles,
%    \texttt{Mathematica} notebooks or among Kalli's diapers?
%    }
$  {\cal L}_g \vel(x) = \left(\Lg_a  - (\Lg_a \ssp) \cdot
\frac{\partial}{\partial \ssp} \right) \vel(\ssp) $ is the {\em
Lie derivative} of the dynamical flow field $\vel$ along the direction
of the infinitesimal group-rotation induced flow $\Lg_a \ssp$.
    \PC{maybe say this too:
    ``\refeq{inftmInv} is the {\em Lie bracket} (or the {\em
    Poisson bracket}\rf{arnold89}) of the two flows.''
    }
The Lie derivative measures how much the
flows induced by the two vector fields commute with each
other; they commute if the Lie derivative
    \PC{perhaps say ``Lie bracket''}
vanishes. % as is the case in \refeq{inftmInv}.
\PC{state this, or refer to similar statement made elsewhere,
    following \refref{RumSch01}:
   ``The equivalence classes
   $\Group \, \ssp = \{  \LieEl \, \ssp | \LieEl \in \Group \}$
   are manifolds called the group orbits.
   We regard the system $\{\pS,f\}$ as a system
   $\{\hat{\pS},\hat{f}\}$ on the set
   of group orbits, \ie\ the orbit
   space $\pS/\Group$. The system $\{\hat{\pS},\hat{f}\}$
   on the orbit space is called the reduced system.''

   Also, might consider this change of nomenclature:
   `trajectory' will her be understood in the dynamical sense,
   both in the full and in the reduced (orbit) \statesp,
   while `orbit' will refer exclusively to `group orbit.'
   `Equilibrium' will refer to fixed point of a map as well,
   with `fixed point' reserved to the group action only.
   }

\begin{example}% SO{n} on R^5 equivariance of \CLf
That \CLf\ \refeq{eq:CLeR} is equivariant,
% $\vel(\ssp)= \LieEl^{-1} \, \vel(\LieEl \, \ssp)$
under $\SOn{2}$ rotations \refeq{eq:RotCLe5d} can be checked
by substituting the antihermitian Lie algebra generator,
  \beq
 \Lg =   \left(\barr{ccccc}
    0  &  1 & 0  &  0 & 0  \\
   -1  &  0 & 0  &  0 & 0 \\
    0  &  0 & 0  &  1 & 0  \\
    0  &  0 &-1  &  0 & 0 \\
    0  &  0 & 0  &  0 & 0
    \earr\right)
 \,,
 \label{ZMgen}
 \eeq
    \PC{is the sign standard?}
and the \stabmat\ for \CLf\ \refeq{eq:CLeR},
  \beq
\Mvar =
  \left(\barr{ccccc}
    -\sigma    	& 0 		& \sigma & 0    &  0 \\
	0 	& -\sigma       & 0      & \sigma   &  0 \\
	\RerCLor-z  &     -\ImrCLor      & -1     & -e & -x_1 \\
	\ImrCLor     & \RerCLor-z       	& e  	& -1       & -x_2 \\
	y_1     & y_2           & x_1    & x_2      & -b
    \earr\right)
\,,
  \ee{CLeStabMat}
into the equivariance condition \refeq{inftmInv}.
\end{example}

    } %end \PCedit{
