% section: Numerically locating relative equilibria and periodic orbits

\subsection{Multipoint shooting for \rpo s}

Here we will present the one-parameter Lie group case. The modifications
for multi-parameter groups are rather obvious. Let the symmetry group $\Gamma$
parametrized by the real parameter $\theta$\ES{Should compactness be emphasized?}. 
Assume that we have an initial guess for a \rpo\ of period 
$T_p$ and phase shift $\phi_p$. 
Let the guess be given as $N$ initial conditions $x_i,\, i=1\ldots N$ 
for each segment of the RPO, along with the flight times $T_i$, such that $\sum_i^N T_i = T_p$, and the
phase shift $\phi_p$. For the true \rpo\ we have
\bea
	f^{\tilde{T}_i}(\tilde{x}_i) & = & \tilde{x}_{i+1}\,,\  i=1,\ldots N-1,\continue
	R(\tilde{\phi}_p)f^{\tilde{T}_N}(\tilde{x}_N) & = & \tilde{x}_{1}\,.
	\label{eq:rpoCond}
\eea

Assuming that our guess is in the linear neighborhoud of the \rpo\ we can Taylor expand \refeq{eq:rpoCond} around our guess to linear order in the small quantities $\Delta x_i=\tilde{x}_i-x_i,\, \Delta T_i=\tilde{T}_i-T_i,\, \Delta \phi_p=\tilde{\phi}_p-\phi_p$ to get
\bea
	J^{T_i}(x_i)\Delta x_i + v_{T_i}\Delta T_i -\Delta x_{i+1}& = & x_{i+1}-f^{T_i}(x_i)\continue
	R(\phi_p)J^{T_N}(x_N)\Delta x_N + R(\phi_p)v_{T_N}\Delta T_N +\Lg R(\phi)f^{T_i}(x_i) -\Delta x_1 & = & x_{1}-R(\phi_p)f^{T_N}(x_N)\,,
	\label{eq:rpoCond}
\eea
where $v_{T_i}$ is shortcut for $v$ evaluated at ${f^{T_i}\left(x_i\right)}$ and $\Lg$ denotes the Lie algebra generator of the group. Interpreting $\Delta x_i,\, \Delta T_i,\, \Delta \phi_p$ as corrections to our guess solution we iteratively improve our approximation
of $\tilde{x}_p$. 
To overcome the difficulties associated
with the two unit eigenvalues of $R(\tilde{\phi}_p)J^{\tilde{T}_p}(\tilde{x})$ we impose 
the conditions
\bea
	v(x_i)\cdot\Delta x_i  &=& 0\,, \label{eq:transpV}\\
	\left(\Lg x_N\right) \cdot \Delta x_N &=& 0\,. \label{eq:transpLie}
\eea

Conditions \refneq{eq:transpV} assures that the correction will be transverse to the eigendirection associated
with time translational invariance, while condition \refneq{eq:transpV} prohibits correction along the direction
of infinitesimal group action.




