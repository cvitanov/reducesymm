% section: Numerically locating relative equilibria and periodic orbits

\subsection{Multipoint shooting for \rpo s}

Here we will present the one-parameter Lie group case. The modifications
for multi-parameter groups are rather obvious. Let the symmetry group $\Gamma$
parameterized by the real parameter $\theta$\ES{Should compactness be emphasized?}.
Assume that we have an initial guess for a \rpo\ of period
$T_p$ and phase shift $\phi_p$.
Let the guess be given as $N$ initial conditions $x_i,\, i=1\ldots N$
for each segment of the \rpo, along with the flight times $T_i$, such that $\sum_i^N T_i = T_p$, and the
phase shift $\phi_p$. For the true \rpo\ we have
\bea
	f^{\tilde{T}_i}(\tilde{x}_i) & = & \tilde{x}_{i+1}\,,\  i=1,\ldots N-1,\continue
	R(\tilde{\phi}_p)f^{\tilde{T}_N}(\tilde{x}_N) & = & \tilde{x}_{1}\,.
	\label{eq:rpoCond}
\eea

Assuming that our guess is in the linear neighborhood of the
\rpo\ we can Taylor expand \refeq{eq:rpoCond} around our guess
to linear order in the small quantities $\Delta
x_i=\tilde{x}_i-x_i,\, \Delta T_i=\tilde{T}_i-T_i,\, \Delta
\phi_p=\tilde{\phi}_p-\phi_p$ to get
    \PC{I do not like ${\tt Lg} = \mathfrak{g}$ as the notation for
        the Lie algebra generator - it confuses it with the finite
        action group element $g$. Something like
        ${\tt Lg} = \mathfrak{a}$ would make more sense. Are you
        following some reputable literature with this notation?}
\bea
	J^{T_i}(x_i)\Delta x_i + v_{T_i}\Delta T_i -\Delta x_{i+1}& = & x_{i+1}-f^{T_i}(x_i)\continue
	R(\phi_p)J^{T_N}(x_N)\Delta x_N + R(\phi_p)v_{T_N}\Delta T_N +\Lg R(\phi_p)f^{T_N}(x_N)\Delta \phi -\Delta x_1 & = & x_{1}-R(\phi_p)f^{T_N}(x_N)\,,
	\label{eq:rpoCond}
\eea
where $v_{T_i}$ denotes $v$ evaluated at ${f^{T_i}\left(x_i\right)}$ and $\Lg$ denotes the Lie algebra generator of the group. Interpreting $\Delta x_i,\, \Delta T_i,\, \Delta \phi_p$ as corrections to our guess solution we iteratively improve our approximation
of $\tilde{x}_p$.
To overcome the difficulties associated
with the two unit eigenvalues of $R(\tilde{\phi}_p)J^{\tilde{T}_p}(\tilde{x})$ we impose
the conditions
\bea
	v(x_i)\cdot\Delta x_i  &=& 0\,, \label{eq:transpV}\\
	\left(\Lg x_N\right) \cdot \Delta x_N &=& 0\,. \label{eq:transpLie}
\eea

Conditions \refeq{eq:transpV} assures that the correction will be transverse to the eigendirection associated
with time translational invariance, while condition \refeq{eq:transpV} prohibits correction along the direction
of infinitesimal group action. In matrix form we have the system
\scriptsize
\beq
    \left( \begin{array}{ccccc|ccccc|c}
        \mathbf{J}^{T_1}(x_1) 	&\hspace{-4pt}-\mathbf{1}	& 					&			& 						&v_{T_1}	&			&			&			&			&\\
				&\hspace{-4pt}\ddots	&\hspace{-4pt}-\mathbf{1}			&			& 						&		&\hspace{-10pt}\ddots	&			&			&			&\\	
				&			&\hspace{-4pt}\mathbf{J}^{T_i}(x_i)	& \hspace{-4pt}-\mathbf{1}	& 						& 		&			& \hspace{-10pt}v_{T_i}	&			&			&\\
				&			&					&\hspace{-4pt}\ddots	&\hspace{-4pt}-\mathbf{1}				&		&			&			&\hspace{-10pt}\ddots	&			&\\
			-\mathbf{1}	&			&					&			&\hspace{-4pt}R(\phi_p)\mathbf{J}^{T_N}(x_N)	&		&			&			&			&\hspace{-10pt}R(\phi_p)v_{T_N}	&\hspace{-4pt}\Lg R(\phi_p)f^{T_N}(x_N)\\ \hline
	v(x_1)			&			&					&			&						&		&			&			&			&			&\\
				&\hspace{-4pt}\ddots	&					&			&						&		&			&			&			&			&\\
				&			&\hspace{-4pt}v(x_i)			&			&						&		&			&			&			&			&\\
				&			&					&\hspace{-4pt}\ddots	&						&		&			&			&			&			&\\
				&			&					&			&	\hspace{-4pt}v(x_N)			&		&			&			&			&			&\\ \hline
				&			&					&			&	\hspace{-4pt}\Lg x_N 			& 		&			&			&			&			&
     \end{array}\right)
     \left(\begin{array}{c}
        \Delta x_1 \\
	\vdots\\
	\Delta x_i \\
	\vdots\\
	\Delta x_N \\
        \Delta T_1 \\
	\vdots	\\
	\Delta T_i \\
	\vdots	\\
	\Delta T_N \\	
	\Delta \phi
     \end{array}\right)
     =
     \left(\begin{array}{c}
	x_2-f^{T_1}(x_1)\\
	\vdots\\	
        x_{i+1}-f^{T_i}(x_i) \\
	\vdots\\
	x_{1}-R(\phi_p)f^{T_N}(x_N)\\
       	0    \\
	\\
	\\
	\vdots\\
	\\
	\\
	0
     \end{array}\right)\,,
     \label{eq:NewtonScheme}
\eeq
\normalsize
