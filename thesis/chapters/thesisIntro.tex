\subsection{\KS\ equation}

The studies of the \KS\ system
\beq
  u_t = F(u) = -{\textstyle\frac{1}{2}}(u^2)_x-u_{xx}-u_{xxxx}
    \,,\qquad   x \in [-L/2,L/2]
\ee{intro:ks}
in \refrefs{Christiansen97,lanCvit07,LanThesis} restricted the dynamics
to the space of antisymmetric functions $u(-x,t)=-u(x,t)$ by imposing
boundary condition $u(-L/2,t)=u(L/2,t)=0$, \cf\ \refsect{sec:KSe} for more
detail. This restriction eliminated the translational symmetry \KS\ equation
has when supplied with more general periodic boundary condition $u(x+L,t)=u(x,t)$ or
in an unbounded domain $x\in(-\infty,\infty)$. Even though working in the antisymmetric
subspace is mathematically and computationally convenient and the dynamics are far
from trivial for sufficiently large system size, many of the interesting solutions,
such as traveling waves are eliminated. We do not need to try hard to motivate the
study of traveling solutions, they are present in all fluid simulations and experiments
mentioned in \refsect{s:hopf} and ubiquitous in physics.

Going on with the discussion in the context of \KS\ system we observe that as soon as
we relax the antisymmetric boundary conditions and choose to work with periodic boundaries
\KS\ equation becomes invariant under the 1-$d$ Lie group of $O(2)$ rotations: if
$u(x,t)$ is a solution, then $u(x+d,t)$ is an equivalent
solution for any $-L/2 < d \leq L/2$.
As a result,
KS can have \rpo\ solutions with nonzero shift
\beq
u(x+d,\period{}) = u(x,0)
\,.
\ee{KSrpos}
where $\period{}$ is the period and recurrence becomes
relative: The periodic orbits that organized \statesp\ in
\refrefs{Christiansen97,lanCvit07,LanThesis} are now not the
only generic solutions, we are also faced with relative
periodic orbits, solutions that repeat them self up to a
translation. were introduced by Poincar\'e in his study of the
3-body problem\rf{ChencinerLink,rtb}. In PDE's they are also
known as modulated traveling waves and they have been found and
studied, for example in \KS\ equation\rf{BrKevr96}, Complex
Ginzburg-Landau equation\rf{lop05rel}, plane Couette
flow\rf{Visw07b}. A striking recent application of \rpo s has
been the discovery of ``choreographies'' of $N$-body problems%
\rf{CheMon00,CGMS02,McCordMontaldi}.

The main purpose of this thesis will be to investigate the role played by relative equilibria (traveling waves) 
and relative periodic orbits in the geometry of spatially extended systems with continuous symmetry. Following
the example of earlier work we concetrate on \KS\ equation as it provides a much simpler system for the illustration
of ideas than more realistic equations such as Navier-Stokes. Yet, we emphasize that we develop methods in
a way that can be applicable in other PDE's\ES{For the defense quote imaginary (but typical) KS paper: We have
demonstrated our method in the example of KSe but it completely general and applicable in any other PDE. Application to Navier-Stokes is left as an exercise to the reader.}. We will concentrate on a system size for which the dynamics
are chaotic but we will not get into the spatiotemporally chaotic regime. Therefore when the term ``turbulence'' is
used in this thesis for \KS\ equation it will be a synonym to chaos but will help  
to remind us that we deal with a spatially extended system. 

\subsection{Symmetry reduction}

In physics symmetry usually leads to simplification of a problem so it will sound as an oxymoron that it actually
complicates matters in our case. Indeed, in linear theories, such as quantum mechanics, symmetry is often exploited
through separation of variables. In Hamiltonian mechanics it leads to conserved quantities which can often be
directly exploited. For instance in the central force problem conservation of angular momentum fixes the plane
of motion. %In field theories Hamiltonians are constructed so that they respect a certain symmetry.
In general exploiting symmetry by identifying points in space or solutions 
related by a symmetry operation is the objective of 
symmetry reduction. The subject has a very long history in Hamiltonian mechanics and for general systems
and group actions it usually is highly non-trivial, see for example \refrefs{marsden_introduction_1999,marsden_hamiltonian_2007,cushman_global_1997}.

Here we will concentrate in the dissipative context and in particular we will consider high-dimensional truncations
of PDEs. The main problems we are facing are: 1) the high dimensionality of phase space, 2) the structure of
the phase space induced by the symmetry group action usually prevents reduction to be carried out globaly.
High dimensionality does not allow us to use a very powerful tool in symmetry reduction, Hilbert bases, 
see \refsect{sec:symRed}, since their determination seems computationally prohibitive for anything larger
than a ten-dimensional space\rf{ChossLaut00,gatermannHab}. The second issue demonstrates itself in different
ways, see \refsect{sec:symRed}, and we will have to overcome it since we are interested in understanding
how the attractor is organized globally.





