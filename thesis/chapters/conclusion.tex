%  conclusion.texHalcrow thesis
% $Author$ $Date$


As a turbulent flow evolves, every so often we
catch a glimpse of a familiar pattern. For any finite spatial
resolution, the flow approximately follows for a finite time
a pattern belonging to a finite alphabet of admissible fluid states,
represented here by a set of exact coherent structures.
Turbulent dynamics visualized in \statesp\ appears pieced together from
close visitations of exact coherent structures
connected by transient interludes.  This is plainly illustrated
by \reffig{f:bigbox}.  The larger cell is clearly tesselated by states
not dissimilar from those presented here.

For \KS\ \eqva, \reqva\ and
periodic solutions embody the vision of turbulence\rf{Hopf48}:
a repertoire of recurrent spatio-temporal
patterns explored by turbulent dynamics.
The new \eqva\ and \reqva\ that we present here
expand and refine this repertoire.

These orbits lend credence to our view of
turbulence as a walk through this set of patterns.
The heteroclinic connections that we present here are the initial steps in drawing
an atlas of \KS\ \statesp; close passages to \eqva\ form a coarse symbolic
dynamics (nodes of a Markov graph), and their heteroclinic connections are the directed links
connecting these nodes.

The emergence and disappearance of these heteroclinic connections can also be
diagnostic. For instance, in the Lorenz system a series of such bifurcations occur
as the Rayleigh number is increased\rf{jackson89}.
They mark changes
in the topology of \statesp.  For \KS, such bifurcations could be used
to mark the onset of turbulence.

Future work in this direction could serve to clarify such points.  It is still
not entirely clear what happens at the global bifurcations involved in the creation
and annihilation of these heteroclinic connections.  Furthermore,  the list of \eqva\ and
their heteroclinic connections we have found so far should by no means be considered to be exhaustive.
Further investigation of \KS\ for these as well as other geometries
will most likely turn up more \eqva\ and their heteroclinic connections.

Currently, a taxonomy for all of these myriad states eludes us.
To organize these states in a useful way requires
a deeper understanding of the connections between them.
From this work, we only see how the various states are related for a fixed geometry.

This connects to the outstanding issue of all studies undertaken
so far, which must be addressed in future work:
the small aspect cell periodicities imposed for computational convenience.
So far, all numerical
work has focused on spanwise-streamwise periodic cells barely large
enough to allow for sustained turbulence. Such small cells introduce dynamical
artifacts such as lack of structural stability, cell-size dependence of the
sustained turbulence states, and boundary-condition dependent coherent structures
unlike those observed in large aspect ratio experiments.
