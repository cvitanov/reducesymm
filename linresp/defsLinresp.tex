% reducesymm/linresp/defsLinresp.tex

% Predrag                       Jul 15 2014

%\usepackage{
%    amsmath,
%    amssymb,
%    latexsym,
%    float,
%    epsfig,
%    subfig
%            }

\usepackage[export]{adjustbox} % http://ctan.org/pkg/adjustbox

\ifPDF
    \bibliographystyle{apsrev4-1} %with DOI hyperlinks
    \usepackage{color} % dvips allows for colors
    \usepackage[colorlinks]{hyperref} %% hyperlinks
    % \usepackage[pdfauthor={D. Lippolis}%
    %           pdftitle={Noise},%
    %           pdfmark,colorlinks]{hyperref} %% hyperlinks
\else % prepare B&W file
    \bibliographystyle{apsrev}
\fi
    \usepackage{graphicx}
    \usepackage{ifthen}

%%%%%%%%%%%%%%%%%%%%%%% set up draft, hyperlinked %%%%%%%%%%%

\ifPDF % hyperlinked pdf, keep homepage flexible:
    \newcommand{\wwwcb}[1]{
                  {\tt \href{http://ChaosBook.org#1}
              {ChaosBook.org#1}}}
    \newcommand{\weblink}[1]{{\tt \href{http://#1}{#1}}}
    \newcommand{\HREF}[2]{{\href{#1}{#2}}}
    \newcommand{\arXiv}[1]{
              {\tt \href{http://arXiv.org/abs/#1}{arXiv:#1}}}
\else  %% prepare for postscript printing:
    \newcommand{\wwwcb}[1]{{\tt ChaosBook.org#1}}
    \newcommand{\weblink}[1]{{\tt #1}}
    \newcommand{\HREF}[2]{{#2}}
    \newcommand{\arXiv}[1]{ {\tt arXiv:#1}}
\fi

%%%%%%%%%%%%%%%%%%%%%% PRLett STYLE COMMENTS %%%%%%%%%%%%%%%%%%%
        \ifboyscout % if draft, display comments in text
   \renewcommand{\PCedit}[1]{{\color{blue}#1}}
   \renewcommand{\PC}[2]{\begin{quote}\PCedit{[#1 Predrag] #2}\end{quote}}
   \newcommand{\DLedit}[1]{{\color{red}#1}}
   \newcommand{\DL}[2]{\begin{quote}\DLedit{[#1 Domenico] #2}\end{quote}}
   \definecolor{darkgreen}{rgb}{0,.6,.25}
   \newcommand{\JMHedit}[1]{{\color{darkgreen}#1}}
   \newcommand{\JMH}[2]{\begin{quote}\JMHedit{[#1 Jeffrey] #2}\end{quote}}
   \newcommand{\toCB}{\marginpar{\footnotesize 2CB}}  % to compare with ChaosBook
   \newcommand{\inCB}{\marginpar{\footnotesize now in CB}}
        \else  % drop comments
   \renewcommand{\PC}[2]{}{}
   \newcommand{\PCedit}[1]{#1}
   \newcommand{\DL}[2]{}{}
   \newcommand{\DLedit}[1]{#1}
   \newcommand{\JMH}[2]{}{}
   \newcommand{\JMHedit}[1]{#1}
   \newcommand{\toCB}{}
   \newcommand{\inCB}{}
        \fi


%%%%%%%%%%%%%%% REFERENCING EQUATIONS ETC. %%%%%%%%%%%%%%%%%%%%%%%%%%%%%%%
\newcommand{\rf}     [1] {~\cite{#1}}
\newcommand{\refref} [1] {ref.~\cite{#1}}
\newcommand{\refRef} [1] {Ref.~\cite{#1}}
\newcommand{\refrefs}[1] {refs.~\cite{#1}}
\newcommand{\refRefs}[1] {Refs.~\cite{#1}}
\newcommand{\refeq}  [1] {(\ref{#1})}
\newcommand{\refeqs} [2]{(\ref{#1}--\ref{#2})}
\newcommand{\refpage}[1] {page~\pageref{#1}}
\newcommand{\reffig} [1] {figure~\ref{#1}}
\newcommand{\reffigs} [2] {figures~\ref{#1} and~\ref{#2}}
\newcommand{\refFig} [1] {Figure~\ref{#1}}
\newcommand{\refFigs} [2] {Figures~\ref{#1} and~\ref{#2}}
\newcommand{\reftab} [1] {table~\ref{#1}}
\newcommand{\refTab} [1] {Table~\ref{#1}}
\newcommand{\reftabs}[2] {tables~\ref{#1} and~\ref{#2}}
\newcommand{\refsect}[1] {sect.~\ref{#1}}
\newcommand{\refsects}[2] {sects.~\ref{#1} and \ref{#2}}
\newcommand{\refSect}[1] {Sect.~\ref{#1}}
\newcommand{\refSects}[2] {Sects.~\ref{#1} and \ref{#2}}
\newcommand{\refappe}[1] {appendix~\ref{#1}}
\newcommand{\refappes}[2] {appendices~\ref{#1} and~\ref{#2}}
\newcommand{\refAppe}[1] {Appendix~\ref{#1}}

%%%%%%%%%%%%%%% EQUATIONS %%%%%%%%%%%%%%%%%%%%%%%%%%%%%%%
\newcommand{\beq}{\begin{equation}}
\newcommand{\continue}{\nonumber \\ }
\newcommand{\nnu}{\nonumber}
\newcommand{\eeq}{\end{equation}}
\newcommand{\ee}[1] {\label{#1} \end{equation}}
\newcommand{\bea}{\begin{eqnarray}}
\newcommand{\ceq}{\nonumber \\ & & }
\newcommand{\eea}{\end{eqnarray}}

%%%%%%%%%%%%%%  Abbreviations %%%%%%%%%%%%%%%%%%%%%%%%%%%%%%%%%%%%%%%%
\newcommand{\etc}{{\em etc.}}       % etcetera in italics
\newcommand{\ie}{{i.e.}}            % APS

\newcommand{\statesp}{state space}
\newcommand{\Statesp}{State space}
\newcommand{\eqv}{equi\-lib\-rium}
\newcommand{\Eqv}{Equi\-lib\-rium}
\newcommand{\eqva}{equi\-lib\-ria}
\newcommand{\Eqva}{Equi\-lib\-ria}
\newcommand{\po}{periodic orbit}
\newcommand{\Po}{Periodic orbit}
\newcommand{\dmn}{\ensuremath{\,d}}  %  n-dimensional
\newcommand{\Fd}{spec\-tral det\-er\-min\-ant}
\newcommand{\fd}{spec\-tral det\-er\-min\-ant}
\newcommand{\stabmat}{stability matrix}     % stability matrix, velocity gradients
\newcommand{\Stabmat}{Stability matrix}     % Stability matrix
\newcommand{\stabmats}{stability matrices}
\newcommand{\monodromyM}{monodromy matrix} % monodromy matrix, Poincare cut
\newcommand{\MonodromyM}{Monodromy matrix} % monodromy matrix, Poincare cut
\newcommand{\optPart}{optimal partition}
\newcommand{\OptPart}{Optimal partition}
\newcommand{\Fokker}{Fokker-Planck}

\newcommand{\ssp}{\ensuremath{x}}    % state space point
\newcommand{\Mvar}{\ensuremath{A}}  % stability matrix
\newcommand{\ExpaEig}{\Lambda}
\newcommand{\Lyap}{\ensuremath{\lambda}}            %Lyapunov exponent
\newcommand{\cl}[1]{{n_{#1}}}   % discrete length of a cycle, Predrag
\newcommand{\msr}{{\rho}}               % measure
\newcommand{\SRB}{{\rho_0}}             % natural measure
\newcommand{\pS}{{\cal M}}          % symbol for state space space
\newcommand{\spaceAver}[1]{\left\langle {#1} \right\rangle}
\newcommand{\diffCon}{\ensuremath{D}}       % diffusion constant
\newcommand{\diffTen}{\ensuremath{\Delta}}  % diffusion tensor
\newcommand{\covMat}{\ensuremath{Q}}             % covariance matrix
\newcommand{\Lnoise}[1]{{\cal L}^{#1}}    % noisy evolution operator
\newcommand{\Lmat}[1]{{{\bf L}_{#1}}}      % evolution matrix
\newcommand{\orbitDist}{{z}}     % Langevin distance from orbit point
\newcommand{\transp}[1]{{#1}{}^\top}
\newcommand{\matId}{\ensuremath{{\bf 1}}}      % matrix identity
\newcommand{\Df}[1]{{f'_{#1}}}
\newcommand{\monodromy}{\ensuremath{M}}   % monodromy matrix, full Poincare cut
\newcommand{\tr}{\mbox{\rm tr}\,}
\newcommand{\cycle}[1]{\ensuremath{\overline{#1}}}

\newcommand{\MatrixII}[4]{\left(
\begin{array}{cc}
{#1}  &  {#2} \\
{#3}  &  {#4} \end{array} \right)}
