\documentclass[12pt]{letter}
\pagestyle{plain}

\begin{document}

% [REPLY MUST BE EITHER PLAIN TEXT OR IN A SINGLE PDF]

Dear Editor,

Please find attached our article, 
considered for publication in PRL.  
Following the assessment of the editors, who found our manuscript
suitable for a specialized journal,  
we would like to submit it to PRE.

Please find below our original submission letter.

Sincerely yours,\\
\phantom{~~~~~~~~~~~~~~~}Domenico Lippolis 


----------------------------------------------------------------------\\
Cover letter for first submission.\\
----------------------------------------------------------------------

Due to its sensitivity to initial conditions, a chaotic dynamical system
can be mainly characterized through its statistical properties, such as
long-time or state-space averages. The computation of the latter requires
knowledge of the so-called stationary distribution, the `natural density'
to which all initial densities asymptotically converge, that acts as a
weight function in all state space averages.

In this work, we present the `optimal partition hypothesis' for a
hyperbolic system, an algorithm that determines the finest attainable
state-space resolution of a $n-$dimensional chaotic system in the
presence of weak noise, to be then used for an efficient and accurate
estimate of the stationary distribution.

We thus fulfill the pledge of generalizing our {\em Phys. Rev. Lett.}
\textbf{104}, 014101 (2010) formalism and algorithms: that work applied only
to either one-dimensional or all-expanding (all-contracting)
$n$-dimensional systems. As shown by our numerical tests, we are able to
successfully tackle paradigmatic discrete-time models, and, most
importantly, we set the stage for further extensions to low- and
high-dimensional flows with turbulence as the ultimate goal, where direct
numerical simulations are computationally very costly.

We believe that the elegance of the formalism, together with the
originality of the method and its non-trivial implementation, as well as
its very promising perspectives, all very much fit the standards of
impact, innovation, and interest set by Physical Review Letters. We thus
hope you will help us communicate the present contribution to a wide
community of physicists.


\end{document}
