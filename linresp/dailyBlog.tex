% GitHub cvitanov/reducesymm/linresp/dailyBlog.tex

% Predrag                                       2015-10-16
	
\chapter{Research blog on linear response}
\label{c-DailyBlog}

\begin{description}

\item[2015-10-16 Predrag] We can write up the narrative starting with
    this file, in this folder, \texttt{reducesymm/linresp/}, with all
    stuff that does not belong to the public version bracketed by
    \texttt{ifboyscout}...\texttt{fi} . You can clip \& paste anything
    from here or from ChaosBook.org, if that saves you LaTeXing time.

\item[2015-10-18 Ben] Added figures to \texttt{linresp/figs/}
\begin{itemize}
  \item \texttt{tentmapexample.png}
    image of tent map with $\epsilon$ perturbation
  \item \texttt{numeric.png}
    plot of numerical and exact analytic calculations of leading
    eigenvalue for perturbed tent map, along with analytic
    calculation of leading eigenvalue for unperturbed map
  \item \texttt{logisticnumeric.png}
    analytic calculation of leading eigenvalue for perturbed
    logistic map
\end{itemize}

\item[2015-10-18 Predrag]
Have a glance at
\refrefs{AbrMaj08,Abramov09,Abramov10,Abramov12,Abramov12a,AbrKje13},
report here if you find/learn anything of interest.

We have to find and read Ruelle\rf{Ruelle12} and
Baladi papers on linear response. Baladi's\rf{BalTod15} in
{\em Linear response for intermittent maps} writes ``This is the first
time that a linear response formula for the SRB measure is obtained in
the setting of slowly mixing dynamics'',

``theme of linear response was explored in a few pioneering
papers\rf{Ruelle98,KKPW89,Ruelle96} in the setting of smooth hyperbolic
dynamics (Anosov or Axiom A)"

``It was soon realised that existence of a spectral gap is not sufficient
to guarantee linear response when bifurcations are present (see e.g.
\refrefs{Baladi07,BalSma08}). In the other direction, neither the
spectral gap nor structural stability is necessary for linear response,
as was shown by Dolgopyat\rf{Dolgopyat04} who obtained a linear response
formula for some rapidly mixing system."

``The intuition that a key sufficient condition is convergence of the sum
was confirmed by \rf{HaiMaj10} [Remark 2.4]"

``Another lesson of recent research [Ru1] \rf{Ruelle09,Baladi14,Baladi08}
[CD] on linear response is that understanding the singularities of the
SRB measure is essential."

She also refers to these:

%[B0]\rf{Baladi07}  [B1]\rf{Baladi08}  [B2]\rf{Baladi14}

 [BMS]{BMS} V. Baladi, S. Marmi, and D. Sauzin,
 {\it Natural boundary for the susceptibility function of generic
piecewise expanding unimodal maps,}
Ergodic Theory Dynam. Systems \textbf{10}
(2013) 1--24.

% [BS]\rf{BalSma08}

[BCV]{BomCasVar12} T. Bomfim, A. Castro, and P. Varandas,
\emph{Differentiability of thermodynamical quantities in
      non\--uni\-formly expanding dynamics,}
arXiv:1205.5361.

 [BT]{BT} H. Bruin and M. Todd,
 {\it Equilibrium states for potentials with
   $\sup \phi -\inf \phi < h_{top}(f)$,} Comm. Math. Phys.
 {\bf 283} (2008) 579--611.

 [CD]{CD} F. Contreras and D. Dolgopyat, {\it Regularity of absolutely
 continuous invariant measures for piecewise expanding unimodal maps,}
 arXiv:1504.04214.

% [Do]\rf{Dolgopyat04}

%[FT]{FT} J.M. Freitas and M. Todd, {\it Statistical stability of
%equilibrium states for interval maps,} Nonlinearity {\bf 22} (2009)
%259--281.
%
%[Go] {Go} S. Gou\"ezel, {\it Sharp polynomial estimates for the
%decay of correlations,} Israel J. Math. {\bf 139} (2004) 29--65.
%
%[Goth] {Goth} S. Gou\"ezel, {\it Vitesse de décorr\'elation et
%th\'eor\`emes limites pour les applications non uniform\'ement
%dilatantes,} PhD thesis, Orsay, 2004.

% [HM]\rf{HaiMaj10} [KKPW]\rf{KKPW89}

[Lu1]{Luca} V. Lucarini, D. Faranda, J. Wouters, and T. Kuna, {\it
Towards a general theory of extremes for observables of chaotic
dynamical systems,}
 J. Stat. Phys. {\bf 154} (2014) 723--750.

[Lu2]{Luca2} V. Lucarini et al., {\it Extremes and Recurrence in
Dynamical Systems,}
 John Wiley and Sons, 2015.
 % (305 p.).

%[Sa]{Sa} O. Sarig, {\it Subexponential decay of correlations,}
% Invent. Math. \textbf{150} (2002) 629--653.

% [Ru]\rf{Ruelle96}  [Ru0]\rf{Ruelle98}

%[Ru1]{Ru1} D. Ruelle, \emph{Structure and $f$-dependence of the
%{A.C.I.M.} for a unimodal map $f$ of {Misiurewicz type},} Comm.
%Math. Phys.,
%    \textbf{287}
%   (2009) 1039--1070.

% [Ru2]\rf{Ruelle09}

[Th1]{Th} M. Thaler, {\it Estimates of the invariant densities of
endomorphisms with indifferent fixed points,}
 Israel J. Math. \textbf{37} (1980) 303--314.

[Th2]{Tha00} M. Thaler,
\emph{The asymptotics of the {Perron-Frobenius} operator of a class of
interval maps preserving infinite measures,}
 Studia Math. \textbf{143} (2000) 103--119.

\item[2015-10-19 Ben] Singling out one more potentially interesting paper from the stack,
Baladi's\rf{BalSma08} {\em Linear response formula for piecewise expanding unimodal maps}.
This is a proof of an extension of Ruelle's linear response formula
{\em Differentiation of SRB states} to the nonuniformly hyperbolic case.

"[for $f_{t}$ a smooth one-parameter family of piecewise expanding
interval maps] ...Linear response is violated if and only if $f_{t}$ is
transversal to the topological class of $f$."

\item[2015-10-19 Predrag]
Some of this literature has to be referred to and explained in the
introduction, so we need to understand it conceptually (but no need of
working through the proofs):

``Topological class of $f$'' presumably means that all admissible orbits
are the same for each element of the class. Not sure one can define that
without having symbolic dynamics. ``$f_{t}$ is transversal to the
topological class of $f$" means that for any value of the parameter $t$
in an open interval around $t=0$ the dynamical system stays in the same
topological class.

``If $f$ is a sufficiently smooth uniformly hyperbolic diffeomorphism
restricted to a transitive attractor,
Ruelle\rf{Ruelle96,Ruelle98,Ruelle98a,Ruelle03} proved that $R (t)$ is
differentiable at $t = 0$.  In addition, Ruelle gave an explicit `linear
response formula' for $R'( 0 )$ , depending on $f_t$ only through its
linear part (the `infinitesimal deformation').''

I agree with that. I expect `linear response formulas' to be correct for such dynamical
systems, but they are rare and non-generic: some piecewise linear 1\dmn\
maps and weakly nonlinear cat maps. Baladi turns to the generic case next,
for which I expect no `linear response formula' can be defined, as any
change in parameter $t$ leads to creation and destruction of infinitely
many cycles:

``A much more difficult situation consists of studying nonuniformly
hyperbolic interval maps $f$, e.g. smooth unimodal maps. For some of
these maps, in particular those which satisfy the Collet-Eckmann
condition, there exists a unique SRB measure.  Two new difficulties are
that structural stability does not hold (in a rather drastic way), and
that $f_t$ will not always have an SRB measure even if $f$
has one.''

I think Smale's statement `a dynamical system is structurally stable'
means the same (and more precisely stated) as ``$f_{t}$ is transversal to
the topological class of $f$''.

`Collet-Eckmann condition' applies to smooth unimodal maps (think
parabola) and says that no matter how close a generic infinite-time orbit
comes to the critical point, it's Lyapunov exponent is strictly bounded
away from zero. So there are no stable limit cycles. Change the parameter
$t$ infinitesimally (in every open neighborhood there are parameter
values for which the limit is a stable periodic orbit) and - boom! - the
attractor is gone, just like in the ChaosBook H{\'e}non map example.

\item[2015-10-19 Predrag]
Young\rf{Young02} apparently defines something called `physical measure'.
Curious what that is.


\end{description}
