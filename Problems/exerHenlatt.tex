\Problems{exerHenlatt}{30nov2020}
% GitHub/reducesymm/Problems/exerHenlatt.tex called by kittens/sidney.tex
% $Author: predrag $ $Date: 2019-08-13 12:09:47 -0500 (Tue, 13 Aug 2019) $

% Predrag                                               2020-11-30

%%%%%%%%%%%%%%%%%%%%%%%%%%%%%%%%%%%%%%%%%%%%%%%%%%%%%%%%%%%%%%%%%%%
\exercise{H\'enon temporal lattice.}{\label{exer:tempHen}

1\dmn\ temporal H\'enon lattice (see
ChaosBook $\!$\HREF{http://ChaosBook.org/chapters/ChaosBook.pdf\#section.3.4}
{Example 3.5}) is given by a 3-term recurrence
\beq \nonumber
\ssp_{n+1} + a \ssp_n^2 - b\,\ssp_{n-1} = 1
\,.
\eeq
The parameter $a$ quantifies the ``stretching'' and $b$ quantifies the
``contraction''.

The single H\'enon map is nice because the system is a nonlinear
generalization of \templatt\ 3-term recurrence
\refeq{catMapNewt}, with no restriction to the unit hypercube XXX, but has
binary dynamics.

There is still a tri-diagonal {\jacobianOrb} $\jMorb$
\refeq{tempCatFixPoint}, but \refeq{Hessian} is now lattice state
dependent. Also, I beleive Han told me that
\refsect{s:Hill}~{\em {\HillDet}:
            stability of an orbit vs. its time-evolution stability}
block matrices derivation of Hill's formula does not work any more.
Neither does the `fundamental fact', as each lattice state's {\jacobianOrb}
is different, and presumably does not count periodic states, as there is
no integer lattice within the {\HillDet} volume.

Does the
\HREF{http://chaosbook.org/chapters/ChaosBook.pdf\#section.27.4} {flow
conservation} sum rule \refeq{Det(jMorb)eights} still works?

The assignement: Implement the variational searches for
periodic states in Matt's \HREF{https://github.com/farom57/Orbit-hunter}
{OrbitHunter}, find all periodic lattice states up to $n=6$.

(a) $a=1.4\,\;b=0.3$, compare with ChaosBook
\HREF{http://chaosbook.org/chapters/ChaosBook.pdf\#section*.550}
{Table~34.2}.

(b) For $b=-1$ the system is time-reversible or `Hamiltonian',
see
ChaosBook $\!$\HREF{http://ChaosBook.org/chapters/ChaosBook.pdf\#chapter.8}
{Example 8.5}.

Note also
\HREF{http://chaosbook.org/chapters/ChaosBook.pdf\#section.J.3}
{sect~A10.3} {\em H\'enon map symmetries}
and
\HREF{http://chaosbook.org/chapters/ChaosBook.pdf\#section.7.3}
{Exer.~7.2} {\em Inverse iteration method}.

The deviation of an approximate trajectory from the 3-term recurrence is
\beq \nonumber
v_n = \ssp_{n+1} - (1 - a \ssp_n^2 + b\,\ssp_{n-1})
\eeq
In classical mechanics force is the gradient of potential, which
Biham-Wenzel\rf{afind} construct as a cubic potential
\beq \nonumber
V_n = \ssp_n(\ssp_{n+1} - b\,\ssp_{n-1} - 1) + a \ssp_n^3
\eeq
With the cubic potential of a single H\'enon map we can start to look for
orbits variationally.


Compare with XXX
    } % end \exercise{exer:catMapGreenInf}
%%%%%%%%%%%%%%%%%%%%%%%%%%%%%%%%%%%%%%%%%%%%%%%%%%%%%%%%%%%%%%%%%%%%%%%%

%%%%%%%%%%%%%%%%%%%%%%%%%%%%%%%%%%%%%%%%%%%%%%%%%%%%%%%%%%%%%%%%%%%
%\exercise{XXX.}{\label{exer:XXX}
%XXX
%    } % end \exercise{exer:XXX}
%%%%%%%%%%%%%%%%%%%%%%%%%%%%%%%%%%%%%%%%%%%%%%%%%%%%%%%%%%%%%%%%%%%%%%%%

%%%%%%%%%%%%%%%%%%%%%%%%%%%%%%%%%%%%%%%%%%%%%%%%%%%%%%%%%%%%%%%%%%%
%\exercise{XXX.}{\label{exer:XXX}
%XXX
%    } % end \exercise{exer:XXX}
%%%%%%%%%%%%%%%%%%%%%%%%%%%%%%%%%%%%%%%%%%%%%%%%%%%%%%%%%%%%%%%%%%%%%%%%

    \ProblemsEnd
