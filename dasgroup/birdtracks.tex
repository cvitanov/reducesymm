% compile by  pdflatex blog; biber blog
% GitHub cvitanov/reducesymm/dasgroup/birdtracks.tex

% Predrag  created              Aug 7 2014
% notes for birdtracks.eu



\chapter{Birdtracks}
\label{c-birdtracks}


Enter here notes of general group-theoretic interest, perhaps for inclusion
into revisions of \wwwgt. The notes are in
\\
\HREF{https://github.com/cvitanov/reducesymm/}
{GitHub.com/cvitanov/reducesymm}. If you download it
\\
\texttt{> cd dasgroup/} \\
\texttt{> pdflatex blog} \\
For anything
technical, please do not email me, but let me give you permissions to edit
this GitHub repository. Then you can  edit directly into the GitHub version,
and let me know by email to \\
\texttt{dasgroup@mail.gatech.edu}\\
when you have \texttt{git push}ed something new to the server.



\section{Notes on Alcock-Zeilinger and Weigert}
\label{s-AlcZei16}

\newcommand{\FPic}[1]{\raisebox{-0.4\height}{\hspace{-0.27mm}\includegraphics{#1}\hspace{-0.27mm}}}
\newcommand*\circled[1]{\tikz[baseline=(char.base)]{
            \node[shape=circle,draw,inner sep=2pt] (char) {#1};}}
\newcommand{\diagram}[2][{}]{\pbox{\textwidth}{\includegraphics[#1]{{#2}}}}
\newcommand{\SUN}{\mathsf{SU}(N)}
\newcommand{\MixedPow}[2]{V^{\otimes
    #1}\otimes\left(V^*\right)^{\otimes #2}}
\newcommand{\Pow}[1]{V^{\otimes #1}}
\newcommand{\DAlg}[1]{\left(V^*\right)^{\otimes #1}}
\newcommand{\Lin}[1]{\mathrm{Lin}\left( #1 \right)}
\newcommand{\API}[1]{\mathsf{API}\left( #1 \right)}
\newcommand{\InvAlg}[1]{A\left[ S_{#1} \right]}
\newcommand{\Rsim}{\stackrel{\mathcal{R}}{\sim}}

\begin{description}
  \item[2016-12-08 Predrag to Heribert:]
My
\HREF{http://chaosbook.org/~predrag/papers/preprints.html\#FiniteFieldTheo}
{``finiteness conjecture''} is based on the observation that if internal
photons are collected into gauge invariant sets, each set contributes a
small, finite amount to what (off-mass shell) is usually assumed to be an
asymptotic series.
A gauge invariant set contributing to $(m+m'+k)$th order consists of $m$
photon ``strands" attached to the incoming electron, $m'$ photon ``strands"
attached to the outgoing electron, and $k$ photon ``strands" crossing the
external photon vertex. I do not have a direct method for evaluating a gauge
set; instead, it takes a few years and a PhD thesis to evaluate these sets.

These photon ``strands" have infrared divergences in individual diagrams,
but as one is evaluating the magnetic moment, their sums do not.

Do you envision using Wilson lines formulation possibly accounting for clouds
of soft photons crossing a QED vertex? Is there a direct calculation one
could do without perturbatively expanding the Wilson lines to the usual
individual multi-photon Feynman diagrams?

  \item[2016-12-02 Predrag]
My notes  on Alcock-Zeilinger and Weigert are in\\
\HREF{https://github.com/cvitanov/reducesymm/}
{GitHub.com/cvitanov/reducesymm/dasgroup/}.

  \item[2016-11-30 Predrag]
Your Sect.~4.2 Proof of Theorem 3 \emph{(generalized propagation rules)}
seems not to need Young tableaux to work. A streamlined derivation might be
to prove it for an individual transposition, the assemble whatever operator
you need from transpositions?

  \item[2016-12-02 Predrag]
You might consider folding your Mathematica codes into
\texttt{FormTracer}\rf{CyMiSt16} see \refsect{s-birdtrBlog}, entry of {\bf
[2016-12-10]}.

  \item[2016-11-30 Predrag]
Skype session with Heribert and Judy, about their 3 birdtracking preprints.
Got through the first one\rf{AlcZei16-1}.

\end{description}

\subsection{Physics motivation}
\label{s-AlcZei16-HEP}

% Physics motivation:


(read up on Larry McLarren propaganda)

Applications of these
tools in a QCD context where factorization invariably involves color
singlet projections of Wilson line correlators, see
several fields with possible
applications:

Marquet and Weigert\rf{MarWei10} {\em New observables to test the {Color
Glass Condensate} beyond the large-{$N_c$} limit}

Weigert\rf{Weigert03}
{\em Non-global jet evolution at finite {$N_c$}}

Falcioni \etal\rf{FGHMW14}
{\em Multiple gluon exchange webs}

Bomhof \etal\rf{BoMuPi06}
{\em The construction of gauge-links in arbitrary hard processes}

Since $\SUN$ is the gauge group of QCD, \Ypo s come into play through the
theory of invariants, which relates the irreducible representations of $\SUN$
over $\Pow{n}$ to the Young tableaux of size $n$\rf{Fulton97,Tung1985na}.
The lack of Hermiticity of \Ypo s  disqualifies them from the application to
QCD calculations: for applications the operators need to be Hermitian (hence
\refrefs{KeppSjo14,AlcZei16-2}) and all singlets are accounted for (hence
\refref{AlcZei16-4}).

Functional evolution evolution equation  for QCD cross sections in high
energy limit (Bjorken $x$ less than $10^{-2}$), as you push up energy make
more and more soft gluons, making the system highly nonlinear. Parton model
picture breaks down. BFKL pomeron equation is in Bjorken $x$, but distributions go
exponentially large; Weigert contributed to formulating the nonlinear
version.

{\em Color Glass Condensate} (within the standard model, only QCD does it):
\\

The Balitsky-\-JIMWLK (Jalilian-\-Marian-\-Iancu-\-McLerran-\-Weigert-\-Leonidov-\-Kovner)
is a tool to calculate the energy dependence of QCD observables at high
energies. Gluon  distribution  in  a  proton  as  a  function  of impact
parameter and rapidity can be described by the functional Langevin version of
the JIMWLK renormalization group equation.

The meson production cross-sections contain four point correlators
whose evolution follows from the JIMWLK framework. The four point correlators
are here computed beyond the large-Nc limit.

$N_c$ limit breaks gauge invariance, Weigert restores it minimally on the
level of Wilson lines.

Needed for Wilson lines, in jet-like situations (scattering experiment jet
observables) need to get all color singlets, SU(n).

Only thing that can happen are color rotations (that's where Wilson lines,
driven by the soft gluons, come in), JIMWLK gives effective field theory for
expectation values of these Wilson lines, globally colorless states.

Came from correlators of Wilson lines, needed to get all color singlets for
$\SUn{n}$.

The theory of invariants,
relates the irreducible representations of $\SUN$ over $\Pow{n}$
to the Young tableaux of size $n$, see~\cite{Fulton97,Tung1985na}
and other standard textbooks.


\subsection{Simplification rules for birdtrack operators}
\label{s-AlcZei16-1}

Notes on Alcock-Zeilinger and Weigert\rf{AlcZei16-1}.

They credit Young\rf{Young1933} for introducing \Ypo s,
and refer to Tung\rf{Tung1985na} as the standard reference for them.

\subsubsection{%         \medskip\noindent Sect.~\emph{
2 Notation, conventions and known results}

The direction of the arrow on the index lines of a birdtrack encodes whether
the line acts on the vector space $V$ (arrow pointing from right to left) or
its dual $V^*$ (arrow pointing from left to right)\rf{PCgr}.
In general birdtracks represent primitive invariants of $\SUN$ over a mixed
algebra $\MixedPow{m}{n}$, where $V^*$ is the dual vector space of $V$.
Here only birdtracks acting on a space $\Pow{m}$ are considered (never on the
dual). As all arrows go from right to left, they can be dropped.

The permutations of $S_n$ are the \emph{primitive invariants}\rf{PCgr}
(of $\SUN$ over $\Pow{n}$).
The real subalgebra of $\Lin{\Pow{n}}$ that is spanned by these primitive
invariants is denoted $\API{\SUN,\Pow{n}}\subset\Lin{\Pow{n}}$.
{API} stands for ``\api.''
One
distinguishes

\begin{description}
  \item[Semi-standard irregular tableaux]
Each number
appears \emph{at most once} within a tableau.
  \item[Young tableaux]
The boxes are top- and left-aligned.
The numbers in the boxes to increase within each row from left to right and
within each column from top to bottom.
  \item[Amputated tableaux]
The \emph{column-amputated tableau} is obtained by removing all columns
which do not overlap with the given row.
The \emph{row-amputated tableau}  is obtained by removing all rows which do
not overlap with the given column.
\end{description}

$A\subset B $ denotes that a \emph{Hermitian} projection operator $A$
projects onto a subspace completely contained in the image of a projection
operator $B$, \ie, $A\subset B$ if and only if
\beq
  \label{eq:OperatorInclusion1}
  A \cdot B = B \cdot A = A
\,.
\eeq
This simplification rule breaks down for the standard Young projection
operators whenever they are not Hermitian.

The main result of this paper are the two kinds of simplification rules
(cancellation or propagation) for birdtrack operators $O$ comprised of
symmetrizers and antisymmetrizers.

\subsubsection{%         \medskip\noindent Sect.~\emph{
3 Cancellation rules}

(1) Cancelation rules : Rules to determine whether certain symmetrizers or
antisymmetrizers within an operator $O$ are redundant, and thus can be
\emph{cancelled} from an operator. They can make a long expression
significantly shorter, and thus easier to work with.

The two main cancellation rules are
the {\em cancellation of wedged \Ypo s}, and
the {\em cancellation of wedged ancestor-operators}.

\subsubsection{%         \medskip\noindent Sect.~\emph{
3.1 Cancellation of wedged \Ypo s}

Theorem 1. Outside S and A, inside a \Ypo\

The example that starts with Eq.~(17), goes to the top of the page 10
motivates the general algorithm to remove inner symmetrizers.

The points 1. 2. and 3. are general, not just for this particular tableaux,
hence:

Corollary~1. {\em Cancellation of wedged ancestor-operators}:
can always get rid of an interior \Ypo.

\subsubsection{%         \medskip\noindent Sect.~\emph{
3.2 Cancellation of factors between bracketing sets}

Cancelation rules :
move sets of symmetrizers or antisymmetrizers
through certain parts of the operator.
\\

\noindent\emph{horizontal permutations} of $\Tilde{\Theta}$:
$\mathbf{h}_{\Tilde{\Theta}}$ is the subset of all permutations in
  $S_n$ that only operate within the rows of $\Tilde{\Theta}$; i.e. that do
  not swap numbers across rows. \\
\emph{vertical permutations} of $\Tilde{\Theta}$:
  $\mathbf{v}_{\Tilde{\Theta}}$ is the subset of permutations in $S_n$
  that only operate within the columns of $\Tilde{\Theta}$.
\\

(I am too lazy to work through Tung's Lemma IV.5)
\\


Corollary~2. {\em Cancellation of parts of the operator}
One can always get rid of an inner \Ypo. They lack explicit formula
for the constants; so make sure non-zero, at the end evaluate
the overall constant by other means (projection operator conditions).

Outer $\mathbf{A}_{\Theta}$ and $\mathbf{S}_{\Theta}$ belong to the same
\Ypo, see Eq.~(23): there exists a (possibly vanishing)
constant $\lambda$ such that
  \begin{equation}
    \label{eq:Cancel-General-O}
\mathbf{S}_{\Theta} \; M \; \mathbf{A}_{\Theta} =
\lambda \cdot Y_{\Theta}
\ .
  \end{equation}

The rest of the section  ensures that the constant is non-zero. It's quite of
bit of work, I skipped it (unless they want me to work through some of it).

\subsubsection{%         \medskip\noindent Sect.~\emph{
Dimensional zeroes}

If any of the antisymmetrizers exceed the length $N$ one has a
\emph{dimensional zero}. So one needs to assume $N$ is high enough.
It should work out once the calculation is done - every polynomial
will have zeros for $N$ to small for a given tableaux.


\subsubsection{%         \medskip\noindent Sect.~\emph{
4 Propagation rules}

rules when things commute

(2) {Propagation rules} :
when it is possible to commute (\emph{propagate}) a symmetrizer through an
antisymmetrizer (or vice versa)? Then the cancellation rules might be
applied, or features of a particular operator $O$, such as its Hermiticity
can be made explicit. The answer:
\[
O =\; \scalebox{0.75}{\FPic{mOpsO}} \; = \;
\scalebox{0.75}{\FPic{mOpsOHermitean}} \; = \;
\scalebox{0.75}{\FPic{mOpsOHC}} \; = O^{\dagger}
\,.
\]
It something I had used in the birdtracks.eu book in inchoate manner - they
make it into a precise algorithm.


Example Eq.~(45) is a bit tough

Example Eq.~(48) through Eq.~(50) is easy.



rest of the section is ``Q. when you can commute?''

A. If can get rectangular tableaux, then it commutes

In particular, works also for semi-standard irregular tableaux,
\\
Theorem~3 {\em (generalized propagation rules)}

Eq.~(57) tells it


\subsubsection{%         \medskip\noindent Sect.~\emph{
4.1 Proof of Theorem 2 (generalized propagation rules)}

Proof is long and painful - I did not go through it. Should I?

\subsubsection{%         \medskip\noindent Sect.~\emph{
5 Conclusion}

Keppeler and Sj{\"o}dahl\rf{KeppSjo14} were the first to offer a simple
method to construct Hermitian operators: their iteration is easy to
understand, and the proofs of hermiticity are simple proofs. However, in
practice, the algorithm is inefficient - the expression balloon quickly.

The methods of this paper are also recursive, but with the recursion cut down
drastically. The gain is illustrated by Fig.~5.2 in the paper, here
reproduced as \reffig{fig:MOLDAdvantage}.


\subsection{Compact Hermitian \Ypo s}
\label{s-AlcZei16-2}

Notes on Alcock-Zeilinger and Weigert\rf{AlcZei16-2}.

\Ypo s are (1) idempotent, (2) orthogonal and (3) complete.
But, as the symmetrizers and antisymmetrizers comprising a given Young
tableau do not necessarily commute,  \Ypo s are in general not Hermitian.

Keppeler and Sj{\"o}dahl\rf{KeppSjo14} were first to construct Hermitian
versions of Young projection operators in the birdtrack formalism, by an
iterative algorithm. However, the KS-operators soon become unwieldy and thus
impractical to work.

The construction algorithm presented here,
based on the simplification rules of \refref{AlcZei16-2},
leads to drastically more compact and explicitly Hermitian
expressions for the projection operators than the
KS-algorithm\rf{KeppSjo14}; an example is given
in \reffig{fig:MOLDAdvantage}.

\begin{figure}% [H]
%\newlength\foo
%\settototalheight\foo{\resizebox{\textwidth}{!}{%
%  \diagram[height=.15cm]{MOLDAdvantageEx2}
%}}
  \begin{center}
\resizebox{\textwidth}{!}{%
\begin{tikzpicture}[every node/.style={inner sep=1pt, outer sep=0pt}]
\node (KS) {\diagram[height=.15cm]{MOLDAdvantageEx2}};
\node (short) at ($(KS) +(0,-0.6cm)$)
      {\diagram[height=.15cm]{Simple-KS}};
\node (MOLD) at ($(short) +(0,-0.6cm)$)
      {\diagram[height=.15cm]{MOLDAdvantageEx4}};
\draw[-{stealth}, line width=0.25pt] (KS) to (short);
\draw[-{stealth}, line width=0.25pt] (short) to (MOLD);
\node[scale=0.4] (Cancel) at ($(KS) +(0.7,-0.3cm)$) {Cancellation
  rules};
\node[scale=0.4] (Propagate) at ($(short) +(0.7,-0.3cm)$) {Propagation rules};
\end{tikzpicture}
}
  \end{center}
\caption{
(top)
A Hermitian birdtrack obtained by the iterative KS-algorithm\rf{KeppSjo14}.
Blow it up on the screen to see the details.
(middle)
The much shorter version obtained by application of the cancellation rules.
(bottom)
The explicitly symmetric (Hermitian) version achieved via the propagation
rules.
}
\label{fig:MOLDAdvantage}
\end{figure}


Repeated here are most of the Keppeler and Sj{\"o}dahl\rf{KeppSjo14} {\em
Hermitian \Ypo s}. Keppeler and Sj{\"o}dahl used iterative methods, see
\refref{AlcZei16-2} bottom p.~18


Eq.~(53) not obvious it is symmetric - \rf{AlcZei16-1} gives simplification rules,
dramatic simplification, see Fig.~5.2

\subsubsection{
%\subsubsection{%         \medskip\noindent Sect.~\emph{
3.3 KS Construction principle for Hermitian \Ypo s}


provides a direct route to bottom Fig.~5.2, paper proves that it really works

\Ypo s not being Hermitian has strange consequences.
Eq.~(12) not true, but for the Hermitian ones it is true.



\subsection{Transition operators}
\label{s-AlcZei16-3}

Notes on Alcock-Zeilinger and Weigert\rf{AlcZei16-3}.

The simplification rules of \refref{AlcZei16-2}
allow here a construction of transition operators between (Hermitian) Young
projection operators corresponding to equivalent irreducible representations
of $\SUN$, and an orthogonal basis for the algebra of invariants on
$\Pow{m}$.


completes the picture, the full algebra of invariants

\subsubsection{%         \medskip\noindent Sect.~\emph{
3 Young projection and transition operators}

gives the counting argument that the number of primitive invariants
equals the sum of diagonal operators and transition operators.

\subsubsection{%         \medskip\noindent Sect.~\emph{
5.2 A full orthogonal basis for {\api}}

write Clebsch, as in Eq.~(71)

Eq.~(73) transition operator between equivalent representations

they are unitary if restricted on the representations (top p 21)

together with the hermitian, they give you the full unitary basis

In Eq.~(55a) algebra is decomposed into subalgebras,
Eq.~(55b) is as simple as can be.

Eq.~(63) is non-Hermitian version

see and compare Fig.~2 (non Hermitian) and Fig.~3: (hermitian)
same as birdtracks.eu, but without the transition operators.

Dimension of the algebra goes factorially, so algorithm works up to 8 or 9
(all algebra in Mathematica, up to 8 on the laptop).

\subsection{Singlets}
\label{s-AlcZei16-4}


Notes on J. Alcock-Zeilinger and H. Weigert\rf{AlcZei16-4}.

This paper says that these projection operators give you all singlets.

The orthogonal basis of \refref{AlcZei16-3} is used to form a basis for the
singlet states necessary to determine all color neutral Wilson line
correlators. This has applications in  QCD, such as \refref{MarWei10}
and

Lappi \etal\rf{LRRW16}
{\em {JIMWLK} evolution of the odderon}



\section{Notes on Keppeler and Sj{\"o}dahl}
\label{s-KeppSjo14}

\begin{description}

\item[2014-07-20 PC] More birdtracking - a construction of orthogonal
(Hermitian) projection operators:

\HREF{https://plus.google.com/111710245682175604723}
{Stefan Keppeler} and Malin Sj{\"o}dahl\rf{KeppSjo14} {\em Hermitian \Ypo s}

Sj{\"o}dahl\rf{Sjodahl13,SjoKep13} {\em Tools for calculations in color
space}, Malin.Sjodahl@thep.lu.se

and Keppeler's student Thor{\'e}n\rf{Thoren14}.

\end{description}

\section{Notes on Tai PhD thesis}
\label{s-groupTheBlog}

\HREF{http://www.math.upenn.edu/~mtai/}
{Matthew Tai}'s 2014 PhD thesis\rf{TaiThesis,Tai13}
{\em Family algebras and the isotypic components of $g \bigotimes g$}
(PhD adviser
\HREF{http://www.math.upenn.edu/~kirillov/}
{Alexandre A. Kirillov}\rf{Ki00,Ki01}, of
 Institute for Information Transmission Problems, Russian Academy of Sciences)
appears to supersede the Casimir and many other discussions of {\wwwgt}.
My 2014-10-17 letter to Tai, mtai@math.upenn.edu:

Dear Matthew

Rumors of my death are exaggerated, so I always wonder about why nobody
tells me anything about advances related to my work? Here you are, my
best birdtracks student, and we have not even been introduced?

Anyway, I've started writing down some notes on your thesis in GitHub,
\\
\HREF{https://github.com/cvitanov/reducesymm/}
{GitHub.com/cvitanov/reducesymm},
\\
\texttt{> cd dasgroup} \\
\texttt{> pdflatex blog} \\
read Sect.~{\em Notes on Tai PhD thesis}. For anything technical, please do
not email me, but edit directly into the GitHub version, and let me know
by email to \\
\texttt{dasgroup@mail.gatech.edu}\\
 when you have \texttt{git
push}ed something new to the server. Here are a few notes, from the first
superficial reading. We can meet to discuss face to face anything any
time on Skype or Google Hangouts.

\begin{enumerate}
  \item
Should I write in {\wwwgt} that chapter ? is superseded by your thesis?
  \item
With an eye on revising {\wwwgt}:
which sections of the thesis in particular I should I study?
  \item
Do you have some clever way of generating your diagrams?
  Mine were all drawn by hand, using xfig.
  Do you want to contribute any of the scripts/programs to {\wwwgt} 'extras'?
  \item
why no link to {\wwwgt}?
  \item
ending lines with white dots rather than symmetrizers on external lines
is clever. (but I would not know how to do that if there are internal
symmetrizers and or several symmetrizer in the same diagram)
  \item
any errors, typos, etc. in {\wwwgt} I should fix?
  \item
I wonder where I got the `Pfaffian' from (in your discussion of $D_r /
SO(2k)$). I have no recollection - you happen to know a good reference?
  I should add Pfaffian to the index.
  \item
`The degrees of the primitive Casimir operators' or
`exponents' are the (Betti numbers-1). Compare
my  {\em Table 7.1 Betti numbers for the simple Lie groups}
with  Tai {\em Table 10.1 Exponents for the exceptional Lie algebras}.
``The name `exponents' comes from the exponents of the hyperplane
arrangement corresponding to the simple reflection planes of the Weyl
Group of the Lie algebra. The exponents can also be considered
topologically [...] also have representation-theoretic interpretations''
  \item
Can you contribute your thesis \texttt{*.bib} to {\wwwgt}?
  \item
for $G_2$, should I check Pieter Mostert unpublished paper?
  \item
for $F_4$, I should check 'Albert algebra' (related to
\HREF{http://www.ams.org/journals/bull/1974-80-06/S0002-9904-1974-13622-0/}
{Albert} of
{\wwwgt} ref.~[70] C. W. Curtis\rf{Curtis1963} ...?)
  \item
My 'defining rep', 'fundamental 1-box Young tableaux representation'
or `defining $n$-dimensional rep' is 'reference representation'
or `standard representation'.
  \item
  \item typos
  \begin{itemize}
    \item[p. 23] Clebsche vertices
%    \item[p. ?]
%    \item[p. ?]
  \end{itemize}
\end{enumerate}



\section{Birdtracks blog}
\label{s-birdtrBlog}

\begin{description}

\item[2000-02-01 PC]
This paper says my methods are not good enough:
van Ritbergen, Schellekens and Vermaseren\rf{RiScVe99}
{\em Group theory factors for {Feynman} diagrams}.

\item[2000-09-09 Malin Sj{\"o}dahl]
I have encountered a group theoretical problem, and I'm hoping that you might
know of a solution to the problem (if a solution has been presented).

In QCD the external particles carry (anti-) quark and gluon indices that have
to be summed, as we don't observe individual colors. The relevant color
factor for the interference between two amplitudes M1 and M2 is thus
\[
\sum{q1...qn, q\bar{1}...q\bar{n},g1...gm}
      M1^{q1...qn, q\bar{1}...q\bar{n},g1...gm}
      M2*^{q1...qn, q\bar{1}...q\bar{n},g1...gm}.
\]
This can be seen as a scalar product between two vectors M1 and M2; it is not
hard to argue that the definition of a scalar product is fulfilled. This
means that any amplitude can be written as a linear combination of basis
states, and it would be nice to know of orthogonal bases for an arbitrary
number of quarks and gluons. (I need to have a basis to do resummation,
people doing loop calculations would benefit from having a basis when there
are many external partons. )

If there are only quarks (and anti-quarks - an incoming anti-quark can be
changed to an outgoing quark) such a basis can be constructed from the Young
tableau projectors, for example as in figure 9.1 in your book.

The remaining problem is thus how to deal with the gluons, and my question is
if you are aware of a systematic treatment of gluon indices to construct
orthogonal states. Can one ``recombine" the quark-indices of the Young
tableaux to create orthogonal states? Or, do you know of an alternative
strategy?

\item[2013-06-20 PC to Malin]

                           /draft of Sep 10, 2010/\\
I'm - in manner of everybody now days - horribly behind, so when and if I
answer is uncertain. But on the face of it the answer appears in this
\HREF{http://birdtrack.edu}{excellent book}, which - if you are too poor to
afford a coffee and a croissant in the Frankfurt airport - can also be
downloaded for a click.

Try studying it, and if it looks like the answer is hidden there, I might be
able to help - it's [draft of the letter stops here]

                          /continued June 20, 2013/\\
OMG - I have not forgotten, you have been on my guilt list for a long time,
but hopefully time heals all wounds... Do you still want me to ponder your
question, or is it all resolved, sealed, delivered, and published by now?

I do not know if it is of relevance to you, but we have a serious error in
\href{http://birdtracks.eu/extras/reviews.html}{Appendix B}, which I have not
corrected yet in the book (Tony Kennedy's fix is the length of the book
itself). There are also errata beyond the ones noted on the website that I
have not listed yet...

I have started thinking again about how we fix gauges (I think now that using
covariant gauges was a bad idea), but returning to QCD to implement my
slicing feels so far beyond my reach... If there is something new and
interesting happening in non-perturbative QCD, please do alert me :)

apologetically yours
Predrag


\item[2012-05-12 Stefan Keppeler] <stefan.keppeler@gmail.com>
Dear Predrag,
over the last year I became a great fan of your birdtracking. Together
with Malin Sj{\"o}dahl I'm in the middle of writing up how to get
decompositions like table 9.4 in your book for n-fold tensor products
of the adjoint rep.

I think I found some typos in section 9.14, also within the rabbit-mouse
birdtrack, of all equations ;-) I marked them in orange in the
\HREF{Sec_9-14_annotated.pdf} {attached pdf}.

I think the arrow in $P_7$ in table~9.4 should point in the other direction
(or the sign in front of the second term be changed from minus to plus).

\item[2016-12-10 PC]
Cyrol, A. K. and Mitter, M. and Strodthoff\rf{CyMiSt16}
{\em {FormTracer - A Mathematica Tracing Package Using FORM}}
reviews the current software offerings.
\emph{FormTracer} includes different group tracing algorithms that are
implemented in FORM\rf{RiScVe99,KuUeVe15}. The most general algorithm is
provided by the FORM color package\rf{RiScVe99} and allows to take traces of
arbitrary simple compact Lie groups. Furthermore we include explicit tracing
algorithms for the fundamental representation in SU(N) , SO(N) and Sp(N),
adapted from routines 4 published with the color package\rf{RiScVe99} that
use the Cvitanovi\'c algorithm\rf{PCgr} with additional support for partial
traces. Finally, we include dedicated tracing algorithms for the fundamental
representations in SU(2) and SU(3) that support partial traces, explicit
numerical indices as well as transposed group generators. The use of explicit
numerical indices requires to work in explicit representations.  For SU(2)
and SU(3) we choose generators proportional to Pauli and Gell-Mann matrices,
respectively. Note that the fundamental SU(N) tracing algorithm also supports
partial traces but does not guarantee the same degree of simplification as
the specific SU(2) and SU(3) routines.

\item[2013-02-22 PC] Fomin and Pylyavskyy\rf{FomPyl12}
{\em Tensor diagrams and cluster algebras}, {\arXiv{1210.1888}},
is a major orgy in birdtracking. Should study it some day.

\item[2014-07-20 PC] More birdtracking:

Gu and Jockers\rf{GuJock14}
 {\em A note on colored {HOMFLY} polynomials for hyperbolic knots
      from {WZW} models}

Kol and Shir\rf{KolShir14} {\em Color structures and permutations}.

\item[2014-12-02 PC] More birdtracking:

Geyer and Lazar\rf{GeyLaz00}
{\em Twist decomposition of nonlocal light-cone operators
{II:} general tensors of 2nd rank}

Costa and Hansen\rf{CosHan14}
{\em Conformal correlators of mixed-symmetry tensors}

Rejon-Barrera and Robbins\rf{RejRob16} {\em Scalar-vector bootstrap}

Costa \etal\rf{Costa2016}
{\em Projectors and seed conformal blocks for traceless mixed-symmetry tensors}

Pang, Rong  and Su\rf{PaRoSu16} {\em {$\phi^3$} theory with {$F_4$} flavor
symmetry in {$6-2\epsilon$} dimensions: 3-loop renormalization and conformal
bootstrap}

\item[2016-12-08 Michael Stone] m-stone5@uiuc.edu
\\
It's quite interesting to read Young's original paper\rf{Young1928} on this
issue. See the discussion after Theorem III on pa 263. This shows how to fix
up the projectors, but the result is not pretty. Essentially the same
discussion appears in the section of Young projectors in D E Littlewood's
``University Algebra,'' which is where I was first alerted to the problem.

\item[2016-12-08 Predrag]
Liu and Zerf\rf{LiuZer16} {\em Irreducible tensor basis and general {Fierz}
relations for {Bhabha} scattering like amplitudes} has Fierz-e birdtracking for
$\SOn{3,1}$.



\end{description}


%\newpage %%%%%%%%%%%%%%%%%%%%%%%%%%%%%%%%%%%%%%%%%%%%%%%%
\printbibliography[heading=subbibintoc,title={References}]
