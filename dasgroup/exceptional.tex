% compile by  pdflatex blog; biber blog
% GitHub cvitanov/reducesymm/dasgroup/exceptional.tex


\chapter{Matters exceptional}
\label{s-groupTheBlog}



\begin{description}

\item[2013-09-08  Brian Swingle]
If a lattice is [8$\times$8], the
the lattice Laplacian has periodic boundary conditions `-1's in the two
corners. If `-1' are instead in 3rd row of 8th column, and transpose,
this is the Cartan matrix for $E_8$, used in constructing matrix models,
says
\HREF{http://pirsa.org/displayFlash.php?id=13050027}
{Brian Swingle}. Cartan matrices are Laplacians in some sense, try to
figure out what sense?

\item[2014-08-07  Predrag]
Phil Morrison told me that Birkhoff discovered $G_2$ while looking for
hidden symmetries of the heat kernel. Did not find the reference (it is
in his thin book on statistical mechanics?) but this is of interest:

Ilka Agricola\rf{Agricola08} writes:
``
In a talk delivered in Leipzig on June 11, 1900, Friedrich
Engel\rf{Engel1900} gave the first public account of his newly discovered
description of the smallest exceptional Lie group $G_2$, and he wrote in
the corresponding note to the Royal Saxonian Academy of Sciences:
Moreover, we hereby obtain a direct definition of our 14-dimensional
simple group $G_2$ which is as elegant as one can wish for. Indeed,
Engel's definition of $G_2$ as the isotropy group of a generic 3-form in
7 dimensions is at the basis of a rich geometry that exists only on
7-dimensional manifolds,
''

This is precisely how I think of $G_2$, so should give Engel and
Reichel credit, and
cite Agricola\rf{Agricola08}.

One immediately discovers that Baez and Huerta\rf{BaeHue14} think of
$G_2$ as a small ball rolling on a big ball, and so on; with this one can
be infinitely distracted.

The\rf{Tai13} {\em Exceptionally simple PDE}: ``
[...] a remarkably uniform generalization of the Cartan-Engel models from
1893 in the $G_2$ case.
''

\item[2015-09-14 Bernard Julia]
I at last have maybe some hint on the difference and on the common origin
of our two distinct magic triangles. I asked P. Deligne for advice but not
very diplomatically I enquired about his relation with your work so total
silence was the answer.
I discussed since with P. Vogel and was wondering whether you had made
progress. My breakthrough comes out of nowhere so it takes time to
decipher.
I suggested to Vogel he should go beyond powers of the adjoint
representation (your idea for the triangle) I probably should do it if it
has not been done before...

Answered 2016-02-06 by dasgroup@gatech.edu email,
complaining that `` I do not know why Deligne is so unwilling to cite my
work that precedes his - he has known about it forever
(\HREF{http://birdtracks.eu/extras/Deligne96.pdf} {click here}). I
have never met him. I had once been in Vogel's office (about 10 years
ago?). The desk was covered by birdtracks :) I'm a bit annoyed that that
things that I did decades before Vogel and are now credited to them, and
I get cited only for Vogletracks, not Vogelsong. But I do have a tenured
job and freedom to do anything, so who cares. I too find Deligne's paper
more beautiful than my own formulation. It's just that I did it first,
but apparently if you are not a "mathematician" it does not count.

Bernard Julia response:
I visited Vogel about 9 months ago about his (super)group manifold and
his version of the E line! He is trying to disprove a conjecture of
Deligne in this field.

Here is the general idea of my program, nothing has been published.
The theory of second order Painlev\'e differential ODE is classical. The
definition of difference Painlev\'e equations is not uniformly agreed upon
however but the main one due to Sakai does indeed suggest that my En
series (equivalently that of the Del Pezzo's) correspond to so called
q-difference equations whereas yours (I mean the A0 A1 A2 end) is closer to
those diagrams appearing in additive difference Painlev\'e equations but $F_4$
and $G_2$ are missing today. I urged the experts to look for them and one more
branch appears with A3 symmetry as an important piece. Furthermore I am
trying to clarify  the delicate connection(s) to (continuous) differential
equations and c-Painlev\'e VI takes a leading role there only.
A very small workshop: One day would cover discretizations of
symmetries and of spaces, one day for the characteristic properties of
Painlev\'es (what makes them polyquitous) and one day for algebraic geometry
(curves appear in string theory but also surfaces ...Del Pezzo's...),
Calabi-Yau's etc... but confinement of singularities and the algebraic
entropy criterion of Viallet et al. is very relevant, you may know about
it from another side, QRT maps etc...)

\item[2016-02-07  Predrag]
Never heard of them until today, but there is huge literature on
\HREF{https://en.wikipedia.org/wiki/Del_Pezzo_surface} {Del Pezzo
surfaces}.


\item[2015-11-11 Bruce Westbury]
I know you no longer work on ``birdtracks" but I thought you might be
interested in my preprint\rf{Westbury15} {\em Extending and quantising
the Vogel plane}. \\
(Answered 2016-02-06 by dasgroup@gatech.edu email,
complaining that ``I'm a bit annoyed that that things that I did decades
before Vogel and Deligne are now credited to them, and I get cited only
for Vogletracks, not Vogelsong.'').

\item[2016-02-08 Bruce Westbury]
I did try to understand spinors but felt I did not succeed. I posted my
attempt in \arXiv{1007.2579}.

I think the problem with the $E_7$ (and also the $E_6$, $F_4$, $D_4$ series) is that
it does not actually form a series. There is nothing written on this but
this does seem to be the consensus. When I say it is not a series, I
mean you end up with a finite list (so no parameter). This is known to
happen for the $G_2$ series. I wrote an account of this in
\arXiv{1011.6197}

However, I still believe the E8 series exists as Deligne conjectured and
am currently working on proving this. This seems to me to be crucial.
\\
(Answered 2016-02-08 by dasgroup@gatech.edu email,
complaining that ``
My only bone is that I do not get how the $E_8$ series that I computed in
a 1977 preprint and officially published in 1981 gets to be called
``Deligne conjecture'' from 1996 on. I explain the history in detail in
birdtracks.eu section 21.2~{\em A brief history of exceptional magic}. They
have recomputed the same formulas, they know of my work, they mention
that they this ``well known to physicists (cf. Cvitanovi{\'c} [83]),'' and
that's it. The total credit for what was a wonderful breakthrough for me.
That's neither nice nor professional. Oh, screw them :)
'')


\item[2016-12-03 Predrag]
Read

M.~A.~Olshanetskii and V.B.K.~Rogov\rf{Olshanetskii1987}
``Adjoint representations of exceptional Lie algebras,''

MacKay\rf{MacKay05}
{\em Introduction to {Yangian} symmetry in integrable field theory}:
``
This, alongside the appearance of X in \refref{PD96,PCgr} and the
unified R-matrix structure of \refref{Westbury02}, suggests that it might be
interesting to investigate the connection between Yangians and the `magic
square' construction of the exceptional Lie groups.
''

Westbury\rf{Westbury06a}
{\em Universal characters from the {MacDonald} identities}


\item[2016-08-20 Predrag]
Anastasiou \etal\rf{AnBoHuNa16}
{Global symmetries of {Yang-Mills} squared in various dimensions}: ``
It should be noted that these are not the only magic triangles of Lie
algebras, a particularly elegant and intriguing example being that of
Cvitanovi{\'c}\rf{C77,PCgr}.
''

\item[2016-08-20 Bernard Julia]
Again, citation is not complete but it is a big story with old references
so you might be interested in (see, for example page 2):

Shimizu and Tachikawa\rf{ShiTac16}
{\em Anomaly of strings of {$6d$ $N = (1,0)$} theories}

Looks like one also has to look at
Kim \etal\rf{KiKiPa16} {\em {6d} strings from new chiral gauge theories}

\item[2016-12-03 Predrag]
Reading Shimizu and Tachikawa\rf{ShiTac16}, \arXiv{1608.05894},
and in particular their footnote $^1$.
They credit Deligne for ``Deligne's exceptional series of groups,'' and then
they say ``also independently found by Cvitanovi{\'c}.'' This after reading
section~21.2 of birdtacks book\rf{PCgr}. I do not get the ethics of
mathematicians, they must think it is a town in Russia. If I discover the
series in 1975-77, and Deligne rediscovers almost 20 years
later\rf{PD96}, fully aware that I had discovered it (click
\HREF{http://birdtracks.eu/extras/Deligne96.pdf}{here}), how is that the
``Deligne series?''

They also refer to {1988} Mathur and Mukhi and Sen paper\rf{MaMuSe88},
Table~4 of Beem \etal\rf{BLLPRR15}, and Table~1 of Lemos and
Liendo\rf{LemLie16}. The 2000 paper of Grassi and Morrison\rf{GraMor00}
refers only to Deligne, does not cite me at all.

\item[2018-10-06 Predrag]
Beem and Rastelli\rf{BeeRas18}
{\em Vertex operator algebras, {Higgs} branches, and modular differential equations},
\arXiv{1707.07679}:

[...] we aim to characterize the connection between the Higgs branch of the
moduli space of vacua (as an algebraic geometric entity) and the associated
vertex operator algebra.
[...] We illustrate these ideas in a number of examples including a series of
rank-one theories associated with the ``Deligne-Cvitanovi{\'c} exceptional series''
of simple Lie algebras, several families of Argyres-Douglas theories, an
assortment of class S theories, and N=2 super Yang-Mills with su(n) gauge group
for small-to-moderate values of n.

[...] Our first set of examples is a collection of nine vertex algebras as that
exhibit a number of remarkable properties. They are the vertex algebras
associated to the Deligne-Cvitanovi{\'c} (DC) exceptional series of simple Lie
algebras
\beq
{a}_0 \subset {a}_1 \subset {a}_2 \subset {g}_2 \subset {d}_4
\subset f_4 \subset e_6 \subset e_7 \subset e_8
\,,
\ee{BeeRas18(4.1)}
at the negative levels
\beq k_{2 d} = - \frac{h^\vee}{6} - 1
\,,
\ee{BeeRas18(4.2)}
where $h^\vee$ is the dual Coxeter number. The theory attached to ${a}_0$
(the trivial Lie algebra) is the Virasoro VOA with central charge
$c_{2d}=-22/5$, which is the value corresponding to the Lee-Yang, or $(2,5)$,
minimal model. Eight of these vertex algebras, namely the cases
$\{{a}_0\,,{a}_1\,,{a}_2\,,{f}_4\,,{e}_6\,,{e}_7\,,{e}_8\}$,
are known to be associated to physical four-dimensional theories. They are the
rank-one SCFTs that arise on the worldvolume of a single $D3$ brane at an
F-theory singularity. The four-dimensional interpretation of the ${g}_2$ and
${f}_4$ cases remains unclear.
[...]

The rest of the discussion I do not know enough to be able to follow.

Mukhi and Muralidhara\rf{MukMur18}
{\em Universal {RCFT} correlators from the holomorphic bootstrap},
\arXiv{1708.06772} is a followup on Beem and Rastelli\rf{BeeRas18},
with lots of explicit calculations and Baby Monster CFT.

Mathur, Mukhi and Sen\rf{MaMuSe88} discovered a theory lying ``between" $E_7$ and
$E_8$, which satisfies most axioms of CFT (except that the ``identity'' is
degenerate). This matches the so-called $E_{7.5}$ algebra\rf{Landsberg2006143}. I
was first\rf{C77} to have it, by 1975-1977, it is the $m=30$ column of
nonreductive algebras in Table 17.1 of my book\rf{PCgr}, with adjoint rep of
dimension 190, as noted by Landsberg and Manivel\rf{Landsberg2006143}.

\item[2018-10-28 Predrag]
Ramesh Chandra and Mukhi\rf{RamMuk18}
{\em Towards a classification of two-character rational conformal field theories},
\arXiv{1810.09472},
of course write
``Subsequently Deligne\rf{PD96} proposed that the corresponding Lie algebras form
a series with special re presentation-theoretic properties (similar observations
have been made by Cvitanovi{\'c}\rf{PCgr})''
explaining their Table~1
{\em The MMS Series}. But I wrote it up\rf{C77} already in
1977? I love Deligne's beautiful papers, but since when doing it 20 years earlier
does not count? Also 11 year before MMS, See Sect.~21.2 {\em A brief history of exceptional magic}, and the
\HREF{http://birdtracks.eu/extras/reviews.html\#letters} {1996 letter}
from Deligne.

\item[2020-05-16 Predrag]
Beem, Meneghelli, Peelaers and Rastelli\rf{BMPR20}
{\em {VOAs} and rank-two instanton {SCFTs}}
have appendix {\em A Some Properties of the Deligne-Cvitanovi\'c
Exceptional Lie Algebras}:

From the Vertex operator algebras (VOA) viewpoint, there is no
obstruction---and in fact it appears quite natural---to include two
additional Lie algebras, $\mathfrak{g}_2$ and $\mathfrak{f}_4$, to the
previously listed seven, thus completing the so-called
Deligne-Cvitanovi\'c series of exceptional Lie algebras\rf{PD96,PCgr}.
Their inclusion is suggested by the observation that the resulting series
of nine current algebras are uniquely singled out as those whose levels
and Virasoro central charges simultaneously saturate three independent
(four-dimensional) unitarity bounds. While their higher-rank cousins are
not known to be singled out in such fashion, we find that the higher-rank
VOAs still behave in a remarkably uniform fashion.

\item[2022-085-29 Predrag]
Thurston\rf{Thurston04}
{\em The {$F_4$} and {$E_6$} families have a finite number of points} (2004).

Dylan credits me globally,
"The graphical techniques that we use to treat the various exceptional
series were pioneered by Cvitanovi{\'c}, who first investigated these
series in the 1970's," so I cannot tell what is new in his paper
without parsing it line by line. To be done.

\item[2022-085-29 Predrag]
Alistair Savage (U. Ottawa) talk:
\HREF{https://alistairsavage.ca/talks/2022-Savage-Diagratification.pdf}
{Diagratification}  (2022).

Gandhi, Savage and Zainoullin\rf{GaSaZa21}
%Raj Gandhi, Alistair Savage, Kirill Zainoulline
{\em Diagrammatics for {$F_4$}} (2021)
\arXiv{2107.12464}:

[...] several of the equations deduced in the current paper can be found
in Cvitanovi{\'c}\rf{PCgr} Ch. 19 and Thurston\rf{Thurston04} in a
different language. However, a complete treatment from the monoidal
category point of view seems to be
new.

They do not credit me with anything in particular,  so I cannot tell what
is new in his paper without parsing it line by line. To be done.

Savage and Westbury\rf{SavWes22}
{\em Quantum diagrammatics for {$F_4$}}  (2022)
\arXiv{2204.11976}

Say ``[...] Inspired by the language of Feynman diagrams in quantum field
theory, invariant tensors for semisimple Lie algebras have also been
computed diagrammatically by Cvitanovi{\'c}\rf{PCgr},.'' but then not
credit me with anything,  so I cannot tell what is new in his paper
without parsing it line by line. To be done.

To the authors: Please fix Cvitanonivi{\'c} $\to$ Cvitanovi{\'c}!


\end{description}
\renewcommand{\ssp}{a}


%\newpage %%%%%%%%%%%%%%%%%%%%%%%%%%%%%%%%%%%%%%%%%%%%%%%%
\printbibliography[heading=subbibintoc,title={References}]
