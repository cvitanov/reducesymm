% reducesymm/dasgroup/groupThe.tex
% Predrag  created              Aug 7 2014
% notes for birdtracks.eu



\chapter{Group theory blog}
\label{c-groupThe}


Enter here notes of general group-theoretic interest, perhaps
for inclusion into revisions of \wwwgt.

\section{Notes on Tai PhD thesis}
\label{s-groupTheBlog}

\HREF{http://www.math.upenn.edu/~mtai/}
{Matthew Tai}'s 2014 PhD thesis\rf{TaiThesis,Tai13}
{\em Family algebras and the isotypic components of $g \bigotimes g$}
(PhD adviser
\HREF{http://www.math.upenn.edu/~kirillov/}
{Alexandre A. Kirillov}\rf{Ki00,Ki01}, of
 Institute for Information Transmission Problems, Russian Academy of Sciences)
appears to supersede the Casimir and many other discussions of {\wwwgt}.
My 2014-10-17 letter to Tai, mtai@math.upenn.edu:

Dear Matthew

Rumors of my death are exaggerated, so I always wonder about why nobody
tells me anything about advances related to my work? Here you are, my
best birdtracks student, and we have not even been introduced?

Anyway, I've started writing down some notes on your thesis in GitHub,
\\
\HREF{https://github.com/cvitanov/reducesymm}
{GitHub.com/cvitanov/reducesymm}
sect.~{\em Notes on Tai PhD thesis}. For anything technical, please do
not email me, but edit directly into the GitHub version, and let me know
by email to \texttt{dasgroup@mail.gatech.edu} when you have \texttt{git
push}ed something new to the server. Here are a few notes, from the first
superficial reading. We can meet to discuss face to face anything any
time on Skype or Google Hangouts.

\begin{enumerate}
  \item
Should I write in {\wwwgt} that chapter ? is superseded by your thesis?
  \item
With an eye on revising {\wwwgt}:
which sections of the thesis in particular I should I study?
  \item
Do you have some clever way of generating your diagrams?
  Mine were all drawn by hand, using xfig.
  Do you want to contribute any of the scripts/programs to {\wwwgt} 'extras'?
  \item
why no link to {\wwwgt}?
  \item
ending lines with white dots rather than symmetrizers on external lines
is clever. (but I would not know how to do that if there are internal
symmetrizers and or several symmetrizer in the same diagram)
  \item
any errors, typos, etc. in {\wwwgt} I should fix?
  \item
I wonder where I got the `Pfaffian' from (in your discussion of $D_r /
SO(2k)$). I have no recollection - you happen to know a good reference?
  I should add Pfaffian to the index.
  \item
`The degrees of the primitive Casimir operators' or
`exponents' are the (Betti numbers-1). Compare
my  {\em Table 7.1 Betti numbers for the simple Lie groups}
with  Tai {\em Table 10.1 Exponents for the exceptional Lie algebras}.
``The name `exponents' comes from the exponents of the hyperplane
arrangement corresponding to the simple reflection planes of the Weyl
Group of the Lie algebra. The exponents can also be considered
topologically [...] also have representation-theoretic interpretations''
  \item
Can you contribute your thesis \texttt{*.bib} to {\wwwgt}?
  \item
for $G_2$, should I check Pieter Mostert unpublished paper?
  \item
for $F_4$, I should check 'Albert algebra' (related to
\HREF{http://www.ams.org/journals/bull/1974-80-06/S0002-9904-1974-13622-0/}
{Albert} of
{\wwwgt} ref.~[70] C. W. Curtis\rf{Curtis1963} ...?)
  \item
My 'defining rep', 'fundamental 1-box Young tableaux representation'
or `defining $n$-dimensional rep' is 'reference representation'
or `standard representation'.
  \item
  \item typos
  \begin{itemize}
    \item[p. 23] Clebsche vertices
%    \item[p. ?]
%    \item[p. ?]
  \end{itemize}
\end{enumerate}

\section{Notes on Keppeler and Sj{\"o}dahl}
\label{s-AlcZei16}



\section{Notes on Alcock-Zeilinger and Weigert}
\label{s-AlcZei16}

Nov 30, 2016 Predrag Skype session with
Heribert and Judith, about their 3 birdtracking preprints.

Came from correlators of Wilson lines, needed to get all color singlets, SU(n).
Important in application that operators are Hermitian.

Physics motivation:

[8] C. Marquet and H. Weigert, ``New observables to test the Color Glass Condensate beyond the large-Nc
limit,''

N-c limit break gauge invariance, Weigert restores it minimally on the level of
Wilson lines.


All arrows go from right to left, as in birdtracks.eu, so they can be dropped.

Repeated most of Keppeler and Sj{\"o}dahl\rf{KeppSjo14}
   {\em {Hermitian Young} operators}.
They used iterative methods,
see
\refref{AlcZei16-3} bottom p.~18

3.3 KS Construction principle for Hermitian Young projection operators

(53) not obvious it is symmetric - \rf{AlcZei16-1} gives simplification rules,
dramatic simplification, see Figure 5.2

\subsection{Simplification rules for birdtrack operators}
\label{s-AlcZei16-1}

Notes on Alcock-Zeilinger and Weigert\rf{AlcZei16-1}.

Need calculation rules:

3 Cancellation rules

Cancelation rules : how many (anti)-symmetrizers are redundant?

3.1 Cancellation of wedged Young projectors

Theorem 1. Outside S and A, inside YPO

(17) to top page (10) removed inner symmetrizers.

extract points 1. 2. and 3. are general (not particular tableaux

Corollary 1. can always get rid of middle one

3.2 Cancellation of factors between bracketing sets

Outer are two are A and S that belong to the same young projection operator
(23)

Corollary 2. can always get rid of middle one (lack explicit formula
for the constants; make sure non-zero, at the end you can evaluate
the overall constant by other means (projection operator conditions))

Rest ensures that the constant is non-zero.

Dimensional zeros

4 Propagation rules

rules when things commute

example (45)
(48) easier

(50)

rest of the section is when you can commute?

If can get rectangular tableaux, then it commutes

4.1 Proof of Theorem 2

Proves it - long and painful

(57) tells it

4.2 Proof of Theorem 3 (generalized propagation rules)
Does not need to be Young tableaux to work

5 Conclusion

Keppeler simplified (iteration easy to understand, simple proofs) but it is
inefficient - result balloons. This paper also recursive, but with the recursion
cut down drastically.

Figure 5.2


\subsection{Compact {Hermitian Young} projection operators}
\label{s-AlcZei16-2}

Notes on Alcock-Zeilinger and Weigert\rf{AlcZei16-2}.




provides a direct route to bottom Figure 5.2, paper proves that it really works

Young proj ops not being Hermitian has strange consequences.
(12) not true, but for the Hermitian ones it is true



\subsection{Transition operators}
\label{s-AlcZei16-3}

Notes on Alcock-Zeilinger and Weigert\rf{AlcZei16-3}.


completes the picture, the full algebra of invariants

3 Young projection and transition operators:

gives the counting argument that the number of primitive invariants
equals the sum of diagonal operators and transition operators.

5.2 A full orthogonal basis for API

write Clebsch, as in (71)

(73) transition operator between equivalent representations

they are unitary if restricted on the representations (top p 21)

together with the hermitian, they give you the full unitary basis

In (55a) algebra is decomposed into subalgebras, (55b) is as simple as can be.

(63) is non-Hermitian version

see and compare Figure 2 (non Hermitian) and Figure 3: (hermitian)
same as birdtracks.eu, but without the transition operators.

Dimension of the algebra goes factorially, so algorithm works up to 8 or 9
(all algebra in Mathematica, up to 8 on the laptop).


J. Alcock-Zeilinger and H. Weigert\rf{AlcZei16-4}
{\em Singlets}

General motivation:

Needed for Wilson lines, in jet-like situations
(scattering experiment jet observables) needed to get all color singlets, SU(n).

Larry McLarren propaganda:

Functional evolution evolution equation  for QCD cross sections in high
energy limit (Bjorken x less than $10^{-2}$), as you push up energy make more
and more soft gluons, making the system highly nonlinear. Parton model
picture breaks down. BFKL equation is in Bjorken x, but distributions go
exponentially large; Weigert contributed to the nonlinear version.

Color glass condensate (in the standard model, only QCD does it).

Only thing that can happen are color rotations (that's where Wilson lines,
driven by the soft gluons, come in),
gives effective field theory for expectation values of these Wilson lines,
globally colorless states.

This paper says that these projection operators give you all singlets.
