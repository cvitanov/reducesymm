% reducesymm/dasgroup/groupThe.tex
% Predrag  created              Aug 7 2014
% notes for birdtracks.eu



\chapter{Group theory blog}
\label{c-groupThe}


Enter here notes of general group-theoretic interest, perhaps
for inclusion into revisions of \wwwgt.

\section{Notes on Tai PhD thesis}
\label{s-groupTheBlog}

\HREF{http://www.math.upenn.edu/~mtai/}
{Matthew Tai}'s 2014 PhD thesis\rf{TaiThesis,Tai13}
{\em Family algebras and the isotypic components of $g \bigotimes g$}
(PhD adviser
\HREF{http://www.math.upenn.edu/~kirillov/}
{Alexandre A. Kirillov}\rf{Ki00,Ki01}, of
 Institute for Information Transmission Problems, Russian Academy of Sciences)
appears to supersede the Casimir and many other discussions of {\wwwgt}.
My 2014-10-17 letter to Tai, mtai@math.upenn.edu:

Dear Matthew

Rumors of my death are exaggerated, so I always wonder about why nobody
tells me anything about advances related to my work? Here you are, my
best birdtracks student, and we have not even been introduced?

Anyway, I've started writing down some notes on your thesis in GitHub,
\\
\HREF{https://github.com/cvitanov/reducesymm/}
{GitHub.com/cvitanov/reducesymm},
\\
\texttt{> cd dasgroup} \\
\texttt{> pdflatex blog} \\
read Sect.~{\em Notes on Tai PhD thesis}. For anything technical, please do
not email me, but edit directly into the GitHub version, and let me know
by email to \\
\texttt{dasgroup@mail.gatech.edu}\\
 when you have \texttt{git
push}ed something new to the server. Here are a few notes, from the first
superficial reading. We can meet to discuss face to face anything any
time on Skype or Google Hangouts.

\begin{enumerate}
  \item
Should I write in {\wwwgt} that chapter ? is superseded by your thesis?
  \item
With an eye on revising {\wwwgt}:
which sections of the thesis in particular I should I study?
  \item
Do you have some clever way of generating your diagrams?
  Mine were all drawn by hand, using xfig.
  Do you want to contribute any of the scripts/programs to {\wwwgt} 'extras'?
  \item
why no link to {\wwwgt}?
  \item
ending lines with white dots rather than symmetrizers on external lines
is clever. (but I would not know how to do that if there are internal
symmetrizers and or several symmetrizer in the same diagram)
  \item
any errors, typos, etc. in {\wwwgt} I should fix?
  \item
I wonder where I got the `Pfaffian' from (in your discussion of $D_r /
SO(2k)$). I have no recollection - you happen to know a good reference?
  I should add Pfaffian to the index.
  \item
`The degrees of the primitive Casimir operators' or
`exponents' are the (Betti numbers-1). Compare
my  {\em Table 7.1 Betti numbers for the simple Lie groups}
with  Tai {\em Table 10.1 Exponents for the exceptional Lie algebras}.
``The name `exponents' comes from the exponents of the hyperplane
arrangement corresponding to the simple reflection planes of the Weyl
Group of the Lie algebra. The exponents can also be considered
topologically [...] also have representation-theoretic interpretations''
  \item
Can you contribute your thesis \texttt{*.bib} to {\wwwgt}?
  \item
for $G_2$, should I check Pieter Mostert unpublished paper?
  \item
for $F_4$, I should check 'Albert algebra' (related to
\HREF{http://www.ams.org/journals/bull/1974-80-06/S0002-9904-1974-13622-0/}
{Albert} of
{\wwwgt} ref.~[70] C. W. Curtis\rf{Curtis1963} ...?)
  \item
My 'defining rep', 'fundamental 1-box Young tableaux representation'
or `defining $n$-dimensional rep' is 'reference representation'
or `standard representation'.
  \item
  \item typos
  \begin{itemize}
    \item[p. 23] Clebsche vertices
%    \item[p. ?]
%    \item[p. ?]
  \end{itemize}
\end{enumerate}

\section{Notes on Keppeler and Sj{\"o}dahl}
\label{s-AlcZei16}

\begin{description}

\item[2014-07-20 PC] More birdtracking - a construction of orthogonal
(Hermitian) projection operators:

Keppeler and Sj{\"o}dahl\rf{KeppSjo14} {\em Hermitian \Ypo s}

Sj{\"o}dahl\rf{Sjodahl13,Sjodahl13a} {\em Tools for calculations in color
space},

and Keppeler's student Thor{\'e}n\rf{Thoren14}.

\end{description}


\section{Notes on Alcock-Zeilinger and Weigert}
\label{s-AlcZei16}

Nov 30, 2016 Predrag Skype session with
Heribert and Judith, about their 3 birdtracking preprints.

Anyway, I've started writing down some notes on Judith thesis in GitHub,
\\
\HREF{https://github.com/cvitanov/reducesymm/}
{GitHub.com/cvitanov/reducesymm},
\\
\texttt{> cd dasgroup} \\
\texttt{> pdflatex blog} \\
read Sect.~{\em Notes on Alcock-Zeilinger and Weigert}.
For anything technical, please do not email me, but edit directly into the
GitHub version, and let me know by email to \\
\texttt{dasgroup@mail.gatech.edu}\\
when you have \texttt{git push}ed something new to the server.

\newcommand*\circled[1]{\tikz[baseline=(char.base)]{
            \node[shape=circle,draw,inner sep=2pt] (char) {#1};}}
\newcommand{\diagram}[2][{}]{\pbox{\textwidth}{\includegraphics[#1]{{#2}}}}
\newcommand{\SUN}{\mathsf{SU}(N)}
\newcommand{\MixedPow}[2]{V^{\otimes
    #1}\otimes\left(V^*\right)^{\otimes #2}}
\newcommand{\Pow}[1]{V^{\otimes #1}}
\newcommand{\DAlg}[1]{\left(V^*\right)^{\otimes #1}}
\newcommand{\Lin}[1]{\mathrm{Lin}\left( #1 \right)}
\newcommand{\API}[1]{\mathsf{API}\left( #1 \right)}
\newcommand{\InvAlg}[1]{A\left[ S_{#1} \right]}
\newcommand{\Rsim}{\stackrel{\mathcal{R}}{\sim}}

\subsection{Physics motivation}
\label{s-AlcZei16-HEP}

% Physics motivation:

(read up on Larry McLarren propaganda)

Applications of these
tools in a QCD context where factorization invariably involves color
singlet projections of Wilson line correlators, see
several fields with possible
applications:

Marquet and Weigert\rf{Marquet:2010cf} {\em New observables to test the {Color
Glass Condensate} beyond the large-{$N_c$} limit}

Weigert\rf{Weigert:2003mm}
{\em Non-global jet evolution at finite {$N_c$}}

Falcioni \etal\rf{Falcioni:2014pka}
{\em Multiple gluon exchange webs}

Bomhof \etal\rf{Bomhof:2006dp}
{\em The construction of gauge-links in arbitrary hard processes}

Since $\SUN$ is the gauge group of QCD, \Ypo s come into play through the
theory of invariants, which relates the irreducible representations of $\SUN$
over $\Pow{n}$ to the Young tableaux of size $n$\rf{Fulton97,Tung:1985na}.
The lack of Hermiticity of \Ypo s  disqualifies them from the application to
QCD calculations: for applications the operators need to be Hermitian (hence
\refrefs{KeppSjo14,AlcZei16-2}) and all singlets are accounted for (hence
\refref{AlcZei16-4}).

Functional evolution evolution equation  for QCD cross sections in high
energy limit (Bjorken $x$ less than $10^{-2}$), as you push up energy make
more and more soft gluons, making the system highly nonlinear. Parton model
picture breaks down. BFKL pomeron equation is in Bjorken $x$, but distributions go
exponentially large; Weigert contributed to formulating the nonlinear
version.

{\em Color Glass Condensate} (within the standard model, only QCD does it):
\\

The Balitsky-\-JIMWLK (Jalilian-\-Marian-\-Iancu-\-McLerran-\-Weigert-\-Leonidov-\-Kovner)
is a tool to calculate the energy dependence of QCD observables at high
energies. Gluon  distribution  in  a  proton  as  a  function  of impact
parameter and rapidity can be described by the functional Langevin version of
the JIMWLK renormalization group equation.

The meson production cross-sections contain four point correlators
whose evolution follows from the JIMWLK framework. The four point correlators
are here computed beyond the large-Nc limit.

$N_c$ limit breaks gauge invariance, Weigert restores it minimally on the
level of Wilson lines.

Needed for Wilson lines, in jet-like situations (scattering experiment jet
observables) need to get all color singlets, SU(n).

Only thing that can happen are color rotations (that's where Wilson lines,
driven by the soft gluons, come in), JIMWLK gives effective field theory for
expectation values of these Wilson lines, globally colorless states.

Came from correlators of Wilson lines, needed to get all color singlets for
$\SUn{n}$.

The theory of invariants,
relates the irreducible representations of $\SUN$ over $\Pow{n}$
to the Young tableaux of size $n$, see~\cite{Fulton97,Tung:1985na}
and other standard textbooks.



\textbf{Predrag question}: should I use Wilson lines formulation to account for soft
photons crossing a QED vertex, in my
\HREF{chaosbook.org/~predrag/papers/preprints.html\#FiniteFieldTheo}
{``finitness conjecture''}?

\subsection{Simplification rules for birdtrack operators}
\label{s-AlcZei16-1}

Notes on Alcock-Zeilinger and Weigert\rf{AlcZei16-1}.

They credit Young\rf{Young:1933} for introducing \Ypo s,
and refer to Tung\rf{Tung:1985na} as the standard reference for them.

\subsubsection{%         \medskip\noindent Sect.~\emph{
2 Notation, conventions and known results}

The direction of the arrow on the index lines of a birdtrack encodes whether
the line acts on the vector space $V$ (arrow pointing from right to left) or
its dual $V^*$ (arrow pointing from left to right)\rf{PCgr}.
In general birdtracks represent primitive invariants of $\SUN$ over a mixed
algebra $\MixedPow{m}{n}$, where $V^*$ is the dual vector space of $V$.
Here only birdtracks acting on a space $\Pow{m}$ are considered (never on the
dual). As all arrows go from right to left, they can be dropped.

The permutations of $S_n$ are the \emph{primitive invariants}\rf{PCgr}
(of $\SUN$ over $\Pow{n}$).
The real subalgebra of $\Lin{\Pow{n}}$ that is spanned by these primitive
invariants is denoted $\API{\SUN,\Pow{n}}\subset\Lin{\Pow{n}}$.
{API} stands for ``\api.''
One
distinguishes

\begin{description}
  \item[Semi-standard irregular tableaux]
Each number
appears \emph{at most once} within a tableau.
  \item[Young tableaux]
The boxes are top- and left-aligned.
The numbers in the boxes to increase within each row from left to right and
within each column from top to bottom.
  \item[Amputated tableaux]
The \emph{column-amputated tableau} is obtained by removing all columns
which do not overlap with the given row.
The \emph{row-amputated tableau}  is obtained by removing all rows which do
not overlap with the given column.
\end{description}

$A\subset B $ denotes that a \emph{Hermitian} projection operator $A$ projects onto a
subspace completely contained in the image of a projection
operator $B$, \ie, $A\subset B$ if and only if
\beq
  \label{eq:OperatorInclusion1}
  A \cdot B = B \cdot A = A
\,.
\eeq
This simplification rule breaks down for the standard Young projection
operators whenever they are not Hermitian.

The main result of this paper are the two kinds of simplification rules
(cancelation or propagation) for birdtrack operators $O$ comprised of
symmetrizers and antisymmetrizers.

\subsubsection{%         \medskip\noindent Sect.~\emph{
3 Cancellation rules}

(1) Cancelation rules : Rules to determine whether certain symmetrizers or
antisymmetrizers within an operator $O$ are redundant, and thus can be
\emph{cancelled} from an operator. They can make a long expression
significantly shorter, and thus easier to work with.

The two main cancellation rules are
the {\em cancellation of wedged \Ypo s}, and
the {\em cancellation of wedged ancestor-operators}.

\subsubsection{%         \medskip\noindent Sect.~\emph{
3.1 Cancellation of wedged \Ypo s}

Theorem 1. Outside S and A, inside a \Ypo\

The example that starts with Eq.~(17), goes to the top of the page 10
motivates the general algorithm to remove inner symmetrizers.

extract points 1. 2. and 3. are general (not particular tableaux

Corollary 1. can always get rid of middle \Ypo\

\subsubsection{%         \medskip\noindent Sect.~\emph{
3.2 Cancellation of factors between bracketing sets}

Cancelation rules :
to  sets of symmetrizers or antisymmetrizers
through certain parts of the operator

Outer are two are A and S that belong to the same young projection operator
Eq.~(23)

Corollary 2. can always get rid of middle one (lack explicit formula
for the constants; make sure non-zero, at the end you can evaluate
the overall constant by other means (projection operator conditions))

Rest ensures that the constant is non-zero.

\subsubsection{%         \medskip\noindent Sect.~\emph{
Dimensional zeros}

\subsubsection{%         \medskip\noindent Sect.~\emph{
4 Propagation rules}

rules when things commute

(2){Propagation rules} :
when it is possible to commute (\emph{propagate}) a symmetrizer through an
antisymmetrizer (or vice versa)? Then the cancellation rules might be
applied, or features of a particular operator $O$, such as its Hermiticity
can be made explicit.




example Eq.~(45)

Eq.~(48) is easier

Eq.~(50)

rest of the section is ``Q. when you can commute?''

A. If can get rectangular tableaux, then it commutes

\subsubsection{%         \medskip\noindent Sect.~\emph{
4.1 Proof of Theorem 2}

Proves it - long and painful

Eq.~(57) tells it

\subsubsection{%         \medskip\noindent Sect.~\emph{
4.2 Proof of Theorem 3 (generalized propagation rules)}

Predrag suggestion: This seems not to need to be Young tableaux to work. A
streamlined derivation might be to prove it for individual transpositions.

\subsubsection{%         \medskip\noindent Sect.~\emph{
5 Conclusion}

Keppeler simplified (iteration easy to understand, simple proofs) but it is
inefficient - result balloons. This paper also recursive, but with the recursion
cut down drastically.

Fig.~5.2


\subsection{Compact Hermitian \Ypo s}
\label{s-AlcZei16-2}

Notes on Alcock-Zeilinger and Weigert\rf{AlcZei16-2}.

\Ypo s are (1) idempotent, (2) orthogonal and (3) complete.
But, as the symmetrizers and antisymmetrizers comprising a given Young
tableau do not necessarily commute,  \Ypo s are in general not Hermitian.

Keppeler and Sj{\"o}dahl\rf{KeppSjo14} were first to construct Hermitian
versions of Young projection operators in the birdtrack formalism, by an
iterative algorithm. However, the KS-operators soon become unwieldy and thus
impractical to work.

The construction algorithm presented here,
based on the simplification rules of \refref{AlcZei16-2},
leads to drastically more compact and explicitly Hermitian
expressions for the projection operators than the
KS-algorithm\rf{KeppSjo14}; an example is given
in \reffig{fig:MOLDAdvantage}.

\begin{figure}% [H]
%\newlength\foo
%\settototalheight\foo{\resizebox{\textwidth}{!}{%
%  \diagram[height=.15cm]{MOLDAdvantageEx2}
%}}
  \begin{center}
\resizebox{\textwidth}{!}{%
\begin{tikzpicture}[every node/.style={inner sep=1pt, outer sep=0pt}]
\node (KS) {\diagram[height=.15cm]{MOLDAdvantageEx2}};
\node (short) at ($(KS) +(0,-0.6cm)$)
      {\diagram[height=.15cm]{Simple-KS}};
\node (MOLD) at ($(short)
      +(0,-0.6cm)$)  {\diagram[height=.15cm]{MOLDAdvantageEx4}};
\draw[-{stealth}, line width=0.25pt] (KS) to (short);
\draw[-{stealth}, line width=0.25pt] (short) to (MOLD);
\node[scale=0.4] (Cancel) at ($(KS) +(0.7,-0.3cm)$) {Cancellation
  rules};
\node[scale=0.4] (Propagate) at ($(short) +(0.7,-0.3cm)$) {Propagation rules};
\end{tikzpicture}
}
  \end{center}
\caption{
(top)
A Hermitian birdtrack obtained by the iterative KS-algorithm\rf{KeppSjo14}.
Blow it up on the screen to see the details.
(middle)
The much shorter version obtained by application of the cancellation rules.
(bottom)
The explicitly symmetric (Hermitian) version achieved via the propagation
rules.
}
\label{fig:MOLDAdvantage}
\end{figure}


Repeated here are most of the Keppeler and Sj{\"o}dahl\rf{KeppSjo14} {\em
Hermitian \Ypo s}. Keppeler and Sj{\"o}dahl used iterative methods, see
\refref{AlcZei16-2} bottom p.~18


Eq.~(53) not obvious it is symmetric - \rf{AlcZei16-1} gives simplification rules,
dramatic simplification, see Fig.~5.2

\subsubsection{
%\subsubsection{%         \medskip\noindent Sect.~\emph{
3.3 KS Construction principle for Hermitian \Ypo s}


provides a direct route to bottom Fig.~5.2, paper proves that it really works

\Ypo s not being Hermitian has strange consequences.
Eq.~(12) not true, but for the Hermitian ones it is true.



\subsection{Transition operators}
\label{s-AlcZei16-3}

Notes on Alcock-Zeilinger and Weigert\rf{AlcZei16-3}.

The simplification rules of \refref{AlcZei16-2}
allow here a construction of transition operators between (Hermitian) Young
projection operators corresponding to equivalent irreducible representations
of $\SUN$, and an orthogonal basis for the algebra of invariants on
$\Pow{m}$.


completes the picture, the full algebra of invariants

\subsubsection{%         \medskip\noindent Sect.~\emph{
3 Young projection and transition operators}

gives the counting argument that the number of primitive invariants
equals the sum of diagonal operators and transition operators.

\subsubsection{%         \medskip\noindent Sect.~\emph{
5.2 A full orthogonal basis for {\api}}

write Clebsch, as in Eq.~(71)

Eq.~(73) transition operator between equivalent representations

they are unitary if restricted on the representations (top p 21)

together with the hermitian, they give you the full unitary basis

In Eq.~(55a) algebra is decomposed into subalgebras,
Eq.~(55b) is as simple as can be.

Eq.~(63) is non-Hermitian version

see and compare Fig.~2 (non Hermitian) and Fig.~3: (hermitian)
same as birdtracks.eu, but without the transition operators.

Dimension of the algebra goes factorially, so algorithm works up to 8 or 9
(all algebra in Mathematica, up to 8 on the laptop).

\subsection{Singlets}
\label{s-AlcZei16-4}


Notes on J. Alcock-Zeilinger and H. Weigert\rf{AlcZei16-4}.

This paper says that these projection operators give you all singlets.

The orthogonal basis of \refref{AlcZei16-3} is used to form a basis for the
singlet states necessary to determine all color neutral Wilson line
correlators. This has applications in  QCD, such as \refref{Marquet:2010cf}
and

Lappi \etal\rf{Lappi:2016gqe}
{\em {JIMWLK} evolution of the odderon}
