% compile by  pdflatex blog; biber blog
% GitHub cvitanov/reducesymm/dasgroup/negative.tex


\chapter{Negative dimensions}
\label{c-negative}

\begin{description}

\item[2011-07-27 PC]
Maru and Kitakado\rf{MarKit97} {\em Negative-dimensional group
extrapolation and dualities in {N=1} supersymmetric gauge theories},
 \arXiv{hep-th/9609230}

\item[2011-07-27 PC]
Astorino\rf{Astor10} {\em Kauffman knot invariant from $\mathrm{SO}(N)$
or $\mathrm{Sp}(N)$ Chern-Simons} theory and the {Potts} model writes
``Jones polynomial arises as special cases: Sp(2), SO(-2), and SL(2,R).
These results are confirmed and extended up to the second order, by means
of perturbation theory, which moreover let us establish a duality
relation between $SO(\pm N)$ and $Sp(\mp N)$ invariants. A correspondence
between the first orders in perturbation theory of SO(-2), Sp(2) or SU(2)
Chern-Simons quantum holonomy's traces and the partition function of the
Q=4 Potts model is built.''

\item[2016-02-06  Predrag] Khudaverdian and Ruben Mkrtchyan\rf{KhuMkr16},
{\em Universal volume of groups and anomaly of {Vogel}'s symmetry},
write: ``
We show that integral representation of universal
volume function of compact simple Lie groups gives rise to six analytic
functions on $CP^2$, which transform as two triplets under group of
permutations of Vogel's projective parameters.  This substitutes
expected invariance under permutations of universal parameters by more
complicated covariance.

 We provide an analytical continuation of these functions and particularly
calculate their change  under  permutations of parameters.  This last
relation is universal generalization, for an arbitrary simple Lie group
and an arbitrary point in Vogel's plane, of the Kinkelin's reflection
relation on Barnes' $G(1+N)$ function. Kinkelin's relation gives asymmetry
of the $G(1+N)$ function (which is essentially the volume function for $SU(N)$
groups)  under $N\leftrightarrow -N$ transformation (which is  equivalent
of the permutation of parameters, for $SU(N)$ groups), and coincides with
universal relation on permutations at the $SU(N)$ line on Vogel's plane.
These results are also applicable to universal partition function of
Chern-Simons theory on three-dimensional sphere.

This effect is  analogous to modular covariance, instead of invariance,
of partition functions of appropriate gauge theories under modular
transformation of couplings.
''

Mironov, Mkrtchyan and Morozov\rf{MiMkMo16}
{\em On universal knot polynomials}
``
We present a universal knot polynomials for 2- and 3-strand torus knots
in adjoint representation, by universalization of appropriate Rosso-Jones
formula. According to universality, these polynomials coincide with
adjoined colored HOMFLY and Kauffman polynomials at SL and SO/Sp lines on
Vogel's plane, respectively and give their exceptional group's
counterparts on exceptional line. We demonstrate that [m,n]=[n,m]
topological invariance, when applicable, take place on the entire Vogel's
plane. We also suggest the universal form of invariant of figure eight
knot in adjoint representation, and suggest existence of such
universalization for any knot in adjoint and its descendant
representations.
''

Have a look at

Mkrtchyan\rf{Mkrtchyan14} {\em On a {Gopakumar-Vafa} form of
partition function of {Chern-Simons} theory on classical and exceptional
lines}

Mkrtchyan and Veselov\rf{MkrVes12}
{\em Universality in {Chern-Simons} theory}

Mkrtchyan is not spring chicken. The funny thing is that, while there are
legions of young Witteninos, all this work seem to be carried out by old
men.



\item[2015-12-02  Predrag]
Email to Ruben:

I have not been working on these problems for a while, so apologies if
everything I write about here is something you already know. We have
uncovered more $N \to -N$ relationships than just $SO(-n)=Sp(n)$
(\HREF{http://birdtracks.eu/refs/index.html}{click here}).

In {\em Spinors in negative dimensions},
Phys. Scripta 26, 5 (1982) Tony Kennedy and I did it for ``spinsters".

In my (1981) {\em Negative dimensions and $E_7$ symmetry}\rf{NegDimE7}
 (as well as in Chapter 20. {\em $E_7$ family and its
negative-dimensional cousins} of the birdtracks book) I have a
negative-dimension mapping where $E_7$  appears as a
negative-dimensional relative of $SO(4)$.

That might be of interest for your Vogel song:)

\item[2016-12-03 Predrag]
Read

Maru and Kitakado\rf{MarKit97}
{\em Negative dimensional group extrapolation and a new chiral-nonchiral
duality in ${\mathcal N} = 1$ supersymmetric gauge theories}

\item[2016-12-03 Predrag]
Read

Garc{\'{\i}}a-Etxebarria andHeidenreich\rf{GarHei16}
{\em S-duality in N=1 orientifold {SCFTs}} write:``
this continuation relates for example an $SU(-N)$ gauge theory to an
$\tilde{SU(N)}$ gauge theory and is often referred to as negative rank duality
although the two related theories are generically not dual in the physical
sense. In particular, they have generically different anomalies.

`negative rank duality' is related to the formal replacement $N \to -N$ in the
corresponding quiver gauge theory (see, e.g., \refref{PCgr} and references
therein), reviewed in \refref{GaHeWr13}. Contrary to the connection between
S-duality and negative rank duality hypothesized in \refref{GaHeWr13},
there are many S-dualities for which related theories which are not negative
rank duals.
''

So this is now called `negative rank duality'.

In Appe.~G $\SOn{N} \leftrightarrow Sp(-N)$ duality leads them to conjecture
a new identity for elliptic hypergeometric integrals.

\item[2017-01-05 Predrag]
Read Dunne\rf{Dunne89}
{\em Negative-dimensional groups in quantum physics}


\end{description}


%\newpage %%%%%%%%%%%%%%%%%%%%%%%%%%%%%%%%%%%%%%%%%%%%%%%%
\printbibliography[heading=subbibintoc,title={References}]
