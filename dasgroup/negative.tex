% compile by  pdflatex blog; biber blog
% GitHub cvitanov/reducesymm/dasgroup/negative.tex


\chapter{Negative dimensions}
\label{c-negative}

\begin{description}

\item[2011-07-27 PC]
Maru and Kitakado\rf{MarKit97} {\em Negative-dimensional group
extrapolation and dualities in {N=1} supersymmetric gauge theories},
 \arXiv{hep-th/9609230}

\item[2011-07-27 PC]
Astorino\rf{Astor10} {\em Kauffman knot invariant from $\mathrm{SO}(N)$
or $\mathrm{Sp}(N)$ Chern-Simons} theory and the {Potts} model writes
``Jones polynomial arises as special cases: Sp(2), SO(-2), and SL(2,R).
These results are confirmed and extended up to the second order, by means
of perturbation theory, which moreover let us establish a duality
relation between $SO(\pm N)$ and $Sp(\mp N)$ invariants. A correspondence
between the first orders in perturbation theory of SO(-2), Sp(2) or SU(2)
Chern-Simons quantum holonomy's traces and the partition function of the
Q=4 Potts model is built.''

\item[2016-02-06  Predrag] Khudaverdian and Ruben Mkrtchyan\rf{KhuMkr16},
{\em Universal volume of groups and anomaly of {Vogel}'s symmetry},
write: ``
We show that integral representation of universal
volume function of compact simple Lie groups gives rise to six analytic
functions on $CP^2$, which transform as two triplets under group of
permutations of Vogel's projective parameters.  This substitutes
expected invariance under permutations of universal parameters by more
complicated covariance.

 We provide an analytical continuation of these functions and particularly
calculate their change  under  permutations of parameters.  This last
relation is universal generalization, for an arbitrary simple Lie group
and an arbitrary point in Vogel's plane, of the Kinkelin's reflection
relation on Barnes' $G(1+N)$ function. Kinkelin's relation gives asymmetry
of the $G(1+N)$ function (which is essentially the volume function for $SU(N)$
groups)  under $N\leftrightarrow -N$ transformation (which is  equivalent
of the permutation of parameters, for $SU(N)$ groups), and coincides with
universal relation on permutations at the $SU(N)$ line on Vogel's plane.
These results are also applicable to universal partition function of
Chern-Simons theory on three-dimensional sphere.

This effect is  analogous to modular covariance, instead of invariance,
of partition functions of appropriate gauge theories under modular
transformation of couplings.
''

Mironov, Mkrtchyan and Morozov\rf{MiMkMo16}
{\em On universal knot polynomials}
``
We present a universal knot polynomials for 2- and 3-strand torus knots
in adjoint representation, by universalization of appropriate Rosso-Jones
formula. According to universality, these polynomials coincide with
adjoined colored HOMFLY and Kauffman polynomials at SL and SO/Sp lines on
Vogel's plane, respectively and give their exceptional group's
counterparts on exceptional line. We demonstrate that [m,n]=[n,m]
topological invariance, when applicable, take place on the entire Vogel's
plane. We also suggest the universal form of invariant of figure eight
knot in adjoint representation, and suggest existence of such
universalization for any knot in adjoint and its descendant
representations.
''

Have a look at

Mkrtchyan\rf{Mkrtchyan14} {\em On a {Gopakumar-Vafa} form of
partition function of {Chern-Simons} theory on classical and exceptional
lines}

Mkrtchyan and Veselov\rf{MkrVes11}
{\em On duality and negative dimensions in the theory of {Lie} groups and 
symmetric spaces} (2011) 

Mkrtchyan and Veselov\rf{MkrVes12}
{\em Universality in {Chern-Simons} theory}

Mkrtchyan is not spring chicken. The funny thing is that, while there are
legions of young Witteninos, all this work seem to be carried out by old
men.



\item[2015-12-02  Predrag]
Email to Ruben:

I have not been working on these problems for a while, so apologies if
everything I write about here is something you already know. We have
uncovered more $N \to -N$ relationships than just $SO(-n)=Sp(n)$
(\HREF{http://birdtracks.eu/refs/index.html}{click here}).

In {\em Spinors in negative dimensions},
Phys. Scripta 26, 5 (1982) Tony Kennedy and I did it for ``spinsters".

In my (1981) {\em Negative dimensions and $E_7$ symmetry}\rf{NegDimE7}
 (as well as in Chapter 20. {\em $E_7$ family and its
negative-dimensional cousins} of the birdtracks book) I have a
negative-dimension mapping where $E_7$  appears as a
negative-dimensional relative of $SO(4)$.

That might be of interest for your Vogel song:)

\item[2016-12-03 Predrag]
Read

Maru and Kitakado\rf{MarKit97}
{\em Negative dimensional group extrapolation and a new chiral-nonchiral
duality in ${\mathcal N} = 1$ supersymmetric gauge theories}

\item[2016-12-03 Predrag]
Read

Garc{\'{\i}}a-Etxebarria and Heidenreich\rf{GarHei16}
{\em S-duality in N=1 orientifold {SCFTs}} write:``
this continuation relates for example an $SU(-N)$ gauge theory to an
$\tilde{SU(N)}$ gauge theory and is often referred to as negative rank duality
although the two related theories are generically not dual in the physical
sense. In particular, they have generically different anomalies.

`negative rank duality' is related to the formal replacement $N \to -N$ in the
corresponding quiver gauge theory (see, e.g., \refref{PCgr} and references
therein), reviewed in \refref{GaHeWr13}. Contrary to the connection between
S-duality and negative rank duality hypothesized in \refref{GaHeWr13},
there are many S-dualities for which related theories which are not negative
rank duals.
''

So this is now called `negative rank duality'.

In Appe.~G $\SOn{N} \leftrightarrow Sp(-N)$ duality leads them to conjecture
a new identity for elliptic hypergeometric integrals.

%\newpage % TEMPORARY
\item[2021-01-02 Predrag] to
\HREF{http://www.mathnet.ru/eng/person19342}
{Aleksei Petrovich Isaev} and
Aleksander Alekseevich Provorov:
    % isaevap@theor.jinr.ru,
    % aleksanderprovorov@gmail.com
 Very briefly, I've done much of the early work related to
your paper in 1970's and early 1980's, see \toBirdtracks{section.21.2}
{Sect.~21.2} {\em A brief history of exceptional magic} and other
history sections of the book. Whether you cite me or not is an
ethical question (should I cite
\HREF{http://birdtracks.eu/extras/Deligne96.pdf}
{the first person who did it}, or ?),
and it fundamentally does not matter in the long run.
However, if you are not already scribbling birdtracks on the side, I
strongly recommend them - many of your calculations are easier to survey
diagrammatically than in the tensor notation. I'll here just point out a
few examples, will gladly answer any specific questions.

\item[2020-09-31 Isaev and Provorov]
{\em Projectors on invariant subspaces of representations
{$\mathrm{ad}^{\otimes 2}$} of {Lie} algebras ${so(N)}$ and {$sp(2r)$}
and {Vogel} parametrization}\rf{IsaPro20},\\
\arXiv{2012.00746}:

Explicit formulae for the projectors onto invariant subspaces of the
$\mathrm{ad}^{\otimes 2}$ representation of the Lie algebras $so(N)$ and $sp(2r)$
have been found by means of the split Casimir operator. These projectors
have also been considered from the viewpoint of the universal complex
simple Lie algebra description by using the Vogel parametrisation.

In the case when $T=\mathrm{ad}$ is the adjoint representation,
construction of the projectors onto the invariant subspaces of the
representation $T\otimes T=\mathrm{ad}\otimes \mathrm{ad}$ is related to
the notion of \textit{the universal Lie algebra}, which was introduced by
P. Vogel\rf{PV99} (see also \refrefs{PD96,Lands06}). The universal Lie
algebra was supposed to be a model of all complex simple Lie algebras
$\mathcal{A}$.

Many quantities [...] can be expressed analytically\cite{MkSeVe12} as
functions of the three Vogel parameters \cite{PV99,lands01}, which take
specific values for each of the complex simple Lie
algebras~$\mathcal{A}$. For the simple Lie algebras of classical series
values of these parameters are given in \reftab{tab3}.

[... I]t was shown, that using the Vogel
parameters one can express the dimensions of an arbitrary representation
$T_{\lambda'}$ with the highest weight\rf{lands01}, the values of the
higher Casimir operators in the adjoint representation of a Lie algebra
$\mathcal{A}$ \cite{MkSeVe12}, as well that
the universal description of complex simple Lie algebras allows to
formulate some types of knot polynomials\rf{MiMkMo16}.

%%%%%%%%%%%%%%%%%%%%%%%%%%%%%%%%%%%%%%%%%%%%%%%%%%%%%%%%
\begin{table}
\small
\centering
\caption{\label{IsaPro20:tab3}
Vogel parameters $(\alpha,\beta,\gamma)$, defined modulo a common
multiplier and an arbitrary permutation, for simple Lie algebras of
classical series\rf{MkSeVe12}. $t\equiv \alpha+\beta+\gamma$ is the
dual Coxeter number.}
\vspace*{1mm}\tabcolsep=1.5em
\renewcommand{\arraystretch}{1.2}
\begin{tabular}{|c|c|c|c|c|c|}
\hline
      & Lie algebra & $\alpha$ & $\beta$ & $\gamma$ & $t$\\
\hline
$A_r$ & $sl(r+1)$ & $-2$ & $2$ & $r+1$ & $r+1$\\
\hline
$B_r$ & $so(2r+1)$ & $-2$ & $4$ & $2r-3$ & $2r-1$\\
\hline
$C_r$ & $sp(2r)$ & $-2$ & $1$ & $r+2$ & $r+1$\\
\hline
$D_r$ & $so(2r)$ & $-2$ & $4$ & $2r-4$ & $2r-2$\\
\hline
\end{tabular}
\end{table}
%%%%%%%%%%%%%%%%%%%%%%%%%%%%%%%%%%%%%%%%%%%%%%%%%%%%%%%%


They write the projectors onto invariant subspaces of the tensor product
of two adjoint representations of the Lie algebra $so(N,\mathbb{C})$ for
$N\ge 3$ and $sp(N,\mathbb{C})$ for $N=2r\ge 2$ in a unified form, with
the correspondence given by $N\to -N$. In terms of the Vogel parameters
their results agree with \refrefs{PV99,MkSeVe12,lands01}.
\bigskip

\item[2021-01-02 Predrag]
\texttt{Some Isaev and Provorov excerpts, with birdtracks.eu links.
Incomplete, can provide more if needed:}

$V_N$ : the space of $so(N,\mathbb{C})$, $sp(N,\mathbb{C})$ defining
representations.

The metric $c_{ij}=\epsilon c_{ji}$ : $\epsilon=+1$
for $so(N,\mathbb{C})$, and $\epsilon=-1$ for $sp(N,\mathbb{C})$.
\\{\bf Predrag}: see
\toBirdtracks{equation.12.0.3} {eq.~(12.3)}.

The inverse matrix $\bar{c}\,^{ij}$ of the metric $c$ :
$\bar{c}\,^{ik}c_{kj}=\delta^i_j$.

Can raise and lower indices:
$z_{i_1}{}^{j_2j_3\dots}=c_{i_1j_1}z^{j_1j_2j_3\dots}$
and~$z^{j_1}{}_{i_2i_3\dots}=\bar{c}\,^{j_1i_1}z_{i_1i_2i_3\dots}$.

$(e_s{}^r)^i{}_k=\delta^r_k\delta^i_s$.
Lowering the index $r$ yields
$(e_{sr})^i{}_k=c_{rk}\delta^i_s$.

The generators of the
Lie algebras $so(N,\mathbb{C})$ and~$sp(N,\mathbb{C})$:
\begin{align}
\label{eq2.2}
&M_{ij}=e_{ij}-\epsilon e_{ji},\\
&(M_{ij})^k{}_l=c_{jl}\delta^k_i-\epsilon c_{il}\delta^k_j=2\delta_{[i}^kc_{j)l},
\label{eq2.3}
\end{align}
where $[ij)$ implies antisymmetrization in the case of
algebra~$so(N,\mathbb{C})$ and symmetrization in the case of
algebra~$sp(N,\mathbb{C})$.
\\{\bf Predrag}: see
\toBirdtracks{equation.10.1.13} {eq.~(10.13)},
\toBirdtracks{equation.12.1.9} {eq.~(12.9)}.


The commutation relation
\begin{equation}
\label{eq2.4}
[M_{ij},M_{kl}]=c_{jk}M_{il}-\epsilon c_{ik}M_{jl}-\epsilon c_{jl}M_{ik}+c_{il}M_{jk}=X_{ij,kl}{}^{mn}M_{mn},
\end{equation}
{\bf Predrag}: see
\toBirdtracks{equation.4.6.54} {eq.~(4.54)}.

The structure constants of the Lie algebras \eqref{eq2.2}:
\begin{equation}
\label{eq2.5}
X_{ij,kl}{}^{mn}=c_{jk}\delta_{i}^{[m}\delta_{l}^{n)}-\epsilon c_{ik}\delta_{j}^{[m}\delta_{l}^{n)}-\epsilon c_{jl}\delta_{i}^{[m}\delta_{k}^{n)}+c_{il}\delta_{j}^{[m}\delta_{k}^{n)}.
\end{equation}
or, concisely:
\begin{equation}
\label{eq2.6}
X_{i_1i_2,j_1j_2}{}^{k_1k_2}
=4\Sym^\epsilon_{1\leftrightarrow 2}(c_{i_2j_1}\delta_{i_1}^{k_1}\delta_{j_2}^{k_2}),
\end{equation}
where $\Sym^\epsilon_{1\leftrightarrow 2}$ denotes (anti)symmetrization over
$(i_1,i_2)$, $(j_1,j_2)$, $(k_1,k_2)$ pairs of indices. For example,
 $\Sym^\epsilon_{1\leftrightarrow 2}(x_{i_1 i_2}) \equiv
 x_{i_1 i_2} - \epsilon x_{i_2 i_1}$.

$\mathfrak{g}_N^\epsilon$ is the algebra $so(N,\mathbb{C})$ if $\epsilon=+1$,
 and $sp(N,\mathbb{C})$ if $\epsilon=-1$.


\bigskip

Let $\mathcal{A}$ be a simple Lie algebra with the basis elements $X_a$
and the structure relations
\begin{equation}
\label{eq3.1}
[X_a,X_b]= C{}_{ab}{}^dX_d,
\end{equation}
where $C{}_{ab}{}^d$~are the structure constants. The Cartan-Killing metric is defined in the standard fashion:
\begin{equation}
\label{eq3.2}
{\sf g}_{ab} \equiv C^{d}_{ac} \, C^{c}_{bd} =
\tr(\mathrm{ad}(X_a)\cdot \mathrm{ad}(X_b)),
\end{equation}
 and the structure constants $C_{abc} \equiv C^{d}_{ab} \, {\sf g}_{dc}$
 are antisymmetric in the indices $(a,b,c)$.
$\mathcal{U}(\mathcal{A})$ denotes the universal enveloping algebra of $\mathcal{A}$. Consider the operator
\begin{equation}
\label{eq3.3}
\widehat{C}  = {\sf g}^{ab} X_a \otimes
  X_b \in \mathcal{A}  \otimes  \mathcal{A}
   \subset \mathcal{U}(\mathcal{A}) \otimes\mathcal{U}(\mathcal{A}),
\end{equation}
 where the matrix $\|{\sf g}^{ab}\|$ is the inverse matrix of $\|{\sf g}_{ab}\|$, which, in turn, defines the Cartan-Killing metric \eqref{eq3.2}:
\begin{equation}
\label{eq3.4}
 {\sf g}^{ab}{\sf g}_{bc}=\delta^a_c.
 \end{equation}
 The operator $\widehat{C}$ is called the \textit{split} (or \textit{polarised}) \textit{Casimir operator of the Lie algebra} $\mathcal{A}$.
This operator is related to the usual quadratic Casimir operator
\begin{equation}
\label{eq3.5}
 C_{(2)} = {\sf g}^{ab}X_a \cdot X_b
\end{equation}
according to the formula
\begin{equation}
\label{eq3.6}
 \Delta(C_{(2)}) = C_{(2)} \otimes I + I \otimes C_{(2)} + 2 \, \widehat{C} ,
\end{equation}
where $\Delta$~is the comultiplication:
\begin{equation}
\label{eq3.7}
\Delta(X_a) = (X_a \otimes I + I \otimes X_a).
\end{equation}
{\bf Predrag}: ``Split'' Casimir, ``polarization'' is simply the
insertion of the adjoint irrep, \toBirdtracks{equation.4.4.35}
{eq.~(4.35)} in any invariant tensor, followed by contraction (the gets
us to quadratic Casimirs $\widehat{C}$ and $C_{(2)}$) of the free adjoint
leg. In quantum mechanics this shows up when evaluating the quadratic
Casimir for the total angular moment as a sum of the angular and spin
angular momenta, $J=L\oplus{S}$, \toBirdtracks{equation.7.4.22}
{eq.~(7.22)}.

The characteristic identity:
\begin{equation}
\label{eq3.11}
{\prod_{\lambda}}^{\, \prime} \biggl(T(\widehat{C})-\frac{1}{2}
(c_2^{(\lambda)} -c_2^{(\lambda_1)}
-c_2^{(\lambda_2)})\biggr) = 0
\,.
\end{equation}
The product ${\prod}^{\, \prime}$ is performed only over the weights
$\lambda$  which correspond to different eigenvalues $c_2^\lambda$.

They give explicit expressions for the split
Casimir operator $T(\widehat{C} )=(T_1\otimes T_2)(\widehat{C} )$ for
orthogonal and symplectic Lie algebras in the case of $T_1$ and
$T_2$ are the adjoint representations: $T_1 = T_2 = \mathrm{ad}$. Then
the characteristic identity \eqref{eq3.11} can be rewritten as
\begin{equation}
\label{eq3.12}
{\prod_{\lambda}}^{\, \prime}
\biggl(\mathrm{ad}^{\otimes 2}(\widehat{C})-\frac{1}{2}(c_2^{(\lambda)}-2c_2^{(\lambda_{\mathrm{ad}})})\biggr)=0,
\end{equation}
where $\lambda_{\mathrm{ad}}$~is the highest weight of the adjoint
representation of~$\mathcal{A}$. Using the characteristic
identity \refeq{eq3.12}, construct
the projectors on the invariant
subspaces~$\mathcal{V}_\lambda$ of the irreducible representations
$T_\lambda$, comprising the decomposition of $\mathrm{ad}^{\otimes 2}$.

{\bf Predrag}: Isaev and Provorov sect.~4  {\em The decomposition of
$\mathrm{ad}^{\otimes 2}(\mathfrak{g}_N^\epsilon)$ into the irreducible
representations} general procedure of constructing characteristic
equations feels very much like my own calculations, but I have not tried
to compare them in detail. I fully agree that the way to get irreps is
through characteristic equations, and projectors constructed from the -
that's the main difference of my book from any other group theory book. I
believe the birdtracks calculation is much shorter and faster, I
recommend going through it.

\begin{equation}
\label{eq4.32}
\begin{aligned}
&\dim(V_{a_1})=\tr \mathcal{P}_1=\frac{1}{8}M(M-1)(M+2)(M-3),\\
&\dim(V_{a_2})=\tr \mathcal{P}_2=\frac{1}{2}M(M-1),\\
&\dim(V_{a_3})=\tr \mathcal{P}_3=1,\\
&\dim(V_{a_4})=\tr \mathcal{P}_4=\frac{1}{12}M(M+1)(M+2)(M-3),\\
&\dim(V_{a_5})=\tr \mathcal{P}_5=\frac{1}{24}M(M-1)(M-2)(M-3),\\
&\dim(V_{a_6})=\tr \mathcal{P}_6=\frac{1}{2}(M-1)(M+2).
\end{aligned}
\end{equation}
The sum
\begin{equation}
\label{eq4.33}
\sum_{i=1}^6\tr\mathcal{P}_i=\frac{M^2}{4}(M-1)^2=\tr \mathbf{I}
\end{equation}
is the dimension of $\mathfrak{g}_N^\epsilon\otimes \mathfrak{g}_N^\epsilon$,
as it should be.

%%%%%%%%%%%%%%%%%%%%%%%%%%%%%%%%%%%%%%%%%%%%%%%%%%%%%%%%
\begin{table}
\small\tabcolsep=1.5em
\centering
\renewcommand{\arraystretch}{1.18}
\vspace*{1mm}
\caption{\label{IsaPro20:tab1}
Dimensions of irreducible representations for algebra~$so(N)$.}
\vspace*{1mm}
\begin{tabular}{|c|c|c|c|c|c|c|}
\hline
$N$ & $\dim_1$ & $\dim_2$ & $\dim_3$ & $\dim_4$ & $\dim_5$ & $\dim_6$\\
\hline
$5$ & $35$ & $10$ & $1$ & $35$ & $5$ & $14$\\
\hline
$7$ & $189$ & $21$ & $1$ & $168$ & $35$ & $27$\\
\hline
$9$ & $594$ & $36$ & $1$ & $495$ & $126$ & $44$\\
\hline
$10$ & $945$ & $45$ & $1$ & $770$ & $210$ & $54$\\
\hline
$11$ & $1434$ & $55$ & $1$ & $1144$ & $330$ & $65$\\
\hline
\end{tabular}

\caption{\label{IsaPro20:tab2}
Dimensions of irreducible representations for algebra~$sp(N)$.}
\vspace*{1mm}
\begin{tabular}{|c|c|c|c|c|c|c|}
\hline
$N$ & $\dim_1$ & $\dim_2$ & $\dim_3$ & $\dim_4$ & $\dim_5$ & $\dim_6$\\
\hline
$4$ & $35$ & $10$ & $1$ & $14$ & $35$ & $5$\\
\hline
$6$ & $189$ & $21$ & $1$ & $90$ & $126$ & $14$\\
\hline
$8$ & $594$ & $36$ & $1$ & $308$ & $330$ & $27$\\
\hline
$10$ & $1430$ & $55$ & $1$ & $780$ & $715$ & $44$\\
\hline
$12$ & $2925$ & $78$ & $1$ & $1650$ & $1365$ & $65$\\
\hline
\end{tabular}
\end{table}
%%%%%%%%%%%%%%%%%%%%%%%%%%%%%%%%%%%%%%%%%%%%%%%%%%%%%%%%
%
\refTab{IsaPro20:tab1}, \reftab{IsaPro20:tab2} list irrep dimensions of $so(N)$, $sp(N)$
for \eqref{eq4.32}, several $N$ values. The dimensions agree with Yamatsu
tables\rf{Yamatsu15}, {\bf Predrag}: and, of course, with the much
earlier Cvitanovi{\'c}\rf{PCgr}. The characteristic identities and the
dimensions \refeq{eq4.32} for~$so(N,\mathbb{C})$ and~$sp(N,\mathbb{C})$
are related by $N\to-N$. This manifests a duality between the algebras
$so(N)$ and $sp(N)$.
{\bf Predrag}: here Isaev and Provorov write ``(see \cite{PCgr} for more
details)'', the only reference to my book in their paper.

{\bf Predrag}: see \toBirdtracks{table.10.3} {table~10.3}. There was no
point for me of publishing the corresponding table for $sp(N,\mathbb{C})$
as it turned out to be a step-by-step repetition of $so(N,\mathbb{C})$
calculations, see \toBirdtracks{chapter.13} {chapter.13}. That is the
reason that I did not want to use the metric $c_{ij}=\epsilon c_{ji}$,
$\epsilon=\pm1$ in \refeq{eq2.3}. The $N\to -N$ is so easy to prove
that it is best to do all calculations for $so(N,\mathbb{C})$ only, than
apply the duality to get all $sp(N,\mathbb{C})$ formulas.

The characteristic identity for the split Casimir operator
$\widehat{C}_{\mathrm{ad}}=\widehat{C}_+ + \widehat{C}_-$:
\begin{equation}
\label{eq4.39}
\widehat{C}_{\mathrm{ad}}\biggl(\widehat{C}_{\mathrm{ad}}+\frac{1}{2}\biggr) (\widehat{C}_{\mathrm{ad}}+1)
\biggl(\widehat{C}_{\mathrm{ad}}+\frac{1}{3}\biggr)\biggl(\widehat{C}_{\mathrm{ad}}-\frac{1}{6}\biggr)=0 .
\end{equation}

In order to build the projectors onto the invariant subspaces $V_{35}$ and $V_{35'}$
we introduce an operator $E_8\colon V_{8}^{\otimes 4}\to V_{8}^{\otimes 4}$ with the following components:
\begin{equation}
(E_8)^{i_i\dots i_4}{}_{j_1\dots j_4}
 \equiv \frac{1}{4!}\,\varepsilon^{i_i\dots i_4}{}_{j_1\dots j_4}=\frac{1}{4!}\,
 c_{j_1i_5} \cdots c_{j_1i_8}\varepsilon^{i_1\dots i_8}.
\end{equation}
Here $\varepsilon^{i_1\dots i_8}$~is the completely antisymmetric invariant tensor, $\varepsilon^{12345678}=1$. This operator, being built from the invariant tensors $\varepsilon^{i_1\dots i_8}$ and $c_{i_1i_2}$, is $\mathrm{ad}$-invariant. Taking into account, that in the case of algebra $so(8)$ we have $c_{i_1i_2}=\delta_{i_1i_2}$, we do not distinguish between upper and lower indices.

the full system of mutually orthogonal and primitive projectors onto the spaces of the irreducible subrepresentations in~$\mathrm{ad}^2(so(8))$:
\begin{equation}
\begin{alignedat}{2}
&\mathcal{P}_1^\prime=\frac{1}{2}(\mathbf{I}-\mathbf{P})+2\widehat{C}_-, &\qquad &\dim=350,\\
&\mathcal{P}_2^\prime=-2\widehat{C}_-, &\qquad &\dim=28,\\
&\mathcal{P}_3^\prime=\frac{1}{28}\mathbf{K}, &\qquad &\dim=1,\\[1mm]
&\mathcal{P}_6|^{}_{M=8}=\frac{1}{6}(\mathbf{I}+\mathbf{P})-2\widehat{C}_+-\frac{1}{12}\,\mathbf{K}-A_4,
&\qquad &\dim=35,\\[1mm]
&\mathcal{\overline{P}}_2=\frac{1}{2}(A_4+E_4), &\qquad &\dim=35,\\[1mm]
&\mathcal{\overline{P}}_3=\frac{1}{2}(A_4-E_4), &\qquad &\dim=35,\\[1mm]
&\mathcal{P}_5^\prime=\frac{1}{3}(\mathbf{I}+\mathbf{P})+2\widehat{C}_++\frac{1}{21}\,\mathbf{K}, &\qquad &\dim=300.
\end{alignedat}
\end{equation}

According to \refrefs{PV99,Lands06}
\\{\bf Predrag}: rederived decades after
my work, and aware of my work. Lansberg is a friend:)
\\
the symmetric
part (in all the cases of simple Lie algebras of classical series, apart
from $sl(2,\mathbb{C})\cong sp(2,\mathbb{C})\cong so(3,\mathbb{C})$,
$sl(3,\mathbb{C})$ and $so(8,\mathbb{C})$), in turn, decomposes into four
irreducible subrepresentations - a singlet $T_0$ and three irreducible
representations denoted by $Y_2^{(\alpha)}$, $Y_2^{(\beta)}$ and
$Y_2^{(\gamma)}$, where $\alpha$, $\beta$ and $\gamma$~are the Vogel
parameters of the algebra $\mathcal{A}$ from \reftab{IsaPro20:tab3}. The
dimensions of these four representations as well as the eigenvalues of
the quadratic Casimir operator $C_{(2)}$ (which was defined
in~\eqref{eq3.5}), acting on the spaces of these representations, equal
\cite{PV99,lands01}
\\({\bf Predrag}: rederived decades after me)
\begin{equation}
\label{eq5.1}
\begin{alignedat}{2}
&\dim T_0(\mathcal{A})=1, &\qquad &c_2^0=0,\\
&\dim Y_2^{(\alpha)}(\mathcal{A})=-\frac{(3\alpha-2t)(\beta-2t)(\gamma-2t)t(\beta+t)(\gamma+t)}{\alpha^2(\alpha-\beta)\beta(\alpha-\gamma)\gamma}, &\qquad &c_2^\alpha=2-\frac{\alpha}{t}\,,\\
&\dim Y_2^{(\beta)}(\mathcal{A})=-\frac{(3\beta-2t)(\alpha-2t)(\gamma-2t)t(\alpha+t)(\gamma+t)}{\beta^2(\beta-\alpha)\alpha(\beta-\gamma)\gamma}, &\qquad &c_2^\beta=2-\frac{\beta}{t}\,,\\
&\dim Y_2^{(\gamma)}(\mathcal{A})=-\frac{(3\gamma-2t)(\beta-2t)(\alpha-2t)t(\beta+t)(\alpha+t)}{\gamma^2(\gamma-\beta)(\beta(\gamma-\alpha)\alpha}, &\qquad &c_2^\gamma=2-\frac{\gamma}{t}\,,
\end{alignedat}
\end{equation}
where $c_2^0$, $c_2^\alpha$, $c_2^\beta$ and $c_2^\gamma$~are the
eigenvalues of the operator $C_{(2)}$, acting in the representations
$T_0$, $Y_2^{(\alpha)}$, $Y_2^{(\beta)}$, $Y_2^{(\gamma)}$ respectively.

{\bf Predrag}: For $so(8)=D_4$ see
\toBirdtracks{table.17.1} {table~17.1},
\toBirdtracks{table.17.2} {table~17.2}.
The above projectors should be special cases of
those in \toBirdtracks{section.17.1}
{sect.~17.1} {\em Two-index tensors},
and \toBirdtracks{equation.21.1.4} {eq.~(21.4)},
but I have not checked.

I knew that $so(8)=D_4$ has triality and is exceptional, presumably have
a pile of calculations in my office (1/3 of the country away), but saw no
need for including them in the birdtracks.eu.

{\bf Predrag}: it's getting late, so I stop here, but can provide more
detail if needed.

\item[2022-03-14 Alexey Isaev and Sergey Krivonos]
If you have time and desire, please take a look to our new paper
Isaev and Krivonos\rf{IsaKro21}
{\em Split Casimir operator for simple Lie algebras, solutions of {Yang-Baxter} equations and {Vogel} parameters}
(2021).

``
we consider a very particular problem of constructing invariant
projectors in representation spaces of $T^{\otimes2}$, where $T$ is the
defining, or adjoint representation but for all simple Lie algebras g.
Our approach is closely related to the one outlined in \refref{PCgr}. In
\refref{PCgr}, such invariant projectors were obtained in terms of
several special invariant operators and the calculations were performed
using a peculiar diagram technique.
''

``peculiar", ha?

This article reproduces much of my book, have to check whether it goes further
in getting to the $E_8$?


\item[2022-03-14 Predrag]
Isaev and Provorov\rf{IsaPro22}
{\em Split {Casimir} operator and solutions of the {Yang-Baxter} equation
for the {osp(M|N) and sl(M|N) Lie} superalgebras, higher {Casimir}
operators, and the {Vogel} parameters}
(2022)

Refers to me only once. Recheck to see whether this is fair...


\item[2017-01-05 Predrag]
Read Dunne\rf{Dunne89}
{\em Negative-dimensional groups in quantum physics}

\item[2017-02-27 Predrag]
Jim Gates is working on holoraumy (Greek holo, German raum, American y)
representations - the lowest one is 36864\dmn\ (see \arXiv{1802.02890},
\arXiv{1712.07826}, \arXiv{1210.0478}, etc). The group is finite, so it
has to be in The Atlas. $36,864=192^2=2^{12}\times3^2$.
Here are some candidates:

\arXiv{1409.6055} see Table 9: Conjugacy classes in $O(L)$.

\arXiv{math/0308069}  number 36864 shows in a numerator in a formula.

PhD Thesis
\HREF{https://www.research.manchester.ac.uk/portal/files/59970048/FULL_TEXT.PDF}
{On Commuting Involution Graphs of Certain Finite Groups}:  number 36864
shows many times in many tables.

In \HREF{http://www.ams.org/journals/proc/2010-138-06/S0002-9939-10-10322-0/S0002-9939-10-10322-0.pdf} {Tensoring generalized characters
with the Steinberg character}
$36864$ shows up in a discriminant associated with exceptional group
$G_2$.

\HREF{doi.org/10.1016/S0747-7171(02)00133-5} {Cannon and Holt}
{\em Automorphism group computation and
isomorphism testing in finite groups} Table~1 has lots of 36,864

\item[2018-10-17 Predrag]
Caputa, de Mello Koch and Diaz\rf{CaMeDi13}
{\em Operators, correlators and free fermions for {SO(N) and Sp(N)}}
write: ``
we obtain a complete generalization of these results to Sp(N) gauge theory by
proving that the finite N physics of SO(N) and Sp(N) gauge theory are related
by exchanging symmetrizations and antisymmetrizations and replacing N by -N.
''

There is lots of stuff in this paper, hard to understand without some serious study.
But they do not that large N limits of the SO(N) and Sp(N) gauge theories
are non-planar, and deal with that. Interesting.



\end{description}


%\newpage %%%%%%%%%%%%%%%%%%%%%%%%%%%%%%%%%%%%%%%%%%%%%%%%
\printbibliography[heading=subbibintoc,title={References}]
