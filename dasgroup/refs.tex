% reducesymm/dasgroup/refs.tex
% was dasgroup/book/chapter/refs.t\exp of 2015-12-20

% Predrag         1dec2016
% Predrag 		  8apr2011
% Predrag         5feb2008
% Predrag         7jul2007
% Tony           14sep2006
% zazy           22Nov2005
% zazy           17Nov2005
% zazy           15Nov2005
% zazy           14Nov2005
% Zazy           10Nov2005
% Zazy           19oct2005
% Predrag        8oct2005
% Predrag        27may2004
% Predrag        29feb2004
% Henriette      references added
% Landsberg      references updated
% Predrag        7may2003
% Dorte          15jan2003
% Predrag        7jan2003
% Dorte          3jan2003
% Dorte          30dec2002
% Predrag        9dec2002
% Dorte          3dec2002
% Predrag        1dec2002
% Westbury       references added
% Predrag        1nov2002
% Dorte          22oct2002
% Predrag        16aug2002
% Dorte          31jul2002
% Predrag        03jun2002
% Dorte          14may2002
% Predrag        19apr2002
% Predrag        25feb2000
% Predrag        15/5-96


\PC{8apr2011 remember: transfer text from PRD14 refs.}

%%AAAAAAAAAAAAAAAAAAAAAAAA

\refref{Abdess02} A. Abdesselam,
    ``A physicist's proof of the Lagrange-Good
      multivariable inversion formula,''
    {\em J. Phys. A \bf  36}, 9471 (2002); % 9471-9477
    \arXiv{math.CO/0208174}.

\refref{Abdess03a} A. Abdesselam,
    {``The Jacobian conjecture as a problem
    of perturbative quantum field theory,''}
    {\em Ann. Inst. Henri Poincar{\'e}  \bf 4}, 199 (2003); % 199-215,
    \arXiv{math.CO/0208173}.

\refref{Abdess03} A.~Abdesselam,
    {``Feynman diagrams in algebraic combinatorics,''}
    \\
    {\em S\'em. Lothar. Combin. \bf 49}, article B49c (2003);
        \arXiv{math.CO/0212121}.

\refref{Abdess04} A. Abdesselam and J.~Chipalkatti,
    {``A regularity result for a locus of Brill type,''}
        \arXiv{math/0405236}.
%@unpublished={
%   title = { A regularity result for a locus of Brill type },
%   authors = { Abdesselam, Abdelmalek and Chipalkatti, Jaydeep },
%   abstract = {   Let n,d be positive integers, with d even (say d=2e). Let X_(n,d) denote the
%locus of degree d hypersurfaces in P^n which consist of two e-fold hyperplanes.
%We bound the regularity of the ideal of this variety. Moreover, we show that
%this variety is r-normal for r at least 2. The proof of the latter part is is a
%result of a tripartite collaboration of algebraic geometry, classical invariant
%theory and theoretical physics. It is executed by reducing the question to a
%combinatorial calculation involving Feynman diagrams and hypergeometric
%functions.
% },
%   URL = { http://arXiv.org/abs/math/0405236 },
%}

\refref{Adler1977} S.~L.~Adler, J.~Lieberman, and Y.~J.~Ng,
``Regularization of the stress-energy tensor for vector and scalar particles
propagating in a general background metric,''
{\em Ann. Phys. \bf 106}, 279 (1977).

\refref{Agrawala1968}
V.~K.~Agrawala and J.~G.~Belinfante,
``Graphical formulation of recoupling for any compact group,''
{\em Ann.  Phys.  (N. Y.) \bf 49}, 130 (1968).
%% Annals Phys.

\refref{Agrawala1980}
V.~K.~Agrawala,
``Micu type invariants of exceptional simple Lie algebras,''
{\em J.~Math.~Phys.  \bf 21}, 1577 (1980).

\refref{Aharony:1999ti}
O.~Aharony, S.~S.~Gubser, J.~Maldacena, H.~Ooguri, and Y.~Oz,
``Large $N$ field theories, string theory and gravity,''
{\em Phys.~Rept.  \bf 323}, 183 (2000);
\arXiv{hep-th/9905111}.
%%CITATION = HEP-TH 9905111;%%
%\href{http://www.slac.stanford.edu/spires/find/hep/www?eprint=HEP-TH/9905111}{SPIRES}

% no birtracks here, inspate Penrose crediting it
% \refref{Aitken39} A.~C.~Aitken,
% {\em Determinants \&\ Matrices}
% (Oliver \&\ Boyd, Edinburgh 1939).

\refref{Allison76} B.~N.~Allison,
        ``A construction of Lie algebras from $J$-ternary algebras,''
        {\em Amer. J. Math. \bf 98}, 285 (1976).  % 285-294

\refref{Andreev1967}
    E.~M.~Andreev, E.~B.~Vinberg, and A.~G.~Elashvili,
    {\em Funct.~Analysis and Appl.  \bf 1}, 257 (1967).

%\refref{Andrews76} G.~E. Andrews,
%    {\em The Theory of Partitions}
%    % , Encyclopedia of Mathematics and its Applications
%    (Addison-Wesley, Reading, MA 1976).
%
%\refref{Angelop01} E. Angelopoulos,
%    ``Classification of simple Lie algebras,''
%    {\em Panamerican Math. Jour. \bf 2}, 65 (2001). % 65-79

%%BBBBBBBBBBBBBBBBBBBBBBBBBBB


\refref%[Bae02]
        {MR1886087} J.~C. Baez,
    ``The octonions,''
    {\em Bull. Amer. Math. Soc.  \bf 39}, 145 (2002); % 145-205
    \weblink{math.ucr.edu/home/baez/octonions/octonions.html}

%\refref{BarNatan95} D. Bar-Natan,
%    ``Weights of Feynman diagrams and the Vassiliev knot invariants,''
%    \weblink{www.math.toronto.edu/$\sim$drorbn/LOP.html}
%    (unpublished preprint, 1991);
%    ``On the Vassiliev knot invariants,''
%    {\em Topology \bf 34}, 423 (1995). %-472.
%% \PC{fix missing tildes}

\refref{Barnes1972} K.~J.~Barnes and R.~Delbourgo,
    {\em J.~Phys. A  \bf 5}, 1043 (1972).

%\refref%[BS02]
%        {barton00} C.~H. Barton and A.~Sudbery,
%    ``Magic squares of Lie algebras'' (2000), unpublished;
%    \arXiv{math.RA/0001083}.
%% Hendryk: you should have a look at
%% for the symmetries of the magic triangle.
%
\PC{update}

%\refref%[BS02]
%        {barton02} C.~H. Barton and A.~Sudbery,
%    ``Magic squares and matrix models of Lie algebras,''
%        {\em Adv. in Math. \bf 180}, 596 (2003); % 596-647
%     \arXiv{math.RA/0203010}

\refref{Behrends1962}
R.~E.~Behrends, J.~Dreitlein, C.~Fronsdal, and B.~W.~Lee,
``Simple groups and strong interaction symmetries,''
{\em Rev.~Mod.~Phys. \bf 34}, 1 (1962).

% same as Behrends1962: \refref{Behrends1966} R.~E.~Behrends {\em et al}. (1966).

\refref{Belinfante1965}  J.~G.~Belinfante,
    {\em Group Theory}, lecture notes
    (Carnegie Tech 1965), unpublished.
    % http://www.math.gatech.edu/$\sim$belinfan/
    % belinfan@math.gatech.edu

%% Annals Phys. 49, 130

\refref{Berezin} F.~A. Berezin,
    {\em Introduction to superanalysis}
    %, Mathematical Physics and Applied Mathematics, 9,
    (Reidel, Dordrecht 1987).

%\refref{Bessen98}
%C.~Bessenrodt,
%{\em Ann. of Comb. \bf 2}, 103 (1998);
%the hook rule %\refeq{Skdim}
%follows directly from Theorem~1.1.

\refref{Bincer1993}
A.~M.~Bincer and K.~Riesselmann,
``Casimir operators of the exceptional group $G_2$,''
{\em J.~Math.~Phys. \bf 34}, 5935 (1993).

\refref{Bincer1994} A.~M. Bincer,
    ``Casimir operators of the exceptional group $F_4$:
    The chain $B_4$ contains $F_4$ contains $D_{13}$,''
    {\em J.~Phys.~A  \bf 27}, 3847 (1994);
    \arXiv{hep-th/9312148}.

\refref{Biritz1875} H.~Biritz,
    {\em Nuovo Cimento \bf 25B}, 449 (1973).

\refref{Blinn02} J.~F.~Blinn,
    ``Quartic discriminants and tensor invariants,''
{\em IEEE Computer Graphics and Appl. \bf 22}, 86 (2002). % pp.86-91

%\refref{Boerner70} H.~Boerner,
%    {\em Representations of Groups}
%    (North-Holland, Amsterdam 1970).

\refref{Boos98} D.~Boos,
        ``Ein tensorkategorieller Zugang zum Satz von Hurwitz
        (A tensor-categorical approach to Hurwitz's theorem),''
    Diplomarbeit ETH Zurich (March 1998), available at
% this is old: http://www.math.ohio-state.edu/$\sim$rost/tensors.html
\HREF{http://www.math.uni-bielefeld.de/~rost/tensors.html}
     {www.math.uni-bielefeld.de/$\sim$rost/tensors.html}

\refref{Bose1976} A.~K.~Bose,
    {\em Computer Phys. Comm.  \bf 11}, 1 (1976).

\refref{Bourbaki} According to a reliable source,
    Bourbaki is dead. The old soldier will not be missed.

\refref{Brauer1935} R.~Brauer,
    ``Sur les invariants int\'egraux des vari\'et\'es
    repr\'esentatives des groupes de Lie simples clos,''
    {\em C.~R. Acad. Sci. Paris \bf 201}, 419 (1935). % 419-421.
% on the Betti numbers of the classical Lie groups,
% purely algebraic treatment based on invariant theory.
%%http://stills.nap.edu/readingroom/books/biomems/rbrauer.html

\refref{Brauer1937} R.~Brauer,
    ``On algebras which are connected with
      the semisimple continuous groups,''
    {\em Ann. Math. \bf 38}, 854 (1937).
% Brauer algebras = birdtracks for SO/Sp:1

\refref{Brig71} J.~S.~Briggs,
    {\em Rev.~Mod.~Phys. \bf 43}, 189 (1971).

\refref{Brink68} D.~M.~Brink and G.~R.~Satchler,
    {\em Angular Momentum}
    (Oxford Univ. Press, Oxford 1968).

\refref{Brown1969} R.~B.~Brown,
    {\em J. Reine Angew. Math.  \bf 236}, 79 (1969).

\refref{Butera:1980na} P.~Butera, G.~M.~Cicuta, and M.~Enriotti,
    ``Group weight and vanishing graphs,''
    {\em Phys.  Rev.  D  \bf 21}, 972 (1980).
%%CITATION = PHRVA,D21,972;%%
%\href{http://www.slac.stanford.edu/spires/find/hep/www?j=PHRVA%2cD21%2c972}{SPIRES}

\refref{Butler1978} P.~H.~Butler, R.~W.~Haase, and B.~G.~Wybourne,
    ``Calculation of $3jm$ factors and the matrix elements
      of $E_7$ group generators,''
    {\em Austr.~J.~Phys. \bf 32}, 137 (1979).  % 137-154

%%CCCCCCCCCCCCCCCCCCCCCCCCCCCC


\refref{Cahalan1976}
R.~F.~Cahalan and D.~Knight,
``Construction of planar diagrams,''
{\em Phys.~Rev.  D \bf 14}, 2126 (1976).

\refref{Callan1976}
C.~G.~Callan, N.~Coote, and D.~J.~Gross,
{\em Phys.~Rev.  D \bf 13}, 1649 (1976).

\refref{Canning1973}
G.~P.~Canning,
{\em Phys.~Rev. D \bf 8}, 1151 (1973).

\refref{Canning1975}
G.~P.~Canning,
{\em Phys.~Rev.  D \bf 12}, 2505 (1975).

%\refref{Canning:1978ee}
%G.~P.~Canning,
%``Diagrammatic group theory in quark models,''
%{\em Phys.~Rev.  D \bf 18}, 395 (1978).
%%%CITATION = PHRVA,D18,395;%%
%%\href{http://www.slac.stanford.edu/spires/find/hep/www?j=PHRVA%2cD18%2c395}{SPIRES}

\refref{Carles1973}
R.~Carles,
{\em Acad.~Sci. Paris  A \bf 276}, 451 (1973).

%\refref{Cartan1894} E.~Cartan (1894),
%{\em Oeuvres Compl\`etes} (Gauthier-Villars, Paris 1952).
%
%\refref{Cartan1914} E.~Cartan,
%{\em Ann.  Sci.  Ecole\ Norm.  Sup.  Paris  \bf 31}, 263 (1914).
%
\refref%[Car93]
        {MR94k:20020} R.~W. Carter,
     ``Conjugacy classes and complex characters,''
     {\em Finite groups of {L}ie type}
     % Wiley Classics Library,
     (John Wiley \& Sons Ltd., Chichester 1985, reprint 1993).

\refref{Cartier00} P. Cartier,
    ``Mathemagics (a tribute to L. Euler and R. Feynman),''
    {\em S\'em. Lothar. Combin. \bf 44}, article B44d (2000);
    \weblink{www.mat.univie.ac.at/$\sim$slc} .
    %, 71 pp.
    %/wpapers/s44cartier1.html}

\refref{Casalbuoni1976}
R.~Casalbuoni, G.~Domokos, and S.~K{\"o}vesi-Domokos,
{\em Nuovo Cimento \bf 31A}, 423 (1976).

\refref{Case1955} K.~M.~Case,
``Biquadratic spinor identities,''
{\em Phys.  Rev.   \bf 97}, 810 (1955).
%% Phys.Rev. 97, 810

\refref{Casimir31} H.~Casimir,
{\em Proc.  Roy.  Acad. Amstd.  \bf 34}, 844 (1931).

\refref{Cayley57} A.~Cayley,
``On the theory of the analytical forms called trees,''
{\em Philos. Mag. \bf 13}, 19 (1857). %pp.~19-30,

\refref{Chadha1981} S.~Chadha and M.~E.~Peskin,
``Implications of chiral dynamics in theories of technicolor. 2. The mass of
the $P^+$,''
{\em Nucl.~Phys.  B \bf 187}, 541 (1981).

\refref%[CP91]
        {MR92h:17010} V.~Chari and A.~Pressley,
``Fundamental representations of {Y}angians
  and singularities of {$R$}-matrices,''
{\em J.~Reine~Angew.~Math. \bf 417}, 87 (1991). % -128
\refref{Cheng:1974nv}
T.~P.~Cheng, E.~Eichten, and L.~Li,
``Higgs phenomena in asymptotically free gauge theories,''
{\em Phys.  Rev.    D \bf 9}, 2259 (1974).
%%CITATION = PHRVA,D9,2259;%%
%\href{http://www.slac.stanford.edu/spires/find/hep/www?j=PHRVA%2cD9%2c2259}{SPIRE$

\refref{Chev53} C.~Chevalley,
{\em Math.  Reviews  \bf 14}, 948 (1953).
%\PC{find}

\refref{Chisholm1963} J.~S.~R.~Chisholm,
{\em Nuovo Cimento  \bf 30}, 426 (1963).
%% Nuovo Cim. 30, 426

\refref%[CK91]
        {MR93f:82019}
H.~J.~Chung and I.~G.~Koh,
``Solutions to the quantum Yang-Baxter
  equation for the exceptional Lie algebras with a spectral parameter,''
{\em J.~Math.~Phys. \bf 32}, 2406 (1991). %2406-2408

\refref{Ciapessoni:1983} E. Ciapessoni and G.~M.~Cicuta,
     ``Gauge sets and $1/N$ expansion,"
    {\em Nucl. Phys. B \bf 219},  513 (1983). % 513-523

\refref{Cicuta:1982} G.~M.~Cicuta, %(Milan U. & INFN, Milan).
        ``Topological expansion for $SO(n)$ and $Sp(2n)$ gauge theories,''
        {\em Lett. Nuovo Cim. \bf 35}, 87 (1982).
% Cicuta: Here I describe the large-n expansion for gauge theories
% with SO(n) or Sp(n) groups. The problem had previously been discussed in a
% more limited way by Canning, Phys.Rev.D12 (1975), 2506. Your diagrammatic
% techniques are used.  The replacement n to -n for gauge invariant
% quantities of the SO(n) and SP(n) models is noted, as well as your
% previous work with Kennedy.

\refref{Cicuta:1984} G.~M.~Cicuta and D.~Gerundino, %(Milan U. & INFN, Milan).
    ``High-energy limit and internal symmetries,"
    {\em Phys. Rev. D \bf 29}, 1258 (1984). % 1258-1266
% Most of the new work in this paper is the evaluation of group factors in
% SU(N) for classes of relevant Feynman graphs. One can perform series
% summations after decomposing contributions with different quantum numbers
% (actually contributions belonging to different representations of the
% group). This is done using projectors. We interacted during the work and
% you made contributions. In the end, instead of an acknowledgment you
% preferred a sentence, which appear as reference n.11.

\refref{Cicuta:1989} G.~M.~Cicuta and A.~Pavone,
    ``Diagonalization of a coloring problem (on a strip),''
    {\em J. Phys. A \bf 22}, 4921 (1989).
% Here I considered a problem exactly solved by Baxter : evaluating the
% number of ways of proper coloring (with 3 colors) the bonds of a lattice
% made by regular exagons. You may view this lattice as made by rectangular
% bricks, the way brick walls are made. And consider a strip of the brick
% wall, the projectors of SU(2), here I am using Penrose and your results.
% The transfer matrix corresponding to a strip of finite width provides an
% approximation to the exact Baxter result, obtained by a completely
% different approach.

\refref{Clifford78} W.~Clifford,
    a letter to Mr.~Sylvester,
    {\em Amer. J. Math. \bf 1}, 126 (1878). % pp.~126-128

%% amc@win.tue.nl
%\refref{CM96} A.~M.~Cohen and R.~de Man,
%    ``Computational evidence for Deligne's conjecture
%      regarding exceptional Lie groups,''
%    {\em C.~R. Acad. Sci. Paris, S\'er. I, \bf 322}, 427 (1996).
%
%% amc@win.tue.nl
%\refref{CM99} A.~M.~Cohen and R.~de Man,
%``On a tensor category for the exceptional groups,''
%in P.~Drexler, G.~O.~Michler, and C.~M.~Ringel, eds.,
%{\em Computational Methods for Representations of Groups and Algebras},
%% pp. 121-138,
%{\em Progress in Math. \bf 173}, chapter 6, 121
%% (Euroconf.\ Proceedings),
%(Birkh\"auser, Basel 1999)

%\refref{Coleman68} A.~J.~Coleman,
%    {\em  Advances in Quantum Chemistry \bf 4}, 83 (1968).

\refref{CorriganFairlie77}
E.~Corrigan and D.~B.~Fairlie,
{\em Phys. Lett.  \bf 67B}, 69 (1977).

\refref{Corrigan77} E.~Corrigan, D.~B.~Fairlie, and R.~G.~Yates,
``SU(2) gauge potentials and their strength,''
Print-77-0375 (U. of Durham preprint, April 1977);
unpublished.
%\refref{Corrigan:1977dr}
%\href{http://www.slac.stanford.edu/spires/find/hep/www?r=print-77-0375\%2F(durham)}{SPIRES entry}

was Coxeter :
\refref{Coxeter12} H.~S.~M.~Coxeter,
    {\em Regular Polytopes}
    (Dover, New York 1973); see p. 92.
%(ISBN 0-486-61480-8)
% AD Kennedy: (to paraphrase: "why are they called
% Dynkin diagrams?"

%\refref{Cremmer:1979up} E.~Cremmer and B.~Julia,
%    ``The $SO(8)$ supergravity,''
%    {\em Nucl.  Phys.  B  \bf 159}, 141 (1979).
%%%CITATION = NUPHA,B159,141;%%
%%\href{http://www.slac.stanford.edu/spires/find/hep/www?j=NUPHA%2cB159%2c141}{SPIR$
%%(was cited by
%%    Renata Kallosh, Barak Kol
%%       E(7) SYMMETRIC AREA OF THE BLACK HOLE HORIZON
%%       Phys.Rev.D53:5344-5348,1996  hep-th/9602014
%%)

%Hendryk Pfeiffer 2 Dec 2002
% I think the first paper on the use of diagrams for SU(N) irreps was in
\refref{Creutz78} M.~Creutz,
    ``On invariant integration over $SU(N)$,''
    {\em J. Math. Phys. \bf 19}, 2043 (1978). % 2043-2046
% We give a graphical algorithm for evaluation of invariant integrals
% of polynomials in SU(N) group elements. Such integrals occur in
% strongly coupled lattice gauge theory. The results are expressed in
% terms of totally antisymmetric tensors and Kronecker delta symbols.

%\refref{Curtis1963} C.~W.~Curtis,
%{\em Studies in modern algebra  \bf 2},
%in A.~A.~Albert, ed.
%(Prentice Hall, New York 1963).

\refref{Cutkosky1963}
R.~E.~Cutkosky,
{\em Ann.~Phys. (N.Y.)  \bf 23}, 415 (1963).

was Cvit74 :
\refref{CviKin74c} P.~Cvitanovi\'c and T.~Kinoshita,
    ``Sixth order magnetic moment of the electron,''
    {\em Phys. Rev.~D \bf 10}, 4007 (1974).

was C76 :
\refref{ PCar} P.~Cvitanovi\'c,
``Group theory for Feynman diagrams in non-Abelian gauge theories,''
    {\em Phys. Rev.  D \bf 14}, 1536 (1976).
%BIRDIE label: \refref{PCar}
%%CITATION = PHRVA,D14,1536;%%
%\href{http://www.slac.stanford.edu/spires/find/hep/www?j=PHRVA%2cD14%2c1536}{SPIRES}

%\refref{C77} P.~Cvitanovi\'c,
%        ``Classical and exceptional Lie algebras as invariance algebras,''
%    Oxford preprint 40/77 (June 1977, unpublished);
%    available on \wwwcb{/refs}.
%
%\refref{Cvit77b} P.~Cvitanovi\'c,
%    ``Asymptotic estimates and gauge invariance,''
%    {\em Nucl. Phys.~B \bf 127}, 176 (1977).

was Cvit78 :
\refref{CvLaPe78} P.~Cvitanovi\'c, B.~Lautrup, and R.~B.~Pearson,
    ``The number and weights of Feynman diagrams,''
    {\em Phys. Rev. D \bf 18}, 1939 (1978).

was Cvit81 :
\refref{Cvit81planar} P.~Cvitanovi\'c,
    ``Planar perturbation expansion,''
    {\em Phys. Lett. \bf 99B}, 49 (1981).

was C81 :
\refref{NegDimE7}  P.~Cvitanovi\'c,
    ``Negative dimensions and $E_7$ symmetry,''
    {\em Nucl. Phys. B \bf 188}, 373 (1981).
% \refref%[Cvi81] {MR82i:81053}

was Cvitanovic:1981bu :
\refref{NPB81} P.~Cvitanovi\'c, P.~G.~Lauwers, and P.~N.~Scharbach,
    ``Gauge invariance structure of Quantum Chromodynamics,''
    {\em Nucl.  Phys.  B \bf 186}, 165 (1981).
%%CITATION = NUPHA,B186,165;%%
%\href{http://www.slac.stanford.edu/spires/find/hep/www?j=NUPHA%2cB186%2c165}{SPIRES}

was Cvitanovic:1981cn :
\refref{CviGreLau81} P.~Cvitanovi\'c, J.~Greensite, and B.~Lautrup,
    ``The crossover points in lattice gauge theories with
    continuous gauge groups,''
    {\em Phys.  Lett.  B \bf 105}, 197 (1981).
%%CITATION = PHLTA,B105,197;%%
%\href{http://www.slac.stanford.edu/spires/find/hep/www?j=PHLTA%2cB105%2c197}{SPIRES}

%\refref{CK82}  P. Cvitanovi\'c and A.~D. Kennedy,
%    ``Spinors in negative dimensions,''
%    {\em Phys. Scripta  \bf 26}, 5 (1982).
%%/spires label: \refref{Cvitanovic:1982bq}
%%%CITATION = PHSTB,26,5;%%
%%\href{http://www.slac.stanford.edu/spires/find/hep/www?j=PHSTB%2c26%2c5}{SPIRES}

\refref{C84}  P. Cvitanovi\'c,
    ``Group theory, part I''
    (Nordita, Copenhagen 1984); incorportated into this monograph.
%{\em Print-84-0261 (NORDITA)}.
%\refref{C86}  P. Cvitanovi\'c,
%BIRDIE label: \refref{PCgr}
%``Group theory, part II,''\\
%{\tt www.nbi.dk/GroupTheory/},
%   (1984-1996); Webbook in preparation.
%\refref{PCgr} P. Cvitanovi\'c, \emph{Group Theory}
%(Nordita, Copen\-hagen, 1984); www.nbi.dk/GroupTheory/.

\refref{CwwwGT} P. Cvitanovi\'c,
    ``Group theory''
    (Nordita and Niels Bohr Institute,
    Copenhagen 1984-2007). Web book version of this monograph;
    {\wwwcb{}}.
%BIRDIE label: \refref{PCgr}
%   (1984-1996); Webbook in preparation.

%\refref{conjug_Fred} P. Cvitanovi\'c, C.~P.~Dettmann,
%    R.~Mainieri, and G.~Vattay,
%        ``Trace formulas for stochastic evolution operators:
%         Smooth conjugation method,''
%        {\em Nonlinearity \bf 12}, 939 (1999);  % 939-953.
%        \arXiv{chao-dyn/9811003}.

%\refref{CFTsketch} P. Cvitanovi\'c,
%        {``Chaotic field theory: a sketch,''}
%        {\em Physica A \bf 288}, 61 (2000);\\ %61-80
%        % (5 Nov 1999; - invited talk, Dynamics Days Asia-Pacific,
%        % 13 - 16 July, 1999)
%        \arXiv{nlin.CD/0001034}.


%%DDDDDDDDDDDDDDDDDDDDDDDDDDDDDDDD


\refref{deAzcarraga:1998ya}
J.~A.~de Azc{\'a}rraga, A.~J.~Macfarlane, A.~J.~Mountain, and J.~C.~P\'erez Bueno,
    ``Invariant tensors for simple groups,''
    {\em Nucl.  Phys.  B \bf 510}, 657 (1998);
    \arXiv{physics/9706006}.
%%CITATION = PHYSICS 9706006;%%
%\href{http://www.slac.stanford.edu/spires/find/hep/www?eprint=PHYSICS/9706006}{SPIRES}

\refref{deAzcarraga:2000} J.~A.~de Azc{\'a}rraga and A.~J.~Macfarlane,
    ``Optimally defined Racah-Casimir operators for $su(n)$ and their
      eigenvalues for various classes of representations,''
    {\em Nucl.  Phys.  B \bf 510}, 657 (2000);
    \arXiv{math-ph/0006013}.

\refref{deAzcarraga:2001} J.~A.~de Azc{\'a}rraga and A.~J.~Macfarlane,
    ``Compilation of identities for the antisymmetric tensors of
      the higher cocyles of $su(n)$,"
    {\em Internat. J. Mod. Phys. \bf A16} 1377 (2001); %1377-1409
    \arXiv{math-ph/0006026}.

\PC{cite \refref{deAzcarraga:2001}}



%Pierre  deligne@math.ias.edu
%\refref{PD96} P.~Deligne,
%    ``La s\'erie exceptionelle de groupes de Lie,''
%    {\em C.~R. Acad. Sci. Paris, S\'er. I, \bf 322}, 321 (1996).

%\refref%[DdM96]
%        {MR97k:22008} P.~Deligne and R.~de~Man,
%    ``La s\'erie exceptionnelle de groupes de {L}ie. {II},''
%    {\em C.~R.~Acad.~Sci. Paris S\'er. I \bf 323}, 577 (1996). % 577-582

\refref{PD96a} P.~Deligne,
    letter to P. Cvitanovi\'c (December 1996);
    \\
    \wwwcb{/extras/reviews.html}.

%\refref{PDBHG2002} P.~Deligne and B.~H.~Gross,
%    ``On the exceptional series, and its descendents,''
%    {\em C.~R. Acad. Sci. Paris, S\'er. I, \bf 335}, 877 (2002).

\refref%[DGZ94]
        {MR95k:81058} G.~W.~Delius, M.~D.~Gould, and Y.~Z.~Zhang,
    ``On the construction of
      trigonometric solutions of the Yang-Baxter equation,''
    {\em Nucl. Phys. B  \bf 432}, 377 (1994). % 377-403

\refref{deWit:1977hx} B.~de~Wit and G.~'t~Hooft,
``Nonconvergence of the $1/N$ expansion for
  $SU(N)$ gauge fields on a lattice,''
    {\em Phys.  Lett.  B  \bf 69}, 61 (1977).
%%CITATION = PHLTA,B69,61;%%
%\href{http://www.slac.stanford.edu/spires/find/hep/www?j=PHLTA%2cB69%2c61}{SPIRES}

\refref{Dickson1907} L.~E.~Dickson (1907).
I am indebted to G.  Seligman for this reference.
% http://www-gap.dcs.st-and.ac.uk/~history/Mathematicians/Dickson.html

\refref{Dirac:1928hu}
P.~A.~Dirac,
``The quantum theory of electron,''
{\em Proc.  Roy.  Soc.  Lond.    A \bf 117}, 610 (1928).
%%CITATION = PRSLA,A117,610;%%
%\href{http://www.slac.stanford.edu/spires/find/hep/www?j=PRSLA%2cA117%2c610}{SPIR$
%% Proc.Roy.Soc.Lond.Sect.A A117, 610

% scaling Potts models provide a structure which unifies the E_8 family
\PC{recheck their numerics}
\refref{Dorey2002a} P. Dorey, A.~Pocklington, and  R.~Tateo,
    ``Integrable aspects of the scaling q-state Potts models I:
    Bound states and bootstrap closure,''
    {\em Nucl. Phys. B \bf 661},  425 (2003); %  425-463
    \arXiv{hep-th/0208111}.

\refref{Dittner1971}
P.~Dittner,
{\em Commun.~Math.~Phys.  \bf 22}, 238 (1971); {\bf 27}, 44 (1972).

\refref{Drouffe:1978py}
J.~M.~Drouffe,
``Transitions and duality in gauge lattice systems,''
{\em Phys.  Rev.    D \bf  18}, 1174 (1978).
%%CITATION = PHRVA,D18,1174;%%
%\href{http://www.slac.stanford.edu/spires/find/hep/www?j=PHRVA%2cD18%2c1174}{SPIR$

\refref{Duca:2000} V.~Del~Duca, L.~J. Dixon, and F.~Maltoni,
        % Vittorio Del Duca, Lance J. Dixon, Fabio Maltoni .
        ``New color decompositions for gauge amplitudes
        at tree and loop level,''
        % SLAC-PUB-8294, DFTT-53-99, Oct 1999. 17pp.
        {\em Nucl. Phys. B \bf 571}, 51 (2000); %51-70
        \arXiv{hep-ph/9910563}.

\refref{Dundarer:1986jt}
R.~Dundarer, F.~Gursey, and C.~Tze,
``Selfduality and octonionic analyticity of $S(7)$ valued antisymmetric fields
in eight-dimensions,''
{\em Nucl.  Phys.   B \bf 266}, 440 (1986).
%%CITATION = NUPHA,B266,440;%%
%\href{http://www.slac.stanford.edu/spires/find/hep/www?j=NUPHA%2cB266%2c440}{SPIRES}

\refref{Dundarer:1984fe}
R.~Dundarer, F.~Gursey, and C.~Tze,
``Generalized vector products, duality and octonionic
identities in $D = 8$ geometry,''
{\em J.  Math.  Phys.    \bf 25}, 1496 (1984).
%%CITATION = JMAPA,25,1496;%%
%\href{http://www.slac.stanford.edu/spires/find/hep/www?j=JMAPA%2c25%2c1496}{SPIRES}

\refref{Dunne89} G.~V.~Dunne,
    ``Negative-dimensional groups in quantum physics,''
    {\em J. Phys. A \bf  22}, 1719 (1989). % 1719-1736

\refref{Durr1968}
H.~P.~D{\"u}rr and F.~Wagner,
{\em Nuovo Cimento  \bf 53A}, 255 (1968).

\refref{Dynkin1957} E.~B.~Dynkin,
{\em Transl. Amer. Math. Soc.~(2)  \bf 6}, 111 (1957).

\refref{Dynkin1962} E.~B.~Dynkin,
{\em Transl. Amer. Math. Soc.~(1)  \bf 9}, 328 (1962).

\refref{Dyson49} F.~J.~Dyson,
``The radiation theories of Tomonaga, Schwinger, and Feynman,''
{\em Phys. Rev. \bf 75}, 486 (1949). % pp.~486-502,

%%EEEEEEEEEEEEEEEEEEEEEEEEEEEEEEEEEEEEE

        % Wigner - Eckart theorem
\refref{Eckart30} C.~Eckart,
``The application of group theory to the quantum dynamics of monatomic
systems,''
{\em Rev. Mod. Phys.  \bf 2}, 305 (1930).

\refref{Ekins1975}
J.~M.~Ekins and J.~F.~Cornwell,
{\em Rep.~Math.~Phys.  \bf 7}, 167 (1975).

\refref{ElBazCastel} E. El Baz and B. Castel,
{\em Graphical Methods of Spin Algebras in Atomic,
Nuclear and Particle Physics}
(Dekker, New York 1972).

\refref{Elduque05}  A. Elduque,
    ``The Magic Square and symmetric compositions II,''
    {\em Rev. Mat. Iberoamericana \bf 23}, 57 (2007); % , 57-84
    \arXiv{math.RT/0507282}.

\refref{Houari97} M. El Houari,
    ``Tensor invariants associated with classical
      Lie algebras: a new classification of simple Lie algebras,''
    {\em Algebras, Groups, and Geometries \bf 17}, 423 (1997).

\refref{Elvang99} H.~Elvang,
    ``Birdtracks, Young projections, colours,''
     MPhys project in Mathematical Physics (1999);
  \HREF{http://www.nbi.dk/$\sim$elvang/rerep.ps}{www.\penalty -10\null nbi.\penalty
  -10\null dk/\penalty -20\null \char `\~elvang/\penalty -20\null
  rerep.\penalty -10\null ps}.

\refref{YoungUn} H.~Elvang, P.~Cvitanovi\'c, and A.~D.~Kennedy,
    ``Diagrammatic Young projection operators for $U(n)$,''
        {\em J. Math. Phys. \bf 46}, 043501 (2005);
    \arXiv{hep-th/0307186}.

\refref{Evans:1993az}
M.~Evans, F.~Gursey, and V.~Ogievetsky,
``From 2-D conformal to 4-D selfdual theories: Quaternionic analyticity,''
{\em Phys.  Rev.    D \bf 47}, 3496 (1993);
\arXiv{hep-th/9207089}.
%%CITATION = HEP-TH 9207089;%%
%\href{http://www.slac.stanford.edu/spires/find/hep/www?eprint=HEP-TH/9207089}{SPIRES}


%%FFFFFFFFFFFFFFFFFFFFFFFFFFFFFFFFFFFFFF

\refref{Farrar:1983wk}
G.~R.~Farrar and F.~Neri,
``How to calculate 35640 $O(\alpha^5)$ Feynman diagrams in less than an
hour,''
{\em Phys.  Lett.    \bf 130B}, 109 (1983).
%%CITATION = PHLTA,130B,109;%%
%\href{http://www.slac.stanford.edu/spires/find/hep/www?j=PHLTA%2c130B%2c109}{SPIRES}

%\refref{Faulkner1971}
%J.~R.~Faulkner,
%{\em Trans.~Amer.~Math.~Soc.  \bf 155}, 397 (1971).

\refref{Faulkner1977} J.~R.~Faulkner and J.~C.~Ferrar,
    ``Exceptional Lie algebras
    and related algebraic and geometric structures,"
    {\em Bull. London Math. Soc.  \bf 9}, 1 (1977).

\refref{Fermi:1934hr} E.~Fermi,
``An attempt of a theory of beta radiation. 1,''
{\em Z.  Phys.    \bf 88}, 161 (1934).
%%CITATION = ZEPYA,88,161;%%
%\href{http://www.slac.stanford.edu/spires/find/hep/www?j=ZEPYA%2c88%2c161}{SPIRES}

% \refref{Feynman49}  R.~P.~Feynman, ``Theory of Positrons,''
%   {\em Phys. Rev. \bf 76}, 749 (1949).

%\refref{FeynmanUr} R.~P.~Feynman,
%``Space-time approach to nonrelativistic quantum mechanics,''
%{\em Rev. Mod. Phys.  \bf 20}, 367 (1948).

\refref{Feynman49a} R.~P.~Feynman,
``Space-time approach to {Quantum Electrodynamics},''
{\em Phys. Rev. \bf 76}, 769 (1949). %769--789

\refref{Fierz1934} M.~Fierz,
{\em Z. Physik   \bf 88}, 161 (1934).

\refref{Finkelstein1963}
D.~Finkelstein, J.~M.~Jauch, and D.~Speiser,
{\em J.~Math.~Phys.  \bf 4}, 136 (1963).

\refref{Fischler80} M.~Fischler,
    % HE: M(ark) Fischler (ikke W(illy) Fischler)
``Young-tableau methods for Kronecker products of representations of the
classical groups,''
{\em J. Math. Phys.   \bf 22}, 637 (1981).
% {\em Fermilab-Pub-80/49-THY} (1980).
% fnalpubs.fnal.gov/archive/1980/pub/Pub-80-049-T.pdf
% www.slac.stanford.edu/spires/find/hep/www?indexer=1&rawcmd=find+r+FERMILAB-PUB-80-049-THY
% @Article{Fischler:1981:YTM,
%   author =       "Mark Fischler",
%   MRclass =      "22C05 (20C15 20C35 81C40)",
%   MRnumber =     "82j:22003",



%\refref{Frame54}
%J.~S.~Frame, D.~de~B.~Robinson, and R.~M. Thrall,
%    {\em Canad. J. Math. \bf 6}, 316 (1954).

\refref{Frampton:1982ik}
P.~H.~Frampton and T.~W.~Kephart,
``Exceptionally simple $E(6)$ theory,''
{\em Phys.  Rev.    D \bf 25}, 1459 (1982).
%%CITATION = PHRVA,D25,1459;%%
%\href{http://www.slac.stanford.edu/spires/find/hep/www?j=PHRVA%2cD25%2c1459}{SPIRES}
%(was cited by
%    J. L. Hewett and T. G. Rizzo
%   LOW-ENERGY PHENOMENOLOGY OF SUPERSTRING INSPIRED E(6) MODELS
%   Phys.Rept.183:193,1989
%)

\refref{Frampton:1982pf}
P.~H.~Frampton and T.~W.~Kephart,
``Dynkin weights and global supersymmetry in grand unification,''
{\em Phys.  Rev.  Lett.    \bf 48}, 1237 (1982).
%%CITATION = PRLTA,48,1237;%%
%\href{http://www.slac.stanford.edu/spires/find/hep/www?j=PRLTA%2c48%2c1237}{SPIRES}

\refref{FraSciSor01} L.~Frappat, A.~Sciarrino, and P.~Sorba,
    {\em A Dictionary of Lie Algebras and Superalgebras}
    (Academic Press, New York 2001).

\refref{Frege1879} G.~Frege,
    {\em Begriffsschrift, eine der arithmetischen
    nachgebildete Formelsprache des reinen Denkens} (Halle 1879),
    translated as {\em  Begriffsschrift, formula language, modeled upon that
    of arithmetic, for pure thought},
    \refref{VanHeijenoort}. I am grateful to P.~Cartier for bringing
    Frege's work to my attention.

\refref{Frege1897} G.~Frege,
    ``On Mr. Peano's conceptual notation and my own,''
    {\em Berichte \"uber die Verhandlungen der
     K\"oniglich-S\"achsischen Gesellschaft
    der Wissenschaften zu Leipzig. Mathematisch-Physische Klasse,
    \bf XLVIII}, 361 (1897). % 361-378.

%\refref{Freudenthal1954} H.~Freudenthal, %Hans Freudenthal,
%    ``Beziehungen der $E_7$ und $E_8$ zur Oktavenebene, I, II,''
%    {\em Indag. Math. \bf 16}, 218, 363 (1954); %218-230, 363-368;
%    % {\em Ned. Akad. Weternsch. Proc.   A \bf 57}, 218 (1954).
%    {\em ``III, IV,''}
%    {\em  Indag. Math. \bf 17}, 151, 277 (1955); %151-157, 277-285;
%    {\em ``V - IX,''}
%    {\em  Indag. Math. \bf 21}, 165 (1959); %165-201, 447-474;
%    {\em `` X, XI,''}
%        {\em Indag. Math. \bf 25}, 457 (1963). % 457-487.

%\refref{Freudenthal1964} H.~Freudenthal,
%``Lie groups in the foundations of geometry,''
%{\em Adv. Math.   \bf 1}, 145 (1964).

\refref{Fritzsch1973}
H.~Fritzsch, M.~Gell-Mann, and H.~Leutwyler,
{\em Phys.~Lett.  \bf 47B}, 365 (1973).


%\refref{Fulton91} W.~Fulton and J.~Harris,
%    {\em Representation Theory}
%    % , Graduate Texts in Mathematics
%    (Springer, Berlin 1991).

was Fulton99 :
\refref{Fulton97} W.~Fulton,
     {\em Young Tableaux, with Applications to
          Representation Theory and Geometry}
        (Cambridge Univ. Press, Cambridge  1999).

%%GGGGGGGGGGGGGGGGGGGGGGGGGGGGGGGGGGGG

\refref{Gelfand50} I. M. Gel'fand,
    {\em Math. Sbornik \bf 26}, 103 (1950).

\refref{Gelfand61} I. M. Gel'fand,
        {\em Lectures on Linear Algebra}
        (Dover, New York 1961).
        % ISBN:         0486660826

%\refref%[GZ84]
%        {MR86i:22024}
%I.~M.~Gel'fand and A.~V.~Zelevinski\u{\i},
%``Models of
%  representations of classical groups and their hidden symmetries,''
%{\em Funktsional. Anal. i Prilozhen. \bf 18}, 14 (1984). % 14-31

\refref{Gell-Mann61} M.~Gell-Mann,
    ``The eightfold way: Theory of strong interaction symmetry,''
    {\em Caltech Report No. CTSL-20} (1961), unpublished.
    Reprinted in M. Gell-Mann and Y. Ne'eman,
    {\em The Eightfold Way: A Review - with Collection of Reprints}
    % Frontiers in Physics, ed. D. Pines,
    (Benjamin, New York 1964).
%
%\PC{this is not the  Gell-Mann matrices ref}
%see: http://www.santafe.edu/sfi/People/mgm/mgmpubs.html - if not
% \refref{Gell-Mann64} M.~Gell-Mann,
% ``Nonleptonic weak decays and the Eightfold Way,''
% {\em Phys. Rev. Lett. \bf 12}, 155 (1964).

%\refref{Georgi99} H.~Georgi,
%     {\em Lie Algebras in Particle Physics}
%    (Perseus Books, Reading, MA 1999).
%

%MUST STUDY THIS REFERENCE:
% giddings@denali.physics.ucsb.edu
\refref{GP95} B.~Giddings and J.~M.~Pierre,
    ``Some exact results in supersymmetric theories
      based on exceptional groups,''
{\em Phys.  Rev.   D \bf 52}, 6065 (1995);
\arXiv{hep-th/9506196}.

\refref{Gilmore1974} R.~Gilmore,
    {\em Lie Groups, Lie Algebras and Some of Their Applications}
    (Wiley, New York 1974).

\refref{Gleick92} J.~Gleick,
    {\em Genius: The Life and Science of Richard Feynman}
    (Pantheon, New York 1992).

%\refref{GoodWallach} R.~Goodman and N.~R. Wallach,
%     {\em Representations and Invariants of the Classical Groups}
%        (Cambridge Univ. Press, Cambridge  1998).

\refref{Gorn:1981sr} M.~Gorn,
``Problems in comparing diagrams with group theory in nonleptonic decays,''
{\em Nucl.  Phys.  B  \bf 191}, 269 (1981).
%%CITATION = NUPHA,B191,269;%%
%\href{http://www.slac.stanford.edu/spires/find/hep/www?j=NUPHA%2cB191%2c269}{SPIRES}

\refref{Gourdin67} M.~Gourdin,
{\em Unitary Symmetries}
(North-Holland, Amsterdam 1967).

\refref{GSWstrings}
M.~Green, J.~Schwarz, and E.~Witten,
    ``String theory.''

\refref{Gross:1980he} D.~J.~Gross and E.~Witten,
    ``Possible third order phase transition in
      the large $N$ lattice gauge theory,''
    {\em Phys.  Rev.    D \bf 21}, 446 (1980).
%%CITATION = PHRVA,D21,446;%%
%\href{http://www.slac.stanford.edu/spires/find/hep/www?j=PHRVA%2cD21%2c446}{SPIRES}

\refref{Gunaydin1975} M.~G{\"u}naydin,
{\em Nuovo Cimento  \bf 29A}, 467 (1975).

\refref{Gunaydin1973}
M.~G{\"u}naydin and F.~G{\"u}rsey,
{\em J.~Math.~Phys.  \bf 14}, 1651 (1973).

\refref{Gursey1975}
F.~G{\"u}rsey,
in {\em Proceedings of the Kyoto Conference on Mathematical Problems in Theoretical
Physics} (Kyoto 1975), unpublished.

%\refref{Gursey:1976dn}
%F.~G{\"u}rsey and P.~Sikivie,
%``$E_7$ as a universal gauge group,''
%{\em Phys.  Rev.  Lett.    \bf 36}, 775 (1976).
%%%CITATION = PRLTA,36,775;%%
%%\href{http://www.slac.stanford.edu/spires/find/hep/www?j=PRLTA%2c36%2c775}{SPIRES}
%%(was cited by
%%     S.M. Bar
%%   E-8 FAMILY UNIFICATION, MIRROR FERMIONS,
%%   AND NEW LOW-ENERGY PHYSICS
%%   Phys.Rev.D37:204,1988
%%)

\refref{Gursey1976}
F.~G{\"u}rsey, P.~Ramond, and P.~Sikivie,
{\em Phys.~Lett.  \bf 60B}, 177 (1976).

\refref{Gursey:1981kf}
F.~G{\"u}rsey and M.~Serdaroglu,
``$E_6$ gauge field theory model revisited,''
{\em Nuovo Cim.    \bf 65A}, 337 (1981).
%%CITATION = NUCIA,65A,337;%%
%\href{http://www.slac.stanford.edu/spires/find/hep/www?j=NUCIA%2c65A%2c337}{SPIRES}

\refref{Gursey1996} F.~G{\"u}rsey and C.-H.~Tze,
    {\em Division, Jordan and Related Algebras in Theoretical Physics}
    (World Sci., Singapore, 1996).


%%HHHHHHHHHHHHHHHHHHHHHHHHHHHHHHHHHHHH

\refref{HAMERMESH} M.~Hamermesh,
{\em Group Theory and Its Application to Physical Problems}
(Dover, New York 1962).

\refref{Habegger01} N.~Habegger,
    % and  S. Gervais,
    ``The topological IHX relation,''
    {\em J. Knot Theory and Its Ramifications  \bf 10}, 309 (2001). %309-329


%\refref{Harter69} W.~G.~Harter,
%    {\em J. Math. Phys. \bf 10}, 4 (1969).
%
%\refref{Harter78} W.~G.~Harter and N.~Dos Santos,
%    ``Double-group theory on the half-shell and the two-level system.
%      I. Rotation and half-integral spin states,''
%    {\em Am. J. Phys. \bf  46}, 251 (1978). % pp.251-263
%\PC{In this pretty paper Harter
%explains how Hamilton (who would have guessed...) used quaternions to
%to extend the discrete Fourier transform I developed in the QFT  course
%for a 1-d circle to the 2-d sphere embedded in 3 dimensions.}
%
%\refref{Harter93} W.~G.~Harter,
%   {\em Principles of Symmetry, Dynamics, and Spectroscopy}
%   (Wiley, New York 1974).

\refref{Harvey1980} J.~A.~Harvey,
``Patterns of symmetry breaking in the  exceptional groups,''
{\em Nucl. Phys. B \bf 163}, 254 (1980).

\PC{update}

\refref{Hearn1970} A.~C.~Hearn,
``REDUCE User's Manual,''
{\em Stanford Artificial Intelligence Project, Memo AIM-133}, (1970).

\refref{HLJP02}
   P.~Henry-Labord\'ere, B.~L.~Julia, and L.~Paulot,
     ``Borcherds symmetries in M-theory,"
\arXiv{hep-th/0203070}.

%\refref{'tHooft:1972fi}
%G.~'t Hooft and M.~Veltman,
%``Regularization and renormalization of gauge fields,''
%{\em Nucl.  Phys. B \bf 44}, 189 (1972).
%%%CITATION = NUPHA,B44,189;%%
%%\href{http://www.slac.stanford.edu/spires/find/hep/www?j=NUPHA%2cB44%2c189}{SPIRES}
%
\refref{tHooftVelt1973} G.~'t Hooft and M.~Veltman,
``DIAGRAMMAR,''
{\em CERN report 73/9}, (1973).
% reprinted in "Particle
%interactions at very high energies". NATO Adv. Study Inst. Series, Sect. B,
%vol. 4B, 177.
% http://www.phys.uu.nl/~thooft/gthpub.html
% http://cdsweb.cern.ch/record/186259/files/CERN-73-09.pdf

%\refref{'tHooft:1974jz}
%G.~'t Hooft,
%``A planar diagram theory for strong interactions,''
%{\em Nucl.  Phys.  B  \bf 72}, 461 (1974).
%%%CITATION = NUPHA,B72,461;%%
%%\href{http://www.slac.stanford.edu/spires/find/hep/www?j=NUPHA%2cB72%2c461}{SPIRES}

%\refref{'tHooft:1976fv}
%G.~'t Hooft,
%``Computation of the quantum effects due to a four-dimensional  pseudoparticle,''
%{\em Phys. Rev. Lett.  \bf 37}, 8 (1976); {\em Phys.  Rev.    D \bf 14}, 3432 (1976).
%%%CITATION = PHRVA,D14,3432;%%
%%\href{http://www.slac.stanford.edu/spires/find/hep/www?j=PHRVA%2cD14%2c3432}{SPIRES}

\refref{hurwitz98} A. Hurwitz,
     ``\"Uber die Komposition der quadratischen Formen von
    beliebig vielen Variabeln,''
    {\em Nachr. Ges. Wiss. G\"ottingen}, 309 (1898). % 309-316.

\refref{hurwitz23} A. Hurwitz,
    ``\"Uber die Komposition der quadratischen Formen,''
    {\em Math. Ann. 88}, 1 (1923). % 1-25.

%%IIIIIIIIIIIIIIIIIIIIIIIIIIIIIIIIIIII

%%JJJJJJJJJJJJJJJJJJJJJJJJJJJJJJJJJJJJ

%\refref{Jacobi1841}
%C.~G.~J. Jacobi,
%      ``De functionibus alternantibus earumque divisione per productum e
%      differentiis elementorum conflatum,'' in {\em Collected Works},
%      Vol.~22, 439;
%    {\em J. Reine Angew.  Math. (Crelle)} (1841). %pp. 439-452


\refref{Jacobson1971} N.~Jacobson,
    {\em Exceptional Lie Algebras}
    (Dekker, New York 1971).

\refref{Jacobson1974} N.~Jacobson,
        {\em Basic Algebra I}
    (Freeman, San Francisco 1974).
    % MR 50:9457

%\refref{James81} G.~James and A.~Kerber,
%    {\em The Representation Theory of the Symmetric Group}
%    % , Encyclopedia of Mathematics and Its Applications
%    (Addison-Wesley, Reading, MA 1981).

\refref%[Jim86]
        {MR87h:58086}
M.~Jimbo,
``Quantum {$R$} matrix for the generalized Toda system,''
{\em Comm. Math. Phys.  \bf 102}, 537 (1986).  %537-547

\refref%[JM95]
        {MR98b:81098} B.-Q. Jin and Z.-Q. Ma,
    ``$R$ matrix for $U\sb qE\sb 7$,''
    {\em Comm.~Theoret.~Phys.  \bf 24}, 403 (1995). % 403-410

%\refref{Jordan:1934vh} P.~Jordan, J.~von Neumann, and E.~Wigner,
%``On an algebraic generalization of the quantum mechanical formalism,''
%    {\em Ann. Math.    \bf 35}, 29 (1934).
%%%CITATION = ANMAA,35,29;%%
%%\href{http://www.slac.stanford.edu/spires/find/hep/www?j=ANMAA%2c35%2c29}{SPIRES}
%
%\refref{Julia81} B.~L. Julia,
%    ``Group disintegrations,'' in
%        S.~Hawking and M.~Rocek, eds.,
%        {\em Superspace and Supergravity}
%        (Cambridge Univ. Press, Cambridge  1981).

\refref{Julia00} B.~L. Julia,
    ``Below and beyond U-duality,"
\arXiv{hep-th/0002035}.

\refref{Julia01} B.~L.~Julia,
    ``Magics of M-gravity,"
\arXiv{hep-th/0105031}.


%%KKKKKKKKKKKKKKKKKKKKKKKKKKKKKKKKKKKK

\refref{Kac:1977em}
V.~G.~Kac,
``Lie superalgebras,''
{\em Adv.  Math.    \bf 26}, 8 (1977).
%%CITATION = ADMTA,26,8;%%
%\href{http://www.slac.stanford.edu/spires/find/hep/www?j=ADMTA%2c26%2c8}{SPIRES}

\refref{Kahane1968} J.~Kahane,
{\em J. Math. Phys.   \bf 9}, 1732 (1968).

\refref{Kamiya89} N.~Kamiya,
``A structure theory of Freudenthal-Kantor triple systems III,''
{\em Mem. Fac. Sci. Shimane Univ. \bf 23}, 33 (1989).   %33-51

\refref{Kamiya91} N.~Kamiya,
``The construction of all simple Lie algebras
over $\complex$ from balanced Freudenthal-Kantor triple systems,''
{\em Contr. to General Algebra \bf 7}, 205 (1991).      %205-213
% (Verlag H\"older-Pichler-Tempsky, Wien 1991). %205-213

\refref{Kantor1973} I.~L.~Kantor,
{\em Soviet Math. Dokl.  \bf 14}, 254 (1973).

\refref{Kaplan1967}
L.~M.~Kaplan and M.~Resnikoff,
{\em J.~Math.~Phys.  \bf 8}, 2194 (1967).

\refref{Kempe85} A.~B.~Kempe,
``On the application of Clifford's graphs to ordinary
  binary quantics,''
{\em Proc.~London Math.~Soc. \bf 17}, 107 (1885). %pp.~107-121

\refref{Kennedy:1981kp}
A.~D.~Kennedy,
``Clifford algebras in two Omega dimensions,''
{\em J.  Math.  Phys.    \bf 22}, 1330 (1981).
%%CITATION = JMAPA,22,1330;%%
%\href{http://www.slac.stanford.edu/spires/find/hep/www?j=JMAPA%2c22%2c1330}{SPIRES}

\refref{Kennedy:1982ei}
%BIRDIE label: \refref{ADKar}
A.~D.~Kennedy,
``Spinography: Diagrammatic methods for spinors in Feynman diagrams,''
{\em Phys.  Rev.  D \bf 26}, 1936 (1982).
%%CITATION = PHRVA,D26,1936;%%
%\href{http://www.slac.stanford.edu/spires/find/hep/www?j=PHRVA%2cD26%2c1936}{SPIRES}
%(was cited by
%    H.W. Braden
%   A NEW EXPRESSION FOR THE D-DIMENSIONAL FIERZ COEFFICIENTS
%   DAMTP-84/3, (Received May 1984)
%     L.V. Avdeev
%   ON FIERZ IDENTITIES IN NONINTEGER DIMENSIONS
%   Theor.Math.Phys.58:203,1984
%)

%\refref{ADKsl}
%A.~D.~Kennedy,
%``Group algebras, Lie algebras, and Clifford algebras,''
% Colloquium, Moscow State University (1997);\\
%      \weblink{www.ph.ed.ac.uk/$\sim$adk/algebra-slides/all.html} .
%%\weblink{www.ph.ed.ac.uk/{$\sim$}adk/algebra-slides/all.html} .
%%\refref{Kephart81} Kephart (1981).
%%DG: is this \refref{Kephart:1981cf}?
%
\refref{Kephart:1981cf}
T.~W.~Kephart and M.~T.~Vaughn,
``Renormalization of scalar quartic and Yukawa couplings
in unified gauge theories,''
% title goes on?: based,''
{\em Z.  Phys. C   \bf 10}, 267 (1981).
%%CITATION = ZEPYA, C10,267;%%
%\href{http://www.slac.stanford.edu/spires/find/hep/www?j=ZEPYA%2cC10%2c267}{SPIRES}


\refref{Kephart:1983gf}
T.~W.~Kephart and M.~T.~Vaughn,
``Tensor methods for the exceptional group $E6$,''
{\em Annals Phys.    \bf 145}, 162 (1983).
%%CITATION = APNYA,145,162;%%
%\href{http://www.slac.stanford.edu/spires/find/hep/www?j=APNYA%2c145%2c162}{SPIRES}
% (was cited by
%     S.B.  Giddings and J.M.  Pierre
%       SOME EXACT RESULTS IN SUPERSYMMETRIC THEORIES BASED ON
%       EXCEPTIONAL GROUPS
%       Phys.Rev.D52:6065-6073,1995      % hep-th/9506196
% )


%\refref{Killing1888} W.~Killing,
%{\em Math. Ann.   \bf 31}, 252 (1888); {\bf 33}, 1 (1889); {\bf 34}, 57
%(1889); {\bf 36}, 161 (1890).

\refref%[KKM91]
        {MR92h:17015}
J.~D. Kim, I.~G.~Koh, and Z.~Q.~Ma,
``Quantum $\check R$ matrix for $E\sb 7$ and $F\sb 4$ groups,''
{\em J.~Math.~Phys.  \bf 32}, 845 (1991). %845-856


%\refref{King1972} R.~C.~King, {\em Canad. J. Math.  \bf 33}, 176 (1972).


\refref{King1985} R.~C.~King and B.~G.~Wybourne,
    ``Holomorphic discrete series and harmonic series unitary
      irreducible representations of non-compact Lie groups:
      $Sp(2n,R)$, $U(p,q)$ and $SO^*(2n)$,''
    {\em J. Math. Phys. \bf 18}, 3113 (1985). % 3113-3139

\refref{King:2000} R.~C.~King and B.~G.~Wybourne,
``Analogies between finite-dimensional irreducible representations of
  $SO(2n)$ and infinite-dimensional irreducible representations of $Sp(2n,R)$.
  I. Characters and products,''
  {\em J. Math. Phys. \bf 41}, 5002 (2000).
% a paper on spinsters, metaplectic reps of Sp(2n)

\refref{King:2002} R.~C.~King and B.~G.~Wybourne,
``Multiplicity-free tensor products of irreducible representations of the
exceptional Lie groups,''
    {\em J. Physics A  \bf 35}, 3489 (2002).
%
% For exceptional Lie groups, a complete determination is given
% of those pairs of finite-dimensional irreducible representations
% whose tensor products (or squares) may be resolved into
% irreducible representations that are multiplicity free, i.e.
% such that no irreducible representation occurs in the decomposition
% of the tensor product more than once. Explicit formulae are presented
% for the decomposition of all those tensor products that are multiplicity
% free, many of which exhibit a stability property.
%

\refref{Kinoshita1981} T.~Kinoshita  and W.~B.~Lindquist,
        ``Eighth-Order Anomalous Magnetic Moment of the Electron,"
        {\em Phys. Rev. Lett.  \bf 47}, 1573  (1981). %1573 - 1576

\refref{KleinNish1929} O.~Klein and Y.~Nishina,
        {\em Z. Physik  \bf 52}, 853 (1929).

\refref%[KM90]
        {MR91b:17017}   I.~G. Koh and Z.~Q.~Ma,
        ``Exceptional quantum groups,''
        {\em Phys.~Lett.~B  \bf 234}, 480 (1990). %480-486

% \refref{Konstein77} S.~E.~Konstein (1977).
% \PC{unpublished, could not find}

\refref{Kontsevich90} M.~Kontsevich,
``Formal (non)commutative symplectic geometry,''
{\em  The Gel`fand Mathematical Seminars 1990--1992}, 173
  (Birkh{\"a}user, Boston, MA 1993).
% (Reviewer: Alexander A. Voronov) 58H15 (17B65 58F05)
% pp~173-187
\refref{Kontsevich94} M.~Kontsevich,
``Feynman diagrams and low-dimensional topology,''
{\em  First European Congress of Mathematics, Vol. II (Paris 1992)},
 {\em Progr. Math. \bf 120}, 97
(Birkh{\"a}user, Basel 1994). % 97-121
% (Reviewer: Anatoly Libgober) 57R57 (14H15 32G15 57M25).
%
%  Kontsevich converted very successfully the
% diagramatics of symplectic invariant tensors to graphs, and coined the
% term: "graph complex" in the first of the above papers. This graph
% complex, and its homology, studied by Kontsevich have had the most
% influence on pertrurbative knot invariants, that I know of. The IHX
% relation is simply the relation that defines the 0-homology in that
% complex.


\refref{Kryukov:1988ta}
A.~P.~Kryukov and A.~Y.~Rodionov,
``Color: Program for calculation of group weights of Feynman
diagrams in nonabelian gauge theories,''
{\em Comput.  Phys.  Commun.    \bf 48}, 327 (1988).
%%CITATION = CPHCB,48,327;%%
%\href{http://www.slac.stanford.edu/spires/find/hep/www?j=CPHCB%2c48%2c327}{SPIRES}

\refref%[KRS81]
        {MR83g:81099}
P.~P. Kulish, N.~Y. Reshetikhin, and E.~K. Sklyanin, ``Yang-Baxter
  equations and representation theory. I,''
{\em Lett.~Math.~Phys. \bf 5}, 393 (1981). % 393-403

\refref%[Kun90]
        {MR91f:82026} A.~Kuniba,
``Quantum $R$-matrix for $G\sb 2$ and a solvable $175$-vertex model,''
{\em J.~Phys.~A  \bf 23}, 1349 (1990). %1349-1362

%%LLLLLLLLLLLLLLLLLLLLLLLLLLLLLLLLLLLL

%\refref%[LM01]
%        {lands01} J.~M. Landsberg and L.~Manivel,
%``Triality, exceptional Lie algebras and Deligne dimension formulas,''
%        {\em Adv.  Math. \bf 171}, 59 (2002); % 59-85.
%        \arXiv{math.AG/0107032}.

%\refref%[LM02]
%        {lands01a} J.~M. Landsberg and L.~Manivel,
%        ``On the projective geometry of Freudenthal's magic square,''
%        {\em J. of Algebra \bf 239}, 477 (2001);  % 477-512
%        \arXiv{math.AG/9908039}.
%
%\refref%[LM02]
%        {lands02} J.~M. Landsberg and L.~Manivel,
%        ``Series of Lie groups,''
%        {\em Michigan Math. J. \bf 52},   453 (2004); %  453 479
%        \arXiv{math.AG/0203241}.
%
%\refref{lands02a} J.~M. Landsberg and L.~Manivel,
%    ``Representation theory and projective geometry,''
%    in V. Popov, ed.,
%    {\em Algebraic transformation groups and algebraic varieties},
%    {\em Encyclopaedia of Mathematical Sciences},
%    {\em Invariant Theory and Algebraic Transformation Groups III}
%%    {\em Proceedings of the Vienna conference on
%%    varieties arising from interesting group actions}
%    (Springer-Verlag, New York 2004);
%    \arXiv{math.AG/0203260}.

\refref{LM1} J.~M. Landsberg and L.~Manivel,
    {``Construction and classification of
    simple Lie algebras via projective geometry,''}
    {\em Selecta Mathematica  \bf 8}, 137 (2002). % p137-159.

\refref{LMW02} J.~M. Landsberg, L.~Manivel, and B.~W. Westbury,
    {``Series of nilpotent orbits,''}
    {\em Experimental Math \bf 13}, 13 (2004); % 13 29
        \arXiv{math.AG/0212270}.

was lands06 :
\refref{Landsberg2006143} J.~M. Landsberg and L.~Manivel,
        ``The sextonions and $E_{7\; 1/2}$,''
        {\em Adv.  Math. \bf 201}, 143-179 (2006); % 143-179.
        \arXiv{math.RT/0402157}.

\refref{lands04a} J.~M. Landsberg and L.~Manivel,
        {``A universal dimension
        formula for complex simple Lie algebras,''}
        {\em Adv.  Math. \bf 201}, 379 (2006); % 379-407.
        \arXiv{math.RT/0401296}.

\refref{lands03} J.~M. Landsberg,
Journal feur die reine und angewandte Mathematik (Crelle) {\bf 562}, 1 (2003). %p 1-3.

was lands02b :
\refref{lands01} J.~M. Landsberg and L.~Manivel,
    "Triality, exceptional Lie algebras, and Deligne dimension formulas,"
    Advances in Mathematics {\bf171}, 59 (2002). %p 59-85.

\refref{lands02c} J.~M. Landsberg and L.~Manivel,
    "Construction and classification of complex simple
    Lie algebras via projective geometry,"
    {\em Selecta Mathematica \bf 8}, 137 (2002). %137-159


%\refref{Lang71} S. Lang, %     Lang, Serge
%        {\em Linear Algebra}
%        (Addison-Wesley, Reading, MA 1971).
%        % 400 p.
%        % QA184 .L38 1971

\refref{LawPet05} S. Lawton and E. Peterson,
        ``Spin networks and $SL(2,C)$-Character varieties.''
        \arXiv{math.QA/0511271}

%\refref{Levinson56} I.~B.~Levinson,
%``Sums of Wigner coefficients and their graphical representation,''
%{\em Proceed. Physical-Technical Inst. Acad. Sci. Lithuanian SSR  \bf 2},
%17 (1956). % 17-30
%
%\refref{LICHTENBERG} D.~B.~Lichtenberg,
%{\em Unitary Symmetry and Elementary Particles}
%(Academic Press, New York 1970).

\refref{Lurie2001} J.~Lurie,   % Jacob Lurie,
    ``On simply laced Lie algebras and
    their minuscule representations,''
    undergraduate thesis (Harvard Univ. 2000);
    {\em Comment. Math. Helv. \bf 76},  515 (2001). % no. 3, 515-575.
% Dylan Thurston (2003):
% This considers exceptional Lie algebras from the point of view of
% their {\em miniscule representations}: the representation with the
% smallest dimension.
% something similar, but more explicit than \refref{Wenzl2003},
% for E_6 and E_7.


%%MMMMMMMMMMMMMMMMMMMMMMMMMMMMMMMMMMMM


\refref%[Ma90a]
        {MR91k:17011} Z.~Q. Ma,
``Rational solution for the minimal representation of $G\sb 2$,''
{\em J.~Phys.~A  \bf 23}, 4415 (1990).

\refref%[Ma90b]
        {MR92a:82036} Z.~Q. Ma,
``The spectrum-dependent solutions to the Yang-Baxter
  equation for quantum $E\sb 6$ and $E\sb 7$,''
{\em J.~Phys.~A  \bf 23}, 5513 (1990).

\refref%[Ma91]
        {MR92b:17019} Z.~Q. Ma,
``The embedding $e\sb 0$ and the spectrum-dependent
  $R$-matrix for $q$-$F\sb 4$,''
{\em J.~Phys.~A  \bf 24}, 433 (1991).

\refref{Macfarlane1968} A.~J.~Macfarlane, A.~Sudbery, and P.~M.~Weisz,
    {\em Comm. Math. Phys.  \bf 11}, 77 (1968);
    {\em Proc. Roy. Soc. Lond.  A \bf 314}, 217 (1970).

% There are many results on
% representation theory formulated in tensor notation. They could be
%translated into diagrams, but here they are presented with many indices:
%
%Hendryk Pfeiffer 2 Dec 2002
\refref{MacPfe00} A.~J.  Macfarlane and H. Pfeiffer,
    ``On characteristic equations, trace identities
  and Casimir operators of simple Lie algebras,"
  {\em J. Math. Phys. \bf 41}, 3192 (2000); % 3192-3225,
  Erratum: {\em \bf 42}, 977 (2001).

%Hendryk Pfeiffer 2 Dec 2002:
% There are many results on
% representation theory formulated in tensor notation. They could be
%translated into diagrams, but here they are presented with many indices:
% Macfarlane Jan 2003: deserves citation:
%\refref{Macfarlane01} A.~J.~Macfarlane,
%    ``Lie algebra and invariant tensor technology for $g_2$,"
%    {\em Internat. J. Mod. Phys.  A \bf 16}, 3067 (2001);  %3067-3097
%    \arXiv{math-ph/0103021}.

% in MacFarlane ref report, metaplectic reference [4]
    % Sec. 4 of [4] employs an operator (7)
\refref{MPW01} A.~J.  Macfarlane, H.  Pfeiffer, and F.  Wagner,
    ``Symplectic and orthogonal Lie algebra technology for
      bosonic and fermionic oscillator models of integrable systems,''
    {\em Internat. J. Mod. Phys. A \bf 16},  1199 (2001); %  1199-1225
        \arXiv{math-ph/0007040}.

%\refref%[Mac91]
%        {MR92j:17013} N.~J. MacKay,
%``Rational $R$-matrices in irreducible representations,''
%{\em J.~Phys.~A  \bf 24}, 4017 (1991).

\refref%[Mac92]
        {MR94c:82029} N.~J. MacKay,
``The full set of $C\sb n$-invariant factorized
  $S$-matrices,''
{\em J.~Phys.~A  \bf 25}, L1343 (1992).

\refref%[Mac02]
        {mackay} N.~J. MacKay,
``Rational $K$-matrices and representations of twisted Yangians,''
{\em J.~Phys.~A  \bf 35}, 7865 (2002); %7865-7876
%?? \arXiv{QA/0205155}.

%\refref{mackay-2006} N.~J. MacKay and A. Taylor,
%    ``Rational {$R$}-matrices, centralizer algebras and tensor identities for
%    {$e_6$} and {$e_7$} exceptional families of {Lie} algebras,''
%       (2006);
%    \arXiv{math/0608248}.

%\refref{Mandula} J.~B.~Mandula,
%    ``Diagrammatic techniques in group theory,''
%    notes taken by S.~N.~Coulson and A.J.G.~Hay
%    (Univ. of Southampton, 1981), unpublished.
%    % Tony.Hey@epsrc.ac.uk  4/2003
%
%\refref{Manivel01} L.~Manivel,
%    {\em Symmetric Functions,
%    Schubert Polynomials and Degeneracy Loci}
%    (American Math. Society, Providence, RI 2001),\\
% \HREF{http://www-fourier.ujf-grenoble.fr/$\sim$manivel/cours.html}{www-\penalty
%  -10\null fourier.\penalty -10\null ujf-\penalty -10\null grenoble.\penalty
%  -10\null fr/\penalty -20\null \char `\~manivel/\penalty -20\null
%  cours.\penalty -10\null html}. % ISBN: 0821821547

%\refref{Maru:1997fq} N.~Maru and S.~Kitakado,
%``Negative dimensional group extrapolation and a new chiral-nonchiral
%duality in $N = 1$ supersymmetric gauge theories,''
%{\em Mod.  Phys.  Lett.    A \bf 12}, 691 (1997);
%\arXiv{hep-th/9609230}.
%%%CITATION = HEP-TH 9609230;%%
%%\href{http://www.slac.stanford.edu/spires/find/hep/www?eprint=HEP-TH/9609230}{SPIRES}

\refref{Mehta66} M.~L.~Mehta,
{\em J.~Math.~Phys.  \bf 7}, 1824 (1966).

\refref{MehtaSriva66} M.~L.~Mehta and P.~K.~Srivastava,
{\em J.~Math.~Phys.  \bf 7}, 1833 (1966).

\refref{McDonald1973} A.~McDonald and S.~P.~Rosen,
{\em J.~Math.~Phys.  \bf 14}, 1006 (1973).

\refref{McKay1977} W.~G.~McKay, J.~Patera, and R.~T.~Sharp,
{\em J. Math. Phys.  \bf 17}, 1371 (1977).

\refref{McKay1981} W.~G.~McKay and J.~Patera,
    ``Tables of dimensions, indices and branching rules for representations of
      simple Lie algebras''
    % {\em Lecture Notes in Pure and Appl. Math. \bf 69}
    (Dekker, New York 1981).

\refref{McKay1986} W.~G.~McKay, R.~V.~Moody, and J.~Patera,
``Decomposition of tensor products of $E8$ representations,''
{\em Alg. Groups Geom. \bf 3}, 286 (1986).

\refref{McKay1990} W.~G.~McKay, J.~Patera, and D.~W.~Rand,
{\em Tables of Representations of Simple Lie Algebras}, vol.{ I}:
{\em Exceptional Simple Lie Algebras}
(Les publications CRM, Universit{\'e} de Montr{\'e}al 1990).

\refref{Mehra94} J.~Mehra,
    {\em The Beat of a Different Drum}
    (Oxford Univ. Press, Oxford 1994).

\refref{Messiah1966} A.~Messiah,
    {\em Quantum Mechanics}
% {\em Appendix C.}
    (North-Holland, Amsterdam 1966).

% \refref{Messiah62}
% A.~Messiah,
% \newblock {\em Quantum Mechanics} Vol.~2 (North-Holland, Amsterdam, 1962).

\refref{Meyberg1960} K.~Meyberg,
{\em Ned. Akad. Wetensch. Proc.  A \bf 71}, 162 (1960).

%\refref%[Mey84]
%        {MR86j:17011} K.~Meyberg,
%    ``Spurformeln in einfachen Lie-Algebren,''
%    {\em Abh. Math. Sem. Univ. Hamburg  \bf 54}, 177 (1984). % 177-189
%%  Hendryk:
%%  Meyberg actually uses the structure of the decomposition of adjoint
%%  tensor adjoint as the input. His calculation shows that the absence
%%  of a primitive quartic Casimir IS directly related to the Deligne
%%  conjecture. Your (7.53) should agree with Meyberg.

\refref{Mey93} K.~Meyberg,
    ``Trace formulas in various algebras and $L$-projections,''
    {\em Nova J. of Algebra and Geometry \bf 2}, 107 (1993). % 107-135
%  Hendryk:
%  By the way, there is another Meyberg article in an obscure journal
%  which is definitely worth looking at,

\refref{Michel1973}
L.~Michel and L.~A.~Radicati,
{\em Ann. Inst. Henri Poincar{\'e}  \bf 18}, 13 (1973).

\refref{Michel:1992kf}
M.~Michel, V.~L.~Fitch, F.~Gursey, A.~Pais, R.~U.~Sexl, V.~L.~Telegdi,
and E.~P.~Wigner,
``Round table on the evolution of symmetries,''
in {\em Sant Feliu de Guixols 1983, Proceedings, Symmetries in Physics % 625
(1600--1980)}.

\refref{Mirman} R. Mirman,
    {\em Group Theory: An Intuitive Approach}
    (World Scientific, Singapore 1995).

%\refref{Mkrtchian:1981bb} R.~L.~Mkrtchyan,
%    ``The equivalence of {$Sp(2n)$} and {$SO(-2n)$} gauge theories,''
%    {\em Phys. Lett. B \bf  105}, 174 (1981).

%\refref{Mkrtchian:2011} R.~L.~Mkrtchyan and A.~P.~Veselov,
%    ``On duality and negative dimensions in the theory
%    of Lie groups and symmetric spaces,''
%    {\em J. Math. Phys. \bf  },   (2011);
%	\arXiv{1011.0151}.

%\refref{Moody}

\refref{Mo89} S. Morita,
``Casson's invariant for homology
$3$-spheres and characteristic classes of surface bundles, {I},''
    {\em Topology  \bf 28}, 305 (1989).  % 305-323

\refref{Mo91} S. Morita,
``On the structure of the Torelli group and the Casson invariant,''
    {\em Topology  \bf 30}, 603 (1991). % 603-621

% There are many results on
% representation theory formulated in tensor notation. They could be
%translated into diagrams, but here they are presented with many indices:
%
%Hendryk Pfeiffer 2 Dec 2002
\refref{Mountain98} A.~J.~Mountain,
``Invariant tensors and Casimir operators for simple compact Lie groups,''
{\em J. Math. Phys. \bf 39}, 5601 (1998);
\arXiv{physics/9802012}.

    \PC{cite \refref{Mountain98}}

\refref{Mukunda1975}
N.~Mukunda and L.~K.~Pandit,
{\em J.~Math.~Phys.  \bf 6}, 746 (1975).
%matrix representation; King's College report.

\refref{Murphy1972}
T.~Murphy,
{\em Proc.~Camb.~Philos.~Soc. \bf 71}, 211 (1972).

\refref{MurRashid1973} G.~Murtaza and M.~A.~Rashid,
{\em J. Math. Phys. \bf 14}, 1196 (1973).


%%NNNNNNNNNNNNNNNNNNNNNNNNNNNNNNNNNNNN

\refref{Neville1963}
D.~E.~Neville,
{\em Phys.~Rev.  \bf 132}, 844 (1963).

\refref{Nomizu79} K. Nomizu, % Nomizu, Katsumi
    {\em Fundamentals of Linear Algebra}
    (Chelsea Publ., New York 1979).
    % 325 p.
    % QA184 .N65

%%OOOOOOOOOOOOOOOOOOOOOOOOOOOOOOOOOOOO


\refref%[OW86]
        {MR87k:82018}
E.~Ogievetsky and P.~Wiegmann,
``Factorized $S$-matrix and the Bethe
  ansatz for simple Lie groups,''
{\em Phys.~Lett.~B  \bf 168}, 360 (1986). % 360-366

\refref{Okubo:1977se}
S.~Okubo,
``Casimir invariants and vector operators in simple
and classical Lie algebras,''
{\em J. Math. Phys. \bf 18}, 2382 (1977). % 2382-2394

\refref{Okubo:1977sc}
S.~Okubo,
``Gauge groups without triangular anomaly,''
{\em Phys.  Rev.    D \bf 16}, 3528 (1977).
%%CITATION = PHRVA,D16,3528;%%
%\href{http://www.slac.stanford.edu/spires/find/hep/www?j=PHRVA%2cD16%2c3528}{SPIRES}

\refref{Okubo:1977sd}
S.~Okubo,
``Constraint on color gauge group,''
{\em Phys. Rev. D \bf 16}, 3535 (1977).
%%CITATION = PHRVA,D16,3535;%%
%\href{http://www.slac.stanford.edu/spires/find/hep/www?j=PHRVA%2cD16%2c3535}{SPIRES}

%\refref{Okubo:1979qe}
%S.~Okubo,
%``Quartic trace identity for exceptional Lie algebras,''
%{\em J.  Math.  Phys.    \bf 20}, 586 (1979).
%%%CITATION = JMAPA,20,586;%%
%%\href{http://www.slac.stanford.edu/spires/find/hep/www?j=JMAPA%2c20%2c586}{SPIRES}

\refref{Okubo:1982td}
S.~Okubo,
``Modified fourth order Casimir invariants and indices for simple Lie algebras,''
{\em J.  Math.  Phys.    \bf 23}, 8 (1982).
%%CITATION = JMAPA,23,8;%%
%\href{http://www.slac.stanford.edu/spires/find/hep/www?j=JMAPA%2c23%2c8}{SPIRES}

\refref{Okubo:1983sv}
S.~Okubo and J.~Patera,
``Symmetrization of product representations and general indices and simple
Lie algebras,''
{\em J.  Math.  Phys.    \bf 24}, 2722 (1983).
%%CITATION = JMAPA,24,2722;%%
%\href{http://www.slac.stanford.edu/spires/find/hep/www?j=JMAPA%2c24%2c2722}{SPIRES}

\refref{Okubo:1984dt}
S.~Okubo and J.~Patera,
``General indices of representations and Casimir invariants,''
{\em J.  Math.  Phys.    \bf 25}, 219 (1984).
%%CITATION = JMAPA,25,219;%%
%\href{http://www.slac.stanford.edu/spires/find/hep/www?j=JMAPA%2c25%2c219}{SPIRES}

\refref{Okubo1985}
S.~Okubo,
``Branching index sum rules for simple Lie algebras,''
{\em J. Math. Phys.  \bf 26}, 2127 (1985).

\PC{add Okubo book}

%\refref{Olshanetskii1987}
%M.~A.~Olshanetskii and V.B.K.~Rogov,
%``Adjoint representations of exceptional Lie algebras,''
%{\em Theor. Math. Phys.  \bf 72}, 679 (1987).

\refref{OShakiban89} P.~J.~Olver and C.~Shakiban,
``Graph theory and classical invariant theory,''
{\em Adv. Math. \bf 75}, 212 (1989). %.pp.~212-245,

\refref{Olver99} P.~J.~Olver,
{\em Classical Invariant Theory}
% London Mathematical Society Student Texts.
(Cambridge Univ. Press, Cambridge 1999).

\refref{Ord1954}
R.~J.~Ord-Smith,
{\em Phys.  Rev. \bf 94}, 1227 (1954).
% The first paper with group-theoretic diagramatic notation?

\refref{Ore1967}
O.~Ore,
{\em The Four-Color Problem}
(Academic Press, New York 1967).

%%PPPPPPPPPPPPPPPPPPPPPPPPPPPP

\refref{Pang:1994hd}
A.~C.~Pang and C.~Ji,
``A spinor technique in symbolic Feynman diagram calculation,''
{\em J.  Comput.  Phys.    \bf 115}, 267 (1994).
%%CITATION = JCTPA,115,267;%%
%\href{http://www.slac.stanford.edu/spires/find/hep/www?j=JCTPA%2c115%2c267}{SPIRES}

%\refref{ParisiSour1979} G.~Parisi and N.~Sourlas (1979)

\refref{Parisi:1979ka}
G.~Parisi and N.~Sourlas,
``Random magnetic fields, supersymmetry and negative dimensions,''
{\em Phys.  Rev.  Lett.    \bf 43}, 744 (1979).
%%CITATION = PRLTA,43,744;%%
%\href{http://www.slac.stanford.edu/spires/find/hep/www?j=PRLTA%2c43%2c744}{SPIRES}

\refref{Patera1970}
J.~Patera,
{\em J.~Math.~Phys.  \bf 11}, 3027 (1970).

\refref{PateraBose1970}
J.~Patera and A.~K.~Bose,
{\em J.~Math.~Phys.  \bf 11}, 2231 (1970).

\refref{Patera1971} J.~Patera,
{\em J.~Math.~Phys.  \bf 12}, 384 (1971).
% gives an explicit representation of $F_4$.

\refref{PateraSank1973} J.~Patera and D.~Sankoff,
{\em Tables of Branching Rules for Representations of Simple Lie Algebras}
(Universit{\'e} de Montr{\'e}al 1973).

\refref{PateraSharp1977}
J.~Patera, R.~T.~Sharp, and P.~Winternitz,
{\em J. Math. Phys.  \bf 17}, 1972 (1977).

\refref{Patera:1981qy}
J.~Patera and R.~T.~Sharp,
``On the triangle anomaly number of $SU(N)$ representations,''
{\em J.  Math.  Phys.    \bf 22}, 2352 (1981).
%%CITATION = JMAPA,22,2352;%%
%\href{http://www.slac.stanford.edu/spires/find/hep/www?j=JMAPA%2c22%2c2352}{SPIRES}

%%from Westbury, "Vogel" (?) bibliography (5 Dec 2002)
%\refref{MR1928841} B.~Patureau-Mirand, % Bertrand},
%    ``Caract\`eres sur l'alg\`ebre de diagrammes $\Lambda$,"
%    {\em Geom. Topol. \bf 6}, 563
%    % Geometry and Topology}, 563--607 (electronic)
%    (2002).

%%from Westbury, "Vogel" (?) bibliography (5 Dec 2002)
%\refref{MR2001a:17011}  B.~Patureau-Mirand, % Bertrand},
%    ``Caract\`eres sur l'alg\`ebre de diagrammes {$\Lambda$},"
%    {\em C.~R. Acad. Sci. Paris \bf 329}, 803 (1999). % 803--806

\refref{Pauli36} W.~Pauli,
    {\em Ann. Inst. Henri Poincar{\'e}  \bf 6}, 109 (1636).

\refref{Penrose1968} R.~Penrose,
    ``Structure of Spacetime,''
     in {\em Battelle Rencontres},
     C.~M.~de~Witt, and J.~A.~Wheeler, eds., p. 121
    (Benjamin, New York 1968).

%\refref{Penrose1971} R.~Penrose,
%    ``Applications of negative dimensional tensors,''
%    in {\em Combinatorial mathematics and its applications},
%    D.J.A.~Welsh, ed.
%    (Academic Press, New York 1971), 221. % pp. 221-244,  % binors
%
%\refref{Penrose71} R. Penrose,
%    ``Angular momentum: An approach to combinatorial space-time,''
%    in {\em Quantum Theory and Beyond},
%    T.~Bastin, ed.
%    (Cambridge Univ. Press, Cambridge 1971).
%%%http://www.phys.canterbury.ac.nz/~physges/Stedman_pub.pdf (p. 4)

was PenroseMacCullen :
\refref{Penrose:1972ia} R.~Penrose and M.A.H. MacCallum,
    {\em Physics Reports \bf 65} (1973).

\refref{PenroseRind84} R. Penrose and W. Rindler,
    {\em Spinors and Space-time}
    % Vol 1 and 2
    % Cambridge Monographs on Mathematical Physics,
    (Cambridge Univ. Press, Cambridge 1984, 1986).


\refref{Pet07} E. Peterson,
        ``A not-so-characteristic equation: the art of linear algebra.''
        \arXiv{0712.2058}

\refref{Pouliot:2001}
P.~Pouliot,
``Spectroscopy of gauge theories based on exceptional Lie groups,''
{\em J. Phys. A  \bf 34},  8631 (2001).
% (19 October 2001) 8631-8658
% E-mail: pouliot@physics.utexas.edu
% We generate by computer a basis of invariants for the
% fundamental representations of the exceptional Lie groups E6 and E7,
% up to degree 18. relevance for the supersymmetric gauge theories,
% and the self-dual exceptional models. We study the chiral ring of G2
% to degree 13, as well as a few classical groups.


%%RRRRRRRRRRRRRRRRRRRRRRRRRRRRRR

\refref{Racah42} G.~Racah,
``Theory of complex spectra. II,''
{\em Phys. Rev.   \bf 62}, 438 (1942).
    % $3j$ symbol (Racah 1942, Wigner 1931, 1959)

\refref{Racah49} G.~Racah,
{\em Phys.~Rev.  \bf 76}, 1352 (1949).

\refref{Racah50} G.~Racah,
{\em Rend. Lincei   \bf 8}, 108 (1950).

%    \PCedit{
\refref{Racah65} G.~Racah, in
{\em Ergebnisse der exakten Naturwissenschaften \bf 37},
%,  G.~H{\"o}hler, ed.
 p.~28 (Springer-Verlag, Berlin 1965). % pp. 28-84
%           } %end \PCedit{

%\refref{Ramond1977}
%P.~Ramond,
%``Is there an exceptional group in your future? $E(7)$ and the travails of the
%symmetry breaking,''
%{\em Nucl. Phys.  B \bf 126}, 509 (1977).

\refref{Ramond1976}
P.~Ramond,
``Introduction to exceptional Lie groups and algebras,''
{\em CALT-68-577}, 58 (CalTech 1976).

\refref{Rashid73}
M.~A.~Rashid and Saifuddin,
{\em J.~Phys.~A.  \bf 5}, 1043 (1972).

\refref{Read67}
R.~C.~Read,
{\em J.~Combinatorial Theory  \bf 4}, 52 (1968).

\refref{vanRitbergen:1999pn}
T.~van Ritbergen, A.~N.~Schellekens, and J.~A.~Vermaseren,
``Group theory factors for Feynman diagrams,''
{\em Int.  J.  Mod.  Phys.    A \bf 14}, 41 (1999);
\arXiv{hep-ph/9802376}.
%%CITATION = HEP-PH 9802376;%%
%\href{http://www.slac.stanford.edu/spires/find/hep/www?eprint=HEP-PH/9802376}{SPIRES}

%\refref{Robinson61} D.~de~B.~Robinson,
%     {\em Representation Theory of the Symmetric Group}
%    (Univ. Toronto Press, Toronto 1961).

\refref{Rockmore75} R.~Rockmore,
    {\em Phys. Rev. D  \bf 11}, 620 (1975).
    The method of this paper is applicable only to $SU(3)$.

%\refref{Rosenfeld1956} B.~A.~Rosenfeld,    % Boris A. Rosenfeld
%``Geometrical interpretation of the compact
%simple Lie groups of the class,''
%{\em Dokl. Akad. Nauk USSR  \bf 106}, 600 (1956) (in Russian).
%    % 600-603.

\refref{Rosenfeld1962} B.~A.~Rosenfeld,
in {\em Algebraical and Topological Foundations of Geometry},  J.~L.~Tits, ed.
    (Pergamon, Oxford 1962).

\refref{Rost96} M.~Rost,
    ``On the dimension of a composition algebra,''
    {\em Documenta Mathematica \bf 1}, 209 (1996); % 209-214,
\HREF{http://www.mathematik.uni-bielefeld.de/DMV-J/vol-01/10.html}
     {www.mathematik.uni-bielefeld.de/DMV-J/vol-01/10.html}

\refref{Rowe1985} D.~J.~Rowe, B.~G.~Wybourne, and P.~H.~Butler,
    ``Unitary representations, branching rules and matrix elements
      for the non-compact symplectic groups,''
    {\em J. Math. Phys. \bf 18}, 939 (1985). % 939-953

\refref{Rumelhart1997} K.~E.~Rumelhart, % Karl
    ``Minimal representations of exceptional $p$-adic groups,''
    {\em Representation Theory \bf 1}, 133 (1997). % 133-181 (electronic).
% Gordan Savin:
% jos jedna konstrukcija izuzetnih algebri:
% Konstrukcija - na moju sugestiju - koristi Z/3Z gradaciju te Jordanove
% algebre ranga 3.

%%SSSSSSSSSSSSSSSSSSSSSSSSSSSSSSSSS

\refref{Sagan01} B.~E.~Sagan,
     {\em The Symmetric Group}
    (Springer-Verlag, New York 2001).

\refref{Samuel:1980vk}
S.~Samuel,
``$U(N)$ integrals, $1/N$, and the Dewit-'t Hooft anomalies,''
{\em J.  Math.  Phys.    \bf 21}, 2695 (1980).
%%CITATION = JMAPA,21,2695;%%
%\href{http://www.slac.stanford.edu/spires/find/hep/www?j=JMAPA%2c21%2c2695}{SPIRES}

\refref{Schafer1966} R.~D.~Schafer,
{\em Introduction to Nonassociative Algebras}
(Academic Press, New York 1966).

\refref{Schrij2007} A. Schrijver,
    ``Tensor subalgebras and first fundamental theorems in invariant theory,"
    \arXiv{math/0604240}.

%\PC{JM: perhaps published by now}

%\refref{Schur1901} I.~Schur,
%    ``\"Uber eine Klasse von Matrizen, die Sich
%      einer Gegebenen Matrix zuordnen Lassen,''
%    in {\em Gesammelte Abhandlungen}, 1
%    (Springer-Verlag, Berlin 1973). % pp. 1-71

\refref{Schweber94} S.~S.~Schweber,
    {\em QED and the Men Who Made It:
         Dyson, Feynman, Schwinger, and Tomonaga}
    (Princeton Univ. Press, Princeton, NJ 1994).

\refref{Schwinger58} J.~Schwinger, ed.,
    {\em Selected Papers on Quantum Electrodynamics}
        (Dover, New York 1958).

% in MacFarlane ref report, metaplectic reference
    % [3]. G.  Segal
\refref{Segal81} G.\  Segal,
    ``Unitary representations of some infinite-dimensional groups,''
    {\em Comm. Math. Phys. \bf 80}, 301 (1981).
    % MR 82k:22004

\refref%[Ser91]
        {MR92e:17020}
S.~M. Sergeev,
``Spectral decomposition of $R$-matrices for
  exceptional Lie algebras,''
{\em Modern Phys. Lett. A  \bf 6}, 923 (1991). % 923-927

\refref{Sirlin:1981pi}
A.~Sirlin,
``A class of useful identities involving correlated direct products of gamma
matrices,''
{\em Nucl.  Phys.  B  \bf 192}, 93 (1981).
%%CITATION = NUPHA,B192,93;%%
%\href{http://www.slac.stanford.edu/spires/find/hep/www?j=NUPHA%2cB192%2c93}{SPIRES

%\refref{Slansky:1981yr}
%R.~Slansky,
%``Group theory for unified model building,''
%{\em Phys.  Rep.    \bf 79}, 1 (1981).
%%%CITATION = PRPLC,79,1;%%
%%\href{http://www.slac.stanford.edu/spires/find/hep/www?j=PRPLC%2c79%2c1}{SPIRES}

\refref{Sloane73}
N.~J.~A.~Sloane,
\HREF{http://www.research.att.com/~njas/sequences/index.html}
     {\em A Handbook of Integer Sequences}
(Academic Press, New York 1973).

\refref{Springer1959} T.~A.~Springer,
{\em Ned. Akad. Wetensch. Proc.   A \bf 62}, 254 (1959).

\refref{Springer1962} T.~A.~Springer,
{\em Ned. Akad. Wetensch. Proc.   A \bf 65}, 259 (1962).

%\refref{Stanley99} R.~P. Stanley,
%     {\em Enumerative Combinatorics 2}
%        (Cambridge Univ. Press, Cambridge  1999);
%  \HREF{http://www-math.mit.edu/$\sim$rstan/ec}{www-\penalty -10\null math.\penalty
%  -10\null mit.\penalty -10\null edu/\penalty -20\null \char `\~rstan/\penalty
%  -20\null ec/\penalty -20\null}.

%\refref{Stedman:1975ts} G.~E.~Stedman,
%``A diagram technique for coupling calculations in compact groups,''
%{\em J.~Phys.  A \bf 8}, 1021 (1975); {A \bf 9}, 1999 (1976).
%%%CITATION = JPAGB,A8,1021;%%
%%\href{http://www.slac.stanford.edu/spires/find/hep/www?j=JPAGB%2cA8%2c1021}{SPIRE

%\refref{StedmanBook} G.~E.~Stedman,
%{\em Diagram Techniques in Group Theory}
%(Cambridge Univ. Press, Cambridge 1990).
%%%http://www.phys.canterbury.ac.nz/people/stedman.shtml
%%% 2008: Emeritus Stedman_pub.pdf (p. 4)

\refref%[Ste02]
        {stem} J.~R. Stembridge,
 ``Multiplicity-free products and restrictions of Weyl characters,''
     \HREF{http://www.math.lsa.umich.edu/~jrs/papers.html}
          {www.math.lsa.umich.edu/$\sim$jrs/papers.html} (2002).

\refref{Sviridov75} D.~T.~Sviridov, Yu.~F.~Smirnov, and V.~N.~Tolstoy,
{\em Rep.~Math.~Phys.  \bf 7}, 349 (1975).

\refref{Sylvester78} J.~J.~Sylvester,
``On an application of the new atomic theory to the graphical
  representation of the invariants and covariants of binary quantics,
  with three appendices,''
{\em  Amer. J. Math. \bf 1}, 64 (1878). % pp.~64-125, 1878.
%% lots of blah-blah and one plate with many figures

%%TTTTTTTTTTTTTTTTTTTTTTTTTTTTTTTTT


\refref{ThiMieg84} J. Thierry-Mieg,
    {\em C.~R. Acad. Sci. Paris \bf 299}, 1309 (1984).

\refref{Thornblad1967} H.~Th{\"o}rnblad,
{\em Nuovo Cimento   \bf 52A}, 161 (1967).

\refref{Tinkham} M.~Tinkham,
{\em Group Theory and Quantum Mechanics}
(McGraw-Hill, New York 1964).

%\refref{Tits1966} J.~Tits, %Jacques Tits,
%    ``Alg\`ebres alternatives, alg\`ebres de Jordan et alg\`ebres
%    de Lie exceptionnelles,''
%    {\em Indag. Math.  \bf 28}, 223 (1966); %223-237
%    %{\em Ned. Akad. Wetensch. Proc.  A \bf 69}, 223 (1966);
%    {\em Math. Reviews   \bf 36}, 2658 (1966).

\refref{Tits67}
J.~Tits,
{\em Lecture Notes in Mathematics  \bf 40}
(Springer-Verlag, New York 1967).

\refref{Tyburski76} L.~Tyburski,
    {\em Phys. Rev. D \bf 13}, 1107 (1976).

%%UUUUUUUUUUUUUUUUUUUUUUUUUUUUUUUUU


%%VVVVVVVVVVVVVVVVVVVVVVVVVVVVVVVVV

\refref{VanHeijenoort} J.~Van Heijenoort,
    {\em Frege and G\"odel: Two Fundamental Texts in Mathematical Logic}
    (Harvard Univ. Press, Cambridge, MA 1970).

\refref{Vermaseren:1998} J.~A.~Vermaseren,
    ``Some problems in loop calculations,''
\arXiv{hep-ph/9807221}. %1 2 Jul 1998 NIKHEF-98-021

%\refref%[Vin94]
%        {MR96d:22001} {\`E}.~B. Vinberg,
%    {\em Lie groups and {L}ie algebras, {III}},
%    {\em Encyclopaedia of Mathematical Sciences \bf 41}
%     (Springer-Verlag, Berlin 1994);
%     ``Structure of Lie groups and Lie algebras,''
%  a translation of {\em
%  Current problems in mathematics. Fundamental directions \bf 41} (Russian),
%  (Akad. Nauk SSSR, Vsesoyuz. Inst. Nauchn. i Tekhn. Inform., Moscow 1990);
%  [MR 91b:22001], translation by V. Minachin [V. V. Minakhin], translation
%  edited by A. L. Onishchik and \`E.\ B. Vinberg.

\refref{Vin66} {\`E}.~B. Vinberg,
    ``A construction of exceptional simple Lie groups (Russian),"
    {\em Tr. Semin. Vektorn. Tensorn. Anal. \bf 13}, 7 (1966). % 7-9.


%%vogel@mathp7.jussieu.fr
%\refref{PV95}
%\HREF{http://www.math.jussieu.fr/~vogel/}{P.~Vogel},
%``Algebraic structures on modules of diagrams,''
%unpublished preprint (1995).

\refref{vogel99}
\HREF{http://www.math.jussieu.fr/~vogel/}{P.~Vogel},
    ``The universal Lie algebra''
     unpublished preprint (1999).

%%WWWWWWWWWWWWWWWWWWWWWWWWWWWWWWWW

%\refref{WAERDEN} B.~L. van~der Waerden,
%     {\em Algebra}, vol.~2, 4th ed.
%    (Springer-Verlag, Berlin 1959).

\refref{Weil} A. Weil,
   ``Sur certains groupes d'op\'erateurs unitaires,''
   {\em Acta Math.} {\bf 111}, 143--211 (1964).

\refref{Weinberg1972} S.~Weinberg,
    {\em Gravitation and Cosmology}
    (Wiley, New York 1972).

\refref{Weinberg1973} S.~Weinberg,
    {\em Phys. Rev. Lett.  \bf 31}, 494 (1973).

%\refref{WolframYoung} E.~Weisstein,
%    ``Young tableaux,''
%  \HREF{http://mathworld.wolfram.com/YoungTableau.html}{mathworld.\penalty
%        -10\null wolfram.\penalty -10\null com/\penalty -20\null
%        YoungTableau.\penalty -10\null html}.

\refref{Wenzl2003} H.~Wenzl, % Hans Wenzl,
    ``On tensor categories of Lie type $E_N$, $N \ne 9$,''
    {\em Adv. Math. \bf  177}, 66 (2003); 66-104
    \weblink{math.ucsd.edu/~wenzl/}.
% Dylan Thurston (2003):
% This considers exceptional Lie algebras from the point of view of
% their {\em miniscule representations}: the representation with the
% smallest dimension.  Hans shows that the invariants (for the "quantum"
% groups in addition to the classical ones) are nearly generated by a
% single tensor in dimension N (on the nose for N=6,7, and in a direct
% summand for other N, N != 9).
% I think \refref{Lurie2001} does something similar,
% but more explicitly, for E_6 and E_7.

%\refref{Westbury02} B.~W.~Westbury,
%         ``R-matrices and the magic square,''
%         {\em J.~Phys.~A \bf 36},  2857 (2003).

%\refref{Westbury03} B.~W.~Westbury,
%    ``Invariant tensors and diagrams,''
%    {\em Internat. J. Mod. Phys. A \bf 18}, % Supplement October
%    49 (2003) % 49-82


\refref{Westbury06} B.~W.~Westbury,
        ``Sextonions and the magic square,''
         {\em J. London Math. Soc. \bf 73}, 455 (2006). % 455-474,

%\refref{Weyl31} H.~Weyl,
%    {\em The Theory of Groups and Quantum Mechanics}
%    % translated by H. P.Robertson.
%    (Methuen, London 1931).

%\refref{Weyl39} H.~Weyl,
%    {\em The Classical Groups, Their Invariants and
%         Representations}
%    (Princeton Univ. Press, Princeton, NJ 1946).

\refref{WeylBrauer1935} H.~Weyl and R.~Brauer,
    ``Spinors in $n$ dimensions,''
    {\em Amer. J. Math.   \bf 57}, 425 (1935).
%%DG: found here:
%%http://stills.nap.edu/readingroom/books/biomems/rbrauer.html

was Wigner59 :
\refref{WIGNER} E.~P.~Wigner,
    {\em Group Theory and Its Application to the Quantum Mechanics
        of Atomic Spectra}
    (Academic Press, New York 1959).
%\refref{Wigner1931} Wigner 1931, Wigner 1959
% {\em symbols} (Wigner 1959) are related to our $6j$ {\em coefficients}

\refref{Wilczek:1982iz}
F.~Wilczek and A.~Zee,
``Families from spinors,''
{\em Phys.  Rev.    D \bf 25}, 553 (1982).
%%CITATION = PHRVA,D25,553;%%
%\href{http://www.slac.stanford.edu/spires/find/hep/www?j=PHRVA%2cD25%2c553}{SPIRES}

\refref{Wilson:1974sk}
K.~G.~Wilson,
``Confinement of quarks,''
{\em Phys.  Rev.    D \bf 10}, 2445 (1974).
%%CITATION = PHRVA,D10,2445;%%
%\href{http://www.slac.stanford.edu/spires/find/hep/www?j=PHRVA%2cD10%2c2445}{SPIRES}

%\refref{Wybourne70} B.~G. Wybourne,
%    {\em Symmetry Principles and Atomic Spectroscopy}
%    (Wiley, New York 1970).

\refref{Wybourne74} B.~G.~Wybourne,
    {\em Classical Groups for Physicists}
    (Wiley, New York 1974).

\refref{Wybourne:1977ff} B.~G.~Wybourne and M.~J.~Bowick,
    ``Basic properties of the exceptional Lie groups,''
    {\em Austral.  J.  Phys.    \bf 30}, 259 (1977).
%%CITATION = AUJPA,30,259;%%
%\href{http://www.slac.stanford.edu/spires/find/hep/www?j=AUJPA%2c30%2c259}{SPIRES}
%(was cited by
%    Pramoda K. Mohapatra
%   A MODEST STEP TOWARDS A YOUNG TABLEAU METHOD FOR E6
%   Phys.Rev.D33:3142,1986
%)

\refref{Wybourne:1980eh} B.~G.~Wybourne,
``Enumeration of group invariant quartic polynomials in Higgs scalar fields,''
    {\em Austral.  J.  Phys.    \bf 33}, 941 (1980).
%%CITATION = AUJPA,33,941;%%
%\href{http://www.slac.stanford.edu/spires/find/hep/www?j=AUJPA%2c33%2c941}{SPIRES}

\PC{find \refref{Wybourne:1980hh}}
\refref{Wybourne:1980hh}
B.~G.~Wybourne,
        ``Young tableaux for exceptional Lie groups.''


%%XXXXXXXXXXXXXXXXXXXXXXXXXXXXXXXXXXXXXXXXXXXXXXXXXXXXX


%%YYYYYYYYYYYYYYYYYYYYYYYYYYYYYYYYYYYYYYYYYYYYYYYYYYYYY

%\refref{Yamaguti1975} K.~Yamaguti and H.~Asano,
%    {\em Proc. Japan Acad.   \bf 51} (1975).

%\refref{Yeung76} P.~S.~Yeung, {\em Phys. Rev. D \bf 13}, 2306 (1976).

\refref{Yoshimura1975} T.~Yoshimura, % Tetz
``Some identities satisfied by representation matrices of the
exceptional Lie algebra $G_2$'' % 4p
{\em (King's Coll., London, Print-75-0424}, 1975).

%\refref{Young1977} A.~Young,
%     {\em The Collected Papers of Alfred Young}
%    (Univ. Toronto Press, 1977),
%     with the eight articles on quantitative substitutional analysis,
%  including: {\em Proc. London Math. Soc. \bf 33}, 97 (1900);
%    {\bf 28}, 255 (1928); {\bf 31}, 253 (1930).
%  %{\em Proc.~London Math.~Soc.~\bf 33} (1900) 97-146; {\bf 28}
%  %(1928) 255-292; {\bf 31} (1930) 253-272.

% \refref{Yutsis62} A.~P.~Yutsis, (1962).
% one can use Yutsis' (1962) notation
% \DG{maybe the one below?}

%\refref{YutsisLevVan62} A.~P.~Yutsis, I.~B.~Levinson, and V.~V.~Vanagas,
%    {\em Matematicheskiy apparat teorii momenta kolichestva dvizheniya}
%    (Gosudarstvennoe izdatel'stvo politicheskoy i nauchnoy
%    literatury Litovskoy SSR, Vilnius 1960); English translation:
%    {\em The Theory of Angular Momentum},
%    Israel Scientific Translation, Jerusalem 1962
%    (Gordon and Breach, New York 1964).


%%ZZZZZZZZZZZZZZZZZZZZZZZZZZZZZZZZZZZZZZZZZZZZZZZZZZZZ

\refref{Zelobenko73} D.~P.~\v{Z}elobenko,
{\em Compact Lie Groups and Their Representations},
trans. of Math. Monographs {\bf 40}
(American Math. Society, Providence, RI 1973).

\refref%[ZGB91]
        {MR92m:17031}
R.~B. Zhang, M.~D. Gould, and A.~J. Bracken,
``From representations of
  the braid group to solutions of the Yang-Baxter equation,''
{\em Nucl.~Phys.~B  \bf 354}, 625 (1991).   %625-652


%%XXXXXXXXXXXXXXXXXXXXXXXXXXXXXX extras  XXXXXXXXXXXXXXXXXXXXXXXXX


% software
\refref{Schellekens}A.~N.~Schellekens,
    {\tt Kac}.

http://tpe.physik.rwth-aachen.de/schweigert/book/Software.html

 Symmetries, Lie algebras and representations,
    J{\"u}rgen Fuchs and Christoph Schweigert, with many links to literature, Lie group programs etc..
    \PC{cite Schweigert book}

\refref{MacfarlaneCocycles} Macfarlane Jan 2003:
There is one very important family of tensors, to which
bird tracks methods could easily be applied, which receives no mention in the
ms. These are the totally antisymmetric tensors that correspond to the
cohomology cocycles of any Lie algebra. Their relation to the symmetric tensors
and Casimir operators that feature centrally in the PC ms is outlined
in \refref{deAzcarraga:1998ya} of the ms.
Their properties, for the $su(n)$ family,
are systematically developed in \refref{MPW01,deAzcarraga:2001}.
Surely they deserve a mention, they are of much greater use in
applications
({\em e.g.} supercharges more than cubic in fermionic operators)
than the Levi-Civita tensors.

% Macfarlane Jan 2003
\refref{OkuboBook} Cite Okubo's book!

%Hendryk Pfeiffer 2 Dec 2002
G Kuperberg, Spiders for rank 2 Lie algebras, Comm Math Phys 180 (1996) 109

%Hendryk Pfeiffer 2 Dec 2002
P Vogel, the material on Vassiliev Theory on his web site


%%Hendryk Pfeiffer 2 Dec 2002
%There is also some connection to
%the extended Freudenthal magic square (I
%have seen that you mention Barton-Sudbery).

% Re:  the paper by Cohen and DeMan.
% We have recently extended some aspects to
% higher order tensor products in
%Hendryk Pfeiffer 2 Dec 2002
% math-ph/0208014

% on some problems with the naive version of this tensor calculus which
% implies the same trouble for the diagrammatical calculus:
%Hendryk Pfeiffer 2 Dec 2002
M Forger, Invariant polynomials and Molien functions, J Math Phys 39 (1998) 1107

% a different flavour of diagrams for quantum groups and
% knot invariants when the over- and undercrossings and the twists in the
% framing are relevant.
%Hendryk Pfeiffer 2 Dec 2002
  N Reshetikhin,VG Turaev, Ribbon graphs and their invariants derived from
    quantum groups, Comm Math Phys 127 (1990) 1
\\
%Hendryk Pfeiffer 2 Dec 2002
  VG Turaev, Quantum invariants of knots and 3-manifolds
    Walter deGruyter, Berlin, 1994
%
% They can be specialized to representations of compact Lie groups, and one
% can prove results about products of 6j-symbols from them:
%Hendryk Pfeiffer 2 Dec 2002
  gr-qc/0211106 and the references therein

% a second type of diagrams using the ideas of
% skein theory in order to accomplish the same,
% for example the `Chain Mail'.
%
%Hendryk Pfeiffer 2 Dec 2002
  DV Boulatov, Quantum deformation of lattice gauge theory,
    Comm Math Phys 186 (1997) 295
\\
%Hendryk Pfeiffer 2 Dec 2002
  JE Roberts, Skein Theory and Turaev-Viro invariants,
    Topology 34 (1995) 771


%%% FIND THESE REFERENCES: if in the book, complete them
%% Comm. Math. Phys. 11, 77
%
%% NOT IN THE BOOK ref.tex:      7jul2007
%
%% Rev. Mod. Phys. 34, 1
%% J. Phys. A A9, 1999
%% J. Math. Phys. 31, 287
%% Adv. Math. 4, 1
%% Adv. Math. 19, 306
%% Proc. Roy. Soc. Lond. Sect.A A218, 345
%% Proc. Roy. Soc. Lond. Sect.A A314, 217
%% Nucl. Phys. B190, 395
