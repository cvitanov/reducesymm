% compile by  pdflatex blog; biber blog
% GitHub cvitanov/reducesymm/dasgroup/dailyBlog.tex


\chapter{Daily group theory blog}
\label{s-groupTheBlog}



\begin{description}

\item[2003-02-05  Bhama Srinivasan] <srinivas@uic.edu>
The reference to an early use of diagrammatic notation is R.
Brauer\rf{Brauer1937} {\em On algebras which are connected with the
semisimple continuous groups} in 1937. The algebra he constructs is now
called the Brauer algebra (essentially birdtracks for SO/Sp:1), but people
now study it algebraically. I had a student who worked on it and found a nice
basis for it in his Ph.D thesis.

\item[2003-03-19  Dirk Kreimer] told me to
read these (perhaps they belong to the QFT blog):

Lavelle and D. McMullan\rf{LavMcM97}
(Predrag not sure if this is the right one, they have many articles together.
Could also be \refrefs{BaLaMcM97,BaLaMcM98})

Suslov\rf{Suslov99}

Rota: On combinatoric (book)
	article on Moebius Function

%\HREF{http://www.yahoo.groups.com}{www.yahoo.groups.com}

\item[2015-12-02  Predrag]
Should one study
Klink and Wickramasekara\rf{KliWic15},
{\em Relativity, Symmetry and the Structure of Quantum Theory I} ?
They say: ``The history of how quantum mechanics was developed is a
fascinating one and underlies the focus of this book; namely, given the
rules that the founders of quantum mechanics developed, is it possible to
find principles that lead to the structure of quantum mechanics as it was
historically formulated? This is the first book in a series of works
considering what particular relativity is applicable to a given dynamical
theory. The series considers Newton, Einstein, and de Sitter
relativities, while this book examines the unitary irreducible
representations of the Galilei group and see how they provide the
framework for Galilean quantum theory.
''

\item[2015-12-02  Predrag] Abbas\rf{Abbas16} {\em Group Theory
in Particle, Nuclear, and Hadron Physics} will be available in July 2016.

\item[2016-04-03 Predrag] I've collected a bunch of group theory e-books,
saved them in \wwwcb{/library}:

MaGu04.pdf\rf{MaGu04}
{\em Problems and Solutions in Group Theory for Physicists}

Dresner98.pdf\rf{Dresner98}
{\em Applications of Lie's Theory of Ordinary and Partial Differential Equations}
discusses, inter alia, similarity and traveling-wave solutions.
I basically never understand this stuff...

Yaglom88.pdf\rf{Yaglom88} {\em Felix Klein and Sophus Lie: Evolution of the
Idea of Symmetry in the Nineteenth Century}

DasOkubo14.pdf\rf{DasOkubo14}
{\em Lie Groups and Lie Algebras for Physicists}.
Their discussion of Lorentz group uses
Clifford algebras, and does not invoke SL(2,C).

Isham\rf{isham99}
{\em Modern Differential Geometry for Physicists}

Fecko06.pdf\rf{Fecko06}
{\em Differential Geometry and Lie Groups for Physicists},

Gilkey \etal\rf{GiPaVa15I,GiPaVa15II,GiPaVa15III} three volume
{\em Aspects of Differential Geometry I, II, III} is available on line
with a Georgia Tech VPN login.

Bergman98.pdf
{\em General (a.k.a. Universal) Algebra},
a highly mysterious and presumably useless subject.

\HREF{https://www.scribd.com/doc/207786199/Q-HO-KIM-Group-Theory-A-Physicist-s-Primer}
{Ho-Kim14.pdf} {\em  Group Theory - A Physicist's Primer}, are lecture
notes. Not sure what they are good for. For example, a chapter on sl(2,C)
does not mention Lorentz group.

From Shlomo Sternberg
\HREF{http://www.math.harvard.edu/~shlomo/}{online books}:
Sternberg04.pdf {\em Lie algebras} is pretty high level.

\item[2017-10-10  Predrag]
Bincer\rf{Bincer12} has a good discussion of SL(2,C) as the covering
group of the proper orthochronous component \SOn{1,3} of the Lorentz group.

The spinor representations $Spin(n)$ were discovered in 1913
by Cartan. What I call Dirac gammas he calls \emph{Clifford} numbers.

Includes biographical sketches of
Galois, Abel, Jacobi, Euler, Lie, Cartan,
Casimir, Weyl, Clebsch, Gordan, Wigner,
Clifford, Schur,
Dynkin, Racah,
Hurwitz, Hamilton, Graves, Cayley, Frobenius,
Lorentz, de Sitter, Liouville, Maxwell, Thomas,
Minkowski, Klein, Gordon, Dirac, Proca,
Poincar\'e, Pauli, Lubanski, Kac, Moody,
Coulomb, Heisenberg, Lenz, and Runge. Does not cite Cvitanovi\'c.

\item[2017-10-10  Predrag]
Ramond\rf{Ramond10} does cite Cvitanovi\v{c}.

\item[2017-11-01 Predrag]
This Russian site has many group-theory books (and any other, I presume):\\
\HREF{http://nozdr.ru/biblio/kolxo3/m/mps} {nozdr.ru}


\item[2017-10-10  Predrag]
A possible addendum to history of birdtracks is in my
{\em Birdtracks - updated history} in\\
\texttt{PHYS-7143-17/notes/weeks/week9.tex},
\HREF{http://birdtracks.eu/courses/PHYS-7143-17/bTrackHistory.pdf}
{online version}.
But looking at figures included in Penrose\rf{Penr04} - I should
really give him all the credit for diagrammatic notation...



\end{description}


%\newpage %%%%%%%%%%%%%%%%%%%%%%%%%%%%%%%%%%%%%%%%%%%%%%%%
\printbibliography[heading=subbibintoc,title={References}]
