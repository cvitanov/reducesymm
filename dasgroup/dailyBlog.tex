% reducesymm/dasgroup/dailyBlog.tex
% Predrag  switched to github.com               jul  8 2013


\chapter{Daily group theory blog}
\label{s-groupTheBlog}



\begin{description}

\item[2013-02-22 PC] Fomin and Pylyavskyy\rf{FomPyl12}
{\em Tensor diagrams and cluster algebras}, {\arXiv{1210.1888}},
is a major orgy in birdtracking. Should study it some day.

\item[2014-07-20 PC] More birdtracking:

Gu and Jockers\rf{GuJock14}
 {\em A note on colored {HOMFLY} polynomials for hyperbolic knots from {WZW} models}

Kol and Shir\rf{KolShir14} {\em Color structures and permutations}.

\item[2014-12-02 PC] More birdtracking:

Geyer and Lazar\rf{GeyLaz00}
{\em Twist decomposition of nonlocal light-cone operators {II:} general tensors of 2nd rank}

Costa and Hansen\rf{CosHan14}
{\em Conformal correlators of mixed-symmetry tensors}

Rejon-Barrera and Robbins\rf{RejRob16} {\em Scalar-vector bootstrap}

Costa \etal\rf{Costa2016}
{\em Projectors and seed conformal blocks for traceless mixed-symmetry tensors}

\item[2014-08-07  Predrag]
Phil Morrison told me that Birkhoff discovered $G_2$ while looking for
hidden symmetries of the heat kernel. Did not find the reference (it is
in his thin book on statistical mechanics?) but this is of interest:

Ilka Agricola\rf{Agricola08} writes:
``
In a talk delivered in Leipzig on June 11, 1900, Friedrich
Engel\rf{Engel1900} gave the first public account of his newly discovered
description of the smallest exceptional Lie group $G_2$, and he wrote in
the corresponding note to the Royal Saxonian Academy of Sciences:
Moreover, we hereby obtain a direct definition of our 14-dimensional
simple group $G_2$ which is as elegant as one can wish for. Indeed,
Engel's definition of $G_2$ as the isotropy group of a generic 3-form in
7 dimensions is at the basis of a rich geometry that exists only on
7-dimensional manifolds,
''

This is precisely how I think of $G_2$, so should give Engel and
Reichel credit, and
cite Agricola\rf{Agricola08}.

One immediately discovers that Baez and Huerta\rf{BaeHue14} think of
$G_2$ as a small ball rolling on a big ball, and so on; with this one can
be infinitely distracted.

\item[2015-12-02  Predrag]
Should one study
Klink and Wickramasekara\rf{KliWic15},
{\em Relativity, Symmetry and the Structure of Quantum Theory I} ?
They say: ``The history of how quantum mechanics was developed is a
fascinating one and underlies the focus of this book; namely, given the
rules that the founders of quantum mechanics developed, is it possible to
find principles that lead to the structure of quantum mechanics as it was
historically formulated? This is the first book in a series of works
considering what particular relativity is applicable to a given dynamical
theory. The series considers Newton, Einstein, and de Sitter
relativities, while this book examines the unitary irreducible
representations of the Galilei group and see how they provide the
framework for Galilean quantum theory.
''

\item[2015-09-14 Bernard Julia]
I at last have maybe some hint on the difference and on the common origin
of our two distinct magic triangles. I asked P. Deligne for advice but not
very diplomatically I enquired about his relation with your work so total
silence was the answer.
I discussed since with P. Vogel and was wondering whether you had made
progress. My breakthrough comes out of nowhere so it takes time to
decipher.
I suggested to Vogel he should go beyond powers of the adjoint
representation (your idea for the triangle) I probably should do it if it
has not been done before...

Answered 2016-02-06 by dasgroup@gatech.edu email,
complaining that `` I do not know why Deligne is so unwilling to cite my
work that precedes his - he has known about it forever
(\HREF{http://birdtracks.eu/extras/Deligne96.pdf} {click here}). I
have never met him. I had once been in Vogel's office (about 10 years
ago?). The desk was covered by birdtracks :) I'm a bit annoyed that that
things that I did decades before Vogel and are now credited to them, and
I get cited only for Vogletracks, not Vogelsong. But I do have a tenured
job and freedom to do anything, so who cares. I too find Deligne's paper
more beautiful than my own formulation. It's just that I did it first,
but apparently if you are not a "mathematician" it does not count.

Bernard Julia response:
I visited Vogel about 9 months ago about his (super)group manifold and
his version of the E line! He is trying to disprove a conjecture of
Deligne in this field.

Here is the general idea of my program, nothing has been published.
The theory of second order Painlev\'e differential ODE is classical. The
definition of difference Painlev\'e equations is not uniformly agreed upon
however but the main one due to Sakai does indeed suggest that my En
series (equivalently that of the Del Pezzo's) correspond to so called
q-difference equations whereas yours (I mean the A0 A1 A2 end) is closer to
those diagrams appearing in additive difference Painlev\'e equations but F4
and G2 are missing today. I urged the experts to look for them and one more
branch appears with A3 symmetry as an important piece. Furthermore I am
trying to clarify  the delicate connection(s) to (continuous) differential
equations and c-Painlev\'e VI takes a leading role there only.
A very small workshop: One day would cover discretizations of
symmetries and of spaces, one day for the characteristic properties of
Painlev\'es (what makes them polyquitous) and one day for algebraic geometry
(curves appear in string theory but also surfaces ...Del Pezzo's...),
Calabi-Yau's etc... but confinement of singularities and the algebraic
entropy criterion of Viallet et al. is very relevant, you may know about
it from another side, QRT maps etc...)

\item[2016-02-07  Predrag]
Never heard of them until today, but there is huge literature on
\HREF{https://en.wikipedia.org/wiki/Del_Pezzo_surface} {Del Pezzo
surfaces}.


\item[2015-11-11 Bruce Westbury]
I know you no longer work on ``birdtracks" but I thought you might be
interested in my preprint\rf{Westbury15} {\em Extending and quantising
the Vogel plane}. \\
(Answered 2016-02-06 by dasgroup@gatech.edu email,
complaining that ``I'm a bit annoyed that that things that I did decades
before Vogel and Deligne are now credited to them, and I get cited only
for Vogletracks, not Vogelsong.'').

\item[2016-02-08 Bruce Westbury]
I did try to understand spinors but felt I did not succeed. I posted my
attempt in \arXiv{1007.2579}.

I think the problem with the E7 (and also the E6, F4 D4 series) is that
it does not actually form a series. There is nothing written on this but
this does seem to be the consensus. When I say it is not a series, I
mean you end up with a finite list (so no parameter). This is known to
happen for the G2 series. I wrote an account of this in
\arXiv{1011.6197}

However, I still believe the E8 series exists as Deligne conjectured and
am currently working on proving this. This seems to me to be crucial.
\\
(Answered 2016-02-08 by dasgroup@gatech.edu email,
complaining that ``
My only bone is that I do not get how the $E_8$ series that I computed in
a 1977 preprint and officially published in 1981 gets to be called
``Deligne conjecture'' from 1996 on. I explain the history in detail in
birdtracks.eu section 21.2 A BRIEF HISTORY OF EXCEPTIONAL MAGIC. They
have recomputed the same formulas, they know of my work, they mention
that they this ``well known to physicists (cf. Cvitanovi\'c [83]),'' and
that's it. The total credit for what was a wonderful breakthrough for me.
That's neither nice nor professional. Oh, screw them :)
'')

\item[2016-02-06  Predrag] Khudaverdian and Ruben Mkrtchyan\rf{KhuMkr16},
{\em Universal volume of groups and anomaly of {Vogel}'s symmetry},
write: ``
We show that integral representation of universal
volume function of compact simple Lie groups gives rise to six analytic
functions on $CP^2$, which transform as two triplets under group of
permutations of Vogel's projective parameters.  This substitutes
expected invariance under permutations of universal parameters by more
complicated covariance.

 We provide an analytical continuation of these functions and particularly
calculate their change  under  permutations of parameters.  This last
relation is universal generalization, for an arbitrary simple Lie group
and an arbitrary point in Vogel's plane, of the Kinkelin's reflection
relation on Barnes' $G(1+N)$ function. Kinkelin's relation gives asymmetry
of the $G(1+N)$ function (which is essentially the volume function for $SU(N)$
groups)  under $N\leftrightarrow -N$ transformation (which is  equivalent
of the permutation of parameters, for $SU(N)$ groups), and coincides with
universal relation on permutations at the $SU(N)$ line on Vogel's plane.
These results are also applicable to universal partition function of
Chern-Simons theory on three-dimensional sphere.

This effect is  analogous to modular covariance, instead of invariance,
of partition functions of appropriate gauge theories under modular
transformation of couplings.
''

Mironov, Mkrtchyan and Morozov\rf{MiMkMo16}
{\em On universal knot polynomials}
``
We present a universal knot polynomials for 2- and 3-strand torus knots
in adjoint representation, by universalization of appropriate Rosso-Jones
formula. According to universality, these polynomials coincide with
adjoined colored HOMFLY and Kauffman polynomials at SL and SO/Sp lines on
Vogel's plane, respectively and give their exceptional group's
counterparts on exceptional line. We demonstrate that [m,n]=[n,m]
topological invariance, when applicable, take place on the entire Vogel's
plane. We also suggest the universal form of invariant of figure eight
knot in adjoint representation, and suggest existence of such
universalization for any knot in adjoint and its descendant
representations.
''

Have a look at

Mkrtchyan\rf{Mkrtchyan14} {\em On a {Gopakumar-Vafa} form of
partition function of {Chern-Simons} theory on classical and exceptional
lines}

Mkrtchyan and Veselov\rf{Mkrtchyan_2012}
{\em Universality in {Chern-Simons} theory}

Mkrtchyan is not spring chicken. The funny thing is that, while there are
legions of young Witteninos, all this work seem to be carried out by old
men.



\item[2015-12-02  Predrag]
Email to Ruben:

I have not been working on these problems for a while, so apologies if
everything I write about here is something you already know. We have
uncovered more $N \to -N$ relationships than just $SO(-n)=Sp(n)$
(\HREF{http://birdtracks.eu/refs/index.html}{click here}).

In {\em Spinors in negative dimensions},
Phys. Scripta 26, 5 (1982) Tony Kennedy and I did it for ``spinsters".

In my (1981) {\em Negative dimensions and $E_7$ symmetry}\rf{NegDimE7}
 (as well as in Chapter 20. {\em E7 family and its
negative-dimensional cousins} of the birdtracks book) I have a
negative-dimension mapping where $E_7$  appears as a
negative-dimensional relative of $SO(4)$.

That might be of interest for your Vogel song:)

BTW, `Parizi' $\to$ `Parisi'

\item[2016-12-03 Predrag]
Read

Maru and Kitakado\rf{Maru:1997fq}
``Negative dimensional group extrapolation and a new chiral-nonchiral
duality in $N = 1$ supersymmetric gauge theories''

M.~A.~Olshanetskii and V.B.K.~Rogov\rf{Olshanetskii1987}
``Adjoint representations of exceptional Lie algebras,''

MacKay\rf{MacKay05}
{\em Introduction to {Yangian} symmetry in integrable field theory}:
``
This, alongside the appearance of X in \refref{PD96,PCgr} and the
unified R-matrix structure of \refref{Westbury02}, suggests that it might be
interesting to investigate the connection between Yangians and the `magic
square' construction of the exceptional Lie groups.
''

Westbury\rf{Westbury06a}
{\em Universal characters from the {MacDonald} identities}


\item[2015-12-02  Predrag] Abbas\rf{Abbas16} {\em Group Theory
in Particle, Nuclear, and Hadron Physics} will be available in July 2016.

\item[2016-04-03 Predrag] I've collected a bunch of group theory e-books,
saved them in \wwwcb{/library}:

Bincer12.pdf\rf{Bincer12}

Ramond10.pdf\rf{Ramond10}

isham99.pdf\rf{isham99}

MaGu04.pdf\rf{MaGu04}

Dresner98.pdf\rf{Dresner98}

Yaglom88.pdf\rf{Yaglom88}

DasOkubo14.pdf\rf{DasOkubo14}

Fecko06.pdf\rf{Fecko06}
{\em Differential Geometry and Lie Groups for Physicists},

Bergman98.pdf

\HREF{https://www.scribd.com/doc/207786199/Q-HO-KIM-Group-Theory-A-Physicist-s-Primer}
{Ho-Kim14.pdf} {\em  Group Theory - A Physicist's Primer}, just lecture
notes.

From Shlomo Sternberg
\HREF{http://www.math.harvard.edu/~shlomo/}{online books}:
Sternberg04.pdf {\em Lie algebras}

\item[2016-08-20 Predrag]
Anastasiou \etal\rf{AnBoHuNa16}
{Global symmetries of {Yang-Mills} squared in various dimensions}: ``
It should be noted that these are not the only magic triangles of Lie
algebras, a particularly elegant and intriguing example being that of
Cvitanovi\'c\rf{C77,PCgr}.
''

\item[2016-08-20 Bernard Julia]
Again, citation is not complete but it is a big story with old references
so you might be interested in (see, for example page 2):

Shimizu and Tachikawa\rf{ShiTac16}
{\em Anomaly of strings of {$6d N = (1,0)$} theories}

Looks like one also has to look at
Kim \etal\rf{KiKiPa16} {\em {6d} strings from new chiral gauge theories}

\item[2016-12-03 Predrag]
Reading Shimizu and Tachikawa\rf{ShiTac16}, and in particular their footnote $^1$.
They credit Deligne for ``Deligne's exceptional series of groups,'' and then
they say ``also independently found by Cvitanovi\'c.'' This after reading
section~21.2 of birdtacks book\rf{PCgr}. I do not get the ethics of
mathematicians, they must think it is a town in Russia. If I discover the
series in 1975-77, and Deligne rediscovers almost 20 years
later\rf{PD96}, fully aware that I had discovered it (click
\HREF{http://birdtracks.eu/extras/Deligne96.pdf}{here}), how is that the
``Deligne series?''

They also refer to {1988} Mathur and Mukhi and Sen paper\rf{MaMuSe88},
Table~4 of Beem \etal\rf{BLLPRR15}, and Table~1 of Lemos and
Liendo\rf{LemLie16}. The 2000 paper of Grassi and Morrison\rf{GraMor00}
refers only to Deligne, does not cite me at all.



%%%%%%%%%%%%%%%%%%%% Bryant\rf{Bryant99} Table~IV %%%%%%%%%%%%%%
% reducesymm/dasgroup/Bryant99tabIV.tex
% Predrag  2015-12-03

%%%%%%%%% Bryant\rf{Bryant99} macros %%%%%%%%%%%%%%%%%%%%%
\def\Lam{\hbox{\sans\char3}}%sans Lambda
\def\Sym{\hbox{\sans S}}
\def\SO{{\rm SO}}\def\SL{{\rm SL}}
\def\Sp{{\rm Sp}}\def\SU{{\rm SU}}
\def\CO{{\rm CO}}\def\GL{{\rm GL}}
\def\Spin{{\rm Spin}}\def\U{{\rm U}}
\def\A{{\rm A}}\def\C{{\rm C}}\def\D{{\rm D}}
\def\E{{\rm E}}\def\O{{\rm O}}
\def\Diff{{\rm Diff}}\def\G{{\rm G}}
\def\bbC{{\Bbb C}}
\def\bbF{{\Bbb F}}
\def\bbH{{\Bbb H}}
%definition of the font for footnotes
\font\ninerm=cmr9%
% font for Schur functor symbols: CM sans serif
\font\sans=cmss10
%%%%%%%%% Bryant\rf{Bryant99} macros %%%%%%%%%%%%%%%%%%%%%
\begin{table}
\centerline{
\vtop{
\offinterlineskip
\halign{
&\vrule#&\strut\quad\hfil#\hfil\quad\cr  %The preamble
\multispan4{\hfil \bf IV. Exotic Symplectic Holonomies \hfil}\cr
\noalign{\smallskip}
\multispan4{\hrulefill}&\cr
height 2 pt&\omit&&\omit&\cr
& H && ${\mathfrak m}$ &\cr
height 2 pt&\omit&&\omit&\cr
\multispan4{\hrulefill}&\cr
height 1 pt&\omit&&\omit&\cr
\multispan4{\hrulefill}&\cr
height 2 pt&\omit&&\omit&\cr
&   &&   &\cr
height 3 pt&\omit&&\omit&\cr
\multispan4{\hrulefill}&\cr
height 2 pt&\omit&&\omit&\cr
& $\SL(2,\bbR)$ && $\bbR^4\simeq\Sym^3(\bbR^2)$ &\cr
height 1 pt&\omit&&\omit&\cr
& $\SL(2,\bbC)$ && $\bbC^4\simeq\Sym^3(\bbC^2)$ &\cr
height 2 pt&\omit&&\omit&\cr
\multispan4{\hrulefill}&\cr
height 2 pt&\omit&&\omit&\cr
& $\SL(2,\bbR)\!\cdot\!\SO(p,q)\,^a$ && $\bbR^{2}\otimes\bbR^{p+q}$ &\cr
height 1 pt&\omit&&\omit&\cr
& $\SL(2,\bbC)\!\cdot\!\SO(n,\bbC)\,^b$ && $\bbC^{2}\otimes\bbC^{n}$ &\cr
height 1 pt&\omit&&\omit&\cr
& $\Sp(1)\!\cdot\!\SO(n,\bbH)\,^c$ && $\bbH^{n}$ &\cr
height 2 pt&\omit&&\omit&\cr
\multispan4{\hrulefill}&\cr
height 2 pt&\omit&&\omit&\cr
& $\Sp(3,\bbR)$ && $\bbR^{14}\simeq \Lam^3_0(\bbR^6)$ &\cr
height 1 pt&\omit&&\omit&\cr
& $\Sp(3,\bbC)$ && $\bbC^{14}\simeq \Lam^3_0(\bbC^6)$ &\cr
height 2 pt&\omit&&\omit&\cr
\multispan4{\hrulefill}&\cr
height 2 pt&\omit&&\omit&\cr
& $\SL(6,\bbR)$ && $\bbR^{20}\simeq\Lam^3(\bbR^6)\phantom{^\bbR}$ &\cr
height 1 pt&\omit&&\omit&\cr
& $\SU(1,5)$ && $\bbR^{20}\simeq\Lam^3(\bbC^6)^\bbR$ &\cr
height 1 pt&\omit&&\omit&\cr
& $\SU(3,3)$ && $\bbR^{20}\simeq\Lam^3(\bbC^6)^\bbR$ &\cr
height 1 pt&\omit&&\omit&\cr
& $\SL(6,\bbC)$ && $\bbC^{20}\simeq\Lam^3(\bbC^6)\phantom{^\bbR}$ &\cr
height 2 pt&\omit&&\omit&\cr
\multispan4{\hrulefill}&\cr
height 2 pt&\omit&&\omit&\cr
& $\Spin(2,10)$ && $\bbR^{32}$ &\cr
height 1 pt&\omit&&\omit&\cr
& $\Spin(6,6)$ && $\bbR^{32}$ &\cr
height 1 pt&\omit&&\omit&\cr
& $\Spin(6,\bbH)$ && $\bbR^{32}$ &\cr
height 1 pt&\omit&&\omit&\cr
& $\Spin(12,\bbC)$ && $\bbC^{32}$ &\cr
height 2 pt&\omit&&\omit&\cr
\multispan4{\hrulefill}&\cr
height 2 pt&\omit&&\omit&\cr
& $\E_7^5$ && $\bbR^{56}$ &\cr
height 1 pt&\omit&&\omit&\cr
& $\E_7^7$ && $\bbR^{56}$ &\cr
height 1 pt&\omit&&\omit&\cr
& $\E_7^\bbC$ && $\bbC^{56}$ &\cr
height 2 pt&\omit&&\omit&\cr
\multispan4{\hrulefill}&\cr
}
}
%\vskip-10pt}
\hskip1pt
\vtop{\offinterlineskip
\halign{
&\vrule#&\strut\quad\hfil#\hfil\quad\cr  %The preamble
\multispan4{\hfil \bf $\E_7$ row \hfil}\cr
\noalign{\smallskip}
\multispan4{\hrulefill}&\cr
height 2 pt&\omit&&\omit&\cr
& algebra && rep. &\cr
height 2 pt&\omit&&\omit&\cr
\multispan4{\hrulefill}&\cr
height 1 pt&\omit&&\omit&\cr
\multispan4{\hrulefill}&\cr
height 2 pt&\omit&&\omit&\cr
& $\U(1)  $ && $\bbR^{2}$ &\cr
height 3 pt&\omit&&\omit&\cr
\multispan4{\hrulefill}&\cr
height 2 pt&\omit&&\omit&\cr
& $\A_1  $ && $\bbR^{4}$ &\cr
height 1 pt&\omit&&\omit&\cr
&   &&   &\cr
height 3 pt&\omit&&\omit&\cr
\multispan4{\hrulefill}&\cr
height 2 pt&\omit&&\omit&\cr
& $3\,\A_1$ && $\bbR^{8}$ &\cr
height 1 pt&\omit&&\omit&\cr
&   &&   &\cr
height 1 pt&\omit&&\omit&\cr
&   &&   &\cr
height 3 pt&\omit&&\omit&\cr
\multispan4{\hrulefill}&\cr
height 2 pt&\omit&&\omit&\cr
& $\C_3$ && $\bbR^{14}$ &\cr
height 1 pt&\omit&&\omit&\cr
&   &&   &\cr
height 3 pt&\omit&&\omit&\cr
\multispan4{\hrulefill}&\cr
height 2 pt&\omit&&\omit&\cr
& $\A_5  $ && $\bbR^{20}$ &\cr
height 1 pt&\omit&&\omit&\cr
&   &&   &\cr
height 1 pt&\omit&&\omit&\cr
&   &&   &\cr
height 1 pt&\omit&&\omit&\cr
&   &&   &\cr
height 3 pt&\omit&&\omit&\cr
\multispan4{\hrulefill}&\cr
height 2 pt&\omit&&\omit&\cr
& $\D_6  $ && $\bbR^{32}$ &\cr
height 1 pt&\omit&&\omit&\cr
&   &&   &\cr
height 1 pt&\omit&&\omit&\cr
&   &&   &\cr
height 1 pt&\omit&&\omit&\cr
&   &&   &\cr
height 3 pt&\omit&&\omit&\cr
\multispan4{\hrulefill}&\cr
height 2 pt&\omit&&\omit&\cr
& $\E_7  $ && $\bbR^{56}$ &\cr
height 1 pt&\omit&&\omit&\cr
&   &&   &\cr
height 1 pt&\omit&&\omit&\cr
&   &&   &\cr
height 3 pt&\omit&&\omit&\cr
\multispan4{\hrulefill}&\cr
}
%\vskip-10pt
}
}
\caption{\label{Bryant99tabIV}
(left)
Bryant Table~IV\rf{Bryant99}.
The $H$ are subgroups of~$\Sp(S)\subset\GL({\mathfrak m})$ for a
nondegenerate skew symmetric bilinear form~$S$ on~${\mathfrak m}$, and the
corresponding $H$-structures have an underlying symplectic structure.
The restrictions are
$(a)$ {\ninerm $p+q\ge3$} (for irreducibility),
$(b)$ {\ninerm $n\ge3$} (for irreducibility),
and
$(c)$ {\ninerm $n\ge2$} (to be nonmetric).
(right)
The $\E_7$ row\rf{NegDimE7} of the Magic Triangle\rf{PCar,C77,PCgr}.
}
\end{table}
%%%%%%%%%%%%%%%%%%%%%%%%%%%%%%%%%%%%%%%%%%%%%%%%%%%%%%%%%%%%%%%%%%%%%%%

%%%%%%%%%%%%%%%%%%%% Bryant\rf{Bryant99} Table~IV %%%%%%%%%%%%%%

\item[2016-10-24 Bernard Julia]
I recently found a niche for your $E_7$ family\rf{NegDimE7}. Please
have a look at Bryant\rf{Bryant99}
and check out \reftab{Bryant99tabIV} in his review of
Berger\rf{Berger55,Berger57} nonmetric holonomies.

You will find among the exotic symplectics \(E_7, D_6, A_5\) (where
everybody agrees), and then your $C_3$, your
$3\,A_1$ (as a possible solution) and your $A_1$.

What do you make of this, except that the symplecticity is just right... ?

I copy this to Pierre <pierre.ramond@gmail.com> who may appreciate.

\item[2016-12-03 Predrag]

My 1975 construction\rf{PCar,C77,PCgr} of the $\E_7$ row of the Magic
Triangle was inspired by Brown's\rf{Brown1969}
observation that defining representation of \(E_7\) has a symmetric quartic
invariant (my construction, however,is a stand-alone derivation that owes
nothing to Brown and Freudethal). I am not able to decode the ``exotic
holonomies'' literature, hopefully the authors can point out if there is such
quartic invariant in their Poisson manifolds derivation. Or identify the
equivalent  class of ternary algebras (Freudenthal triple systems), see
Faulkner\rf{Faulkner1971} and Yamaguti and Asano\rf{Yamaguti1975}.

As nobody understood my diagrammatic notation, in 1980 I rewrote my (very
compact in diagrammatic notation) derivation of the $\E_7$ family in the more
standard and lengthy index notation\rf{NegDimE7}, also pretty much in vain.


Reading \HREF{https://math.duke.edu/people/robert-bryant} {Robert Bryant}
% <bryant@math.duke.edu>, Duke University.
review paper\rf{Bryant99}...
He is not only the current President of the American Mathematical Society.
American Mathematical Society, he is also a NAS, and Landsberg's thesis
adviser, etc. He explains the idea of ``holonomy'' by a
\HREF{http://www.maa.org/meetings/calendar-events/the-idea-of-holonomy}
{basketball}.
Bryant\rf{Bryant99} {\em Recent advances in the theory of holonomy},
\\ \arXiv{math/9910059}
\\~[arXiv now includes blog links:
\HREF{https://en.wikipedia.org/wiki/Holonomy}{wiki/Holonomy}
and
\HREF{http://mathoverflow.net/questions/6475/what-is-the-relationship-between-various-things-called-holonomic/6550\#6550}
{mathoverflow}]:


{\it Exotic Holonomies.}
The full list of exotic holonomies (so called because the were omitted the
initial list of Berger\rf{Berger57}), including those in
\reftab{Bryant99tabIV}, was compiled by
Chi, Merkulov, Schwachh\"ofer, and Bryant\rf{ChMeSc96,Bryant87,Bryant91};
see, in particular, the tables in Merkulov and
Schwachh{\"o}fer\rf{MerSch99,MerSch99add}.

The first two examples in \reftab{Bryant99tabIV}, each with a torsion-free
connection, were analyzed in \refref{Berger57}. The
study\rf{Bryant87,Bryant91} of the moduli space of rational curves on a
complex surface with normal bundle~${\cal O}(3)$ turned up the next two
entries in \reftab{Bryant99tabIV} that were omitted from Berger's nonmetric
list\rf{Bryant91}. The `exotic' refers to any nonmetric
subgroup~$H\subset\GL({\mathfrak m})$ that satisfies Berger's criteria but
that does not appear on Berger's original nonmetric list.
The construction \rf{Bryant91} uncovered a number of unexpected identities.
Chi, Merkulov, and Schwachh\"ofer\rf{ChMeSc96} found other exotic symplectic
examples, and noticed that the reconstruction technique generalized in the
context of Poisson geometry (\ie, Poisson bracket, or a symplectic
skew-symmetric invariant).

% Even for the exceptional holonomies, there are explicit examples of
% cohomogeneity one\rf{Bryant87,BrySal89}.

Merkulov and
Schwachh{\"o}fer\rf{MerSch99} note
``
Another by-product result is  the striking supersymmetry property of their
real forms which occur as holonomies of torsion-free affine connections. With
any symplectic Lie algebra $g \subset \Sp(V)$, dim~$g = m$, dim~$V = 2n$, one
may associate naturally an $(m|2n)$-dimensional supermanifold
''

That actually was a motivation for my $\E_7$ family paper\rf{NegDimE7}, but I
have not identified my $\SOn{4}$ family in the holonomies literature. It
might be in this paper, for all I know...



\end{description}
\renewcommand{\ssp}{a}
