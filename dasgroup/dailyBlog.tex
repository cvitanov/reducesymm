% compile by  pdflatex blog; biber blog
% GitHub cvitanov/reducesymm/dasgroup/dailyBlog.tex


\chapter{Daily group theory blog}
\label{s-groupTheBlog}



\begin{description}

\item[2017-11-01 Predrag]
This Russian site has many group-theory books (and any other, I presume):\\
\HREF{http://nozdr.ru/biblio/kolxo3/m/mps} {nozdr.ru}

\item[2016-04-03 Predrag] I've collected a bunch of group theory e-books,
saved them in \wwwcb{/library}:

MaGu04.pdf\rf{MaGu04}
{\em Problems and Solutions in Group Theory for Physicists}

Dresner98.pdf\rf{Dresner98}
{\em Applications of Lie's Theory of Ordinary and Partial Differential Equations}
\CBlibrary{Dresner98}
discusses, inter alia, similarity and traveling-wave solutions.
I basically never understand this stuff...

Yaglom88.pdf\rf{Yaglom88} {\em Felix Klein and Sophus Lie: Evolution of the
Idea of Symmetry in the Nineteenth Century}
\CBlibrary{Yaglom88}

DasOkubo14.pdf\rf{DasOkubo14}
{\em Lie Groups and Lie Algebras for Physicists}
\CBlibrary{DasOkubo14}
Their discussion of Lorentz group uses
Clifford algebras, and does not invoke SL(2,C).

Isham\rf{isham99}
{\em Modern Differential Geometry for Physicists}
\CBlibrary{isham99}

Fecko06.pdf\rf{Fecko06}
{\em Differential Geometry and Lie Groups for Physicists}
\CBlibrary{Fecko06}

Gilkey \etal\rf{GiPaVa15I,GiPaVa15II,GiPaVa15III} three volume
{\em Aspects of Differential Geometry I, II, III} is available on line
with a Georgia Tech VPN login.

Bergman98.pdf
{\em General (a.k.a. Universal) Algebra},
a highly mysterious and presumably useless subject.

\HREF{https://www.scribd.com/doc/207786199/Q-HO-KIM-Group-Theory-A-Physicist-s-Primer}
{Ho-Kim14.pdf} {\em  Group Theory - A Physicist's Primer}, are lecture
notes. Not sure what they are good for. For example, a chapter on sl(2,C)
does not mention Lorentz group.

From Shlomo Sternberg
\HREF{http://www.math.harvard.edu/~shlomo/}{online books}:
Sternberg04.pdf {\em Lie algebras} is pretty high level.

\item[2022-03-14 Predrag] A few more textbooks:

Isaev and Rubakov\rf{IsaRub18}
{\em Theory of Groups and Symmetries. Finite Groups, Lie Groups, and
Lie Algebras} (2018) \CBlibrary{IsaRub18}

Isaev and Rubakov\rf{IsaRub20}
{\em Theory of Groups and Symmetries.
Representations of Groups and Lie Algebras, Applications}
(2020) \CBlibrary{IsaRub20}

Bradley and Cracknell\rf{ChrBra10}
{\em The Mathematical Theory of Symmetry in Solids:
Representation Theory for Point Groups and Space Groups}
(2010) \CBlibrary{ChrBra10}

Powell\rf{Powell10}
{\em Symmetry, Group Theory, and the Physical Properties of Crystals}
(2010) \CBlibrary{ChrBra10}

Zee\rf{Zee16} {\em Group Theory in a Nutshell for Physicists}
(2016) \CBlibrary{Zee16}

Bruus and Flensberg\rf{Zee16}
{\em Many-Body Quantum Theory in Condensed Matter Physics: An Introduction}
(2004) \CBlibrary{BruFle04}

\item[2017-11-01  Predrag] a new textbook
Campoamor-Stursberg and Rausch de Traubenberg\rf{CamTra19}
{\em Group Theory in Physics} focuses on applications to space-time
symmetries, the Wigner method for representations and applications to
relativistic wave equations, kinematical algebras and groups,
contractions, central extensions and projective representations, gauge
symmetries, the Standard Model, Grand-Unified Theories.

\item[2003-03-19  Dirk Kreimer] told me to
read these (perhaps they belong to the QFT blog):

Lavelle and McMullan\rf{LavMcM97}
(Predrag not sure if this is the right one, they have many articles together.
Could also be \refrefs{BaLaMcM97,BaLaMcM98})

Suslov\rf{Suslov99}

Rota: On combinatoric (book)
	article on Moebius Function

%\HREF{http://www.yahoo.groups.com}{www.yahoo.groups.com}

\item[2015-12-02  Predrag]
Should one study
Klink and Wickramasekara\rf{KliWic15},
{\em Relativity, Symmetry and the Structure of Quantum Theory I} ?
They say: ``The history of how quantum mechanics was developed is a
fascinating one and underlies the focus of this book; namely, given the
rules that the founders of quantum mechanics developed, is it possible to
find principles that lead to the structure of quantum mechanics as it was
historically formulated? This is the first book in a series of works
considering what particular relativity is applicable to a given dynamical
theory. The series considers Newton, Einstein, and de Sitter
relativities, while this book examines the unitary irreducible
representations of the Galilei group and see how they provide the
framework for Galilean quantum theory.
''

\item[2015-12-02  Predrag] Abbas\rf{Abbas16} {\em Group Theory
in Particle, Nuclear, and Hadron Physics} will be available in July 2016.

\item[2017-10-10  Predrag]
Bincer\rf{Bincer12} has a good discussion of SL(2,C) as the covering
group of the proper orthochronous component \SOn{1,3} of the Lorentz group.

The spinor representations $Spin(n)$ were discovered in 1913
by Cartan. What I call Dirac gammas he calls \emph{Clifford} numbers.

Includes biographical sketches of
Galois, Abel, Jacobi, Euler, Lie, Cartan,
Casimir, Weyl, Clebsch, Gordan, Wigner,
Clifford, Schur,
Dynkin, Racah,
Hurwitz, Hamilton, Graves, Cayley, Frobenius,
Lorentz, de Sitter, Liouville, Maxwell, Thomas,
Minkowski, Klein, Gordon, Dirac, Proca,
Poincar\'e, Pauli, Lubanski, Kac, Moody,
Coulomb, Heisenberg, Lenz, and Runge. Does not cite Cvitanovi{\'c}.

\item[2017-10-10  Predrag]
Ramond\rf{Ramond10} does cite Cvitanovi\v{c}.

\item[2017-10-10  Predrag]
A possible addendum to history of birdtracks is in my
{\em Birdtracks - updated history} in\\
\texttt{PHYS-7143-17/notes/weeks/week9.tex},
\HREF{http://birdtracks.eu/courses/PHYS-7143-17/bTrackHistory.pdf}
{online version}.
But looking at figures included in Penrose\rf{Penr04} - I should
really give him all the credit for diagrammatic notation...

\item[2019-05-01 Judith Alcock-Zeilinger] and Stefan Keppeler
are working on birdtracking quarks$~\times~$antiquarks Young tableaux.
Their current construction algorithm of the irreducible multiplets is based on
King\rf{King70} {\em Generalized {Young Tableaux} and the general linear group}
that Predrag is not familiar with it. In this paper, King derives what he calls
back-to-back tableaux from the Littlewood-Richardson rules of tableau
multiplication. King writes:

By making use of the theory of characters, Murnaghan and
Littlewood\rf{Littlewood50}  were able to reduce the inner and outer products of
IR's of the symmetric groups into IR's of the symmetric groups. The importance of
Young's work was particularly stressed by Robinson\rf{Robinson61} who derived
procedural rules for carrying out these reductions. These rules were expressed
solely in terms of Young tableaux and thus obviated the necessity of calculating
characters. The results obtained by these authors are applicable to those IR's of
GL(n) whose bases are either covariant or contravariant tensors.

More generally, mixed tensors may be used to form the bases of irreps of GL(n),
but it has been customary to raise or lower indices appropriately in order to
make use of the duality of GL(n) as expressed in the Young tableaux. Here a
generalization of the Young tableau is defined to study some of the properties of
the irreps of GL(n) induced in a space defined by mixed tensors, without recourse
to raising or lowering of indices.

\item[2019-09-21  Predrag].

Arnold\rf{Arnold19} {\em {Landau-Pomeranchuk-Migdal} effect in sequential
bremsstrahlung: {From large- N {QCD} to N=3} via the {SU}(N) analog of
{Wigner }6-j symbols} writes:
``
The latter will involve the \SUn{N} generalization of Wigner 6-j
coefficients. The original purpose of Wigner 6-j coefficients can be
thought of as describing for angular momentum [symmetry group \SUn{2}]
the relation of different bases for spin singlet states made from four
spins. In our case, we will need the generalization from spin to color.
''

I'm bit puzzled, because ``the generalization'' is in my \emph{Group
Theory}\rf{PCgr},
\HREF{http://birdtracks.eu/version9.0/GroupTheory.pdf\#section.5.1} {\em
Chapter~5} {\em Recouplings}. Moreover, I started out believing that
Wigner 6-$j$ were specific to  boring atomic / nuclear physics angular
momentum calculation. Only after I've written the central parts of my
monograph did I understand that Wigner already knew that $3n-j$'s apply
to all semisimple Lie groups - it's just that the main applications of
that time were to \SUn{2}, but his deep insight was general.

In the introduction I write\rf{PCgr}:
``For me, the greatest surprise of all is that in spite of all the magic
and the strange diagrammatic notation, the resulting manuscript is in
essence not very different from Wigner's 1931 classic. Regardless of
whether one is doing atomic, nuclear, or particle physics, all physical
predictions (``spectroscopic levels'') are expressed in terms of
Wigner's 3n-j coefficients, which can be evaluated by means of recursive or
combinatorial algorithms.

Now, I have not {\em really} looked through Wigner 1931 since 1984. So am
I giving him too much credit? Am I to credit myself with generalizing
Wigner from \SUn{2} $\to$ all semisimple Lie groups? That seems very
unlikely.

Arnold notes that $s$- $u$- $t$-channel graphs discrete symmetry is
$\Dn{2}=\Zn{2}\times\Zn{2}$ and uses that to go to irreps of \Dn{2} in
order to get the $gggg$ singlets, see his Table.~II.
Probably should do that myself, rather
than only looking at the $gg\to{gg}$ channels?

He says 3 singlets (mixed symmetry) decouple, leaving $[5\!\times\!5]$
matrices mixing remaining (symmetric) 5 singlets. For $N\geq4$ that
becomes 6 songlets. His ``harmonic oscillator'' (1.5) has 3 symmetric
$[5\!\times\!5]$ sub-matrices, making calculations laborious. He does not
know how to find a closed-form solution for this ``harmonic oscillator''
propagator. Derivation of the Hamiltonian is a bit detailed, I have not
gone through it.


\end{description}


%\newpage %%%%%%%%%%%%%%%%%%%%%%%%%%%%%%%%%%%%%%%%%%%%%%%%
\printbibliography[heading=subbibintoc,title={References}]
