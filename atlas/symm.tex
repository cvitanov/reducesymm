% siminos/atlas/symm.tex  pdflatex atlas
% $Author$ $Date$

\section{Dynamics and symmetry: dancers and drifters}
\label{s:symm}

What is a symmetry? Fluid dynamics affords an easy illustration,
\reffig{fig:A27-pipeSymms}.

A {\em \reqv} is a dynamical orbit whose velocity field lies within the
group tangent space,
\(
\vel(\ssp) = c \cdot \groupTan(\ssp)
\) %\label{phaseVel}
with a constant {\phaseVel} $c$, and whose time evolution is thus
confined to the group orbit. Depending on the context, \reqva\
are also called traveling waves and/or rotational waves.
A {\em \rpo} $\pS_p$ is a trajectory which exactly recurs
\beq
\ssp(\zeit) = \LieEl_p \, \ssp(\zeit + \period{p} )
    \,,\qquad
\ssp(\zeit) \in \pS_p
\ee{RPOrelper1}
after a fixed {relative period} $\period{p}$, but shifted by a fixed
group action ${\LieEl}$ that maps the endpoint $\ssp (\period{}) $ back
into the initial point cycle point $\ssp (0) $. The pipe flow sketches
\reffig{fig:A27-pipeSymms} and \cLf\ \reffig{fig:CLf01group}\,(b) are two
examples.

% \subsection{Pipes and planes}

 %% A27*-pipeSymms.* - read dasbuch/book/FigSrc/inkscape/00ReadMe.txt
 \begin{figure}
 \begin{center}
  \setlength{\unitlength}{0.20\textwidth}
  %% \unitlength = units used in the Picture Environment
(a)
  \begin{picture}(1,0.52454249)%
    \put(0,0){\includegraphics[width=\unitlength]{A27a-pipeSymms}}%
    \put(0.61583231,0.13683004){\color[rgb]{0,0,0}\makebox(0,0)[lb]{\smash{$z$}}}%
    \put(0.00611823,0.27217453){\color[rgb]{0,0,0}\makebox(0,0)[lb]{\smash{$\theta$}}}%
  \end{picture}%
(b)
  \begin{picture}(1,0.52454249)%
    \put(0,0){\includegraphics[width=\unitlength]{A27b-pipeSymms}}%
    \put(0.61583231,0.13683004){\color[rgb]{0,0,0}\makebox(0,0)[lb]{\smash{$z$}}}%
    \put(0.00611823,0.27217453){\color[rgb]{0,0,0}\makebox(0,0)[lb]{\smash{$\theta$}}}%
  \end{picture}%
\\
(c)
  \begin{picture}(1,0.52454249)%
    \put(0,0){\includegraphics[width=\unitlength]{A27c-pipeSymms}}%
    \put(0.61583231,0.13683004){\color[rgb]{0,0,0}\makebox(0,0)[lb]{\smash{$z$}}}%
    \put(0.00611823,0.27217453){\color[rgb]{0,0,0}\makebox(0,0)[lb]{\smash{$\theta$}}}%
  \end{picture}%
(d)
  \begin{picture}(1,0.52454249)%
    \put(0,0){\includegraphics[width=\unitlength]{A27d-pipeSymms}}%
    \put(0.61583231,0.13683004){\color[rgb]{0,0,0}\makebox(0,0)[lb]{\smash{$z$}}}%
    \put(0.00611823,0.27217453){\color[rgb]{0,0,0}\makebox(0,0)[lb]{\smash{$\theta$}}}%
  \end{picture}%
 \end{center}
 \caption[$\On{2}_\theta \times \SOn{2}_z$ symmetry of flow in a stream-wise
          periodic pipe]{\label{fig:A27-pipeSymms}
Any fluid state in a stream-wise periodic pipe translated or rotated is
also a solution. In particular, a \rpo\ $p$ is a state of the fluid
 (a)
that reappears
 (b)
period $\period{}$ later, translated by downstream shift $\shift$
(in contrast to a \reqv, a constant shape that travels
downstream with constant {\phaseVel} $\velRel$); such solutions are
stream-wise $\SOn{2}_z$ equivariant; or
 (c)
period $\period{}$ later, translated by downstream shift $\shift$ and
rotated azimuthaly by $\gSpace_p$; $\SOn{2}_{\theta} \times \SOn{2}_z$
equivariant; or
 (d)
period $\period{}$ later, reflected and rotated azimuthaly by
$\gSpace$; $\On{2}_{\theta}$ equivariant
(from \wwwcb{}).
 }%
 \end{figure}
													\toCB
    \PC{emphasize that our turbulent states are \emph{not} localized
    three-dimensional solitary waves as quasi-particles -
    \reffig{fig:A27-pipeSymms} might be misleading. Our solutions are
    global, distributed over the whole volume.}
The symmetry group $\Gpipe$ of stream-wise periodic pipe flow contains
two commuting \SOn{2} rotations (\reffig{fig:A27-pipeSymms}). Each
\SOn{2} group orbit is topologically a circle, but here a very
complicated \statesp\ curve composed of many Fourier modes nonlinearly
coupled by turbulent dynamics and thus of comparable magnitudes. Together
the two \SOn{2} sweep out a very contorted and hard to visualize $T^2$
torus that is shown in \refref{ACHKW11}, but we shall illustrate the key
ideas by a much simpler example, the $\SOn{2}$-equivariant Gibbon and
McGuinness\rf{GibMcCLE82,FowlerCLE82} \cLe\ of geophysics and laser
physics,
\bea
	\dot{x}_1 &=& -\sigma x_1 + \sigma y_1\continue
	\dot{x}_2 &=& -\sigma x_2 + \sigma y_2\continue
	\dot{y}_1 &=& (\RerCLor-z) x_1 - \ImrCLor x_2 -y_1-e y_2 \continue
	\dot{y}_2 &=& \ImrCLor x_1 + (\RerCLor-z) x_2 + e y_1- y_2\continue
	\dot{z} \; &=& -b z + x_1 y_1 + x_2 y_2
    \,.
\label{eq:CLeR}
\eea
Here the parameters are set to Siminos values $\RerCLor=28,\,
\ImrCLor=0,\, b=8/3,\, \sigma=10,\, e= 1/10$. (For the background and an
in-depth investigation of the model  see \refrefs{SiminosThesis}.)

\begin{figure}
  	\begin{center}
  	\setlength{\unitlength}{0.20\textwidth}
  (a)
  	\begin{picture}(1,1.07802818)%
    	\put(0,0){\includegraphics[width=\unitlength]{CLWattractor}}%
    	\put(0.55152995,1.0139628){\color[rgb]{0,0,0}\makebox(0,0)[lb]{\smash{$z$}}}%
    	\put(0.05573445,0.0739776){\color[rgb]{0,0,0}\makebox(0,0)[lb]{\smash{$x_1$}}}%
    	\put(0.90013492,0.16491708){\color[rgb]{0,0,0}\makebox(0,0)[lb]{\smash{$x_2$}}}%
  	\end{picture}%	
  (b)
  	\begin{picture}(1,1.06440474)%
    	\put(0,0){\includegraphics[width=\unitlength]{CLEWurst01}}%
   		\put(0.55961552,1.00214901){\color[rgb]{0,0,0}\makebox(0,0)[lb]{\smash{$z$}}}%
   		\put(0.07008555,0.07304272){\color[rgb]{0,0,0}\makebox(0,0)[lb]{\smash{$x_1$}}}%
    	\put(0.90381504,0.16283301){\color[rgb]{0,0,0}\makebox(0,0)[lb]{\smash{$x_2$}}}%
  	\end{picture}	
    \end{center}
  \caption
  [\CLf: $\cycle{01}$ {\rpo} group orbit]{
  (a)
  The strange attractor of \cLf.
  (b)
  \CLf: The initial \reqv\ $\REQV{}{1}$ point is shown by the red dot,
  and its group orbit / trajectory by the dashed red line. One period of
  the $\cycle{01}$ {\rpo} is shown by the solid blue line. The group
  orbit of its (arbitrary) starting point is shown by the dashed blue
  line: after one period the trajectory has returned to the group orbit
  but with a different phase. The group orbit of the $\cycle{01}$
  trajectory (dark blue) is shown by the cyan surface. Trajectory of the
  further 15 repeats of $\cycle{01}$ (faint dotted lines) traces out
  ergodically the torus generated by the$\cycle{01}$ group; in that time
  the slowly drifting \reqv\ $\REQV{}{1}$ has advanced to the next red
  dot (red line).
  }
\label{fig:CLf01group}
\end{figure}

The strange attractor of the \cLf\ is shown in \reffig{fig:CLf01group}\,(a). It's a mess. Why?
The problem of dynamics in presence of symmetry can perhaps understood this way:
dissipative flows are lazy, and as drifting is energetically cheap, rather than do work,
solutions tend to drift along non-shape-changing symmetry directions which burn no energy.

%%%%%%%%%%%%%%%%%%%%%%%%%%%%%%%%%%%%%%%%%%%%%%%%%
% 2011-09-09, 2012-03-30 Predrag: add BeThMovFr to
%            continuous.tex overheads, and ChaosBook
% replace A27movFrame*.* everywhere
\begin{figure}
 \begin{center}
  \setlength{\unitlength}{0.20\textwidth}
  %% \unitlength = units used in the Picture Environment
(a)~~
  \begin{picture}(1,0.98655417)%
    \put(0,0){\includegraphics[width=\unitlength]{BeThTrajTeX}}%
    \put(0.35976094,0.91875614){\color[rgb]{0,0,0}\rotatebox{-31.32889204}{\makebox(0,0)[lb]{\smash{$\pS_{\ssp(\zeit)}$}}}}%
        \put(0.60333631,0.42274457){\color[rgb]{0,0,0}\rotatebox{-40.8073288}{\makebox(0,0)[lb]{\smash{$\pS_{\ssp(0)}$}}}}%
    \put(0.66001383,0.16959019){\color[rgb]{0,0,0}\rotatebox{0.0313674}{\makebox(0,0)[lb]{\smash{$\ssp(0)$}}}}%
    \put(0.5058276,0.64524238){\color[rgb]{0,0,0}\rotatebox{0.0313674}{\makebox(0,0)[lb]{\smash{$\ssp(\zeit)$}}}}%
    \put(0.13110825,0.05766516){\color[rgb]{0,0,0}\rotatebox{0.11031334}{\makebox(0,0)[lb]{\smash{$\pS$}}}}%
  \end{picture}%
~~(b)
  \begin{picture}(1,0.98655417)%
    \put(0,0){\includegraphics[width=\unitlength]{BeThMovFr}}%
    \put(0.20559239,0.64023845){\color[rgb]{0,0,0}\rotatebox{0.0313674}{\makebox(0,0)[lb]{\smash{$\ssp(\zeit)$}}}}%
    \put(0.67382401,0.35781161){\color[rgb]{0,0,0}\rotatebox{0.0313674}{\makebox(0,0)[lb]{\smash{$\ssp(0)$}}}}%
    \put(0.61221026,0.74589514){\color[rgb]{0,0,0}\rotatebox{0.0313674}{\makebox(0,0)[lb]{\smash{$\sspRed(\zeit)$}}}}%
    \put(0.35760559,0.8662057){\color[rgb]{0,0,0}\rotatebox{0.0313674}{\makebox(0,0)[lb]{\smash{$\LieEl(\zeit)$}}}}%
  \end{picture}%
 \end{center}
  \caption{\label{fig:BeThMovFr}
(a)
The group orbit $\pS_{\ssp(0)}$ of \statesp\ point $\ssp(0)$, and the
group orbit $\pS_{\ssp(\zeit)}$ reached by the trajectory $\ssp(\zeit)$ time $t$
later.
(b)
The two physically equivalent flows $\ssp(\zeit)=\LieEl(\zeit)\,\sspRed(\zeit)$ are related
by, in general, an arbitrary, time dependent {\em moving frame} transformation $\LieEl(\zeit)$.
(from \wwwcb{}).
  }
\end{figure}
%%%%%%%%%%%%%%%%%%%%%%%%%%%%%%%%%%%%%%%%%%%%%%%%%%

As the $SOn{2}$ transformations act on the \cLf\ only through the
simplest, $m=1$ Fourier mode, all group orbits are circles, which appear
elliptical in most $d=5 \to 3$~dimensions projections. Nevertheless, even
the wurst traced out by the very simplest, short \rpo\ $\cycle{01}$ shown
in \reffig{fig:CLf01group}\,(b) in is not so easy to gets one's head around:
you are looking at a 3\dmn\ projection of a \emph{torus} embedded in 5
dimensions.
