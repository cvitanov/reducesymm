% siminos/atlas/concl.tex  pdflatex atlas
% $Author$ $Date$

\section{Conclusion and perspectives}
\label{s:concl}
% former siminos/atlas/concl.tex


As a turbulent flow evolves, every so often we catch a glimpse of a
familiar structure. For any finite spatial resolution, the flow
follows for a finite time a coherent structure belonging to an
alphabet of admissible fluid states, represented here by a set of \reqv\
and \rpo\ solutions of \NS. These are not the `modes' of the fluid; {they
do not provide a decomposition of the flow into a sum of components at
different wavelengths, or a basis for low-dimensional
modeling.} Each such solution spans the whole range of physical scales of
the turbulent fluid, from the outer wall-to-wall scale, down to the
viscous dissipation scale. Numerical computations require sufficient
resolution to cover all of these scales, so no {global} dimension
reduction is likely. The role of invariant solutions of \NS\ is, instead,
to partition the $\infty$-dimensional \statesp\ into a finite set of
neighborhoods visited by a typical long-time turbulent fluid state.

Motivated by the recent observations of \recurrStr s in experimental and
numerical turbulent flows, we initiated here an exploration of the
hierarchy of \reqva\ and \rpo s of fully-resolved transitionally
turbulent pipe flow in order to describe its spatio-temporally chaotic
dynamics. The $3D$ fluid states captured by the short pipe invariant
solutions and their unstable manifolds are strikingly similar to states
observed both in experiments and in numerical simulations of longer pipes
\rf{science04}, while the turbulent dynamics visualized in \statesp\
appears pieced together from close visitations to \cohStr s connected by
transient interludes.

For pipe flow \reqva\ and \rpo s embody a vision of turbulence as a
repertoire of recurrent spatio-temporal coherent structures explored by
turbulent dynamics. Given a set of invariant solutions, the next
step is to understand how the dynamics interconnects the neighborhoods of
the invariant solutions discovered so far.
        %
Currently, a taxonomy of these myriad states eludes us, but emboldened by
successes in applying periodic orbit theory to the simpler \KS\ problem
\rf{Christiansen97,lanCvit07,SCD07}, we are optimistic.

In a co-moving frame, moving with the mean phase velocity of a given
solution, a \rpo\ reduces to a \po.
However, as each solution travels with its own mean downstream velocity,
there is no single co-moving frame that can simultaneously reduce
\emph{all} traveling solutions.
    \PC{disparage co-moving frames: they are misleading for \rpo s, as they
    misrepresent time segments for which a \po\ might be glued to
    barely unstable \eqv. A co-moving frame rotates with the constant
    mean
    $\timeAver{\velRel}= \shift_p/\period{p}
    \,.
    $ We \emph{emphatically} do not work in
    co-moving frames.}
    \PC{unhappy about ``moving frames'': misleading, as slice is
     \emph{emphatically} stationary. ''Covariant frames'' move, but we
     do not know how to use them
     }


The reader might rightfully wonder what the short pipe periodic cells
studied here and in \pCf\ have to do with physical,
wall-bounded shear flows in general, with large aspect ratios and
physical boundary conditions. Indeed, the outstanding issue
that must be addressed in future work is the small-aspect cell
periodicities imposed for computational efficiency. In case of the pipe
flow, most computations of invariant solutions have focused on
stream-wise periodic cells barely long enough to allow for sustained
turbulence. Such small cells introduce dynamical artifacts such as lack
of structural stability and cell-size dependence of the sustained
turbulence states.

The main message of this paper is that if a problem has a continuous
symmetry, the symmetry \emph{must} be used to simplify it. Ignore it at
your own peril, as has been done so far in \KS\rf{Christiansen97}
and \pCf\rf{GHCW07}; the invariant solutions found by restricting
searches to the discrete-symmetry invariant subspaces have little if
anything to do with the full \statesp\ explored by turbulence, not more
than the \eqv\ points of the Lorenz equation have to do with its strange
attractor. Symmetry reduction by \mslices\ is numerically efficient.
Coupled with our \statesp\ visualizations allows for explorations of
high-dimensional flows that hitherto were unthinkable. Symmetry reduction
is here achieved, and now all pipe flow solutions can be plotted
together, as one happy family: all points equivalent by symmetries are
represented by a single point, families of solutions are mapped to a
single solution, \reqva\ become \eqva, and \rpo s become \po s. Without
symmetry reduction, no full understanding of pipe and plane \pCf s is
possible.
    %
    \PC{2011-10-18 incorporate this paragraph:``
Note also that the rotation of a fluid flow into a \slice\ {\em is not}
an average over the 3D pipe azimuthal angle, it is the full snapshot of
the flow embedded in the $\infty$-dimensional \statesp. Symmetry
reduction is not a dimensional-reduction scheme, or flow modeling by
fewer degrees of freedom: the \reducedsp\ is also $\infty$-dimensional
and no information is lost, one can go freely between solutions in the
full and reduced \statesp s by integrating the associated
\emph{reconstruction equations}.
''}

\begin{acknowledgments}
We are indebted to
M.~Avila,
D.~Barkley,
R.~L.~Davidchack,
S.~Froehlich,
E.~Siminos
L.~S.~Tuckerman,
and
A.~P.~Willis
% D.~Viswanath
for inspiring discussions.
P.~C.\ thanks G.~Robinson,~Jr.\ for support, and
Max-Planck-Institut f\"ur Dynamik und Selbstorganisation,
G\"ottingen for hospitality.
P.~C.\ was partly supported by NSF grant DMS-0807574
and
2009 Forschungspreis der Alexander von Humboldt-Stiftung.
\end{acknowledgments}
