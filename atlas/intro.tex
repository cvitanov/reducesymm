% siminos/atlas/intro.tex  pdflatex atlas
% $Author$ $Date$

\begin{quotation}
    \ifdraft\color{blue}
The ``lead paragraph'' is formatted as a single paragraph before the first
section heading. Numbered references are allowed.
        \PC{
    The first paragraph of the article should be a Lead Paragraph and
    will be highlighted in the journal in boldface type. This paragraph,
    which essentially advertises the main points of the article, must
    describe in terms accessible to the nonspecialist reader the context
    and significance of the research problem studied and the importance
    of the results. The Editors will pay special attention to the clarity
    and accessibility of this paragraph, and in many cases may rewrite it
        }
    \color{black}\fi
\end{quotation}

    \PC{2012-01-03 experiment with
    \ensuremath{\hat{\ssp}}, \ensuremath{\bar{\ssp}} or \ensuremath{\tilde{\ssp}}
    as the \reducedsp\ coordinate.
    }

\section{Introduction}
\label{s:intro}
% former siminos/atlas/intro.tex

    \ifdraft\color{blue}
    \PC{
{2012-03-12} A putative outline of the paper is in
\refsect{chap:outline}.
    }
Goal: chart the \statesp\ explored by chaotic dynamics,
a curved manifold embedded in a high-dimensional \statesp.

Key notion: recurrence.
To quantify `near' need the notion of distance, or 'norm'.

Problem: evolution in time decomposes \statesp\ into spaghetti of time
orbits or trajectories. Symmetries stratify it into layers of an onion.
Need to pick a single point for each trajectory (section it) and each group orbit
(slice it).

(template)

Cover the curved manifold by the shortest-distance sections (for time
recurrence) and \slice s (for continuous transformations). In the limit of longer
and longer cycles this leads to the usual curved manifold geometry,
measured locally by Euclidean distances.
    \color{black}\fi


Over the last decade, new insights into the dynamics of moderate
\Reynolds\ turbulent flows have been gained through visualizations of
their $\infty$-dimensional \statesp s by means of dynamically invariant,
representation independent coordinate frames\rf{GHCW07} constructed from
physically prominent unstable coherent states, hereafter referred to
{\em \template s'}. The most recent advance made possible by these new
approaches is the determination of \rpo s that in part shape turbulence
observed in pipe flows\rf{ACHKW11}. Navigating and charting the geometry
of these extremely high-dimensional \statesp s necessitates a
reexamination of two of the basic tools of the theory of dynamical
systems, Poincar\'e sections and symmetry reduction by the
{\mslices}\rf{rowley_reconstruction_2000,BeTh04,SiCvi10,FrCv11}. We
explain here the key geometrical ides in simple but illustrative
settings, eschewing the technicalities of fluid dynamics.

We begin by defining the dynamical concepts that we shall need here.

%%%%%%%%%%%%%%%%%%%%%%%%%%%%%%%%%%%%%%%%%%%%%%%%%%%%%%%%%%%%%%%%%%%%%
\begin{figure}
   \centering
  \setlength{\unitlength}{0.20\textwidth}
(a)~~~
  \begin{picture}(1,0.98239821)%
    \put(0,0){\includegraphics[width=\unitlength]{A28tangent3}}%
    \put(0.91612064,0.70682767){\color[rgb]{0,0,0}\makebox(0,0)[lb]{\smash{$\vel$}}}%
    \put(0.48698745,0.90266503){\color[rgb]{0,0,0}\makebox(0,0)[lb]{\smash{$\ssp(\zeit)$}}}%
    \put(0.2624318,0.5347756){\color[rgb]{0,0,0}\makebox(0,0)[lb]{\smash{$\groupTan_1$}}}%
    \put(0.80471037,0.38188675){\color[rgb]{0,0,0}\makebox(0,0)[lb]{\smash{$\groupTan_2$}}}%
    \put(0.538343,0.25344355){\color[rgb]{0,0,0}\makebox(0,0)[lb]{\smash{$\LieEl\ssp$}}}%
    \put(0.47864531,0.56060893){\color[rgb]{0,0,0}\makebox(0,0)[lb]{\smash{$\ssp$}}}%
  \end{picture}%
~~(b)\includegraphics[width=0.20\textwidth]{A27traj}
\\
(c)\includegraphics[width=0.20\textwidth]{A27gOrbit}
(d)\includegraphics[width=0.20\textwidth]{A27wurst}
   \caption{\label{fig:A27wurst}
   (a)
In presence of $N$-continuous parameter symmetry, each \statesp\ point
$\ssp$ owns $(N\!+\!1)$ tangent vectors: one $\vel(\ssp)$ along the time
flow $\ssp(\zeit)$, and the $N$ group tangents  $\groupTan_1(\ssp), \,
\groupTan_2(\ssp) ,\,\cdots, \groupTan_N(\ssp)$ along infinitesimal
symmetry shifts, tangent to the group orbit $\LieEl\ssp$.
    (b)
Trajectory.
    (c)
Group orbit.
    (d)
Wurst.
}
\end{figure}
%%%%%%%%%%%%%%%%%%%%%%%%%%%%%%%%%%%%%%%%%%%%%%%%%%%%%%%%%%%%%%%%%%%%%

\begin{itemize}

  \item
A {\em dynamical system} $\{\pS,\map^t\}$ is defined globally by
specifying the flow $\map^t$ and the \statesp\ $\pS$ on which the flow
acts.
  \item
After some experimentation with and observations of a given turbulent
flow, one can identify a set of dynamically important unstable
{\recurrStr s}.
We shall refer to this catalogue of $M$ representative snapshots or
`reference states', either precomputed or experimentally measured, as
\emph{\template s}\rf{rowley_reconstruction_2000}, each an
instantaneous state of the $3D$ fluid flow represented by a \emph{point}
$\slicep{}^{(j)}$, $j=1,2,\cdots,M$, in the \statesp\ $\pS$ of the
system.
  \item
If a group $\Group$ of continuous parameter transformations acts on a
continuous time flow, each \statesp\ point owns a set of tangent vectors,
\reffig{fig:A27wurst}\,(a). Integrated globally, the velocity vector
$\vel(\ssp)$ traces out the time orbit, \reffig{fig:A27wurst}\,(b),
hereafter referred to as the {\em trajectory} $\flow{\zeit}{\ssp}$, and the
group tangents trace out the group orbit, \reffig{fig:A27wurst}\,(c),
hereafter often referred to as the {\em orbit}  $\pS_\ssp =
\{\LieEl\,\ssp \mid \LieEl \in {\Group}\}$. Together the two trace out a
complicated smooth manifold, \reffig{fig:A27wurst}\,(d), hereafter
affectionately referred to as the {\em wurst}.
  \item
A flow $\dot{\ssp}= \vel(\ssp)$ is
said to have symmetry $\Group$ or be $\Group$-\emph{equivariant}
if the form of evolution equations \refeq{symbolicNS} is left invariant
by the set of transformations $\LieEl$,
\beq
\vel(\ssp)=\LieEl^{-1} \, \vel(\LieEl \, \ssp)
\,,\qquad \mbox{for all } \LieEl \in {\Group}
\,.
\ee{eq:FiniteRot}
While the flow equations are invariant under $\Group$, the state of a turbulent flow
typically is not.
  \item section {\PoincS} vs slice \pSRed

  \item
dynamical system  with symmetry \Group\ vs reduced dynamics
$\{\pSRed,\mapRed^t\}$ , hereafter affectionately referred to as the {\em
sliced wurst}.
  \item strobing $\sim$ method of connections
  \item reduction vs projection
\end{itemize}

Our goals here are two-fold:
(i) In \refsect{s:cut} we review the method of Poincar\'e sections, but with
    emphasis on aspects applicable to high-dimensional flows: construction of
    multiple local linear charts and determination of their borders, and then in
(ii) \refsect{s:slice} show how the same set of tools applied to
    reduction of continuous symmetries enables us to commence a
    systematic charting of the long-time dynamics of high-dimensional
    flows with continuous symmetries.
