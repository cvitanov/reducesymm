% siminos/atlas/intro.tex  pdflatex atlas
% $Author$ $Date$

\begin{quotation}
    \PublicPrivate{}{
The ``lead paragraph'' is formatted as a single paragraph before the first
section heading. Numbered references are allowed.
        \PC{
    The first paragraph of the article should be a Lead Paragraph and
    will be highlighted in the journal in boldface type. This paragraph,
    which essentially advertises the main points of the article, must
    describe in terms accessible to the nonspecialist reader the context
    and significance of the research problem studied and the importance
    of the results. The Editors will pay special attention to the clarity
    and accessibility of this paragraph, and in many cases may rewrite it
        }
    }
\end{quotation}

    \PC{2012-01-03 experiment with
    \ensuremath{\hat{\ssp}}, \ensuremath{\bar{\ssp}} or \ensuremath{\tilde{\ssp}}
    as the \reducedsp\ coordinate.
    }

\section{Introduction}
\label{s:intro}
% former siminos/atlas/intro.tex

    \PublicPrivate{}{
{\bf 2012-03-12 Predrag} A putative outline of the paper is in
\refsect{chap:atlas}, search for {\bf [2012-03-12 Predrag]}.
    }


The understanding of chaotic dynamics in higher-dimensional systems that
has emerged in the last decade offers a promising dynamical framework to
study turbulence. Here, turbulence is viewed as a walk through a forest
of exact solutions of the governing equations, each solution shaping the
local \statesp\ dynamics.

In this approach, dynamics of moderate \Reynolds\ turbulent flows is
visualized in the $\infty$-dimensional \stateDsp\  using \eqv\ solutions
of the \NSe\ to define dynamically invariant, intrinsic, and
representation independent coordinate frames.
These results inform a new way of thinking about the role {\recurrStr s}
play in shaping turbulence:
The observed {\cohStr s} are the physical images of the flow's
least unstable invariant solutions, with
turbulent dynamics arising from a sequence of transitions between
these states, and
the intrinsic low-dimensionality of turbulence resulting from the low
number of unstable eigendirections for each state.
The unstable \po s are of particular
importance, as they provide the skeleton underpinning the
turbulent dynamics\rf{DasBuch}: the geometry of the \statesp\ flow
near onset of turbulence is shaped by the chaotic saddle, a set of
unstable solutions and their heteroclinic connections.

Importance of a given invariant solution is made precise by periodic
orbit theory which assigns a deterministic weight with which the solution
contributes to any dynamical average over chaotic component of the flow
\rf{DasBuch}. Consideration of continuous symmetries extends this
theory to sums over \emph{relative} periodic orbits\rf{Cvi07},
time-dependent solutions which recur periodically in co-moving frames
translating and/or rotating along the pipe axis with given stream-wise
and azimuthal velocities, different velocities for each solution.

%%%%%%%%%%%%%%%%%%%%%%%%%%%%%%%%%%%%%%%%%%%%%%%%%
% 2011-10-23 Predrag: replace this Ashley' simulation
%            continuous.tex overheads, and ChaosBook
% TEMPORARY: from siminos/rpo_ks/arxiv-v2/figs, \refref{cont:SCD07})
%
\begin{figure}
\centering
(a)%\includegraphics[width=0.45\textwidth,clip=true]{2841GO3a}
~~(b)%\includegraphics[width=0.45\textwidth,clip=true]{2841GO3b}
  \caption{\label{f:MeanVelocityFrame}
Symmetry reduction $\pS \to \pSRed$ replaces each
(a)
full \statesp\ trajectory $\ssp(\zeit)$ by
(b)
a simpler \reducedsp\ trajectory $\sspRed(\zeit)$, with continuous group
induced drifts quotiented out. Here this is illustrated by the \rpo\
$\RPO{36.92}$ (see \reffig{fig:M1Orb})
(a) %(red)
traced in the full {\statesp} for two $\period{}=36.92$ periods, in the
frame moving with the constant mean axial flow speed $U$ defined in
\refeq{NavStokesDev};
(b) %(blue)
restricted to the symmetry-\reducedsp. Both are projected onto the
$3$\dmn\ frame \refeq{FrenetFrame1}. In the full \statesp\ a \rpo\ traces
out quasi-periodically a highly contorted 2-torus; in the \reducedsp\ it
closes a \po\ in one period $\period{}$.
            }
\end{figure}
%%%%%%%%%%%%%%%%%%%%%%%%%%%%%%%%%%%%%%%%%%%%%%%%%%

[...]
This problem is here resolved by the
{\mslices}\rf{rowley_reconstruction_2000,BeTh04,SiCvi10,FrCv11}, in
which the group orbit of any full-flow structure is represented by a
single point (see \reffig{fig:BeThTraj}), the group orbit's intersection
with a fixed hypersurface, or the \emph{`\slice'}. A
\slice\ fixes only the symmetry group phases: a continuous time full space
orbit is reduced to a continuous time orbit in the symmetry-\reducedsp,
as in \reffig{f:MeanVelocityFrame}.

Our goals here are two-fold.
(i) First we review the method of Poincar\'e sections,
    in order to motivate the need for the need for symmetry reduction. We
    then explain what symmetry reduction is, and how with it the geometry
    of \statesp\ dynamics is revealed;
(2) we demonstrate next that this tool enables us to commence a systematic
exploration of the hierarchy of dynamically important invariant solutions
of flows with continuous symmetries. The $\infty$-dimensional \stateDsp\
representation\rf{GHCW07} of PDEs, such as \reffig{f:MeanVelocityFrame},
enables us to track the unstable manifolds of invariant
solutions, the heteroclinic connections between them\rf{GHCV08}, and
{provides us with} new insights into the nonlinear \statesp\ geometry and
dynamics of moderate \Reynolds\ wall-bounded flows.

We review  ??? flows, their visualization, and their symmetries in
\refsect{s:review}. The {\mslices}  is described in \refsect{s:slice},
and the computation of invariant solutions and their stability
eigenvalues and eigenvectors in \refsects {sect:TimeOrb}{s:algorithm}.
The main advances reported in this paper are the symmetry \reducedsp\
visualization and [...], (\refsect{s:rpos}). Outstanding challenges are
discussed in \refsect{s:concl}.
