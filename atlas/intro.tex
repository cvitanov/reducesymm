% siminos/atlas/intro.tex  pdflatex atlas
% $Author$ $Date$

\begin{quotation}
    \ifdraft\color{blue}
The ``lead paragraph'' is formatted as a single paragraph before the first
section heading. Numbered references are allowed.
        \PC{
    The first paragraph of the article should be a Lead Paragraph and
    will be highlighted in the journal in boldface type. This paragraph,
    which essentially advertises the main points of the article, must
    describe in terms accessible to the nonspecialist reader the context
    and significance of the research problem studied and the importance
    of the results. The Editors will pay special attention to the clarity
    and accessibility of this paragraph, and in many cases may rewrite it
        }
    \color{black}\fi
\end{quotation}

    \PC{2012-01-03 experiment with
    \ensuremath{\hat{\ssp}}, \ensuremath{\bar{\ssp}} or \ensuremath{\tilde{\ssp}}
    as the \reducedsp\ coordinate.
    }

\section{Introduction}
\label{s:intro}
% former siminos/atlas/intro.tex

    \ifdraft\color{blue}
    \PC{
a more winged title? Current one sounds like several previous ones...
``Revealing the geometry of chaotic flows by slicing'' (?)
    }
    \PC{
{2012-03-12} A putative outline of the paper is in
\refsect{chap:outline}.
    }
Goal: chart the \statesp\ explored by chaotic dynamics,
a curved manifold embedded in a high-dimensional \statesp.

Problem: evolution in time decomposes \statesp\ into spaghetti of time
orbits or trajectories. Symmetries stratify it into layers of an onion.
Need to pick a single point for each trajectory (section it) and each group orbit
(slice it).

(template)

Cover the curved manifold by the shortest-distance sections (for time
recurrence) and \slice s (for continuous transformations). In the limit of longer
and longer cycles this leads to the usual curved manifold geometry,
measured locally by Euclidean distances.
    \color{black}\fi


Over the last decade, new insights into the dynamics of moderate
\Reynolds\ turbulent flows have been gained through visualizations of
their $\infty$-dimensional \statesp s by means of dynamically invariant,
representation independent coordinate frames\rf{GHCW07} constructed from
physically prominent unstable {\cohStr s}, hereafter referred to {\em
\template s}.
    \DB{2012-04-10}{
    Since we are talking about coherent structures in the context
    of turbulence, should we distinguish `exact' coherent structures from
    Lagrangian coherent structures, POD modes, etc?
    }
The most recent advance within this bold new framework is
the first determination of \rpo s that in part shape turbulence observed
in pipe flows\rf{ACHKW11}. Navigating and charting the geometry of these
extremely high-dimensional \statesp s necessitates a reexamination of two
of the basic tools of the theory of dynamical systems: Poincar\'e
sections and symmetry
reduction\rf{rowley_reconstruction_2000,BeTh04,SiCvi10,FrCv11}. We strive
here to explain the key geometrical ideas in simple but illustrative
settings, eschewing the fluid dynamical and group theoretical
technicalities.
%    \DB{2012-04-10}{ I thought
%    the method of slices was not in everybody's bag of tricks. We also
%    discuss some further methods, at least as per the outline.
%    Predrag: agreed}


%%%%%%%%%%%%%%%%%%%%%%%%%%%%%%%%%%%%%%%%%%%%%%%%%%%%%%%%%%%%%%%%%%%%%
\begin{figure}
   \centering
  \setlength{\unitlength}{0.20\textwidth}
(a)~~~
  \begin{picture}(1,0.98239821)%
    \put(0,0){\includegraphics[width=\unitlength]{A28tangent3}}%
    \put(0.91612064,0.70682767){\color[rgb]{0,0,0}\makebox(0,0)[lb]{\smash{$\vel$}}}%
    \put(0.48698745,0.90266503){\color[rgb]{0,0,0}\makebox(0,0)[lb]{\smash{$\ssp(\zeit)$}}}%
    \put(0.2624318,0.5347756){\color[rgb]{0,0,0}\makebox(0,0)[lb]{\smash{$\groupTan_1$}}}%
    \put(0.80471037,0.38188675){\color[rgb]{0,0,0}\makebox(0,0)[lb]{\smash{$\groupTan_2$}}}%
    \put(0.538343,0.25344355){\color[rgb]{0,0,0}\makebox(0,0)[lb]{\smash{$\LieEl\ssp$}}}%
    \put(0.47864531,0.56060893){\color[rgb]{0,0,0}\makebox(0,0)[lb]{\smash{$\ssp$}}}%
  \end{picture}%
~~(b)\includegraphics[width=0.20\textwidth]{A27traj}
\\
(c)\includegraphics[width=0.20\textwidth]{A27gOrbit}
(d)\includegraphics[width=0.20\textwidth]{A27wurst}
   \caption{\label{fig:A27wurst}
   (a)
In presence of $N$-continuous parameter symmetry, each \statesp\ point
$\ssp$ owns $(N\!+\!1)$ tangent vectors: one $\vel(\ssp)$ along the time
flow $\ssp(\zeit)$, and the $N$ group tangents  $\groupTan_1(\ssp), \,
\groupTan_2(\ssp) ,\,\cdots, \groupTan_N(\ssp)$ along infinitesimal
symmetry shifts, tangent to the group orbit $\LieEl\ssp$.
    (b)
Trajectory.
    (c)
Group orbit.
    (d)
Wurst.
}
\end{figure}
%%%%%%%%%%%%%%%%%%%%%%%%%%%%%%%%%%%%%%%%%%%%%%%%%%%%%%%%%%%%%%%%%%%%%

A flow $\map^t$ and the \statesp\ $\pS$ on which the flow acts comprise a
{dynamical system}. If a group $\Group$ of continuous transformations
acts on a continuous time flow, each \statesp\ point has a set of
tangent vectors (\reffig{fig:A27wurst}\,(a)). Integrated globally, the
velocity vector $\vel(\ssp)$ traces out the {\em trajectory}
$\flow{\zeit}{\ssp}$ ( \reffig{fig:A27wurst}\,(b)). Applying the continuous
transformations traces out the {group orbit} (or, from now on, just
\emph{orbit})
\(
\pS_\ssp = \{\LieEl\,\ssp \mid \LieEl \in {\Group}\}
% \,,\qquad \pS_\ssp \subset \pS
\,
\) %ee{sspOrbit}
(\reffig{fig:A27wurst}\,(c)). Together they trace out a complicated smooth
manifold (hereafter affectionately referred to as the {\em wurst}, see
figures~\ref{fig:A27wurst}\,(d), \ref{fig:CLf01group}\,(b) and
\ref{fig:sliceimage}), that we shall teach you here how to slice.

A flow is said to have symmetry $\Group$ if the form of evolution
equations $\dot{\ssp} = \vel(\ssp)$ is left invariant, i.e.,
\(
\vel(\ssp)=\LieEl^{-1} \, \vel(\LieEl \, \ssp)
% \,,\qquad \mbox{for all }
\,,
\) %ee{eq:FiniteRot}
by the set of transformations $\LieEl \in {\Group}$. Physicists love
symmetry, but Nature does not care: turbulence breaks all symmetries,
and while the flow equations may be invariant under $\Group$, their
solutions typically are not.

The key to chaotic dynamics is the notion of
recurrence. To quantify how close the state of the system now is to a previously visited
state, we need the notion of distance between two points in
\statesp. The simplest (but far from the only, or the most natural) is
the Euclidean norm
\beq
  \Norm{\ssp-\ssp'}^2  = \braket{\ssp-\ssp'}{\ssp-\ssp'} =
\sum_j^d
(\ssp-\ssp')_j^2
\,.
\ee{innerproduct}
Given distances and neighborhoods \DB{\color{blue}Neighborhoods not defined up to this point in the text. Should they be?}, the next key notion is  \emph{measure}, or
how likely a typical trajectory is to visit a given neighborhood.
After some observations of a given turbulent flow, one can identify a set
of representative \emph{\template s}\rf{rowley_reconstruction_2000}, {points}
$\slicep{}^{(j)}$, $j=1,2,\cdots$ in the \statesp\ $\pS$, which are the
dynamically most important unstable {\recurrStr s} of the flow.

Our goals here are two-fold:
(i) In \refsect{s:cut} we review the method of Poincar\'e sections, with
    emphasis on aspects applicable to high-dimensional flows: construction of
    multiple local linear `charts' and determination of their borders and 
(ii) in \refsect{s:slice} we show how the same set of tools applied to
    reduction of continuous symmetries enables us to commence a
    systematic charting of the long-time dynamics of high-dimensional
    flows with continuous symmetries (\refsect{s:chart}).

    \ifdraft\color{blue}
still to discuss:
\begin{itemize}

  \item section {\PoincS} vs slice \pSRed

  \item
dynamical system $\{\pS,\map^t\}$ with symmetry \Group\ vs reduced dynamics
$\{\pSRed,\mapRed^t\}$
  \item strobing $\sim$ method of connections
  \item reduction vs projection
\end{itemize}
    \color{black}\fi
