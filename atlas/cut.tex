% siminos/atlas/cut.tex  pdflatex atlas
% $Author$ $Date$

\section{\Statesp\ visualization}

\subsection{Group orbit}
    \PublicPrivate{}{
%%%%%%%%%%%%%%%%%%%%%%%%%%%%%%%%%%%%%%%%%%%%%%%%%%%%%%%%%%%%%%%%%%%%%
\begin{figure}
   \centering
   %\includegraphics[width=0.45\textwidth]{???}
   \caption{\label{fig:tangents}
3 tangents: one $\vel(\ssp)$  and two group tangents
$\groupTan(\ssp)^{(1)}$, $\groupTan(\ssp)^{(2)}$.
}
\end{figure}
%%%%%%%%%%%%%%%%%%%%%%%%%%%%%%%%%%%%%%%%%%%%%%%%%%%%%%%%%%%%%%%%%%%%%

define
\begin{itemize}
  \item dynamical system $\{\pS,f^t,\Group\}$
        vs reduced dynamics $\{\PoincS/\Group,f\}$
  \item \statesp\ vs tangent space, see \reffig{fig:tangents}
  \item trajectory vs orbit
  \item template
        \\
        there is always tension between mathematics - linear problem eigenmodes
        (Fourier for translations and rotations) and physics - the fact that
        nonlinear dynamics states are far away from such axes, as they
        always involve a number of such linear modes strongly entangled.
  \item section vs slice
  \item strobing $\sim$ method of connections
  \item reduction vs projection
\end{itemize}
will explain later on
    }

The \emph{group orbit} $\pS_\ssp $ of a \statesp\ point $\ssp \in \pS$ is
traced out by the set of all group actions
\beq
\pS_\ssp = \{\LieEl\,\ssp \mid \LieEl \in {\Group}\}
% \,,\qquad \pS_\ssp \subset \pS
\,.
\ee{sspOrbit}
Any state in the  group orbit set $\pS_{\ssp}$
is physically equivalent to any other. The action of a symmetry group
thus foliates the \statesp\ into a union of group orbits,
\reffig{fig:BeThTraj}\,(a).

\subsection{\cLe}
\subsection{Ring of Fire}
\subsection{Experimentalist description: a video 1D to 3D arrays of pixels}
\subsection{Theorist description: $\infty$-\dmn\ \statesp}
\subsection{Time orbit: point is a point, line is a line in all dimensions}
\label{sect:TimeOrb}

\subsection{Physical dimension: covariant Lyapunov vectors}

\section{Poincar\'e sections}
\label{s:cut}

In general there are two strategies for replacing a continuous-time flow
by iterated mappings; by cutting it by Poincar\'e sections, or by
\emph{strobing} it at a sequence of instants in time. Think of your
partner moving to the beat in a disco: a sequence of frozen stills. While
`strobing' is what any numerical integrator does, by representing a
trajectory by a sequence of time-integration step separated points,
strobing is in general not a reduction of a flow, as the sequence of
strobed points still resides in the full \statesp\ $\pS$, of
dimensionality $d$.

In the {\em Poincar\'e section method} one records the coordinates of a
trajectory whenever the trajectory crosses a prescribed trigger. This
triggering event can be as simple as vanishing of one of the coordinates,
or as complicated as the trajectory cutting through a curved
hypersurface. A Poincar\'e section (or, in the remainder of this chapter,
just `section') is {\em not} a projection onto a lower-dimensional space:
Rather, it is a local change of coordinates to a direction along the
flow, and the remaining coordinates (spanning the section) transverse to
it. No information about the flow is lost by reducing it to its set of
Poincar\'e section points and the return maps connecting them; the full
space trajectory can always be reconstructed by integration from the
nearest point in the section.

Reduction of a continuous time flow to its Poincar\'e section is a
powerful visualization tool. But, the method of sections is more than
visualization; it is also a fundamental tool of dynamics - to fully
unravel the geometry of a chaotic flow, one {\em has to} quotient all of
its symmetries, and evolution in time is one of these.

\subsection{Local chart}
After some experimentation and observations of turbulence in a given
flow, one can identify a set of dynamically important unstable
{\recurrStr s}.

\subsection{R\"ossler {\poincBord}}
\subsection{R\"ossler two-chart atlas}
\subsection{R\"ossler unstable manifold curvilinear distance}
\subsection{R\"ossler return map}
\subsection{$N$-chart atlas, forward maps}
\subsection{Ring of Fire return map\rf{lanCvit07}}
