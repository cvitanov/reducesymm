% siminos/atlas/cut.tex  pdflatex atlas
% $Author$ $Date$

\section{Section}
\label{s:cut}

In the {\em \PoincSec} method one records the coordinates
$\sspRed_n$ of the trajectory $\ssp(\zeit)$ at the instants $\zeit_n$
when it traverses a fixed oriented hypersurface $\PoincS$ of codimension
1. For high-dimensional flows that we have in mind, the practical choice
is a hyperplane, the only type of \PoincSec\ (from now on, just a
\emph{section}) we shall consider here. Such a section captures
important features of the flow in an open neighborhood of the
section-fixing \template.

As an example consider the system of R\"ossler\rf{ross},
\index{R\"ossler system}
\beq
\begin{split}
  \dot{x} &= -y \,-\,z \\
  \dot{y} &= x + a y \\
  \dot{z} &= b + z (x - c)
  \,,
  \label{eq:Rossler}
\end{split}
\eeq
where $a = b = 0.2$ and $c = 5.7$. This flow has two prominent invariant
states, the `inner' and the `outer' unstable \eqva\ $\slicep{}^{(-)}$  and
$\slicep{}^{(+)}$, which we pick as {\em \template s} (\refFig{fig:RoessTrjs}).

We orient the sections so the plane $\PoincS_{-}$ contains the 1\dmn\
stable eigenvector of $\slicep{}^{(-)}$ (\reffig{fig:RoessTrjs}\,(b)),
and the other section $\PoincS_{+}$ contains the 1\dmn\ unstable
eigenvector of $\slicep{}^{(+)}$ (\reffig{fig:RoessFarEq}\,(a)), thus
capturing the local spiral-in, spiral-out dynamics. The remaining freedom
to rotate each section can be used to orient them in such a way that the
ridge (the intersection of the two sections) lies approximately
between the two templates (\reffig{fig:RoessFarEq}\,(b)).

A well chosen section captures the dynamics in the neighborhood of its
\template, but how far does this neighborhood extend?
The answer is that the section captures neighboring trajectories as long
as it cuts them transversally; it fails the moment the velocity field at
a point $\sspRSing$ fails to pierce the section. At these locations, the
velocity either vanishes (\eqv) or is tangent to the section, \ie,
orthogonal to the section normal $\hat{n}$,
\beq
    \hat{n} \cdot \vel(\sspRSing) = 0
\,,\qquad
    \sspRSing \in \cal{S}
\,.
\ee{eq:sspRSing}
For a smooth flow such points form a smooth $(d\!-\!2)$\dmn\
\emph{\poincBord} ${\cal S} \subset \PoincS$, which encloses the open
neighborhood of the {\template} characterized by qualitatively similar
flow. We shall refer to this region of the section as a `chart' of the
{\template} neighborhood (see
\reffig{fig:RoessTrjs}\,(b) and \reffig{fig:RoessFarEq}). Beyond the border, the flow
pierces the section hyperplane in the `wrong' direction and the dynamics
are qualitatively different.

%%%%%%%%%%%%%%%%%%%%%%%%%%%%%%%%%%%%%%%%%%%%%%%%%%%%%%%%%%%%%%%%%%%%%
\begin{figure}
(a)\includegraphics[width=0.24\textwidth]{Rossler_Equilibria}%{RoessTrjs}%
(b)\includegraphics[width=0.18\textwidth,clip=true]{RoessNearEq2a}
    \caption{
(a)
\DBedit{\Huge [DRAW THIS!]}
R\"ossler \eqva\ and their invariant manifolds. The stable manifold of
the inner {\eqv} $\slicep{}^{(-)}$  is 1-dimensional and the unstable one
is a spiral-out focus. For the outer {\eqv} $\slicep{}^{(+)}$  the stable
manifold is a spiral-in focus (basin boundary for initial conditions that
either fall into the chaotic attractor, or escape to infinity) and the
unstable manifold is 1-dimensional.
(b)
\PoincSec\ plane through the inner {\eqv} $\slicep{}^{(-)}$ and
its stable eigenvector. The chart $\PoincS_{-}$ of the $\slicep{}^{(-)}$
neighborhood, bounded by its \poincBord, is highlighted by (light) green.
    }
\label{fig:RoessTrjs}
\end{figure}
%%%%%%%%%%%%%%%%%%%%%%%%%%%%%%%%%%%%%%%%%%%%%%%%%%%%%%%%%%%%%%%%%%%%%

%%%%%%%%%%%%%%%%%%%%%%%%%%%%%%%%%%%%%%%%%%%%%%%%%%%%%%%%%%%%%%%%%%%%%
% supercedes \label{fig:RoessBothEq}
\begin{figure}%[H]
\begin{center}
(a)\includegraphics[width=0.23\textwidth,clip=true]{RoessFarEq2}%
(b)\includegraphics[width=0.23\textwidth,clip=true]{RoessSctAtlas2}
\end{center}
  \caption[R\"ossler section, outer {\eqv}]{
(a)
  A section through the outer {\eqv} $\slicep{}^{(+)}$  and its unstable
  eigenvector. The velocity $\vel(\sspRed_n)$ at the $n$th section is
  indicated by (yellow) vector.
(b)
  A two-chart atlas of R\"ossler flow, with charts $\PoincS_{-}$ and
  $\PoincS_{+}$ oriented and combined so that the ridge (intersection of
  the two sections, indicated by the dotted brown line) lies
  approximately between the \template s.
  } \label{fig:RoessFarEq}
\end{figure}
%%%%%%%%%%%%%%%%%%%%%%%%%%%%%%%%%%%%%%%%%%%%%%%%%%%%%%%%%%%%%%%%%%%%%

For R\"ossler flow, the border condition \refeq{eq:sspRSing} yields a
quadratic condition in 3 dimensions, so \poincBord s\ drawn in
\reffig{fig:RoessTrjs}\,(b) and \reffig{fig:RoessFarEq} are conic
sections. The two charts meet at a ridge, and together do a pretty good
job as the 2-chart atlas of the interesting R\"ossler dynamics,
\reffig{fig:RoessFarEq}. As explained in ChaosBook.org\cite{DasBuch}, due
to extreme contraction rate, the section in \reffig{fig:RoessTrjs}\,(b)
is for all practical purposes 1\dmn, and the associated return map yields
all \po s of the 3\dmn\ flow.

In 3 dimensions everything -sections, ridges, \poincBord s-  can be
drawn, and a $\chi\alpha\rho\tau\eta\varsigma$ does fit on a sheet of
papyrus. But what about high-dimensional flows? The point of the
cartography enterprize undertaken here is that while it is impossible to
visualize  $(d\!-\!2)$\dmn\ {\poincBord} of the $(d\!-\!1)$\dmn\ slab
that is now our chart, a point is a point, and a line is a line in a
projection from any number of dimensions, so a trajectory crossing of
both a section and a {\poincBord} can be easily determined and visualized
in any dimension.

To summarize:
Evolution in time decomposes the \statesp\ into a spaghetti of 1\dmn\
trajectories $\ssp(\zeit)$, each determined by a single point $\ssp(0)$
on it. A well chosen set of \emph{sections} of codimension 1 allows us to
`quotient' the continuous time parameter $\zeit$. For unstable
trajectories one needs, in addition, a notion of recurrence to the
section. The set of points $\{\sspRed_n\} = \{\ssp(\zeit_n)\}$  separated
by short-time section to section flights, then captures the transverse
dynamics without losing any information about the chaotic flow.
We can thus chart the regions of \statesp\ of interest, by a picking a
sufficient number of \template s and the associated charts over their
neighborhoods, each bounded by \poincBord s and ridges.

Two concluding remarks on what sections \emph{are not}:

(1) A \PoincSec\ is {\em not} a projection onto a
lower-dimensional space: Rather, it is a local change of coordinates to a
direction along the flow $\vel(\sspRed)$, and the remaining coordinates
transverse to it. No information about the flow is lost; the full space
trajectory can always be reconstructed by integration from its point
$\sspRed$ in the section.

(2) The method of \PoincSec s is {\em not} equivalent to
\emph{strobing} a flow at a sequence of instants in time. While
`strobing' is what any numerical integrator does, by representing a
trajectory by a sequence of time-integration step separated points, this
is no reduction of a flow to a codimension 1 manifold, as the sequence of
strobed points still resides in the full \statesp\ $\pS$, of
dimensionality $d$.
