% siminos/atlas/cut.tex  pdflatex atlas
% $Author$ $Date$

\section{Poincar\'e sections}
\label{s:cut}

In the {\em Poincar\'e section method} one records the coordinates of a
trajectory at the instant it traverses an oriented fixed hypersurface
$\PoincS$ of codimension 1. For the high-dimensional flows that we have in
mind the only feasible choice is a hyperplane, the type of Poincar\'e
section (or, from now on, just a `section')  we shall consider in this
paper. Such section is almost always only local, and should
capture important features of the flow in an open neighborhood of the
section-fixing \template.

%%%%%%%%%%%%%%%%%%%%%%%%%%%%%%%%%%%%%%%%%%%%%%%%%%%%%%%%%%%%%%%%%%%%%
\begin{figure}
  \includegraphics[width=0.3\textwidth]{AmLeAg06Im1}
    \caption{[From \refref{AmLeAg06}]
R\"ossler \eqva\ and their invariant manifolds. The stable manifold of
the inner {\eqv} $\ssp_{-}$  is 1-dimensional and the unstable one is a
spiral-out focus. The stable manifold at the outer {\eqv} $\ssp_{+}$ is a
spiral-in focus and the unstable manifold is 1-dimensional.
    }
\label{fig:AmLeAg06Im1}
\end{figure}
%%%%%%%%%%%%%%%%%%%%%%%%%%%%%%%%%%%%%%%%%%%%%%%%%%%%%%%%%%%%%%%%%%%%%

As an example consider the system of R\"ossler\rf{ross},
\index{R\"ossler system}
\beq
\begin{split}
  \dot{x} &= -x \,-\,z \\
  \dot{y} &= x + a y \\
  \dot{z} &= b + z (x - c)
  \,,
  \label{eq:Rossler}
\end{split}
\eeq
where $a = b = 0.2$ and $c = 5.7$. This flow has two prominent invariant states,
the `inner' unstable \eqv\ $\ssp_{-}$ and the `outer' $\ssp_{+}$ (Fig. \ref{fig:AmLeAg06Im1})
    \PC{ Draw (un)stable eigenvectors as in \reffig{fig:AmLeAg06Im1}.
    Also, run the unstable spiral(s) for the upper brunch further out, to
    trace out the basin boundary.}
which we pick as {\em
\template s}. These are points through which the sections pass and they determine the dynamical behaviors that will be captured by the section. We orient the first section $\PoincS_{-}$
(\reffig{fig:RoessSct1}) so that it contains the
1\dmn\ stable eigenvector of $\ssp_{-}$ and captures the local outward spiraling dynamics. The other section $\PoincS_{+}$ contains the 1\dmn\ unstable eigenvector (\reffig{fig:RoessSct2}) of $\ssp_{+}$, thus capturing the trajectories as they spiral onto its unstable manifold and either fall into the chaotic attractor or move on to infinity. The
remaining freedom to rotate each section can be used to orient them in
such a fashion that the Euclidean distance from the first \template\ to
the ridge (intersection of the two sections, indicated by the brown line
in individual sections of the preceding figures), and then to the second
\template\ is approximately minimized (\reffig{fig:RoessSctAtlas}). \PC{DB: Not really sure how this last bit fits into this part at this time. It is unclear why we are doing this. I would consider moving this to discussion of 2-chart atlas}

A properly chosen section captures the dynamics in the neighborhood of its template, but this raises the obvious question: How far does the neighborhood of a {\template} extend along the hyperplane?

The answer is that section faithfully captures neighboring orbits as
long as it cuts them transversally; it fails the moment the velocity
field at point $\sspRSing$ fails to pierce the section. At these locations,
the velocity either vanishes (\eqv) or is tangent to the section and, thus, orthogonal to the
section \PC{DB: I think in this context the template does not have normal, the section does.} normal $\hat{n}$, such that
\beq
    \hat{n} \cdot \vel(\sspRSing) = 0
\,,\qquad
    \sspRSing \in \cal{S}
\,,
\ee{eq:sspRSing}
For a smooth flows such points form a smooth $(d\!-\!2)$\dmn\ \emph{\poincBord} ${\cal S} \subset \PoincS$ \PC{DB: Are we using $S$ or ${\cal S}$ as the variable for the section border? Both are used in the text.} encompassing the open neighborhood of the {\template} characterized by qualitatively similar flow. This region of the section hyperplane comprises the chart \PC{DB: As of 4/10 `chart' is undefined in the text up to this point} of the {\template} neighborhood.

For R\"ossler flow \refeq{eq:Rossler}
[blah blah]
%\subsection{R\"ossler two-chart atlas}

\subsection{R\"ossler unstable manifold curvilinear distance}
\subsection{R\"ossler return map}
\subsection{$N$-chart atlas, forward maps}
% \subsection{Ring of Fire return map\rf{lanCvit07}}

    \ifdraft\color{blue}


The two charts
\reffigs{fig:RoessSct1}{fig:RoessSct2} illustrate \poincBord,
and \reffigs{fig:RoessSctAtlas} the combined 2-chart atlas.

%%%%%%%%%%%%%%%%%%%%%%%%%%%%%%%%%%%%%%%%%%%%%%%%%%%%%%%%%%%%%%%%%%%%%
\begin{figure}%[H]
\begin{center}
  \includegraphics[width=0.30\textwidth,clip=true]{RoessSct1}
\end{center}
  \caption{\label{fig:RoessSct1}
  A R\"ossler flow Poincar\'e section $\PoincS_{-}$ centered on the inner
  {\eqv} $\ssp_{-}$ \template\ and its stable eigenvector.
}
\end{figure}
%%%%%%%%%%%%%%%%%%%%%%%%%%%%%%%%%%%%%%%%%%%%%%%%%%%%%%%%%%%%%%%%%%%%%

Cannibalize from this text:
Lorenz flow cut by  $y=x$ Poincar\'e section plane $\PoincS$ through the
$z$ axis and both $\EQV{1,2}$ \eqva. Points where flow pierces into
section % $\PoincS$ are marked by dots. To aid visualization of the flow
near the $\EQV{0}$ \eqv, the flow is cut by the second Poincar\'e
section,  $\PoincS'$, through $y=-x$ and the $z$ axis.

%%%%%%%%%%%%%%%%%%%%%%%%%%%%%%%%%%%%%%%%%%%%%%%%%%%%%%%%%%%%%%%%%%%%%
\begin{figure}%[H]
\begin{center}
  \includegraphics[width=0.30\textwidth,clip=true]{RoessSct2}
\end{center}
  \caption[R\"ossler section, outer {\eqv}]{
  A Poincar\'e section for R\"ossler flow correctly \PC{DB: As of 4/10 the figure about good and bad Poincare sections went to Flotsam. Are we going to put it back or are we dropping it and should ``correctly'' be dropped} centered on the outer
  {\eqv} $\ssp_{+}$ \template.
  } \label{fig:RoessSct2}
\end{figure}
%%%%%%%%%%%%%%%%%%%%%%%%%%%%%%%%%%%%%%%%%%%%%%%%%%%%%%%%%%%%%%%%%%%%%

%%%%%%%%%%%%%%%%%%%%%%%%%%%%%%%%%%%%%%%%%%%%%%%%%%%%%%%%%%%%%%%%%%%%%
\begin{figure}%[H]
\begin{center}
  \includegraphics[width=0.30\textwidth,clip=true]{RoessSctAtlas}
\end{center}
  \caption{
  A two-section atlas for R\"ossler flow, with the local sections of
  \reffigs{fig:RoessSct1}{fig:RoessSct2} oriented and combined so the
  Euclidean distance from the first \template\ to the ridge (intersection
  of the two sections, indicated by the brown line in individual sections
  of the preceding figures), and then to the second \template\ is
  approximately minimized.
  } \label{fig:RoessSctAtlas}
\end{figure}
%%%%%%%%%%%%%%%%%%%%%%%%%%%%%%%%%%%%%%%%%%%%%%%%%%%%%%%%%%%%%%%%%%%%%

%%%%%%%%%%%%%%%%%%%%%%%%%%%%%%%%%%%%%%%%%%%%%%%%%%%%%%%%%%%%%%%%%%%%%
\begin{figure}
\begin{center}
(a) % \includegraphics[width=0.30\textwidth,clip=true]{RoessSct1}
(b) \includegraphics[width=0.30\textwidth,clip=true]{RoessRetMap}
\end{center}
  \caption{
(a) R\"ossler strange attractor section of \reffig{RoessSct1}
(b) R\"ossler return map for curvilinear distance as measured
along the unstable manifold section (the method of
\refref{Christiansen97}, see \refrefs{DasBuchBasu07}).
  }
\label{fig:RoessRetMap}
\end{figure}
%%%%%%%%%%%%%%%%%%%%%%%%%%%%%%%%%%%%%%%%%%%%%%%%%%%%%%%%%%%%%%%%%%%%%

In general there are two strategies for replacing a continuous-time flow
by iterated mappings; by cutting it by Poincar\'e sections, or by
\emph{strobing} it at a sequence of instants in time. While
`strobing' is what any numerical integrator does, by representing a
trajectory by a sequence of time-integration step separated points,
strobing is in general not a reduction of a flow, as the sequence of
strobed points still resides in the full \statesp\ $\pS$, of
dimensionality $d$.

A Poincar\'e section is {\em not} a projection onto a lower-dimensional
space: Rather, it is a local change of coordinates to a direction along
the flow, and the remaining coordinates (spanning the section) transverse
to it. No information about the flow is lost; the full space trajectory
can always be reconstructed by integration from the nearest point in the
section.


An example of a wurst is the set of group orbits traced out by a single
short \rpo\ in \reffig{fig:CLf01group}. As \cLe\ exhibits only the simplest,
$m=1$ Fourier mode, all group orbits are circles, which appear elliptical in
most $d=5 \to 3$~dimensions projections.
    \PC{define \cLe, refer to \reffig{fig:CLf01group}}
    \color{black}\fi

Any state in the  group orbit set $\pS_{\ssp}$
is physically equivalent to any other. The action of a symmetry group
thus stratifies the \statesp\ into a union of group orbits,
\reffig{fig:BeThTraj}\,{(a)}.

\subsection{Norms - distances between states}

We have seen that in presence of the continuous $\SOn{2}$ symmetry
\reqva\ and \rpo s are 2- and 3-dimensional manifolds of physically
equivalent states. How are we to compare a pair of such states? We shall
do this here by determining the minimal distance between them.

In order to quantify
whether two states are close to or far from each other, one
needs a notion of distance between two points in \statesp, which 
is invariant under all symmetries of the flow, such that
$\Norm{\ssp-\LieEl\,\slicep}=\Norm{\sspRed-\slicep}$. 

Here, we will use
\beq
  \Norm{\ssp-\ssp'}^2  = \braket{\ssp-\ssp'}{\ssp-\ssp'} =
\frac{1}{V}
\int_\bCell \! d \bx \;
(\vec{u}-\vec{u}') \cdot (\vec{u}-\vec{u}')
\,.
\ee{innerproduct}

There is no compelling reason to use this {`energy norm'}, other the fact that 
velocity fields are what is usually computed in numerical simulations. One's choice
of norm actually depends very much on the application; the importance (and arbitrariness) 
of the choice of norm cannot be overemphasized. For example, in the study of 
`optimal perturbations' that move a laminar solution to a turbulent one, both energy
\citep{TeHaHe10} and dissipation \citep{LoCaCoPeGo11} norms have been
used. For experimental data sets, pattern recognition type norms may be a practical option.

\subsection{\Statesp\ visualization}

On perils of thinking linearly: bases such as Fourier modes are
perfectly natural for problems such a bifurcation of a steady state, and
other weak perturbations. They are absolutely unnatural for strongly
nonlinear problems, with many Fourier modes of comparable magnitude and
strongly entangled.

\subsection{\CLe}

The \CLe $\,$ \cite{GibMcCLE82} are given by 

\beq
\begin{split}
  \dot{x} &= -\sigma x \,+\,\sigma y \\
  \dot{y} &= (\rho-z)\,x\,-\,a y \\
  \dot{z} &= \dfrac{1}{2}(x y^* + x^* y) - b z
  \,,
  \label{eq:ComplexLorenz}
\end{split}
\eeq

where $x$, $y$ are complex variables, $z$ is real, the parameters $\sigma$ and $b$ are real, and the parameters $\rho = \rho_1 + i \rho_2$ and $a = 1 - i e$ are complex.

\subsection{Experimentalist description: a video 1D to 3D arrays of pixels}
\subsection{Theorist description: $\infty$-\dmn\ \statesp}
\subsection{Time orbit: point is a point, line is a line in all dimensions}
\label{sect:TimeOrb}

\subsection{Physical dimension: covariant Lyapunov vectors}
