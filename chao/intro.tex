%           %experimenting with svn-multi

\svnkwsave{$RepoFile: halcrow/dupre/intro.tex $}
\svnidlong {$HeadURL$}
{$LastChangedDate$}
{$LastChangedRevision$} {$LastChangedBy$}
\svnid{$Id$}

\chapter{Continuous symmetry reduction by the \mslices}
\label{chap:CLF}

\begin{quote}
{\bf Abstract:}

When a dynamical system has a ...
    \PC{\color{red}
Stefan, write this: often! this might be the only part
of this text that most people glance at.
%PC 2010-09-30: planted an error into the abstract, just to see how
%   often do you edit it.
	}
     \PC{
   When you write
   a project report or a research article, you always write abstract, introduction
   and conclusions first, and then keep rewriting them often.
   They are the most important parts of the text, as that is
   for most people only parts they will look at.
   }
    %
    \Private{ % subversion label pages
$\footnotemark\footnotetext{{\tt \svnkw{RepoFile}}, rev. \svnfilerev:
 last edit by \svnFullAuthor{\svnfileauthor},
 \svnfilemonth/\svnfileday/\svnfileyear}$
    } % end \Private{

\end{quote}



\section{Introduction}
\label{sect:intro}

{\bf PC}{ this is just a placeholder}

\bigskip
\noindent {\bf Acknowledgments.}
%This report is written in collaboration with
% % E.~Siminos and
%P.~Cvitanovi\'c.
D.W. work was supported by the National Science Foundation
grant DMR~0820054.
P.C. thanks Glen Robinson Jr. for support. 	
