\ifsvnmulti
\svnkwsave{$RepoFile: siminos/chao/ChaosBook.tex $}
\svnidlong {$HeadURL$}
{$LastChangedDate$}
{$LastChangedRevision$} {$LastChangedBy$}
\svnid{$Id$}
\fi


\chapter{ChaosBook.org blog}
\label{chap:ChaosBook}

\begin{description}

\item[2011-08-24 PC to Chao]
I have created this chapter to collect all your ChaosBook.org
notes in one place.

\item[2011-03-21 PC] You have the dasbuch/book/chapter/*.tex source
code, you can just clip and paste formulas to here.

\item[2011-09-12 PC] I never refer to a chapter by it's current number,
as chapter numbers change from edition to edition - latter on (years
hence) trying to figure out what ``Chapter 17'' is can be quite
confusing. Internally, each chapter is kept track off by its file name,
for example, in this blog ``smale'' refers to  \refchap{c-smale} {\em
Stretch, fold, prune}.

\item[2011-12-10 Predrag] Ciao Chao: mark here [ ] when
all relevant stuff is moved to siminos/blog/ChaosBook.tex.

\end{description}

%%-----   Billiards
\section{Chapter: Billiards}
\label{c-billiards}\noindent dasbuch/book/chapter/billiards.tex
\begin{description}

\item[2011-03-25 PC]
Glad you asked the question about choice of billiard coordinates. Please
write up the solution to \refexer{ex_birkhoff}; it is worth doing it in
class as well, a concrete example of how symplectic invariance preserves
area for each $(q,p)$ dual coordinate  pair.

\item[2011-04-19 CS] Done - I have verified in \refexer{ex_birkhoff} that
the 2-form/wedge product is conserved on the \Poincare section.

\item[2011-04-19 PC] It's a bit inelegant, no? Maybe you can discuss it with
Francois and Adam and rewrite it elegantly.

\item[2011-04-12 PC] Remember to write up your solution to
\refexer{ex_birkhoff} before you forget it. If it is good, we might
rewrite the Chapter 8 {\em Billiards} before the study group takes it up.

\item[2011-04-05 CS] Worked out \refexer{ex_birkhoff}. Finished Chapter 9
with all the details in the examples, have had exemplified pictures of
symmetry and group actions.

\item[2011-04-06 PC] Not so fast - \refexer{ex_birkhoff} is not finished
until you write down your solution.

\item[2011-04-20 PC]  I think that in \refexer{ex_birkhoff} you want to
have different radii $a_i$ for the two disks. If you do that, you have
the general map for a billiard with a smooth boundary, as you only need the
local radius of curvature. Also, while showing that
the 2-form/wedge product is conserved on the \Poincare section might make
Adam and friends happy, what you really need is the explicit $[2\!\times\!2]$
Jacobian matrix, because you will need to compute its eigenvalues - I do not see
how you do that from the wedge product alone. Once you have the \jacobianM,
the area preservation is immediate, as you will show that $|\det \jMP |= 1$.
% {\jMP}{\ensuremath{\hat{J}}}   % jacobian matrix, Poincare return

\item[2011-4-22 CS to PC 06:32am]  Generalized the proof in the exercise
to different radius and wrote out the Jacobian matrix. But what's
interesting as you can below is that as long as my previous proof when
the two circles are identical is correct, there's no way that the
Jacobian in the case of unequal radius is one. You can easily see that
the map: $(\theta_1, \sin(\phi_1))\Rightarrow(\theta_2, \sin(\phi_2))$ is
still volume-preserving in same way. Plus the fact that $s=a\theta$,
there must be a factor of ratio of two radius comes in. You can visualize
this effect when you expand or compress the circle in the righthand side.
So I guess Birkhoff coordinates $(s,p)$ just preserve phase space volume
when the radius are equal. Right?

\item[2011-4-22 PC] No.

\item[2011-4-22 CS]
You're right. I checked the proof again and found out that in the map of
$p$, I missed a factor $\frac{a_1}{a_2}$ in front of $p_1$ and other two
$\frac{1}{a_2}$ factors before the remaining two terms. Now the result is
happy. It seems that caffeine cannot perfectly replace sleep.

\item[2011-04-22 PC]
I'm feeling better, too. I'll try to unconvolute your derivation (your
`sine laws' are formulas for the impact parameters in my hand-drawn
billiard maps), but you will feel even more alert if you can
verify (or reduce your formulas to) the ChaosBook formulas
\refeq{hor}, \refeq{eq_her} and \refeq{eq_her}.


\end{description}
